% Options for packages loaded elsewhere
\PassOptionsToPackage{unicode}{hyperref}
\PassOptionsToPackage{hyphens}{url}
%
\documentclass[
]{ltjarticle}
\usepackage{lmodern}
\usepackage{amssymb,amsmath}
\usepackage{ifxetex,ifluatex}
\ifnum 0\ifxetex 1\fi\ifluatex 1\fi=0 % if pdftex
  \usepackage[T1]{fontenc}
  \usepackage[utf8]{inputenc}
  \usepackage{textcomp} % provide euro and other symbols
\else % if luatex or xetex
  \usepackage{unicode-math}
  \defaultfontfeatures{Scale=MatchLowercase}
  \defaultfontfeatures[\rmfamily]{Ligatures=TeX,Scale=1}
\fi
% Use upquote if available, for straight quotes in verbatim environments
\IfFileExists{upquote.sty}{\usepackage{upquote}}{}
\IfFileExists{microtype.sty}{% use microtype if available
  \usepackage[]{microtype}
  \UseMicrotypeSet[protrusion]{basicmath} % disable protrusion for tt fonts
}{}
\makeatletter
\@ifundefined{KOMAClassName}{% if non-KOMA class
  \IfFileExists{parskip.sty}{%
    \usepackage{parskip}
  }{% else
    \setlength{\parindent}{0pt}
    \setlength{\parskip}{6pt plus 2pt minus 1pt}}
}{% if KOMA class
  \KOMAoptions{parskip=half}}
\makeatother
\usepackage{xcolor}
\IfFileExists{xurl.sty}{\usepackage{xurl}}{} % add URL line breaks if available
\IfFileExists{bookmark.sty}{\usepackage{bookmark}}{\usepackage{hyperref}}
\hypersetup{
  pdftitle={補充書01},
  pdfauthor={廣野秀樹},
  hidelinks,
  pdfcreator={LaTeX via pandoc}}
\urlstyle{same} % disable monospaced font for URLs
\usepackage[left=3cm,right=2cm,top=3.5cm,bottom=2.7cm]{geometry}
\setlength{\emergencystretch}{3em} % prevent overfull lines
\providecommand{\tightlist}{%
  \setlength{\itemsep}{0pt}\setlength{\parskip}{0pt}}
\setcounter{secnumdepth}{5}
\ifluatex
  \usepackage{selnolig}  % disable illegal ligatures
\fi

\title{補充書01}
\author{廣野秀樹}
\date{2021-04-02 06:06:36 +0900}

\begin{document}
\maketitle

{
\setcounter{tocdepth}{4}
\tableofcontents
}
\hypertarget{ux88abux544aux767aux4ebaux5c0fux5cf6ux88d5ux53f2ux88c1ux5224ux9577ux306bux3064ux3044ux3066}{%
\section{被告発人小島裕史裁判長について}\label{ux88abux544aux767aux4ebaux5c0fux5cf6ux88d5ux53f2ux88c1ux5224ux9577ux306bux3064ux3044ux3066}}

\hypertarget{ux798fux4e95ux5973ux5b50ux4e2dux5b66ux751fux6bbaux5bb3ux4e8bux4ef6ux306eux9006ux8ee2ux6709ux7f6aux5224ux6c7a}{%
\subsection{福井女子中学生殺害事件の逆転有罪判決}\label{ux798fux4e95ux5973ux5b50ux4e2dux5b66ux751fux6bbaux5bb3ux4e8bux4ef6ux306eux9006ux8ee2ux6709ux7f6aux5224ux6c7a}}

\hypertarget{ux677eux6d66ux7531ux8a18ux592bux691cux5bdfux5b98ux59163ux4ebaux3060ux3063ux305fux540dux53e4ux5c4bux9ad8ux7b49ux691cux5bdfux5e81ux91d1ux6ca2ux652fux90e8ux798fux4e95ux5973ux5b50ux4e2dux5b66ux751fux6bbaux5bb3ux4e8bux4ef6ux306eux63a7ux8a34ux5be9}{%
\subsubsection{松浦由記夫検察官外3人だった名古屋高等検察庁金沢支部(福井女子中学生殺害事件の控訴審)}\label{ux677eux6d66ux7531ux8a18ux592bux691cux5bdfux5b98ux59163ux4ebaux3060ux3063ux305fux540dux53e4ux5c4bux9ad8ux7b49ux691cux5bdfux5e81ux91d1ux6ca2ux652fux90e8ux798fux4e95ux5973ux5b50ux4e2dux5b66ux751fux6bbaux5bb3ux4e8bux4ef6ux306eux63a7ux8a34ux5be9}}

:LOGBOOK:CLOCK: {[}2021-04-02 金 06:13{]}:END:

:CATEGORIES: @kanazawabengosi \#金沢弁護士会 @JFBAsns
日本弁護士連合会(日弁連) \#法務省 @MOJ\_HOUMU \#松浦由記夫検事
\#被告発人小島裕史裁判長

〉〉〉:Emacs: 2021-04-02(金曜日)06:13  〉〉〉

 告発状は,一昨日の3月31日,昨日の4月1日の2回に分けて同じ能都郵便局から郵送しました。本日4月2日より補充書の作成に取り掛かります。今回より記述範囲の時刻の記録を再開します。できるだけ整理したかたちで作成するのと,既述の時期を特定し前後の流れを明確化するためです。

 ネットで福井女子中学生殺害事件の控訴審の判決文が出てきたのですが,これもきっかけは奈良の被疑者変死告発事件について調べ始めたことで,松尾昭彦裁判官の名前がジャーナリストの江川紹子氏のツイートに出てきたことです。

〉〉〉 kk\_hironoのリツイート 〉〉〉

\begin{itemize}
\tightlist
\item
  RT
  kk\_hirono(刑事告発・非常上告_金沢地方検察庁御中)|hirono\_hideki(奉納\さらば弁護士鉄道・泥棒神社の物語)
  日時:2021-04-02 06:22/2021/04/02 05:40 URL:
  \url{https://twitter.com/kk\_hirono/status/1377733059798134785} 
  \url{https://twitter.com/hirono\_hideki/status/1377722596494438402} 
  \textgreater{} 名古屋高等裁判所金沢支部 平成2年(う)71号 判決 -
  大判例 \url{https://t.co/CbDtCctst3} 
  殺人被告事件について、平成二年九月二六日福井地方裁判所が言い渡した判決に対し、検察官から控訴の申立があったので、当裁判所は、検察官加藤元章、同寺坂衛、同松浦由記夫出席の上審理し、次のとおり判決する
\end{itemize}

 控訴審で高検検事の名前が複数出てきたのも初めて見たと思うのですが,それも支部である名古屋高等検察庁金沢支部で,告発状でも取り上げた松浦由記夫検事の名前が出てきました。

 ghatena201912\textbar grep 松浦 の結果はゼロでした。twilog-serch
とは違い告発状でご説明をしていなかったかもしれませんが,はてなのブログの記事の情報をデータベースに登録し,一覧で出力する自作のコマンドになります。なお,現時点での最終投稿は次の記事になります。

\begin{itemize}
\tightlist
\item
  1293:2021-03-31\_09:36:23 \#告発の事実・補充書準備編 \#告発状202103
  \#\#\#
  被告発人らの殺人未遂の計画性を物語る被害者安藤文さんの裏駐車場での行動(非弁ハンター・魑魅魍魎・深澤諭史弁護士&向原栄大朗弁護士編)
  \url{https://hirono-hideki.hatenadiary.jp/entry/2021/03/31/093618} 
\end{itemize}

 ここで気がついたのですが,最後のエントリーをはてなブログ等に投稿し忘れていました。

\begin{itemize}
\tightlist
\item
  1294:2021-04-02\_06:35:44 \#告発の事実・補充書準備編 \#告発状202103
  \#\#\#
  被告発人らの殺人未遂の計画性を物語る被害者安藤文さんの裏駐車場での行動(次へ繋ぐ総括編)
  \url{https://hirono-hideki.hatenadiary.jp/entry/2021/04/02/063541} 
\end{itemize}

 一審判決で不本意な判決が出た場合,地方検察庁から「上級庁と協議」というはお決まりに近いものを感じますが,高等検察庁というのもその1つではと思われます。もう一つというのが最高検察庁ですが,上級庁で括られていると思うのでそれ以上はわかりません。

\begin{itemize}
\tightlist
\item
  2021年04月02日06時41分の登録:
  「上級庁」を@hirono\_hideki @kk\_hirono @s\_hironoで検索 18件の該当 2021-04-02\_06:41の記録
  \url{https://kk2020-09.blogspot.com/2021/04/hironohidekikkhironoshirono182021-04.html} 
\end{itemize}

〉〉〉 kk\_hironoのリツイート 〉〉〉

\begin{itemize}
\tightlist
\item
  RT
  kk\_hirono(刑事告発・非常上告_金沢地方検察庁御中)|hirono\_hideki(奉納\さらば弁護士鉄道・泥棒神社の物語)
  日時:2021-04-02 06:44/2011/05/25 10:12 URL:
  \url{https://twitter.com/kk\_hirono/status/1377738677653102593} 
  \url{https://twitter.com/hirono\_hideki/status/73194608800178176} 
  \textgreater{} @amneris84
  普通は上級庁と協議とか、高検と協議ともいうのが定番だと思うのですが、確かに水戸地検と言っていました。支部判決とはいえ、妙な気はしました。事実上、控訴は断念するほかないとは思いますが、裁判所も検察の顔を立てたのかもしれないです。
\end{itemize}

〉〉〉 kk\_hironoのリツイート 〉〉〉

\begin{itemize}
\tightlist
\item
  RT
  kk\_hirono(刑事告発・非常上告_金沢地方検察庁御中)|amneris84(Shoko
  Egawa) 日時:2021-04-02 06:44/2011/05/25 08:34 URL:
  \url{https://twitter.com/kk\_hirono/status/1377738690009571328} 
  \url{https://twitter.com/amneris84/status/73170004312457216} 
  \textgreater{}
  3)もちろん、警察や検察に大きな問題があったのは事実だが、布川事件は裁判所が作った冤罪だと思う。警察や検察にだまされた、では済まない。にもかかわらず、裁判長は2人に謝罪することも、長年の労をねぎらうことすらしていない。裁判所は、2人に謝るべきだ。(了)
\end{itemize}

 ジャーナリストの江川紹子氏のツイートをリツイートしたものと思っていたのですが,そうではなく私のジャーナリストの江川紹子氏のツイートに対する返信ツイートでした。この辺りはテキスト情報のみのツイートで気をつけなければならない誤解釈です。

〉〉〉 kk\_hironoのリツイート 〉〉〉

\begin{itemize}
\tightlist
\item
  RT
  kk\_hirono(刑事告発・非常上告_金沢地方検察庁御中)|hirono\_hideki(奉納\さらば弁護士鉄道・泥棒神社の物語)
  日時:2021-04-02 06:48/2021/03/23 21:49 URL:
  \url{https://twitter.com/kk\_hirono/status/1377739728854732802} 
  \url{https://twitter.com/hirono\_hideki/status/1374342413884387328} 
  \textgreater{}
  17歳への強姦罪、被告に無罪 日や場所「合理的疑い」:朝日新聞デジタル
  \url{https://t.co/pZoeuAP92t} 
  横浜地検の安藤浄人次席検事は「判決内容を精査し、上級庁とも協議の上、適切に対応したい」とコメントした。(大平要、神宮司実玲)
\end{itemize}

 横浜地検が気になったのですが,記事をかくにんしたところやはり横浜地検川崎支部となっていました。ほとんど全てが次席検事のコメントなので横浜地検となるのでしょう。最近のニュースとして記憶にあったのですが,同じ頃,大津地裁でも似たような事件内容で無罪判決がありました。

 大津地裁の裁判長が年度替わりに転勤になったのか気になっていますが,まだ確認はしていません。昨日4月1日の夜に調べて確認したのは金沢弁護士会の役員の顔ぶれです。全て一新されていましたが,副会長の数が一人増えたようにも思え気になっています。

\begin{itemize}
\tightlist
\item
  長女へわいせつ、父親に無罪判決 大津地裁「証言に疑問」:中日新聞Web
  \url{https://t.co/ETu0qz3ttc} 
  大津地裁(大西直樹裁判長)は二十二日、無罪判決を言い渡した。
\end{itemize}

 最新の大津地裁の裁判官の名前を調べたつもりだったのですが,新しい裁判長が赴任したような情報は見当たりませんでした。大西直樹の名前で次に最新情報を調べてみたいと思います。検事正の場合は新任会見が決まってニュースになるという気がするのですが,地裁の場合ははっきりしません。

\begin{itemize}
\tightlist
\item
  大津地方裁判所 担当裁判官一覧 \textbar{} 裁判所
  \url{https://t.co/oBfftRrYF2} 
  合議Aイ係大西直樹,大森直子,髙橋孝治,横井裕美,齊藤隆広,水谷翔,林宏樹
\end{itemize}

 裁判長が転任したのに裁判所のホームページが更新していないことはありえないという前提ですが,これで大西直樹裁判長は平成31年4月から4年目の任期に入ったのではと思われます。4年目の任期というのはこれまでほとんど見ていないかもしれません。

\begin{itemize}
\item
  「責任重く受け止めている」 大津地裁の大西裁判長が言及:中日新聞Web
  \url{https://t.co/BL3iMmKPYW} 
  「裁判官の一人として責任を重く受け止めている」と述べた。裁判官が判決後の説諭で、こうした言及をするのは異例。
\item
  「刑事司法全体への問題提起」裁判長、渾身の説諭 湖東記念病院再審 |社会|地域のニュース|京都新聞
  \url{https://t.co/vSxjQtjTri}  2020年3月31日 21:00
\end{itemize}

 この大西直樹裁判長については,別のところで取り上げておきたいと考えておりますが,非常に考えさせられることの多い裁判長で,その影響力の大きさというのも他人事ではないと考えています。一番の問題点だと個人的に思うのは,弁護士に対する疑念がいささかも感じられず,被暗示性も強そうです。

 大津地方裁判所 担当裁判官一覧も裁判所のホームページからは見つけられず,これは森岡地裁でも金沢地裁でも経験していることです。しかし,検察庁の担当検事一覧というのは見たことがなく,名古屋高裁金沢支部に至っては一人の検察官が対応しているのかとも考えさせられていました。

 これまで高等検察庁の検事が話題になったのは,特殊なケースであった黒川弘務東京高検検事長や退任後に逮捕されたり,あるいはスキャンダルになった検事長ぐらいしか記憶にないのですが,名古屋高等検察庁のホームページが見つかったのは意外で,それも手作り感のあるページでした。

\begin{itemize}
\tightlist
\item
  名古屋高等検察庁 \url{https://t.co/BeBDBLcmwx} 
\end{itemize}

\begin{quote}
《引用の始まり》
\end{quote}

\begin{quote}
名古屋高等検察庁検事長

    なかがわ    ~せいめい  中 川  清 明

昭和33年富山県生まれ昭和59年検事任官

法務省大臣官房秘書課長,法務省大臣官房審議官,高知地検検事正,静岡地検検事正,最高検公安部長,公安調査庁長官などを経て,令和2年5月29日就任
\end{quote}

\begin{quote}
《引用の終わり》
\end{quote}

\begin{itemize}
\tightlist
\item
  検事長の紹介:名古屋高等検察庁
  \url{http://www.kensatsu.go.jp/kakuchou/h\_nagoya/a\_nagoyakoken/a10\_kenjityo/kenjityo.html} 
\end{itemize}

 写真付きの検事長の紹介で,最終更新日が2020年6月5日なっています。新任の挨拶であれば,まだ一年中の任期になりそうです。中川清明検事になりそうですが,これまでに見た検事の名前ではないと思います。上級庁として,私の告発事件に関わり,影響するかもしれないと考えた発見です。

〈〈〈:Linux Emacs: 2021-04-02(金曜日)07:23  〈〈〈

\hypertarget{section}{%
\subsubsection{}\label{section}}

\begin{quote}
《引用の始まり》
\end{quote}

\begin{quote}
(法令の適用)

被告人の判示所為は、刑法一九九条に該当するところ、所定刑中有期懲役刑を選択し、右は心神耗弱者の行為であるから同法三九条二項、六八条三号により法律上の減軽をした刑期の範囲内で被告人を懲役七年に処し、同法二一条を適用して原審における未決勾留日数中六五〇日を右刑に算入し、原審及び当審における訴訟費用については刑訴法一八一条一項本文により被告人に負担させることとして、主文のとおり判決する。

(量刑の理由)

本件は、被告人が、シンナーを吸引するための場所を求めて訪問した先の市街地にある団地の一室において、深夜、一人で留守番をしていた女子中学生といさかいになって、シンナー吸入による影響もあって激高の余り、同女に対し、ガラス製灰皿でその頭部を数回殴打し、電気コードで首を絞め、包丁でその顔面、頸部、胸部等をめった突きにするなどして同女を殺害したものであるが、短気な性格のうえに従来からのシンナーへのたんできが原因であって、動機につき酌量すべき事情は認められないばかりか、執拗で残虐な行為態様に照らせば、犯行は悪質かつ非道というほかなく、凄惨をきわめた遺体の状況には目を覆わざるをえず、中学の卒業式を当日終え、高校進学を目前にしていた矢先に突如として一命を奪われた被害者及びその遺族の無念は察するに十分である。それにもかかわらず、被告人は不合理な弁解を繰り返して本件犯人であることを否認するなど反省悔悟の態度が全く認められないばかりか、遺族に対し何ら慰謝の措置を講じておらず、遺族の被害感情は今なお強烈であると考えられる。また、本件犯行が市街地内の団地で敢行されたことは、団地住民のみならず、地域社会にも衝撃を与え、その不安と恐怖は深刻であったこともうかがわれる。これら諸点に、これまでに何度も入院する等してシンナーの害悪を知悉しながら、定職にも就かず不良交遊の中で又もやシンナーにたんできしていた当時の被告人の生活態度をも勘案すると、被告人の刑事責任は重大である。しかしながら、他方、本件が計画的な犯行ではないことや被告人の年齢、自由刑に処せられるのは今回が初めてであることなど被告人にとって有利ないし斟酌すべき事情をも併せ考慮して、有期懲役刑を選択した上、犯行時心神耗弱の状態にあったから法律上の減軽をし、被告人を主文掲記の懲役七年に処するのを相当と判断した。

(裁判長裁判官 小島裕史 裁判官 松尾昭彦 裁判官 田中敦)
\end{quote}

\begin{quote}
《引用の終わり》
\end{quote}

\begin{itemize}
\tightlist
\item
  名古屋高等裁判所金沢支部 平成2年(う)71号 判決 - 大判例
  \url{https://daihanrei.com/l/\%E5\%90\%8D\%E5\%8F\%A4\%E5\%B1\%8B\%E9\%AB\%98\%E7\%AD\%89\%E8\%A3\%81\%E5\%88\%A4\%E6\%89\%80\%E9\%87\%91\%E6\%B2\%A2\%E6\%94\%AF\%E9\%83\%A8\%20\%E5\%B9\%B3\%E6\%88\%90\%EF\%BC\%92\%E5\%B9\%B4\%EF\%BC\%88\%E3\%81\%86\%EF\%BC\%89\%EF\%BC\%97\%EF\%BC\%91\%E5\%8F\%B7\%20\%E5\%88\%A4\%E6\%B1\%BA} 
\end{itemize}

\end{document}
