% Options for packages loaded elsewhere
\PassOptionsToPackage{unicode}{hyperref}
\PassOptionsToPackage{hyphens}{url}
%
\documentclass[
]{ltjarticle}
\usepackage{lmodern}
\usepackage{amssymb,amsmath}
\usepackage{ifxetex,ifluatex}
\ifnum 0\ifxetex 1\fi\ifluatex 1\fi=0 % if pdftex
  \usepackage[T1]{fontenc}
  \usepackage[utf8]{inputenc}
  \usepackage{textcomp} % provide euro and other symbols
\else % if luatex or xetex
  \usepackage{unicode-math}
  \defaultfontfeatures{Scale=MatchLowercase}
  \defaultfontfeatures[\rmfamily]{Ligatures=TeX,Scale=1}
\fi
% Use upquote if available, for straight quotes in verbatim environments
\IfFileExists{upquote.sty}{\usepackage{upquote}}{}
\IfFileExists{microtype.sty}{% use microtype if available
  \usepackage[]{microtype}
  \UseMicrotypeSet[protrusion]{basicmath} % disable protrusion for tt fonts
}{}
\makeatletter
\@ifundefined{KOMAClassName}{% if non-KOMA class
  \IfFileExists{parskip.sty}{%
    \usepackage{parskip}
  }{% else
    \setlength{\parindent}{0pt}
    \setlength{\parskip}{6pt plus 2pt minus 1pt}}
}{% if KOMA class
  \KOMAoptions{parskip=half}}
\makeatother
\usepackage{xcolor}
\IfFileExists{xurl.sty}{\usepackage{xurl}}{} % add URL line breaks if available
\IfFileExists{bookmark.sty}{\usepackage{bookmark}}{\usepackage{hyperref}}
\hypersetup{
  hidelinks,
  pdfcreator={LaTeX via pandoc}}
\urlstyle{same} % disable monospaced font for URLs
\usepackage[left=3cm,right=2cm,top=3.5cm,bottom=2.7cm]{geometry}
\setlength{\emergencystretch}{3em} % prevent overfull lines
\providecommand{\tightlist}{%
  \setlength{\itemsep}{0pt}\setlength{\parskip}{0pt}}
\setcounter{secnumdepth}{5}
\ifluatex
  \usepackage{selnolig}  % disable illegal ligatures
\fi

\author{}
\date{}

\begin{document}

{
\setcounter{tocdepth}{4}
\tableofcontents
}
 高検検事を廃棄物中間処理場とした落合洋司弁護士(東京弁護士会)のことで,一緒に思い出したのが,その秋田の弁護士刺殺に関する落合洋司弁護士(東京弁護士会)の過去のツイートでした。けっこうな数見かけていたと思って,Twitter検索しました。

\begin{itemize}
\tightlist
\item
  ``秋田'' (from:yjochi) - Twitter検索 / Twitter
  https://twitter.com/search?lang=ja\&q=\%22\%E7\%A7\%8B\%E7\%94\%B0\%22\%20(from\%3Ayjochi)\&src=typed\_query
\end{itemize}

 検索ボックスが見当たらないと思っていた落合洋司弁護士(東京弁護士会)のブログですが,どうもレイアウトが崩れて,本来,左サイドに来るブロック要素が,下に落ちていました。HTMLを編集し,タグの閉じ忘れでよく起こる現象です。過去に何度か経験があります。

 その落合洋司弁護士(東京弁護士会)のブログの方で,秋田を検索し始めた頃,ちょっとある場所に目を向けると,すぐに2021年の能登町の電話帳が見つかり,そのまま作業の中断となったのです。

〉〉〉 kk\_hironoのリツイート 〉〉〉\\
- RT
kk\_hirono(刑事告発・非常上告_金沢地方検察庁御中)|s\_hirono(非常上告-最高検察庁御中\_ツイッター)
日時:2021-02-09 08:06/2021/02/09 08:05 URL:
https://twitter.com/kk\_hirono/status/1358915091563900929
https://twitter.com/s\_hirono/status/1358914996688674817\\
\textgreater{}
2021-02-09-080453\_検索ボックスが見当たらないと思っていた落合洋司弁護士(東京弁護士会)のブログですが,どうもレイアウトが崩れて,本来,左サイドに来るブロック要.jpg
https://t.co/e6lYM1IxtH

\begin{itemize}
\tightlist
\item
  元検事の弁護士です(お問い合わせ→https://t.co/YxQlCZWV2A) -
  弁護士落合洋司(東京弁護士会)の日々是好日 https://t.co/vih8WJT8Yi \n
  2064-03-31
\end{itemize}

 よく考えてみると,落合洋司弁護士(東京弁護士会)がHTMLを編集することはなさそうです。トップの記事の日付が2064年という遥か未来となっていますが,

\end{document}
