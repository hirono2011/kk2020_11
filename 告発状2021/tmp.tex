% Options for packages loaded elsewhere
\PassOptionsToPackage{unicode}{hyperref}
\PassOptionsToPackage{hyphens}{url}
%
\documentclass[
]{ltjarticle}
\usepackage{amsmath,amssymb}
\usepackage{lmodern}
\usepackage{iftex}
\ifPDFTeX
  \usepackage[T1]{fontenc}
  \usepackage[utf8]{inputenc}
  \usepackage{textcomp} % provide euro and other symbols
\else % if luatex or xetex
  \usepackage{unicode-math}
  \defaultfontfeatures{Scale=MatchLowercase}
  \defaultfontfeatures[\rmfamily]{Ligatures=TeX,Scale=1}
\fi
% Use upquote if available, for straight quotes in verbatim environments
\IfFileExists{upquote.sty}{\usepackage{upquote}}{}
\IfFileExists{microtype.sty}{% use microtype if available
  \usepackage[]{microtype}
  \UseMicrotypeSet[protrusion]{basicmath} % disable protrusion for tt fonts
}{}
\makeatletter
\@ifundefined{KOMAClassName}{% if non-KOMA class
  \IfFileExists{parskip.sty}{%
    \usepackage{parskip}
  }{% else
    \setlength{\parindent}{0pt}
    \setlength{\parskip}{6pt plus 2pt minus 1pt}}
}{% if KOMA class
  \KOMAoptions{parskip=half}}
\makeatother
\usepackage{listings}
\lstset{
language=sh,
basicstyle=\ttfamily\scriptsize,
commentstyle=\textit,
classoffset=1,
keywordstyle=\bfseries,
frame=tRBl,
framesep=5pt,
showstringspaces=false,
%numbers=left,
%stepnumber=1,
%numberstyle=\tiny,
tabsize=2,
breaklines = true,
}

% \input{/home/a66/Dropbox/macro.tex}

\usepackage{xcolor}
\IfFileExists{xurl.sty}{\usepackage{xurl}}{} % add URL line breaks if available
\IfFileExists{bookmark.sty}{\usepackage{bookmark}}{\usepackage{hyperref}}
\hypersetup{
  pdftitle={(インターネットの公開資料)},
  pdfauthor={廣野秀樹},
  hidelinks,
  pdfcreator={LaTeX via pandoc}}
\urlstyle{same} % disable monospaced font for URLs
\usepackage[left=3cm,right=2cm,top=3.5cm,bottom=2.7cm]{geometry}
\setlength{\emergencystretch}{3em} % prevent overfull lines
\providecommand{\tightlist}{%
  \setlength{\itemsep}{0pt}\setlength{\parskip}{0pt}}
\setcounter{secnumdepth}{5}
\ifLuaTeX
  \usepackage{selnolig}  % disable illegal ligatures
\fi

\title{(インターネットの公開資料)}
\author{廣野秀樹}
\date{2021-07-22}

\begin{document}
\maketitle
\fontsize{16pt}{0cm}\selectfont
\begin{center}
告発状
\par\end{center}
\vspace{5mm}
\fontsize{11pt}{11pt}\selectfont
\vspace{3mm}
\begin{flushright}
令和3年○月○日
\par\end{flushright}
〒920-0912金沢市大手町6番15号 金沢地方検察庁御中

\fontsize{10pt}{10pt}\selectfont
\vspace{3mm}
\hspace{90mm}被告発人 金沢弁護士会所属 木梨松嗣弁護士

\hspace{90mm}被告発人 金沢弁護士会所属 岡田進弁護士

\hspace{90mm}被告発人 金沢弁護士会所属 長谷川紘之弁護士 

\hspace{90mm}被告発人 金沢弁護士会所属 若杉幸平弁護士

\hspace{90mm}被告発人 元名古屋高裁金沢支部裁判長 小島裕史 

\hspace{90mm}被告発人 元金沢地方裁判所裁判官 古川龍一

\hspace{90mm}被告発人 松平日出男

\hspace{90mm}被告発人 池田宏美 

\hspace{90mm}被告発人 梅野博之

\hspace{90mm}被告発人 安田繁克 

\hspace{90mm}被告発人 安田敏 

\hspace{90mm}被告発人 東渡好信 

\hspace{90mm}被告発人 多田敏明 

\hspace{90mm}被告発人 浜口卓也 

\hspace{90mm}被告発人 大網健二

\hspace{90mm}

\fontsize{11pt}{11pt}\selectfont
告発人\\
\\
〒927-0431 石川県鳳珠郡能登町宇出津山分10-3 \hspace{60mm}廣野秀樹

\vspace{6mm}
\begin{center}
記
\par\end{center}
\vspace{1mm}

\section{告発の趣旨}
 被告発人らの所為は,市場急配センター(所在地: 〒920-0025 石川県金沢市駅西本町5丁目10−20)における殺人未遂(刑法第203条,第199条)の共謀共同正犯(刑法第60条)として法的評価すべきもの,また,弁護士,裁判官らの立場と職権で隠ぺいした片面的事後共犯であると思料するので,犯情甚だ悪質につき,無期懲役刑として処罰することを求め,ここに告発に及びます。
\clearpage

{
\setcounter{tocdepth}{4}
\tableofcontents
}
\hypertarget{ux544aux767aux306bux81f3ux308bux7d4cux7def}{%
\section{告発に至る経緯}\label{ux544aux767aux306bux81f3ux308bux7d4cux7def}}

\hypertarget{ux91d1ux6ca2ux5730ux65b9ux691cux5bdfux5e81}{%
\subsection{金沢地方検察庁}\label{ux91d1ux6ca2ux5730ux65b9ux691cux5bdfux5e81}}

\hypertarget{ux4ee4ux548cuxff13ux5e74uxff13ux6708uxff13uxff11ux65e5ux4ed8ux544aux767aux72b6ux306bux3064ux3044ux3066}{%
\subsubsection{令和3年3月31日付告発状について}\label{ux4ee4ux548cuxff13ux5e74uxff13ux6708uxff13uxff11ux65e5ux4ed8ux544aux767aux72b6ux306bux3064ux3044ux3066}}

\hypertarget{ux30c7ux30a4ux30eaux30fcux65b0ux6f6eux306e330ux5186ux306eux6709ux6599ux8a18ux4e8bux3092ux77e5ux308bux304dux3063ux304bux3051ux3068ux306aux3063ux305fux5c0fux5ba4ux3055ux3093ux6cd5ux5f8bux306bux3053ux3060ux308fux308aux3059ux304eux5f01ux8b77ux58ebux305fux3061ux304bux3089ux306eux30a2ux30c9ux30d0ux30a4ux30b9ux3068ux3044ux3046ux5f01ux8b77ux58ebux30c9ux30c3ux30c8ux30b3ux30e0ux306eux8a18ux4e8b}{%
\paragraph{デイリー新潮の330円の有料記事を知る,きっかけとなった「「小室さん、法律にこだわりすぎ」弁護士たちからのアドバイス」という弁護士ドットコムの記事}\label{ux30c7ux30a4ux30eaux30fcux65b0ux6f6eux306e330ux5186ux306eux6709ux6599ux8a18ux4e8bux3092ux77e5ux308bux304dux3063ux304bux3051ux3068ux306aux3063ux305fux5c0fux5ba4ux3055ux3093ux6cd5ux5f8bux306bux3053ux3060ux308fux308aux3059ux304eux5f01ux8b77ux58ebux305fux3061ux304bux3089ux306eux30a2ux30c9ux30d0ux30a4ux30b9ux3068ux3044ux3046ux5f01ux8b77ux58ebux30c9ux30c3ux30c8ux30b3ux30e0ux306eux8a18ux4e8b}}

\begin{itemize}
\tightlist
\item
  〉〉〉 Linux Emacs: 2021/04/24 10:48:32 〉〉〉
\end{itemize}

:CATEGORIES: @kanazawabengosi \#金沢弁護士会 @JFBAsns
日本弁護士連合会(日弁連) \#法務省 @MOJ\_HOUMU \#弁護士ドットコム

 忘れていたところ,ブラウザの戻るボタンで確認すると,次の記事が最初のきっかけで,Twitterのトレンドで見ていたことを思い出しました。

\begin{itemize}
\tightlist
\item
  カエルの声は騒音か 訴え退ける - Yahoo!ニュース
  \url{https://news.yahoo.co.jp/pickup/6391497} 
\end{itemize}

 トレンドに東京地裁とあったので,東京地裁の建物や敷地内でカエルが発見されたのかと最初に想像したのですが,民事裁判のことでした。カエルで思い出すのは,長野県の諏訪大社のご神事と,御柱祭の死亡事故を告発した弁護士らのことです。

\begin{itemize}
\tightlist
\item
  ---- ¥\n 保育園ができると、子供の声がうるさいから作るな。
  除夜の鐘の音がうるさいから、大晦日の昼間に撞け。
  盆踊りの民謡がうるさいから、イヤホンで聴きながら踊れ。
  嫌な世の中になったもんだ。 ¥\n \#Yahooニュースのコメント
  \url{https://t.co/lNoxOCmkoM} 
\end{itemize}

 上記のコメントも1つの発見でした。保育所の件は,淡路島の弁護士と能登高校の女子高校生殺害事件との一つのつながりが2017年3月にありました。いずれも弁護士が拍車をかけているような印象のある社会問題です。

 twilog-serch で屋根を検索すると次のツイートが出てきました。

〉〉〉 kk\_hironoのリツイート 〉〉〉

\begin{itemize}
\tightlist
\item
  RT
  kk\_hirono(刑事告発・非常上告_金沢地方検察庁御中)|Hideo\_Ogura(小倉秀夫)
  日時:2021-04-24 11:04/2012/02/09 21:47 URL:
  \url{https://twitter.com/kk\_hirono/status/1385776666450952198} 
  \url{https://twitter.com/Hideo\_Ogura/status/167590539754090497} 
  \textgreater{}
  私たち弁護士が、子どもに義務教育より上の教育を受けさせたり、屋根の下で暮らしたりできている間は、既得権!既得権!という糾弾が収まる気配なんてありやしないんだし。
\end{itemize}

〉〉〉 kk\_hironoのリツイート 〉〉〉

\begin{itemize}
\tightlist
\item
  RT
  kk\_hirono(刑事告発・非常上告_金沢地方検察庁御中)|Hideo\_Ogura(小倉秀夫)
  日時:2021-04-24 11:05/2013/03/03 12:49 URL:
  \url{https://twitter.com/kk\_hirono/status/1385776900065284100} 
  \url{https://twitter.com/Hideo\_Ogura/status/308061649313611778} 
  \textgreater{}
  弁護士が屋根の下に住み、市場で売られている食材で作られた料理を食べ、寒さをしのげる衣服を着て生活するコストも、不当に市場に転嫁されていることに法哲学的にはなるんだろうなあ。
\end{itemize}

〉〉〉 kk\_hironoのリツイート 〉〉〉

\begin{itemize}
\tightlist
\item
  RT
  kk\_hirono(刑事告発・非常上告_金沢地方検察庁御中)|Hideo\_Ogura(小倉秀夫)
  日時:2021-04-24 11:06/2014/12/27 21:55 URL:
  \url{https://twitter.com/kk\_hirono/status/1385777044911394819} 
  \url{https://twitter.com/Hideo\_Ogura/status/548824541732950016} 
  \textgreater{}
  弁護士が屋根の下で暮らせること自体おかしいという前提に立てば、国選報酬は1万円でももらいすぎと言われそうですね。RT
  @kaien\_law:
  この辺の話も、弁護士様の保証されるべき年収の議論が背景にありますよね。RT
  \emph{hznf}:
\end{itemize}

〉〉〉 kk\_hironoのリツイート 〉〉〉

\begin{itemize}
\tightlist
\item
  RT
  kk\_hirono(刑事告発・非常上告_金沢地方検察庁御中)|Hideo\_Ogura(小倉秀夫)
  日時:2021-04-24 11:06/2015/06/28 12:43 URL:
  \url{https://twitter.com/kk\_hirono/status/1385777249010421766} 
  \url{https://twitter.com/Hideo\_Ogura/status/615002517470994433} 
  \textgreater{}
  法科大学院では、「弁護士が屋根の下で起臥寝食し、市場に流通している食材を食して生きていくなんて、国民が許さない」って教わりませんでしたか。RT
  @bengoshi\_black:
  そこから奨学金と修習の貸与金を引くと月収17万円くらいかな?
  それを希望するのが、贅沢ですか?
\end{itemize}

 探していたのはスクリーンショットになりますが,次のツイートになります。

〉〉〉 kk\_hironoのリツイート 〉〉〉

\begin{itemize}
\tightlist
\item
  RT
  kk\_hirono(刑事告発・非常上告_金沢地方検察庁御中)|s\_hirono(非常上告-最高検察庁御中\_ツイッター)
  日時:2021-04-24 11:09/2017/03/02 20:39 URL:
  \url{https://twitter.com/kk\_hirono/status/1385777943771635713} 
  \url{https://twitter.com/s\_hirono/status/837266130670518272} 
  \textgreater{}
  2017-03-02-203931\_屋根にペンキででっかく名前書いとくべきだったなあw.jpg
  \url{https://t.co/vpsEm9RuZM} 
\end{itemize}

 リツイートの直前になって気が付き,見落とした可能性もあるのですが,北海道旭川市の中村元弥弁護士のリツイートとして記録したスクリーンショットでした。

\begin{itemize}
\tightlist
\item
  ---- ¥\n
  【佐藤みのりさんのコメント】ご近所同士の騒音トラブルは、時々、こうして裁判になることがあります。裁判では、「受忍限度」を超えるか否かが問題となります。「受忍限度」という\ldots{}
  ¥\n \#Yahooニュースのコメント \url{https://t.co/SUNxQFzSzK} 
\end{itemize}

 32分前という表示がありますが,一番上に弁護士のコメントがあることに気が付きました。名前も女性らしいですが髪の長い女性らしい顔写真のアイコンもあります。

 「こんな案件に弁護士入れて裁判したのか?」という一部分の表示のコメントで気がついたのですが,ボタンに「もっと見る(4358件)とありました。コメントのTwitterのボタンが,そのままコメント本文の引用ツイートになっていることも今日初めて知ったように思います。

\begin{itemize}
\tightlist
\item
  ---- ¥\n こんな案件に弁護士入れて裁判したのか?
  弁護士は何もアドバイスしなかったのかそれとも勝てると見込んで訴えたのか?
  こんなことで勝訴するならうちの近所の人はみんな田植えの季節にうるさいと訴えますよ。
  日\ldots{} ¥\n \#Yahooニュースのコメント \url{https://t.co/tyFh7toAEC} 
\end{itemize}

 いつものように記事のページの右側にアクセスランキングなどの記事の見出しの一覧があるのですが,次のようにリンクを辿りました。

\begin{quote}
《引用の始まり》
\end{quote}

\begin{quote}
神奈川県出身。中学時代、友人の非行がきっかけで、少年事件に携わりたいとの思いから弁護士を志す。2012年3月、慶応義塾大学大学院法務研究科修了後、同年9月に司法試験に合格。2015年5月、佐藤みのり法律事務所開設。少年非行、いじめ、児童虐待に関する活動に参加し、いじめに関する第三者委員やいじめ防止授業の講師、日本弁護士連合会(日弁連)主催の小中高校生向け社会科見学講師を務めるなど、現代の子どもと触れ合いながら、子どもの問題に積極的に取り組む。弁護士活動の傍ら、ニュース番組の取材協力、執筆活動など幅広く活動。女子中高生の性の問題、学校現場で起こるさまざまな問題などにコメントしている。
\end{quote}

\begin{quote}
《引用の終わり》
\end{quote}

\begin{itemize}
\tightlist
\item
  佐藤みのりさんのページ \textbar{} Yahoo!ニュース
  \url{https://news.yahoo.co.jp/profile/commentator/satominori/comments} 
\end{itemize}

 上記のプロフィールのページを開いたためか,ブラウザの戻るボタンの履歴が全て消えていました。佐藤みのり弁護士ですが,Twitterでは見ていない弁護士かと思います。顔写真には見覚えがありますが,注目することはなかったとも思います。これも確認をしておきます。

 「twilog-serch 佐藤みのり」の結果はゼロでした。意外な結果です。

 uで更新すると,結果が4件になりました。uはコマンドのエイリアス(別名)でwhichコマンドの結果は,「u:
aliased to update-twitterAPI-to-twilog\_csv\_text」となっています。

 リストに追加するコマンドを実行すると,「\textgreater\textgreater\textgreater\textgreater\textgreater\textgreater\textgreater\textgreater\textgreater\textgreater\textgreater\textgreater\textgreater\textgreater\textgreater\textgreater\textgreater\textgreater\textgreater\textgreater\textgreater\textgreater\textgreater\textgreater\textgreater\textgreater\textgreater\textgreater\textgreater\textgreater\textgreater\textgreater\textgreater{}
\url{https://twitter.com/minoripost/status/1097753848343539712} を
追加しました」という結果でした。これまで未登録だったことになります。次が固定されたツイートです。

〉〉〉 kk\_hironoのリツイート 〉〉〉

\begin{itemize}
\tightlist
\item
  RT
  kk\_hirono(刑事告発・非常上告_金沢地方検察庁御中)|minoripost(弁護士
  佐藤みのり) 日時:2021-04-24 11:38/2019/02/19 16:04 URL:
  \url{https://twitter.com/kk\_hirono/status/1385785300698353664} 
  \url{https://twitter.com/minoripost/status/1097753848343539712} 
  \textgreater{} 著書が、タイトル新たに新書判でも発売されることに!
  受験合格や夢の実現に向けてがんばる方に読んでいただけたらうれしいです。
  「自己流こそが最大の武器」 ただいま、Amazonにて予約受付中♪
  \url{https://t.co/3F6Nsdl4Xx}  \#勉強法 \#大学受験 \#自己流 \#自己分析 \#独学
\end{itemize}

 奉納\さらば弁護士鉄道・泥棒神社の物語(@hirono\_hideki)のアカウントでもリツイートが出来ました。今まで目に触れることなかった実名弁護士Twitterアカウントになるので,ずっと前にブロックされていた可能性というのも考えました。

 リンクにアメブロのブログがあって,プロフィールに「自己紹介:神奈川県立希望ヶ丘高校卒業
慶應義塾大学法学部政治学科卒業(首席)
」などとありました。才女という印象ですが,Twitterのフォロワー数が278と意外に少なく,その辺りのギャップも気になっています。

 さきほどざっと目を通したときは気が付かなかったようですが,Twitterのプロフィールにも「(首席)」とありました。Yahooニュースのコメント一覧にも旭川市の少女自殺問題など参考になるものがあるので,のちほど個別に取り上げておきたい佐藤みのり弁護士です。

 ブラウザのリンクは辿れなくなったのですが,いずれも奉納\さらば弁護士鉄道・泥棒神社の物語(@hirono\_hideki)でツイートをしていたと思います。なかなか線の濃い,巡り合わせのようなつながりがありました。

〉〉〉 kk\_hironoのリツイート 〉〉〉

\begin{itemize}
\tightlist
\item
  RT
  kk\_hirono(刑事告発・非常上告_金沢地方検察庁御中)|hirono\_hideki(奉納\さらば弁護士鉄道・泥棒神社の物語)
  日時:2021-04-24 11:56/2021/04/24 10:42 URL:
  \url{https://twitter.com/kk\_hirono/status/1385789765853667330} 
  \url{https://twitter.com/hirono\_hideki/status/1385771023912488964} 
  \textgreater{} - 小室圭さん母「年金詐取」計画 口止めメール
  28枚文書で判明【先出し全文】(文春オンライン) - Yahoo!ニュース
  \url{https://t.co/l64zXucqXJ} 
\end{itemize}

〉〉〉 kk\_hironoのリツイート 〉〉〉

\begin{itemize}
\tightlist
\item
  RT
  kk\_hirono(刑事告発・非常上告_金沢地方検察庁御中)|hirono\_hideki(奉納\さらば弁護士鉄道・泥棒神社の物語)
  日時:2021-04-24 11:56/2021/04/24 10:40 URL:
  \url{https://twitter.com/kk\_hirono/status/1385789784153460741} 
  \url{https://twitter.com/hirono\_hideki/status/1385770543249465347} 
  \textgreater{} -
  「客を慰めることで自分も癒やされた」 震災直後、デリヘル嬢たちはなぜ仕事に戻ったのか【衝撃ルポ】(デイリー新潮)
  - Yahoo!ニュース \url{https://t.co/mPzNod0zrx} 
\end{itemize}

〉〉〉 kk\_hironoのリツイート 〉〉〉

\begin{itemize}
\tightlist
\item
  RT
  kk\_hirono(刑事告発・非常上告_金沢地方検察庁御中)|hirono\_hideki(奉納\さらば弁護士鉄道・泥棒神社の物語)
  日時:2021-04-24 11:56/2021/04/24 10:38 URL:
  \url{https://twitter.com/kk\_hirono/status/1385789818982920193} 
  \url{https://twitter.com/hirono\_hideki/status/1385770074137530368} 
  \textgreater{} -
  ``事件屋''一味にしゃぶり尽くされた大阪の名刹「正圓寺」 「20億円」で寺の土地を勝手に売りに出され刑事告訴(デイリー新潮)
  - Yahoo!ニュース \url{https://t.co/p3Eyvvypxg} 
\end{itemize}

〉〉〉 kk\_hironoのリツイート 〉〉〉

\begin{itemize}
\tightlist
\item
  RT
  kk\_hirono(刑事告発・非常上告_金沢地方検察庁御中)|hirono\_hideki(奉納\さらば弁護士鉄道・泥棒神社の物語)
  日時:2021-04-24 11:57/2021/04/24 10:24 URL:
  \url{https://twitter.com/kk\_hirono/status/1385789846547963905} 
  \url{https://twitter.com/hirono\_hideki/status/1385766685009793024} 
  \textgreater{} -
  「小室さん、法律にこだわりすぎ」弁護士たちからのアドバイス「答案を書いても解決しないよ」(弁護士ドットコムニュース)
  - Yahoo!ニュース \url{https://t.co/GtxuYCpVXo} 
\end{itemize}

〉〉〉 kk\_hironoのリツイート 〉〉〉

\begin{itemize}
\tightlist
\item
  RT
  kk\_hirono(刑事告発・非常上告_金沢地方検察庁御中)|hirono\_hideki(奉納\さらば弁護士鉄道・泥棒神社の物語)
  日時:2021-04-24 11:57/2021/04/24 10:21 URL:
  \url{https://twitter.com/kk\_hirono/status/1385789877111988226} 
  \url{https://twitter.com/hirono\_hideki/status/1385765865497399298} 
  \textgreater{} - 最高裁 飯塚事件の再審認めず - Yahoo!ニュース
  \url{https://t.co/NO7JO0X9JY} 
\end{itemize}

〉〉〉 kk\_hironoのリツイート 〉〉〉

\begin{itemize}
\tightlist
\item
  RT
  kk\_hirono(刑事告発・非常上告_金沢地方検察庁御中)|hirono\_hideki(奉納\さらば弁護士鉄道・泥棒神社の物語)
  日時:2021-04-24 11:57/2021/04/24 10:21 URL:
  \url{https://twitter.com/kk\_hirono/status/1385789907247923201} 
  \url{https://twitter.com/hirono\_hideki/status/1385765765475749888} 
  \textgreater{} - カエルの声は騒音か 訴え退ける - Yahoo!ニュース
  \url{https://t.co/IVEjQDFkr5} 
\end{itemize}

〉〉〉 kk\_hironoのリツイート 〉〉〉

\begin{itemize}
\tightlist
\item
  RT
  kk\_hirono(刑事告発・非常上告_金沢地方検察庁御中)|nhk\_news(NHKニュース)
  日時:2021-04-24 11:57/2021/04/23 21:22 URL:
  \url{https://twitter.com/kk\_hirono/status/1385789994606858240} 
  \url{https://twitter.com/nhk\_news/status/1385569680237203459} 
  \textgreater{}
  「隣の庭のカエルがうるさい」として住民が騒音の差し止めやカエルの駆除を求めた訴えについて、東京地方裁判所は「カエルの鳴き声は自然音で騒音には当たらない」として退ける判決を言い渡しました。
  \url{https://t.co/iZawRinkOU}  \#nhk\_video \url{https://t.co/RCthw1nwcy} 
\end{itemize}

 奉納\さらば弁護士鉄道・泥棒神社の物語(@hirono\_hideki)のタイムラインをみると,最初に税込み330円の記事のリンクを開いたのは,次の記事になるのかと思います。

\begin{itemize}
\item
  ``事件屋''一味にしゃぶり尽くされた大阪の名刹「正圓寺」 「20億円」で寺の土地を勝手に売りに出され刑事告訴(デイリー新潮)
  - Yahoo!ニュース \url{https://t.co/6hQ5d1pU50}  記事提供終了日:2021/12/19(日)
  ¥\n 4/23(金) 10:00配信
\item
  〈〈〈 2021/04/24 12:02:40 Linux Emacs: 〈〈〈
\item
  〉〉〉 :Linux Atom: 2021-04-24 13:19  〉〉〉
\item
  「小室さん、法律にこだわりすぎ」弁護士たちからのアドバイス「答案を書いても解決しないよ」(弁護士ドットコムニュース)
  - Yahoo!ニュース \url{https://t.co/6KsTSfWGhM} 
\end{itemize}

\begin{quote}
《引用の始まり》
\end{quote}

\begin{quote}
●9割が文書を「評価しない」

弁護士ドットコムは4月16日から21日にかけて、小室圭さんの文書公表についての見解を募集。登録弁護士31人からの回答が寄せられた。

文書については、およそ9割が「評価しない」。また全体のおよそ8割が、金銭トラブルは解決しないとの見方を示した。

具体的には、「ボタンの掛け違いが進みすぎたので、もう挽回は厳しい」「相手の気持ちを考えない交渉はあり得ない」というものだ。
\end{quote}

\begin{quote}
《引用の終わり》
\end{quote}

\begin{itemize}
\tightlist
\item
  「小室さん、法律にこだわりすぎ」弁護士たちからのアドバイス「答案を書いても解決しないよ」(弁護士ドットコムニュース)
  - Yahoo!ニュース
  \url{https://news.yahoo.co.jp/articles/7aa4e07eadc46a7c6f7a7ef6a3a30bb3104bbb95} 
\end{itemize}

\begin{quote}
《引用の始まり》
\end{quote}

\begin{quote}
「紛争の実態によっては、法的な解決手法が必ずしも最善とはいえない」

「法律万能主義では何事も解決しない。法律を駆使すれば解決は遠のくことを思い知ったのではないか」

弁護士は、法律の専門家であるとともに、紛争を現実的に解決する「トラブルシューター」「実務家」でもある。

そうした見地からすれば、一連の対応は「悪手」で、公表された文書も現実離れした「答案」に過ぎないと言うことなのだろう。

「多くの国民は、金員の交付が貸付けか、贈与かといった法的議論ではなく、小室さん母子の元婚約者に対する忘恩行為を批判している」

「法的な判断、解決が求められている場面ではない。ご自分の正当性を主張するのではなく、思いやり、徳の高さを見せるべきだった」

など、法律論に拘泥するあまり、方向性を誤ったという意見が多く見られる。
\end{quote}

\begin{quote}
《引用の終わり》
\end{quote}

\begin{itemize}
\tightlist
\item
  「小室さん、法律にこだわりすぎ」弁護士たちからのアドバイス「答案を書いても解決しないよ」(弁護士ドットコムニュース)
  - Yahoo!ニュース
  \url{https://news.yahoo.co.jp/articles/7aa4e07eadc46a7c6f7a7ef6a3a30bb3104bbb95} 
\end{itemize}

 上記に引用をしましたが,「法律を駆使すれば解決は遠のくことを思い知ったのではないか」,「ご自分の正当性を主張するのではなく、思いやり、徳の高さを見せるべきだった」などとあります。

 数年前からの社会の関心事ですが,どちらの側にも背後に弁護士がいて,その弁護士がいいように操縦をしているのでは,と思えていたところです。株式にも相場を変動させる操縦のようなものがあると聞きますが,マスコミが疑いもなく追従しているように見えるのも,由々しきところと考えていました。

\begin{itemize}
\tightlist
\item
  小室圭さん母「年金詐取」計画 口止めメール
  28枚文書で判明【先出し全文】(文春オンライン) - Yahoo!ニュース
  \url{https://t.co/TtWpWzXrCG}  ¥\n 記事提供終了日:2022/4/16(土) ¥\n
  4/21(水) 16:00配信
\end{itemize}

 有料記事として最初に目についたのが上記の記事であったように思います。「本文:6,924文字 写真:10枚」ともあります。

\begin{quote}
《引用の始まり》
\end{quote}

\begin{quote}
元婚約者の金銭提供を贈与と主張してきた佳代さんだが、報道されるまで贈与税を支払っていなかった。経緯を検証すると、佳代さんが遺族年金を詐取しようとしていた疑惑が。婚約直後、元婚約者に送ったメールには綿密な計画と共に「内密にして頂きたい」――。
\end{quote}

\begin{quote}
《引用の終わり》
\end{quote}

\begin{itemize}
\tightlist
\item
  小室圭さん母「年金詐取」計画 口止めメール
  28枚文書で判明【先出し全文】(文春オンライン) - Yahoo!ニュース
  \url{https://news.yahoo.co.jp/articles/8514f183f2887c972d0472ead5145bcb3545027f?source=pc-paiddetail-subcolumnmore} 
\end{itemize}

 上記の引用が公開部分になりますが,貼り付けた後にEmacsの自作のコマンドで文字数をカウントすると120文字でした。記事でみると文字数が多めに見えて,140文字のツイートにはぎりぎりの文字数になりそうだと見込んでいました。

 こちらはYahooニュースでも,文春オンラインの記事のようです。先程見ていた有料記事はデイリー新潮となっていたので,330円という有料記事の価格設定はデイリー新潮のものと考えていたのですが,大きな勘違いをしていたようです。

 330円といえば,一昔前の週刊新潮,週刊文春の価格に相当しそうです。はっきりとした記憶はないですが,平成4年当時も週刊誌は300円ぐらいが相場で,200円台というのは見ていなかったかもしれません。なにぶん昔のことなのでずいぶん曖昧な記憶とはなっています。

 紙媒体としての流通もないのに記事の1つが税込みで330円というのは,途方もない収益にも繋がりそうですし,情報の持つ価値,値段ということにも改めて考えさせられるところがあります。

 実際,330円で記事を購読している人の数は多くはないような気がしますが,1つでも当たれば大きいということで,弁護士商売の本質にも繋がっていると思え,マスコミと弁護士の一体化のようなものをあらためて強く感じたところです。

 必ずしも記事の有料化に反対し異議を唱えているつもりはないのですが,逆にこれまで長い間,無料で記事を読めていたことが不思議なぐらいで,一つ参考になったのが弁護士ドットコムの仕組みでした。弁護士ドットコムも記者を雇って記事を配信しているようです。

 今朝は,まずはじめに先日入会したAmazonプライムのビデオと,別の有料配信で視聴した映画のことを取り上げるつもりだったのですが,決済が出来ず断念した映画の視聴も価格がレンタルで330円になっていたような記憶です。別のサービスで407円ほどで視聴しました。

\begin{itemize}
\tightlist
\item
  〈〈〈 2021/04/24 13:56:52 Linux Emacs: 〈〈〈
\end{itemize}

\hypertarget{ux6170ux8b1dux6599uxff12uxff10ux4e07ux5186ux306bux5bfeux3057ux3066ux5f01ux8b77ux58ebux8cbbux7528uxff16uxff10ux4e07ux5186ux4f59ux308aux3092ux8a8dux5bb9ux3067ux5408ux8a08uxff18uxff10ux4e07ux5186ux8d85ux304bux753bux671fux7684ux3060ux3068ux601dux3063ux305fux3089ux88abux544aux672cux4ebaux8a34ux8a1fux306aux306eux306dux3068ux3044ux3046ux66f4ux65b0ux304cux6fc0ux6e1bux3057ux305fux6df1ux6fa4ux8aedux53f2ux5f01ux8b77ux58ebux306eux30c4ux30a4ux30fcux30c8}{%
\paragraph{「慰謝料20万円に対して弁護士費用60万円余りを認容で、合計80万円超か。画期的だと思ったら、被告本人訴訟なのね。」という更新が激減した深澤諭史弁護士のツイート}\label{ux6170ux8b1dux6599uxff12uxff10ux4e07ux5186ux306bux5bfeux3057ux3066ux5f01ux8b77ux58ebux8cbbux7528uxff16uxff10ux4e07ux5186ux4f59ux308aux3092ux8a8dux5bb9ux3067ux5408ux8a08uxff18uxff10ux4e07ux5186ux8d85ux304bux753bux671fux7684ux3060ux3068ux601dux3063ux305fux3089ux88abux544aux672cux4ebaux8a34ux8a1fux306aux306eux306dux3068ux3044ux3046ux66f4ux65b0ux304cux6fc0ux6e1bux3057ux305fux6df1ux6fa4ux8aedux53f2ux5f01ux8b77ux58ebux306eux30c4ux30a4ux30fcux30c8}}

\begin{itemize}
\tightlist
\item
  〉〉〉 Linux Emacs: 2021/04/24 14:14:25 〉〉〉
\end{itemize}

:CATEGORIES: @kanazawabengosi \#金沢弁護士会 @JFBAsns
日本弁護士連合会(日弁連) \#法務省 @MOJ\_HOUMU \#深澤諭史弁護士
\#弁護士費用

\begin{itemize}
\item
  奉納\危険生物・弁護士脳汚染除去装置\金沢地方検察庁御中\_2020:
  \深澤諭史 @fukazawas\(・∀・)近時の裁判例、ネット投稿について、慰謝料20万円に対して弁護士費用60万円余りを認容で、合計80万円超か。画期的だと思ったら、
  \url{https://t.co/BAczD27thY} 
\item
  (4/100) TW fukazawas(深澤諭史) 日時: 2021-04-16 11:30 URL:
  \url{https://twitter.com/fukazawas/status/1382883997642989571\textgreater} {}
  (・∀・)近時の裁判例、ネット投稿について、慰謝料20万円に対して弁護士費用60万円余りを認容で、合計80万円超か。画期的だと思ったら、被告本人訴訟なのね。\textgreater{}
  (^ω^)やはり、この種の事件で、弁護士費用以上の賠償金を得るには、発信・・・
  \url{https://t.co/Dh75n7udRy} 
\item
  TW fukazawas(深澤諭史) 日時: 2021/04/16 11:30:12 URL:
  \url{https://twitter.com/fukazawas/status/1382883997642989571} 
  \textgreater{}
  (・∀・)近時の裁判例、ネット投稿について、慰謝料20万円に対して弁護士費用60万円余りを認容で、合計80万円超か。画期的だと思ったら、被告本人訴訟なのね。\\
  \textgreater{}
  (^ω^)やはり、この種の事件で、弁護士費用以上の賠償金を得るには、発信者が弁護士付けずに対応するかどうかがポイントですね。
\end{itemize}

 2つ目のツイートの掲載は,TwitterAPIでツイートの情報を取得し,現時点でツイートが削除されていないことを確認したことになります。いずれ近いうちに,Twitterのアカウント自体が消えてしまうかもしれないと見ている深澤諭史弁護士のツイートでもあります。

 ツイートの数ではなく質の問題だったと思いますが,控えめか穏当なツイートやリツイートばかりになったと感じた2日目か3日目に,その深澤諭史弁護士のタイムラインで見かけたのが「弁護士費用60万円余りを認容で、合計80万円超か。画期的だと思ったら、被告本人訴訟なのね。」になります。

 その後,一日だけ従来の深澤諭史弁護士らしさを感じるツイート群を見かけたことがあったと,現在の私の印象には残っています。次のツイートの記録はたぶんその日のものです。

\begin{itemize}
\tightlist
\item
  2021年04月19日19時12分の登録:
  \深澤諭史 @fukazawas\(^ω^)弁護士会からFATF第4次対日相互審査対応ワーキンググループの委員選任通知がきたお・・・・。
  \url{https://kk2020-09.blogspot.com/2021/04/fukazawas\_19.html} 
\item
  2021年04月19日20時06分の登録:
  \深澤諭史 @fukazawas\(・∀・)稀にこういう人いますが、弁護士の仕事の邪魔をして、敵に塩を送るだけなんですけれどもねぇ。。。。
  \url{https://kk2020-09.blogspot.com/2021/04/fukazawas\_96.html} 
\end{itemize}

 弁護士としての万能感に酔い痴れた従来の深澤諭史弁護士らしいツイートと,同日のようですが,「弁護士会からFATF第4次対日相互審査対応ワーキンググループの委員選任通知がきたお」というツイートが記録されていました。そういえばタイムラインで近くに並んでみたようなツイートです。

\begin{itemize}
\tightlist
\item
  FATF第4次対日相互審査 - Google 検索 \url{https://t.co/AMTRZZV38m} 
\end{itemize}

\begin{quote}
《引用の始まり》
\end{quote}

\begin{quote}
国際機関FATFによる第4次対日相互審査が2019年秋に迫った。2008年に実施された前回審査では日本の金融機関におけるマネー・ローンダリング及びテロ資金供与対策の甘さが指摘され、国際的な信頼が大きく揺らぐという苦い経緯がある。では、いま国内の金融機関に求められる対応とは―。ICTと金融法務に精通した弁護士である増島
雅和氏に話を聞いた。
\end{quote}

\begin{quote}
《引用の終わり》
\end{quote}

\begin{itemize}
\tightlist
\item
  FATF勧告とは?第4次対日相互審査前後に取るべき対応やリスクベース・アプローチについて解説
  \textbar{} nec wisdom \textbar{}
  ビジネス・テクノロジーの最先端情報メディア
  \url{https://wisdom.nec.com/ja/business/2019083001/index.html} 
\end{itemize}

 多少気になったものの前のときは調べなかったのですが,「FATF勧告」というのを調べてみると,マネーロンダリングの話が出てきました。

 2,3日前から深澤諭史弁護士のタイムラインは,これまでになく更新頻度が低くなっています。

\begin{itemize}
\item
  RT
  fukazawas(深澤諭史)|chihirotsukada(弁護士•米国公認会計士 塚田智宏)
  日時:2021-04-22 12:18/2021-04-22 09:05 URL:
  \url{https://twitter.com/fukazawas/status/1385070523927842818} 
  \url{https://twitter.com/chihirotsukada/status/1385021791148544000} 
  \textgreater{}
  ニューヨーク州司法試験に無事合格。何事も、周りの人の支えあってこそで一人では成し遂げられないこと、小さな一歩の積み重ねこそ肝要であることに改めて気付くことができました。何歳になっても初心を忘れず、耳を傾けられる素直さを、これからはより大切にしていきたい。
\item
  RT fukazawas(深澤諭史)|panda09panda(編むぱんだ) 日時:2021-04-22
  12:23/2021-04-22 12:23 URL:
  \url{https://twitter.com/fukazawas/status/1385071799591784453} 
  \url{https://twitter.com/panda09panda/status/1385071734508769281} 
  \textgreater{} @fukazawas
  深澤先生いつも著書にお世話になっております。仰る通り、Q\&Aのことなのですが、「確かこの話読んだな〜」とは思うのですが、いつもどこに書いてたか探すのに手間取ってしまいまして、、、、。
\item
  TW fukazawas(深澤諭史) 日時: 2021-04-22 12:23 URL:
  \url{https://twitter.com/fukazawas/status/1385071836040294403} 
  \textgreater{} @panda09panda 貴重なご意見ありがとうございます😊
\item
  RT fukazawas(深澤諭史)|Shingo\_Nakao((入院中)中尾慎吾)
  日時:2021-04-22 18:10/2021-04-22 17:21 URL:
  \url{https://twitter.com/fukazawas/status/1385159122748772354} 
  \url{https://twitter.com/Shingo\_Nakao/status/1385146816606261249} 
  \textgreater{} 病室内にて通話打ち合わせ終わり\\
  \textgreater{}\\
  \textgreater{} 現場に出向かない仕事は普通に再開しています\\
  \textgreater{}\\
  \textgreater{} 今後ともよろしくどうぞ
\item
  RT fukazawas(深澤諭史)|rippy08(りっぴぃ) 日時:2021-04-22
  19:26/2021-04-22 18:34 URL:
  \url{https://twitter.com/fukazawas/status/1385178240818376709} 
  \url{https://twitter.com/rippy08/status/1385165221694640128} 
  \textgreater{} 床にタイルカーペットが貼られました✨
  \url{https://t.co/mTasTKID69} 
\item
  RT fukazawas(深澤諭史)|ryouheitakaki(弁護士 高木良平を名乗る人物)
  日時:2021-04-22 19:26/2021-04-22 17:59 URL:
  \url{https://twitter.com/fukazawas/status/1385178283390562304} 
  \url{https://twitter.com/ryouheitakaki/status/1385156306646945797} 
  \textgreater{} 恫喝に利用されたり、濫発されないといいんですけどね〜
  \url{https://t.co/hDBDgazBj8} 
\item
  RT fukazawas(深澤諭史)|nhk\_seikatsu(NHK生活・防災)
  日時:2021-04-22 19:26/2021-04-21 21:23 URL:
  \url{https://twitter.com/fukazawas/status/1385178300193009664} 
  \url{https://twitter.com/nhk\_seikatsu/status/1384845200753930244} 
  \textgreater{} SNSで\\
  \textgreater{} ひぼう中傷を受けた被害者が\\
  \textgreater{} 投稿者の情報開示を求める\\
  \textgreater{} 裁判手続きが簡素化されます\\
  \textgreater{}\\
  \textgreater{} これまでは\\
  \textgreater{} 2回必要だった手続きが\\
  \textgreater{} 1回で済むようになります\\
  \textgreater{}\\
  \textgreater{} 詳しい内容は↓↓↓\\
  \textgreater{} \url{https://t.co/wZY91K4fCF}  \url{https://t.co/VEiVot6sYZ} 
\item
  RT fukazawas(深澤諭史)|sakaisusumu\_vb(酒井将) 日時:2021-04-23
  10:10/2021-04-23 08:27 URL:
  \url{https://twitter.com/fukazawas/status/1385400777439473665} 
  \url{https://twitter.com/sakaisusumu\_vb/status/1385374766270357505} 
  \textgreater{}
  最近、ベリーベスト法律事務所では弁護士中途採用の面接を多数受け付けているが、採用になる弁護士は少ないそうだ。良い人がいたら是非とも採用したいのだが、そう簡単ではない。
\item
  RT fukazawas(深澤諭史)|shimadayusuke66(島田雄左) 日時:2021-04-23
  21:27/2021-04-23 18:57 URL:
  \url{https://twitter.com/fukazawas/status/1385571048653918209} 
  \url{https://twitter.com/shimadayusuke66/status/1385533372798173191} 
  \textgreater{}
  弁護士や司法書士の資格取得後、以前は、丁稚奉公したら独立するのスタンダードでした。でも、今は独立に限らず、専門性を高めたり事務所の総合化などに伴い、色んなキャリアプランが増えたように感じます。そう考えると、資格者の役割も多様化してるので、資格取得後のキャリアプランは無限大です。
\item
  TW fukazawas(深澤諭史) 日時: 2021-04-24 14:16 URL:
  \url{https://twitter.com/fukazawas/status/1385824908366684163} 
  \textgreater{}
  (・∀・)Twitterで著名な先生(期は近い)の書面を拝見しているが、非常に勉強になる。さすがである。\\
  \textgreater{} (^ω^)同時に反省を迫られるお・・。
\end{itemize}

 上記10件のツイートは,「tun fukazawas
10」というコマンドで取得した深澤諭史弁護士の最新10件のツイートになります。いつの間にか1つ更新されていたらしく,それも深澤諭史弁護士本人のツイートで,14時16分とあります。

 次が今月4月に入ってからの深澤諭史弁護士の日毎にまとめたツイートの記録になります。

\begin{itemize}
\tightlist
\item
  2021年04月01日05時28分の登録:
  ツイートの記録資料:\法務検察・石川県警察宛\/深澤諭史(@fukazawas)/''2021年03月31日'':50件
  \url{https://kk2020-09.blogspot.com/2021/04/fukazawas2021033150.html} 
\item
  2021年04月02日04時15分の登録:
  ツイートの記録資料:\法務検察・石川県警察宛\/深澤諭史(@fukazawas)/''2021年04月01日'':23件
  \url{https://kk2020-09.blogspot.com/2021/04/fukazawas2021040123.html} 
\item
  2021年04月03日00時21分の登録:
  ツイートの記録資料:\法務検察・石川県警察宛\/深澤諭史(@fukazawas)/''2021年04月02日'':35件
  \url{https://kk2020-09.blogspot.com/2021/04/fukazawas2021040235.html} 
\item
  2021年04月04日06時40分の登録:
  ツイートの記録資料:\法務検察・石川県警察宛\/深澤諭史(@fukazawas)/''2021年04月03日'':67件
  \url{https://kk2020-09.blogspot.com/2021/04/fukazawas2021040367.html} 
\item
  2021年04月05日02時14分の登録:
  ツイートの記録資料:\法務検察・石川県警察宛\/深澤諭史(@fukazawas)/''2021年04月04日'':70件
  \url{https://kk2020-09.blogspot.com/2021/04/fukazawas2021040470.html} 
\item
  2021年04月06日03時58分の登録:
  ツイートの記録資料:\法務検察・石川県警察宛\/深澤諭史(@fukazawas)/''2021年04月05日'':35件
  \url{https://kk2020-09.blogspot.com/2021/04/fukazawas2021040535.html} 
\item
  2021年04月07日06時20分の登録:
  ツイートの記録資料:\法務検察・石川県警察宛\/深澤諭史(@fukazawas)/''2021年04月06日'':38件
  \url{https://kk2020-09.blogspot.com/2021/04/fukazawas2021040638.html} 
\item
  2021年04月08日04時20分の登録:
  ツイートの記録資料:\法務検察・石川県警察宛\/深澤諭史(@fukazawas)/''2021年04月07日'':41件
  \url{https://kk2020-09.blogspot.com/2021/04/fukazawas2021040741.html} 
\item
  2021年04月09日02時53分の登録:
  ツイートの記録資料:\法務検察・石川県警察宛\/深澤諭史(@fukazawas)/''2021年04月08日'':32件
  \url{https://kk2020-09.blogspot.com/2021/04/fukazawas2021040832.html} 
\item
  2021年04月10日00時55分の登録:
  ツイートの記録資料:\法務検察・石川県警察宛\/深澤諭史(@fukazawas)/''2021年04月09日'':41件
  \url{https://kk2020-09.blogspot.com/2021/04/fukazawas2021040941.html} 
\item
  2021年04月11日00時33分の登録:
  ツイートの記録資料:\法務検察・石川県警察宛\/深澤諭史(@fukazawas)/''2021年04月10日'':20件
  \url{https://kk2020-09.blogspot.com/2021/04/fukazawas2021041020.html} 
\item
  2021年04月12日00時20分の登録:
  ツイートの記録資料:\法務検察・石川県警察宛\/深澤諭史(@fukazawas)/''2021年04月11日'':20件
  \url{https://kk2020-09.blogspot.com/2021/04/fukazawas2021041120.html} 
\item
  2021年04月13日00時43分の登録:
  ツイートの記録資料:\法務検察・石川県警察宛\/深澤諭史(@fukazawas)/''2021年04月12日'':16件
  \url{https://kk2020-09.blogspot.com/2021/04/fukazawas2021041216.html} 
\item
  2021年04月14日00時07分の登録:
  ツイートの記録資料:\法務検察・石川県警察宛\/深澤諭史(@fukazawas)/''2021年04月13日'':23件
  \url{https://kk2020-09.blogspot.com/2021/04/fukazawas2021041323.html} 
\item
  2021年04月15日05時49分の登録:
  ツイートの記録資料:\法務検察・石川県警察宛\/深澤諭史(@fukazawas)/''2021年04月14日'':29件
  \url{https://kk2020-09.blogspot.com/2021/04/fukazawas2021041429.html} 
\item
  2021年04月16日00時05分の登録:
  ツイートの記録資料:\法務検察・石川県警察宛\/深澤諭史(@fukazawas)/''2021年04月15日'':22件
  \url{https://kk2020-09.blogspot.com/2021/04/fukazawas2021041522.html} 
\item
  2021年04月17日01時59分の登録:
  ツイートの記録資料:\法務検察・石川県警察宛\/深澤諭史(@fukazawas)/''2021年04月16日'':13件
  \url{https://kk2020-09.blogspot.com/2021/04/fukazawas2021041613.html} 
\item
  2021年04月18日00時05分の登録:
  ツイートの記録資料:\法務検察・石川県警察宛\/深澤諭史(@fukazawas)/''2021年04月17日'':12件
  \url{https://kk2020-09.blogspot.com/2021/04/fukazawas2021041712.html} 
\item
  2021年04月19日04時40分の登録:
  ツイートの記録資料:\法務検察・石川県警察宛\/深澤諭史(@fukazawas)/''2021年04月18日'':9件
  \url{https://kk2020-09.blogspot.com/2021/04/fukazawas202104189.html} 
\item
  2021年04月20日01時52分の登録:
  ツイートの記録資料:\法務検察・石川県警察宛\/深澤諭史(@fukazawas)/''2021年04月19日'':17件
  \url{https://kk2020-09.blogspot.com/2021/04/fukazawas2021041917.html} 
\item
  2021年04月21日00時11分の登録:
  ツイートの記録資料:\法務検察・石川県警察宛\/深澤諭史(@fukazawas)/''2021年04月20日'':18件
  \url{https://kk2020-09.blogspot.com/2021/04/fukazawas2021042018.html} 
\item
  2021年04月23日00時17分の登録:
  ツイートの記録資料:\法務検察・石川県警察宛\/深澤諭史(@fukazawas)/''2021年04月22日'':14件
  \url{https://kk2020-09.blogspot.com/2021/04/fukazawas2021042214.html} 
\item
  2021年04月24日00時11分の登録:
  ツイートの記録資料:\法務検察・石川県警察宛\/深澤諭史(@fukazawas)/''2021年04月23日'':2件
  \url{https://kk2020-09.blogspot.com/2021/04/fukazawas202104232.html} 
\end{itemize}

 4月17日から13件,12件,9件,17件,18件,14件,2件と減少傾向です。他に忙しくしているとも考えられなくはないですが,noteの記事の紹介も最近は見かけなくなっています。

\begin{itemize}
\tightlist
\item
  TW fukazawas(深澤諭史) 日時: 2021/04/16 11:30:12 URL:
  \url{https://twitter.com/fukazawas/status/1382883997642989571} 
  \textgreater{}
  (・∀・)近時の裁判例、ネット投稿について、慰謝料20万円に対して弁護士費用60万円余りを認容で、合計80万円超か。画期的だと思ったら、被告本人訴訟なのね。\\
  \textgreater{}
  (^ω^)やはり、この種の事件で、弁護士費用以上の賠償金を得るには、発信者が弁護士付けずに対応するかどうかがポイントですね。
\end{itemize}

 再掲になりますが4月16日11時30分の深澤諭史弁護士のツイートです。これとよく似た内容のツイートは他にもいくつかあって,より金額の大きいものもあったように思いますが,深澤諭史弁護士らしさを踏襲,再確認となったツイートになります。

 他にも同じ頃に,郷原信郎弁護士のネット記事とジャーナリストの江川紹子氏のツイートにぜひとも取り上げておきたいものがあったのですが,深澤諭史弁護士のこのツイートをまず一番に着手したことになります。

\begin{itemize}
\tightlist
\item
  〈〈〈 2021/04/24 14:58:14 Linux Emacs: 〈〈〈
\end{itemize}

\hypertarget{ux4e00ux756aux3059ux3054ux3044ux3082ux306eux3067ux6170ux8b1dux6599uxff12uxff10ux4e07ux5186ux306bux5bfeux3057ux3066ux5f01ux8b77ux58ebux8cbbux7528uxff19uxff15ux4e07ux5186ux3092ux8a8dux5bb9ux3057ux305fux3082ux306eux304cux3042ux308bux3068ux3044ux3046ux88c1ux5224ux5b98ux3068ux5f01ux8b77ux58ebux306eux7570ux5e38ux6027ux3092ux5370ux8c61ux3065ux3051ux308bux6df1ux6fa4ux8aedux53f2ux5f01ux8b77ux58ebux306eux904eux53bbux306eux30c4ux30a4ux30fcux30c8}{%
\paragraph{「一番すごいもので慰謝料20万円に対して、弁護士費用95万円を認容したものがある。」という裁判官と弁護士の異常性を印象づける深澤諭史弁護士の過去のツイート}\label{ux4e00ux756aux3059ux3054ux3044ux3082ux306eux3067ux6170ux8b1dux6599uxff12uxff10ux4e07ux5186ux306bux5bfeux3057ux3066ux5f01ux8b77ux58ebux8cbbux7528uxff19uxff15ux4e07ux5186ux3092ux8a8dux5bb9ux3057ux305fux3082ux306eux304cux3042ux308bux3068ux3044ux3046ux88c1ux5224ux5b98ux3068ux5f01ux8b77ux58ebux306eux7570ux5e38ux6027ux3092ux5370ux8c61ux3065ux3051ux308bux6df1ux6fa4ux8aedux53f2ux5f01ux8b77ux58ebux306eux904eux53bbux306eux30c4ux30a4ux30fcux30c8}}

\begin{itemize}
\tightlist
\item
  〉〉〉 Linux Emacs: 2021/04/24 15:02:48 〉〉〉
\end{itemize}

:CATEGORIES: @kanazawabengosi \#金沢弁護士会 @JFBAsns
日本弁護士連合会(日弁連) \#法務省 @MOJ\_HOUMU \#深澤諭史弁護士
\#弁護士費用

\begin{itemize}
\tightlist
\item
  1330:2021-04-24\_14:58:57 \#告発状 \#\#\#\#
  「慰謝料20万円に対して弁護士費用60万円余りを認容で、合計80万円超か。画期的だと思ったら、被告本人訴訟なのね。」という更新が激減した深澤諭史弁護士のツイート
  \url{https://hirono-hideki.hatenadiary.jp/entry/2021/04/24/145854} 
\end{itemize}

 内容は上記エントリーの続きになります。令和3年3月31日付告発状にも取り上げたように思いますが,改めてまとめを作成し,言及をしておきたいことがあります。

 私の記憶では,Twitterではなく,深澤諭史弁護士のnoteの記事にそれを見たのですが,過去のツイートを調べたか,たぶん作成したまとめ記事で,過去のツイートにも深澤諭史弁護士が同じような内容のツイートをしていたことを発見しました。まずは「弁護士費用」を含む深澤諭史弁護士の記録です。

\begin{lstlisting}
py37_env ❯ d|grep fukazawas|grep 弁護士費用
\end{lstlisting}

\begin{itemize}
\tightlist
\item
  2017年10月07日20時24分の登録:
  \深澤諭史 @fukazawas RT: @\_hznf\_\司法改革の一環として弁護士費用敗訴者負担が導入されそうになり、それに対して全国のの主に消費者系弁護士が猛烈な反対運動
  \url{http://hirono2014sk.blogspot.com/2017/10/fukazawasrthznf.html} 
\item
  2017年10月09日09時03分の登録:
  \深澤諭史 @fukazawas\賠償責任保険は、本当にオススメ。というか、必須。
  月額500円程度で入れるので、家族全員で入りましょう。
  賠償金はもちろんのこと、弁護士費用特約が
  \url{http://hirono2014sk.blogspot.com/2017/10/fukazawas-500.html} 
\item
  2018年04月09日11時09分の登録:
  \深澤諭史 @fukazawas\弁護士に依頼すれば,10割は難しいが,8割のリスクは回避・削減できるのに,それで得られる利益の10分の1の弁護士費用も渋るって,人間,合
  \url{http://hirono2014sk.blogspot.com/2018/04/fukazawas\_9.html} 
\item
  2019年03月06日13時29分の登録:
  \深澤諭史 @fukazawas\たとえば、よくネットでいう「発信者情報開示請求でかかった弁護士費用は投稿者に請求できる」というのも、例の東京高裁裁判例以来、完全に間違い
  \url{http://hirono2014sk.blogspot.com/2019/03/fukazawas\_39.html} 
\item
  2019年03月18日23時16分の登録:
  \深澤諭史 @fukazawas\クラウドファンディングで弁護士費用を集める場合、弁護士職務基本規程と弁護士広告規程との関係で、かなり慎重な配慮が必要なんだが・・。¥\n相手
  \url{http://hirono2014sk.blogspot.com/2019/03/fukazawas\_87.html} 
\item
  2019年08月08日16時44分の登録:
  \深澤諭史 @fukazawas\(・∀・)私もコメントしています。¥\n(^ω^)弁護士費用の話もしちゃいましたお。
  \url{http://hirono2014sk.blogspot.com/2019/08/fukazawas\_85.html} 
\item
  2020年02月06日20時58分の登録:
  \深澤諭史 @fukazawas\ネット上の名誉毀損投稿については、特定(発信者情報開示)のための弁護士費用実費が認められるかが、被害回復のメルクマールの一つです。¥\nとい
  \url{http://hirono2014sk.blogspot.com/2020/02/fukazawas\_6.html} 
\item
  2020年02月06日21時22分の登録:
  \深澤諭史 @fukazawas\インターネット上の名誉毀損投稿について、慰謝料20万円に加えて、弁護士費用約90万円、合計約110万円の賠償が認められた事案が最近ありま
  \url{http://hirono2014sk.blogspot.com/2020/02/fukazawas\_2.html} 
\item
  2020年04月05日22時17分の登録:
  \深澤諭史 @fukazawas\最近だと慰謝料300万円請求中30万円認容、弁護士費用実費約220万円請求中30万円認容というのがありますね。¥\n発信者が弁護士つけて応訴
  \url{http://hirono2014sk.blogspot.com/2020/04/fukazawas\_34.html} 
\item
  2020年04月20日12時34分の登録:
  \深澤諭史 @fukazawas\弁護士費用について誤解を招くような表現をして広告するのは,弁護士業務広告規程違反の問題ががが・・・。¥\n(・∀・;;)
  \url{http://hirono2014sk.blogspot.com/2020/04/fukazawas\_67.html} 
\item
  2020年07月07日16時19分の登録:
  \深澤諭史 @fukazawas\余り扇情的なことをいうつもりはないけど,ネットの誹謗中傷投稿について,150万円の慰謝料と,弁護士費用等400万円が認められた判決も最近
  \url{http://hirono2014sk.blogspot.com/2020/07/fukazawas\_71.html} 
\item
  2020年07月07日16時35分の登録:
  REGEXP:''弁護士費用約90万円、合計約110万円の賠償''/深澤諭史(@fukazawas)の検索(2020-02-06〜2020-02-06/2020年07月07日16時35分の記録1件)
  \url{http://hirono2014sk.blogspot.com/2020/07/regexpfukazawas2020-02-062020-02.html} 
\item
  2020年07月07日16時35分の登録:
  REGEXP:''弁護士費用約90万円''/深澤諭史(@fukazawas)の検索(2020-02-06〜2020-02-06/2020年07月07日16時35分の記録1件)
  \url{http://hirono2014sk.blogspot.com/2020/07/regexpfukazawas2020-02-062020-02\_7.html} 
\item
  2020年07月20日22時01分の登録:
  \深澤諭史 @fukazawas\「敗訴時の弁護士費用、カバーします」 新サービス開始:朝日新聞デジタル
  https:// \url{http://hirono2014sk.blogspot.com/2020/07/fukazawas-https.html} 
\item
  2020年10月03日20時43分の登録:
  \深澤諭史 @fukazawas\Q10.発信者情報開示請求を検討していますが,弁護士費用の方が高くついてしまわないでしょうか。
  \url{http://kk2020-09.blogspot.com/2020/10/fukazawas\_3.html} 
\item
  2020年11月14日15時30分の登録:
  REGEXP:''弁護士費用''/深澤諭史(@fukazawas)の検索(2013-02-12〜2020-10-17/2020年11月14日15時30分の記録171件)
  \url{http://kk2020-09.blogspot.com/2020/11/regexpfukazawas2013-02-122020-10.html} 
\item
  2020年11月26日10時34分の登録:
  \深澤諭史 @fukazawas\最近、無理筋案件(請求が認められないか、認められるとして、弁護士費用で足が出るか、労力も考えると、トントンになりそうな案件も含む。)で請
  \url{http://kk2020-09.blogspot.com/2020/11/fukazawas\_40.html} 
\item
  2020年12月31日18時29分の登録:
  \深澤諭史 @fukazawas\(・∀・)弁護士会の会費が高額なのは、弁護士費用が捻出できない方への、無料・廉価の法律サービス提供にも、使われているからです。(^ω^)
  \url{http://kk2020-09.blogspot.com/2020/12/fukazawas\_0.html} 
\item
  2021年02月05日19時54分の登録:
  \深澤諭史 @fukazawas\発信者情報開示請求の弁護士費用は賠償請求できるか?|深澤諭史
  @fukazawas \#note \url{https://note.com/fuk} 
  \url{http://kk2020-09.blogspot.com/2021/02/fukazawas-fukazawas-note-httpsnotecomfuk.html} 
\item
  2021年02月18日11時54分の登録:
  \深澤諭史 @fukazawas\「私が弁護士費用、いくらつかったと思っているんですか?」(・∀・)いや、あなた、私の依頼者じゃなくて、相手方でしょ?それはあなたの弁護士
  \url{https://kk2020-09.blogspot.com/2021/02/fukazawas\_92.html} 
\item
  2021年02月25日14時20分の登録:
  \KBブラック02 @battamonblack02\返信先:
  @fukazawasさん基本はそうですね!「示談金の方が先生の弁護士費用よりも安いですよね!?」というケー
  \url{https://kk2020-09.blogspot.com/2021/02/battamonblack02-fukazawas.html} 
\item
  2021年03月02日06時55分の登録:
  REGEXP:''弁護士費用''/深澤諭史(@fukazawas)の検索(2013-02-12〜2021-03-01/2021年03月02日06時55分の記録182件)
  \url{https://kk2020-09.blogspot.com/2021/03/regexpfukazawas2013-02-122021-03.html} 
\item
  2021年03月02日07時13分の登録:
  %@fukazawas 深澤諭史%一番すごいもので慰謝料20万円に対して、弁護士費用95万円を認容したものがある。
  \url{https://kk2020-09.blogspot.com/2021/03/fukazawas\_2.html} 
\item
  2021年03月14日18時28分の登録:
  \深澤諭史 @fukazawas\発信者情報開示請求の弁護士費用は賠償請求できるか?|深澤諭史
  @fukazawas \#note \url{https://note.com/fuk} 
  \url{https://kk2020-09.blogspot.com/2021/03/fukazawas-fukazawas-note-httpsnotecomfuk.html} 
\item
  2021年03月22日19時27分の登録:
  「(深澤諭史\textbar{}@fukazawas).*弁護士費用」を@hirono\_hideki @kk\_hirono @s\_hironoで検索 103件の該当 2021-03-22\_19:27の記録
  \url{https://kk2020-09.blogspot.com/2021/03/fukazawashironohidekikkhironoshirono103.html} 
\item
  2021年04月16日14時56分の登録:
  \深澤諭史 @fukazawas\(・∀・)近時の裁判例、ネット投稿について、慰謝料20万円に対して弁護士費用60万円余りを認容で、合計80万円超か。画期的だと思ったら、
  \url{https://kk2020-09.blogspot.com/2021/04/fukazawas\_16.html} 
\end{itemize}

 2,3記憶にないものが記録されていたのですが,とりわけ次が驚きの内容です。

\begin{itemize}
\item
  2020年07月07日16時19分の登録:
  \深澤諭史 @fukazawas\余り扇情的なことをいうつもりはないけど,ネットの誹謗中傷投稿について,150万円の慰謝料と,弁護士費用等400万円が認められた判決も最近
  \url{http://hirono2014sk.blogspot.com/2020/07/fukazawas\_71.html} 
\item
  TW fukazawas(深澤諭史) 日時: 2020/07/07 14:15:53 URL:
  \url{https://twitter.com/fukazawas/status/1280369926843203584} 
  \textgreater{}
  余り扇情的なことをいうつもりはないけど,ネットの誹謗中傷投稿について,150万円の慰謝料と,弁護士費用等400万円が認められた判決も最近出ていますね(欠席判決ではない)。\\
  \textgreater{}
  被害者・投稿者・プロバイダを弁護してる経験上いえることですが,自己判断でネット情報と心中前に悩まず弁護士へ。
\end{itemize}

 にわかに信じがたいところですが,ネットの誹謗中傷投稿で150万円の慰謝料と弁護士費用等400万円が認容された判決と深澤諭史弁護士は紹介しています。等の部分に深澤諭史弁護士らしいトリックを感じますが,開示請求費用を含めているのかもしれません。

 いったい,どのようなネットの誹謗中傷投稿で,450万円の損害賠償を負うことにされるのか,想像もつかないですが,それほど悪質で責任重大であれば,名誉毀損やあるいは業務妨害で刑事事件になっていそうな気もします。慰謝料のみなので業務妨害はないのかもしれません。

 数日前に,両親に対する慰謝料の具体的金額を見たことを思い出したのですが,子供が死亡したり,生涯介護が必要となる重大な後遺障害を負ったとしても,両親に対する慰謝料というのはずいぶんと安いものだという判例を前にも見ています。せいぜい200万円程度だったと思います。

\begin{itemize}
\tightlist
\item
  奉納\危険生物・弁護士脳汚染除去装置\金沢地方検察庁御中\_2020:
  「慰謝料」を@hirono\_hideki @kk\_hirono @s\_hironoで検索 441件の該当 2021-04-24\_15:34の記録
  \url{https://t.co/pLAFTPEOn1} 
\end{itemize}

2010-06-30 19:15:36
``探していた岡田進弁護士に関する情報は、現在プライベートモード中のHatena::Diaryのブログでみつかりました。タイトルは「\protect\hyperlink{ux5f01ux8b77ux58eb}{弁護士}生涯介護にあたることになった両親に慰謝料各々275万円の民事裁判
」。2009-04-18の登録エントリでした。''
\url{https://twitter.com/hirono\_hideki/status/17404335220} 

\begin{itemize}
\tightlist
\item
  TW hirono\_hideki(奉納\さらば弁護士鉄道・泥棒神社の物語) 日時:
  2010/06/30 19:15:36 URL:
  \url{https://twitter.com/hirono\_hideki/status/17404335220} 
  \textgreater{}
  探していた岡田進弁護士に関する情報は、現在プライベートモード中のHatena::Diaryのブログでみつかりました。タイトルは「\protect\hyperlink{ux5f01ux8b77ux58eb}{弁護士}生涯介護にあたることになった両親に慰謝料各々275万円の民事裁判
  」。2009-04-18の登録エントリでした。
\end{itemize}

 1番目が上記の被告発人岡田進弁護士に関するツイートでした。275万円という具体的な金額は記憶になかったですが,たしか小学生の姉と弟の二人乗りでの自転車の事故で,自転車側にずいぶんな過失割合が認められ,かなり疑問のある判決でした。

 「twilog-serch
"岡田進.*自転車"」の結果がさきほどのツイートのみでした。納得のいかない結果ですが,このような不本意な結果がきっかけで,調べ直し,新たな発見につながることもあったように思います。

〉〉〉 kk\_hironoのリツイート 〉〉〉

\begin{itemize}
\tightlist
\item
  RT
  kk\_hirono(刑事告発・非常上告_金沢地方検察庁御中)|hirono\_hideki(奉納\さらば弁護士鉄道・泥棒神社の物語)
  日時:2021-04-24 15:50/2010/06/30 19:21 URL:
  \url{https://twitter.com/kk\_hirono/status/1385848629416521731} 
  \url{https://twitter.com/hirono\_hideki/status/17404573402} 
  \textgreater{} 子供自転車二人乗り \url{http://goo.gl/F17D} 
  どんなに検索しても見つからなかったのですが、ページ自体は存在しリンク切れになっていませんでした。裁判について非常に考えさせられる内容です。「被告訴訟代理人弁護士 岡田 進」になっているので、個人的にはなおさらです。
\end{itemize}

 こういうところで,いずれはと懸念していた短縮URLのリンク切れを経験しました。

\begin{itemize}
\tightlist
\item
  生涯介護にあたることになった両親に慰謝料各々275万円の民事裁判 -
  告発\金沢地方検察庁\最高検察庁\法務省\石川県警察御中
  \url{https://t.co/usc4dAGT8H} 
\end{itemize}

 慰謝料各々275万円の意味が理解できたのですが,各々500万円から45%の過失相殺で引かれた金額でした。平成15年の事故で,「金沢地裁 平成18年10月11日判決(碓定)」とのことです。

\begin{itemize}
\tightlist
\item
  〈〈〈 2021/04/24 16:10:52 Linux Emacs: 〈〈〈
\end{itemize}

\hypertarget{ux5e744ux67089ux65e53ux5a6635ux5206ux306bux5c4aux3044ux3066ux3044ux305famazonux30d7ux30e9ux30a4ux30e0ux3078ux3088ux3046ux3053ux305dux3068ux3044ux3046ux30e1ux30fcux30eb}{%
\paragraph{2021年4月9日3婦35分に届いていた「Amazonプライムへようこそ」というメール}\label{ux5e744ux67089ux65e53ux5a6635ux5206ux306bux5c4aux3044ux3066ux3044ux305famazonux30d7ux30e9ux30a4ux30e0ux3078ux3088ux3046ux3053ux305dux3068ux3044ux3046ux30e1ux30fcux30eb}}

\begin{itemize}
\tightlist
\item
  〉〉〉 Linux Emacs: 2021/04/28 09:15:20 〉〉〉
\end{itemize}

:CATEGORIES: @kanazawabengosi \#金沢弁護士会 @JFBAsns
日本弁護士連合会(日弁連) \#法務省 @MOJ\_HOUMU

 さきほどメールを確認したところ,年4月9日3婦35分に「Amazonプライムへようこそ」というメールが届いていました。Amazonフォトの利用が利用が当初の目的だったのですが,Amazonプライムビデオもちょくちょく利用するようになり,これが生活スタイルの変化にもなっています。

\begin{itemize}
\tightlist
\item
  〈〈〈 2021/04/28 09:21:29 Linux Emacs: 〈〈〈
\end{itemize}

\hypertarget{ux5e744ux670814ux65e5ux304bux3089ux8996ux8074ux958bux59cbux3068ux306aux3063ux3066ux3044ux308bamazonux30d7ux30e9ux30a4ux30e0ux30d3ux30c7ux30aaux3053ux308cux307eux3067ux306eux8996ux8074ux5c65ux6b74}{%
\paragraph{2021年4月14日から視聴開始となっているAmazonプライムビデオ,これまでの視聴履歴}\label{ux5e744ux670814ux65e5ux304bux3089ux8996ux8074ux958bux59cbux3068ux306aux3063ux3066ux3044ux308bamazonux30d7ux30e9ux30a4ux30e0ux30d3ux30c7ux30aaux3053ux308cux307eux3067ux306eux8996ux8074ux5c65ux6b74}}

\begin{itemize}
\tightlist
\item
  〉〉〉 Linux Emacs: 2021/04/28 09:26:23 〉〉〉
\end{itemize}

:CATEGORIES: @kanazawabengosi \#金沢弁護士会 @JFBAsns
日本弁護士連合会(日弁連) \#法務省 @MOJ\_HOUMU

 視聴履歴をみると次のようになっています。中断しながら3,4日掛けて視聴し終えたものもあったと思うのですが,履歴には日付が一つあるだけで,視聴開始の時刻なのかと思いますが,はっきりせず,連日の視聴のようになっています。

\begin{itemize}
\item
  2021/04/14 復習するは我にあり
\item
  2021/04/15 東京流れ者
\item
  2021/04/16 それでもボクはやっていない
\item
  2021/04/21 「拳銃無頼帖 不敵に笑う男」,「疑惑」
\item
  2021/04/23 「日本沈没」,「小さいおうち」
\item
  2021/04/26 無敵が俺を呼んでいる
\item
  2021/04/27 銀河鉄道999
\item
  2021/04/28 銀河鉄道999
\end{itemize}

 同じ4月23日に,「日本沈没」,「小さいおうち」の2作を視聴したことになっていますが,どちらも朝の同じ時間帯に視聴をしたような記憶となっています。

\begin{lstlisting}
py37_env ❯ twilog-serch 小さいおうち
\end{lstlisting}

\begin{itemize}
\tightlist
\item
  ./hirono\_hideki2021-04-28\_073532.csv:2021-04-23 14:53:40 ``-
  小さいおうち 羽咋 ロケ地 \url{https://t.co/JHghlkCODC''} 
  \url{https://twitter.com/hirono\_hideki/status/1385471917126082563} 
\item
  ./hirono\_hideki2021-04-28\_073532.csv:2021-04-23 12:53:09
  ``小さいおうち - Wikiwand \url{https://t.co/9KesYP0tlN} 
  2014年、監督・山田洋次、主演・松たか子により映画化された。出演した黒木華は第64回ベルリン国際映画祭最優秀女優賞(銀熊賞)を受賞。''
  \url{https://twitter.com/hirono\_hideki/status/1385441586301964291} 
\item
  ./hirono\_hideki2021-04-28\_073532.csv:2021-04-23 12:52:25
  ``小さいおうち - Wikiwand \url{https://t.co/9KesYP0tlN} 
  中島京子による日本の小説。『別册文藝春秋』(文藝春秋)にて2008年11月号(第278号)から2010年1月号(第285号)まで連載された。第143回直木三十五賞受賞作。''
  \url{https://twitter.com/hirono\_hideki/status/1385441402943721472} 
\item
  ./hirono\_hideki2021-04-28\_073532.csv:2021-04-23 12:26:13
  ``\url{https://t.co/Z63EwYEvqJ:}  小さいおうちを観る \textbar{} Prime Video
  \url{https://t.co/Cp5AUQ1DoT}  ¥\n小さいおうち''
  \url{https://twitter.com/hirono\_hideki/status/1385434810554880000} 
\item
  ./hirono\_hideki2021-04-28\_073532.csv:2020-11-30 09:48:29
  ``松たか子&黒木華『小さいおうち』予告編 - YouTube
  \url{https://www.youtube.com/watch?v=jHWyJq3MN-o''} 
  \url{https://twitter.com/hirono\_hideki/status/1333211262558212097} 
\item
  ./hirono\_hideki2021-04-28\_073532.csv:2020-11-30 09:45:44
  ``小さいおうち - Wikiwand
  \url{https://www.wikiwand.com/ja/\%E5\%B0\%8F\%E3\%81\%95\%E3\%81\%84\%E3\%81\%8A\%E3\%81\%86\%E3\%81\%A1\#/cite\_ref-1} 
  イタクラの記念館には赤い屋根の家が描かれた作品があった。板倉は独身を貫いたようだった。3年前に平井家の息子と連絡をとったことがあると学芸員に聞かされた健史は、恋人と共に彼の住む石川県に向かう。''
  \url{https://twitter.com/hirono\_hideki/status/1333210570808381440} 
\item
  ./hirono\_hideki2021-04-28\_073532.csv:2020-11-30 09:36:40
  ``小さいおうち - Wikiwand
  \url{https://www.wikiwand.com/ja/\%E5\%B0\%8F\%E3\%81\%95\%E3\%81\%84\%E3\%81\%8A\%E3\%81\%86\%E3\%81\%A1} 
  昭和期のタキを演じる黒木華は、「クラシックな顔立ち」が決め手となり起用された{[}15{]}。撮影は、2013年3月1日から同年5月末まで行われた{[}14{]}。''
  \url{https://twitter.com/hirono\_hideki/status/1333208290075967494} 
\item
  ./hirono\_hideki2021-04-28\_073532.csv:2020-11-30 09:28:23
  ``小さいおうち - Wikiwand
  \url{https://www.wikiwand.com/ja/\%E5\%B0\%8F\%E3\%81\%95\%E3\%81\%84\%E3\%81\%8A\%E3\%81\%86\%E3\%81\%A1} 
  その後、タキは女中を辞して帰郷する。平井夫妻は、空襲で死亡し、恭一の行方は分からなかった。回想録はタキの死去により、絶筆となる。''
  \url{https://twitter.com/hirono\_hideki/status/1333206202499534849} 
\item
  ./hirono\_hideki2021-04-28\_073532.csv:2020-11-30 09:25:09
  ``小さいおうち - Wikiwand
  \url{https://www.wikiwand.com/ja/\%E5\%B0\%8F\%E3\%81\%95\%E3\%81\%84\%E3\%81\%8A\%E3\%81\%86\%E3\%81\%A1} 
  2014年、監督・山田洋次、主演・松たか子により映画化された。出演した黒木華は第64回ベルリン国際映画祭最優秀女優賞(銀熊賞)を受賞。''
  \url{https://twitter.com/hirono\_hideki/status/1333205389869948929} 
\item
  ./hirono\_hideki2021-04-28\_073532.csv:2015-02-26 10:24:08
  ``「小さいおうち」がテレビに出るらしい。3月1日日曜日。割と最近話題になっていた映画。録画していたドラマのCM。''
  \url{https://twitter.com/hirono\_hideki/status/570756165015117824} 
\item
  ./kk\_hirono2021-04-27\_203222.csv:2019-02-04 16:23:50
  ``黒木華という女優は、上記の引用部分になりますが、『小さいおうち』という作品が話題となっていたとき、情報番組で初めて知ったという記憶です。番宣以外で映画のシーンは見ていませんが、昭和30年代から昭和40年代という時代の雰囲気がよく出ているように感じたのが印象的でした。''
  \url{https://twitter.com/kk\_hirono/status/1092322823245164544} 
\end{itemize}

 「2021-04-23 14:53:40 "-
小さいおうち 羽咋 ロケ地」というツイートは,視聴後間もない時間の投稿になります。北陸の石川とセリフがありながら石川県内の景色ではないと思って調べたのですが,柴垣の海岸で堤防の先端に赤い灯台があると初めて知りました。

 次に「日本沈没」です。

\begin{itemize}
\item
  ./hirono\_hideki2021-04-28\_073532.csv:2021-04-23 20:33:14
  ``\url{https://t.co/Z63EwYEvqJ:}  Prime Video \url{https://t.co/rKXKLyviL3} 
  ¥\n日本沈没''
  \url{https://twitter.com/hirono\_hideki/status/1385557369925476354} 
\item
  ./hirono\_hideki2021-04-28\_073532.csv:2021-04-23 21:03:48 ``日本沈没
  - Wikiwand \url{https://t.co/96iBjdxjeN} 
  小松自身は、題名を「『日本滅亡』――果てしなき流れの果てに・・・、出発の日」とつけていたが、担当編集者であった浜井武の「『日本沈没』のほうが``滅亡''よりユーモラスだ」という主張により『日本沈没』となったという{[}4{]}。"
  \url{https://twitter.com/hirono\_hideki/status/1385565062362898432} 
\end{itemize}

 ツイートの数を2つに絞りましたが,4月23日の18時43分に火災の場面を撮影した写真がありました。少し思い出したのですが,買い物に出掛ける予定を変更して18時過ぎから視聴を開始したように思います。朝に見たというのは勘違いで,同じ日の朝に「小さいおうち」を視聴していました。

 (※)最初の視聴が「復習するは我にあり」だったというのも意外だったのですが,タイトルだけは印象と記憶にあった作品で,これが同じ作家の作品ということで島根県の冤罪事件とつながりがあります。これも個別に取り上げておきたい事柄で,弁護士の手弁当という問題性があります。

 最初の頃は時間を使って視聴をするのがもったいなくって,中断が多かったのですが,得るものも大きいと感じるようになり,視聴する作品の数も増えました。事実関係の表現や構成の参考にしていきたいと考えています。

\begin{itemize}
\tightlist
\item
  〈〈〈 2021/04/28 10:04:01 Linux Emacs: 〈〈〈
\end{itemize}

\hypertarget{ux5e744ux670827ux65e5ux304bux30895ux67081ux65e5ux306bux304bux3051ux8996ux8074ux3057ux305famazonux30d7ux30e9ux30a4ux30e0ux30d3ux30c7ux30aaux306eux9280ux6cb3ux9244ux9053uxff19uxff19uxff19ux30b7ux30fcux30baux30f31ux30a8ux30d4ux30bdux30fcux30c9113}{%
\paragraph{2021年4月27日から5月1日にかけ視聴したAmazonプライムビデオの「銀河鉄道999」シーズン1,エピソード(113)}\label{ux5e744ux670827ux65e5ux304bux30895ux67081ux65e5ux306bux304bux3051ux8996ux8074ux3057ux305famazonux30d7ux30e9ux30a4ux30e0ux30d3ux30c7ux30aaux306eux9280ux6cb3ux9244ux9053uxff19uxff19uxff19ux30b7ux30fcux30baux30f31ux30a8ux30d4ux30bdux30fcux30c9113}}

\begin{itemize}
\tightlist
\item
  〉〉〉 Linux Emacs: 2021/05/03 11:02:58 〉〉〉
\end{itemize}

:CATEGORIES: @kanazawabengosi \#金沢弁護士会 @JFBAsns
日本弁護士連合会(日弁連) \#法務省 @MOJ\_HOUMU \#深澤諭史弁護士

〉〉〉 kk\_hironoのリツイート 〉〉〉

\begin{itemize}
\item
  RT
  kk\_hirono(刑事告発・非常上告_金沢地方検察庁御中)|otakulawyer(山口貴士
  aka無駄に感じが悪いヤマベン) 日時:2021-05-03 11:04/2021/05/03 03:29
  URL: \url{https://twitter.com/kk\_hirono/status/1389038105382952962} 
  \url{https://twitter.com/otakulawyer/status/1388923612233539585} 
  \textgreater{} 楽天のトップが意味がわからないんじゃ監視を強化しないと
  \url{https://t.co/FAhT27tFiw} 
\item
  \url{https://t.co/hEEXK0Lxkh:}  銀河鉄道999を観る \textbar{} Prime Video
  \url{https://t.co/HmRKOzHCfq} 
\end{itemize}

 ページをスクロールダウンすると,第25話「鋼鉄天使」までリンク付きの一覧があり,さらに「113件のエピソードをすべて表示」というリンクもあるのですが,第113話の最終話を視聴してしばらくしてから気が付きました。

 それまでは「次のエピソード」というリンクを辿って第33話まで視聴し,そのあとは飛ばしながら第113話の最終話まで視聴したのですが,次のエピソードが3,4ほど飛ぶことがあり,最終話まで視聴しました。最終話の視聴は一昨日の5月1日だったように思います。

 はっきりしたことは憶えていないのですが,第1話から連続して第18話の「泥のメーテル」まで視聴していたのですが,その次が第22話の「海賊船クイーン・エメラルダス」になったように思います。

 その最終話のあとに視聴した「泥のメーテル」ですが,鉄郎が泥のメーテルに池の中に引きずり込まれる場面までは視聴した記憶があったのですが,そのあとが余り思い出せない場面で,列車内で泥のメーテルが砂になってしまう場面は,不思議なほど記憶にないものでした。

 そういう不可解さもあったので視聴した回を,パソコンの画面をスマホで撮影し記録を残しておくようにしました。第34話の写真もあるかもしれないですが,すぐに視聴をやめ,飛ばし始めました。かなりの数,飛ばしていますが,シーズン1がどこまで続くのか不明の状態でした。

 最終的にシーズン1は第113話まで続いたのですが,これはWikipediaのページで確認していたものと同じでした。テレビで放映された全作になるようですが,そのすべてが「シーズン1」となっているようです。同時になぜAmazonプライムビデオに「エピソード」となっているのかも気になっていました。

\begin{itemize}
\tightlist
\item
  銀河鉄道999 (アニメ) - Wikiwand
  \url{https://www.wikiwand.com/ja/\%E9\%8A\%80\%E6\%B2\%B3\%E9\%89\%84\%E9\%81\%93999\_(\%E3\%82\%A2\%E3\%83\%8B\%E3\%83\%A1)} 
\end{itemize}

 上記のページに,1978年9月14日が第1話「出発のバラード」で16話まで,16話が12月28日の放送で「蛍の街」となっていることに気が付きました。

 1979年が第17話から第59話まで,1980年が第60話から第101話まで,次が1981年・1982年となっていますが,第102話から第113話まで,最終回と思われるその第113話の放送が1981年3月26日で,1982年というのは4月5日の総集編「少年の旅立ちと別れ」だけのようです。

 1978年9月14日から1981年3月26日の放送ですが,これは昭和53年から昭和56年になります。令和3年3月31日付告発状のPDFファイルで確認したところ,能都中学校の入学が昭和52年の4月で石川県立水産高校小木分校への入学が昭和55年4月となっていました。

 昭和56年6月に小木分校を自主退学したことはよく憶えているので入学した年が昭和55年ということもはっきりしているのですが,中学生ではなく高校1年生のときに銀河鉄道999のテレビ放送があったというのは意外です。

 今まで意識したことはまるでなかったのですが,その昭和56年3月26日が銀河鉄道999の最終話の放送で,それから3ヶ月ほどあとの6月の終わりか7月の初めに,私は社会に旅立ち,金沢市問屋町でガソリンスタンドの店員の仕事を初めたことになります。

 銀河鉄道999は雰囲気が好きでお気に入りの番組で,印象にも強く残っていたのですが,不思議と視聴した回の記憶が少なく,はっきりしているのは初回の「旅立ちのバラード」だけかもしれません。他にもいくつか記憶に残る場面がAmazonプライムビデオにあったのですが,曖昧模糊としています。

 西部劇の宿場の場面が唯一,Amazonプライムビデオの視聴前から記憶にあったのですが,それらしい場面が3話ほど確認しています。最初が第2話で「火星の赤い風」でした。正直,もっと古い記憶が喚起されるのかと期待していたのですが,その点では期待はずれの残念な結果でした。

 第113話の最終話は「青春の幻影 さらば999
後編」ですが,前回の前編に続き最終駅の近くのブラックホールが場面にありました。これは全く記憶になかったのですが,ホームの向かいの列車に別の少年といるメーテルの姿は,ずっと前に,たぶんネット動画で観ていることを思い出しました。

 昭和56年3月26日というテレビアニメ,銀河鉄道999の最終話ですが,その日に石川県で放送があったのか,近いうちに図書館の北國新聞縮小版で確認をしておきたいと考えています。放送の時間帯に別の番組をみていた可能性を含めて。

 銀河鉄道999は少年キングの連載マンガということも最近になって意識したのですが,考えてみると少年キングを時々読んだのは,昭和50年4月の初めまで住んでいた辺田の浜の頃で,宇出津に来てからは見ることがなく,その間の連載だったようです。

 4,5日前,Twitterで銀河鉄道999を検索すると,漫画のメーテルや鉄郎が出てきたのですが,記憶にあるレタッチの描写で,アニメとはずいぶん違ったものとなっていました。記憶以上にメーテルがスタンレーの魔女に似ているとも思いました。

 宇出津の三番町の古本屋やスタンレーの魔女のことは,昨日辺りにみつけた次のエントリーに,こまかい記載がありました。そして,珠洲道路の食堂の古いテレビで観た,バイキングの放送で何が問題とされていたのかよく理解が出来ました。松本零士の漫画とアニメのメーテルの違いです。

\begin{itemize}
\tightlist
\item
  銀河鉄道999の作者、「漫画家の松本零士さん、イタリアで倒れ重体 脳卒中か」という芝原章吾弁護士のツイート:ジャーナリストの江川紹子氏のTwitterタイムライン(2019年11月分)
  - 告発\金沢地方検察庁\最高検察庁\法務省\石川県警察御中
  \url{https://hirono-hideki.hatenablog.com/entry/2019/11/17/003433} 
\end{itemize}

 エントリーの記事に,芝原章吾弁護士のツイートが出てくるのですが,どうもツイートの掲載をし忘れていたらしく,それがきっかけで芝原章吾弁護士のツイートを調べたところ,それがまたいくつかの発見に繋がりました。

\begin{itemize}
\tightlist
\item
  〈〈〈 2021/05/03 12:13:41 Linux Emacs: 〈〈〈
\end{itemize}

\hypertarget{ux5e745ux670811ux65e5ux306eamazonux30d7ux30e9ux30a4ux30e0ux306eux5229ux7528ux505cux6b62ux3068ux3044ux308fux304fux306e4ux670815ux65e5ux4ee5ux6765ux3068ux306aux3063ux305fux5c0fux6728ux6e2fux6771ux4e00ux6587ux5b57ux5824ux9632ux3067ux306eux30a2ux30b8ux91e3ux308a}{%
\paragraph{2021年5月11日のAmazonプライムの利用停止と,いわくの4月15日以来となった小木港東一文字堤防でのアジ釣り}\label{ux5e745ux670811ux65e5ux306eamazonux30d7ux30e9ux30a4ux30e0ux306eux5229ux7528ux505cux6b62ux3068ux3044ux308fux304fux306e4ux670815ux65e5ux4ee5ux6765ux3068ux306aux3063ux305fux5c0fux6728ux6e2fux6771ux4e00ux6587ux5b57ux5824ux9632ux3067ux306eux30a2ux30b8ux91e3ux308a}}

\begin{itemize}
\tightlist
\item
  〉〉〉 Linux Emacs: 2021/05/12 09:02:29 〉〉〉
\end{itemize}

:CATEGORIES: @kanazawabengosi \#金沢弁護士会 @JFBAsns
日本弁護士連合会(日弁連) \#法務省 @MOJ\_HOUMU \#石川県警察

\begin{itemize}
\tightlist
\item
  2021年05月12日09時04分の登録:
  @kk\_hirono(刑事告発・非常上告_金沢地方検察庁御中)のツイート ''.*'' 3249/3249:2021-04-02\_1909〜2021-05-12\_0903 2021年05月12日09時03分の記録
  \url{https://kk2020-09.blogspot.com/2021/05/kkhirono324932492021-04-0219092021-05.html} 
\end{itemize}

 これからは掲載するツイートの数をなるべく減らそうと考えています。

 「ku3
hirono\_hideki」とコマンドを実行したのですが,再捜査要請書_警察庁・石川県警察御中(@kk\_hirono)のまとめ記事が出来ていました。昨日辺り修正したスクリプトに問題があり,クリップボードにあるツイートのURLでユーザを特定したようです。

 スクリプトを確認するとユーザを特定する条件分岐が次のようになっていました。明らかなミスです。

\begin{lstlisting}
tw_user = `xsel -b`
if tw_user =~ /https:\/\/.*/
    tw_user = tw_user.scan(/https:\/\/twitter\.com\/(\w+)(\/status\/\d+)?/)[0][0]
else
    tw_user = ARGV[0].dup
end
\end{lstlisting}

 条件式を「if tw\_user =\textasciitilde{} /\url{https://.} */ \&\&
ARGV{[}0{]}.nil?」と変更したところ今度はうまく行ったようです。テストのつもりでしたが,非常上告-最高検察庁御中\_ツイッター(@s\_hirono)のまとめ記事も作成しました。

\begin{itemize}
\tightlist
\item
  2021年05月12日09時12分の登録:
  @hirono\_hideki(奉納\さらば弁護士鉄道・泥棒神社の物語)のツイート ''.*'' 3237/3237:2021-04-24\_2027〜2021-05-12\_0904 2021年05月12日09時11分の記録
  \url{https://kk2020-09.blogspot.com/2021/05/hironohideki323732372021-04-2420272021.html} 
\item
  2021年05月12日09時17分の登録:
  @s\_hirono(非常上告-最高検察庁御中\_ツイッター)のツイート ''.*'' 3250/3250:2021-01-21\_1603〜2021-05-12\_0132 2021年05月12日09時17分の記録
  \url{https://kk2020-09.blogspot.com/2021/05/shirono-325032502021-01-2116032021-05.html} 
\end{itemize}

 コマンドはku3です。昨日もご紹介したと思いますが,3200件前後の最新ツイートを取得し,テキストデータとしてまとめ記事をBloggerのブログに投稿します。

 これまでは埋め込みツイートで見やすい表示を重視していたのですが,埋め込みツイートが表示されないことも多くなり,取得するツイートも100件と少ないので,これからはテキストで3200件前後のまとめ記事を軸に取り扱いをしたいと思います。

 3200件前後のツイートを目で追うには無理があり,ブラウザのページ内検索の活用が必須になるのではと思います。Google
Chromeだとエンターキーで次の該当位置に移動してくれます。

\begin{itemize}
\tightlist
\item
  (124/3237)
  @hirono\_hideki(奉納\さらば弁護士鉄道・泥棒神社の物語)のツイート ''.*'' 3237/3237:2021-04-24\_2027〜2021-05-12\_0904
  2021年05月12日09時11分の記録\\
  TW hirono\_hideki(奉納\さらば弁護士鉄道・泥棒神社の物語) 日時:
  2021-05-11 14:13 URL:
  \url{https://twitter.com/hirono\_hideki/status/1391984779646574594} 
  \textgreater{} 覚醒剤 密輸 判決 - Twitter検索 / Twitter
  \url{https://t.co/UFLkCz8YJ2} 
\end{itemize}

 だいたいの位置を探すのに「密輸」でページ内検索をしました。買い物から戻った後になりますが,宇出津図書館での思いがけない発見がきっかけで行った検索です。その前に「金沢地裁」で検索したのですが目的の情報は見つかりませんでした。

 昨日5月11日は,前日の10日の午後に電話があり,11日の16時に小木港の東一文字堤防にアジ釣りに行くことが決まっていたのですが,午前中にAmazonプライムビデオの利用停止があり,Amazonカードを買うために,北國銀行のATMに寄ってからコンビニに行ったのです。

 コンビニでは3千円のAmazonカードを買いました。そのあと宇出津図書館に行き,桶川ストーカー殺人事件の清水潔氏の本を取り寄せで注文しました。そして前に見つからなかった正木ひろし弁護士の本を探してもらい借りてきました。

 すぐに見つけ出してくれたのですが,偉人伝のシリーズ集のような場所にあり,すっかり盲点となっていました。

 Amazonカードの支払い方法変更でAmazonプライムビデオの利用停止が解除されたのですが,12日の今朝になって再び利用停止が表示されるようになり,形態からモバイルモードでリンクを開いても形態決済のページが表示されなくなっています。

 Amazonプライムビデオの料金は月500円だったと思いますが,確か30日間の無料期間があり,形態決済として申込みをしていました。その形態決済ができず利用停止となってようですが,形態決済のページが表示できない状態が続いています。

 Amazonphotosで写真の表示は出来ているのですが,Amazonプライムビデオは「プライム会員資格は停止されています お支払い方法を更新してくださいお支払い方法が確認されると、このメッセージは表示されなくなります。」という表示が出たままとなっています。

 「Amazonギフト券・Amazon種類別商品券またはクーポン」という支払い方法もあるのですが,コードは使われているとされ,Amazonギフト券のポイント加算に回されたようです。

 不可解なことは4月15日にもあったのですが,それも小木港東一文字堤防にアジ釣りに出掛ける直前のことで,そのときはバイクで郵便局に行っています。いずれタイミングをみて記述する予定です。

\begin{itemize}
\tightlist
\item
  TW kk\_hirono(刑事告発・非常上告_金沢地方検察庁御中) 日時:
  2021/05/12 08:56:51 URL:
  \url{https://twitter.com/kk\_hirono/status/1392267490210717699} 
  \textgreater{} 2021-04-15\_142304_.jpg\\
  \textgreater{} \url{https://t.co/JUNaGo4m3W} 
\end{itemize}

 さきほど試しにTwitterに投稿したのですが,Googleフォトで一枚の写真を共有からTwitter投稿したものです。これまでアルバムは共有のリンクを作成して公開していましたが,個別の写真をTwitterで共有したのは今回が初めてかもしれません。

 Twitterカードとも呼ばれるようですが,サムネイルのような写真がタイムラインのツイートに表示され,ひと目でだいたいどんな写真かわかるようになっています。手順が少し面倒にも感じますが,サムネイルが表示されないリンクの画像もあります。

 Googleフォトの場合,まもなく無制限のアップロードができなくなるということもありますが,アルバム数が気になっています。以前,アルバム数に上限があるという情報を見かけていたのですが,撤廃されたのか,その後は探しても情報が見つかりませんでした。

 このようなWebサービスの制約,上限というのは,ギリギリのところを狙ってくるアカウントに対処するためか,具体的な基準が示されていないことが多く,それでいきなり使えなくなったということもGoogleフォトと連携したBloggerの画像投稿で経験しています。

 自作のppコマンドの更新が出来たくなったのもそのためですが,画像や写真を探し出すというのもなかなかやっかいなことがあり,画像や写真に識別しやすい名前をつけていくというのも手間の掛かる作業でした。

\begin{itemize}
\item
  2021-04-15\_144335_.jpg ¥\n \url{https://t.co/RbcOGjrNYj}  ¥\n
   イカの駅つくモール 巨大イカのモニュメント
\item
  2021-04-15\_144414_.jpg ¥\n \url{https://t.co/QutUvrmBhU}  ¥\n
   イカの駅つくモール 九十九湾から見える立山連峰
\end{itemize}

 ネットで問題になっていた能登町の巨大イカのモニュメントですが,4月15日に初めて見に行きました。昨日も早めに出かけてどんな様子か見てこようと考えていたのですが,根魚釣りの道具を探すのに時間がかかったりして,出発が遅れました。まだ早いのですが,キジハタ狙いです。

\begin{itemize}
\tightlist
\item
  (123/3237)
  @hirono\_hideki(奉納\さらば弁護士鉄道・泥棒神社の物語)のツイート ''.*'' 3237/3237:2021-04-24\_2027〜2021-05-12\_0904
  2021年05月12日09時11分の記録\\
  RT
  hirono\_hideki(奉納\さらば弁護士鉄道・泥棒神社の物語)|ShunsukeTodo(藤堂俊介)
  日時:2021-05-11 14:13/2021-05-11 14:06 URL:
  \url{https://twitter.com/hirono\_hideki/status/1391984835074297863} 
  \url{https://twitter.com/ShunsukeTodo/status/1391983003660812294} 
  \textgreater{} 覚醒剤をキャリーケースに隠し密輸の女に実刑判決|NHK
  石川県のニュース \url{https://t.co/kJ5jImhnGB} 
\end{itemize}

 ツイートをしただけでリンクの記事は読んでいなかったのかもしれません。リンクを開いて,まずそこにあるニュース動画の視聴をしたのですが,被告人の精神障害の話が出てきました。図書館の新聞で見ただけで,昨日ネットで探したツイートには,触れたものがなかったように思います。

\begin{quote}
《引用の始まり》
\end{quote}

\begin{quote}
裁判では被告が荷物の中身を違法薬物と認識していたかが争点になり、検察は「キャリーケースを持ち帰るだけで、報酬をもらえると聞けば、常識的に考えて密輸を疑う」と主張し、弁護側は「被告が精神障害を抱え、指示役のうそを見抜く力はなかった」と主張しました。10日の裁判で大村陽一裁判長は、「被告人は『税関に中身を確認したいと言われたときに困る』という理由で、キャリーケースの鍵の暗証番号を依頼人に尋ねており違法薬物が隠匿されている疑念を抱きながら、それでも構わないと思い日本に持ち帰ったことが強く推認できる」と指摘しました。その上で「自ら考えて依頼人に質問できていたため、被告人の精神障害は社会的認知機能に特段の影響を及ぼしていない」として、懲役7年の実刑判決を言い渡しました。
\end{quote}

\begin{quote}
《引用の終わり》
\end{quote}

\begin{itemize}
\tightlist
\item
  覚醒剤をキャリーケースに隠し密輸の女に実刑判決|NHK
  石川県のニュース \url{https://www3.nhk.or.jp/lnews/kanazawa/20210511/3020008111.htmln} 
\end{itemize}

 最近は滅多にテレビをつけないし,新聞も図書館に行ったときぐらいしかみることがないのですが,この小松空港の覚醒剤密輸事件は,これまで3回ぐらいピンポイントのタイミングで報道を見ていました。

 判決が出るというニュースも見ていたと思うのですが,すっかり忘れていました。Amazonプライムの利用停止がなければ,昨日,図書館に行くことはなく,新聞記事も見ていなかったと思います。アジ釣りの道具の用意をするのに,ただでさえバタバタとしていました。

 上記のNHK石川県内ニュースは,比較的詳細なニュースだと思いますが,新聞の方が詳しく,弁護士の名前も出ていました。最初に北國銀行の記事を読んで,一度,図書館から出たタイミングで思い出し,引き返して北陸中日新聞の記事も読んでおきました。

 弁護士の意気込みが強く伝わる新聞記事で,被告人に控訴をすすめるともあったと思いますが,判決の言い渡しで被告人が倒れたという記載もありました。弁護士が被告人に有利となる判決を期待させ,その落差の衝撃が大きかったとも想像されるところです。

 新聞の刑事裁判で,実名の弁護士の名前をみたのも数年ぶりのことでした。前のことは憶えていないですが,4,5年かそれ以上は経っているように思います。これはかねて疑念があったのですが,新聞社の方で弁護士の宣伝を回避する意向があるのかと考えることもありました。

 余り見覚えなのない弁護士の名前でしたがまだ調べていません。上の名前はよくある名前だったと思います。大久保だったような気もするのですが,よく憶えていません。

 メモの代わりになるようなものを確認したのですが,山本啓二弁護士となっていました。これは見たことのない弁護士の名前になると思います。昨日は先を急いでいたこともあり,深く考えずにいました。

 Twitterで検索すると,「「山本啓二弁護士」の検索結果はありません」となっていました。

\begin{itemize}
\tightlist
\item
  山本 啓二|北都法律事務所|金沢弁護士会に所属|弁護士データベース
  \url{https://t.co/yioLzH93mL} 
\end{itemize}

 北都法律事務所というのも見聞きした憶えがないですが,北海道にありそうな名称です。身近なところでは金沢に北都運輸と北都高速がありました。北都運輸は正確には松任市になるかもしれません。他に宇出津の堤防のプレートに,北都組という施工業者の名前を見ています。

\begin{itemize}
\tightlist
\item
  (株)北都組|トンネル、海岸・湾岸、道路舗装、河川・砂防、下水道、造成、土木等建設工事業
  -石川県金沢市- \url{https://t.co/26hnZDEqrS} 
\end{itemize}

 金沢では余り見た憶えのない建設会社名だったのですが,金沢市内に本社がある建設会社だとわかりました。

\begin{itemize}
\tightlist
\item
  北都法律事務所 - Google 検索 \url{https://t.co/SQxPh5IWFr} 
\end{itemize}

 念のためGoogleのアカウントをログアウトさせて「北都法律事務所」を検索したのですが,ほとんどが金沢の情報で,わずかに函館市にも「北都法律事務所」が存在するような情報がありました。「弁護士法人ほくと総合法律事務所」といのもあって,支店がいくつかある規模のようです。

 北陸の都という意味で北都なのかもしれないですが,個人的に北都といえば北海道の方をイメージします。よく使われていそうな名称ですが,意外に少なく,Googleでは金沢がメインになっているようですが,所属弁護士が一人らしいという情報しか見つかっていません。

\begin{itemize}
\tightlist
\item
  覚醒剤をキャリーケースに隠し密輸の女に実刑判決|NHK 石川県のニュース
  \url{https://www3.nhk.or.jp/lnews/kanazawa/20210511/3020008111.html} 
\end{itemize}

 そういえばどちらの席も2人座っていたと思い出し,上記の記事にある動画の静止画を確認したのですが,ここで意外な発見がありました。どちらが検察なのかわからないと思いながら眺めていたところ,前に薄緑の大きな長椅子があることに気が付きました。

 法廷の静止画や映像というのは刑事裁判のニュースで見かけることが多いですが,この長椅子に気がついたのは初めてです。被告人に並んで連行した刑務官が座る席だと思います。私自身,座った経験がありますが,弁護士の前に移動したのは,被害者安藤文さんの兄が暴れた次の公判からです。

 金沢地方裁判所の建物は数年前に建て替えられていますが,写真で見ただけで実際に見たことも行ったこともありません。場所は前の建物とほぼ同じようですが,内部の構造は全く情報を見ていません。

 法廷の左右とも同じような壁となっていますが,前の金沢地方裁判所の法廷は,裁判官に向かって右手に通路と窓があったように思います。被告人が刑務官と一緒に出入りする通路があったことははっきりしていますが,外の光が入り込む窓が通路の間にあったと思うのです。

 ニュース動画を再生させ,ズームアップされた大村陽一裁判長を再び見ましたが,ずいぶん若く見える裁判長だと思いました。この裁判長の名前は記憶にあったのですが,ずいぶん前の金沢地方裁判所の裁判長の名前かと思い,それも退官が近いような高齢の裁判長のイメージがありました。

\begin{itemize}
\tightlist
\item
  金沢地方裁判所 - Wikipedia \url{https://t.co/ZNh5Tb4dlF} 
\end{itemize}

 過去に似たような名前の裁判長がいなかったのか確認したかったのですが,上記のWikipediaには,歴代所長があり,1992年から現職までざっと数えて13人の名前があります。見覚えのある名前が2,3あるのですが,同姓同名か似たような名前の別人かと思います。

 裁判所の所長の権限とか責任問題を聞いた覚えがないのですが,同じ石川県に住んでいても見聞きしたことのない名前が金沢地方裁判所の所長として並んでいるのは,どういう存在なのかと謎に思いました。検察庁だと検事正に相当する役職になるのかもしれません。

 そもそも裁判所の責任が問われたという話は聞かないのですが,その裁判所の代表責任者が所長になるのかとは思います。裁判官の独立は憲法に保証されていたようにも思うのですが,中央が人事権を握っていて,ヒラメ裁判官が多いという話もありました。

\begin{itemize}
\tightlist
\item
  「伝説の裁判官」が実名告発!なぜ裁判官は政府に逆らえないのか?(岩瀬
  達哉) \textbar{} 現代ビジネス \textbar{} 講談社(1/4)
  \url{https://t.co/luYd1b1yOB} 
\end{itemize}

 ヒラメ裁判官でGoogle検索をしました。

\begin{quote}
《引用の始まり》
\end{quote}

\begin{quote}
第15代最高裁長官を務めた町田顯は、2004年10月、裁判官として採用された新任判事補への辞令交付式の挨拶でこう述べた。

「上級審の動向や裁判長の顔色ばかりうかがう『ヒラメ裁判官』がいると言われている。私はそんな人はいないと思うが、少なくとも全く歓迎していない」。

皮肉なことに、これは町田長官に対する批判でしばしば使われる言葉だった。

町田が、東京高等裁判所の民事部裁判長だった時のことだ。知り合いの法曹関係者に、こう零していた。「僕は、高裁長官にはなれないのかねえ」。

当の法曹関係者は、その時の驚きを、いまも鮮明に覚えている。
\end{quote}

\begin{quote}
《引用の終わり》
\end{quote}

\begin{itemize}
\tightlist
\item
  「伝説の裁判官」が実名告発!なぜ裁判官は政府に逆らえないのか?(岩瀬
  達哉) \textbar{} 現代ビジネス \textbar{}
  講談社(1/4) \url{https://gendai.ismedia.jp/articles/-/51985n} 
\end{itemize}

 上記の引用部分に,「上級審の動向や裁判長の顔色ばかりうかがう『ヒラメ裁判官』がいると言われている。」とあります。裁判長の顔色をうかがう裁判官という話は聞いたことがなかったのですが,佐藤倫子弁護士のツイートがきっかけで三宅俊一郎裁判長のことを書いていたところです。

\begin{quote}
《引用の始まり》
\end{quote}

\begin{quote}
跳ね返りの若手裁判官を震え上がらせるとともに、政府の理解を得て「平賀書簡問題」を収めるには、スケープゴートが必要だった。

そのターゲットとされたのが、青法協の中心メンバーであった宮本康昭判事補(当時35歳)である。

裁判官は、10年ごとにその適格性を審査され、不適格と認定されると、裁判官の地位を失う。宮本は、理由を告げられることなく、再任拒否となり、裁判所を追われた。

宮本は、冒頭の町田元長官とは司法修習生の同期で、ともに青法協のアクティブメンバーだった。

町田は脱会し、順調にエリート街道を歩み、宮本は会員に留まったことで、弁護士への転身を余儀なくされた。それほどまでに、最高裁の青法協への人事政策は徹底していた。

当時、書簡を受け取った福島は、退官後の現在、郷里の富山市で弁護士をしている。多くの裁判官を翻弄した「平賀書簡」は、いったい、誰が、マスコミに流出させたのか。

86歳の高齢ながら矍鑠としていて、一徹な性格そのまま、書簡流出の真相をはじめて語った。

「いろいろ考えた挙げ句、かなり早い段階で、僕が、新聞社に渡した。これを放置したんじゃ、いずれまた、同じような裁判干渉が起こる。

だけど、裁判所の中だけで問題にしても、結局、もみ消されるだけですから。まずは、公表しないと戦えない。世論を喚起しようと、地元の北海道新聞に送ったわけです」
\end{quote}

\begin{quote}
《引用の終わり》
\end{quote}

\begin{itemize}
\tightlist
\item
  「伝説の裁判官」が実名告発!なぜ裁判官は政府に逆らえないのか?(岩瀬
  達哉) \textbar{} 現代ビジネス \textbar{}
  講談社(3/4) \url{https://gendai.ismedia.jp/articles/-/51985?page=3n} 
\end{itemize}

 上記に引用しましたが,「宮本康昭判事補(当時35歳)」という名前が出てきました。再任拒否となり郷里の富山市で弁護士になったとあります。ブルーバージは少し見かけた覚えがありますが,平賀書簡問題というのは少なくとも記憶にありませんでした。

 「その会議で、「平賀書簡」問題が取り上げられるのを知って、自分たちの手許にある書簡が本物との確信を得た。以後、激しい取材合戦が繰り広げられ、日本中を巻き込んだ一大騒動へと発展していくのである。」ともありますが,まだ時期を理解していません。

 4ページに分かれ記事を一通り読んでから前に戻って確認し,理解したのですが,「この裁判を担当したのが、札幌地裁民事1部の福島重雄裁判長(当時39歳)だった。」という裁判官がこの記事の主役だったようです。

 偶然なのか小松空港の覚醒剤密輸事件の判決のニュースでは,見出しに含まれる金沢地裁という裁判所名がなかったようなのですが,確か,この検索がきっかけで,意外な他にニュースに出会いました。

 わずか一日前のことですが,いちいち正確に記憶できているわけではありません。ku3のまとめ記事を見ながら流れを追い経過を調べます。

 その前にさきほど中央としていた全国の裁判官の人事を握る場所ですが,Googleの検索結果を見ていると,最高裁の事務総局や最高裁事務局とあります。検察庁だと最高検になりそうですが,検事が人事権を握られ,出世が,なんとかという話は聞いたことがないように思います。

\begin{itemize}
\item
  (114/3237)
  @hirono\_hideki(奉納\さらば弁護士鉄道・泥棒神社の物語)のツイート ''.*'' 3237/3237:2021-04-24\_2027〜2021-05-12\_0904
  2021年05月12日09時11分の記録\\
  RT
  hirono\_hideki(奉納\さらば弁護士鉄道・泥棒神社の物語)|fastdoor2(fastdoor)
  日時:2021-05-11 14:44/2021-03-15 16:29 URL:
  \url{https://twitter.com/hirono\_hideki/status/1391992698412822530} 
  \url{https://twitter.com/fastdoor2/status/1371362898161770499} 
  \textgreater{} 覚醒剤使用と危険運転で歩行者死亡させた罪
  女に懲役5年の判決 \textbar{} 事件 \textbar{} NHKニュース\\
  \textgreater{}\\
  \textgreater{} これさ、やっぱりおかしいと思う。\\
  \textgreater{}
  騙されて密輸に加担してしまった外国の人は、懲役6年だったよ。麻薬だと知らなかったとしても、疑わずに断らなかったからなんだって。
  \url{https://t.co/3pmQa6tmQD} 
\item
  (124/3237)
  @hirono\_hideki(奉納\さらば弁護士鉄道・泥棒神社の物語)のツイート ''.*'' 3237/3237:2021-04-24\_2027〜2021-05-12\_0904
  2021年05月12日09時11分の記録\\
  TW hirono\_hideki(奉納\さらば弁護士鉄道・泥棒神社の物語) 日時:
  2021-05-11 14:13 URL:
  \url{https://twitter.com/hirono\_hideki/status/1391984779646574594} 
  \textgreater{} 覚醒剤 密輸 判決 - Twitter検索 / Twitter
  \url{https://t.co/UFLkCz8YJ2} 
\end{itemize}

 「逃走」でページ内検索をして見つけました。「金沢地裁」ではなく,その次の「覚醒剤 密輸 判決」の検索結果に出てきたニュースでした。事故の直後の報道をテレビでみていました。

 現場で逃走し,刑事裁判でも反省していないような話なのですが,覚醒剤で酩酊中の死亡事故で逃走し,懲役5年というのは疑問に思いました。最近このように疑問に思う判決の量刑が多いのですが,求刑自体が懲役6年だったらしくそちらを基準にすれば,重めの判決を出したという見方もありそうです。

 Google
Chromeは全角文字で検索しても11年が該当するようですが,該当は5件で2011年が多いようです。その検索で確認したかったのは求刑の懲役11年のことです。

 手っ取り早く,新聞の写真の方で確認しましたが,やはり求刑が懲役11年と罰金400万円となっていました。その判決が懲役7年と罰金300万円となっています。罰金300万円の方は忘れていましたが,これも被告人は相当のダメージがあったと思われ,労役場だとどれくらい支払いに掛かるのか考えました。

 覚醒剤などの違法薬物の密輸事件では,薬物の量が大きく量刑に反映され無期懲役も少なくないと聞きます。新聞には覚醒剤約1.8キロをスーツケースに,とあります。量刑相場など素人にはわからないですが,求刑の懲役11年と判決の懲役7年の開きが気になります。

 弁護士が頑張ったので求刑が大きくなったのか,それとも弁護士の頑張りで懲役11年の求刑が懲役7年で済んだのかという受け取り方も人様々で,同じ境遇同じ立場になった時,弁護士を頼るというインセンティブは,弁護士の宣伝効果として大きなものがありそうで,検察にしてみれば逆効果。

 時刻は14時18分です。ようやくAmazonプライムビデオの形態決済ができました。

 昨夜は予定より30分程早く戻ったのですが,深澤弁護士でTwitter検索をしたところ,一般のツイートでAbemaPrimeに深澤諭史弁護士が出演したような情報があり,リンクがあったのか忘れましたが,動画ニュースを飛ばしながら視聴しました。

 その少し前に,ずっとTwitterの更新のない深澤諭史弁護士が新型コロナウイルスに感染し病状が悪化して隔離病棟に収容されスマホも使えない状況下にある可能性がいくらか高まったように思え,少し調べていました。

 出演者として深澤諭史弁護士の名前を確認しなかったのですが,主に前半に出演がありました。後半は姿を見かけなかったのですが,稲田朋美氏が出演していました。

 今調べて確認すると元防衛大臣だったのですが,昨夜は元法務大臣だったように思え,深澤諭史弁護士に続けた出演だったこともあり,疲れて寝ぼけているのかとも思うことがありました。むしろ疲れを感じないのが不思議な状態でもありました。

 昨日は未明の4時頃に目が覚め,そのまま起きていたのですが,寝た時間も0時30分は過ぎていました。最近は睡眠が十分でも魚釣りに行って帰ると疲れることが多く,昨夜は疲れが出ないのが不思議でした。1つ理由として考えられるのは数日前に視聴した地磯でのYouTube動画です。

 たしか行きに車から地磯まで2時間半,崖を登る帰りは3時間半という話で,17時間あたりがないまま重そうなジギングロッドを振り続けているという話でした。地磯ではめったに使う人がいないGTロッドという話でもあったと思います。

 魚釣りという趣味を極めているという見方もありましたが,同時に,法律の専門家として想像を絶する言動をTwitterでやってきたのが深澤諭史弁護士になります。令和3年3月31日付告発状でも取り上げましたが,盲腸と真実だけでもその思考の一端の凄まじさがお伝えできると思います。

\begin{itemize}
\tightlist
\item
  TW fukazawas(深澤諭史) 日時: 2016/05/23 11:32:14 URL:
  \url{https://twitter.com/fukazawas/status/734572617651347459} 
  \textgreater{}
  病院には,「俺,医療には詳しいんで,それで盲腸手術なんか簡単なんでしょ?だから,自分でやってみたんすけれど,敗血症になったので,治して下さい。」っていう患者は滅多に来ないだろうが,法律事務所には似たような状況の人がしょっちゅう来る。
\end{itemize}

\begin{lstlisting}
py37_env ❯ d|grep @fukazawas|grep 真実
\end{lstlisting}

\begin{itemize}
\tightlist
\item
  2017年10月12日20時33分の登録:
  \深澤諭史 @fukazawas\まあ、当たり前のことですが、相場って、真実ではなくて、気持ちで動くものなのですよね。
  なるほどとおもった1日でした
  \url{http://hirono2014sk.blogspot.com/2017/10/fukazawas-1.html} 
\item
  2017年11月05日10時23分の登録:
  \深澤諭史 @fukazawas\負け筋だったり、コミュニケーションに問題がある人ほど、真実に反して、弁護士が自分の事件を受任してくれない理由を、金銭に求めたがるという傾
  \url{http://hirono2014sk.blogspot.com/2017/11/fukazawas\_43.html} 
\item
  2018年08月08日03時46分の登録:
  \深澤諭史 @fukazawas\ですね(・∀・;)
  死刑について論じると,「ネットで真実に目覚めた」方々から,こういう反応がありますね・・。
  こういう人達が,何とかブロ
  \url{http://hirono2014sk.blogspot.com/2018/08/fukazawas\_8.html} 
\item
  2019年03月08日12時12分の登録:
  \深澤諭史 @fukazawas\「説明責任」だの「世間をおさがわせ」だの「真実を明らかに」だの、とにかく、メディアは、取材対象の個人に甘え過ぎ。¥\n「甘えるな。取材対象者
  \url{http://hirono2014sk.blogspot.com/2019/03/fukazawas\_42.html} 
\item
  2019年03月30日19時12分の登録:
  \深澤諭史 @fukazawas\出世には影響なく、取締役にもなれる!?「自己破産」の真実(幻冬舎ゴールドオンライン)
  - Yahoo!ニュース
  \url{http://hirono2014sk.blogspot.com/2019/03/fukazawas-yahoo.html} 
\item
  2019年04月21日12時29分の登録:
  \深澤諭史 @fukazawas\これ、同じこと思った。¥\n語られたことではなくて、語られなかったことで真実を推定するという法曹テクニック。¥\n(・∀・)
  \url{http://hirono2014sk.blogspot.com/2019/04/fukazawas\_29.html} 
\item
  2019年05月14日21時24分の登録:
  \深澤諭史 @fukazawas\もはや常識ですが、真実でも名誉毀損は成立し得ます。
  \#ネット表現の法務
  \url{http://hirono2014sk.blogspot.com/2019/05/fukazawas\_45.html} 
\item
  2019年05月23日19時00分の登録:
  \深澤諭史 @fukazawas\ブロックは自由でございますが、それにしても、なぜ、丁寧に御社が真実を説明なさった部分を削除してしまわれたのですか?¥\n私はログを持っている
  \url{http://hirono2014sk.blogspot.com/2019/05/fukazawas\_36.html} 
\item
  2019年05月25日19時17分の登録:
  \深澤諭史 @fukazawas RT: @fukazawas\ブロックは自由でございますが、それにしても、なぜ、丁寧に御社が真実を説明なさった部分を削除してしまわれたの
  \url{http://hirono2014sk.blogspot.com/2019/05/fukazawasrtfukazawas.html} 
\item
  2019年06月21日17時41分の登録:
  \深澤諭史 @fukazawas\もっとシンプルに「誰も1000万円の借金背負って年収300万円の不安定な職業につきたくない」でおわりかと。¥\nこの指摘が真実ならば、例えば
  \url{http://hirono2014sk.blogspot.com/2019/06/fukazawas1000300.html} 
\item
  2019年07月05日12時51分の登録:
  \深澤諭史 @fukazawas\(・∀・)真実であっても名誉毀損は成立します。¥\n(^ω^)統計をとったわけではありませんが、名誉権を侵害されている側が不名誉で非難に値す
  \url{http://hirono2014sk.blogspot.com/2019/07/fukazawas\_16.html} 
\item
  2019年07月29日21時33分の登録:
  \深澤諭史 @fukazawas\(;・∀・)デマ,誹謗中傷の類は,「面白い」ことが多いです。「退屈な真実は,面白い嘘に勝てない」というのが,問題なんですよね・・・。
  \url{http://hirono2014sk.blogspot.com/2019/07/fukazawas\_934.html} 
\item
  2019年07月30日03時30分の登録:
  \深澤諭史 @fukazawas\(;・∀・)デマ,誹謗中傷の類は,「面白い」ことが多いです。「退屈な真実は,面白い嘘に勝てない」というのが,問題なんですよね・・・。
  \url{http://hirono2014sk.blogspot.com/2019/07/fukazawas\_268.html} 
\item
  2019年08月24日23時39分の登録:
  \深澤諭史 @fukazawas\「過払い金に続く弁護士のビジネス」という話の真実
  : 弁護士 深澤諭史のブログ
  \url{http://hirono2014sk.blogspot.com/2019/08/fukazawas\_25.html} 
\item
  2019年10月08日11時15分の登録:
  \深澤諭史 @fukazawas\破産,廃業周りって,ネットde真実が本当に多いな。¥\n下手すりゃ命に関わるのでやめてほしい。インチキ医療並みの危険度。¥\n(・∀・;)
  \url{http://hirono2014sk.blogspot.com/2019/10/fukazawasde\_8.html} 
\item
  2019年10月08日11時29分の登録:
  \深澤諭史 @fukazawas\(;・∀・)この3倍,5倍酷い案件でも免責通りますよね・・・。¥\n(^ω^)「ネットde真実」の前に,弁護士に相談だお!
  \url{http://hirono2014sk.blogspot.com/2019/10/fukazawas-de.html} 
\item
  2019年11月23日10時56分の登録:
  \深澤諭史 @fukazawas\【前田恒彦さんのコメント】「あなたにできる最低限の償いは真実を述べることだと思いますが、真実を述べないこ\ldots{}
  ▼新潟女児殺害、被告に死
  \url{http://hirono2014sk.blogspot.com/2019/11/fukazawas\_40.html} 
\item
  2019年11月24日20時34分の登録:
  \深澤諭史 @fukazawas\②無罪っぽいんで刑を軽くします。¥\n③否認する被告人に真実話さないと償えないと問う裁判所←NEW!!
  \url{http://hirono2014sk.blogspot.com/2019/11/fukazawas\_24.html} 
\item
  2020年03月14日19時31分の登録:
  \深澤諭史 @fukazawas\法的請求を受ける側,被告側で本人訴訟すると,ネットde真実に目覚めて都合の良い法律情報デマと心中するのって,大病にかかった人が,カルトや
  \url{http://hirono2014sk.blogspot.com/2020/03/fukazawasde.html} 
\item
  2020年03月22日12時27分の登録:
  \深澤諭史 @fukazawas\コロナの件で怖いのは,「政府は真実を隠している!」とか根拠のない陰謀論が出てきて,それに「証拠ないじゃん」っていっても,「記録を抹消,ね
  \url{http://hirono2014sk.blogspot.com/2020/03/fukazawas\_50.html} 
\item
  2020年03月22日12時35分の登録:
  \深澤諭史 @fukazawas\あの人たち,デマが嫌いなわけでも,真実を尊重したいわけでも無くて,デマをやっつけることを大義名分に,他人を馬鹿にしたり,賢い振りをしたり
  \url{http://hirono2014sk.blogspot.com/2020/03/fukazawas\_44.html} 
\item
  2020年04月02日21時31分の登録:
  \深澤諭史 @fukazawas\「こういう事件では、弁護士なんかつけなくても訴訟は大丈夫!みんなでネットde真実に目覚めよう!」みたいな話はTwitterでもたまに聞く
  \url{http://hirono2014sk.blogspot.com/2020/04/fukazawastwitter.html} 
\item
  2020年04月03日13時14分の登録:
  \深澤諭史 @fukazawas\少額訴訟という制度は「ネットde真実に目覚めたが,この制度を使えば,本人訴訟で簡単に,正しい俺の言い分が通るはずだ」というものではありま
  \url{http://hirono2014sk.blogspot.com/2020/04/fukazawas\_90.html} 
\item
  2020年04月15日11時01分の登録:
  REGEXP:''(ネットde真実|ネットde真実)''/深澤諭史(@fukazawas)の検索(2017-07-20〜2020-04-13/2020年04月15日11時01分の記録27件)
  \url{http://hirono2014sk.blogspot.com/2020/04/regexpdefukazawas2017-07-202020-04.html} 
\item
  2020年05月09日18時52分の登録:
  \深澤諭史 @fukazawas\法律トラブルで、同じ立場に置かれた当事者同士で「情報交換」して、ネットde真実すなわち都合の良いだけのデマをかき集め、自分達に不都合な情
  \url{http://hirono2014sk.blogspot.com/2020/05/fukazawas\_9.html} 
\item
  2020年05月11日15時44分の登録:
  \深澤諭史 @fukazawas\「ネットde真実」よりも「専門家で現実」 :
  弁護士 深澤諭史のブログ
  \url{http://hirono2014sk.blogspot.com/2020/05/fukazawas\_11.html} 
\item
  2020年05月20日21時42分の登録:
  \深澤諭史 @fukazawas\最近沢山検索でアクセスされているので,紹介・・。
  本人訴訟あるある,本人訴訟のネットde真実系のデマとかも,なるべく網羅しています・・。
  \url{http://hirono2014sk.blogspot.com/2020/05/fukazawas\_31.html} 
\item
  2020年05月29日13時28分の登録:
  \深澤諭史 @fukazawas\お互いに都合の良い事実は真実で,不都合な事実は「工作」だって思い込む迷路(・∀・;)
  \url{http://hirono2014sk.blogspot.com/2020/05/fukazawas\_63.html} 
\item
  2020年06月06日21時10分の登録:
  \深澤諭史 @fukazawas\別に本人訴訟やっちゃダメとは言わないし、本人訴訟が適切な事例もあるけれども、「判例で決まるから大丈夫!ネットde真実の法律情報!弁護士は
  \url{http://hirono2014sk.blogspot.com/2020/06/fukazawas\_10.html} 
\item
  2020年06月18日21時52分の登録:
  \深澤諭史 @fukazawas\ネット上の表現トラブルを扱っていると、ネットde真実に目覚めて、違法投稿して責任追及されて、相手に弁護士がついているのに、それでもネット
  \url{http://hirono2014sk.blogspot.com/2020/06/fukazawas\_56.html} 
\item
  2020年06月19日16時41分の登録:
  \深澤諭史 @fukazawas\専門家に相談すると都合の悪い話もあるかもしれないので怖いってのがあるかもしれないですね。
  ネットなら,有利な情報はネットde真実!,不利
  \url{http://hirono2014sk.blogspot.com/2020/06/fukazawas\_19.html} 
\item
  2020年06月23日17時11分の登録:
  REGEXP:''ネット.*真実''/深澤諭史(@fukazawas)の検索(2014-10-29〜2020-06-20/2020年06月23日17時11分の記録73件)
  \url{http://hirono2014sk.blogspot.com/2020/06/regexpfukazawas2014-10-292020-06.html} 
\item
  2020年07月06日11時20分の登録:
  \深澤諭史 @fukazawas\真実は、法テラスが弁護士を食べさせているのではなくて、
  法テラスに弁護士が食べられているところですよね。 (・∀・;)
  \url{http://hirono2014sk.blogspot.com/2020/07/fukazawas\_6.html} 
\item
  2020年07月16日16時13分の登録:
  \深澤諭史 @fukazawas\逆にすぐに相談しないとまずい案件に限って、「ネットde真実」の法律情報を探すことに時間を費消して、最後に差し押さえ食らってから相談に来る
  \url{http://hirono2014sk.blogspot.com/2020/07/fukazawas\_88.html} 
\item
  2020年08月11日12時28分の登録:
  \深澤諭史 @fukazawas\「ネットde真実
  法律情報版」に騙されないためには,その情報提供の仕方,文言が,感情的でないかは,一つのメルクマールですね。
  やたら感情 \url{http://hirono2014sk.blogspot.com/2020/08/fukazawas\_11.html} 
\item
  2020年09月15日11時22分の登録:
  \深澤諭史 @fukazawas\それにしても,本当にネット上の表現トラブルで,ネットde真実の法律情報に目覚めてしまいましたって,パターンが多いですね。紛争とネットとの
  \url{http://hirono2014sk.blogspot.com/2020/09/fukazawas\_15.html} 
\item
  2020年09月16日19時41分の登録:
  \深澤諭史 @fukazawas\えー,いわば刑訴法は,まさに真実発見と人権保障の調和の観点で制定されておりまして,この論点の論証をしていく中におきまして,(ここでペット
  \url{http://hirono2014sk.blogspot.com/2020/09/fukazawas\_93.html} 
\item
  2020年09月16日20時55分の登録:
  REGEXP:''真実''/深澤諭史(@fukazawas)の検索(2014-04-14〜2020-09-16/2020年09月16日20時55分の記録242件)
  \url{http://hirono2014sk.blogspot.com/2020/09/regexpfukazawas2014-04-142020-09.html} 
\item
  2020年09月21日00時21分の登録:
  REGEXP:''真実''/深澤諭史(@fukazawas)の検索(2014-04-14〜2020-09-16/2020年09月21日00時20分の記録242件)
  \url{http://kk2020-09.blogspot.com/2020/09/regexpfukazawas2014-04-142020-09.html} 
\item
  2020年09月28日21時28分の登録:
  \深澤諭史 @fukazawas\「ネットde真実の法律情報」に目覚めてしまうのは,法的紛争が持つ,非常に大きなプレッシャー,不安感が背景にあると思う。弁護士としては,デ
  \url{http://kk2020-09.blogspot.com/2020/09/fukazawas\_12.html} 
\item
  2020年09月28日22時49分の登録:
  REGEXP:''ネットde真実''/深澤諭史(@fukazawas)の検索(2019-09-16〜2020-09-28/2020年09月28日22時49分の記録33件)
  \url{http://kk2020-09.blogspot.com/2020/09/regexpfukazawas2019-09-162020-09.html} 
\item
  2020年10月18日11時45分の登録:
  REGEXP:''ネット.*真実''/深澤諭史(@fukazawas)の検索(2014-10-29〜2020-10-11/2020年10月18日11時44分の記録87件)
  \url{http://kk2020-09.blogspot.com/2020/10/regexpfukazawas2014-10-292020-10.html} 
\item
  2020年11月02日13時24分の登録:
  REGEXP:''真実''/深澤諭史(@fukazawas)の検索(2014-04-14〜2020-10-27/2020年11月02日13時24分の記録247件)
  \url{http://kk2020-09.blogspot.com/2020/11/regexpfukazawas2014-04-142020-10.html} 
\item
  2020年11月05日10時06分の登録:
  \深澤諭史 @fukazawas\6年前の本ですが、「マスコミが伝えない真実」とか「ネットde真実」が発生するメカニズムについて解説しています。今も変わらないですね。(・
  \url{http://kk2020-09.blogspot.com/2020/11/fukazawas\_41.html} 
\item
  2020年11月15日12時15分の登録:
  \深澤諭史 @fukazawas\ネット上の匿名の親切な人「本人訴訟でOK!裁判官は証拠と判例に基づき公正に判断するから大丈夫!ネットde真実の法律情報!」
  \url{http://kk2020-09.blogspot.com/2020/11/fukazawas\_15.html} 
\item
  2020年11月30日08時00分の登録:
  @fukazawas(深澤諭史)のツイート ''真実'' 23/3225:2020-09-27\_1813〜2020-11-30\_0741 2020年11月30日08時00分の記録
  \url{http://kk2020-09.blogspot.com/2020/11/fukazawas2332252020-09-2718132020-11.html} 
\item
  2020年11月30日08時02分の登録:
  %@fukazawas 深澤諭史%デマは面白く,真実は退屈である。¥\nデマは都合良く,真実は不都合である。¥\nデマはわかりやすく,真実はわかりにくい。¥\nしたがって,真実
  \url{http://kk2020-09.blogspot.com/2020/11/fukazawasnnn.html} 
\item
  2020年12月18日12時32分の登録:
  \深澤諭史 @fukazawas\この時期だと、弁護士に相談しにくくて、発信者側が、「ネットde真実の法律情報」の罠にハマって、
  \url{http://kk2020-09.blogspot.com/2020/12/fukazawas\_56.html} 
\item
  2020年12月23日00時08分の登録:
  \深澤諭史 @fukazawas\自著でも解説しているが、ほんこれ。聞いてもいないのに、ペラペラしゃべることに真実があったりする。(・∀・)
  \url{http://kk2020-09.blogspot.com/2020/12/fukazawas\_28.html} 
\item
  2020年12月27日08時51分の登録:
  REGEXP:''真実''/深澤諭史(@fukazawas)の検索(2014-04-14〜2020-12-22/2020年12月27日08時51分の記録258件)
  \url{http://kk2020-09.blogspot.com/2020/12/regexpfukazawas2014-04-142020-12.html} 
\item
  2020年12月31日15時29分の登録:
  REGEXP:''真実''/深澤諭史(@fukazawas)の検索(2014-04-14〜2020-12-28/2020年12月31日15時29分の記録259件)
  \url{http://kk2020-09.blogspot.com/2020/12/regexpfukazawas2014-04-142020-12\_31.html} 
\item
  2021年01月11日12時14分の登録:
  \深澤諭史 @fukazawas\ネットde真実の法律情報・・・ネット上の法的紛争の情報(特に実体験系)が不正確なのは、感情的になっていて判断力が低下してネットde真実に目覚
  \url{http://kk2020-09.blogspot.com/2021/01/fukazawas\_11.html} 
\item
  2021年01月14日23時47分の登録:
  \深澤諭史 @fukazawas\なので、安易にネットに公にして、「ネットde真実(の法律情報)」に目覚めましょう!みたいになるので、本当に注意が必要です。(;・∀・)
  \url{http://kk2020-09.blogspot.com/2021/01/fukazawas\_48.html} 
\item
  2021年01月16日16時15分の登録:
  \深澤諭史 @fukazawas\「ネットde真実」な法律情報に騙されないように、ってことでも書いていたら、今年中にQ100まで行きそうだ・・・。
  \url{http://kk2020-09.blogspot.com/2021/01/fukazawas100.html} 
\item
  2021年01月30日10時04分の登録:
  \深澤諭史 @fukazawas\(・∀・)ほんこれ。(^ω^)生兵法で「『ネットde真実』の法律情報」をやる前に、自分の問題はすぐに弁護士へ!!
  \url{http://kk2020-09.blogspot.com/2021/01/fukazawas\_30.html} 
\item
  2021年01月30日10時08分の登録:
  \深澤諭史 @fukazawas\相談者「私は、「『ネットde真実』の法律情報」に目覚めた者です。それによれば、この方法を使えば、全て、私の思い通りになるはずなので、私の
  \url{http://kk2020-09.blogspot.com/2021/01/fukazawas\_92.html} 
\item
  2021年01月30日12時10分の登録:
  \深澤諭史 @fukazawas\(・∀・)ぜひ!私を呼んで欲しい!!!(*^ω^)お待ちしておりますお!!本音で現実と真実を語りますお!
  \url{http://kk2020-09.blogspot.com/2021/01/fukazawas\_13.html} 
\item
  2021年02月13日12時29分の登録:
  \深澤諭史 @fukazawas\ネットで持ち寄って、伝言ゲームで都合良く改変されて、「ネットde真実の法律情報」に目覚めた人が大量発生するという・・・。
  \url{https://kk2020-09.blogspot.com/2021/02/fukazawas\_13.html} 
\item
  2021年03月17日10時19分の登録:
  \深澤諭史 @fukazawas\法律デマ、つまり「ネットde真実の法律情報」の見分け方ですが、激しい対立のある分野、わかりやすく断言する、言い方が感情的、無理に法律用語
  \url{https://kk2020-09.blogspot.com/2021/03/fukazawas\_17.html} 
\item
  2021年03月21日19時24分の登録:
  \深澤諭史 @fukazawas\請求者も、それを知っていて、弁護士付けずに仲間内で情報交換で大丈夫と誤信し、ネットde真実の法律情報みたいな対応に期待してやるケースもあ
  \url{https://kk2020-09.blogspot.com/2021/03/fukazawas\_99.html} 
\item
  2021年03月24日00時42分の登録:
  REGEXP:''ネット.*真実''/深澤諭史(@fukazawas)の検索(2014-10-29〜2021-03-21/2021年03月24日00時42分の記録111件)
  \url{https://kk2020-09.blogspot.com/2021/03/regexpfukazawas2014-10-292021-03.html} 
\item
  2021年04月30日19時47分の登録:
  REGEXP:''真実''/深澤諭史(@fukazawas)の検索(2014-04-14〜2021-04-19/2021年04月30日19時47分の記録276件)
  \url{https://kk2020-09.blogspot.com/2021/04/regexpfukazawas2014-04-142021-04.html} 
\end{itemize}

 そういえば,昨夜視聴したAbemaPrimeでの深澤諭史弁護士の開設が,アスリートの女性のわいせつ加工動画のような事件だったのですが,ネットで少しだけ見かけていたように思ったのですが,図書館で新聞を開いた時,右側の紙面の左側にそれらしい見出しの記事がありました。

 Twitterのトレンドでは見かけていないように思うのですが,探せばすぐに見つかるニュースだと思います。弁護士がツイートをしているのも見ていないように思うのですが,ネットのニュース記事のページにあるランキングのようなところで見出しを見かけていたのかもしれません。

\begin{quote}
《引用の始まり》
\end{quote}

\begin{quote}
容疑者はおととし5月、テレビ番組で放送された女性アスリートの画像39枚をアダルトサイトに無断でアップロードした疑いがもたれています。小山容疑者はわいせつなコメントをつけて画像を掲載し、10年間で少なくとも1億2000万円の広告収入を得ていたとみられています。
\end{quote}

\begin{quote}
《引用の終わり》
\end{quote}

\begin{itemize}
\tightlist
\item
  女性アスリート画像をアダルトサイトに転載の男逮捕|TBS
  NEWS \url{https://news.tbs.co.jp/newseye/tbs\_newseye4265736.htmln} 
\end{itemize}

 11時13時45分が配信時刻のようです。深澤諭史弁護士が出演したAbemaPrimeではボードに1億2千万円の収益だけが強調された印象だったのですが,上記の記事には「10年間で少なくとも1億2000万円の広告収入」とあって,別の収入源が混じっていそうな印象を受けました。

 正確にどのような言葉遣いがあったのか,動画の視聴をやり直さなければ確認できないですが,こういうネット犯罪絡みで1億円以上の収益とは,同じ問題にたずさわる弁護士というのも,相当の収益がありそうに思えました。

 深澤諭史弁護士といえば,何度でも,次のツイートをご紹介しておきたいと思います。弁護士と裁判官という関係でも深く考えさせられる純度の高い貴重な資料となっています。

\begin{itemize}
\item
  TW fukazawas(深澤諭史) 日時: 2021/04/16 11:30:12 URL:
  \url{https://twitter.com/fukazawas/status/1382883997642989571} 
  \textgreater{}
  (・∀・)近時の裁判例、ネット投稿について、慰謝料20万円に対して弁護士費用60万円余りを認容で、合計80万円超か。画期的だと思ったら、被告本人訴訟なのね。\\
  \textgreater{}
  (^ω^)やはり、この種の事件で、弁護士費用以上の賠償金を得るには、発信者が弁護士付けずに対応するかどうかがポイントですね。
\item
  2021年04月16日14時56分の登録:
  \深澤諭史 @fukazawas\(・∀・)近時の裁判例、ネット投稿について、慰謝料20万円に対して弁護士費用60万円余りを認容で、合計80万円超か。画期的だと思ったら、
  \url{https://kk2020-09.blogspot.com/2021/04/fukazawas\_16.html} 
\end{itemize}

 深澤諭史弁護士の出演したAbemaPrimeはこれまでいくつか視聴していますが,被害の回復がとても難しいと印象づけるのも深澤諭史弁護士のあるあるで,弁護士に依頼し裁判官を説得すればなんとかできるかも,という道筋を強く印象づける深夜番組のテレビ通販の説明を見る気分になります。

 最終的な結果を出せるのが裁判官なので,法律のプロとして弁護士が働くのは一見当然のことのようですが,深澤諭史弁護士ほどバランスが極端になると,これはとんでもない話だと考えるのが慣れて普通になっていきました。そういう意味では,いい教師のような恩恵をくれた深澤諭史弁護士になります。

 時刻は15時10分です。今も確認しましたが,深澤諭史弁護士のタイムラインに更新はありませんでした。一日に何度かTwitterAPIで深澤諭史弁護士関連の検索をしているのですが,同業者である弁護士や法クラのツイートでの反応は1件も確認しておらず,何事もなかったように推移しています。

 深澤諭史弁護士がもし裁判官になっていたらと想像しますが,少なくともこれまでのようなTwitterでの内心,本心の吐露というのはなく,ありがちな情報をつなぎあわせた補強で,認知や理解をしていたように思います。言い換えれば,法曹界における稀有な発見が深澤諭史弁護士にあります。

 権力が人を狂わせるという話は昔からよく聞くと思いますが,弁護士の場合は,その権力を監視し叩く立場にみえるもので,少なからず火事場泥棒のような先導やたかみの見物があるのではと考えるようになりました。検察や警察に向けた攻撃性と無視と放置とのメリハリもはっきりしています。

\begin{itemize}
\tightlist
\item
  (88/3237)
  @hirono\_hideki(奉納\さらば弁護士鉄道・泥棒神社の物語)のツイート ''.*'' 3237/3237:2021-04-24\_2027〜2021-05-12\_0904
  2021年05月12日09時11分の記録\\
  RT
  hirono\_hideki(奉納\さらば弁護士鉄道・泥棒神社の物語)|movement26({[}雑賀衆{]}movement)
  日時:2021-05-11 23:36/2021-05-11 21:30 URL:
  \url{https://twitter.com/hirono\_hideki/status/1392126386987421701} 
  \url{https://twitter.com/movement26/status/1392094650278694919} 
  \textgreater{}
  もうちょい深澤弁護士へいろいろ疑問をぶつけてほしかったな @ABEMA
  で視聴中 \url{https://t.co/S1a5BO7drf}  \#アベプラ
\end{itemize}

 リツイートの時間が昨夜の23時36分となっていました。

\begin{itemize}
\tightlist
\item
  変わる報道番組\#アベプラ①稲田朋美&カマたくと考えるLGBT法案/MCロンブー淳
  \textbar{} 【ABEMA】テレビ&ビデオエンターテインメント
  \url{https://t.co/AXoWC1vog0} 
\end{itemize}

 この法曹では,深澤諭史弁護士の話を真剣な表情で聞いている出演者の姿というのもこれまでになく印象的でした。深澤諭史弁護士のタイムラインがなぞの更新停止を続けているということで,謎への探究心が感覚と研ぎ澄ませていたのかもしれません。

 最近は数日に1回しかテレビをつけず長く視聴することもないのですが,長い時間テレビをつけたままにしていた頃も,バラエティ番組というのは余り見ないでいました。MCロンブー淳とありますが,よく見覚えのある芸能人で,テレビではいつのまにか見かけなくなっていたとも思います。

 テレビでのレギュラー出演もあるのかもしれず,たまたま見かけていなかっただけかもしれません。昨夜は思い出したこともあって,確認のため調べたのですが,これまでの認識を上回るような内容となっていました。ただ,理由はよくわかっていないところがあります。それほど調べてもいないので。

\begin{itemize}
\tightlist
\item
  (33/3237)
  @hirono\_hideki(奉納\さらば弁護士鉄道・泥棒神社の物語)のツイート ''.*'' 3237/3237:2021-04-24\_2027〜2021-05-12\_0904
  2021年05月12日09時11分の記録\\
  TW hirono\_hideki(奉納\さらば弁護士鉄道・泥棒神社の物語) 日時:
  2021-05-12 01:31 URL:
  \url{https://twitter.com/hirono\_hideki/status/1392155341408727043} 
  \textgreater{}
  保護者が``子どもに見せたくない''番組、9年連続『ロンハー』がワースト首位
  \textbar{} ORICON NEWS \url{https://t.co/anEtkdj7nN} 
  お笑いコンビ・ロンドンブーツ1号2号が司会を務める『ロンドンハーツ』(テレビ朝日系)が9年連続で1位となった。
\end{itemize}

 余り記憶がはっきりしないものの,お昼ごろの情報番組でもスタジオで姿を見ていた時期があるように思います。前から頭の回転が早く司会進行がうまそうな人という印象はありました。

\begin{itemize}
\tightlist
\item
  (21/3237)
  @hirono\_hideki(奉納\さらば弁護士鉄道・泥棒神社の物語)のツイート ''.*'' 3237/3237:2021-04-24\_2027〜2021-05-12\_0904
  2021年05月12日09時11分の記録\\
  TW hirono\_hideki(奉納\さらば弁護士鉄道・泥棒神社の物語) 日時:
  2021-05-12 01:34 URL:
  \url{https://twitter.com/hirono\_hideki/status/1392156217913479169} 
  \textgreater{} ロンQ!ハイランド 「プープー星人の逆襲」 - YouTube
  \url{https://t.co/4lJHUINOpn} 
\end{itemize}

 
ロンQ!ハイランドという番組名はほとんど記憶にも印象にもなかったのですが,プープー星人の逆襲というのはとても印象的にずっと覚えていてよく思い出していました。その番組の司会者だったと昨夜は確認したのですが,この一致がけっこうな驚きでした。深澤諭史弁護士経由です。

\begin{itemize}
\tightlist
\item
  2021年05月12日15時36分の登録:
  「プープー星人」を@hirono\_hideki @kk\_hirono @s\_hironoで検索 13件の該当 2021-05-12\_15:36の記録
  \url{https://kk2020-09.blogspot.com/2021/05/hironohidekikkhironoshirono132021-05.html} 
\end{itemize}

2019-11-30 23:35:04
``テレビに出てきた。小倉優子。ずいぶん久しぶりですっかり忘れていた。コリン星だったと思い出す。同じ頃かいくらか前に深夜帯のテレビでみていたのが「プープー星人」。そのまま弁護士星人につながった。''
\url{https://twitter.com/hirono\_hideki/status/1200785320041578498} 

 昨夜も見かけていましたが,プープー星人の逆襲のある番組は,2007年の放送という情報がありました。平成19年です。羽咋市のアパートで,深夜番組だったと思います。何曜日の放送なのかもその場限りで理解しないまま,テレビをつけたままにしているとよく始まる番組でした。

 その後,弁護士という世界の広さ,異様さ,超越した存在感を身にしみて教えてくれたのも,モトケンこと矢部善朗弁護士(京都弁護士会),小倉秀夫弁護士,そして深澤諭史弁護士になります。他にも星の数の弁護士はいますが,最も大きく輝いて見えたのが,この惑星のような世界観になります。

 最近の小倉秀夫弁護士は特におかしいと感じるツイートを見かけることが少なくなっていますが,コロナ禍の生活変化で考え方が変わったという可能性もあるいはあるのかもしれないという気がしています。

 他の弁護士らの逆張りを打算的にやっている可能性というのも否定はできないのですが,コロナ禍での人心の困窮や不安を,ウキウキ踊りだすような感じでツイートしていた時期が深澤諭史弁護士にもありました。昨日か一昨日の取り上げた昨年2020年4月のツイートのまとめです。

 余りはっきりした情報というのがネットで調べても見当たらなかったのですが,1つに15億円の報酬を東日本大震災で得たという福永活也弁護士のことがあります。なにがきっかけだったのか,他の対立した弁護士らとの情報がぴたりと見当たらなくなっているのですが,色々ありました。

\begin{lstlisting}
py37_env ❯ tun fukunagakatsuya 5
\end{lstlisting}

\begin{itemize}
\item
  TW fukunagakatsuya(福永活也@ひとり親支援法律事務所) 日時:
  2021-04-21 11:50 URL:
  \url{https://twitter.com/fukunagakatsuya/status/1384700967858302986} 
  \textgreater{}
  昨日放送された「ザ!世界仰天ニュース」の監修を担当させていただきました。\\
  \textgreater{}\\
  \textgreater{} \#世界仰天ニュース\\
  \textgreater{} \#ひとり親支援法律事務所 \url{https://t.co/1z9bpfcjpD} 
\item
  TW fukunagakatsuya(福永活也@ひとり親支援法律事務所) 日時:
  2021-04-23 10:07 URL:
  \url{https://twitter.com/fukunagakatsuya/status/1385399804319080448} 
  \textgreater{} 今日配信予定です!\\
  \textgreater{} しばらく無料で聴けるみたいです。
  \url{https://t.co/hZL0lU3JqH} 
\item
  TW fukunagakatsuya(福永活也@ひとり親支援法律事務所) 日時:
  2021-04-23 15:08 URL:
  \url{https://twitter.com/fukunagakatsuya/status/1385475528799391751} 
  \textgreater{}
  誹謗中傷の裁判(仮処分含まず)だけで50件以上が係属中だけど、返すボールが残っているものは一つもない。\\
  \textgreater{}
  期日前夜に来ても、当日朝までには返すけど、自分が期日1週間前提出を守らなかったせいで再反論の要否が検討できてないとかやめて欲しい。
\item
  TW fukunagakatsuya(福永活也@ひとり親支援法律事務所) 日時:
  2021-04-29 11:00 URL:
  \url{https://twitter.com/fukunagakatsuya/status/1387587546641625088} 
  \textgreater{} @Mentalist\_DaiGo
  立花さん、ゆたぼん親子の代理人で参戦させてください!笑
\item
  RT
  fukunagakatsuya(福永活也@ひとり親支援法律事務所)|ZATSUDAN\_JP(ZATSUDAN@プロフェッショナル同士の雑談)
  日時:2021-05-09 19:53/2021-05-09 15:00 URL:
  \url{https://twitter.com/fukunagakatsuya/status/1391345655734366215} 
  \url{https://twitter.com/ZATSUDAN\_JP/status/1391271720808374279} 
  \textgreater{} 【30日間毎日「ZATSUDAN」配信告知】\\
  \textgreater{}\\
  \textgreater{} 5/10(月) 17:30\textasciitilde17:45\\
  \textgreater{} 明日はコラボ配信です🎉\\
  \textgreater{}\\
  \textgreater{} 🌸\#堀江貴文 氏 ❎ \#福永活也 氏の予定です🌸\\
  \textgreater{}\\
  \textgreater{} 最大2週間無料の新規登録はこちら⬇️\\
  \textgreater{} \url{https://t.co/JCXhTPyzPk} 
  \textgreater{}\\
  \textgreater{} ※ 予定変更となる場合がございます。\\
  \textgreater{}\\
  \textgreater{} @takapon\_jp @fukunagakatsuya \url{https://t.co/Pmmh1ShKlV} 
\end{itemize}

\textbackslash end\{lstlisting\}

 2日ほど間を開けていたように思いますが,福永活也弁護士のタイムラインをみると,リツイートが一つ更新されていて,それが,\#堀江貴文
氏 ❎
\#福永活也 ,という内容でした。親交があるという話は見かけていましたが,対談という組み合わせは意外性を感じました。

\begin{itemize}
\tightlist
\item
  堀江貴文(Takafumi Horie)さん (@takapon\_jp) / Twitter
  \url{https://twitter.com/takapon\_jp} 
\end{itemize}

 ヘッダ画像に,何かを口に加えた顔写真と,「死なないように稼ぐ。堀江貴文 生き残るビジネスと人材」という文字列があります。おそらく著書のタイトルと紹介になるのかと思います。フォロワー数は350万となっていて,それほど変わっていないようです。

 テキーラ事件で話題となっているのを見かけ,それに対する文章というのも読みました。そのあとは名前も余り見かけていなかったように思います。

 堀江貴文氏は,検察を強く批判していた時期もありましたが,検察に人生を狂わされたという思いと関わりがあったことは確かな事実と考えています。私もまだ弁護士よりは検察や警察の不手際や責任が大きいと考えていた時期,今振り返れば,弁護士病に罹患し思考力が低下していた時期になります。

 そうそう深澤諭史弁護士といえば,弁護士に対する逆恨みが星空を駆け巡るように散りばめられた,芸術作品のようなものがありました。オペラとか演劇の人類の遺産に近いものを感じたのですが,弁護士だけの都合による世界観が見事に表現されていました。弁護士鉄道の歴史遺産です。

\begin{lstlisting}
py37_env ❯ d|grep @fukazawas|grep 逆恨み
\end{lstlisting}

\begin{itemize}
\tightlist
\item
  2017年10月05日10時45分の登録:
  %@fukazawas 深澤諭史%凶行に走るストーカーの「真意を確かめたい」率は異常に高い。
  更に「相手が自分のいうとおりになれば『こんなこと』にならなかった」のだから「自分は悪くない」として,社会を逆恨み
  \url{http://hirono2014sk.blogspot.com/2017/10/fukazawas\_5.html} 
\item
  2018年10月04日18時59分の登録:
  \深澤諭史 @fukazawas\ただ,残り1割が,「自分は真意を確かめたかっただけだ!」「ストーカー扱いされた!」って逆恨みして,非常に執着して,数十人分の犯行をするの
  \url{http://hirono2014sk.blogspot.com/2018/10/fukazawas\_82.html} 
\item
  2019年03月01日10時15分の登録:
  \深澤諭史 @fukazawas\これでは、「予備校に法曹養成の役割を奪われたことを反省どころか逆恨みし、受験資格制限でポスト確保を企み、弁護士のみ大増員で既存法曹に復讐
  \url{http://hirono2014sk.blogspot.com/2019/03/fukazawas.html} 
\item
  2019年04月17日22時17分の登録:
  \深澤諭史 @fukazawas\④②の様な事が起きなく(できなく)なる。¥\nという話をやっていたが、②ができなくなった人から逆恨みリスクは高いなぁ、と思ったり。¥\n(・∀・
  \url{http://hirono2014sk.blogspot.com/2019/04/fukazawas\_55.html} 
\item
  2019年06月16日10時56分の登録:
  \深澤諭史 @fukazawas\逆恨みを買いやすいが慣れているので、なんとも思わない。¥\nでも、一緒に働く仲間に矛先が向かうと戦闘モードに。
  \#法曹でない人が嘘だと思うけ
  \url{http://hirono2014sk.blogspot.com/2019/06/fukazawas\_48.html} 
\item
  2019年06月17日13時25分の登録:
  \深澤諭史 @fukazawas\感謝もされるが、それ以上に逆恨みをされる職業である。
  \#法曹でない人が嘘だと思うけど本当の事言え
  \url{http://hirono2014sk.blogspot.com/2019/06/fukazawas\_69.html} 
\item
  2019年06月23日16時47分の登録:
  \深澤諭史 @fukazawas\あるある。¥\n少なくとも「そういう思い込みをして逆恨みをする人」大勢を相手にする時点で、相当大変だってば(・∀・;)
  \url{http://hirono2014sk.blogspot.com/2019/06/fukazawas\_93.html} 
\item
  2020年02月09日19時01分の登録:
  \深澤諭史 @fukazawas\そもそも、法曹養成において予備校との自由競争に敗北し、それを逆恨みして、今の制度を作っておいて、この言い草はないでしょう。¥\nまずは、大学
  \url{http://hirono2014sk.blogspot.com/2020/02/fukazawas\_55.html} 
\item
  2020年03月18日21時55分の登録:
  \深澤諭史 @fukazawas\もしかして:逆恨み
  \url{http://hirono2014sk.blogspot.com/2020/03/fukazawas\_62.html} 
\item
  2020年06月29日10時20分の登録:
  \深澤諭史 @fukazawas\(・∀・)弁護士を逆恨みしている人って,ネットでは,弁護士になりすまして,弁護士に攻撃的な言動をする傾向がありますね。たぶん,本人として
  \url{http://hirono2014sk.blogspot.com/2020/06/fukazawas\_29.html} 
\item
  2020年08月05日17時44分の登録:
  \深澤諭史 @fukazawas\たまに、こういう、弁護士全体を逆恨みしているみたいな人には遭遇しますね。相談者にもいますし、ネットにはもっといっぱいいます。
  属性は統一 \url{http://hirono2014sk.blogspot.com/2020/08/fukazawas\_93.html} 
\item
  2020年08月06日08時19分の登録:
  REGEXP:''逆恨み''/深澤諭史(@fukazawas)の検索(2014-09-22〜2020-08-05/2020年08月06日08時19分の記録55件)
  \url{http://hirono2014sk.blogspot.com/2020/08/regexpfukazawas2014-09-222020-08.html} 
\item
  2020年08月08日22時08分の登録:
  \深澤諭史 @fukazawas\弁護士って、仕事柄、相手方から恨まれるというのは想定していたけれども、世の中には、「弁護士っていうだけで、逆恨みの感情をぶつける」人たち
  \url{http://hirono2014sk.blogspot.com/2020/08/fukazawas\_88.html} 
\item
  2020年08月18日11時15分の登録:
  REGEXP:''逆恨み''/深澤諭史(@fukazawas)の検索(2014-09-22〜2020-08-08/2020年08月18日11時15分の記録56件)
  \url{http://hirono2014sk.blogspot.com/2020/08/regexpfukazawas2014-09-222020-08\_18.html} 
\item
  2020年08月19日17時02分の登録:
  \深澤諭史 @fukazawas\検事も逆恨みを買うから大変ですね。盗撮とのことですが,性犯罪,ストーカー系は,そうなる割合を高く感じます。
  \url{http://hirono2014sk.blogspot.com/2020/08/fukazawas\_47.html} 
\item
  2020年08月29日11時00分の登録:
  REGEXP:''逆恨み''/深澤諭史(@fukazawas)の検索(2014-09-22〜2020-08-23/2020年08月29日11時00分の記録59件)
  \url{http://hirono2014sk.blogspot.com/2020/08/regexpfukazawas2014-09-222020-08\_29.html} 
\item
  2020年09月05日20時03分の登録:
  \深澤諭史 @fukazawas\要するにあなたみたいになりたくてもなれなかった人が,あなたを妬んで,逆恨みしているだけであり,そもそも知り合いどころか,仕事も業界も全く
  \url{http://hirono2014sk.blogspot.com/2020/09/fukazawas\_5.html} 
\item
  2020年10月07日19時11分の登録:
  \深澤諭史 @fukazawas\弁護士になる前から、関係者相手方に逆恨みされやすい仕事であることは知っていたが、なってみると、法曹特に弁護士全体を逆恨みしている人も少数
  \url{http://kk2020-09.blogspot.com/2020/10/fukazawas\_7.html} 
\item
  2020年11月07日09時38分の登録:
  REGEXP:''逆恨み''/深澤諭史(@fukazawas)の検索(2014-09-22〜2020-10-07/2020年11月07日09時37分の記録61件)
  \url{http://kk2020-09.blogspot.com/2020/11/regexpfukazawas2014-09-222020-10.html} 
\item
  2020年12月06日23時36分の登録:
  \深澤諭史 @fukazawas\弁護士って、クレーマー、ストーカー、DV加害者等から、職業全体として、逆恨み買いやすいですからね。仕事柄、最初の思い通りにならない「敵」
  \url{http://kk2020-09.blogspot.com/2020/12/fukazawas\_6.html} 
\item
  2020年12月18日23時25分の登録:
  \深澤諭史 @fukazawas\事件関係者だけではなくて、弁護士全般に逆恨みの感情を燃やしている人が一定数いるので、用心しましょう。
  \#新人弁護士に言いたいこと
  \url{http://kk2020-09.blogspot.com/2020/12/fukazawas\_86.html} 
\item
  2021年03月29日05時29分の登録:
  REGEXP:''逆恨み''/深澤諭史(@fukazawas)の検索(2014-09-22〜2020-12-18/2021年03月29日05時28分の記録65件)
  \url{https://kk2020-09.blogspot.com/2021/03/regexpfukazawas2014-09-222020-12.html} 
\end{itemize}

 「凶行に走るストーカーの「真意を確かめたい」率は異常に高い。」というフレーズは,深澤諭史弁護士のツイートで度々みかけてきたのですが,弁護士鉄道の闇の中でもがきながら身にしみて感じてきた,私自身の客観的,社会的な立場でもありました。

 長くなりましたが,その辺りも踏まえ,次に進めていきたいと思います。個別のエントリーとして取り上げることも当初の考えにあったのですが,こまごまとした本件告発事件と直接的な関わりのない事柄をこの5月11日という時点を軸にまとめ上げました。

\begin{itemize}
\tightlist
\item
  〈〈〈 2021/05/12 16:24:21 Linux Emacs: 〈〈〈
\end{itemize}

\hypertarget{ux65e5ux672cux306eux691cux5bdfux306eux30e2ux30ceux306eux8003ux3048ux65b9ux306fux7121ux610fux5473ux306aux60c5ux71b1ux6709ux5bb3ux306aux62d8ux308aux3068ux3082ux898bux3048ux308bux3042ux306eux30e2ux30c1ux30d9ux30fcux30b7ux30e7ux30f3ux3068ux3044ux3046ux6df1ux6fa4ux8aedux53f2ux5f01ux8b77ux58ebux306eux5171ux8457ux8005ux3068ux3044ux3046ux533fux540dux5f01ux8b77ux58ebux306eux30c4ux30a4ux30fcux30c8}{%
\paragraph{「日本の検察のモノの考え方は・・・無意味な情熱、有害な拘りとも見えるあのモチベーション」という深澤諭史弁護士の共著者という匿名弁護士のツイート}\label{ux65e5ux672cux306eux691cux5bdfux306eux30e2ux30ceux306eux8003ux3048ux65b9ux306fux7121ux610fux5473ux306aux60c5ux71b1ux6709ux5bb3ux306aux62d8ux308aux3068ux3082ux898bux3048ux308bux3042ux306eux30e2ux30c1ux30d9ux30fcux30b7ux30e7ux30f3ux3068ux3044ux3046ux6df1ux6fa4ux8aedux53f2ux5f01ux8b77ux58ebux306eux5171ux8457ux8005ux3068ux3044ux3046ux533fux540dux5f01ux8b77ux58ebux306eux30c4ux30a4ux30fcux30c8}}

\begin{itemize}
\tightlist
\item
  〉〉〉 Linux Emacs: 2021/06/08 14:03:10 〉〉〉
\end{itemize}

:CATEGORIES: @kanazawabengosi \#金沢弁護士会 @JFBAsns
日本弁護士連合会(日弁連) \#法務省 @MOJ\_HOUMU \#深澤諭史弁護士
\#小倉秀夫弁護士

 昨日の夕方になると思いますが、小倉秀夫弁護士のタイムラインでこの弁護士のツイートを見かけました。小倉秀夫弁護士に対する今後の方針を方向転換させるような関連性で、一晩寝かせていたのですが、朝になっていろいろと調べ記録の作成をしていました。

 もともと過食弁護士というようなプロフィールの名前でしたが、現在は過食Bとなっていて、現在の@motaberarenaiyoというメンションでは記録が比較的新しいものしか見当たらず、これまでに蓄積した記録を調べたところ、アカウント自体はそのままでユーザ名が変更されていました。

 今、自分のTwitterアカウントの設定で確認をしたのですが、ユーザ名はメンションと同じでした。アカウント名というのかとも考えたのですが、TwitterAPIでは英語でスクリーンネームともなっていてけっこうややこしいのです。

\begin{itemize}
\tightlist
\item
  1408:2021-06-07\_11:13:05 \#告発状 \#\#\#\#
  「亀戸の弁当店「キッチンDIVE」あごマスク男、菓子折持って意気消沈で来店」という問題に対するモトケンこと矢部善朗弁護士(京都弁護士会)の厳しさ
  \url{https://hirono-hideki.hatenadiary.jp/entry/2021/06/07/111301} 
\end{itemize}

 告発状作成の作業は上記エントリーの続きになりますが、三浦義隆弁護士の「処女」というキーワードをご紹介した辺りまでは記憶に残っています。その後の記録になりますが、これまで次の投稿を行い本件告発事件の参考資料としての記録をしています。

\begin{itemize}
\tightlist
\item
  2021年06月07日10時30分の登録:
  REGEXP:''処女''/ystk(@lawkus)の検索(2011-09-19〜2019-06-19/2021年06月07日10時30分の記録64件)
  \url{https://kk2020-09.blogspot.com/2021/06/regexpystklawkus2011-09-192019-06.html} 
\item
  2021年06月07日10時54分の登録:
  REGEXP:''ノーキャンドットコム''/データベース登録済みツイートの検索:2019-12-06〜2021-06-07/2021年06月07日10時54分の記録:ユーザ・投稿:3/4件
  \url{https://kk2020-09.blogspot.com/2021/06/regexp2019-12-062021-06-0720210607105434.html} 
\item
  2021年06月07日11時21分の登録:
  \渡辺輝人 @nabeteru1Q78\ウサマ・ビンラディンと同じ方法か。
  \url{https://kk2020-09.blogspot.com/2021/06/nabeteru1q78.html} 
\item
  2021年06月07日11時28分の登録:
  REGEXP:''A級戦犯7人、太平洋に散骨''/データベース登録済みツイートの検索:2021-06-07〜2021-06-07/2021年06月07日11時27分の記録:ユーザ・投稿:3/4件
  \url{https://kk2020-09.blogspot.com/2021/06/regexpa72021-06-072021-06.html} 
\item
  2021年06月07日11時49分の登録:
  REGEXP:''トヨタ.*自殺''/データベース登録済みツイートの検索:2021-06-07〜2021-06-07/2021年06月07日11時48分の記録:ユーザ・投稿:19/23件
  \url{https://kk2020-09.blogspot.com/2021/06/regexp2021-06-072021-06.html} 
\item
  2021年06月07日16時45分の登録:
  REGEXP:''キッチンDIVE''/データベース登録済みツイートの検索:2021-06-04〜2021-06-07/2021年06月07日16時44分の記録:ユーザ・投稿:4/12件
  \url{https://kk2020-09.blogspot.com/2021/06/regexpdive2021-06-042021-06.html} 
\item
  2021年06月07日17時13分の登録:
  REGEXP:''議事録''/データベース登録済みツイートの検索:2009-12-25〜2021-06-07/2021年06月07日17時08分の記録:ユーザ・投稿:108/269件
  \url{https://kk2020-09.blogspot.com/2021/06/regexp2009-12-252021-06.html} 
\item
  2021年06月07日17時18分の登録:
  \りっぴぃ @rippy08\破産申立てでも,B型肝炎訴訟でも,資料は「揃ったものからばらばらと」お送りいただきたい派です!!!きちんと全部揃えてからと思い詰めず,キャッ
  \url{https://kk2020-09.blogspot.com/2021/06/rippy08.html} 
\item
  2021年06月07日17時31分の登録:
  REGEXP:''銀河英雄伝''/データベース登録済みツイートの検索:2016-10-06〜2021-06-07/2021年06月07日17時31分の記録:ユーザ・投稿:5/11件
  \url{https://kk2020-09.blogspot.com/2021/06/regexp2016-10-062021-06.html} 
\item
  2021年06月07日19時41分の登録:
  \小倉秀夫 @chosakukenho\女性は神聖にして不可侵という話をいつまで続けるんですかね。
  \url{https://kk2020-09.blogspot.com/2021/06/chosakukenho\_7.html} 
\item
  2021年06月07日19時43分の登録:
  \小倉秀夫 @chosakukenho\学力試験に強いが出自に問題がある奴等を弁護士にするべきでないという声はよく聞き、それがロースクール制度として実現したと理解していま
  \url{https://kk2020-09.blogspot.com/2021/06/chosakukenho\_61.html} 
\item
  2021年06月07日19時47分の登録:
  \小倉秀夫 @chosakukenho\燃やす側が目的を達成するか、飽きるまで。燃やす側が命を奪うことを目的に設定している場合、死ぬか飽きるか。
  \url{https://kk2020-09.blogspot.com/2021/06/chosakukenho\_85.html} 
\item
  2021年06月07日19時49分の登録:
  \小倉秀夫 @chosakukenho\ただ、もはや無実だと考えられている袴田事件でさえああだということは、死刑制度やばいと考えさせる一つの要因ではありますね。
  \url{https://kk2020-09.blogspot.com/2021/06/chosakukenho\_27.html} 
\item
  2021年06月07日19時49分の登録:
  \小倉秀夫 @chosakukenho\袴田事件で今なお徹底抗戦する検察の考え方と、その抵抗に応える一部裁判官のモチベーションがどこにあるのかは私にもよくわかりません。
  \url{https://kk2020-09.blogspot.com/2021/06/chosakukenho\_34.html} 
\item
  2021年06月07日19時51分の登録: \過食B @motaberarenaiyo\返信先:
  @chosakukenhoさん日本の検察のモノの考え方は、現在の日本の法制度からの唯一の帰結なんでしょうか。無意味な
  \url{https://kk2020-09.blogspot.com/2021/06/bmotaberarenaiyo-chosakukenho.html} 
\item
  2021年06月07日19時53分の登録:
  \小倉秀夫 @chosakukenho\袴田さんについてなお死に相応しいやつだと検察が考え続けている点に考えてみる。
  \url{https://kk2020-09.blogspot.com/2021/06/chosakukenho\_26.html} 
\item
  2021年06月07日20時10分の登録:
  %@aphros67 小動物を愛するしんさん%\url{https://twitter.com/aphros67/status/1401525708720017412} 
  \url{https://kk2020-09.blogspot.com/2021/06/aphros67httpstwittercomaphros67status14.html} 
\item
  2021年06月07日20時11分の登録:
  %@aphros67 小動物を愛するしんさん%死刑廃止国が犯人を射殺した件数を調査してみた
  - 痩せるコーラ \url{https://kk2020-09.blogspot.com/2021/06/aphros67.html} 
\item
  2021年06月07日20時12分の登録:
  @aphros67(小動物を愛するしんさん)のツイート ''.*'' 3227/3227:2021-05-08\_2113〜2021-06-07\_1726 2021年06月07日20時12分の記録
  \url{https://kk2020-09.blogspot.com/2021/06/aphros67322732272021-05-0821132021-06.html} 
\item
  2021年06月07日20時15分の登録: \坂本正幸 @sakamotomasayuk\返信先:
  @aphros67さんそんな気がしてた
  \url{https://kk2020-09.blogspot.com/2021/06/sakamotomasayuk-aphros67.html} 
\item
  2021年06月07日20時40分の登録:
  REGEXP:''岡口''/データベース登録済みツイートの検索:2021-06-05〜2021-06-07/2021年06月07日20時39分の記録:ユーザ・投稿:17/25件
  \url{https://kk2020-09.blogspot.com/2021/06/regexp2021-06-052021-06.html} 
\item
  2021年06月07日20時41分の登録:
  「@aphros67」を@hirono\_hideki @kk\_hirono @s\_hironoで検索 158件の該当 2021-06-07\_20:41の記録
  \url{https://kk2020-09.blogspot.com/2021/06/aphros67hironohidekikkhironoshirono1582.html} 
\item
  2021年06月07日21時16分の登録: \豪弁 足立敬太
  @ザンギの極み @keita\_adachi\顧問先から大変満足いただいてるとお褒めいただいてのご褒美
  \url{https://kk2020-09.blogspot.com/2021/06/keitaadachi.html} 
\item
  2021年06月07日21時18分の登録:
  \MakotoAkishige(civilista) @akishigemakoto\3か月間の禁酒と節制生活のおかげで、700を超えてたγ-gtpが100台にもう一息
  \url{https://kk2020-09.blogspot.com/2021/06/makotoakishigecivilistaakishigemakoto37.html} 
\item
  2021年06月07日21時19分の登録:
  \福山和人 @kaz\_fukuyama\竹中氏が会長を務めるパソナは東京五輪のスポンサーとして、大会運営業務を受託して過去最高の営業利益を荒稼ぎ。パソナのために国民は犠牲
  \url{https://kk2020-09.blogspot.com/2021/06/kazfukuyama.html} 
\item
  2021年06月07日21時26分の登録: \太田
  伸二 @shin2\_ota\また、これから作られる法律で、一時金を国から受け取ることができるようになる見込みです。もし「どういうことか話を聞こう」と思ったら各地の
  \url{https://kk2020-09.blogspot.com/2021/06/shin2ota.html} 
\item
  2021年06月07日21時35分の登録:
  REGEXP:''発言削除''/データベース登録済みツイートの検索:2018-04-04〜2021-06-07/2021年06月07日21時34分の記録:ユーザ・投稿:7/10件
  \url{https://kk2020-09.blogspot.com/2021/06/regexp2018-04-042021-06.html} 
\item
  2021年06月07日21時44分の登録:
  REGEXP:''立民議員''/データベース登録済みツイートの検索:2021-06-07〜2021-06-07/2021年06月07日21時44分の記録:ユーザ・投稿:4/4件
  \url{https://kk2020-09.blogspot.com/2021/06/regexp2021-06-072021-06-0720210607214444.html} 
\item
  2021年06月08日00時39分の登録:
  REGEXP:''東條英機''/データベース登録済みツイートの検索:2015-10-02〜2021-06-07/2021年06月08日00時39分の記録:ユーザ・投稿:6/8件
  \url{https://kk2020-09.blogspot.com/2021/06/regexp2015-10-022021-06-0720210608003968.html} 
\item
  2021年06月08日00時43分の登録:
  \モトケン @motoken\_tw\男性と警察を敵にまわして痴漢被害をなくそうというのはかなり無理があると思うけどな。
  \url{https://kk2020-09.blogspot.com/2021/06/motokentw\_8.html} 
\item
  2021年06月08日00時48分の登録:
  2021-06-07の投稿一覧\検察・石川県警察宛記録資料\奉納\危険生物・弁護士脳汚染除去装置\金沢地方検察庁御中:40件
  \url{https://kk2020-09.blogspot.com/2021/06/2021-06-0740.html} 
\item
  2021年06月08日00時48分の登録:
  ツイートの記録資料:\法務検察・石川県警察宛\/深澤諭史(@fukazawas)/''2021年06月07日'':7件
  \url{https://kk2020-09.blogspot.com/2021/06/fukazawas202106077.html} 
\item
  2021年06月08日00時48分の登録:
  ツイートの記録資料:\法務検察・石川県警察宛\/小倉秀夫(@chosakukenho)/''2021年06月07日'':42件
  \url{https://kk2020-09.blogspot.com/2021/06/chosakukenho2021060742.html} 
\item
  2021年06月08日00時48分の登録:
  ツイートの記録資料:\法務検察・石川県警察宛\/モトケン(@motoken\_tw)/''2021年06月07日'':12件
  \url{https://kk2020-09.blogspot.com/2021/06/motokentw2021060712.html} 
\item
  2021年06月08日01時35分の登録:
  \うの字を名乗る?物 @un\_co\_the2nd\性交同意年齢の引き上げを、常に「おっさんと女子中学生」の設定で語ってる人は偏見がキツすぎて使い物にならん。中学生と中学生
  \url{https://kk2020-09.blogspot.com/2021/06/uncothe2nd\_8.html} 
\item
  2021年06月08日09時14分の登録:
  REGEXP:''再審''/データベース登録済みツイートの検索:2021-06-04〜2021-06-07/2021年06月08日09時13分の記録:ユーザ・投稿:18/27件
  \url{https://kk2020-09.blogspot.com/2021/06/regexp2021-06-042021-06\_8.html} 
\item
  2021年06月08日09時42分の登録:
  \小倉秀夫 @chosakukenho\袴田さんについてなお死に相応しいやつだと検察が考え続けている点に考えてみる。
  \url{https://kk2020-09.blogspot.com/2021/06/chosakukenho\_8.html} 
\item
  2021年06月08日09時43分の登録:
  \教皇ノースライム @noooooooorth\「相手を試すためにネガティブなことを言う」のは愚行だと思う。相手を試してはいけない。
  \url{https://kk2020-09.blogspot.com/2021/06/noooooooorth.html} 
\item
  2021年06月08日11時14分の登録:
  REGEXP:''@motaberarenaiyo''/データベース登録済みツイートの検索:2021-04-21〜2021-06-08/2021年06月08日11時11分の記録:ユーザ・投稿:75/168件
  \url{https://kk2020-09.blogspot.com/2021/06/regexpmotaberarenaiyo2021-04-212021-06.html} 
\item
  2021年06月08日11時29分の登録:
  「過食」を@hirono\_hideki @kk\_hirono @s\_hironoで検索 419件の該当 2021-06-08\_11:29の記録
  \url{https://kk2020-09.blogspot.com/2021/06/hironohidekikkhironoshirono4192021-06.html} 
\item
  2021年06月08日11時29分の登録:
  「@motaberarenaiyo」を@hirono\_hideki @kk\_hirono @s\_hironoで検索 27件の該当 2021-06-08\_11:28の記録
  \url{https://kk2020-09.blogspot.com/2021/06/motaberarenaiyohironohidekikkhironoshir.html} 
\item
  2021年06月08日11時37分の登録:
  REGEXP:''@juntaba1''/データベース登録済みツイートの検索:2012-07-10〜2021-03-26/2021年06月08日11時33分の記録:ユーザ・投稿:122/259件
  \url{https://kk2020-09.blogspot.com/2021/06/regexpjuntaba12012-07-102021-03.html} 
\item
  2021年06月08日11時37分の登録:
  %@motaberarenaiyo 過食B%\url{https://twitter.com/motaberarenaiyo/status/826363506333081600} 
  \url{https://kk2020-09.blogspot.com/2021/06/motaberarenaiyobhttpstwittercommotabera.html} 
\item
  2021年06月08日11時39分の登録:
  %@fukazawas 深澤諭史%(・∀・)(・∀・)(・∀・)¥\n\#私のことどう思ってるか引用RTで正直に容赦なく言ってみてほしい
  \url{https://kk2020-09.blogspot.com/2021/06/fukazawasnrt.html} 
\item
  2021年06月08日12時02分の登録:
  %@motaberarenaiyo 過食B%検察幹部って、鬱憤晴らすために嘘ばらまくTwitter界隈の馬鹿な素人と同じレベルで務まるのか?
  \url{https://kk2020-09.blogspot.com/2021/06/motaberarenaiyobtwitter.html} 
\item
  2021年06月08日12時05分の登録:
  %@motaberarenaiyo 過食B%\url{https://twitter.com/motaberarenaiyo/status/1233598320355856384} 
  \url{https://kk2020-09.blogspot.com/2021/06/motaberarenaiyobhttpstwittercommotabera\_8.html} 
\item
  2021年06月08日12時08分の登録:
  %@motaberarenaiyo 過食B%弁護側の戦術はどうか、検察の進め方はどうかと検討する必要があります。¥\n無罪って結構な確率で検察の敵失に乗じて獲得されているのです。
  \url{https://kk2020-09.blogspot.com/2021/06/motaberarenaiyobn.html} 
\item
  2021年06月08日12時11分の登録:
  %@motaberarenaiyo 過食B%\url{https://twitter.com/motaberarenaiyo/status/1211978547150999553} 
  \url{https://kk2020-09.blogspot.com/2021/06/motaberarenaiyobhttpstwittercommotabera\_49.html} 
\item
  2021年06月08日12時16分の登録:
  %@motaberarenaiyo 過食B%\url{https://twitter.com/motaberarenaiyo/status/1377069828591128577} 
  \url{https://kk2020-09.blogspot.com/2021/06/motaberarenaiyobhttpstwittercommotabera\_57.html} 
\item
  2021年06月08日12時18分の登録:
  %@motaberarenaiyo 過食B%検察ってローカル権力な影響下にはないと思うんだけどどーなんだろ。気になる。¥\nまあ告訴状持ってって受け付けてもらえることなんて私風情だと10回に1度くらいですが。
  \url{https://kk2020-09.blogspot.com/2021/06/motaberarenaiyobn\_8.html} 
\end{itemize}

 まだ内容を閲覧していなかったまとめ記事になると思いますが、次のツイートを見つけました。鴨志田裕美弁護士のツイートです。

\begin{itemize}
\tightlist
\item
  TW kamo629782(かもん弓(鴨志田 祐美)) 日時: 2021/06/06 12:07:11
  URL: \url{https://twitter.com/kamo629782/status/1401375084703469570} 
  \textgreater{} 何と2週連続で日曜日の京都新聞に載せていただきました!\\
  \textgreater{}\\
  \textgreater{}
  大崎事件第4次再審は、いよいよ今週となった澤野尋問で再審開始に大きく近づきます。\\
  \textgreater{}\\
  \textgreater{} ご注目下さい‼️ \url{https://t.co/AMtlDRQMmn} 
\end{itemize}

 最近の再審の話題は、この鴨志田裕美弁護士と大崎事件が独走状態という感じです。ついでたまに見かけるのが袴田事件で、その袴田事件に対する小倉秀夫弁護士のツイートが過食弁護士のツイートの発見に繋がりました。元を辿ると、これもブロックされているお馴染のアカウントのツイートがありました。

 令和3年3月31日付告発状でも少し取り上げているように思うのですが、Twitterのヘッダ画像がブルーハーツの「キスしてほしい」という曲を思い出させるものとなっていました。令和3年3月31日付告発状で触れたのか憶えていないですが、羽咋市のパチンコ店、ダイナムにあったパチスロ機のことです。

 スクリーンキャストで画面の動画を撮影するつもりはなかったのですが、Twitterのページを開くと下から風船が出てきて、これは珍しく、なにか特別なめぐり合わせなのかと思い、記念もかねて動画の記録を作成し、YouTubeにアップロードしました。

\begin{itemize}
\tightlist
\item
  2021年06月07日20時14分22秒の記録_「袴田事件で今なお徹底抗戦する検察の考え方と、その抵抗に応える一部裁判官」という小倉秀夫弁護士のツイートを遡った記録
  - YouTube \url{https://t.co/IDJfZz0quo}  3 回視聴•2021/06/07 ¥\n  ¥\n 0
  ¥\n  ¥\n 0 ¥\n  ¥\n 共有 ¥\n  ¥\n 保存 ¥\n 
\end{itemize}

 カラフルな風船が飛び出した小動物を愛するしんさん、というTwitterアカウントについては別に取り上げる予定ですが、ずいぶん前から見かけているアカウントで早くからブロックされ、モトケンこと矢部善朗弁護士(京都弁護士会)のブログの常連コメンテーターの可能性を考えたこともありました。

 どうも過食弁護士がその小動物を愛するしんさん、というTwitterアカウントのツイートに反応し、連鎖的に小倉秀夫弁護士が反応したようでした。プロフィールにはジャーナリストともあり、死刑に関する専門的な資料をネットで公開されていましたが、ずいぶんと前の年月となっていたように思います。

 時刻は6月9日12時58分です。確認したところ次のツイートから中断をしていました。

 クリップボードが機能しなくなっているので、このあと実行中のスクリプトが終了すればパソコンを再起動するつもりです。昨日もパソコンに不具合が発生したのですが、GPUのドライバーがシステムのアップデートで保留となっていました。

 現象自体は前にも一度経験していたのですが、サスペンドから復帰すると画面が真っ暗の状態です。GPUの問題でした。

\begin{itemize}
\tightlist
\item
  TW kk\_hirono(刑事告発・非常上告_金沢地方検察庁御中) 日時:
  2021-06-08 15:03 URL:
  \url{https://twitter.com/kk\_hirono/status/1402144112262873090} 
\end{itemize}

 スクリプトが終了するとクリップボードの不具合が直っていました。kk2020-09\_ajx-all-user-mysql-REGEXP\_blogger.rb
``@kamatatylaw'' ``2021-05-03/2021-06-09 13:09'' を次に実行します。

 昨夜もまた新たな発見がいろいろとあって、そちらに集中してしまったのですが、ずいぶんと視野が広まり、考えの整理もできたように思います。1つの匿名弁護士アカウントがきっかけで、ブロックされているものと思い込んでいたのですが、作業終了後にリツイートができていました。

 勘違いしたものを作っていたのでやり直しをしました。

\begin{itemize}
\tightlist
\item
  2021年06月09日13時17分の登録:
  REGEXP:''@kamatatylaw''/データベース登録済みツイートの検索:2021-05-03〜2021-06-09/2021年06月09日13時15分の記録:ユーザ・投稿:62/174件
  \url{https://kk2020-09.blogspot.com/2021/06/regexpkamatatylaw2021-05-032021-06.html} 
\item
  2021年06月09日13時21分の登録:
  「@kamatatylaw」を@hirono\_hideki @kk\_hirono @s\_hironoで検索 1149件の該当 2021-06-09\_13:21の記録
  \url{https://kk2020-09.blogspot.com/2021/06/kamatatylawhironohidekikkhironoshirono1.html} 
\end{itemize}

 目的は「「@kamatatylaw」を@hirono\_hideki @kk\_hirono @s\_hironoで検索 1149件の該当 2021-06-09\_13:21の記録」の方で、最初にブロックされていることを確認したスクリーンショットの記録を調べることでした。ページ内検索です。

2017-11-05 11:22:06
``2017-11-05-112201\_ブロックされているため、@kamatatylawさんのフォローや@kamatatylawさんのツイートの表示はできません。詳細はこちら.jpg
\url{http://pic.twitter.com/qRzPHJKgSn''} 
\url{https://twitter.com/s\_hirono/status/926998024324128768} 

 2017年11月5日のスクリーンショットが最初に出てきました。

\begin{itemize}
\tightlist
\item
  奉納\さらば弁護士鉄道・泥棒神社の物語(@hirono\_hideki)/「弁護士にとって,国相手に勝訴するのって,アドレナリンが爆発するぐらいのものじゃないかなあ。」の検索結果
  - Twilog \url{https://t.co/24Stbrj08R} 
\end{itemize}

 もとの高橋雄一郎弁護士のツイートが削除されているようですが、Twilogの仕様としてリツイートしたツイートが残っていました。2021-06-09\_13:21の記録で見つけたものですが、初めて高橋雄一郎弁護士のツイートをリツイートしたもののようです。

 高橋雄一郎弁護士については2015年以降に知ったアカウントという可能性も頭にあったのですが、2011年2月19日にリツイートをしていたことになります。

\begin{itemize}
\item
  奉納\さらば弁護士鉄道・泥棒神社の物語(@hirono\_hideki)/2011年02月19日
  - Twilog \url{https://t.co/3MqCH2rEuy} 
\item
  RT
  hirono\_hideki(奉納\さらば弁護士鉄道・泥棒神社の物語)|motoken\_tw(モトケン)
  日時:2011/02/19 21:51:25/2011/02/19 21:43 URL:
  \url{https://twitter.com/hirono\_hideki/status/38943736964710400} 
  \url{https://twitter.com/motoken\_tw/status/38941842682810368\textgreater} {}
  RT @motoken\_tw:
  そういう事実を制度設計にどう反映させるべきだとお考えですか?RT
  @amneris84: @motoken\_tw
  事件の犯人であっても、その詳細について検察側の筋書きを押し付けられたことで反発し、被害者に対する謝罪の気持ちより、検察との戦闘モードになってしまった人もいます。捜査・・・
\end{itemize}

 2012年ではなく2011年だったので訂正しました。当時は多かった非公式RTですが、ジャーナリストの江川紹子氏のツイートにコメントをつけ、そのジャーナリストの江川紹子氏のツイートはモトケンこと矢部善朗弁護士(京都弁護士会)に対する返信という感じです。

〉〉〉 kk\_hironoのリツイート 〉〉〉

\begin{itemize}
\tightlist
\item
  RT
  kk\_hirono(刑事告発・非常上告_金沢地方検察庁御中)|amneris84(Shoko
  Egawa) 日時:2021-06-09 13:44/2011/02/19 21:39 URL:
  \url{https://twitter.com/kk\_hirono/status/1402486840679010304} 
  \url{https://twitter.com/amneris84/status/38940776587075584} 
  \textgreater{} @motoken\_tw
  事件の犯人であっても、その詳細について検察側の筋書きを押し付けられたことで反発し、被害者に対する謝罪の気持ちより、検察との戦闘モードになってしまった人もいます。捜査側がその筋書きに沿った「反省」を強いることが、逆効果になることも。
\end{itemize}

 やはりモトケンこと矢部善朗弁護士(京都弁護士会)に対するジャーナリストの江川紹子氏の返信ツイートでしたが、普通にTwitterやGoogleで検索すると見つかりませんでした。

 次がブロックされ、スクリーンショットの記録も作成しているはずと思いながら見込みが外れたことを確認したまとめ記事になります。

\begin{itemize}
\tightlist
\item
  奉納\危険生物・弁護士脳汚染除去装置\金沢地方検察庁御中\_2020:
  「@jmjhjmwtad」を@hirono\_hideki @kk\_hirono @s\_hironoで検索 283件の該当 2021-06-09\_12:38の記録
  \url{https://kk2020-09.blogspot.com/2021/06/jmjhjmwtadhironohidekikkhironoshirono28.html} 
\end{itemize}

 しばらく前にもリツイートをしていた記録があったのですが、昨日の6月8日までは間がありました。

 261から最後の283のツイートまで次に引用します。短いツイートをいくつかリツイートします。

2021-05-26 10:46:17 ``RT @9jtCdbGf3lih8Fe: @jmjhjmwtad
加害者の弁護人に対してと、被害者の代理人に対してとではね、¥\n検事が行うコミュニケーションの内容が違うのですよ。''
\url{https://twitter.com/kk\_hirono/status/1397368459730718722} 

2021-05-26 10:46:20 ``RT @jmjhjmwtad: @9jtCdbGf3lih8Fe
具体的にどう違うんですか?''
\url{https://twitter.com/kk\_hirono/status/1397368474268168196} 

2021-05-26 10:46:24 ``RT @9jtCdbGf3lih8Fe: @jmjhjmwtad
どうでしょうね。¥\nご自身で確かめたらいかがですか?''
\url{https://twitter.com/kk\_hirono/status/1397368490856620034} 

2021-05-26 10:46:33 ``RT @jmjhjmwtad:
私は「痴漢事件において示談不成立の場合検察官は被疑者が何をやっても略式で落とす」という貴職の見解の根拠を質問させて頂きました。¥\n当然「検察内部の運用を検事から聞いて知った」など、相応の根拠があって、ツイッターという公開の場で発言されていると思います。¥\n根拠を教えて頂けますか?
\url{https://t.co/C9L6ZA5EZK''} 
\url{https://twitter.com/kk\_hirono/status/1397368525648392193} 

2021-05-26 10:46:51 ``RT @9jtCdbGf3lih8Fe: @jmjhjmwtad
諸事情により、今までお答えした以上のお答えはお断りします。¥\n別に回答の義務は負っていません。''
\url{https://twitter.com/kk\_hirono/status/1397368604895551490} 

2021-05-26 11:26:15
``2021-05-26-100344\_弁護士 岸本 学@9jtCdbGf3lih8Fe返信先: @jmjhjmwtadさん諸事情により、今までお答えした以上のお答えはお断りします.jpg
\url{https://t.co/otpdef2nWL''} 
\url{https://twitter.com/s\_hirono/status/1397378518988722181} 

2021-05-26 11:26:33
``2021-05-26-100536\_弁護士 岸本 学@9jtCdbGf3lih8Fe返信先: @jmjhjmwtadさんどうでしょうね。ご自身で確かめたらいかがですか?.jpg
\url{https://t.co/Oss0sym0yd''} 
\url{https://twitter.com/s\_hirono/status/1397378592640618502} 

2021-05-26 11:29:02 ``2021-05-26\_10:05
奉納\\#危険生物・弁護士脳汚染除去装置\\#金沢地方検察庁御中\_2020:
\弁護士 岸本 学 @9jtCdbGf3lih8Fe\返信先:
@jmjhjmwtadさんどうでしょうね。ご自身で確かめたらいかがですか?
\url{https://t.co/CjuLqHO2pH''} 
\url{https://twitter.com/hirono\_hideki/status/1397379218254663686} 

2021-05-28 20:41:38 ``RT @jmjhjmwtad:
これ多分キャバ嬢だよね。¥\nよく分からんのが客の自宅にキャバ嬢が来るって、そんな頻繁にあるケースなのか?自宅まで来た過程はどやったんやろか。¥\n¥\n千葉地検が弁護士を追起訴 自宅に飲食店の女性連れ込み、殴ってわいせつ疑い(千葉日報オンライン)¥\n\#Yahooニュース¥\n\url{https://t.co/gVulAF0uEN''} 
\url{https://twitter.com/hirono\_hideki/status/1398243062044782596} 

2021-05-28 20:43:03 ``RT @jmjhjmwtad:
日本人って、別に真面目でも何でもなくて「赤信号皆で渡れば怖くない」の民族なんで、飲食店の間で、自粛要請に従わないのが普通の空気になっちゃうと、一気に殆どの飲食店が営業開始しそうですね。多分、小池と吉村がキレてもガン無視されるな。''
\url{https://twitter.com/hirono\_hideki/status/1398243417835012104} 

2021-05-28 20:43:31 ``RT @jmjhjmwtad:
人権派弁護士と清楚系AV女優は対偶の関係が成立しています。''
\url{https://twitter.com/hirono\_hideki/status/1398243535174832139} 

2021-05-28 20:43:42 ``RT @jmjhjmwtad:
これあんまツッコミ入ってないけどキャバ嬢自宅まで普通こないよなぁ。自宅まで来てなんで殴るんや?よう分からん事件やな。
\url{https://t.co/fuzltX3n3K''} 
\url{https://twitter.com/hirono\_hideki/status/1398243579298942980} 

2021-05-28 20:44:55 ``RT @jmjhjmwtad:
いや俺、キャバクラって全く行かないんでよく分からんのやけど、いわゆるアフターでも、客の自宅にキャバ嬢が行くって、かなりハードルが高い行為やと思うんよな。複数のキャバ嬢が1、2月の期間に自宅に来るって、結構珍しくないか。「連れ込み」とかサラッと書いてるがここが多分大事な気がするな。''
\url{https://twitter.com/hirono\_hideki/status/1398243888502956034} 

2021-05-28 20:49:17
``2021-05-28-204043\_K 9 9 9 9さんがリツイート7286@jmjhjmwtad·5月27日これ多分キャバ嬢だよね。よく分からんのが客の自宅にキャバ嬢が.jpg
\url{https://t.co/tYxsMU4STS''} 
\url{https://twitter.com/s\_hirono/status/1398244988115251204} 

2021-05-28 20:49:35
``2021-05-28-204352\_7286@jmjhjmwtad·10時間これあんまツッコミ入ってないけどキャバ嬢自宅まで普通こないよなぁ。自宅まで来てなんで殴るんや?よう分.jpg
\url{https://t.co/gwDXzSrJ5l''} 
\url{https://twitter.com/s\_hirono/status/1398245060722847747} 

2021-05-28 20:49:52
``2021-05-28-204848\_7286@jmjhjmwtad·5月27日大手事務所の事務所内穴兄弟を量産するヤリマ○秘書集めて、座談会開いて欲しいですな。「知財系の先生は.jpg
\url{https://t.co/FyUWVYXekX''} 
\url{https://twitter.com/s\_hirono/status/1398245132923666433} 

2021-05-28 20:50:22 ``2021-05-28\_20:46
奉納\\#危険生物・弁護士脳汚染除去装置\\#金沢地方検察庁御中\_2020:
\7286 @jmjhjmwtad\しかし大手法律事務所のヤリマ〇秘書、実在しても、よその事務所には貞操堅そうで、なんかムカつくな。どの事務所なの?あぁ、I合同、別に会っ
\url{https://t.co/Lrab6gw2J2''} 
\url{https://twitter.com/hirono\_hideki/status/1398245259742564353} 

2021-05-28 20:51:17 ``RT @jmjhjmwtad:
法テラで受けなかったらキレる弁護士いるって、地方はしんどそうですな。ツイート見るだけで分かる明らかに偏りある人が要職ついてると地獄やね。弊害は無能上司の比ではない。''
\url{https://twitter.com/hirono\_hideki/status/1398245490446135307} 

2021-05-28 20:51:26 ``RT @jmjhjmwtad:
キャバ嬢って、自分の腐った性格を治さないと、まず貯金が増えない。幸せになる云々の前に、性格糞だと金が貯まらない。間違いない。''
\url{https://twitter.com/hirono\_hideki/status/1398245525212696576} 

2021-06-08 21:40:09
``2021-06-08-213657\_高橋雄一郎さんがリツイート7286@jmjhjmwtad·6時間ドットコムはメディアの要件満たしてないと言われても仕方がない。メディア名乗り.jpg
\url{https://t.co/yUGOUqnFYy''} 
\url{https://twitter.com/s\_hirono/status/1402244055724032008} 

2021-06-08 21:40:39 ``2021-06-08\_21:36
奉納\\#危険生物・弁護士脳汚染除去装置\\#金沢地方検察庁御中\_2020:
\7286 @jmjhjmwtad\ドットコムはメディアの要件満たしてないと言われても仕方がない。メディア名乗りたいんであれば、最低限の中立性と、いわゆる一般人を揶揄しな
\url{https://t.co/iqDE5C6l4N''} 
\url{https://twitter.com/hirono\_hideki/status/1402244180563218434} 

2021-06-09 03:05:17
``2021-06-09-021506\_7286@jmjhjmwtad·6月5日返信先: @harrier0516oskさんまぁ淡々と処理するだけですよね。なんか性犯罪の示談につい.jpg
\url{https://t.co/fTjpOVlOe7''} 
\url{https://twitter.com/s\_hirono/status/1402325876658753538} 

2021-06-09 03:07:21 ``2021-06-09\_02:14
奉納\\#危険生物・弁護士脳汚染除去装置\\#金沢地方検察庁御中\_2020:
\7286 @jmjhjmwtad\返信先:
@harrier0516oskさんまぁ淡々と処理するだけですよね。なんか性犯罪の示談について、ネットで変な風潮ができてて、弁
\url{https://t.co/OwlWCnaEvY''} 
\url{https://twitter.com/hirono\_hideki/status/1402326397045985280} 

 プロフィールの名前が7286となっていて、このような数字だけの名前というのは他に記憶にないのですが、岸本学弁護士とのやりとりのあったアカウントということは憶えておらず、高橋雄一郎弁護士のタイムラインで、弁護士ドットコムのモラハラ連載記事を批判するようなツイートを見て注目したのが昨夜の始まりだったようです。

 時刻は2021年6月9日14時07分です。これまで告発状の文章をTwitterに投稿するのに全部全角文字の場合140文字という制限で、文章の書き直しを行っていたのですが、これからは無理に1つのツイートにまとめず、部分的にツイートすることにします。

 自分で作ったスクリプトというプログラムですが、文字列を範囲選択し、続けてボタンを押すと、範囲選択した文字列がツイートで投稿するようにしています。これだと文字数オーバーの場合、エラーが出るのでわかりやすくなっています。半角が交じると140文字を超えますがカウントがややこしくなります。

 上記のツイートは142文字となっていましたが、半角の数字が3つあったのでツイートに成功していました。なお、私は行頭に字下げの全角スペースを入れているのですが、ツイートでは行頭の全角スペースが自動で削除されています。

 2件のキャバ嬢に対する強制性交等致傷で逮捕された千葉市の弁護士のことは、けっこう重点的に取り上げたことがあると思いますが、疑問を投げかけない弁護士らに疑問があって、そういう中で1つだけみかけていたのが、7286というプロフィールの名前の匿名弁護士のツイートでした。

 なお、これから試験的に告発\市場急配センター殺人未遂事件\金沢地方検察庁・石川県警察御中(@kk\_hirono)のアカウントで、奉納\さらば弁護士鉄道・泥棒神社の物語(@hirono\_hideki)のアカウントをミューとしてみたいと思います。通知がどうなるのかという確認です。

 ミュートすると通知から奉納\さらば弁護士鉄道・泥棒神社の物語(@hirono\_hideki)のツイートが消え、ミュートを解除するとまた表示されていました。返信の数の多さでは小倉秀夫弁護士に苦情の指摘を受けたこともあったのですが、それだけ注目度が高いお知らせという事実上の意味も重視しています。

 小倉秀夫弁護士は、つい最近も、ブロックされているのでツイートが読めないとツイートをしていましたが、Twitterはログアウトした状態で制限なく閲覧ができると思います。返信ツイートとなるとフォロー関係がないと表示がないという話もありますが、その辺りはかなりややこしく感じています。

〉〉〉 kk\_hironoのリツイート 〉〉〉

\begin{itemize}
\tightlist
\item
  RT
  kk\_hirono(刑事告発・非常上告_金沢地方検察庁御中)|C1m2YqMVIfk7VHj(田中みなみになりた医@婚活中)
  日時:2021-06-09 14:31/2021/06/06 00:20 URL:
  \url{https://twitter.com/kk\_hirono/status/1402498652069715968} 
  \url{https://twitter.com/C1m2YqMVIfk7VHj/status/1401197287561338884} 
  \textgreater{}
  実は先日マッチングアプリで女医好きの弁護士との軽い合コンがあった。
  なぜ女医が好きか問うと「頑張ってるから」バンドマン風経営者「わかります、頑張ってますよね。よければ紹介しますよ」頑張ってるの意味が分からないし苛々🤗そして帰り道家に持ち帰ろうとしてきた弁護士さん🤗
\end{itemize}

〉〉〉 kk\_hironoのリツイート 〉〉〉

\begin{itemize}
\tightlist
\item
  RT
  kk\_hirono(刑事告発・非常上告_金沢地方検察庁御中)|jmjhjmwtad(7286)
  日時:2021-06-09 14:31/2021/06/09 13:37 URL:
  \url{https://twitter.com/kk\_hirono/status/1402498668620500995} 
  \url{https://twitter.com/jmjhjmwtad/status/1402485079071346691} 
  \textgreater{} 率直に行って、新興大手の一部の事件処理は疑問が残る。
  え、こんな低額で和解してくれるのwwあざすww
  訴訟になった時の見通し説明した上で、依頼者を説得する作業大変でしたねお疲れ様です
  みたいなケースが過去ウチあったよ。俺はその事務所に感謝した。
\end{itemize}

〉〉〉 kk\_hironoのリツイート 〉〉〉

\begin{itemize}
\tightlist
\item
  RT
  kk\_hirono(刑事告発・非常上告_金沢地方検察庁御中)|jmjhjmwtad(7286)
  日時:2021-06-09 14:32/2021/06/09 13:29 URL:
  \url{https://twitter.com/kk\_hirono/status/1402498748291321858} 
  \url{https://twitter.com/jmjhjmwtad/status/1402482855435923457} 
  \textgreater{}
  強制性行罪の改正、審議会のメンツからもう結論は決まってると思ってたけど違うんかな?いや早い地域やと中学で初体験()みたいな話結構あるしなぁ。ああいうの全部刑法犯なるんか。
\end{itemize}

〉〉〉 kk\_hironoのリツイート 〉〉〉

\begin{itemize}
\tightlist
\item
  RT
  kk\_hirono(刑事告発・非常上告_金沢地方検察庁御中)|nise\_mike\_ross(ノーネクタイのマイクロス)
  日時:2021-06-09 14:32/2021/06/08 21:08 URL:
  \url{https://twitter.com/kk\_hirono/status/1402498852955955200} 
  \url{https://twitter.com/nise\_mike\_ross/status/1402236176124153860} 
  \textgreater{}
  Twitter上でチヤホヤされている法クラの人が、実際だと人格等に問題があるという例は複数サンプル確認済なので、まだ見ぬ法クラの方についても過度な期待はせずに3割引くらいしておこうねと白金台のラボエムでマダム達が話していました
\end{itemize}

 奉納\さらば弁護士鉄道・泥棒神社の物語(@hirono\_hideki)のアカウントでも、本当はプロフィールに、ツイートの数が多いので必要に応じてミューとしてください、などと書いておきたいぐらいなのですが、不思議なほどフォロワーの数に変動がないので、自然に任せた状態にしています。フォロー返しもしなくなってけっこう経ちます。

 時刻は14時38分です。予定では朝8時から小木港東一文字堤防で魚釣りだったのですが、夕方に宇出津新港の買い物から戻ったタイミングで予定変更の電話がありました。朝の8時過ぎに電話があって決まっていたのです。それで昨夜は早く寝るつもりだったのですが、それも予定が変わりました。

 ブロックされている過食弁護士については、まだ取り上げておきたいことがあって、初めてすぐに中断し脱線して行ったのですが、けっこう共通した問題点もあって、参考にしていただきたい資料です。いったん区切りをつけて、続けていきます。

\begin{itemize}
\tightlist
\item
  〈〈〈 2021/06/09 14:43:31 Linux Emacs: 〈〈〈
\end{itemize}

\hypertarget{ux591aux767aux3059ux308bux6027ux72afux7f6aux793aux8ac7ux306eux76f8ux5834ux306fux3044ux304fux3089-ux5f01ux8b77ux58ebux304cux89e3ux8aacux3059ux308bux30c9ux30adux30e5ux30e1ux30f3ux30c8-ux793aux8ac7ux306eux73feux5834ux3068ux3044ux3046ux7530ux7551ux6df3ux5f01ux8b77ux58ebux306eux6587ux6625ux30aaux30f3ux30e9ux30a4ux30f3ux306eux8a18ux4e8b}{%
\paragraph{「多発する性犯罪・・・・・・``示談の相場''はいくら? 弁護士が解説する「ドキュメント 示談の現場」」という田畑淳弁護士の文春オンラインの記事}\label{ux591aux767aux3059ux308bux6027ux72afux7f6aux793aux8ac7ux306eux76f8ux5834ux306fux3044ux304fux3089-ux5f01ux8b77ux58ebux304cux89e3ux8aacux3059ux308bux30c9ux30adux30e5ux30e1ux30f3ux30c8-ux793aux8ac7ux306eux73feux5834ux3068ux3044ux3046ux7530ux7551ux6df3ux5f01ux8b77ux58ebux306eux6587ux6625ux30aaux30f3ux30e9ux30a4ux30f3ux306eux8a18ux4e8b}}

\begin{itemize}
\tightlist
\item
  〉〉〉 Linux Emacs: 2021/06/09 14:51:42 〉〉〉
\end{itemize}

:CATEGORIES: @kanazawabengosi \#金沢弁護士会 @JFBAsns
日本弁護士連合会(日弁連) \#法務省 @MOJ\_HOUMU \#深澤諭史弁護士 \#示談
\#性犯罪

 現在は過食BとなっているTwitterアカウント、以前は過食弁護士となっていたと思いますが、ユーザ名が変更になっていたことも昨日辺りに調べて確認しています。

 わかりやすく過食弁護士としておきますが、アカウントをみた早い段階で、プロフィールだったのかツイートだったのかはっきり記憶にないですが、明らかに実名を特定できる情報がありました。深澤諭史弁護士らとの共著で、ただ一人実名ではないTwitterアカウントなどと記されていたことです。

〉〉〉 kk\_hironoのリツイート 〉〉〉

\begin{itemize}
\tightlist
\item
  RT
  kk\_hirono(刑事告発・非常上告_金沢地方検察庁御中)|hirono\_hideki(奉納\さらば弁護士鉄道・泥棒神社の物語)
  日時:2021-06-09 14:59/2021/06/08 09:56 URL:
  \url{https://twitter.com/kk\_hirono/status/1402505626572320772} 
  \url{https://twitter.com/hirono\_hideki/status/1402066960221802529} 
  \textgreater{}
  溝の口駅の弁護士・法律相談|神奈川県川崎市高津区の溝の口法律事務所
  \url{https://t.co/QSjVBRSGjK} 
\end{itemize}

〉〉〉 kk\_hironoのリツイート 〉〉〉

\begin{itemize}
\tightlist
\item
  RT
  kk\_hirono(刑事告発・非常上告_金沢地方検察庁御中)|hirono\_hideki(奉納\さらば弁護士鉄道・泥棒神社の物語)
  日時:2021-06-09 14:59/2021/06/08 10:08 URL:
  \url{https://twitter.com/kk\_hirono/status/1402505671526940676} 
  \url{https://twitter.com/hirono\_hideki/status/1402069874097082368} 
  \textgreater{} 田畑 淳 プロフィール \textbar{} 文春オンライン
  \url{https://t.co/5ZkkAaGSyb} 
\end{itemize}

〉〉〉 kk\_hironoのリツイート 〉〉〉

\begin{itemize}
\tightlist
\item
  RT
  kk\_hirono(刑事告発・非常上告_金沢地方検察庁御中)|hirono\_hideki(奉納\さらば弁護士鉄道・泥棒神社の物語)
  日時:2021-06-09 14:59/2021/06/08 10:08 URL:
  \url{https://twitter.com/kk\_hirono/status/1402505712039698434} 
  \url{https://twitter.com/hirono\_hideki/status/1402069978119954435} 
  \textgreater{}
  多発する性犯罪・・・・・・``示談の相場''はいくら? 弁護士が解説する「ドキュメント 示談の現場」
  \textbar{} 文春オンライン \url{https://t.co/pPjxP8L4q3} 
\end{itemize}

〉〉〉 kk\_hironoのリツイート 〉〉〉

\begin{itemize}
\tightlist
\item
  RT
  kk\_hirono(刑事告発・非常上告_金沢地方検察庁御中)|hirono\_hideki(奉納\さらば弁護士鉄道・泥棒神社の物語)
  日時:2021-06-09 15:01/2021/06/08 11:30 URL:
  \url{https://twitter.com/kk\_hirono/status/1402506014298054660} 
  \url{https://twitter.com/hirono\_hideki/status/1402090616410959875} 
  \textgreater{} 2021-06-08\_11:29
  奉納\\#危険生物・弁護士脳汚染除去装置\\#金沢地方検察庁御中\_2020:
  「@motaberarenaiyo」を@hirono\_hideki @kk\_hirono @s\_hironoで検索 27件の該当 2021-06-08\_11:28の記録
  \url{https://t.co/6WZ3zwS3Df} 
\end{itemize}

〉〉〉 kk\_hironoのリツイート 〉〉〉

\begin{itemize}
\tightlist
\item
  RT
  kk\_hirono(刑事告発・非常上告_金沢地方検察庁御中)|hirono\_hideki(奉納\さらば弁護士鉄道・泥棒神社の物語)
  日時:2021-06-09 15:01/2021/06/08 11:30 URL:
  \url{https://twitter.com/kk\_hirono/status/1402506044849344514} 
  \url{https://twitter.com/hirono\_hideki/status/1402090642893721600} 
  \textgreater{} 2021-06-08\_11:29
  奉納\\#危険生物・弁護士脳汚染除去装置\\#金沢地方検察庁御中\_2020:
  「過食」を@hirono\_hideki @kk\_hirono @s\_hironoで検索 419件の該当 2021-06-08\_11:29の記録
  \url{https://t.co/n0qr2CYuAU} 
\end{itemize}

〉〉〉 kk\_hironoのリツイート 〉〉〉

\begin{itemize}
\tightlist
\item
  RT
  kk\_hirono(刑事告発・非常上告_金沢地方検察庁御中)|hirono\_hideki(奉納\さらば弁護士鉄道・泥棒神社の物語)
  日時:2021-06-09 15:01/2021/06/08 12:18 URL:
  \url{https://twitter.com/kk\_hirono/status/1402506129326804992} 
  \url{https://twitter.com/hirono\_hideki/status/1402102675617968142} 
  \textgreater{} 2021-06-08\_11:37
  奉納\\#危険生物・弁護士脳汚染除去装置\\#金沢地方検察庁御中\_2020:
  %@motaberarenaiyo 過食B%\url{https://t.co/XwJuuciKD3} 
  \url{https://t.co/L7BsNfI9mL} 
\end{itemize}

〉〉〉 kk\_hironoのリツイート 〉〉〉

\begin{itemize}
\tightlist
\item
  RT
  kk\_hirono(刑事告発・非常上告_金沢地方検察庁御中)|hirono\_hideki(奉納\さらば弁護士鉄道・泥棒神社の物語)
  日時:2021-06-09 15:01/2021/06/08 12:18 URL:
  \url{https://twitter.com/kk\_hirono/status/1402506184318361611} 
  \url{https://twitter.com/hirono\_hideki/status/1402102702117580821} 
  \textgreater{} 2021-06-08\_11:37
  奉納\\#危険生物・弁護士脳汚染除去装置\\#金沢地方検察庁御中\_2020:
  REGEXP:''@juntaba1''/データベース登録済みツイートの検索:2012-07-10〜2021-03-26/2021年06月08日11時33分の記録:ユーザ・投稿:122/259件
  \url{https://t.co/LPd9yZQiAo} 
\end{itemize}

〉〉〉 kk\_hironoのリツイート 〉〉〉

\begin{itemize}
\tightlist
\item
  RT
  kk\_hirono(刑事告発・非常上告_金沢地方検察庁御中)|hirono\_hideki(奉納\さらば弁護士鉄道・泥棒神社の物語)
  日時:2021-06-09 15:02/2021/06/08 12:18 URL:
  \url{https://twitter.com/kk\_hirono/status/1402506264425291776} 
  \url{https://twitter.com/hirono\_hideki/status/1402102728575254532} 
  \textgreater{} 2021-06-08\_11:39
  奉納\\#危険生物・弁護士脳汚染除去装置\\#金沢地方検察庁御中\_2020:
  %@fukazawas 深澤諭史%(・∀・)(・∀・)(・∀・)¥\n\#私のことどう思ってるか引用RTで正直に容赦なく言ってみてほしい
  \url{https://t.co/ptghZR0ar2} 
\end{itemize}

〉〉〉 kk\_hironoのリツイート 〉〉〉

\begin{itemize}
\tightlist
\item
  RT
  kk\_hirono(刑事告発・非常上告_金沢地方検察庁御中)|hirono\_hideki(奉納\さらば弁護士鉄道・泥棒神社の物語)
  日時:2021-06-09 15:02/2021/06/08 12:18 URL:
  \url{https://twitter.com/kk\_hirono/status/1402506337754370053} 
  \url{https://twitter.com/hirono\_hideki/status/1402102755104223263} 
  \textgreater{} 2021-06-08\_12:02
  奉納\\#危険生物・弁護士脳汚染除去装置\\#金沢地方検察庁御中\_2020:
  %@motaberarenaiyo 過食B%検察幹部って、鬱憤晴らすために嘘ばらまくTwitter界隈の馬鹿な素人と同じレベルで務まるのか?
  \url{https://t.co/WwV2KvbSHe} 
\end{itemize}

〉〉〉 kk\_hironoのリツイート 〉〉〉

\begin{itemize}
\item
  RT
  kk\_hirono(刑事告発・非常上告_金沢地方検察庁御中)|hirono\_hideki(奉納\さらば弁護士鉄道・泥棒神社の物語)
  日時:2021-06-09 15:02/2021/06/08 12:18 URL:
  \url{https://twitter.com/kk\_hirono/status/1402506427478921216} 
  \url{https://twitter.com/hirono\_hideki/status/1402102808036270081} 
  \textgreater{} 2021-06-08\_12:08
  奉納\\#危険生物・弁護士脳汚染除去装置\\#金沢地方検察庁御中\_2020:
  %@motaberarenaiyo 過食B%弁護側の戦術はどうか、検察の進め方はどうかと検討する必要があります。¥\n無罪って結構な確率で検察の敵失に乗じて獲得されているのです。
  \url{https://t.co/FO9NKNWuwK} 
\item
  奉納\さらば弁護士鉄道・泥棒神社の物語(@hirono\_hideki)/2021年06月08日
  - Twilog \url{https://t.co/ncCVkJU3pv} 
\end{itemize}

 一日経つと流れが掴みづらく、作業を進める効率が著しく低下します。スクリーンショットの記録もいくつか作ってるので、そちらの方が視覚的にわかりやすい面があると思います。

〉〉〉 kk\_hironoのリツイート 〉〉〉

\begin{itemize}
\tightlist
\item
  RT
  kk\_hirono(刑事告発・非常上告_金沢地方検察庁御中)|s\_hirono(非常上告-最高検察庁御中\_ツイッター)
  日時:2021-06-09 15:12/2021/06/08 11:25 URL:
  \url{https://twitter.com/kk\_hirono/status/1402508936050208768} 
  \url{https://twitter.com/s\_hirono/status/1402089434258890755} 
  \textgreater{}
  2021-06-08-100941\_実際の示談については当然どの事件も守秘義務の対象となりますので、ここでは架空の事例を一つ挙げてみましょう。.jpg
  \url{https://t.co/S6YoRM7iUb} 
\end{itemize}

〉〉〉 kk\_hironoのリツイート 〉〉〉

\begin{itemize}
\tightlist
\item
  RT
  kk\_hirono(刑事告発・非常上告_金沢地方検察庁御中)|s\_hirono(非常上告-最高検察庁御中\_ツイッター)
  日時:2021-06-09 15:13/2021/06/08 11:26 URL:
  \url{https://twitter.com/kk\_hirono/status/1402509065536688128} 
  \url{https://twitter.com/s\_hirono/status/1402089506275160071} 
  \textgreater{}
  2021-06-08-102742\_加害者を罰するためには検察官が奔走しますが、被害者を救い、平穏な生活を過ごすための助力は、まだまだこの社会に欠けているように感じます。.jpg
  \url{https://t.co/GwjKO3Aqst} 
\end{itemize}

〉〉〉 kk\_hironoのリツイート 〉〉〉

\begin{itemize}
\tightlist
\item
  RT
  kk\_hirono(刑事告発・非常上告_金沢地方検察庁御中)|s\_hirono(非常上告-最高検察庁御中\_ツイッター)
  日時:2021-06-09 15:13/2021/06/08 11:26 URL:
  \url{https://twitter.com/kk\_hirono/status/1402509108566102019} 
  \url{https://twitter.com/s\_hirono/status/1402089579029549057} 
  \textgreater{}
  2021-06-08-111424\_過食B@motaberarenaiyo·6月7日日本の検察のモノの考え方は、現在の日本の法制度からの唯一の帰結なんでしょうか。無意味な情熱、.jpg
  \url{https://t.co/UaOAZaivVp} 
\end{itemize}

〉〉〉 kk\_hironoのリツイート 〉〉〉

\begin{itemize}
\tightlist
\item
  RT
  kk\_hirono(刑事告発・非常上告_金沢地方検察庁御中)|s\_hirono(非常上告-最高検察庁御中\_ツイッター)
  日時:2021-06-09 15:13/2021/06/08 12:13 URL:
  \url{https://twitter.com/kk\_hirono/status/1402509218272272387} 
  \url{https://twitter.com/s\_hirono/status/1402101461316562944} 
  \textgreater{}
  2021-06-08-113454\_過食B@motaberarenaiyoブロックされています@motaberarenaiyoさんのフォローやツイートの表示はできません。詳細は.jpg
  \url{https://t.co/LE4cMJDZJQ} 
\end{itemize}

〉〉〉 kk\_hironoのリツイート 〉〉〉

\begin{itemize}
\tightlist
\item
  RT
  kk\_hirono(刑事告発・非常上告_金沢地方検察庁御中)|s\_hirono(非常上告-最高検察庁御中\_ツイッター)
  日時:2021-06-09 15:13/2021/06/08 12:13 URL:
  \url{https://twitter.com/kk\_hirono/status/1402509257921110018} 
  \url{https://twitter.com/s\_hirono/status/1402101533441880064} 
  \textgreater{}
  2021-06-08-113522\_過食B@motaberarenaiyoブロックされています@motaberarenaiyoさんのフォローやツイートの表示はできません。詳細は.jpg
  \url{https://t.co/VMrwOUCxRu} 
\end{itemize}

〉〉〉 kk\_hironoのリツイート 〉〉〉

\begin{itemize}
\tightlist
\item
  RT
  kk\_hirono(刑事告発・非常上告_金沢地方検察庁御中)|s\_hirono(非常上告-最高検察庁御中\_ツイッター)
  日時:2021-06-09 15:14/2021/06/08 12:14 URL:
  \url{https://twitter.com/kk\_hirono/status/1402509310379192320} 
  \url{https://twitter.com/s\_hirono/status/1402101605512605696} 
  \textgreater{}
  2021-06-08-113651\_過食B@motaberarenaiyo·2017年1月31日返信先: @fukazawasさん共著者ナカーマ.jpg
  \url{https://t.co/d6Nd4oLPGN} 
\end{itemize}

〉〉〉 kk\_hironoのリツイート 〉〉〉

\begin{itemize}
\tightlist
\item
  RT
  kk\_hirono(刑事告発・非常上告_金沢地方検察庁御中)|s\_hirono(非常上告-最高検察庁御中\_ツイッター)
  日時:2021-06-09 15:14/2021/06/08 12:14 URL:
  \url{https://twitter.com/kk\_hirono/status/1402509355430277121} 
  \url{https://twitter.com/s\_hirono/status/1402101677491056649} 
  \textgreater{}
  2021-06-08-113821\_過食B@motaberarenaiyo返信先: @fukazawasさん共著者ナカーマ.jpg
  \url{https://t.co/xx4Ru42Osq} 
\end{itemize}

〉〉〉 kk\_hironoのリツイート 〉〉〉

\begin{itemize}
\tightlist
\item
  RT
  kk\_hirono(刑事告発・非常上告_金沢地方検察庁御中)|s\_hirono(非常上告-最高検察庁御中\_ツイッター)
  日時:2021-06-09 15:14/2021/06/08 12:14 URL:
  \url{https://twitter.com/kk\_hirono/status/1402509385809555463} 
  \url{https://twitter.com/s\_hirono/status/1402101749578559491} 
  \textgreater{}
  2021-06-08-113845\_深澤諭史@fukazawas(・∀・)(・∀・)(・∀・)#私のことどう思ってるか引用RTで正直に容赦なく言ってみてほしい.jpg
  \url{https://t.co/1v68KlJXzb} 
\end{itemize}

〉〉〉 kk\_hironoのリツイート 〉〉〉

\begin{itemize}
\tightlist
\item
  RT
  kk\_hirono(刑事告発・非常上告_金沢地方検察庁御中)|s\_hirono(非常上告-最高検察庁御中\_ツイッター)
  日時:2021-06-09 15:14/2021/06/08 12:14 URL:
  \url{https://twitter.com/kk\_hirono/status/1402509417652756489} 
  \url{https://twitter.com/s\_hirono/status/1402101821930229762} 
  \textgreater{}
  2021-06-08-113949\_%@fukazawas 深澤諭史%(・∀・)(・∀・)(・∀・)#私のことどう思ってるか引用RTで正直に容赦なく言ってみてほしい.jpg
  \url{https://t.co/c01HP2BYWT} 
\end{itemize}

〉〉〉 kk\_hironoのリツイート 〉〉〉

\begin{itemize}
\tightlist
\item
  RT
  kk\_hirono(刑事告発・非常上告_金沢地方検察庁御中)|s\_hirono(非常上告-最高検察庁御中\_ツイッター)
  日時:2021-06-09 15:14/2021/06/08 12:15 URL:
  \url{https://twitter.com/kk\_hirono/status/1402509445838508038} 
  \url{https://twitter.com/s\_hirono/status/1402101893858414592} 
  \textgreater{}
  2021-06-08-114000\_%@fukazawas 深澤諭史%(・∀・)(・∀・)(・∀・)#私のことどう思ってるか引用RTで正直に容赦なく言ってみてほしい.jpg
  \url{https://t.co/kvV9rd0alm} 
\end{itemize}

〉〉〉 kk\_hironoのリツイート 〉〉〉

\begin{itemize}
\tightlist
\item
  RT
  kk\_hirono(刑事告発・非常上告_金沢地方検察庁御中)|s\_hirono(非常上告-最高検察庁御中\_ツイッター)
  日時:2021-06-09 15:14/2021/06/08 12:15 URL:
  \url{https://twitter.com/kk\_hirono/status/1402509499190046721} 
  \url{https://twitter.com/s\_hirono/status/1402101965937545216} 
  \textgreater{}
  2021-06-08-114035\_%@fukazawas 深澤諭史%(・∀・)(・∀・)(・∀・)#私のことどう思ってるか引用RTで正直に容赦なく言ってみてほしい.jpg
  \url{https://t.co/TcjcsyZLOZ} 
\end{itemize}

〉〉〉 kk\_hironoのリツイート 〉〉〉

\begin{itemize}
\tightlist
\item
  RT
  kk\_hirono(刑事告発・非常上告_金沢地方検察庁御中)|s\_hirono(非常上告-最高検察庁御中\_ツイッター)
  日時:2021-06-09 15:15/2021/06/08 12:15 URL:
  \url{https://twitter.com/kk\_hirono/status/1402509542433476611} 
  \url{https://twitter.com/s\_hirono/status/1402102038004068352} 
  \textgreater{}
  2021-06-08-114559\_過食B@motaberarenaiyo·2019年7月6日「女性側が不貞して、DVをデッチ上げて離婚しようとする」「不倫がバレたのでレイプさ.jpg
  \url{https://t.co/aepVKW7KDm} 
\end{itemize}

〉〉〉 kk\_hironoのリツイート 〉〉〉

\begin{itemize}
\tightlist
\item
  RT
  kk\_hirono(刑事告発・非常上告_金沢地方検察庁御中)|s\_hirono(非常上告-最高検察庁御中\_ツイッター)
  日時:2021-06-09 15:15/2021/06/08 12:16 URL:
  \url{https://twitter.com/kk\_hirono/status/1402509598074949634} 
  \url{https://twitter.com/s\_hirono/status/1402102110150295553} 
  \textgreater{}
  2021-06-08-115730\_過食B@motaberarenaiyo·2月24日私は年間100件以上無料相談をうけていますが実際に頭おかしいです(笑).jpg
  \url{https://t.co/mC7XHrtfix} 
\end{itemize}

〉〉〉 kk\_hironoのリツイート 〉〉〉

\begin{itemize}
\item
  RT
  kk\_hirono(刑事告発・非常上告_金沢地方検察庁御中)|s\_hirono(非常上告-最高検察庁御中\_ツイッター)
  日時:2021-06-09 15:15/2021/06/08 12:16 URL:
  \url{https://twitter.com/kk\_hirono/status/1402509665997520896} 
  \url{https://twitter.com/s\_hirono/status/1402102182334263300} 
  \textgreater{}
  2021-06-08-120240\_この「検察幹部」は弁護人に一体何を求めてるんだ?検察幹部って、鬱憤晴らすために嘘ばらまくTwitter界隈の馬鹿な素人と同じレベルで務まるの.jpg
  \url{https://t.co/lMwpRDLsuk} 
\item
  非常上告-最高検察庁御中\_ツイッター(@s\_hirono)/2021年06月08日 -
  Twilog \url{https://t.co/ttS9ThS9Lr} 
\end{itemize}

 他にもあるのですが、ブログの記事1つあたりの埋め込みツイートは、これぐらいでやめておこうと思います。ページの読み込みに支障が出る可能性があるからです。

 「溝の口駅の弁護士・法律相談|神奈川県川崎市高津区の溝の口法律事務所」となっていましたが、ページにある地図を拡大すると、ほとんど聞いたことのない駅がずいぶん大きな駅に見えました。さらに地図を拡大すると、武蔵小杉の神原元弁護士の法律事務所の表示も出てきました。東京都内に近いこともわかりました。

 この田畑淳弁護士の法律事務所について調べたのは初めてではなかったと思いますが、文春オンラインの記事の方は、初めてみたように思いました。まるで探偵物の小説を読む気分でしたが、こういう世界もあるのだと視野が開かれました。

 性犯罪の被害者に対する救済という内容にもなっていたと思いますが、昨日の15時ごろに開いた記事だとスクリーンショットでわかりました。

\begin{quote}
《引用の始まり》
\end{quote}

\begin{quote}
11月20日、「ミスター慶応」ファイナリストの元慶大生が20代女性に性的暴行を加えて強制性交の容疑で逮捕された。この事件では、容疑者が過去5回も準強制性交や準強制わいせつなどの疑いで逮捕されながらも、被害者との間で示談が成立したため不起訴となっていたことに対して、世間からは「大金を握らせて被害者を黙らせたのではないか」「無理やり示談にさせたのではないか」といった声が湧きあがった。

 性犯罪における示談の現場はどういうものなのか。現役の弁護士が解説する。
\end{quote}

\begin{quote}
《引用の終わり》
\end{quote}

\begin{itemize}
\tightlist
\item
  多発する性犯罪・・・・・・``示談の相場''はいくら? 弁護士が解説する「ドキュメント 示談の現場」
  \textbar{} 文春オンライン \url{https://bunshun.jp/articles/-/42079} 
\end{itemize}

 記事は「11月20日、「ミスター慶応」ファイナリストの元慶大生が20代女性に性的暴行を加えて強制性交の容疑で逮捕された。この事件では、」という出だしで始まっていました。令和3年3月31日付告発状でも少しは取り上げていると思いますが、ちょうど小川賢司裁判官に注目した時期で焦点を当てていたと思います。

 「刑事事件の示談書は、複雑な事件でなければ1ページ程度の短い内容であることが多い。」というのは具体的でわかりやすく、初めて知る弁護士の仕事ぶりでした。

 なるほどと視野が開けた思いでしたが、このときは参考程度だったと思います。

 「とすれば、「連絡を受けるかどうか」「いくらで示談を受け入れるか」は完全に被害者の手に委ねられており、被害者は強い立場でいくらでも条件を吊り上げられそうにも一見、見えます。」ともあります。

 「これは事実ですが、この事実が被害者を追い詰めることもあります。すなわち、自分が示談を拒絶したために逆恨みされることはないか、と言った不安が被害者にはあります。」という部分は弁護士側の都合という立場でこじつけがあるようにも感じました。

 「全く事件と関係のない第三者が「示談で事件をねじ伏せた」と論評するとき、そこには被害者の意思はありません。ましてや示談金の金額について是非を論じるのは、被害者として最も知られたくない内容の一つでしょう。」というのは、弁護士の仕事の本質に迫るものを感じました。

 「性犯罪について考えるとき、「加害者を罰する」ことより「被害者の力になること」を主眼に置いてみる必要があると私は思います。加害者を罰するためには検察官が奔走しますが、被害者を救い、平穏な生活を過ごすための助力は、まだまだこの社会に欠けているように感じます。」と田畑淳弁護士の文春オンラインの記事は締めくくられています。

 4ページに別れた文春オンラインの記事ですが、最も引っかかりを強く感じたともいえる1ページ目に戻ります。ずっともやもやが残る部分です。

 「一口に性犯罪と言われるものにも多様な側面があります。実際の示談については当然どの事件も守秘義務の対象となりますので、ここでは架空の事例を一つ挙げてみましょう。」という本題に入る直前の断り書きのような部分です。

 深澤諭史弁護士の場合は、ブログの記事やそのあとのnoteで、特定できないように修正するという趣旨のことを当然のように話していましたが、それだと何でもありで、作り話と区別ができないと思いました。同じ口で、警察や検察の調書、作文と罵倒してきたのも弁護士鉄道の歴史であり、伝統芸です。

 年間100件の無料法律相談をしているというツイートもありましたが、この無料法律相談のありようというのも深澤諭史弁護士に似たものを感じました。

 性犯罪の示談については、これまでも弁護士らの話を読んできましたが、示談で賠償がもらえる機会は弁護士がついている間だけ、刑事裁判が終われば無関係というのが基本的スタイルといったところです。

 次に取り上げたいのは、夜に読んだ盗撮で300万円という示談金のツイートでした。強姦でも300万円の示談金というのは高額な部類に入るレアなケースと聞いたように思います。

\begin{itemize}
\tightlist
\item
  〈〈〈 2021/06/09 15:56:04 Linux Emacs: 〈〈〈
\end{itemize}

\hypertarget{ux5f37ux5236ux308fux3044ux305bux3064ux3067450ux4e07ux5186ux76d7ux64aeux3067300ux4e07ux5186ux306eux793aux8ac7ux91d1ux3092ux3082ux3089ux3063ux3066ux3044ux307eux3059ux3068ux3044ux3046ux30adux30e3ux30d0ux5b22ux306eux30c4ux30a4ux30fcux30c8ux3068ux305dux308cux306bux5bfeux3059ux308bux5f01ux8b77ux58ebtwitterux30a2ux30abux30a6ux30f3ux30c8ux306eux53cdux5fdc}{%
\paragraph{強制わいせつで450万円、盗撮で300万円の示談金をもらっています、というキャバ嬢のツイートと、それに対する弁護士Twitterアカウントの反応}\label{ux5f37ux5236ux308fux3044ux305bux3064ux3067450ux4e07ux5186ux76d7ux64aeux3067300ux4e07ux5186ux306eux793aux8ac7ux91d1ux3092ux3082ux3089ux3063ux3066ux3044ux307eux3059ux3068ux3044ux3046ux30adux30e3ux30d0ux5b22ux306eux30c4ux30a4ux30fcux30c8ux3068ux305dux308cux306bux5bfeux3059ux308bux5f01ux8b77ux58ebtwitterux30a2ux30abux30a6ux30f3ux30c8ux306eux53cdux5fdc}}

\begin{itemize}
\tightlist
\item
  〉〉〉 Linux Emacs: 2021/06/09 17:26:51 〉〉〉
\end{itemize}

:CATEGORIES: @kanazawabengosi \#金沢弁護士会 @JFBAsns
日本弁護士連合会(日弁連) \#法務省 @MOJ\_HOUMU \#示談

〉〉〉 kk\_hironoのリツイート 〉〉〉

\begin{itemize}
\tightlist
\item
  RT
  kk\_hirono(刑事告発・非常上告_金沢地方検察庁御中)|jmjhjmwtad(7286)
  日時:2021-06-09 17:27/2021/06/05 17:16 URL:
  \url{https://twitter.com/kk\_hirono/status/1402542902098632710} 
  \url{https://twitter.com/jmjhjmwtad/status/1401090558169260035} 
  \textgreater{}
  こういうの真に受けて本当に交渉しようとすると、弁護人としては、電話の録音の反訳と、報告書を検察官に上げて供託して終わり。こういうネット情報を真に受けてるなって、弁護人も検察官も分かるんだよなあ。
  \url{https://t.co/6Wccd3tZzi} 
\end{itemize}

〉〉〉 kk\_hironoのリツイート 〉〉〉

\begin{itemize}
\tightlist
\item
  RT
  kk\_hirono(刑事告発・非常上告_金沢地方検察庁御中)|harrier0516osk(向原総合法律事務所 弁護士向原)
  日時:2021-06-09 17:27/2021/06/05 18:49 URL:
  \url{https://twitter.com/kk\_hirono/status/1402542951100600321} 
  \url{https://twitter.com/harrier0516osk/status/1401113945352511490} 
  \textgreater{} @jmjhjmwtad ・この金額が示唆されたこと
  ・当方は50万円(これが当該事案での示談金として一般的に必要十分であることが前提)での示談を提案したこと
  ・それで応じてもらえなかったこと を報告書に記載して終わりですね。
  それで不起訴等になってから後で「もっとよこせ」と言われても「もう役目終わりました」
\end{itemize}

〉〉〉 kk\_hironoのリツイート 〉〉〉

\begin{itemize}
\tightlist
\item
  RT
  kk\_hirono(刑事告発・非常上告_金沢地方検察庁御中)|perorin2018(キャバ嬢さくら🌸キャバクラ求人🌸)
  日時:2021-06-09 17:27/2021/06/05 20:05 URL:
  \url{https://twitter.com/kk\_hirono/status/1402542976425881608} 
  \url{https://twitter.com/perorin2018/status/1401133164345499650} 
  \textgreater{} @harrier0516osk @jmjhjmwtad 無能弁護士😊
  世の中、無能弁護士であふれて被害者側も迷惑してますので勘弁してください。
\end{itemize}

〉〉〉 kk\_hironoのリツイート 〉〉〉

\begin{itemize}
\tightlist
\item
  RT
  kk\_hirono(刑事告発・非常上告_金沢地方検察庁御中)|perorin2018(キャバ嬢さくら🌸キャバクラ求人🌸)
  日時:2021-06-09 17:28/2021/06/04 14:39 URL:
  \url{https://twitter.com/kk\_hirono/status/1402543012882702344} 
  \url{https://twitter.com/perorin2018/status/1400688538560065539} 
  \textgreater{} 6月3日 弁護士とのやり取り。弁護士の声がうわずっていた
  \url{https://t.co/Adh0mPIbYr} 
\end{itemize}

〉〉〉 kk\_hironoのリツイート 〉〉〉

\begin{itemize}
\tightlist
\item
  RT
  kk\_hirono(刑事告発・非常上告_金沢地方検察庁御中)|nUO902QkUUp6xT4(愛でたいインコ)
  日時:2021-06-09 17:29/2021/06/04 17:45 URL:
  \url{https://twitter.com/kk\_hirono/status/1402543249626001415} 
  \url{https://twitter.com/nUO902QkUUp6xT4/status/1400735419650625536} 
  \textgreater{} @perorin2018
  あと女性の性被害にとても強い弁護士の先生で岸本学さんという方がおられますツイッターされてますので良ければ探して見てください
\end{itemize}

〉〉〉 kk\_hironoのリツイート 〉〉〉

\begin{itemize}
\tightlist
\item
  RT
  kk\_hirono(刑事告発・非常上告_金沢地方検察庁御中)|perorin2018(キャバ嬢さくら🌸キャバクラ求人🌸)
  日時:2021-06-09 17:29/2021/06/04 17:51 URL:
  \url{https://twitter.com/kk\_hirono/status/1402543311840104455} 
  \url{https://twitter.com/perorin2018/status/1400737069807271938} 
  \textgreater{} @nUO902QkUUp6xT4 ありがとうございます
  岸本先生、3年くらい前にお世話になってますが、頭のいい先生ですよね!
  高額を請求するためには弁護士は使わないほうがいいのと自分でできるので私は自力でやる方針です
  でもありがとうございます💕
\end{itemize}

〉〉〉 kk\_hironoのリツイート 〉〉〉

\begin{itemize}
\tightlist
\item
  RT
  kk\_hirono(刑事告発・非常上告_金沢地方検察庁御中)|nUO902QkUUp6xT4(愛でたいインコ)
  日時:2021-06-09 17:29/2021/06/04 17:54 URL:
  \url{https://twitter.com/kk\_hirono/status/1402543377686548484} 
  \url{https://twitter.com/nUO902QkUUp6xT4/status/1400737665830375429} 
  \textgreater{} @perorin2018
  ご存知でしたか!失礼致しました!さくらさんとても頭の切れる方なので大丈夫ですね!!!トラウマになって二度と同じ間違い犯さないよう徹底的に懲らしめてやってください!!!
\end{itemize}

〉〉〉 kk\_hironoのリツイート 〉〉〉

\begin{itemize}
\tightlist
\item
  RT
  kk\_hirono(刑事告発・非常上告_金沢地方検察庁御中)|perorin2018(キャバ嬢さくら🌸キャバクラ求人🌸)
  日時:2021-06-09 17:29/2021/06/04 18:12 URL:
  \url{https://twitter.com/kk\_hirono/status/1402543416739725318} 
  \url{https://twitter.com/perorin2018/status/1400742273722163209} 
  \textgreater{} @nUO902QkUUp6xT4
  いえいえ!見ているフォロワーさんの参考にもなると思うので、名前を出していただきありがとうございます✨😊
  犯人側の弁護士は無限にいるのに、被害者側の弁護士ってなかなかいないんですよね💦💦
\end{itemize}

〉〉〉 kk\_hironoのリツイート 〉〉〉

\begin{itemize}
\tightlist
\item
  RT
  kk\_hirono(刑事告発・非常上告_金沢地方検察庁御中)|perorin2018(キャバ嬢さくら🌸キャバクラ求人🌸)
  日時:2021-06-09 17:29/2021/06/05 23:27 URL:
  \url{https://twitter.com/kk\_hirono/status/1402543472855289856} 
  \url{https://twitter.com/perorin2018/status/1401183852660359173} 
  \textgreater{} @QxPiWkXs3XCPgZu
  そういうのもあるのですね!勉強になります。ありがとうございます。
  いくらで私に口をつぐませるかのオークションなのと金額は謝罪のお気持ちととらえています😊
\end{itemize}

〉〉〉 kk\_hironoのリツイート 〉〉〉

\begin{itemize}
\tightlist
\item
  RT
  kk\_hirono(刑事告発・非常上告_金沢地方検察庁御中)|sk84545584(sk)
  日時:2021-06-09 17:30/2021/06/04 17:36 URL:
  \url{https://twitter.com/kk\_hirono/status/1402543531000930309} 
  \url{https://twitter.com/sk84545584/status/1400733236028854280} 
  \textgreater{} @perorin2018
  私昨年、電車内でスカート内、盗撮され、声かけて駅長室きて欲しいといい、了承受けた後、電車のドアが開いたら物凄いスピードで逃げられ、追いかけましたが間に合わず・・・駅員は殆ど無視され、
  警察行ってもだめでした 結局嫌な思いしただけだったので
  お金もらえるなら貰いたかったです・・・し支払うべき
\end{itemize}

〉〉〉 kk\_hironoのリツイート 〉〉〉

\begin{itemize}
\tightlist
\item
  RT
  kk\_hirono(刑事告発・非常上告_金沢地方検察庁御中)|harrier0516osk(向原総合法律事務所 弁護士向原)
  日時:2021-06-09 17:31/2021/06/05 18:52 URL:
  \url{https://twitter.com/kk\_hirono/status/1402543955003154435} 
  \url{https://twitter.com/harrier0516osk/status/1401114612766887940} 
  \textgreater{} @jmjhjmwtad あとで裁判やろうとしても、
  ・被疑者の住所(送達先)調べる ・訴状作成等の期日対応
  だから大変だろうと思います。上記示談金はそうした煩瑣分をこちらも掛けずに済むからその分を先払いすべく少し乗せた金額で出すようにしてます
\end{itemize}

〉〉〉 kk\_hironoのリツイート 〉〉〉

\begin{itemize}
\tightlist
\item
  RT
  kk\_hirono(刑事告発・非常上告_金沢地方検察庁御中)|harrier0516osk(向原総合法律事務所 弁護士向原)
  日時:2021-06-09 17:31/2021/06/05 18:53 URL:
  \url{https://twitter.com/kk\_hirono/status/1402543964901675015} 
  \url{https://twitter.com/harrier0516osk/status/1401114943722688513} 
  \textgreater{} @jmjhjmwtad
  言い換えたら、弁護人の役割が終わった後の民事訴訟にかかるであろう被疑者のコスト分を、訴訟コストにするくらいなら被害者に渡す、という感覚です
  しかし、それがダメなら、それぞれが費用を負担して訴訟の場で決着するしかないと思います。
\end{itemize}

〉〉〉 kk\_hironoのリツイート 〉〉〉

\begin{itemize}
\tightlist
\item
  RT
  kk\_hirono(刑事告発・非常上告_金沢地方検察庁御中)|jmjhjmwtad(7286)
  日時:2021-06-09 17:32/2021/06/05 19:21 URL:
  \url{https://twitter.com/kk\_hirono/status/1402544013098381317} 
  \url{https://twitter.com/jmjhjmwtad/status/1401122060605136896} 
  \textgreater{} @harrier0516osk
  まぁ淡々と処理するだけですよね。なんか性犯罪の示談について、ネットで変な風潮ができてて、弁護人としては、ある意味やりやすい気がします。
\end{itemize}

〉〉〉 kk\_hironoのリツイート 〉〉〉

\begin{itemize}
\tightlist
\item
  RT
  kk\_hirono(刑事告発・非常上告_金沢地方検察庁御中)|harrier0516osk(向原総合法律事務所 弁護士向原)
  日時:2021-06-09 17:32/2021/06/05 19:44 URL:
  \url{https://twitter.com/kk\_hirono/status/1402544062524055554} 
  \url{https://twitter.com/harrier0516osk/status/1401127788166320135} 
  \textgreater{} @jmjhjmwtad そうですよね。
  他方、Youtube動画とかで、NTR事例で「間男と嫁から莫大な慰謝料を勝ち取った」みたいな事例を見ると、うーん、となってしまいます。気持ちはわかるんですが、法的解決を志向する限りそんなふうにはならないよ、という現実とは乖離を感じるところです。
\end{itemize}

〉〉〉 kk\_hironoのリツイート 〉〉〉

\begin{itemize}
\tightlist
\item
  RT
  kk\_hirono(刑事告発・非常上告_金沢地方検察庁御中)|jmjhjmwtad(7286)
  日時:2021-06-09 17:32/2021/06/05 20:32 URL:
  \url{https://twitter.com/kk\_hirono/status/1402544099115233286} 
  \url{https://twitter.com/jmjhjmwtad/status/1401139778519633922} 
  \textgreater{} @LiarLawyer800 @harrier0516osk
  ですね。あと、普通に話してる内容を勝手に「弁護士をいい負かしてやった」みたいな改変されてるんも、あるあると言いますか・・・
\end{itemize}

 昨夜は他の弁護士のツイートも見えていたようなと思っていたのですが、告発\市場急配センター殺人未遂事件\金沢地方検察庁・石川県警察御中(@kk\_hirono)でログインしていることに気が付きました。ブロックされているアカウントのツイートもあったのかもしれません。

※ @kk\_hironoのアカウントがブロックされ,リツイートに失敗したツイート

\begin{itemize}
\tightlist
\item
  TW LiarLawyer800(うそつきべ んごし。) 日時:2021/06/05 20:22:30
  URL: \url{https://twitter.com/LiarLawyer800/status/1401137349191376896} 
  \textgreater{} @harrier0516osk @jmjhjmwtad
  弁護士からすれば、エアプで誤った知識を流布されるの困りますよねぇ。
\end{itemize}

※ @kk\_hironoのアカウントがブロックされ,リツイートに失敗したツイート

\begin{itemize}
\tightlist
\item
  TW LiarLawyer800(うそつきべ んごし。) 日時:2021/06/05 20:49:24
  URL: \url{https://twitter.com/LiarLawyer800/status/1401144119775219717} 
  \textgreater{} @jmjhjmwtad @harrier0516osk
  元ツイの人が、リプ付けてきてますが・・・\\
  \textgreater{}
  仮に実話だとしても、極めて例外的な事例を持ち出して、さも成功談のように吹聴するのもなぁ・・・と思いますね・・・\\
  \textgreater{}\\
  \textgreater{} (こちとら何件条例違反の弁護として示談まとめてきて
  ると思ってんねん!と言いたい気分ですw)
\end{itemize}

 向原栄大朗弁護士のタイムラインを遡っているところです。昨夜最初に示談関連のツイートを見たのが向原栄大朗弁護士のタイムラインだったように考えています。

〉〉〉 kk\_hironoのリツイート 〉〉〉

\begin{itemize}
\tightlist
\item
  RT
  kk\_hirono(刑事告発・非常上告_金沢地方検察庁御中)|document35(Document35)
  日時:2021-06-09 17:39/2021/06/05 23:23 URL:
  \url{https://twitter.com/kk\_hirono/status/1402545974938013696} 
  \url{https://twitter.com/document35/status/1401182795263725572} 
  \textgreater{}
  すぐ弁護士に金払って依頼すべきなのに、自分でなんとかしようとして事態を悪化させている人もいれば、弁護士に相談したところでどうしようもないことで弁護士巡りをする人もいる。
\end{itemize}

 昨夜、見かけた覚えはなかったのですが、向原栄大朗弁護士のタイムラインでリツイートを見かけたツイートです。深澤諭史弁護士と考えがよく似ています。

 向原栄大朗弁護士のタイムラインですが、「ツイート」だけのタイムラインにしていました。「ツイートと返信」に切り替えました。

〉〉〉 kk\_hironoのリツイート 〉〉〉

\begin{itemize}
\tightlist
\item
  RT
  kk\_hirono(刑事告発・非常上告_金沢地方検察庁御中)|harrier0516osk(向原総合法律事務所 弁護士向原)
  日時:2021-06-09 17:49/2021/06/05 10:38 URL:
  \url{https://twitter.com/kk\_hirono/status/1402548440979050503} 
  \url{https://twitter.com/harrier0516osk/status/1400990461628665858} 
  \textgreater{}
  鹿児島にいたせいかもしれないんですけど、ホント警察ってのはすごく強大で、カーストの頂点って感じなんですよね(仕事でもよく「相手には警察の親戚がいて・・・」だけで脅威を感じる人もおられる)。
  その原因はやはり「そういう雰囲気づくり」が成功しているから。
\end{itemize}

 昨夜、向原栄大朗弁護士のタイムラインでそこまで遡っていなかったので見ていなかったツイートになりそうです。先程のツイートだけのタイムラインでも見ていて、これは過ぎているように感じるきっかけでした。

※ @kk\_hironoのアカウントがブロックされ,リツイートに失敗したツイート

\begin{itemize}
\tightlist
\item
  TW mizuno\_ryo\_law(福岡の弁護士 水野遼) 日時:2021/06/04 18:56:19
  URL:
  \url{https://twitter.com/mizuno\_ryo\_law/status/1400753271128563712} 
  \textgreater{}
  マスコミは被害者の実名報道の意義云々を言うけれども、事件報道を国家権力のチェックや再発防止の検証でなく、野次馬根性をかきたてるだけのエンターテインメントにしているのは他ならぬマスコミ自身なのだから、物言えば唇寒しと言わざるを得ない。
\end{itemize}

 向原栄大朗弁護士のタイムラインで、向原栄大朗弁護士のリツイートです。ブロックされているアカウントの可能性が高そうに思っていたのですが、当たっていました。

 いくつか向原栄大朗弁護士のタイムラインで坂本正幸弁護士のツイートも見かけたのですが、昨夜は示談関連でも見かけたような気がするのです。スクリーンショットに記録が残っているかもしれません。

 同じ返信欄で見かけていたように思っていたのですが、違っていたのかもしれません。スクリーンショットに「ふて寝べん@hirune\_b」と「gimu13@gimu13」がありました。別の方法で調べてみます。

\begin{itemize}
\tightlist
\item
  〈〈〈 2021/06/09 18:03:33 Linux Emacs: 〈〈〈
\end{itemize}

\hypertarget{ux5f37ux5236ux308fux3044ux305bux3064ux3067450ux4e07ux5186ux76d7ux64aeux3067300ux4e07ux5186ux306eux793aux8ac7ux91d1ux3092ux3082ux3089ux3063ux3066ux3044ux307eux3059ux3068ux3044ux3046ux30adux30e3ux30d0ux5b22ux306eux30c4ux30a4ux30fcux30c8ux3068ux305dux308cux306bux5bfeux3059ux308bux5f01ux8b77ux58ebtwitterux30a2ux30abux30a6ux30f3ux30c8ux306eux53cdux5fdc-gimu13-gimu13}{%
\paragraph{強制わいせつで450万円、盗撮で300万円の示談金をもらっています、というキャバ嬢のツイートと、それに対する弁護士Twitterアカウントの反応 %gimu13 @gimu13%}\label{ux5f37ux5236ux308fux3044ux305bux3064ux3067450ux4e07ux5186ux76d7ux64aeux3067300ux4e07ux5186ux306eux793aux8ac7ux91d1ux3092ux3082ux3089ux3063ux3066ux3044ux307eux3059ux3068ux3044ux3046ux30adux30e3ux30d0ux5b22ux306eux30c4ux30a4ux30fcux30c8ux3068ux305dux308cux306bux5bfeux3059ux308bux5f01ux8b77ux58ebtwitterux30a2ux30abux30a6ux30f3ux30c8ux306eux53cdux5fdc-gimu13-gimu13}}

\begin{itemize}
\tightlist
\item
  〉〉〉 Linux Emacs: 2021/06/09 18:29:02 〉〉〉
\end{itemize}

:CATEGORIES: @kanazawabengosi \#金沢弁護士会 @JFBAsns
日本弁護士連合会(日弁連) \#法務省 @MOJ\_HOUMU \#示談

 「ふて寝べん@hirune\_b」の方は、示談のツイートと関連が確認できませんでした。確認できたのは、ここで取り上げる「gimu13@gimu13」のツイートですが、キャバ嬢のツイートに関連づいていたのかもしれません。

〉〉〉 kk\_hironoのリツイート 〉〉〉

\begin{itemize}
\tightlist
\item
  RT
  kk\_hirono(刑事告発・非常上告_金沢地方検察庁御中)|lawyer\_freenote(法律家の自由帳)
  日時:2021-06-09 18:32/2021/06/05 16:38 URL:
  \url{https://twitter.com/kk\_hirono/status/1402559184155987972} 
  \url{https://twitter.com/lawyer\_freenote/status/1401080867720597510} 
  \textgreater{} @popohito @gimu13
  やったことはないですが、できるみたいですよ。
  当会で、できたという報告が数年前にありました。
\end{itemize}

〉〉〉 kk\_hironoのリツイート 〉〉〉

\begin{itemize}
\tightlist
\item
  RT
  kk\_hirono(刑事告発・非常上告_金沢地方検察庁御中)|nob0117(たけうち
  のぶやす❇️🐈LINE相談はじめました) 日時:2021-06-09 18:32/2021/06/05
  20:46 URL: \url{https://twitter.com/kk\_hirono/status/1402559225251831814} 
  \url{https://twitter.com/nob0117/status/1401143484363403264} 
  \textgreater{} @lawyer\_freenote @popohito @gimu13
  ああ、たぶん私が報告したときのやつですね
  一部弁済という形でさえなければ、「供託者が相当と考えた金額」とかいう付記で可能だったという記憶が
\end{itemize}

〉〉〉 kk\_hironoのリツイート 〉〉〉

\begin{itemize}
\tightlist
\item
  RT
  kk\_hirono(刑事告発・非常上告_金沢地方検察庁御中)|lawyer\_freenote(法律家の自由帳)
  日時:2021-06-09 18:32/2021/06/05 21:01 URL:
  \url{https://twitter.com/kk\_hirono/status/1402559281619030017} 
  \url{https://twitter.com/lawyer\_freenote/status/1401147061458800651} 
  \textgreater{} @nob0117 @popohito @gimu13 そうです。先生のご報告です。
  とても参考になったので覚えています。残念ながら使う機会がなかったのですが(笑)
\end{itemize}

〉〉〉 kk\_hironoのリツイート 〉〉〉

\begin{itemize}
\tightlist
\item
  RT
  kk\_hirono(刑事告発・非常上告_金沢地方検察庁御中)|bengodan\_single(弁護団ひとり)
  日時:2021-06-09 18:32/2021/06/05 17:14 URL:
  \url{https://twitter.com/kk\_hirono/status/1402559316691881990} 
  \url{https://twitter.com/bengodan\_single/status/1401089986036785156} 
  \textgreater{} @gimu13
  一連のツイートを全部は読んでないのですが(めんどくさいため読む気がない)、なぜ弁護士費用の額を知られているのでしょうか。弁護士費用の額より安いのが嫌だというのは斬新ですが人情としてはもっともな理由だなと思いました、弁護士側から知らせてしまっているなら余分なこと言ったなと思います。
\end{itemize}

〉〉〉 kk\_hironoのリツイート 〉〉〉

\begin{itemize}
\tightlist
\item
  RT
  kk\_hirono(刑事告発・非常上告_金沢地方検察庁御中)|GpoT7eeDYIOJhLC(毒舌あざらし@補助金総合アカウント)
  日時:2021-06-09 18:33/2021/06/05 22:19 URL:
  \url{https://twitter.com/kk\_hirono/status/1402559393107890177} 
  \url{https://twitter.com/GpoT7eeDYIOJhLC/status/1401166800369250311} 
  \textgreater{} @bengodan\_single @gimu13
  横ですが、55万という数値は実際に弁護士さんが言及していますね。
  \url{https://t.co/jWj89YsolG} 
\end{itemize}

〉〉〉 kk\_hironoのリツイート 〉〉〉

\begin{itemize}
\tightlist
\item
  RT
  kk\_hirono(刑事告発・非常上告_金沢地方検察庁御中)|perorin2018(キャバ嬢さくら🌸キャバクラ求人🌸)
  日時:2021-06-09 18:33/2021/06/05 19:39 URL:
  \url{https://twitter.com/kk\_hirono/status/1402559494668779524} 
  \url{https://twitter.com/perorin2018/status/1401126408668078081} 
  \textgreater{} @LiarLawyer800 @gimu13
  撮った画像が残って犯人も認めて防犯カメラにもうつってるので不起訴にならないと思います
\end{itemize}

〉〉〉 kk\_hironoのリツイート 〉〉〉

\begin{itemize}
\tightlist
\item
  RT
  kk\_hirono(刑事告発・非常上告_金沢地方検察庁御中)|perorin2018(キャバ嬢さくら🌸キャバクラ求人🌸)
  日時:2021-06-09 18:33/2021/06/05 19:40 URL:
  \url{https://twitter.com/kk\_hirono/status/1402559528382590977} 
  \url{https://twitter.com/perorin2018/status/1401126748507447298} 
  \textgreater{} @gimu13 過去に盗撮で350万円いただいてます
\end{itemize}

〉〉〉 kk\_hironoのリツイート 〉〉〉

\begin{itemize}
\tightlist
\item
  RT
  kk\_hirono(刑事告発・非常上告_金沢地方検察庁御中)|kouya7977(林宏弥)
  日時:2021-06-09 18:34/2021/06/06 06:29 URL:
  \url{https://twitter.com/kk\_hirono/status/1402559755638300672} 
  \url{https://twitter.com/kouya7977/status/1401290137703620608} 
  \textgreater{}
  素人にしてはなかなか交渉上手なキャバ嬢さんですね(\textsuperscript{-})d笑
  \url{https://t.co/haY2ntV5rU} 
\end{itemize}

〉〉〉 kk\_hironoのリツイート 〉〉〉

\begin{itemize}
\tightlist
\item
  RT
  kk\_hirono(刑事告発・非常上告_金沢地方検察庁御中)|abcabcabc999666(ねこパ〜スタ)
  日時:2021-06-09 18:34/2021/06/05 23:34 URL:
  \url{https://twitter.com/kk\_hirono/status/1402559820666859526} 
  \url{https://twitter.com/abcabcabc999666/status/1401185658861850631} 
  \textgreater{}
  強気やなと言う印象だが刑事手続き中で足元見れる今ならそれなりの金額を取ることも不可能ではないと思う
  \url{https://t.co/aZdU1XvUkF} 
\end{itemize}

〉〉〉 kk\_hironoのリツイート 〉〉〉

\begin{itemize}
\tightlist
\item
  RT
  kk\_hirono(刑事告発・非常上告_金沢地方検察庁御中)|ykrngt87(ほとんど普通だいふく)
  日時:2021-06-09 18:35/2021/06/05 20:58 URL:
  \url{https://twitter.com/kk\_hirono/status/1402559945095057412} 
  \url{https://twitter.com/ykrngt87/status/1401146417826066435} 
  \textgreater{}
  こんな要求されても普通の弁護士なら示談打ち切って検察庁に報告書出すだけだよな・・・
  \url{https://t.co/XNSEJAP3O1} 
\end{itemize}

〉〉〉 kk\_hironoのリツイート 〉〉〉

\begin{itemize}
\tightlist
\item
  RT
  kk\_hirono(刑事告発・非常上告_金沢地方検察庁御中)|CPA93754048(CPA取得を目指す弁護士@慶應通信経済学部74期)
  日時:2021-06-09 18:35/2021/06/05 18:29 URL:
  \url{https://twitter.com/kk\_hirono/status/1402560033779486722} 
  \url{https://twitter.com/CPA93754048/status/1401108794751414278} 
  \textgreater{}
  弁護士の引用リツイートが、軒並みこんな弁護士はありえない、って内容で草なんだけど、73期?74期?くらいだったらあり得るのかもね。
  \url{https://t.co/1388MpQzxV} 
\end{itemize}

〉〉〉 kk\_hironoのリツイート 〉〉〉

\begin{itemize}
\tightlist
\item
  RT
  kk\_hirono(刑事告発・非常上告_金沢地方検察庁御中)|hitujiken(羊犬)
  日時:2021-06-09 18:36/2021/06/05 17:42 URL:
  \url{https://twitter.com/kk\_hirono/status/1402560151270264834} 
  \url{https://twitter.com/hitujiken/status/1401097063794806784} 
  \textgreater{}
  絶対嘘だろ。弁護人なら「許せません」なんて言わないし、このやりとりを検察官に出して、50万提示しましだが多額を請求されたので示談できませんでした、で終わり。
  \url{https://t.co/UCURriGoG3} 
\end{itemize}

〉〉〉 kk\_hironoのリツイート 〉〉〉

\begin{itemize}
\tightlist
\item
  RT
  kk\_hirono(刑事告発・非常上告_金沢地方検察庁御中)|monobotto(者bot)
  日時:2021-06-09 18:37/2021/06/05 17:11 URL:
  \url{https://twitter.com/kk\_hirono/status/1402560442396942340} 
  \url{https://twitter.com/monobotto/status/1401089337937121280} 
  \textgreater{} 示談経験者 \url{https://t.co/mIkfaiO9gH} 
\end{itemize}

〉〉〉 kk\_hironoのリツイート 〉〉〉

\begin{itemize}
\tightlist
\item
  RT
  kk\_hirono(刑事告発・非常上告_金沢地方検察庁御中)|htsj11011207(深海猫)
  日時:2021-06-09 18:37/2021/06/05 17:06 URL:
  \url{https://twitter.com/kk\_hirono/status/1402560505277865984} 
  \url{https://twitter.com/htsj11011207/status/1401088012709101569} 
  \textgreater{}
  マジで弁護士がこれ言ってるなら相当の無能。依頼人の妻子の有無とかぽろっと言っていい情報ちゃうでしょ
  \url{https://t.co/ZV0j40TKD7} 
\end{itemize}

〉〉〉 kk\_hironoのリツイート 〉〉〉

\begin{itemize}
\item
  RT
  kk\_hirono(刑事告発・非常上告_金沢地方検察庁御中)|tosyokupipinyan(当職ぴぴにゃん)
  日時:2021-06-09 18:37/2021/06/05 15:51 URL:
  \url{https://twitter.com/kk\_hirono/status/1402560570457423879} 
  \url{https://twitter.com/tosyokupipinyan/status/1401069225565843457} 
  \textgreater{} まぁ交渉の仕方として上手くないね。
  \url{https://t.co/S4afBKqmxe} 
\item
  〉〉〉 アカウント(@gimu13)は,@kk\_hironoをブロックしています。リツイートできませんでした。
  〉〉〉 ¥\n ¥\n \url{https://t.co/zcbReVCcTp} 
\item
  〉〉〉 アカウント(@luckymangan)は,@kk\_hironoをブロックしています。リツイートできませんでした。
  〉〉〉 ¥\n ¥\n \url{https://t.co/qjYsvZCebW} 
\item
  〉〉〉 アカウント(@gimu13)は,@kk\_hironoをブロックしています。リツイートできませんでした。
  〉〉〉 ¥\n ¥\n \url{https://t.co/zZK3QjN58N} 
\item
  〉〉〉 アカウント(@luckymangan)は,@kk\_hironoをブロックしています。リツイートできませんでした。
  〉〉〉 ¥\n ¥\n \url{https://t.co/Dx3lSUY9dv} 
\item
  〉〉〉 アカウント(@LiarLawyer800)は,@kk\_hironoをブロックしています。リツイートできませんでした。
  〉〉〉 ¥\n ¥\n \url{https://t.co/Tlldq01Mfd} 
\item
  〉〉〉 アカウント(@kame\_ishi)は,@kk\_hironoをブロックしています。リツイートできませんでした。
  〉〉〉 ¥\n ¥\n \url{https://t.co/if1wLQ96CP} 
\item
  〉〉〉 アカウント(@jmjhjmwtad)は,@kk\_hironoをブロックしています。リツイートできませんでした。
  〉〉〉 ¥\n ¥\n \url{https://t.co/4k7pGFdPIU} 
\item
  〉〉〉 アカウント(@jmjhjmwtad)は,@kk\_hironoをブロックしています。リツイートできませんでした。
  〉〉〉 ¥\n ¥\n \url{https://t.co/XmqqpVw6Qq} 
\item
  〉〉〉 アカウント(@jmjhjmwtad)は,@kk\_hironoをブロックしています。リツイートできませんでした。
  〉〉〉 ¥\n ¥\n \url{https://t.co/niJXPxLKYE} 
\item
  〉〉〉 アカウント(@LiarLawyer800)は,@kk\_hironoをブロックしています。リツイートできませんでした。
  〉〉〉 ¥\n ¥\n \url{https://t.co/xptYnldYsB} 
\end{itemize}

※ @kk\_hironoのアカウントがブロックされ,リツイートに失敗したツイート

\begin{itemize}
\tightlist
\item
  TW gimu13(gimu13) 日時:2021/06/05 15:43:14 URL:
  \url{https://twitter.com/gimu13/status/1401067067101511685} 
  \textgreater{}
  まあ法外な請求されたんで示談無理って報告書と相場の金額の供託をして終わりかなあ・・・。
  \url{https://t.co/UcNbGcE2Rc} 
\end{itemize}

※ @kk\_hironoのアカウントがブロックされ,リツイートに失敗したツイート

\begin{itemize}
\tightlist
\item
  TW luckymangan(リーチ一発ツモ裏1) 日時:2021/06/05 15:49:02 URL:
  \url{https://twitter.com/luckymangan/status/1401068527344181252} 
  \textgreater{} @gimu13
  不起訴になる可能性もありますから、あまりね・・・\\
  \textgreater{}
  なお、検察官と相談したり、検察官から説明してもらったりとかもやりますね。
\end{itemize}

※ @kk\_hironoのアカウントがブロックされ,リツイートに失敗したツイート

\begin{itemize}
\tightlist
\item
  TW gimu13(gimu13) 日時:2021/06/05 16:01:21 URL:
  \url{https://twitter.com/gimu13/status/1401071628709105669} 
  \textgreater{} @luckymangan
  こういうケースって検察官も相手がヤカラだということはわかってるんですよね。
\end{itemize}

※ @kk\_hironoのアカウントがブロックされ,リツイートに失敗したツイート

\begin{itemize}
\tightlist
\item
  TW luckymangan(リーチ一発ツモ裏1) 日時:2021/06/05 16:21:36 URL:
  \url{https://twitter.com/luckymangan/status/1401076725677764610} 
  \textgreater{} @gimu13
  輩かどうかはさておき、被害者の情報や言動も検察官は考慮しますよね。
\end{itemize}

※ @kk\_hironoのアカウントがブロックされ,リツイートに失敗したツイート

\begin{itemize}
\tightlist
\item
  TW LiarLawyer800(うそつきべ んごし。) 日時:2021/06/05 17:15:45
  URL: \url{https://twitter.com/LiarLawyer800/status/1401090353017417730} 
  \textgreater{} @gimu13
  報告書上げて、弁護人が相場の示談金既に預かってること示しておけば、被害者がヤカラってことで、不起訴になる率上がるやつですからねぇ。
\end{itemize}

※ @kk\_hironoのアカウントがブロックされ,リツイートに失敗したツイート

\begin{itemize}
\tightlist
\item
  TW kame\_ishi(かめもち) 日時:2021/06/05 21:23:57 URL:
  \url{https://twitter.com/kame\_ishi/status/1401152812394844162} 
  \textgreater{} いやこれ、弁護士がかなりアレ \url{https://t.co/atCtRkQMbP} 
\end{itemize}

※ @kk\_hironoのアカウントがブロックされ,リツイートに失敗したツイート

\begin{itemize}
\tightlist
\item
  TW jmjhjmwtad(7286) 日時:2021/06/05 17:59:42 URL:
  \url{https://twitter.com/jmjhjmwtad/status/1401101413732601858} 
  \textgreater{}
  弁護人への支払が着手金と成功報酬だけで55万って、どうやって知ったんやろかな?ホームページで公開してる事務所なんかな?
  \url{https://t.co/6Wccd3tZzi} 
\end{itemize}

※ @kk\_hironoのアカウントがブロックされ,リツイートに失敗したツイート

\begin{itemize}
\tightlist
\item
  TW jmjhjmwtad(7286) 日時:2021/06/05 17:17:51 URL:
  \url{https://twitter.com/jmjhjmwtad/status/1401090879469748227} 
  \textgreater{}
  とりあえずキャバクラの仕事紹介してるアカウントなんやなあ、これ。
  \url{https://t.co/6Wccd3tZzi} 
\end{itemize}

※ @kk\_hironoのアカウントがブロックされ,リツイートに失敗したツイート

\begin{itemize}
\tightlist
\item
  TW jmjhjmwtad(7286) 日時:2021/06/05 17:16:34 URL:
  \url{https://twitter.com/jmjhjmwtad/status/1401090558169260035} 
  \textgreater{}
  こういうの真に受けて本当に交渉しようとすると、弁護人としては、電話の録音の反訳と、報告書を検察官に上げて供託して終わり。こういうネット情報を真に受けてるなって、弁護人も検察官も分かるんだよなあ。
  \url{https://t.co/6Wccd3tZzi} 
\end{itemize}

※ @kk\_hironoのアカウントがブロックされ,リツイートに失敗したツイート

\begin{itemize}
\tightlist
\item
  TW LiarLawyer800(うそつきべ んごし。) 日時:2021/06/05 17:13:45
  URL: \url{https://twitter.com/LiarLawyer800/status/1401089847939403777} 
  \textgreater{}
  これ、罰金払った方が安いからってことで示談しないやつやな。
  \url{https://t.co/N3zMVmkjZG} 
\end{itemize}

 @jmjhjmwtad(7286)にブロックされたようです。

 坂本正幸弁護士のツイートが見当たらないのが気になっていたのですが、別の流れになっていたようです。

\begin{itemize}
\item
  奉納\危険生物・弁護士脳汚染除去装置\金沢地方検察庁御中\_2020:
  %@sakamotomasayuk 坂本正幸%https://twitter.com/sakamotomasayuk/status/1401160309893603340
  \url{https://kk2020-09.blogspot.com/2021/06/sakamotomasayukhttpstwittercomsakamotom.html} 
\item
  TW gimu13(gimu13) 日時:2021/06/05 15:43:14 URL:
  \url{https://twitter.com/gimu13/status/1401067067101511685} 
  \textgreater{}
  まあ法外な請求されたんで示談無理って報告書と相場の金額の供託をして終わりかなあ・・・。
  \url{https://t.co/UcNbGcE2Rc} 
\end{itemize}

 以前からたまに見かけては、気になることが多かった@gimu13という匿名弁護士らしいアカウントですが、さりげなく「まあ法外な請求されたんで示談無理って報告書と相場の金額の供託をして終わりかなあ・・・。」と結論づけている割には、弁護士の業のような奥深さを感じました。

 ヤカラという言葉が出てきて意味を調べたのですが、モンスターや妖怪扱いしているように感じました。

 他の弁護士らも受け止めが落ち着きなれて見えるのですが、それが異様にも映ります。盗撮で300万円の示談金というのは、民法上の公序良俗に抵触しないのかも疑問ですし、逆に脅迫や恐喝をネタに交渉を進めてくる弁護士もいそうで、それ故の罠を仕掛けているようにも思えるからです。

 示談金の相場というのは素人には分かりづらく、処分権主義として原則は自由な契約にはなりそうです。しかし、子供が死んでも両親の慰謝料というのは300万円程度が相場だと法律の本で読んだ記憶があります。示談金と慰謝料は違いもあるでしょうが、300万円というのは年収に匹敵する金額で、強姦でも高くてそれぐらいだったような。

\begin{itemize}
\tightlist
\item
  〈〈〈 2021/06/09 19:01:24 Linux Emacs: 〈〈〈
\end{itemize}

\hypertarget{ux5211ux4e8bux6c11ux4e8bux306eux5404ux7a2eux6cd5ux898fux306eux30ebux30fcux30ebux3092ux7121ux8996ux3059ux308bux59ffux52e2ux306eux30a2ux30abux30a6ux30f3ux30c8ux306fux30d6ux30edux30c3ux30afux3057ux307eux3059ux3068ux3044ux3046ux5f01ux8b77ux58ebux3057ux306eux3060ux5948ux4fddux5b50ux7acbux61b2ux9053uxff17ux533aux7dcfux652fux90e8ux9577yorisoibengoshiux306bux30d6ux30edux30c3ux30afux3055ux308cux3066ux3044ux305fux4ef6}{%
\paragraph{「刑事民事の各種法規のルールを無視する姿勢のアカウントはブロックします。」という弁護士しのだ奈保子(立憲・道7区総支部長)@yorisoibengoshiにブロックされていた件}\label{ux5211ux4e8bux6c11ux4e8bux306eux5404ux7a2eux6cd5ux898fux306eux30ebux30fcux30ebux3092ux7121ux8996ux3059ux308bux59ffux52e2ux306eux30a2ux30abux30a6ux30f3ux30c8ux306fux30d6ux30edux30c3ux30afux3057ux307eux3059ux3068ux3044ux3046ux5f01ux8b77ux58ebux3057ux306eux3060ux5948ux4fddux5b50ux7acbux61b2ux9053uxff17ux533aux7dcfux652fux90e8ux9577yorisoibengoshiux306bux30d6ux30edux30c3ux30afux3055ux308cux3066ux3044ux305fux4ef6}}

\begin{itemize}
\tightlist
\item
  〉〉〉 Linux Emacs: 2021/06/09 21:34:32 〉〉〉
\end{itemize}

:CATEGORIES: @kanazawabengosi \#金沢弁護士会 @JFBAsns
日本弁護士連合会(日弁連) \#法務省 @MOJ\_HOUMU

 時間的物理的な制約もあるので、全ての問題をこの告発状で取り上げるわけにはいかないですが、大きな発見です。久しぶりに見たと思った弁護士しのだ奈保子(立憲・道7区総支部長)@yorisoibengoshiにブロックされていました。それも三浦義隆弁護士のツイート経由でした。

 昨夜のうちに見かけていた横須賀の杉山程彦弁護士が戒告の懲戒処分を受けたという話題がきっかけですが、その懲戒処分を最初に知ったのも同じくブロックされている都行志弁護士のツイートでした。

 歴史的な動き、弁護士業界の地殻変動のようなものを最初に感じていたのですが、24時間も経っていないと思うのに、弁護士しのだ奈保子(立憲・道7区総支部長)@yorisoibengoshiにブロックされているのに気がつくという大きな発見がありました。

 なぜブロックされたのか余り心当たりがないだけに興味津々なのですが、スクリーンショットの記録で4分ほど前のツイートというタイミングで見かけたのも、見出しに設定した弁護士しのだ奈保子(立憲・道7区総支部長)@yorisoibengoshiのツイートです。

\begin{itemize}
\tightlist
\item
  2021年06月09日21時45分の登録:
  「@yorisoibengoshi」を@hirono\_hideki @kk\_hirono @s\_hironoで検索 123件の該当 2021-06-09\_21:45の記録
  \url{https://kk2020-09.blogspot.com/2021/06/yorisoibengoshihironohidekikkhironoshir.html} 
\end{itemize}

 これからまとめ記事を開いて確認するところですが、これほどブロックに心当たりのない弁護士アカウントも初めてのことで、それも実名弁護士で、弁護士しのだ奈保子(立憲・道7区総支部長)@yorisoibengoshiとあります。

 次は記録されたラストの5件のツイートになります。割と最近のものがありますが、余り印象になかったものです。ちょっとしたメモ程度で記録したものかと思います。

2021-03-27 11:50:43 ``RT @sato\_\_michiko: @ekinan\_lawyer @o2441
@yorisoibengoshi
お二方、お忙しいなかありがとうございます。11月以来かずみさんからブロックされてたのと、もう一方はミュートにしてたのとでよく分かりませんでしたが、お陰様で大変スッキリしました''
\url{https://twitter.com/kk\_hirono/status/1375641404521648133} 

2021-04-19 06:23:41 ``2021-04-19\_06:19
奉納\\#危険生物・弁護士脳汚染除去装置\\#金沢地方検察庁御中\_2020:
%@yorisoibengoshi 弁護士しのだ奈保子🐸(立憲・道7区総支部長)%配偶者が子連れで逃げたら、ああ、私色々負担かけて悪かったわ、と思うし、相手が会わせたくな
\url{https://t.co/2ArvURl9PL''} 
\url{https://twitter.com/hirono\_hideki/status/1383894023199358978} 

2021-04-20 23:42:20
``2021-04-20-234153\_えきなんローヤーさんがリツイート弁護士しのだ奈保子(立憲・道7区総支部長)@yorisoibengoshi·2時間別居親の事案では、と.jpg
\url{https://t.co/yI9mTAyKIq''} 
\url{https://twitter.com/s\_hirono/status/1384517796470562818} 

2021-05-03 23:18:23 ``2021-05-03\_23:17
奉納\\#危険生物・弁護士脳汚染除去装置\\#金沢地方検察庁御中\_2020:
\弁護士しのだ奈保子🐸(立憲・道7区総支部長) @yorisoibengoshi\国民が主役な現憲法から、国民がお国のために尽くす脇役にされる憲法に変えたい勢力の対立なの
\url{https://kk2020-09.blogspot.com/2021/05/yorisoibengoshi.html''} 
\url{https://twitter.com/hirono\_hideki/status/1389222811030286340} 

2021-06-09 21:25:04
``2021-06-09-211945\_弁護士しのだ奈保子(立憲・道7区総支部長)@yorisoibengoshi返信先: @lawkusさん誤ったことを書いています。.jpg
\url{https://t.co/l7XlgfE6mf''} 
\url{https://twitter.com/s\_hirono/status/1402602646255509513} 

〉〉〉 kk\_hironoのリツイート 〉〉〉

\begin{itemize}
\tightlist
\item
  RT
  kk\_hirono(刑事告発・非常上告_金沢地方検察庁御中)|yorisoibengoshi(弁護士しのだ奈保子🐸(立憲・道7区総支部長))
  日時:2021-06-09 21:54/2021/06/09 21:37 URL:
  \url{https://twitter.com/kk\_hirono/status/1402610108131995649} 
  \url{https://twitter.com/yorisoibengoshi/status/1402605825949007874} 
  \textgreater{}
  私に対するものについての対応です。これまで、ほとんどブロックしてはいませんが、やはり、しっかりと対応する場面があると思いますから、適宜の判断をして参ります!
  \url{https://t.co/6K1Fkl5bdQ} 
\end{itemize}

 また妙なタイミングで、弁護士しのだ奈保子(立憲・道7区総支部長)@yorisoibengoshiのツイートが更新されていたのですが、その辺りもスクリーンショット等で記録しました。また時間の問題かもしれないですが、現時点で告発\市場急配センター殺人未遂事件\金沢地方検察庁・石川県警察御中(@kk\_hirono)はブロックされていないようです。

〉〉〉 kk\_hironoのリツイート 〉〉〉

\begin{itemize}
\tightlist
\item
  RT
  kk\_hirono(刑事告発・非常上告_金沢地方検察庁御中)|s\_hirono(非常上告-最高検察庁御中\_ツイッター)
  日時:2021-06-09 22:03/2021/06/09 21:58 URL:
  \url{https://twitter.com/kk\_hirono/status/1402612199563005954} 
  \url{https://twitter.com/s\_hirono/status/1402611126236377094} 
  \textgreater{}
  2021-06-09-215755\_刑事告発・非常上告_金沢地方検察庁御中@kk\_hirono·9秒返信先: @yorisoibengoshiさんブロックされています@yori.jpg
  \url{https://t.co/pQpBWSwdUg} 
\end{itemize}

〉〉〉 kk\_hironoのリツイート 〉〉〉

\begin{itemize}
\tightlist
\item
  RT
  kk\_hirono(刑事告発・非常上告_金沢地方検察庁御中)|s\_hirono(非常上告-最高検察庁御中\_ツイッター)
  日時:2021-06-09 22:03/2021/06/09 21:57 URL:
  \url{https://twitter.com/kk\_hirono/status/1402612228658925568} 
  \url{https://twitter.com/s\_hirono/status/1402610906433867776} 
  \textgreater{}
  2021-06-09-215620\_弁護士しのだ奈保子(立憲・道7区総支部長)@yorisoibengoshi私に対するものについての対応です。これまで、ほとんどブロックして.jpg
  \url{https://t.co/ZWspTqK8Dx} 
\end{itemize}

〉〉〉 kk\_hironoのリツイート 〉〉〉

\begin{itemize}
\tightlist
\item
  RT
  kk\_hirono(刑事告発・非常上告_金沢地方検察庁御中)|s\_hirono(非常上告-最高検察庁御中\_ツイッター)
  日時:2021-06-09 22:03/2021/06/09 21:57 URL:
  \url{https://twitter.com/kk\_hirono/status/1402612242999250944} 
  \url{https://twitter.com/s\_hirono/status/1402610834057035786} 
  \textgreater{}
  2021-06-09-215600\_弁護士しのだ奈保子(立憲・道7区総支部長)@yorisoibengoshi刑事民事の各種法規のルールを無視する姿勢のアカウントはブロックし.jpg
  \url{https://t.co/LWuBY5uPJb} 
\end{itemize}

〉〉〉 kk\_hironoのリツイート 〉〉〉

\begin{itemize}
\tightlist
\item
  RT
  kk\_hirono(刑事告発・非常上告_金沢地方検察庁御中)|s\_hirono(非常上告-最高検察庁御中\_ツイッター)
  日時:2021-06-09 22:03/2021/06/09 21:57 URL:
  \url{https://twitter.com/kk\_hirono/status/1402612268378951688} 
  \url{https://twitter.com/s\_hirono/status/1402610760971259912} 
  \textgreater{}
  2021-06-09-215514\_弁護士しのだ奈保子(立憲・道7区総支部長)@yorisoibengoshi·17分私に対するものについての対応です。これまで、ほとんどブロ.jpg
  \url{https://t.co/nnQeQlB9UO} 
\end{itemize}

〉〉〉 kk\_hironoのリツイート 〉〉〉

\begin{itemize}
\tightlist
\item
  RT
  kk\_hirono(刑事告発・非常上告_金沢地方検察庁御中)|s\_hirono(非常上告-最高検察庁御中\_ツイッター)
  日時:2021-06-09 22:03/2021/06/09 21:57 URL:
  \url{https://twitter.com/kk\_hirono/status/1402612280982839306} 
  \url{https://twitter.com/s\_hirono/status/1402610687029825537} 
  \textgreater{}
  2021-06-09-215509\_弁護士しのだ奈保子(立憲・道7区総支部長)@yorisoibengoshi·17分私に対するものについての対応です。これまで、ほとんどブロ.jpg
  \url{https://t.co/Ttt8dk1YR9} 
\end{itemize}

〉〉〉 kk\_hironoのリツイート 〉〉〉

\begin{itemize}
\tightlist
\item
  RT
  kk\_hirono(刑事告発・非常上告_金沢地方検察庁御中)|s\_hirono(非常上告-最高検察庁御中\_ツイッター)
  日時:2021-06-09 22:03/2021/06/09 21:46 URL:
  \url{https://twitter.com/kk\_hirono/status/1402612304647114755} 
  \url{https://twitter.com/s\_hirono/status/1402608100813676547} 
  \textgreater{}
  2021-06-09-213120\_弁護士しのだ奈保子🐸(立憲・道7区総支部長)@yorisoibengoshiブロックされています@yorisoibengoshiさんのフォロ.jpg
  \url{https://t.co/myAD9QSX5D} 
\end{itemize}

〉〉〉 kk\_hironoのリツイート 〉〉〉

\begin{itemize}
\tightlist
\item
  RT
  kk\_hirono(刑事告発・非常上告_金沢地方検察庁御中)|s\_hirono(非常上告-最高検察庁御中\_ツイッター)
  日時:2021-06-09 22:03/2021/06/09 21:46 URL:
  \url{https://twitter.com/kk\_hirono/status/1402612316886102018} 
  \url{https://twitter.com/s\_hirono/status/1402608028440879117} 
  \textgreater{}
  2021-06-09-212928\_弁護士しのだ奈保子カエルの顔(立憲・道7区総支部長)@yorisoibengoshi·2分刑事民事の各種法規のルールを無視する姿勢のアカウン.jpg
  \url{https://t.co/frURF6JxYt} 
\end{itemize}

〉〉〉 kk\_hironoのリツイート 〉〉〉

\begin{itemize}
\tightlist
\item
  RT
  kk\_hirono(刑事告発・非常上告_金沢地方検察庁御中)|s\_hirono(非常上告-最高検察庁御中\_ツイッター)
  日時:2021-06-09 22:03/2021/06/09 21:46 URL:
  \url{https://twitter.com/kk\_hirono/status/1402612336922218500} 
  \url{https://twitter.com/s\_hirono/status/1402607955913043970} 
  \textgreater{}
  2021-06-09-212745\_弁護士しのだ奈保子🐸(立憲・道7区総支部長)@yorisoibengoshi返信先: @lawkusさん誤ったことを書いています。.jpg
  \url{https://t.co/QmpTi6s2T2} 
\end{itemize}

〉〉〉 kk\_hironoのリツイート 〉〉〉

\begin{itemize}
\tightlist
\item
  RT
  kk\_hirono(刑事告発・非常上告_金沢地方検察庁御中)|s\_hirono(非常上告-最高検察庁御中\_ツイッター)
  日時:2021-06-09 22:03/2021/06/09 21:25 URL:
  \url{https://twitter.com/kk\_hirono/status/1402612353879871489} 
  \url{https://twitter.com/s\_hirono/status/1402602646255509513} 
  \textgreater{}
  2021-06-09-211945\_弁護士しのだ奈保子(立憲・道7区総支部長)@yorisoibengoshi返信先: @lawkusさん誤ったことを書いています。.jpg
  \url{https://t.co/l7XlgfE6mf} 
\end{itemize}

〉〉〉 kk\_hironoのリツイート 〉〉〉

\begin{itemize}
\tightlist
\item
  RT
  kk\_hirono(刑事告発・非常上告_金沢地方検察庁御中)|s\_hirono(非常上告-最高検察庁御中\_ツイッター)
  日時:2021-06-09 22:03/2021/06/09 20:54 URL:
  \url{https://twitter.com/kk\_hirono/status/1402612392110919683} 
  \url{https://twitter.com/s\_hirono/status/1402594854874451972} 
  \textgreater{}
  2021-06-09-205244\_ピピピーッ@O59K2dPQH59QEJx杉山弁護士に対する懲戒理由を見たけど、重過ぎるという印象。これが、「戒告」相当なのか? どんどんハ.jpg
  \url{https://t.co/d1YAq9vKfP} 
\end{itemize}

〉〉〉 kk\_hironoのリツイート 〉〉〉

\begin{itemize}
\tightlist
\item
  RT
  kk\_hirono(刑事告発・非常上告_金沢地方検察庁御中)|s\_hirono(非常上告-最高検察庁御中\_ツイッター)
  日時:2021-06-09 22:03/2021/06/09 20:53 URL:
  \url{https://twitter.com/kk\_hirono/status/1402612411975180295} 
  \url{https://twitter.com/s\_hirono/status/1402594782849900561} 
  \textgreater{}
  2021-06-09-204859\_ystk@lawkus·1時間懲戒権の濫用・拡大を警戒する弁護士の気持ちも俺は我が身のこととしてわかるけど、「同業者に対して業として犯罪を行.jpg
  \url{https://t.co/JMYFQHcPMZ} 
\end{itemize}

〉〉〉 kk\_hironoのリツイート 〉〉〉

\begin{itemize}
\tightlist
\item
  RT
  kk\_hirono(刑事告発・非常上告_金沢地方検察庁御中)|s\_hirono(非常上告-最高検察庁御中\_ツイッター)
  日時:2021-06-09 22:03/2021/06/09 20:53 URL:
  \url{https://twitter.com/kk\_hirono/status/1402612436503388171} 
  \url{https://twitter.com/s\_hirono/status/1402594710279966721} 
  \textgreater{}
  2021-06-09-204732\_ystk@lawkus·1時間提供してません。信頼あるマスメディアならまだしも、あのような胡散臭いサイトに議決書を提供することなどないです。.jpg
  \url{https://t.co/Z49mQJUwfo} 
\end{itemize}

〉〉〉 kk\_hironoのリツイート 〉〉〉

\begin{itemize}
\tightlist
\item
  RT
  kk\_hirono(刑事告発・非常上告_金沢地方検察庁御中)|s\_hirono(非常上告-最高検察庁御中\_ツイッター)
  日時:2021-06-09 22:04/2021/06/09 20:53 URL:
  \url{https://twitter.com/kk\_hirono/status/1402612473627172867} 
  \url{https://twitter.com/s\_hirono/status/1402594638167347202} 
  \textgreater{}
  2021-06-09-204534\_7286@jmjhjmwtad·2時間私はやってないーけっぱくだー私はやってないーけっぱくだー.jpg
  \url{https://t.co/A1SuqY5zqH} 
\end{itemize}

〉〉〉 kk\_hironoのリツイート 〉〉〉

\begin{itemize}
\tightlist
\item
  RT
  kk\_hirono(刑事告発・非常上告_金沢地方検察庁御中)|s\_hirono(非常上告-最高検察庁御中\_ツイッター)
  日時:2021-06-09 22:04/2021/06/09 20:44 URL:
  \url{https://twitter.com/kk\_hirono/status/1402612491725672463} 
  \url{https://twitter.com/s\_hirono/status/1402592518991073286} 
  \textgreater{}
  2021-06-09-204344\_ギタ弁(B'z大好き)@guitar\_benブロックされています@guitar\_benさんのフォローやツイートの表示はできません。詳細はこち.jpg
  \url{https://t.co/HhanXk1i3b} 
\end{itemize}

〉〉〉 kk\_hironoのリツイート 〉〉〉

\begin{itemize}
\tightlist
\item
  RT
  kk\_hirono(刑事告発・非常上告_金沢地方検察庁御中)|s\_hirono(非常上告-最高検察庁御中\_ツイッター)
  日時:2021-06-09 22:04/2021/06/09 20:44 URL:
  \url{https://twitter.com/kk\_hirono/status/1402612521631055875} 
  \url{https://twitter.com/s\_hirono/status/1402592447004250113} 
  \textgreater{}
  2021-06-09-204153\_7286さんがリツイートギタ弁(B'z大好き)@guitar\_ben·2時間「警察が被害届を受理しない」→「法改正して構成要件を緩和しよう」.jpg
  \url{https://t.co/XehugDbsi2} 
\end{itemize}

〉〉〉 kk\_hironoのリツイート 〉〉〉

\begin{itemize}
\tightlist
\item
  RT
  kk\_hirono(刑事告発・非常上告_金沢地方検察庁御中)|s\_hirono(非常上告-最高検察庁御中\_ツイッター)
  日時:2021-06-09 22:04/2021/06/09 20:44 URL:
  \url{https://twitter.com/kk\_hirono/status/1402612547086295040} 
  \url{https://twitter.com/s\_hirono/status/1402592374904168450} 
  \textgreater{}
  2021-06-09-204036\_7286@jmjhjmwtad·45分結構他にもヤバい人がいそうな気がします・・・.jpg
  \url{https://t.co/LifrMls0sz} 
\end{itemize}

〉〉〉 kk\_hironoのリツイート 〉〉〉

\begin{itemize}
\tightlist
\item
  RT
  kk\_hirono(刑事告発・非常上告_金沢地方検察庁御中)|s\_hirono(非常上告-最高検察庁御中\_ツイッター)
  日時:2021-06-09 22:04/2021/06/09 20:43 URL:
  \url{https://twitter.com/kk\_hirono/status/1402612567571238912} 
  \url{https://twitter.com/s\_hirono/status/1402592302736961538} 
  \textgreater{}
  2021-06-09-202809\_らめーん@shouwarame返信先: @uwaaaaさんはい、どのような事例が立件されて有罪になり、どのような事例が被害届も受理されないか.jpg
  \url{https://t.co/p0s0F8MWgQ} 
\end{itemize}

〉〉〉 kk\_hironoのリツイート 〉〉〉

\begin{itemize}
\tightlist
\item
  RT
  kk\_hirono(刑事告発・非常上告_金沢地方検察庁御中)|s\_hirono(非常上告-最高検察庁御中\_ツイッター)
  日時:2021-06-09 22:04/2021/06/09 20:31 URL:
  \url{https://twitter.com/kk\_hirono/status/1402612584834953219} 
  \url{https://twitter.com/s\_hirono/status/1402589185278545923} 
  \textgreater{}
  2021-06-09-202709\_サイ太@uwaaaa返信先: @okumuraosakaさんらめーん先生からこういう発言があったのですが,先生の視点ではいかがでしょうか。.jpg
  \url{https://t.co/CKToKnKeV1} 
\end{itemize}

〉〉〉 kk\_hironoのリツイート 〉〉〉

\begin{itemize}
\tightlist
\item
  RT
  kk\_hirono(刑事告発・非常上告_金沢地方検察庁御中)|s\_hirono(非常上告-最高検察庁御中\_ツイッター)
  日時:2021-06-09 22:04/2021/06/09 20:31 URL:
  \url{https://twitter.com/kk\_hirono/status/1402612607584899078} 
  \url{https://twitter.com/s\_hirono/status/1402589112859717634} 
  \textgreater{}
  2021-06-09-202023\_サイ太さんがリツイートystk@lawkus·57分「弁護士自治を考える会」とかいうサイトに議決書が丸々引用されていたので私も驚きました。ど.jpg
  \url{https://t.co/AJJrQ0Zitb} 
\end{itemize}

〉〉〉 kk\_hironoのリツイート 〉〉〉

\begin{itemize}
\tightlist
\item
  RT
  kk\_hirono(刑事告発・非常上告_金沢地方検察庁御中)|s\_hirono(非常上告-最高検察庁御中\_ツイッター)
  日時:2021-06-09 22:04/2021/06/09 20:31 URL:
  \url{https://twitter.com/kk\_hirono/status/1402612628736798720} 
  \url{https://twitter.com/s\_hirono/status/1402589040646311937} 
  \textgreater{}
  2021-06-09-201921\_サイ太さんがリツイートokumuraosaka@okumuraosaka·55分児童淫行罪を否定したときに出てくる青少年条例というのが要件緩.jpg
  \url{https://t.co/bcaIOGqnI2} 
\end{itemize}

〉〉〉 kk\_hironoのリツイート 〉〉〉

\begin{itemize}
\tightlist
\item
  RT
  kk\_hirono(刑事告発・非常上告_金沢地方検察庁御中)|s\_hirono(非常上告-最高検察庁御中\_ツイッター)
  日時:2021-06-09 22:04/2021/06/09 20:30 URL:
  \url{https://twitter.com/kk\_hirono/status/1402612653571276803} 
  \url{https://twitter.com/s\_hirono/status/1402588968525303811} 
  \textgreater{}
  2021-06-09-201818\_サイ太さんがリツイート米山 隆一@RyuichiYoneyama·3時間東京地方裁判所に継続しておりましたツイッター上の「黒瀬深」なるアカウ.jpg
  \url{https://t.co/NPzLiltaNv} 
\end{itemize}

〉〉〉 kk\_hironoのリツイート 〉〉〉

\begin{itemize}
\tightlist
\item
  RT
  kk\_hirono(刑事告発・非常上告_金沢地方検察庁御中)|s\_hirono(非常上告-最高検察庁御中\_ツイッター)
  日時:2021-06-09 22:04/2021/06/09 20:30 URL:
  \url{https://twitter.com/kk\_hirono/status/1402612684164456449} 
  \url{https://twitter.com/s\_hirono/status/1402588895116546055} 
  \textgreater{}
  2021-06-09-201701\_弁護士落合洋司桜感染拡大を招く東京(頭狂)オリンピック中止!桜@yjochi1964広島県生。修道→早大法1987卒。1986司法試験合格(.jpg
  \url{https://t.co/zfu3tJz8bh} 
\end{itemize}

〉〉〉 kk\_hironoのリツイート 〉〉〉

\begin{itemize}
\tightlist
\item
  RT
  kk\_hirono(刑事告発・非常上告_金沢地方検察庁御中)|s\_hirono(非常上告-最高検察庁御中\_ツイッター)
  日時:2021-06-09 22:05/2021/06/09 20:30 URL:
  \url{https://twitter.com/kk\_hirono/status/1402612696869064705} 
  \url{https://twitter.com/s\_hirono/status/1402588822043389954} 
  \textgreater{}
  2021-06-09-201646\_伊藤真@ito\_\_makoto·2012年5月10日論究ジュリスト2012春号「特集憲法最高裁判例を読み直す」を読んだ。安念潤司先生の「憲法.jpg
  \url{https://t.co/TTvklCc0JH} 
\end{itemize}

〉〉〉 kk\_hironoのリツイート 〉〉〉

\begin{itemize}
\tightlist
\item
  RT
  kk\_hirono(刑事告発・非常上告_金沢地方検察庁御中)|s\_hirono(非常上告-最高検察庁御中\_ツイッター)
  日時:2021-06-09 22:05/2021/06/09 20:29 URL:
  \url{https://twitter.com/kk\_hirono/status/1402612732315115525} 
  \url{https://twitter.com/s\_hirono/status/1402588749750411266} 
  \textgreater{}
  2021-06-09-201548\_弁護士落合洋司桜感染拡大を招く東京(頭狂)オリンピック中止!桜@yjochi·4時間よらば大樹で破滅する国民性だから。→報道の中立に疑念・・・大.jpg
  \url{https://t.co/VXxljDsNge} 
\end{itemize}

〉〉〉 kk\_hironoのリツイート 〉〉〉

\begin{itemize}
\tightlist
\item
  RT
  kk\_hirono(刑事告発・非常上告_金沢地方検察庁御中)|s\_hirono(非常上告-最高検察庁御中\_ツイッター)
  日時:2021-06-09 22:05/2021/06/09 20:29 URL:
  \url{https://twitter.com/kk\_hirono/status/1402612744172359685} 
  \url{https://twitter.com/s\_hirono/status/1402588677264445440} 
  \textgreater{}
  2021-06-09-201440\_Shoko Egawa@amneris84返信先: @yjochiさん黒川さんの時も、言わなかったです.jpg
  \url{https://t.co/cEshzIbgAi} 
\end{itemize}

〉〉〉 kk\_hironoのリツイート 〉〉〉

\begin{itemize}
\tightlist
\item
  RT
  kk\_hirono(刑事告発・非常上告_金沢地方検察庁御中)|s\_hirono(非常上告-最高検察庁御中\_ツイッター)
  日時:2021-06-09 22:05/2021/06/09 20:29 URL:
  \url{https://twitter.com/kk\_hirono/status/1402612764619657220} 
  \url{https://twitter.com/s\_hirono/status/1402588604837240845} 
  \textgreater{}
  2021-06-09-190541\_7286@jmjhjmwtadブロックされています@jmjhjmwtadさんのフォローやツイートの表示はできません。詳細はこちら.jpg
  \url{https://t.co/JYuKRnRYIO} 
\end{itemize}

〉〉〉 kk\_hironoのリツイート 〉〉〉

\begin{itemize}
\tightlist
\item
  RT
  kk\_hirono(刑事告発・非常上告_金沢地方検察庁御中)|s\_hirono(非常上告-最高検察庁御中\_ツイッター)
  日時:2021-06-09 22:05/2021/06/09 20:28 URL:
  \url{https://twitter.com/kk\_hirono/status/1402612787147247616} 
  \url{https://twitter.com/s\_hirono/status/1402588532414111752} 
  \textgreater{}
  2021-06-09-182711\_gimu13@gimu13まあ法外な請求されたんで示談無理って報告書と相場の金額の供託をして終わりかなあ・・・。.jpg
  \url{https://t.co/zL46DVfYY5} 
\end{itemize}

〉〉〉 kk\_hironoのリツイート 〉〉〉

\begin{itemize}
\tightlist
\item
  RT
  kk\_hirono(刑事告発・非常上告_金沢地方検察庁御中)|s\_hirono(非常上告-最高検察庁御中\_ツイッター)
  日時:2021-06-09 22:05/2021/06/09 20:28 URL:
  \url{https://twitter.com/kk\_hirono/status/1402612834526138375} 
  \url{https://twitter.com/s\_hirono/status/1402588460318289922} 
  \textgreater{}
  2021-06-09-134434\_Shoko Egawa@amneris84·2011年2月19日返信先: @motoken\_twさん@motoken\_tw 事件の犯人であっ.jpg
  \url{https://t.co/gAZ7yJj8fZ} 
\end{itemize}

〉〉〉 kk\_hironoのリツイート 〉〉〉

\begin{itemize}
\tightlist
\item
  RT
  kk\_hirono(刑事告発・非常上告_金沢地方検察庁御中)|s\_hirono(非常上告-最高検察庁御中\_ツイッター)
  日時:2021-06-09 22:05/2021/06/09 20:28 URL:
  \url{https://twitter.com/kk\_hirono/status/1402612846563774469} 
  \url{https://twitter.com/s\_hirono/status/1402588388067143686} 
  \textgreater{}
  2021-06-09-133045\_高橋雄一郎@kamatatylaw弁護士にとって,国相手に勝訴するのって,アドレナリンが爆発するぐらいのものじゃないかなあ。ある薬害事件で勝.jpg
  \url{https://t.co/QFNZspOh20} 
\end{itemize}

〉〉〉 kk\_hironoのリツイート 〉〉〉

\begin{itemize}
\tightlist
\item
  RT
  kk\_hirono(刑事告発・非常上告_金沢地方検察庁御中)|s\_hirono(非常上告-最高検察庁御中\_ツイッター)
  日時:2021-06-09 22:05/2021/06/09 20:28 URL:
  \url{https://twitter.com/kk\_hirono/status/1402612861348708354} 
  \url{https://twitter.com/s\_hirono/status/1402588315958747138} 
  \textgreater{}
  2021-06-09-124311\_弁護士QUEENBEE(クインビ)@uzumeam·6月6日返信先: @jmjhjmwtadさんしかし、メンヘラは遺伝する事が判明しているし.jpg
  \url{https://t.co/LF4wHd8UDK} 
\end{itemize}

〉〉〉 kk\_hironoのリツイート 〉〉〉

\begin{itemize}
\tightlist
\item
  RT
  kk\_hirono(刑事告発・非常上告_金沢地方検察庁御中)|s\_hirono(非常上告-最高検察庁御中\_ツイッター)
  日時:2021-06-09 22:05/2021/06/09 20:27 URL:
  \url{https://twitter.com/kk\_hirono/status/1402612875722514436} 
  \url{https://twitter.com/s\_hirono/status/1402588243711856640} 
  \textgreater{}
  2021-06-09-124245\_弁護士QUEENBEE(クインビ)@uzumeamブロックされています@uzumeamさんのフォローやツイートの表示はできません。詳細はこち.jpg
  \url{https://t.co/yFdRK65YEw} 
\end{itemize}

〉〉〉 kk\_hironoのリツイート 〉〉〉

\begin{itemize}
\tightlist
\item
  RT
  kk\_hirono(刑事告発・非常上告_金沢地方検察庁御中)|s\_hirono(非常上告-最高検察庁御中\_ツイッター)
  日時:2021-06-09 22:05/2021/06/09 20:27 URL:
  \url{https://twitter.com/kk\_hirono/status/1402612886644477957} 
  \url{https://twitter.com/s\_hirono/status/1402588171569823756} 
  \textgreater{} 2021-06-09-124233\_status/1401469361307619328.jpg
  \url{https://t.co/K9ZmfxIl8y} 
\end{itemize}

〉〉〉 kk\_hironoのリツイート 〉〉〉

\begin{itemize}
\tightlist
\item
  RT
  kk\_hirono(刑事告発・非常上告_金沢地方検察庁御中)|s\_hirono(非常上告-最高検察庁御中\_ツイッター)
  日時:2021-06-09 22:05/2021/06/09 12:40 URL:
  \url{https://twitter.com/kk\_hirono/status/1402612913362280456} 
  \url{https://twitter.com/s\_hirono/status/1402470611692097536} 
  \textgreater{}
  2021-06-09-112300\_小倉秀夫@chosakukenho·35分中学生での性体験率が概ね10〜20%くらいなので、このくらいの割合の数の男性を刑務所に送り込むこと.jpg
  \url{https://t.co/e5LDVCp23X} 
\end{itemize}

〉〉〉 kk\_hironoのリツイート 〉〉〉

\begin{itemize}
\tightlist
\item
  RT
  kk\_hirono(刑事告発・非常上告_金沢地方検察庁御中)|s\_hirono(非常上告-最高検察庁御中\_ツイッター)
  日時:2021-06-09 22:05/2021/06/09 12:40 URL:
  \url{https://twitter.com/kk\_hirono/status/1402612930353369090} 
  \url{https://twitter.com/s\_hirono/status/1402470539558461443} 
  \textgreater{}
  2021-06-09-112033\_高橋雄一郎@kamatatylaw·19分ある行政訴訟で相手方行政庁の指定代理人と裏取引が成立して準備書面のデータを交換することになった。こ.jpg
  \url{https://t.co/LiMKFsMhGR} 
\end{itemize}

〉〉〉 kk\_hironoのリツイート 〉〉〉

\begin{itemize}
\tightlist
\item
  RT
  kk\_hirono(刑事告発・非常上告_金沢地方検察庁御中)|s\_hirono(非常上告-最高検察庁御中\_ツイッター)
  日時:2021-06-09 22:06/2021/06/09 10:14 URL:
  \url{https://twitter.com/kk\_hirono/status/1402612950934777857} 
  \url{https://twitter.com/s\_hirono/status/1402433764848934915} 
  \textgreater{}
  2021-06-09-094705\_モトケン@motoken\_tw·19分現実を知らず(知ろうともせず)独りよがりの観念の世界で考えている人は、みんな似たようなことを言う。要す.jpg
  \url{https://t.co/3eGmEQQVwZ} 
\end{itemize}

〉〉〉 kk\_hironoのリツイート 〉〉〉

\begin{itemize}
\tightlist
\item
  RT
  kk\_hirono(刑事告発・非常上告_金沢地方検察庁御中)|s\_hirono(非常上告-最高検察庁御中\_ツイッター)
  日時:2021-06-09 22:06/2021/06/09 10:13 URL:
  \url{https://twitter.com/kk\_hirono/status/1402612969662386179} 
  \url{https://twitter.com/s\_hirono/status/1402433692614623233} 
  \textgreater{}
  2021-06-09-093155\_ルビック貯金箱@yashi108·6月1日北斗の拳で、モヒカンが奴隷の人力で発電して明かりを灯すシーンがあるんだけど、弁護士会のいう社会を法.jpg
  \url{https://t.co/rcNr9uw76a} 
\end{itemize}

〉〉〉 kk\_hironoのリツイート 〉〉〉

\begin{itemize}
\tightlist
\item
  RT
  kk\_hirono(刑事告発・非常上告_金沢地方検察庁御中)|s\_hirono(非常上告-最高検察庁御中\_ツイッター)
  日時:2021-06-09 22:06/2021/06/09 10:13 URL:
  \url{https://twitter.com/kk\_hirono/status/1402612984594145284} 
  \url{https://twitter.com/s\_hirono/status/1402433618601934851} 
  \textgreater{}
  2021-06-09-092950\_そらまめ@sollamame菊池前日弁連会長が東大法曹会の講演で要旨「LACは大きなマーケットになった、ある地方で若手に主な収入源を尋ねたと.jpg
  \url{https://t.co/QC37LPW8jQ} 
\end{itemize}

〉〉〉 kk\_hironoのリツイート 〉〉〉

\begin{itemize}
\tightlist
\item
  RT
  kk\_hirono(刑事告発・非常上告_金沢地方検察庁御中)|s\_hirono(非常上告-最高検察庁御中\_ツイッター)
  日時:2021-06-09 22:06/2021/06/09 09:23 URL:
  \url{https://twitter.com/kk\_hirono/status/1402612999177719814} 
  \url{https://twitter.com/s\_hirono/status/1402420941506809859} 
  \textgreater{}
  2021-06-09-091032\_浜木綿弁右衛門@leplusallez·23時間基地外に基地外と言わない優しさを感じた.jpg
  \url{https://t.co/e3FvlXE4ZM} 
\end{itemize}

〉〉〉 kk\_hironoのリツイート 〉〉〉

\begin{itemize}
\tightlist
\item
  RT
  kk\_hirono(刑事告発・非常上告_金沢地方検察庁御中)|s\_hirono(非常上告-最高検察庁御中\_ツイッター)
  日時:2021-06-09 22:06/2021/06/09 09:22 URL:
  \url{https://twitter.com/kk\_hirono/status/1402613014352711682} 
  \url{https://twitter.com/s\_hirono/status/1402420868735664130} 
  \textgreater{}
  2021-06-09-090820\_浜木綿弁右衛門さんがリツイートメーテル@\_maeter·19時間どんな事件だろうと必ず迷宮入りさせる神奈川県警さんと、人生がギャンブルな人た.jpg
  \url{https://t.co/7kSer9F8vj} 
\end{itemize}

〉〉〉 kk\_hironoのリツイート 〉〉〉

\begin{itemize}
\tightlist
\item
  RT
  kk\_hirono(刑事告発・非常上告_金沢地方検察庁御中)|s\_hirono(非常上告-最高検察庁御中\_ツイッター)
  日時:2021-06-09 22:06/2021/06/09 09:22 URL:
  \url{https://twitter.com/kk\_hirono/status/1402613027451506693} 
  \url{https://twitter.com/s\_hirono/status/1402420795771547650} 
  \textgreater{}
  2021-06-09-090745\_浜木綿弁右衛門@leplusallez·18時間これはタイトル考えた方がいいんじゃないんですか・・・あと実際の事件からの話の導出もさけたほうが。.jpg
  \url{https://t.co/UlVUaQVnxV} 
\end{itemize}

〉〉〉 kk\_hironoのリツイート 〉〉〉

\begin{itemize}
\tightlist
\item
  RT
  kk\_hirono(刑事告発・非常上告_金沢地方検察庁御中)|s\_hirono(非常上告-最高検察庁御中\_ツイッター)
  日時:2021-06-09 22:06/2021/06/09 03:05 URL:
  \url{https://twitter.com/kk\_hirono/status/1402613040902672392} 
  \url{https://twitter.com/s\_hirono/status/1402326021077045250} 
  \textgreater{}
  2021-06-09-030258\_ふて寝べん@hirune\_b·6月6日岸本先生のツイートを見て、「世の常男は酷い奴ばかりだが、彼は例外で女性のことを理解している」と思う人が.jpg
  \url{https://t.co/wZLS84FQap} 
\end{itemize}

〉〉〉 kk\_hironoのリツイート 〉〉〉

\begin{itemize}
\tightlist
\item
  RT
  kk\_hirono(刑事告発・非常上告_金沢地方検察庁御中)|s\_hirono(非常上告-最高検察庁御中\_ツイッター)
  日時:2021-06-09 22:06/2021/06/09 03:05 URL:
  \url{https://twitter.com/kk\_hirono/status/1402613057977651200} 
  \url{https://twitter.com/s\_hirono/status/1402325948758859778} 
  \textgreater{}
  2021-06-09-025453\_gimu13@gimu13·6月5日まあ法外な請求されたんで示談無理って報告書と相場の金額の供託をして終わりかなあ・・・。.jpg
  \url{https://t.co/mD3XQ2OWoM} 
\end{itemize}

〉〉〉 kk\_hironoのリツイート 〉〉〉

\begin{itemize}
\tightlist
\item
  RT
  kk\_hirono(刑事告発・非常上告_金沢地方検察庁御中)|s\_hirono(非常上告-最高検察庁御中\_ツイッター)
  日時:2021-06-09 22:06/2021/06/09 03:05 URL:
  \url{https://twitter.com/kk\_hirono/status/1402613074465484803} 
  \url{https://twitter.com/s\_hirono/status/1402325876658753538} 
  \textgreater{}
  2021-06-09-021506\_7286@jmjhjmwtad·6月5日返信先: @harrier0516oskさんまぁ淡々と処理するだけですよね。なんか性犯罪の示談につい.jpg
  \url{https://t.co/fTjpOVlOe7} 
\end{itemize}

〉〉〉 kk\_hironoのリツイート 〉〉〉

\begin{itemize}
\tightlist
\item
  RT
  kk\_hirono(刑事告発・非常上告_金沢地方検察庁御中)|s\_hirono(非常上告-最高検察庁御中\_ツイッター)
  日時:2021-06-09 22:06/2021/06/09 03:05 URL:
  \url{https://twitter.com/kk\_hirono/status/1402613087551725581} 
  \url{https://twitter.com/s\_hirono/status/1402325804541878273} 
  \textgreater{}
  2021-06-09-020058\_過食B@motaberarenaiyo·5時間告訴状出しに行ったときの知能犯係かよ!.jpg
  \url{https://t.co/hr0N8SCn5P} 
\end{itemize}

〉〉〉 kk\_hironoのリツイート 〉〉〉

\begin{itemize}
\tightlist
\item
  RT
  kk\_hirono(刑事告発・非常上告_金沢地方検察庁御中)|s\_hirono(非常上告-最高検察庁御中\_ツイッター)
  日時:2021-06-09 22:06/2021/06/09 03:04 URL:
  \url{https://twitter.com/kk\_hirono/status/1402613101095124997} 
  \url{https://twitter.com/s\_hirono/status/1402325732580159490} 
  \textgreater{}
  2021-06-09-015514\_過食B@motaberarenaiyo·2時間返信先: @km0bakeさんやりましょう.jpg
  \url{https://t.co/D3tk66mnh1} 
\end{itemize}

〉〉〉 kk\_hironoのリツイート 〉〉〉

\begin{itemize}
\tightlist
\item
  RT
  kk\_hirono(刑事告発・非常上告_金沢地方検察庁御中)|s\_hirono(非常上告-最高検察庁御中\_ツイッター)
  日時:2021-06-09 22:06/2021/06/09 03:04 URL:
  \url{https://twitter.com/kk\_hirono/status/1402613113388556288} 
  \url{https://twitter.com/s\_hirono/status/1402325660647903233} 
  \textgreater{}
  2021-06-09-015338\_ふて寝べんさんがリツイート北白川@GUv4i6·6月7日これはわりとまじでそうなんじゃないかな。特に弁護士なってから、数年、自分の中で報酬と.jpg
  \url{https://t.co/NnAmQRqW6m} 
\end{itemize}

〉〉〉 kk\_hironoのリツイート 〉〉〉

\begin{itemize}
\tightlist
\item
  RT
  kk\_hirono(刑事告発・非常上告_金沢地方検察庁御中)|s\_hirono(非常上告-最高検察庁御中\_ツイッター)
  日時:2021-06-09 22:06/2021/06/09 03:04 URL:
  \url{https://twitter.com/kk\_hirono/status/1402613126336450563} 
  \url{https://twitter.com/s\_hirono/status/1402325588459720707} 
  \textgreater{}
  2021-06-09-014617\_ふて寝べん@hirune\_b·6月8日土日とか夜間もかなりヤバいけど、深夜や明け方に電話してくる人は完全に異常。.jpg
  \url{https://t.co/KySmT3RQyw} 
\end{itemize}

〉〉〉 kk\_hironoのリツイート 〉〉〉

\begin{itemize}
\tightlist
\item
  RT
  kk\_hirono(刑事告発・非常上告_金沢地方検察庁御中)|s\_hirono(非常上告-最高検察庁御中\_ツイッター)
  日時:2021-06-09 22:06/2021/06/09 03:03 URL:
  \url{https://twitter.com/kk\_hirono/status/1402613139003219972} 
  \url{https://twitter.com/s\_hirono/status/1402325516296671238} 
  \textgreater{}
  2021-06-09-014446\_ふて寝べんさんがリツイートたろう teacher@tomo\_law\_·6月8日意見を見ていると「少年、少女の健全な成長」の為に処罰を求める意.jpg
  \url{https://t.co/11hJjjBvvz} 
\end{itemize}

〉〉〉 kk\_hironoのリツイート 〉〉〉

\begin{itemize}
\tightlist
\item
  RT
  kk\_hirono(刑事告発・非常上告_金沢地方検察庁御中)|s\_hirono(非常上告-最高検察庁御中\_ツイッター)
  日時:2021-06-09 22:06/2021/06/09 01:44 URL:
  \url{https://twitter.com/kk\_hirono/status/1402613154379534340} 
  \url{https://twitter.com/s\_hirono/status/1402305435625873411} 
  \textgreater{}
  2021-06-09-014258\_ふて寝べんさんがリツイートうの字を名乗るうんち物@un\_co\_the2nd·6月8日未成年同士だったら罰しなくていいとかいう恋愛脳はとっとと.jpg
  \url{https://t.co/qAnzgFWX4C} 
\end{itemize}

〉〉〉 kk\_hironoのリツイート 〉〉〉

\begin{itemize}
\tightlist
\item
  RT
  kk\_hirono(刑事告発・非常上告_金沢地方検察庁御中)|s\_hirono(非常上告-最高検察庁御中\_ツイッター)
  日時:2021-06-09 22:06/2021/06/09 01:43 URL:
  \url{https://twitter.com/kk\_hirono/status/1402613167381835781} 
  \url{https://twitter.com/s\_hirono/status/1402305363043516419} 
  \textgreater{}
  2021-06-09-013830\_ふて寝べんさんがリツイートうの字を名乗るうんち物@un\_co\_the2nd·16時間ツッコミどころが多すぎてどこを突くか悩むレベル・・・よし、こ.jpg
  \url{https://t.co/1W9huawHDw} 
\end{itemize}

〉〉〉 kk\_hironoのリツイート 〉〉〉

\begin{itemize}
\tightlist
\item
  RT
  kk\_hirono(刑事告発・非常上告_金沢地方検察庁御中)|s\_hirono(非常上告-最高検察庁御中\_ツイッター)
  日時:2021-06-09 22:06/2021/06/09 01:43 URL:
  \url{https://twitter.com/kk\_hirono/status/1402613178647736329} 
  \url{https://twitter.com/s\_hirono/status/1402305290872115202} 
  \textgreater{}
  2021-06-09-012526\_うの字を名乗る💩物@un\_co\_the2nd·4時間そろそろ「裁判とかやらずにけしからん奴はリンチして殴り殺せばいいのでは。我々が望んでいる.jpg
  \url{https://t.co/eRs7DQ4W1t} 
\end{itemize}

〉〉〉 kk\_hironoのリツイート 〉〉〉

\begin{itemize}
\tightlist
\item
  RT
  kk\_hirono(刑事告発・非常上告_金沢地方検察庁御中)|s\_hirono(非常上告-最高検察庁御中\_ツイッター)
  日時:2021-06-09 22:06/2021/06/09 01:43 URL:
  \url{https://twitter.com/kk\_hirono/status/1402613191629148163} 
  \url{https://twitter.com/s\_hirono/status/1402305218864238600} 
  \textgreater{}
  2021-06-09-005005\_小倉秀夫@chosakukenho·2時間著作権法の専門家としていうと、ドン引きレベルの難癖ではないということなのだけど。.jpg
  \url{https://t.co/U78SXbCOmB} 
\end{itemize}

〉〉〉 kk\_hironoのリツイート 〉〉〉

\begin{itemize}
\tightlist
\item
  RT
  kk\_hirono(刑事告発・非常上告_金沢地方検察庁御中)|s\_hirono(非常上告-最高検察庁御中\_ツイッター)
  日時:2021-06-09 22:07/2021/06/09 00:45 URL:
  \url{https://twitter.com/kk\_hirono/status/1402613209861804036} 
  \url{https://twitter.com/s\_hirono/status/1402290724809314307} 
  \textgreater{}
  2021-06-08-214621\_さて、この「モラ夫バスター弁護日誌」では、実際の離婚事案などを紐解きながら、「モラ夫とは何か」「結婚とは何か」を考えていきたい。そして、結婚.jpg
  \url{https://t.co/60zwl4sOMM} 
\end{itemize}

 上記に53件のリツイートをしましたが、リツイートしていないスクリーンショットの記録がずいぶんとたまっていました。

〉〉〉 kk\_hironoのリツイート 〉〉〉

\begin{itemize}
\tightlist
\item
  RT
  kk\_hirono(刑事告発・非常上告_金沢地方検察庁御中)|yorisoibengoshi(弁護士しのだ奈保子🐸(立憲・道7区総支部長))
  日時:2021-06-09 22:08/2021/06/09 22:02 URL:
  \url{https://twitter.com/kk\_hirono/status/1402613631842340870} 
  \url{https://twitter.com/yorisoibengoshi/status/1402612083196313602} 
  \textgreater{}
  川湯温泉に行きたい。疲れた時にいつも思う。窓から斜里岳がみたい。自信を無くした時にいつも思う。私の育ちの原点だから。
  さて、週末の野外活動を楽しみに、がんばろ!
\end{itemize}

 まだ、ブロックされていないようですが、しばらく様子見でねかせておきたいと思います。

\begin{itemize}
\tightlist
\item
  〈〈〈 2021/06/09 22:09:45 Linux Emacs: 〈〈〈
\end{itemize}

\hypertarget{ux5e746ux670810ux65e50ux6642ux904eux304eux306bux9ad8ux6a4bux96c4ux4e00ux90ceux5f01ux8b77ux58ebux306eux30bfux30a4ux30e0ux30e9ux30a4ux30f3ux3067ux767aux898bux3057ux305fux56fdux9078ux5f01ux8b77ux4eba3ux4ebaux3067ux7121ux7f6aux5224ux6c7a900ux4e07ux5186ux3068ux3044ux3046ux30c4ux30a4ux30fcux30c8ux5211ux5f01ux306eux795eux69d8ux306bux81f3ux308bux767aux898b}{%
\paragraph{2021年6月10日0時過ぎに高橋雄一郎弁護士のタイムラインで発見した国選弁護人3人で無罪判決900万円というツイート〜「刑弁」の神様に至る発見}\label{ux5e746ux670810ux65e50ux6642ux904eux304eux306bux9ad8ux6a4bux96c4ux4e00ux90ceux5f01ux8b77ux58ebux306eux30bfux30a4ux30e0ux30e9ux30a4ux30f3ux3067ux767aux898bux3057ux305fux56fdux9078ux5f01ux8b77ux4eba3ux4ebaux3067ux7121ux7f6aux5224ux6c7a900ux4e07ux5186ux3068ux3044ux3046ux30c4ux30a4ux30fcux30c8ux5211ux5f01ux306eux795eux69d8ux306bux81f3ux308bux767aux898b}}

\begin{itemize}
\tightlist
\item
  〉〉〉 Linux Emacs: 2021/06/10 13:34:03 〉〉〉
\end{itemize}

:CATEGORIES: @kanazawabengosi \#金沢弁護士会 @JFBAsns
日本弁護士連合会(日弁連) \#法務省 @MOJ\_HOUMU \#高橋雄一郎弁護士
\#国選弁護 \#冤罪

 まず、スクリーンショットからご紹介したいと思います。本日未明0時9分頃の記録となっていました。

〉〉〉 kk\_hironoのリツイート 〉〉〉

\begin{itemize}
\tightlist
\item
  RT
  kk\_hirono(刑事告発・非常上告_金沢地方検察庁御中)|s\_hirono(非常上告-最高検察庁御中\_ツイッター)
  日時:2021-06-10 13:35/2021/06/10 11:38 URL:
  \url{https://twitter.com/kk\_hirono/status/1402846955496108032} 
  \url{https://twitter.com/s\_hirono/status/1402817392464457731} 
  \textgreater{}
  2021-06-10-000934\_高橋雄一郎@kamatatylaw·1時間国選弁護はノブリスオブリージュであり無償が原則、国選報酬は厳密な意味での報酬ではなく、法テラスの恩.jpg
  \url{https://t.co/e4JqDaWart} 
\end{itemize}

 次にツイートの内容です。高橋雄一郎弁護士が引用したツイートに鳥取地裁で国選弁護人3人で無罪判決900万円という内容があるのですが、結論からいってそららしい刑事裁判というのは見当たらず、副作用的な発見がいくつかありました。

※ @kk\_hironoのアカウントがブロックされ,リツイートに失敗したツイート

\begin{itemize}
\tightlist
\item
  TW kamatatylaw(高橋雄一郎) 日時:2021/06/09 22:45:54 URL:
  \url{https://twitter.com/kamatatylaw/status/1402622988558553092} 
  \textgreater{}
  国選弁護はノブリスオブリージュであり無償が原則、国選報酬は厳密な意味での報酬ではなく、法テラスの恩恵で付与されるところの、頑張った弁護士への御褒美に過ぎないので権利性はない、という説明ならいちおう辻褄はあうよ。
  \url{https://t.co/NgIKyshwQA} 
\end{itemize}

〉〉〉 kk\_hironoのリツイート 〉〉〉

\begin{itemize}
\tightlist
\item
  RT
  kk\_hirono(刑事告発・非常上告_金沢地方検察庁御中)|sk123454321(木下宗一郎【弁護士/福岡県久留米市】)
  日時:2021-06-10 13:42/2021/06/09 18:03 URL:
  \url{https://twitter.com/kk\_hirono/status/1402848748200357889} 
  \url{https://twitter.com/sk123454321/status/1402551865779232772} 
  \textgreater{} 鳥取地判平成30年4月27日
  無罪を勝ち取った国選弁護人3人が法テラスに対する報酬請求の期限14日を数日徒過した。報酬約900万円が払われなくなり,弁護人3人は法テラスを相手に請求。
  鳥取地裁は完全に請求棄却。無慈悲。
  弁護人選任され6か月経過後は報酬等の中間払請求ができること等が理由。
\end{itemize}

 Twitterには3時間と表示されていますが、滝本太郎弁護士の3つのツイートが返信になっていました。

〉〉〉 kk\_hironoのリツイート 〉〉〉

\begin{itemize}
\tightlist
\item
  RT
  kk\_hirono(刑事告発・非常上告_金沢地方検察庁御中)|takitaro2(滝本太郎)
  日時:2021-06-10 13:45/2021/06/10 10:31 URL:
  \url{https://twitter.com/kk\_hirono/status/1402849477396865034} 
  \url{https://twitter.com/takitaro2/status/1402800550077296642} 
  \textgreater{} @sk123454321
  判例サイトには「控訴」と書いておらず地裁どまりだったのですかね。高裁で和解するのが、バランス感覚というものだろうが。
  3人―報酬等請求報告書の提出を不当に遅らせたものではなく,同報告書の記載を適切に行うために準備を行っていたものの,結果的に,本件刑事事件の特殊性,担当弁護士の**
\end{itemize}

〉〉〉 kk\_hironoのリツイート 〉〉〉

\begin{itemize}
\tightlist
\item
  RT
  kk\_hirono(刑事告発・非常上告_金沢地方検察庁御中)|takitaro2(滝本太郎)
  日時:2021-06-10 13:46/2021/06/10 10:32 URL:
  \url{https://twitter.com/kk\_hirono/status/1402849520090636295} 
  \url{https://twitter.com/takitaro2/status/1402800846635618304} 
  \textgreater{} @sk123454321
  健康状態,同事件により遅延を余儀なくされた原告Bの事務所開業準備作業,裁判所との調整作業の必要性等の諸事情から,**14営業日以内に行うことが事実上不可能な状況となったもので,「やむを得ない事由」
  ⇒ とあり、実に同情すべきこと。これで払われなきゃ「国選やってらんない」だと。
\end{itemize}

〉〉〉 kk\_hironoのリツイート 〉〉〉

\begin{itemize}
\tightlist
\item
  RT
  kk\_hirono(刑事告発・非常上告_金沢地方検察庁御中)|takitaro2(滝本太郎)
  日時:2021-06-10 13:47/2021/06/10 10:35 URL:
  \url{https://twitter.com/kk\_hirono/status/1402849915772936193} 
  \url{https://twitter.com/takitaro2/status/1402801604294680577} 
  \textgreater{} @sk123454321
  まあ、法テラス契約弁護士で実質ストライキをしないのか。多数で交渉せず、裁判を始めてしまったのかな。
  ―ともあれ、日本の弁護士全体に知られるべきことだったろうが、私知らなかった。知られていたことなのか。3人は「運動」はしなかったのかな。
\end{itemize}

 私もちょうど、祭礼委員会の会計として決算報告書の作成に取り掛からなければと考えていた折りですが、報酬等請求報告書とありますが、無罪判決が出る前に準備できなかったものかのか疑問です。

 滝本太郎弁護士の上記3件のツイートは、リツイートの数がいずれも1件となっていて、一つ開いてみると私のこの告発\市場急配センター殺人未遂事件\金沢地方検察庁・石川県警察御中(@kk\_hirono)のアカウントのリツイートとなっていました。

 滝本太郎弁護士のTwitterアカウントも稀に見かけるもので、これまで余り意識することはなかったのですが、国選刑事弁護の報酬に問題意識をお持ちというのも意外でした。

\begin{quote}
《引用の始まり》
\end{quote}

\begin{quote}
滝本太郎@takitaro2市井の弁護士、「創価学会**滝本太郎」は別の人です。友人坂本一家がいなくなったからオウム真理教と対応し、何とか生き残ってます。今も足が洗えずカルト問題にも相対。一般事件もちゃんとやってます。FBはこちら。 https://facebook.com/taro.takimoto.94?ref=bookmarks・・・神奈川県大和市sky.ap.teacup.com/takitaro/2016年6月からTwitterを利用しています807
フォロー中4,672 フォロワー
\end{quote}

\begin{quote}
《引用の終わり》
\end{quote}

\begin{itemize}
\tightlist
\item
  滝本太郎さん (@takitaro2) / Twitter \url{https://twitter.com/takitaro2} 
\end{itemize}

 成り行きになりますが、プロフィールに神奈川県大和市とありました。その前にオウム真理教事件の坂本弁護士の友人とあったので、神奈川県弁護士会所属の弁護士の可能性は高そうに予想はしていました。

 神奈川県大和市は長距離トラックの仕事で行ったことがあったのですが、変わった仕事で出張していた技術者の人を浜田漁業金沢工場からポンコツの大型平ボディ車に同乗させたので、特に印象に残っています。

 東大和市は東京都だったと思いますが、平成3年のちょうどクリスマスイブの頃にベニアを運んだ仕事がありました。

 平成3年のクリスマスイブの日は、朝に池袋から練馬区の方に向かって、首都高に乗り、そこから茨城県古河市の山三青果に向かったような記憶もあります。

 市場急配センターで朝に池袋で荷降ろしをしたのは、現在の記憶の中で一度だけなのですが、これは平成4年1月13日の月曜日のこととはっきりしています。12日の土曜日に片山津温泉のせきやで一泊の新年会があり、翌日の日曜日にトナミ航空で展示会の荷物を積み込んだのでよく憶えています。

 前にも書いていると思いますが、早朝に池袋の三越百貨店で荷降ろしをしたあと、広い道路をしばらく走った後、路上駐車で公衆電話から会社に電話をしたのですが、そのときに見たと記憶にある都会の街の風景が、高橋雄一郎弁護士のヘッダ画像とよく似ています。

 高橋雄一郎弁護士のTwitterアカウントは、最初に見た頃からヘッダ画像の写真が変わっていないように思うのですが、今見ると道路の道幅が思っていたより狭く感じられました。アイコンの顔写真はずいぶん前に変更になっています。

 高橋雄一郎弁護士は弁護士として独自の理想と世界観を持っていると強く感じる、個性の強いTwitterの弁護士で、ツイートの内容のメッセージ性も強く感じております。そんな長年の積み重ねの中での昨夜の発見でした。

 昨夜、鳥取地裁の国選絡みの判決を調べていると、氷見強姦冤罪事件でジャーナリストの江川紹子氏の名前のある資料のような記事が出てきて、他の検索結果にも見かけていたのですが、真犯人の自供が鳥取県米子市とありました。

 鳥取県で鳥取県警ということは早い段階からニュースなどで知っていたのですが、鳥取市の方を思い浮かべイメージしていたので米子市というのは意外でした。このあと島根県の話をすることになりますが、いずれも国道9号線沿いで、長距離トラックの仕事でよく行っていました。

 他に鳥取地裁と何の関係があるのか不思議だったのですが、検索結果に2件ほど、岡山市の弁護士が国選弁護の不正請求をしたというニュースがあり、これも印象的によく憶えているニュースなのですが、元検事の弁護士で、それも司法試験合格者ではなく検察事務官からという異色の経歴でした。

 今はどうなっているのか詳しく調べていないですが、昭和の終わりから平成の初めの鳥取県米子市というのは、国道9号線沿いでも道幅が広く繁華街の中心地のようになっていました。しばらく行くと島根県に入りましたが、それが奇石で印象に残る島根県安来市でした。

 鳥取県米子市から島根県出雲市の間というのは、国道9号線でも昼はけっこう混み合って通過するのも夜間に比較すると倍ぐらいは時間が掛かっていたように思います。九州に向かうときはほとんど午後から夕方にかけて通行していました。

 金沢刑務所の拘置所で読んだ佐木隆三の「闇の中の光」という冤罪の本があって、今朝、奉納\さらば弁護士鉄道・泥棒神社の物語(@hirono\_hideki)のTwilogで調べると検索結果がゼロで驚いたのですが、キーワードを変更しながら調べていくと、自分の作成したまとまった記事を発見しました。

\begin{itemize}
\tightlist
\item
  参考資料/再審/2015年大阪の同居少女強姦無罪判決/【西論】「魂の殺人」に目が曇ったか・・・司法の大失態、「正義の危うさ」自覚せよ - 産経WEST
  - 告発\金沢地方検察庁\最高検察庁\法務省\石川県警察御中
  \url{https://t.co/eWtWd7bMoN} 
\end{itemize}

 ざっと目を通しただけでも刑事補償の金額や弁護士に対する1000万円の謝礼など記憶になかった記述もあったのですが、見出しには含まれていないものの山中事件のことも取り上げているようです。全部に目を通しているわけではないですが、映画「砂の器」が接点になっているはずです。

 改めて記事の日付を確認すると2019年1月7日となっていました。はてなブログでページ内の見出しと目次をつけていた時期のものですが、目次に4つある項目には、島根の事件のことも山中事件のことも見当たらず、脱線しながら書き記していたようです。

※ @kk\_hironoのアカウントがブロックされ,リツイートに失敗したツイート

\begin{itemize}
\tightlist
\item
  TW
  yjochi(弁護士落合洋司🌸感染拡大を招く東京(頭狂)オリンピック中止!🌸)
  日時:2018/05/26 17:33:41 URL:
  \url{https://twitter.com/yjochi/status/1000293886164586496} 
  \textgreater{} むかし児島惟謙、いま岡口基一(違)。
\end{itemize}

 上記の落合洋司弁護士(東京弁護士会)のツイートがきっかけで、児島惟謙→勝ちを制するに至れり、という本→作者の佐木隆三→闇の中の光→石見町女児殺人事件という流れになったようです。

 落合洋司弁護士(東京弁護士会)のツイートに児島惟謙の名前があったのも記憶がなく、それを落合洋司弁護士(東京弁護士会)が岡口基一裁判官と肩を並べる司法の人物のようにツイートしていました。

 児島惟謙という大審院院長の名前を知る人は少ないと思いますが、昨日の夕方に、津田三蔵巡査のことを思い出していました。大津事件と呼ばれ、歴史の教科書にも出ているのかもしれません。どれぐらい前になるのか思い出せないですが、ネットで津田三蔵巡査の日誌や手紙のようなものを読みました。

 津田三蔵巡査が金沢の軍隊にいたことと、西南戦争に従軍していたことを昨日の夕方に思い出していました。

 吾郷計宜というのが探していた弁護士の名前になりますが、このはてなブログの記事に辿り着く少し前に見つけていたかもしれません。再度、吾郷計宜弁護士をGoogleで調べ直したところ、出てきたのがこのはてなブログの記事であったように思います。午前中のことですが、正確には思い出せません。

 「一審判決が出たのは平成2年3月15日、初公判から8年半、弁護費用の補償としてとして裁判所が認めたのは91万5千円とのことです。」とあります。8年半というのも忘れていたのですが、典型的な冤罪事件とされながら、ネットで情報を見かけることはまずなかったと思います。

 金沢刑務所の拘置所で購読していた北國新聞に「闇の中の光」という本の広告があったからの発見であり、出会いでした。泥酔で記憶が飛び、警察に真犯人であるという自白をさせられていた、初公判で否認したという話になっていたかと思います。

\begin{itemize}
\tightlist
\item
  石見町女児殺人事件 - Enpedia \url{https://t.co/rV57MmDOCe} 
\end{itemize}

 上記のはてなブログの記事にあったリンクですが、午前中に最初に見つけて読んだWikipediaのページとは、内容に違いがありました。Enpediaを見かけたのも久しぶりに思ったのですが、表示が違うだけで内容はWikipediaと同じだと思っていました。

〉〉〉 kk\_hironoのリツイート 〉〉〉

\begin{itemize}
\tightlist
\item
  RT
  kk\_hirono(刑事告発・非常上告_金沢地方検察庁御中)|hirono\_hideki(奉納\さらば弁護士鉄道・泥棒神社の物語)
  日時:2021-06-10 15:21/2021/06/10 11:54 URL:
  \url{https://twitter.com/kk\_hirono/status/1402873478156455939} 
  \url{https://twitter.com/hirono\_hideki/status/1402821489456418817} 
  \textgreater{} - 石見町女児殺人事件 - Wikipedia
  \url{https://t.co/k7Vm3Y7Xzz} 
\end{itemize}

 邑南町をGoogleマップで見ると、国道261号線が島根県江津市から広島県北広島町に通じているようです。北広島町と町がつくのは意外にも感じたのですが、中国自動車道に北広島というインターがあったと思います。昭和60年のはじめ頃は、その北広島インターで降りて広島市内に入っていました。

 広島県三好市の中国自動車道三好インターから島根県松江市には国道54号線がありますが、この国道は広島県の方から松江市に向かったことがあったものの、道幅は広いもののドライブインがありそうな道路には思えなかったことが印象に残っていて、ちょうど山口県の津和野市の辺りと似ていました。

\begin{itemize}
\tightlist
\item
  石見銀山 から 島根県邑智郡邑南町 - Google マップ
  \url{https://t.co/CgvTIQey2Z} 
\end{itemize}

 石見銀山は人気で有名な観光地らしくテレビの旅番組でも見たことがあって調べてみたのですが、邑南町とはずいぶん離れていました。33.9kmとあります。

 観光地であれば交通量が多いことも考えたのですが、寝ていた寝室から女児が連れ去られたという話は他に聞いたことがなく、弁護士の主張通り、泥酔状態で犯行は不可能というのであれば、近くのガソリンスタンドで寝ていたことを含め、謎めいた事件です。

 仮眠をしていたトラック運転手なのかと思ったのですが、翌日の捜索に加わっていたとあるので地元民なのでしょう。靴跡が強姦事件の決め手となり、その後、無罪となったことも氷見強姦冤罪事件と似ています。氷見の場合は、靴跡の大きさが違っていたという話だったかもしれません。

 事件は1981年とあるので昭和56年ですが、昭和50年代というのは奥能登でも廃墟になりかけた古いドライブインや廃墟となったドライブインの建物もちらほらとあったような記憶で、当時の記憶の風景とも重なるところがありますが、それにしても謎の多い事件です。

 ネットの情報というのはこのWikipediaのページがほとんどではと思いますが、乏しい情報で、これは石川県珠洲市の蛸島事件とも似ていると思ったのですが、蛸島事件については、今年に入って「石川県奥能登でおきた蛸島学童殺人事件裁判記録」を読んだことでいっきに判断材料が増えました。

\begin{itemize}
\tightlist
\item
  CiNii 論文~-~ 代用監獄の病巣--虚偽自白の集積-11-幼女強姦殺人事件
  \url{https://t.co/MOZZbxDDTo} 
\end{itemize}

 吾郷計宜弁護士の名前だけがある上記の「CiNii 論文~-~
代用監獄の病巣--虚偽自白の集積-11-幼女強姦殺人事件」という論文のようなものが存在することは数年前からわかっていたのですが、門外不出で素人には入手が困難と思われる資料です。

 吾郷計宜弁護士の名前はずいぶん珍しい名前だと思ったのですが、ほとんど記憶に残らずにいたのも不思議に思え、計算の計という漢字が名前にあるのもとてもめずらしく感じました。

 今朝になって、鳥取地裁と無罪判決の組み合わせで調べたのですが、過去に見覚えのある事件が出てきて、一審で懲役18年、控訴審で無罪、最高裁で差し戻しとなり、差し戻し審で無期懲役判決が出たというところまで調べました。鳥取地裁となっていたように思います。

 無罪判決を出した控訴審が、広島高裁松江支部というのも意外に感じ、松江支部というのは余り聞いたことがないように思いました。鳥取地裁米子支部というのも聞いた覚えがないのですが、鳥取県で地方裁判所が1つだけとも考えにくく、他に裁判所がありそうなのは米子ぐらいと思います。

 前に鳥取県鳥取市と米子市の人口を調べたことがあったのですが、やはり米子市の方が多くなっていました。この関係性というのは山口県の山口市と下関市にも似ているのですが、現地での実感でした。県庁所在地より大きい市があるというのは、この鳥取と山口以外になかったと思います。

 今確認のため調べたところ、米子市が14.8万人、鳥取市が19.37万人と鳥取市の方が多くなっていました。鳥取市内は岡山方面の出入りに中心部を通過しているはずですが、大きな街には思えませんでした。

 ネットにある他のデータをみると200年代に入ってから鳥取市の人口が15万人台から20万人台に一気に増えていました。これは市町村合併だと考えられます。もともと土地は広そうな鳥取市で、国道9号線を走ると、遠くから砂漠の向こうにある町に見えたものです。

\begin{itemize}
\tightlist
\item
  管内の裁判所の所在地 \textbar{} 裁判所 \url{https://t.co/Rib4YhBgwn} 
\end{itemize}

 鳥取地裁は米子支部がありましたが、他に倉吉支部もありました。この倉吉市というのも一度は、長距離トラックの仕事で通過したことがあったのですが、なにか近くに幻想的な温泉街を夜に通過したような記憶があり、印象に残っています。倉吉という地名が記憶にあったのも卯辰山の相撲大会でした。

 米子で他に思い出すのは、弁護士鉄道の歴史になりますが、法クラを喜ばせた名言が判例にあったことです。隣に境港市があって、私が知ったのは、あるいは平成10年代になっていたかもしれないですが、ゲゲゲの鬼太郎の作者、水木しげるで有名です。ここで弁護士と妖怪が急接近しました。

\begin{lstlisting}
base ❯ d|grep いやらしいこと    
- 2017年10月10日18時51分の登録: %@uwaaaa サイ太%「あの人は好かんわ。いやらしいことばかりする。」の判例(最判S30.12.9)って,犯人性の証明に疑義があるとして差し戻してるのか \url{http://hirono2014sk.blogspot.com/2017/10/uwaaaas30129.html} 
- 2017年10月10日18時54分の登録: \仙猫カリン@ラグビー1列目 @Bibendum65 RT: @Qu2_law\「あの人は好かんわ・・・(女声)」 「いやらしいことばかりする・・・(女声)・・・。あっ、しまっ・・・」 「連れて行け」  #法クラ狩り \url{http://hirono2014sk.blogspot.com/2017/10/1bibendum65rtqu2law.html} 
- 2017年12月25日21時44分の登録: \サイ太 @uwaaaa\今度から「あの人はすかんわ,いやらしいことばかりするんだ」事件と,ちゃんと呼称するようにしましょう。 \url{http://hirono2014sk.blogspot.com/2017/12/uwaaaa_43.html} 
- 2017年12月28日20時32分の登録: \渡辺輝人 @nabeteru1Q78\「あのひとは好かんわ、いやらしいことばかりする」は、我が国の刑事訴訟法(証拠法の分野)を解釈する上で、最も有名かつ重要な文言だと言 \url{http://hirono2014sk.blogspot.com/2017/12/nabeteru1q78_1.html} 
- 2019年02月23日19時11分の登録: \とろろ @lit_soc\えきなん先生昨日から「いやらしい話」ツイートが続いているので、「あの人は好かんわ、いやらしいことばかりする」を地でいってる \url{http://hirono2014sk.blogspot.com/2019/02/litsoc.html} 
\end{lstlisting}

 「石川県奥能登でおきた蛸島学童殺人事件裁判記録」がきっかけで調べ直し、気がついたように思いますが、鳥取県に近い兵庫県の海沿いで余部鉄橋の辺りをイメージしていたのですが、調べ直すと米子市の皆生温泉だったとうテレビドラマがあって、共通したのが、女性の行商でした。

〉〉〉 kk\_hironoのリツイート 〉〉〉

\begin{itemize}
\tightlist
\item
  RT
  kk\_hirono(刑事告発・非常上告_金沢地方検察庁御中)|hirono\_hideki(奉納\さらば弁護士鉄道・泥棒神社の物語)
  日時:2021-06-10 16:43/2021/06/10 11:02 URL:
  \url{https://twitter.com/kk\_hirono/status/1402894075225796613} 
  \url{https://twitter.com/hirono\_hideki/status/1402808347594690565} 
  \textgreater{} 【裁判】米子強盗殺人、懲役18年の1審破棄し逆転無罪判決
  \url{https://t.co/Njmj8jt7EB} 
\end{itemize}

〉〉〉 kk\_hironoのリツイート 〉〉〉

\begin{itemize}
\tightlist
\item
  RT
  kk\_hirono(刑事告発・非常上告_金沢地方検察庁御中)|hirono\_hideki(奉納\さらば弁護士鉄道・泥棒神社の物語)
  日時:2021-06-10 16:43/2021/06/10 11:00 URL:
  \url{https://twitter.com/kk\_hirono/status/1402894097988296705} 
  \url{https://twitter.com/hirono\_hideki/status/1402807881481691139} 
  \textgreater{}
  差し戻し審で無期懲役 ホテル支配人強殺―鳥取地裁:時事ドットコム
  \url{https://t.co/q0l2deDN3Z}  2020年11月30日19時12分
\end{itemize}

〉〉〉 kk\_hironoのリツイート 〉〉〉

\begin{itemize}
\tightlist
\item
  RT
  kk\_hirono(刑事告発・非常上告_金沢地方検察庁御中)|hirono\_hideki(奉納\さらば弁護士鉄道・泥棒神社の物語)
  日時:2021-06-10 16:43/2021/06/10 10:56 URL:
  \url{https://twitter.com/kk\_hirono/status/1402894124198498305} 
  \url{https://twitter.com/hirono\_hideki/status/1402806725447614465} 
  \textgreater{}
  ホテル強盗殺人事件、被告に無罪判決 広島高裁松江支部:朝日新聞デジタル
  \url{https://t.co/b6APFWwQ3P} 
  判決が27日、広島高裁松江支部であった。栂村明剛(つがむらあきよし)裁判長は懲役18年とした一審・鳥取地裁の判決は「事実誤認がある」とし、有罪と結論づけた主要部分を破棄し、無罪を言い渡
\end{itemize}

〉〉〉 kk\_hironoのリツイート 〉〉〉

\begin{itemize}
\tightlist
\item
  RT
  kk\_hirono(刑事告発・非常上告_金沢地方検察庁御中)|hirono\_hideki(奉納\さらば弁護士鉄道・泥棒神社の物語)
  日時:2021-06-10 16:43/2021/06/10 10:55 URL:
  \url{https://twitter.com/kk\_hirono/status/1402894147929853955} 
  \url{https://twitter.com/hirono\_hideki/status/1402806590550339591} 
  \textgreater{}
  ホテル強盗殺人事件、被告に無罪判決 広島高裁松江支部:朝日新聞デジタル
  \url{https://t.co/b6APFWwQ3P} 
  判決後、弁護人の吉岡伸幸弁護士は「よく判断していただいた」と評価し、広島高検の玉置俊二次席検事は「判決内容を詳細に検討したうえ、上級庁とも協議し、適切に対応したい」との談話を出した。
\end{itemize}

 ホテル支配人強殺事件ともありますが、本日はラブホテルだったという情報も一つ見かけました。ラブホテルは従業員と客の接触を極力避けると聞いたことがあり、接客業を束ねる支配人とはイメージが異なるのですが、皆生温泉の可能性を考え始めていた矢先のことでした。

 長距離トラックの仕事で皆生温泉という標識はよく見かけたという記憶があったのですが、数年前に米子市にあると知ったときは驚いたように思います。他の温泉地と取り違えていた可能性はありますが、鳥取県でも兵庫県に近く海に近いと考えていたようです。

 時刻は17時00分です。Googleマップのストリートビューで、島根県安来市から松江市に向かっていたつもりだったのですが、かなり前方に進めた後、松江市内に入ったのかと思ったら、米子市内で右折で境港方面という道路標識が出ていました。町並みは昭和の終わり頃とはまるで違い、何も残っていないと思えるぐらいです。

\begin{itemize}
\tightlist
\item
  米子トラックステーション - Google マップ \url{https://t.co/obna9LfJFE} 
\end{itemize}

 Googleマップというかストリートビューでは初めての経験になると思いますが、何度見てもセブンイレブンのコンビニとワークマンとかいう作業着店しか見えず、タイムラインを2012年にすると、同じ場所に米子のトラックステーションが出てきました。他のトラックステーションより建物が小さいのが特徴でした。

 今日はずいぶん暑いので30度を超えているのかもしれません。

 スマホで能登町をみると24度でしたが、精度を疑いたくなる暑さです。扇風機の用意も必要に思えてきました。

 昼は出雲のそばというのを食べたのですが、賞味期限が後4日の6月14日で、今年の正月にどんたく宇出津店で半額になっていたのを買ってきて、そのままずっと冷蔵庫に入れていました。2食入りなのでもう1食分残っているのですが、なかなか食べずにいました。

〉〉〉 kk\_hironoのリツイート 〉〉〉

\begin{itemize}
\tightlist
\item
  RT
  kk\_hirono(刑事告発・非常上告_金沢地方検察庁御中)|s\_hirono(非常上告-最高検察庁御中\_ツイッター)
  日時:2021-06-10 17:26/2021/06/10 17:25 URL:
  \url{https://twitter.com/kk\_hirono/status/1402904918546022410} 
  \url{https://twitter.com/s\_hirono/status/1402904817635184647} 
  \textgreater{} 2021-06-10 12.38.40.jpg \url{https://t.co/gDY5pH9JvW} 
\end{itemize}

 出雲のそばというだけで余り見ないで買ったのですが、「出雲十割縁起そば」とあり、980円が半額の490円となっていました。味に劣化は感じなかったのですが、早めに食べた方が美味しかったのかもしれません。最初からなかなか手を付ける気にならず、賞味期限に近くづことはなんとなく予想していました。

 昨夜、高橋雄一郎弁護士のタイムラインで見かけたツイートですが、国選弁護人が3人だったという話も珍しく、前にも書いていると思いますが、国選弁護で弁護士が複数になったとニュースで聞いたのは今一女児殺害事件(のちに栃木女児殺害事件)ぐらいでした。その上、無罪判決で、弁護士費用が900万円、深刻遅れでご破算になったとも。

 鳥取地裁とありましたが、鳥取県といえば全国で最も人口の少ない都道府県になっていたはずと思います。人口が少ないだけに重大事件の発生率や、無罪判決になる確率も低そうですが、それよりなにより、探しても情報が見つからないというのがミステリーです。

 これまでにも何度か見覚えのあるアカウントのツイートでしたが、実名で弁護士となっていました。プロフィールをみて久留米の弁護士というのがいくらか意外な情報には感じたのですが、弁護士が実名ででたらめな裁判例を紹介するとは考えにくいところです。

 今一度、鳥取地裁、無罪判決で検索を行ってみます。

 検索結果の6ページ目に「も的確な検索結果を表示するために、上の 58
件と似たページは除外されています。検索結果をすべて表示するには、ここから再検索してください。」と出てきました。大津地裁の無罪判決も出てきたのですが、鳥取地裁の無罪判決はありませんでした。

 3人の弁護士の取りっぱぐれた国選刑事弁護の費用が900万円というのも、どういう計算で900万円になったのかとても気になるのですが、3人ということで思い出したのが、飯降山とかいう日本昔ばなしです。テレビで最後の放送になったという情報も見かけました。3人の弁護士による伝説のようなお話です。

 引用ツイートが多いことに気が付きました。埋め込みツイートのため稿を改めることにします。

\begin{itemize}
\tightlist
\item
  〈〈〈 2021/06/10 18:03:34 Linux Emacs: 〈〈〈
\end{itemize}

\hypertarget{ux5e746ux670810ux65e50ux6642ux904eux304eux306bux9ad8ux6a4bux96c4ux4e00ux90ceux5f01ux8b77ux58ebux306eux30bfux30a4ux30e0ux30e9ux30a4ux30f3ux3067ux767aux898bux3057ux305fux56fdux9078ux5f01ux8b77ux4eba3ux4ebaux3067ux7121ux7f6aux5224ux6c7a900ux4e07ux5186ux3068ux3044ux3046ux30c4ux30a4ux30fcux30c8ux5211ux5f01ux306eux795eux69d8ux306bux81f3ux308bux767aux898bux5f15ux7528ux30c4ux30a4ux30fcux30c8ux306eux307eux3068ux3081}{%
\paragraph{2021年6月10日0時過ぎに高橋雄一郎弁護士のタイムラインで発見した国選弁護人3人で無罪判決900万円というツイート〜「刑弁」の神様に至る発見:引用ツイートのまとめ}\label{ux5e746ux670810ux65e50ux6642ux904eux304eux306bux9ad8ux6a4bux96c4ux4e00ux90ceux5f01ux8b77ux58ebux306eux30bfux30a4ux30e0ux30e9ux30a4ux30f3ux3067ux767aux898bux3057ux305fux56fdux9078ux5f01ux8b77ux4eba3ux4ebaux3067ux7121ux7f6aux5224ux6c7a900ux4e07ux5186ux3068ux3044ux3046ux30c4ux30a4ux30fcux30c8ux5211ux5f01ux306eux795eux69d8ux306bux81f3ux308bux767aux898bux5f15ux7528ux30c4ux30a4ux30fcux30c8ux306eux307eux3068ux3081}}

\begin{itemize}
\tightlist
\item
  〉〉〉 Linux Emacs: 2021/06/10 18:07:47 〉〉〉
\end{itemize}

:CATEGORIES: @kanazawabengosi \#金沢弁護士会 @JFBAsns
日本弁護士連合会(日弁連) \#法務省 @MOJ\_HOUMU \#国選弁護 \#法テラス

〉〉〉 kk\_hironoのリツイート 〉〉〉

\begin{itemize}
\tightlist
\item
  RT
  kk\_hirono(刑事告発・非常上告_金沢地方検察庁御中)|zaccozacco1(zacco-zacco)
  日時:2021-06-10 18:09/2021/06/10 17:42 URL:
  \url{https://twitter.com/kk\_hirono/status/1402915791511511048} 
  \url{https://twitter.com/zaccozacco1/status/1402908907018735616} 
  \textgreater{}
  これってもしかしたら報酬だけでなくて立替費用も含んでの金額では・・・伝え聞くところでは国選報酬だけでこんな金額にならなさそうだし・・・業界の人でないから知らんけど・・・
  \url{https://t.co/37IjU6JUat} 
\end{itemize}

〉〉〉 kk\_hironoのリツイート 〉〉〉

\begin{itemize}
\tightlist
\item
  RT
  kk\_hirono(刑事告発・非常上告_金沢地方検察庁御中)|masskiii0326(すぎゃ)
  日時:2021-06-10 18:09/2021/06/10 17:22 URL:
  \url{https://twitter.com/kk\_hirono/status/1402915816966742019} 
  \url{https://twitter.com/masskiii0326/status/1402903956980781058} 
  \textgreater{} これはひどい。
  そもそも期限が短すぎるし、期限の趣旨から言っても活動内容がはっきりわかってれば14日過ぎても問題ないはず。
  判決文読んでみよかな。 \url{https://t.co/N6WYhbUDxP} 
\end{itemize}

〉〉〉 kk\_hironoのリツイート 〉〉〉

\begin{itemize}
\tightlist
\item
  RT
  kk\_hirono(刑事告発・非常上告_金沢地方検察庁御中)|\_devilsadvocate(弁護士ばやし(若林翔))
  日時:2021-06-10 18:09/2021/06/10 15:24 URL:
  \url{https://twitter.com/kk\_hirono/status/1402915843126665219} 
  \url{https://twitter.com/\_devilsadvocate/status/1402874321639407623} 
  \textgreater{} 法テラス、無慈悲すぎる・・・😭 \url{https://t.co/FW6YcMff07} 
\end{itemize}

〉〉〉 kk\_hironoのリツイート 〉〉〉

\begin{itemize}
\tightlist
\item
  RT
  kk\_hirono(刑事告発・非常上告_金沢地方検察庁御中)|inyo\_rt\_geinin(RT芸人)
  日時:2021-06-10 18:09/2021/06/10 12:50 URL:
  \url{https://twitter.com/kk\_hirono/status/1402915876647555083} 
  \url{https://twitter.com/inyo\_rt\_geinin/status/1402835489875398665} 
  \textgreater{} 法テラスはクソ \url{https://t.co/Jmma274cTP} 
\end{itemize}

〉〉〉 kk\_hironoのリツイート 〉〉〉

\begin{itemize}
\tightlist
\item
  RT
  kk\_hirono(刑事告発・非常上告_金沢地方検察庁御中)|harrier0516osk(向原総合法律事務所 弁護士向原)
  日時:2021-06-10 18:09/2021/06/10 11:35 URL:
  \url{https://twitter.com/kk\_hirono/status/1402915904548085761} 
  \url{https://twitter.com/harrier0516osk/status/1402816548721418240} 
  \textgreater{}
  新人研修で法テラス契約させるわけですが、情報の非対称性が解消されているとはいえない段階での契約誘導は不公正であると考えます。
  こうした、契約通りにしか扱ってもらえない現実もキチンと教示してから、契約するかどうか判断させるべきだと思います。
  \url{https://t.co/CM7uHP45FJ} 
\end{itemize}

〉〉〉 kk\_hironoのリツイート 〉〉〉

\begin{itemize}
\tightlist
\item
  RT
  kk\_hirono(刑事告発・非常上告_金沢地方検察庁御中)|idleness\_venomy(venomy)
  日時:2021-06-10 18:10/2021/06/10 10:19 URL:
  \url{https://twitter.com/kk\_hirono/status/1402915998903144448} 
  \url{https://twitter.com/idleness\_venomy/status/1402797494573027328} 
  \textgreater{}
  なぜ報道されず、めぼしい判例検索サイトにも掲載されていないのだろう。。。こんなものは、判決内容の当否はさておくとしても、批判の対象として晒されるべきものでしょう。(鳥取地裁に行って閲覧するのはしんどい)
  \url{https://t.co/xBqUvoQGim} 
\end{itemize}

〉〉〉 kk\_hironoのリツイート 〉〉〉

\begin{itemize}
\tightlist
\item
  RT
  kk\_hirono(刑事告発・非常上告_金沢地方検察庁御中)|1023kokuto(こくとー@誤字王)
  日時:2021-06-10 18:10/2021/06/10 10:08 URL:
  \url{https://twitter.com/kk\_hirono/status/1402916036773498880} 
  \url{https://twitter.com/1023kokuto/status/1402794651849949184} 
  \textgreater{} これ日弁とか各会から声明でないのかなぁ。
  国選や法テラスの案件の担い手いなくなるよ。 \url{https://t.co/4qydOCFUVz} 
\end{itemize}

〉〉〉 kk\_hironoのリツイート 〉〉〉

\begin{itemize}
\tightlist
\item
  RT
  kk\_hirono(刑事告発・非常上告_金沢地方検察庁御中)|fxatty(中川@モンハンやってます)
  日時:2021-06-10 18:10/2021/06/10 09:04 URL:
  \url{https://twitter.com/kk\_hirono/status/1402916129094344704} 
  \url{https://twitter.com/fxatty/status/1402778762274435074} 
  \textgreater{}
  弁護士ではなく一人のビジネスパーソンというか、大人としていうとだな、期限を徒過する方がいけないと思うが、、、
  引用リツイートを見ると、この点を責める弁護士が少ないのが、弁護士が時間にルーズと言われる一因を担っている気がする。
  \url{https://t.co/FyPwi0psoO} 
\end{itemize}

〉〉〉 kk\_hironoのリツイート 〉〉〉

\begin{itemize}
\tightlist
\item
  RT
  kk\_hirono(刑事告発・非常上告_金沢地方検察庁御中)|butasan\_tyokin(豚さん貯金箱(こし餡研究所
  首席研究員)) 日時:2021-06-10 18:10/2021/06/10 08:44 URL:
  \url{https://twitter.com/kk\_hirono/status/1402916160178294786} 
  \url{https://twitter.com/butasan\_tyokin/status/1402773694343114753} 
  \textgreater{} 法はね 無慈悲なんよ 誰に対しても平等なんや
  (やべぇなこりゃ) \url{https://t.co/xlGqEn4cBy} 
\end{itemize}

〉〉〉 kk\_hironoのリツイート 〉〉〉

\begin{itemize}
\tightlist
\item
  RT
  kk\_hirono(刑事告発・非常上告_金沢地方検察庁御中)|sho\_ya(shoya)
  日時:2021-06-10 18:10/2021/06/10 06:41 URL:
  \url{https://twitter.com/kk\_hirono/status/1402916190968635395} 
  \url{https://twitter.com/sho\_ya/status/1402742652152340481} 
  \textgreater{}
  事案や規定、そして判決の詳細を存じ上げないのですが、結論だけを見ると違和感を感じる判決ではありますね。
  \url{https://t.co/uM8x4tQc1d} 
\end{itemize}

〉〉〉 kk\_hironoのリツイート 〉〉〉

\begin{itemize}
\tightlist
\item
  RT kk\_hirono(刑事告発・非常上告_金沢地方検察庁御中)|ailuv2u(愛)
  日時:2021-06-10 18:11/2021/06/10 06:15 URL:
  \url{https://twitter.com/kk\_hirono/status/1402916263551139841} 
  \url{https://twitter.com/ailuv2u/status/1402736020483805184} 
  \textgreater{} 醜い \#法テラス \url{https://t.co/3HDtt0kvLV} 
\end{itemize}

〉〉〉 kk\_hironoのリツイート 〉〉〉

\begin{itemize}
\tightlist
\item
  RT
  kk\_hirono(刑事告発・非常上告_金沢地方検察庁御中)|H\_Tomoko\_\_(Tomoko)
  日時:2021-06-10 18:11/2021/06/10 00:57 URL:
  \url{https://twitter.com/kk\_hirono/status/1402916337953894403} 
  \url{https://twitter.com/H\_Tomoko\_\_/status/1402656041364430849} 
  \textgreater{}
  「ひどい」という弁護士の引用リツイートが多いけど、自分の都合で決まりを破った後に、その決まりがおかしいと主張する方がおかしい。
  契約、約束ってなんなんですか?
  弁護士が法を無視することを平気で言うとは恐ろしい。
  他人に厳しく自分に甘いのか。 \url{https://t.co/67H9aBQqyO} 
\end{itemize}

〉〉〉 kk\_hironoのリツイート 〉〉〉

\begin{itemize}
\tightlist
\item
  RT
  kk\_hirono(刑事告発・非常上告_金沢地方検察庁御中)|dmmomoko0718(🎗️momo🍿)
  日時:2021-06-10 18:11/2021/06/10 00:29 URL:
  \url{https://twitter.com/kk\_hirono/status/1402916374645665795} 
  \url{https://twitter.com/dmmomoko0718/status/1402649037723947013} 
  \textgreater{} 法テラスは誰が運営してるの?国?日弁連?
  \url{https://t.co/rwoyRsCYnL} 
\end{itemize}

〉〉〉 kk\_hironoのリツイート 〉〉〉

\begin{itemize}
\tightlist
\item
  RT
  kk\_hirono(刑事告発・非常上告_金沢地方検察庁御中)|hibi\_kian(ビー玉)
  日時:2021-06-10 18:11/2021/06/09 23:51 URL:
  \url{https://twitter.com/kk\_hirono/status/1402916430870286340} 
  \url{https://twitter.com/hibi\_kian/status/1402639561021267977} 
  \textgreater{}
  これ一般の私人間取引でも同じ判断になるの?「14営業日以内に疎明資料を付して書面により報酬請求しない場合は、対価の請求権を失う」なんて条項が有効?
  \url{https://t.co/EfBJCFkRpu} 
\end{itemize}

〉〉〉 kk\_hironoのリツイート 〉〉〉

\begin{itemize}
\tightlist
\item
  RT kk\_hirono(刑事告発・非常上告_金沢地方検察庁御中)|GEHAO6(T)
  日時:2021-06-10 18:12/2021/06/09 23:45 URL:
  \url{https://twitter.com/kk\_hirono/status/1402916463216762882} 
  \url{https://twitter.com/GEHAO6/status/1402637965147336704} 
  \textgreater{} 無罪で900万て、私選より高くないですか?
  \url{https://t.co/lWlha1FFrn} 
\end{itemize}

〉〉〉 kk\_hironoのリツイート 〉〉〉

\begin{itemize}
\tightlist
\item
  RT
  kk\_hirono(刑事告発・非常上告_金沢地方検察庁御中)|azukariguchi(弁護士の預り口)
  日時:2021-06-10 18:12/2021/06/09 23:44 URL:
  \url{https://twitter.com/kk\_hirono/status/1402916492908191744} 
  \url{https://twitter.com/azukariguchi/status/1402637692773429252} 
  \textgreater{} これって控訴審どうなったんだろ。
  \url{https://t.co/zuy4F6Prl9} 
\end{itemize}

〉〉〉 kk\_hironoのリツイート 〉〉〉

\begin{itemize}
\tightlist
\item
  RT
  kk\_hirono(刑事告発・非常上告_金沢地方検察庁御中)|TTMKurihara(弁護士
  栗原 務) 日時:2021-06-10 18:12/2021/06/09 23:38 URL:
  \url{https://twitter.com/kk\_hirono/status/1402916539020365829} 
  \url{https://twitter.com/TTMKurihara/status/1402636219129892866} 
  \textgreater{} マジか。酷いな。。 いろいろと、、
  こんなんじゃ、法テラス案件はやりたくないね。
  やったことないけど。やらないし。 \url{https://t.co/OfCzZsYsgw} 
\end{itemize}

〉〉〉 kk\_hironoのリツイート 〉〉〉

\begin{itemize}
\tightlist
\item
  RT
  kk\_hirono(刑事告発・非常上告_金沢地方検察庁御中)|MAKOTOMurakami7(飛びたいブタ🐷は・・・・・・)
  日時:2021-06-10 18:12/2021/06/09 23:22 URL:
  \url{https://twitter.com/kk\_hirono/status/1402916611711897605} 
  \url{https://twitter.com/MAKOTOMurakami7/status/1402632116622413836} 
  \textgreater{} 法テラスに対する報酬請求権は委任契約に基づくもの。
  そうすると、消滅時効は5年間のはず。
  時効期間って、契約で如何ようにも短縮できるんだっけ?もう、忘れちった・・・・・・
  \url{https://t.co/PGwHqyhfFC} 
\end{itemize}

〉〉〉 kk\_hironoのリツイート 〉〉〉

\begin{itemize}
\tightlist
\item
  RT
  kk\_hirono(刑事告発・非常上告_金沢地方検察庁御中)|Jisyou\_Zenryou(自称善良な弁護士)
  日時:2021-06-10 18:13/2021/06/09 22:57 URL:
  \url{https://twitter.com/kk\_hirono/status/1402916723553050625} 
  \url{https://twitter.com/Jisyou\_Zenryou/status/1402625798910058506} 
  \textgreater{} みんな一緒にやめちゃおうぜ \url{https://t.co/R1UDudESRP} 
\end{itemize}

〉〉〉 kk\_hironoのリツイート 〉〉〉

\begin{itemize}
\tightlist
\item
  RT
  kk\_hirono(刑事告発・非常上告_金沢地方検察庁御中)|rgsmama(アンドロイドT)
  日時:2021-06-10 18:13/2021/06/09 22:41 URL:
  \url{https://twitter.com/kk\_hirono/status/1402916821984968708} 
  \url{https://twitter.com/rgsmama/status/1402621990549168132} 
  \textgreater{} これはひどい・・・ \url{https://t.co/imsIQzkpOd} 
\end{itemize}

〉〉〉 kk\_hironoのリツイート 〉〉〉

\begin{itemize}
\tightlist
\item
  RT
  kk\_hirono(刑事告発・非常上告_金沢地方検察庁御中)|kikuyamahiroki(喜久山大貴)
  日時:2021-06-10 18:13/2021/06/09 22:39 URL:
  \url{https://twitter.com/kk\_hirono/status/1402916849549930496} 
  \url{https://twitter.com/kikuyamahiroki/status/1402621403048800266} 
  \textgreater{}
  これはえぐい。。裁判員事件の報酬請求は立会時間を細かく記載させられるから通常の事件より手間が掛かるのに。
  \url{https://t.co/zKju0FpAaT} 
\end{itemize}

〉〉〉 kk\_hironoのリツイート 〉〉〉

\begin{itemize}
\tightlist
\item
  RT
  kk\_hirono(刑事告発・非常上告_金沢地方検察庁御中)|GoMatsuhira5(GO
  MATSUHIRA) 日時:2021-06-10 18:13/2021/06/09 22:28 URL:
  \url{https://twitter.com/kk\_hirono/status/1402916885297987584} 
  \url{https://twitter.com/GoMatsuhira5/status/1402618600704249861} 
  \textgreater{}
  法テラス関係は国分の刑事のみやってたけど、今、弁護人やってるのを最後にやらないと決めた。
  \url{https://t.co/ADj1ekS7RZ} 
\end{itemize}

〉〉〉 kk\_hironoのリツイート 〉〉〉

\begin{itemize}
\tightlist
\item
  RT
  kk\_hirono(刑事告発・非常上告_金沢地方検察庁御中)|ryu\_goma(剛馬)
  日時:2021-06-10 18:14/2021/06/09 21:56 URL:
  \url{https://twitter.com/kk\_hirono/status/1402916961370066944} 
  \url{https://twitter.com/ryu\_goma/status/1402610626720002053} 
  \textgreater{}
  第一法規のD-1には搭載されていなかったです。凄く気になる判決。
  \url{https://t.co/S6bkNNCukQ} 
\end{itemize}

〉〉〉 kk\_hironoのリツイート 〉〉〉

\begin{itemize}
\tightlist
\item
  RT
  kk\_hirono(刑事告発・非常上告_金沢地方検察庁御中)|SYO42657(syo)
  日時:2021-06-10 18:14/2021/06/09 21:50 URL:
  \url{https://twitter.com/kk\_hirono/status/1402917006706282497} 
  \url{https://twitter.com/SYO42657/status/1402609009668939781} 
  \textgreater{}
  業務委託報酬だと思うけど、なぜ報酬請求期限で請求権が消滅するんだろうか??
  かなり長期の事件で、そもそもの請求権発生の時期がかなり前とか??
  ただ無罪判決という大仕事(結果に関わらずかなりの負担)をした弁護人に対しての態度として気に入らない。
  \url{https://t.co/qFGMD4K37p} 
\end{itemize}

〉〉〉 kk\_hironoのリツイート 〉〉〉

\begin{itemize}
\tightlist
\item
  RT
  kk\_hirono(刑事告発・非常上告_金沢地方検察庁御中)|G4FYHAAjWMixyjS(あお)
  日時:2021-06-10 18:14/2021/06/09 21:23 URL:
  \url{https://twitter.com/kk\_hirono/status/1402917066638643204} 
  \url{https://twitter.com/G4FYHAAjWMixyjS/status/1402602187457396744} 
  \textgreater{}
  図々しいかもしれんが、このような人達が腐らずに刑事弁護を続けていてもらいたいと希望してしまう自分がいる
  \url{https://t.co/0yxzTCM1IB} 
\end{itemize}

〉〉〉 kk\_hironoのリツイート 〉〉〉

\begin{itemize}
\tightlist
\item
  RT
  kk\_hirono(刑事告発・非常上告_金沢地方検察庁御中)|rurikorin(おっくん@キュアローヤー)
  日時:2021-06-10 18:15/2021/06/09 21:03 URL:
  \url{https://twitter.com/kk\_hirono/status/1402917392552894465} 
  \url{https://twitter.com/rurikorin/status/1402597302145978370} 
  \textgreater{} これなんで徒過したの? \url{https://t.co/YBMfZS7XsG} 
\end{itemize}

〉〉〉 kk\_hironoのリツイート 〉〉〉

\begin{itemize}
\tightlist
\item
  RT
  kk\_hirono(刑事告発・非常上告_金沢地方検察庁御中)|kojin\_syugi(橋本太地(弁護士・あなたのみかた法律事務所))
  日時:2021-06-10 18:16/2021/06/09 21:00 URL:
  \url{https://twitter.com/kk\_hirono/status/1402917469316993024} 
  \url{https://twitter.com/kojin\_syugi/status/1402596451755053062} 
  \textgreater{} 気をつけよう甘い言葉と以下自粛 \url{https://t.co/Kon4VDzFbY} 
\end{itemize}

〉〉〉 kk\_hironoのリツイート 〉〉〉

\begin{itemize}
\tightlist
\item
  RT
  kk\_hirono(刑事告発・非常上告_金沢地方検察庁御中)|bengoshimentaru(家系弁護士)
  日時:2021-06-10 18:16/2021/06/09 20:59 URL:
  \url{https://twitter.com/kk\_hirono/status/1402917509544644615} 
  \url{https://twitter.com/bengoshimentaru/status/1402596332829741056} 
  \textgreater{} ま? \url{https://t.co/RVEO3qsL5y} 
\end{itemize}

〉〉〉 kk\_hironoのリツイート 〉〉〉

\begin{itemize}
\tightlist
\item
  RT
  kk\_hirono(刑事告発・非常上告_金沢地方検察庁御中)|NOlHT1yemE0873v(弁護士α)
  日時:2021-06-10 18:16/2021/06/09 20:55 URL:
  \url{https://twitter.com/kk\_hirono/status/1402917658534703106} 
  \url{https://twitter.com/NOlHT1yemE0873v/status/1402595188455854081} 
  \textgreater{} 某テラスは超法規的組織。 時効の強行法規性?
  そんなの関係ねえ、そんなの関係ねえ! \url{https://t.co/IrzE2wajMq} 
\end{itemize}

〉〉〉 kk\_hironoのリツイート 〉〉〉

\begin{itemize}
\tightlist
\item
  RT
  kk\_hirono(刑事告発・非常上告_金沢地方検察庁御中)|537s(kariben/ダイエット中)
  日時:2021-06-10 18:16/2021/06/09 20:42 URL:
  \url{https://twitter.com/kk\_hirono/status/1402917692571480066} 
  \url{https://twitter.com/537s/status/1402592051305218049} 
  \textgreater{} こんな事例があったのか。 \url{https://t.co/26apAoYMxF} 
\end{itemize}

〉〉〉 kk\_hironoのリツイート 〉〉〉

\begin{itemize}
\tightlist
\item
  RT
  kk\_hirono(刑事告発・非常上告_金沢地方検察庁御中)|goroceo1(mod7)
  日時:2021-06-10 18:17/2021/06/10 18:13 URL:
  \url{https://twitter.com/kk\_hirono/status/1402917780572098560} 
  \url{https://twitter.com/goroceo1/status/1402916919909392386} 
  \textgreater{} 時効期間を短縮する特約は有効らしい。
  私も何か買うときは、代金請求の時効期間を14日にする特約を結ぼう。
  しかし、「弁護人選任され6か月経過後は報酬等の中間払請求ができること等が理由」ということであれば、単に時効期間を極端に短くする特約は、公序良俗違反で無効になりうるということ?
  \url{https://t.co/JAXZyaBIB2} 
\end{itemize}

〉〉〉 kk\_hironoのリツイート 〉〉〉

\begin{itemize}
\tightlist
\item
  RT
  kk\_hirono(刑事告発・非常上告_金沢地方検察庁御中)|mt1q7q(マニアの受難@歩く不謹慎)
  日時:2021-06-10 18:17/2021/06/09 20:42 URL:
  \url{https://twitter.com/kk\_hirono/status/1402917880694337537} 
  \url{https://twitter.com/mt1q7q/status/1402591901836926983} 
  \textgreater{} ((( ;゚Д゚)))ガクガクブルブル \url{https://t.co/AOCBmv2bX3} 
\end{itemize}

〉〉〉 kk\_hironoのリツイート 〉〉〉

\begin{itemize}
\tightlist
\item
  RT
  kk\_hirono(刑事告発・非常上告_金沢地方検察庁御中)|sakekai(sakekai)
  日時:2021-06-10 18:17/2021/06/09 20:38 URL:
  \url{https://twitter.com/kk\_hirono/status/1402917926554923012} 
  \url{https://twitter.com/sakekai/status/1402590858793291776} 
  \textgreater{} 14日の消滅時効か・・・ \url{https://t.co/BDDMfQ038F} 
\end{itemize}

〉〉〉 kk\_hironoのリツイート 〉〉〉

\begin{itemize}
\tightlist
\item
  RT
  kk\_hirono(刑事告発・非常上告_金沢地方検察庁御中)|idleness\_venomy(venomy)
  日時:2021-06-10 18:18/2021/06/09 20:37 URL:
  \url{https://twitter.com/kk\_hirono/status/1402917958926569475} 
  \url{https://twitter.com/idleness\_venomy/status/1402590554467147781} 
  \textgreater{} こ、これは・・・ \url{https://t.co/xBqUvoQGim} 
\end{itemize}

〉〉〉 kk\_hironoのリツイート 〉〉〉

\begin{itemize}
\tightlist
\item
  RT
  kk\_hirono(刑事告発・非常上告_金沢地方検察庁御中)|houtojitumu(法と実務)
  日時:2021-06-10 18:18/2021/06/09 20:33 URL:
  \url{https://twitter.com/kk\_hirono/status/1402917998281641986} 
  \url{https://twitter.com/houtojitumu/status/1402589633385426952} 
  \textgreater{}
  顧問先からこの判決を示され我が社でもスーパー短期消滅時効を導入したいのですがと言われたら何と答えればよいのだろう
  \url{https://t.co/0jd4kRou7a} 
\end{itemize}

〉〉〉 kk\_hironoのリツイート 〉〉〉

\begin{itemize}
\tightlist
\item
  RT
  kk\_hirono(刑事告発・非常上告_金沢地方検察庁御中)|Pyc2Ry0kSBIHU6x(弁護士
  中川昂 Takashi NAKAGAWA) 日時:2021-06-10 18:18/2021/06/09 20:05
  URL: \url{https://twitter.com/kk\_hirono/status/1402918037674598404} 
  \url{https://twitter.com/Pyc2Ry0kSBIHU6x/status/1402582571150548993} 
  \textgreater{} これは地獄 \url{https://t.co/WsE7Yabqls} 
\end{itemize}

〉〉〉 kk\_hironoのリツイート 〉〉〉

\begin{itemize}
\tightlist
\item
  RT
  kk\_hirono(刑事告発・非常上告_金沢地方検察庁御中)|titokisa123(ちと☆きさ)
  日時:2021-06-10 18:18/2021/06/09 18:45 URL:
  \url{https://twitter.com/kk\_hirono/status/1402918118951833605} 
  \url{https://twitter.com/titokisa123/status/1402562594443993089} 
  \textgreater{}
  こんなのあまりに不当だし、もちろん控訴したんだろうが、上級審はどう判断したのだろうか・・・
  \url{https://t.co/i3A1ugWo7W} 
\end{itemize}

〉〉〉 kk\_hironoのリツイート 〉〉〉

\begin{itemize}
\item
  RT
  kk\_hirono(刑事告発・非常上告_金沢地方検察庁御中)|FLetlRmdM7gs5vS(ふなざわひろゆき)
  日時:2021-06-10 18:18/2021/06/09 18:26 URL:
  \url{https://twitter.com/kk\_hirono/status/1402918166502592516} 
  \url{https://twitter.com/FLetlRmdM7gs5vS/status/1402557750928822273} 
  \textgreater{}
  法テラスの無慈悲ぶりが際立つ事件。900万円の仕事を無事に果たした弁護士に報告期間徒過で報酬払わないって、建築請負仕事終了後、発注者に挨拶ないから請負代金払わないと言うようなものでしょ。契約上こうなるのでしょうが納得感ゼロ。
  \url{https://t.co/lcTdu9SbrQ} 
\item
  〉〉〉 アカウント(@tomopop21)は,@kk\_hironoをブロックしています。リツイートできませんでした。
  〉〉〉 ¥\n ¥\n \url{https://t.co/UPADVVZPTr} 
\item
  〉〉〉 アカウント(@motaberarenaiyo)は,@kk\_hironoをブロックしています。リツイートできませんでした。
  〉〉〉 ¥\n ¥\n \url{https://t.co/q1GdvJvBIk} 
\item
  〉〉〉 アカウント(@akishigemakoto)は,@kk\_hironoをブロックしています。リツイートできませんでした。
  〉〉〉 ¥\n ¥\n \url{https://t.co/aFelu0j8fj} 
\item
  〉〉〉 アカウント(@LiarLawyer800)は,@kk\_hironoをブロックしています。リツイートできませんでした。
  〉〉〉 ¥\n ¥\n \url{https://t.co/LjZDZkh81R} 
\item
  〉〉〉 アカウント(@k999941457035)は,@kk\_hironoをブロックしています。リツイートできませんでした。
  〉〉〉 ¥\n ¥\n \url{https://t.co/y22XpjD4N6} 
\item
  〉〉〉 アカウント(@harvey61616)は,@kk\_hironoをブロックしています。リツイートできませんでした。
  〉〉〉 ¥\n ¥\n \url{https://t.co/xW8V16l0qP} 
\item
  〉〉〉 アカウント(@O59K2dPQH59QEJx)は,@kk\_hironoをブロックしています。リツイートできませんでした。
  〉〉〉 ¥\n ¥\n \url{https://t.co/Xh6ZHy1p6w} 
\end{itemize}

※ @kk\_hironoのアカウントがブロックされ,リツイートに失敗したツイート

\begin{itemize}
\tightlist
\item
  TW tomopop21(ナカジマ) 日時:2021/06/10 10:56:05 URL:
  \url{https://twitter.com/tomopop21/status/1402806744632369155} 
  \textgreater{} @kitaguni\_b これ可哀想\\
  \textgreater{} \url{https://t.co/xDAfsNKAxN} 
\end{itemize}

※ @kk\_hironoのアカウントがブロックされ,リツイートに失敗したツイート

\begin{itemize}
\tightlist
\item
  TW motaberarenaiyo(過食B) 日時:2021/06/10 09:45:58 URL:
  \url{https://twitter.com/motaberarenaiyo/status/1402789100508172290} 
  \textgreater{} これは酷い。\\
  \textgreater{}
  弁護人は無謬ではないけれど、そもそも14日の期間が短すぎるし、そこまで短くする意味がそもそもない。
  \url{https://t.co/BSURwqv3n3} 
\end{itemize}

※ @kk\_hironoのアカウントがブロックされ,リツイートに失敗したツイート

\begin{itemize}
\tightlist
\item
  TW akishigemakoto(MakotoAkishige(civilista)) 日時:2021/06/10
  06:32:54 URL:
  \url{https://twitter.com/akishigemakoto/status/1402740514235899905} 
  \textgreater{} 時効制度の潜脱のような気もするけど
  \url{https://t.co/ZkhBDXA9FB} 
\end{itemize}

※ @kk\_hironoのアカウントがブロックされ,リツイートに失敗したツイート

\begin{itemize}
\tightlist
\item
  TW kamatatylaw(高橋雄一郎) 日時:2021/06/09 22:45:54 URL:
  \url{https://twitter.com/kamatatylaw/status/1402622988558553092} 
  \textgreater{}
  国選弁護はノブリスオブリージュであり無償が原則、国選報酬は厳密な意味での報酬ではなく、法テラスの恩恵で付与されるところの、頑張った弁護士への御褒美に過ぎないので権利性はない、という説明ならいちおう辻褄はあうよ。
  \url{https://t.co/NgIKyshwQA} 
\end{itemize}

※ @kk\_hironoのアカウントがブロックされ,リツイートに失敗したツイート

\begin{itemize}
\tightlist
\item
  TW LiarLawyer800(うそつきべ んごし。) 日時:2021/06/09 21:19:39
  URL: \url{https://twitter.com/LiarLawyer800/status/1402601284104986626} 
  \textgreater{}
  法テラスの規定は、どうして時効よりも超短期にそんなに強力な規定になってんの??
  \url{https://t.co/lwTjxak6TK} 
\end{itemize}

※ @kk\_hironoのアカウントがブロックされ,リツイートに失敗したツイート

\begin{itemize}
\tightlist
\item
  TW k999941457035(🐯K 9 9 9 9🐯) 日時:2021/06/09 21:08:44 URL:
  \url{https://twitter.com/k999941457035/status/1402598535745265668} 
  \textgreater{}
  国選弁護人なんていなくなってもよいという裁判所からのメッセージですね
  \url{https://t.co/BfERC3v3x8} 
\end{itemize}

※ @kk\_hironoのアカウントがブロックされ,リツイートに失敗したツイート

\begin{itemize}
\tightlist
\item
  TW harvey61616(ついぶる) 日時:2021/06/09 20:57:58 URL:
  \url{https://twitter.com/harvey61616/status/1402595826103377921} 
  \textgreater{}
  法テラスは、時間をかけて申立直前までいっていた破産の依頼者が飛んで辞任したら、ただでさえ安い着手金の8割方返金と言われたので、切った。\\
  \textgreater{}\\
  \textgreater{} おかげさまで、その後は売上倍増しました。\\
  \textgreater{} ありがとう法テラス。 \url{https://t.co/Ppa1PIsdBb} 
\end{itemize}

※ @kk\_hironoのアカウントがブロックされ,リツイートに失敗したツイート

\begin{itemize}
\item
  TW O59K2dPQH59QEJx(ピピピーッ) 日時:2021/06/09 20:42:36 URL:
  \url{https://twitter.com/O59K2dPQH59QEJx/status/1402591957583417351} 
  \textgreater{} ひでー話だ。\\
  \textgreater{} なぜ時効期間を短くできるのか、不思議。
  \url{https://t.co/1UsTzS04tn} 
\item
  〈〈〈 2021/06/10 18:22:08 Linux Emacs: 〈〈〈
\end{itemize}

\hypertarget{ux5c0fux5009ux79c0ux592bux5f01ux8b77ux58ebux306bux5bfeux3059ux308bux540dux8a89ux6bc0ux640dux306eux5211ux4e8bux624bux7d9aux306eux5fc5ux8981ux6027ux3092ux65b0ux305fux306aux6c7aux610fux3068ux3055ux305bux308bux304dux3063ux304bux3051ux3068ux306aux3063ux305fux6df1ux6fa4ux8aedux53f2ux5f01ux8b77ux58ebux306eux30bfux30a4ux30e0ux30e9ux30a4ux30f3ux306bux304aux3051ux308bux5211ux88c1ux30b5ux30a4ux592aux306eux30c4ux30a4ux30fcux30c8ux5ca1ux53e3ux57faux4e00ux88c1ux5224ux5b98ux306eux5f3eux52beux88c1ux5224ux6c7aux5b9aux3078ux3068ux7d9aux3044ux305f}{%
\paragraph{小倉秀夫弁護士に対する名誉毀損の刑事手続の必要性を新たな決意とさせるきっかけとなった、深澤諭史弁護士のタイムラインにおける刑裁サイ太のツイート、岡口基一裁判官の弾劾裁判決定へと続いた}\label{ux5c0fux5009ux79c0ux592bux5f01ux8b77ux58ebux306bux5bfeux3059ux308bux540dux8a89ux6bc0ux640dux306eux5211ux4e8bux624bux7d9aux306eux5fc5ux8981ux6027ux3092ux65b0ux305fux306aux6c7aux610fux3068ux3055ux305bux308bux304dux3063ux304bux3051ux3068ux306aux3063ux305fux6df1ux6fa4ux8aedux53f2ux5f01ux8b77ux58ebux306eux30bfux30a4ux30e0ux30e9ux30a4ux30f3ux306bux304aux3051ux308bux5211ux88c1ux30b5ux30a4ux592aux306eux30c4ux30a4ux30fcux30c8ux5ca1ux53e3ux57faux4e00ux88c1ux5224ux5b98ux306eux5f3eux52beux88c1ux5224ux6c7aux5b9aux3078ux3068ux7d9aux3044ux305f}}

\begin{itemize}
\tightlist
\item
  〉〉〉 Linux Emacs: 2021/06/20 22:44:30 〉〉〉
\end{itemize}

:CATEGORIES: @kanazawabengosi \#金沢弁護士会 @JFBAsns
日本弁護士連合会(日弁連) \#法務省 @MOJ\_HOUMU \#深澤諭史弁護士
\#刑裁サイ太 \#岡口基一裁判官 \#小倉秀夫弁護士
\#モトケンこと矢部善朗弁護士(京都弁護士会)

 まず、深澤諭史弁護士のタイムラインから関連したツイートを告発\市場急配センター殺人未遂事件\金沢地方検察庁・石川県警察御中(@kk\_hirono)のアカウントでリツイートしていきます。ブロックも多いでしょうが、それも重要な記録です。

〉〉〉 kk\_hironoのリツイート 〉〉〉

\begin{itemize}
\tightlist
\item
  RT
  kk\_hirono(刑事告発・非常上告_金沢地方検察庁御中)|gatsu73(ガツ)
  日時:2021-06-20 22:47/2021/06/16 15:19 URL:
  \url{https://twitter.com/kk\_hirono/status/1406609681284403202} 
  \url{https://twitter.com/gatsu73/status/1405047236380164098} 
  \textgreater{} 法学部生「法律難しい」 ロー生「法律分からない」
  修習生「法律苦手」 弁護士「法律未だに分からない」
  学者「未だに不勉強な部分が多い・・・」
  自称法律に詳しいおじさん「私は法律に詳しい!弁護士さんの言ってることは間違っている」
  弁護士に知り合いがいると自称するおじさん「そうだそうだ!」
  \url{https://t.co/MAy5away4m} 
\end{itemize}

〉〉〉 kk\_hironoのリツイート 〉〉〉

\begin{itemize}
\item
  RT
  kk\_hirono(刑事告発・非常上告_金沢地方検察庁御中)|chosakukenho(小倉秀夫)
  日時:2021-06-20 22:48/2021/06/16 16:29 URL:
  \url{https://twitter.com/kk\_hirono/status/1406609880161558528} 
  \url{https://twitter.com/chosakukenho/status/1405065069721911300} 
  \textgreater{}
  たった1回の被害者供述さえあれば、客観証拠と矛盾していようとも、有罪判決を出せと言うことなんではないですかね。
  \url{https://t.co/fkUsaUGuVF} 
\item
  〉〉〉 アカウント(@uwaaaa)は,@kk\_hironoをブロックしています。リツイートできませんでした。
  〉〉〉 ¥\n ¥\n \url{https://t.co/2SwBML6jC8} 
\item
  〉〉〉 アカウント(@uwaaaa)は,@kk\_hironoをブロックしています。リツイートできませんでした。
  〉〉〉 ¥\n ¥\n \url{https://t.co/uDvsgM2GCU} 
\item
  〉〉〉 アカウント(@fukazawas)は,@kk\_hironoをブロックしています。リツイートできませんでした。
  〉〉〉 ¥\n ¥\n \url{https://t.co/CrlznEZqt0} 
\item
  〉〉〉 アカウント(@fukazawas)は,@kk\_hironoをブロックしています。リツイートできませんでした。
  〉〉〉 ¥\n ¥\n \url{https://t.co/9l2vd4fhES} 
\end{itemize}

※ @kk\_hironoのアカウントがブロックされ,リツイートに失敗したツイート

\begin{itemize}
\tightlist
\item
  TW uwaaaa(サイ太) 日時:2021/06/16 15:48:14 URL:
  \url{https://twitter.com/uwaaaa/status/1405054592358916098} 
  \textgreater{}
  こういう起訴するかどうかの判断にも大きく影響する程度に人質司法が蔓延っているの国に,人質司法とは無縁の国の制度を接ぎ木すること自体の問題はあると思うんですよ
\end{itemize}

※ @kk\_hironoのアカウントがブロックされ,リツイートに失敗したツイート

\begin{itemize}
\tightlist
\item
  TW uwaaaa(サイ太) 日時:2021/06/16 15:43:00 URL:
  \url{https://twitter.com/uwaaaa/status/1405053276416925700} 
  \textgreater{}
  >性被害に遭った障害者の代理人経験がある杉浦ひとみ弁護士も「性犯罪は目撃者がいない密室で起きることが多く、被害を受けたのが障害者の場合、物証が特に堅固でないと検察は起訴に慎重だ」と難しさを語る。\\
  \textgreater{}\\
  \textgreater{}
  身柄をどんどん解放してどんどん起訴してどんどん無罪出せばいいと思うのよ
\end{itemize}

※ @kk\_hironoのアカウントがブロックされ,リツイートに失敗したツイート

\begin{itemize}
\tightlist
\item
  TW fukazawas(深澤諭史) 日時:2021/06/16 12:21:55 URL:
  \url{https://twitter.com/fukazawas/status/1405002674181787648} 
  \textgreater{}
  弱者男性差別は存在するから知ってください、フェミニストはこれ以上差別しないでください|88cmと8kg
  \#note \url{https://t.co/5xvKGbobVf} 
  \textgreater{} (・∀・)ふーむ、考えさせられる。\\
  \textgreater{} (^ω^)ガラスの天井とガラスの地下室か。
\end{itemize}

※ @kk\_hironoのアカウントがブロックされ,リツイートに失敗したツイート

\begin{itemize}
\tightlist
\item
  TW fukazawas(深澤諭史) 日時:2020/11/15 09:14:05 URL:
  \url{https://twitter.com/fukazawas/status/1327766786171813889} 
  \textgreater{} 司法修習生「民事訴訟難しい」\\
  \textgreater{} 新人弁護士「民事訴訟難しい」\\
  \textgreater{} 中堅弁護士「民事訴訟難しい」\\
  \textgreater{} ベテラン弁護士「民事訴訟難しい」\\
  \textgreater{} 裁判所「民事訴訟難しい」\\
  \textgreater{}
  ネット上の匿名の親切な人「本人訴訟でOK!裁判官は証拠と判例に基づき公正に判断するから大丈夫!ネットde真実の法律情報!」
\end{itemize}

 深澤諭史弁護士のタイムラインで見たときは気が付かなかったのですが、2020年11月15日のリツイートとはいうものの、未だに「ネット上の匿名の親切な人「本人訴訟でOK!裁判官は証拠と判例に基づき公正に判断するから大丈夫!ネットde真実の法律情報!」という考えが抜けてはいないようです。

 時刻は22時55分です。6月16日というのは4日前ですが、特別な記憶に残る日でした。後で見たツイートでは夕方の16時30分頃が始まりだったと思うのですが、上述の深澤諭史弁護士のタイムラインで見かけた刑裁サイ太のツイートがきっかけで、スクリーンキャストの記録を開始しました。

 最初に小倉秀夫弁護士のツイートを記録し、次にモトケンこと矢部善朗弁護士(京都弁護士会)のツイートの記録をしたのですが、エラーが出て動画の作成に失敗していました。再度試みたのですが、それも失敗して、スマホでの画面の撮影に切り替えてのです。

 他に予定があったとも思うのですが、その予定が狂ってしまい、一通りの作業を終えたと思ったところで、最初に目にしたのが町村泰貴教授のツイートで、岡口基一裁判官の弾劾裁判の訴追が決定したという情報を知ったのです。これも偶然とは思えないタイミングに思えました。

\begin{itemize}
\tightlist
\item
  2021年06月16日10時26分の登録:
  Ex-REGEXP:''和歌山.*事件''/データベース登録済みツイートの検索:2021-06-14〜2021-06-15/2021年06月16日10時26分の記録:ユーザ・投稿:4/10件
  \url{https://kk2020-09.blogspot.com/2021/06/ex-regexp2021-06-142021-06.html} 
\item
  2021年06月16日10時34分の登録:
  REGEXP:''女子校のプールの水になりたい''/データベース登録済みツイートの検索:2015-02-12〜2021-06-16/2021年06月16日10時32分の記録:ユーザ・投稿:68/98件
  \url{https://kk2020-09.blogspot.com/2021/06/regexp2015-02-122021-06.html} 
\item
  2021年06月16日10時41分の登録:
  REGEXP:''和歌山.*長女''/データベース登録済みツイートの検索:2021-06-14〜2021-06-16/2021年06月16日10時40分の記録:ユーザ・投稿:5/23件
  \url{https://kk2020-09.blogspot.com/2021/06/regexp2021-06-142021-06\_16.html} 
\item
  2021年06月16日11時08分の登録:
  REGEXP:''準抗告''/データベース登録済みツイートの検索:2020-12-28〜2021-06-15/2021年06月16日11時03分の記録:ユーザ・投稿:147/283件
  \url{https://kk2020-09.blogspot.com/2021/06/regexp2020-12-282021-06.html} 
\item
  2021年06月16日11時22分の登録:
  REGEXP:''準抗告''/データベース登録済みツイートの検索:2021-06-09〜2021-06-15/2021年06月16日11時17分の記録:ユーザ・投稿:111/189件
  \url{https://kk2020-09.blogspot.com/2021/06/regexp2021-06-092021-06.html} 
\item
  2021年06月16日11時38分の登録:
  @hirono\_hideki(奉納\さらば弁護士鉄道・泥棒神社の物語)のツイート ''.*'' 3245/3245:2021-06-01\_1149〜2021-06-16\_1122 2021年06月16日11時37分の記録
  \url{https://kk2020-09.blogspot.com/2021/06/hironohideki324532452021-06-0111492021.html} 
\item
  2021年06月16日13時15分の登録:
  REGEXP:''岡口''/データベース登録済みツイートの検索:2021-06-07〜2021-06-16/2021年06月16日13時14分の記録:ユーザ・投稿:28/38件
  \url{https://kk2020-09.blogspot.com/2021/06/regexp2021-06-072021-06\_51.html} 
\item
  2021年06月16日16時19分の登録:
  \7286 @jmjhjmwtad\ツイッターでの女子高のプールになりたい発言を擁護した弁護士を委員から外せ、みたいなムーブが起きてるけど、完全に関西に良くいる関わっては
  \url{https://kk2020-09.blogspot.com/2021/06/7286jmjhjmwtad\_16.html} 
\item
  2021年06月16日16時20分の登録:
  \渡辺輝人 @nabeteru1Q78\「お上」が、緊急事態宣言解除のサインを出すと、下々はその時点から活動を再開する。解除後に行儀良く、少しずつ、社会活動を再開するわけ
  \url{https://kk2020-09.blogspot.com/2021/06/nabeteru1q78\_16.html} 
\item
  2021年06月16日16時22分の登録:
  \サイ太 @uwaaaa\弁護士の費用が「着手金・成功報酬」に分けられることを必ずしも意識しているようには思われない。不法行為は着手金も含めて事件解決後に支払う契約にする
  \url{https://kk2020-09.blogspot.com/2021/06/uwaaaa\_16.html} 
\item
  2021年06月16日16時24分の登録:
  \サイ太 @uwaaaa\身柄をどんどん解放してどんどん起訴してどんどん無罪出せばいいと思うのよ
  \url{https://kk2020-09.blogspot.com/2021/06/uwaaaa\_92.html} 
\item
  2021年06月16日16時25分の登録:
  \深澤諭史 @fukazawas\(・∀・)ふーむ、考えさせられる。(^ω^)ガラスの天井とガラスの地下室か。
  \url{https://kk2020-09.blogspot.com/2021/06/fukazawas\_16.html} 
\item
  2021年06月16日16時58分の登録:
  %@hirono\_hideki 奉納\さらば弁護士鉄道・泥棒神社の物語%\url{https://twitter.com/hirono\_hideki/status/1403390029926322177} 
  \url{https://kk2020-09.blogspot.com/2021/06/hironohidekihttpstwittercomhironohideki.html} 
\item
  2021年06月16日18時22分の登録:
  REGEXP:''サイ太''/データベース登録済みツイートの検索:2021-05-29〜2021-06-16/2021年06月16日18時21分の記録:ユーザ・投稿:12/367件
  \url{https://kk2020-09.blogspot.com/2021/06/regexp2021-05-292021-06\_16.html} 
\item
  2021年06月16日18時55分の登録:
  \MakotoAkishige(civilista) @akishigemakoto\顧問弁護士はウハウハかも引用ツイート
  \url{https://kk2020-09.blogspot.com/2021/06/makotoakishigecivilistaakishigemakoto\_16.html} 
\item
  2021年06月16日19時17分の登録:
  \過食B @motaberarenaiyo\まあ色々アレだけど、他に酷い裁判官がゴマンといる中で、1人だけってのがね
  \url{https://kk2020-09.blogspot.com/2021/06/bmotaberarenaiyo1.html} 
\item
  2021年06月16日19時28分の登録:
  \7286 @jmjhjmwtad\ツイッターというSNSなんて単なる掃き溜めなんで、まぁこんなとこの意見をまともに受ける必要はないよ。結局、状況は、自分が苦しんで悩んだ
  \url{https://kk2020-09.blogspot.com/2021/06/7286jmjhjmwtadsns.html} 
\item
  2021年06月16日19時29分の登録:
  \うの字を名乗る?物 @un\_co\_the2nd\当該文章を「うん、キモい妄想だね」って評価して終わる話を、まるで自分の体を現に弄られたかのように大騒ぎしてるのが異常だよ
  \url{https://kk2020-09.blogspot.com/2021/06/uncothe2nd\_96.html} 
\item
  2021年06月16日19時30分の登録:
  \うの字を名乗る?物 @un\_co\_the2nd\ここまでやるなら、裁判所のパワハラ事案ではないかな
  \url{https://kk2020-09.blogspot.com/2021/06/uncothe2nd\_66.html} 
\item
  2021年06月16日19時30分の登録: \ふて寝べん @hirune\_b\返信先:
  @un\_co\_the2ndさん加害行為と評価できるようにするための理屈をあれこれ考えているような印象を受けるんですよね。
  \url{https://kk2020-09.blogspot.com/2021/06/hiruneb-uncothe2nd.html} 
\item
  2021年06月16日19時43分の登録:
  \うの字を名乗る?物 @un\_co\_the2nd\ええ・・・。「洗脳」はウヒョッてなったけど、罷免するしないのレベルの話ではなかろう。自己が当事者となっている事件の話題で、高
  \url{https://kk2020-09.blogspot.com/2021/06/uncothe2nd\_67.html} 
\item
  2021年06月16日19時57分の登録:
  REGEXP:''岡口''/サイ太(@uwaaaa)の検索(2013-10-21〜2021-06-04/2021年06月16日19時57分の記録86件)
  \url{https://kk2020-09.blogspot.com/2021/06/regexpuwaaaa2013-10-212021-06.html} 
\item
  2021年06月16日20時26分の登録:
  \小倉秀夫 @chosakukenho\立法担当者が、自分の身の周りで起こりうることしか想定できないようだと困るんですよw。
  \url{https://kk2020-09.blogspot.com/2021/06/chosakukenhow.html} 
\item
  2021年06月16日20時29分の登録: \Shoko Egawa @amneris84\臓器提供
  家族の葛藤 移植待つ娘はドナーになった \textbar{} NHKニュース
  \url{https://kk2020-09.blogspot.com/2021/06/shoko-egawaamneris84-nhk.html} 
\item
  2021年06月16日20時33分の登録:
  \サイ太 @uwaaaa\このツイートくらいしかソースが見当たらなかったんですけど,特捜検事が弁護人接見を装って取調べをしたみたいな事件があったような記憶なんですが,どな
  \url{https://kk2020-09.blogspot.com/2021/06/uwaaaa\_0.html} 
\item
  2021年06月16日20時39分の登録:
  \サイ太 @uwaaaa\最近のエロ業界の流行を幟で教えてくれる近所の店に「ウーマナイザーあります!」という文字列が踊っていた
  \url{https://kk2020-09.blogspot.com/2021/06/uwaaaa\_97.html} 
\item
  2021年06月16日20時40分の登録:
  \サイ太 @uwaaaa\刑事責任能力に疑問があるとなぜ匿名報道になるのかよく分からない。刑事責任能力だろうがなんだろうが有罪にならない可能性はどんな事件にも存在するはず
  \url{https://kk2020-09.blogspot.com/2021/06/uwaaaa\_84.html} 
\item
  2021年06月16日20時41分の登録:
  \奥村徹弁護士 @okumuraosaka\には精神疾患があり、同署は名前を公表せず刑事責任能力を調べる。マンションの一室で写真撮影、モデル少女の体触る 兵庫・芦屋の無職
  \url{https://kk2020-09.blogspot.com/2021/06/okumuraosaka\_16.html} 
\item
  2021年06月16日20時45分の登録: \弁護士 市川
  寛 @imarockcaster42\朝井リョウ「正欲」(新潮社)。参考文献の筆頭になんと拙著「検事失格」が挙がっていたのに驚いて即買い。とても嬉しく思い
  \url{https://kk2020-09.blogspot.com/2021/06/imarockcaster42\_16.html} 
\item
  2021年06月16日20時53分の登録:
  \まんごう @nan5o\「あんたのは奇案や」と何度言われたか
  \url{https://kk2020-09.blogspot.com/2021/06/nan5o.html} 
\item
  2021年06月16日20時55分の登録: \芝原章吾(Shogo
  Shibahara) @shogoshibahara\鹿児島地方裁判所谷山支部が欲しい。誰におねだりすれば良いんだろう?
  \url{https://kk2020-09.blogspot.com/2021/06/shogo-shibaharashogoshibahara\_16.html} 
\item
  2021年06月16日20時56分の登録:
  \赤木真也(弁護士・LEC専任講師) @akagilaw\責任は存否(有無)だけの議論ではない。責任ありなら、次はどう取るか、の問題。なぜこの国のマスコミは責任ありと為政
  \url{https://kk2020-09.blogspot.com/2021/06/lecakagilaw\_16.html} 
\item
  2021年06月16日21時12分の登録:
  \ガツ @gatsu73\法学部生「法律難しい」ロー生「法律分からない」修習生「法律苦手」弁護士「法律未だに分からない」学者「未だに不勉強な部分が多い・・・」自称法律に詳しい
  \url{https://kk2020-09.blogspot.com/2021/06/gatsu73.html} 
\item
  2021年06月16日21時36分の登録:
  \弁護士落合洋司?感染拡大を招く東京(頭狂)オリンピック中止!? @yjochi\交番や駐在所って、ラブホテル代わりに使われてるんだな。日本警察は大丈夫か。川路大警視がお
  \url{https://kk2020-09.blogspot.com/2021/06/yjochi\_16.html} 
\item
  2021年06月16日21時55分の登録:
  REGEXP:''岡口''/データベース登録済みツイートの検索:2021-06-16〜2021-06-16/2021年06月16日21時50分の記録:ユーザ・投稿:123/231件
  \url{https://kk2020-09.blogspot.com/2021/06/regexp2021-06-162021-06.html} 
\end{itemize}

 これからリンクを開きたいところですが、「REGEXP:''岡口''/データベース登録済みツイートの検索:2021-06-07〜2021-06-16/2021年06月16日13時14分の記録:ユーザ・投稿:28/38件」という記録がずいぶん意外に思えました。内容を確認しますが、まずは驚きです。

\begin{itemize}
\item
  (14/38) TW @zVNkc6c8wnU54aD(炭竈法律事務所(寝屋川市)) 日時:
  2021-06-10 21:32:16 +0900 URL:
  \url{https://twitter.com/zVNkc6c8wnU54aD/status/1402966844424658951\textgreater} {}
  岡口裁判官、異例の再聴取 不適切投稿で国会訴追委(共同通信)\textgreater{}
  \#Yahooニュース\textgreater{} \url{https://t.co/JYDKjd9gMs} 
\item
  (19/38) TW @shin2\_ota(太田 伸二) 日時: 2021-06-12 10:31:57
  +0900 URL:
  \url{https://twitter.com/shin2\_ota/status/1403525449188143108\textgreater} {}
  東北弁連の夏期研修の案内が届き、2日目に岡口裁判官の「要件事実の本質と実務」という講義があるようだ。岡口裁判官の要件事実講義を聞いてみたかったので、日程に入れることにしよう。
\item
  (21/38) TW @showonelaw(ねここ@若手弁■士) 日時: 2021-06-12
  13:14:54 +0900 URL:
  \url{https://twitter.com/showonelaw/status/1403566456311218177\textgreater} {}
  ただ、ワケわかんない理由で懲戒請求とか受けそうで、諸刃の剣ですね。\textgreater{}
  実名tweetの危険は、岡口Jやタヒタヒ先生からも明らかですし、自分は実名アカにはしないと思います。
  \url{https://t.co/oRORZrvfrs} 
\item
  (22/38) TW @harrier0516osk(向原総合法律事務所 弁護士向原) 日時:
  2021-06-13 01:31:41 +0900 URL:
  \url{https://twitter.com/harrier0516osk/status/1403751873811787776\textgreater} {}
  @Prosecutor\_46JR @tamago88tama 岡口裁判官も法クラですよ!(\^{}\^{})
\item
  (24/38) TW @kk\_hirono(刑事告発・非常上告_金沢地方検察庁御中)
  日時: 2021-06-14 16:45:55 +0900 URL:
  \url{https://twitter.com/kk\_hirono/status/1404344333164105730\textgreater} {}
  脳機能障害の深澤諭史弁護士のツイートのまとめは2件でしたが、もう1件が岡口基一裁判官のツイートのリツイートとなっていました。Twitter社に永久凍結された岡口基一裁判官のアカウントです。
\item
  (25/38) RT @koli\_san(こりさん)|Sankei\_news(産経ニュース)
  日時:2021-06-14 20:23:32 +0900/2021-06-14 20:20:00 +0900 URL:
  \url{https://twitter.com/koli\_san/status/1404399101781823488} 
  \url{https://twitter.com/Sankei\_news/status/1404398311365177350\textgreater} {}
  訴追可否16日にも判断 不適切投稿判事、国会委
  \url{https://t.co/urDBB5LgKv\textgreater\textgreater} {}
  仙台高裁の岡口基一判事(55)について、国会の裁判官訴追委員会(委員長・新藤義孝衆院議員)が会期末の16日にも訴追の可否を決める方針であることが14日、関係者への取材で分かった。
\end{itemize}

 全く気が付かずにいたようですが、6月14日20時20分の産経ニュースのツイートで、「訴追可否16日にも判断 不適切投稿判事、国会委」という実質、緊急告知のような知らせは出ていたようです。

\begin{quote}
《引用の始まり》
\end{quote}

\begin{quote}
衆議院議員(立憲民主党、当選6回)。元・日本銀行職員。自民党の林芳正参院議員と共著で『\#国会議員の仕事』(中公新書)を出版。専門は経済政策、科学技術政策。\#女性天皇
\#尊厳死合法化 にも取り組む。社労士受験生。尊敬する政治家は
\#与謝野馨、\#江田三郎。好きな作家は
\#向田邦子、\#綿矢りさ。元モノノフ。岡山県(岡山市、玉野市、瀬戸内市)tsumura.org誕生日:
10月27日2011年5月からTwitterを利用しています433 フォロー中1万 フォロワー
\end{quote}

\begin{quote}
《引用の終わり》
\end{quote}

\begin{itemize}
\item
  \begin{enumerate}
  \def\labelenumi{(\arabic{enumi})}
  \setcounter{enumi}{3}
  \tightlist
  \item
    津村啓介さん (@Tsumura\_Keisuke) / Twitter
    \url{https://twitter.com/Tsumura\_Keisuke} 
  \end{enumerate}
\end{itemize}

 Twitterアカウントのプロフィールをみると、弁護士というのはなく、リストに入れていたのか疑問に思ったのですが、リストに登録済みであったことを確認しました。プロフィールには岡山市、玉野市、瀬戸内市とあります。瀬戸内市というのは初めて見たように思います。

 ふと思い出してキーワード検索し、新たにまとめ記事を作成しておくということはあるのですが、6月16日の13時15分というタイミングは全く意外でした。産経ニュースのみでしたが、最初の報道の日時を確認する目的もあって、メディア関係も複数リストに加えています。

 見落としがあるかもしれませんが、「岡口」をキーワードにしたツイートの次のまとめが「2021年06月16日21時55分の登録」となっていました。

\begin{itemize}
\tightlist
\item
  (002/231) TW @jijicom(時事ドットコム(時事通信ニュース)) 日時:
  2021-06-16 17:20:01 +0900 URL:
  \url{https://twitter.com/jijicom/status/1405077693972959234\textgreater} {}
  【速報】\textgreater{}
  国会の裁判官訴追委員会は、岡口基一仙台高裁判事を弾劾裁判所に訴追する決定をした
  \url{https://t.co/PZVtYgzOBL} 
\end{itemize}

 私の観測というか自前のプログラムを使った記録の範囲ですが、上記の6月16日17時20分の「時事ドットコム(時事通信ニュース)」のツイートが、岡口基一仙台高裁判事を弾劾裁判所に訴追する決定というニュースの第一報となっていました。

 これまでの経験則で「岡口」というキーワード一本に絞っているのですが、岡口基一裁判官、岡口判事、岡口さん、岡口Jなどという多彩なバリエーションがあり、通常のTwitter検索で交じるのは和歌山城にあるらしい岡口門ぐらいということも確認して記録をやっています。

 けっこうな数があるようですが、今月2021年6月に入ってから「岡口」をキーワードに含む記録が次になります。

\begin{itemize}
\tightlist
\item
  2021年06月04日12時21分の登録:
  \サイ太 @uwaaaa\この件,強盗殺人と強盗強姦未遂の事案なんだけど,伊藤和子大先生は岡口さんの肩を持っていて,どうしても普段のスタンスと整合しない
  \url{https://kk2020-09.blogspot.com/2021/06/uwaaaa\_4.html} 
\item
  2021年06月04日12時25分の登録:
  REGEXP:''岡口基一''/データベース登録済みツイートの検索:2021-02-27〜2021-06-04/2021年06月04日12時24分の記録:ユーザ・投稿:5/6件
  \url{https://kk2020-09.blogspot.com/2021/06/regexp2021-02-272021-06-0420210604122456.html} 
\item
  2021年06月04日13時14分の登録:
  REGEXP:''岡口まつり''/データベース登録済みツイートの検索:2013-12-14〜2021-06-04/2021年06月04日13時13分の記録:ユーザ・投稿:13/33件
  \url{https://kk2020-09.blogspot.com/2021/06/regexp2013-12-142021-06.html} 
\item
  2021年06月04日13時38分の登録:
  \椎名つよし(何でも屋さん) @t\_417\_kawasaki\訴追委の実務はそこそこ知ってますが、岡口判事の行為は弾劾法第2条の「著しい」要件で跳ねられるべきと直感的に思
  \url{https://kk2020-09.blogspot.com/2021/06/t417kawasaki.html} 
\item
  2021年06月04日13時49分の登録:
  REGEXP:''岡口''/データベース登録済みツイートの検索:2008-10-08〜2021-06-04/2021年06月04日13時28分の記録:ユーザ・投稿:547/4566件
  \url{https://kk2020-09.blogspot.com/2021/06/regexp2008-10-082021-06.html} 
\item
  2021年06月04日20時18分の登録:
  \椎名つよし(何でも屋さん) @t\_417\_kawasaki\岡口判事の案件は、訴追委から最高裁宛に調査嘱託くらいまではしてるかもしれないなぁ。ただ、実際本人に出頭を求め
  \url{https://kk2020-09.blogspot.com/2021/06/t417kawasaki\_4.html} 
\item
  2021年06月04日23時01分の登録:
  @uwaaaa(サイ太)のツイート ''岡口'' 1/3215:2020-11-05\_1718〜2021-06-04\_2212 2021年06月04日23時01分の記録
  \url{https://kk2020-09.blogspot.com/2021/06/uwaaaa132152020-11-0517182021-06.html} 
\item
  2021年06月04日23時02分の登録:
  @lawkus(ystk)のツイート ''岡口'' 0/3226:2021-02-23\_0047〜2021-06-04\_1735 2021年06月04日23時02分の記録
  \url{https://kk2020-09.blogspot.com/2021/06/lawkusystk032262021-02-2300472021-06.html} 
\item
  2021年06月04日23時07分の登録:
  @otakulawyer(山口貴士 aka無駄に感じが悪いヤマベン)のツイート ''岡口'' 0/3195:2020-12-12\_1111〜2021-06-04\_2019 2021年06月04日23時07分の記録
  \url{https://kk2020-09.blogspot.com/2021/06/otakulawyeraka031952020-12-1211112021.html} 
\item
  2021年06月04日23時10分の登録:
  @keita\_adachi(豪弁 足立敬太 @ザンギの極み)のツイート ''岡口'' 0/3237:2021-05-12\_1244〜2021-06-04\_2306 2021年06月04日23時10分の記録
  \url{https://kk2020-09.blogspot.com/2021/06/keitaadachi032372021-05-1212442021-06.html} 
\item
  2021年06月04日23時11分の登録:
  @t\_hirai(平井利明)のツイート ''岡口'' 0/3186:2014-08-29\_1939〜2021-06-02\_2233 2021年06月04日23時11分の記録
  \url{https://kk2020-09.blogspot.com/2021/06/thirai031862014-08-2919392021-06.html} 
\item
  2021年06月05日06時10分の登録:
  REGEXP:''岡口''/データベース登録済みツイートの検索:2021-06-04〜2021-06-05/2021年06月05日06時08分の記録:ユーザ・投稿:36/223件
  \url{https://kk2020-09.blogspot.com/2021/06/regexp2021-06-042021-06\_5.html} 
\item
  2021年06月05日07時27分の登録:
  REGEXP:''岡口''/データベース登録済みツイートの検索:2021-06-04〜2021-06-05/2021年06月05日07時25分の記録:ユーザ・投稿:34/47件
  \url{https://kk2020-09.blogspot.com/2021/06/regexp2021-06-042021-06\_77.html} 
\item
  2021年06月07日20時40分の登録:
  REGEXP:''岡口''/データベース登録済みツイートの検索:2021-06-05〜2021-06-07/2021年06月07日20時39分の記録:ユーザ・投稿:17/25件
  \url{https://kk2020-09.blogspot.com/2021/06/regexp2021-06-052021-06.html} 
\item
  2021年06月08日23時58分の登録:
  \MakotoAkishige(civilista) @akishigemakoto\あの岡口裁判官も興味津々の様子
  \url{https://kk2020-09.blogspot.com/2021/06/makotoakishigecivilistaakishigemakoto\_8.html} 
\item
  2021年06月16日13時15分の登録:
  REGEXP:''岡口''/データベース登録済みツイートの検索:2021-06-07〜2021-06-16/2021年06月16日13時14分の記録:ユーザ・投稿:28/38件
  \url{https://kk2020-09.blogspot.com/2021/06/regexp2021-06-072021-06\_51.html} 
\item
  2021年06月16日19時57分の登録:
  REGEXP:''岡口''/サイ太(@uwaaaa)の検索(2013-10-21〜2021-06-04/2021年06月16日19時57分の記録86件)
  \url{https://kk2020-09.blogspot.com/2021/06/regexpuwaaaa2013-10-212021-06.html} 
\item
  2021年06月16日21時55分の登録:
  REGEXP:''岡口''/データベース登録済みツイートの検索:2021-06-16〜2021-06-16/2021年06月16日21時50分の記録:ユーザ・投稿:123/231件
  \url{https://kk2020-09.blogspot.com/2021/06/regexp2021-06-162021-06.html} 
\item
  2021年06月17日01時42分の登録:
  \伊藤たける|憲法マニアの弁護士@とやま @itotakeru\岡口判事、ついに弾劾裁判所に訴追。。。本人のブログによると、ラジオ出演も原因事実とのこと。寺西判事補事件の
  \url{https://kk2020-09.blogspot.com/2021/06/itotakeru.html} 
\item
  2021年06月17日02時29分の登録: \弁護士
  吉峯耕平(「カンママル」撲滅委員会) @kyoshimine\岡口裁判官は、それにしても酷い大惨事になってしまったね。裁判官は、Twitter怖がらないでほしい
  \url{https://kk2020-09.blogspot.com/2021/06/kyoshiminetwitter.html} 
\item
  2021年06月17日03時06分の登録:
  \宮武嶺 @raymiyatake\岡口判事を訴追、罷免を判断する「弾劾裁判」へ・・・弁護団「極めて遺憾」
  人として感心しないことをすることと、人権保障のために憲法上特に保障
  \url{https://kk2020-09.blogspot.com/2021/06/raymiyatake.html} 
\item
  2021年06月17日03時16分の登録:
  %@aphros67 小動物を愛するしんさん%なかなかわかりにくい裁判官の罷免とか懲戒処分だけれど、岡口基一判事がいることで身近になっている気はするよね(・ω・)¥\n¥\n
  \url{https://kk2020-09.blogspot.com/2021/06/aphros67nn.html} 
\item
  2021年06月17日08時34分の登録:
  \自称善良な弁護士 @Jisyou\_Zenryou\岡口さんが弾劾されるなら同程度の発言をした国会議員は除名の上公民権剥奪でしくよろ
  \url{https://kk2020-09.blogspot.com/2021/06/jisyouzenryou.html} 
\item
  2021年06月17日08時36分の登録:
  \高橋雄一郎 @kamatatylaw\もし岡口裁判官が弾劾裁判所で罷免の裁判を受けたら弁護士にもなれないんだよね。FBの投稿だけで職を失い数千万円の退職金も失い法曹資格
  \url{https://kk2020-09.blogspot.com/2021/06/kamatatylawfb.html} 
\item
  2021年06月17日10時30分の登録:
  \北白川 @GUv4i6\岡口さんの件は,表現の自由で勝負してしまうと共感されないだろうな。
  \url{https://kk2020-09.blogspot.com/2021/06/guv4i6\_17.html} 
\item
  2021年06月17日10時31分の登録:
  \過食B @motaberarenaiyo\頭脳や知識、経験が貴重なのであって、岡口さんに顧客対応して貰いたいわけではないというのは分かるw
  \url{https://kk2020-09.blogspot.com/2021/06/bmotaberarenaiyow.html} 
\item
  2021年06月17日10時32分の登録:
  \過食B @motaberarenaiyo\素人はともかく、プロで岡口さんが罷免相当だと思ってる人って、逆張り系みたいな人以外ほとんど聞かないけど、どうなんでしょうね。第
  \url{https://kk2020-09.blogspot.com/2021/06/bmotaberarenaiyo\_17.html} 
\item
  2021年06月17日17時43分の登録:
  \7286 @jmjhjmwtad\岡口判事の罷免によって、黒い法服に埋もれていた裁判官の権力性が、日本史上、初めて顕らかになる。その時、日本の司法の歴史が変わるだろう。
  \url{https://kk2020-09.blogspot.com/2021/06/7286jmjhjmwtad\_17.html} 
\item
  2021年06月17日17時47分の登録: \山口貴士
  aka無駄に感じが悪いヤマベン @otakulawyer\岡口判事のツイートほ良くないという評価と、罷免(裁判官クビだけではなく、退職金や法曹資格まで奪う)に
  \url{https://kk2020-09.blogspot.com/2021/06/akaotakulawyer.html} 
\item
  2021年06月17日22時00分の登録:
  \?ふくろう弁 @bgsh\_owl\あれで職を失うのでは裁判官の独立は絵に描いた餅でしょ。\#岡口基一判事弾劾訴追に反対する法曹の会
  \url{https://kk2020-09.blogspot.com/2021/06/bgshowl\_17.html} 
\item
  2021年06月18日04時15分の登録:
  REGEXP:''岡口''/データベース登録済みツイートの検索:2021-06-16〜2021-06-18/2021年06月18日04時06分の記録:ユーザ・投稿:230/668件
  \url{https://kk2020-09.blogspot.com/2021/06/regexp2021-06-162021-06\_18.html} 
\item
  2021年06月18日04時19分の登録:
  \高橋雄一郎 @kamatatylaw\キチガイ社長がいるブラック企業なら「洗脳するような組織だ」と組織批判したらすぐに従業員は懲戒解雇でしょう。法的効果は別にして。岡口
  \url{https://kk2020-09.blogspot.com/2021/06/kamatatylaw\_18.html} 
\item
  2021年06月18日04時21分の登録:
  \ささきりょう @ssk\_ryo\岡口裁判官の件、民間企業ならクビというツイートも散見されますが、よくない投稿をしたとしても、それで解雇が有効になるかは別ですよ。しかも退
  \url{https://kk2020-09.blogspot.com/2021/06/sskryo\_18.html} 
\item
  2021年06月19日11時49分の登録: %@katepanda2 弁護士
  太田啓子 「これからの男の子たちへ」(大月書店)%岡口裁判官の発言に批判はあるだろうけど、だからといってこれで弾劾訴追はいきすぎです。
  \url{https://kk2020-09.blogspot.com/2021/06/katepanda2\_19.html} 
\item
  2021年06月19日11時49分の登録: \弁護士
  上瀧浩子 @sanngatuusagino\\#岡口基一判事弾劾訴追に反対する法曹の会入りたい!
  \url{https://kk2020-09.blogspot.com/2021/06/sanngatuusagino.html} 
\item
  2021年06月19日11時51分の登録:
  \橋本太地(弁護士・あなたのみかた法律事務所) @kojin\_syugi\弾劾されるべき話ではない。
  \#岡口基一判事弾劾訴追に反対する法曹の会
  \url{https://kk2020-09.blogspot.com/2021/06/kojinsyugi.html} 
\item
  2021年06月19日11時51分の登録:
  %@yorinobu2 國本依伸%罷免はあり得ない。¥\n¥\n\#岡口基一判事弾劾訴追に反対する法曹の会
  \url{https://kk2020-09.blogspot.com/2021/06/yorinobu2nn.html} 
\item
  2021年06月19日11時54分の登録:
  REGEXP:''#岡口基一判事弾劾訴追に反対する法曹の会''/データベース登録済みツイートの検索:2021-06-17〜2021-06-18/2021年06月19日11時52分の記録:ユーザ・投稿:32/40件
  \url{https://kk2020-09.blogspot.com/2021/06/regexp2021-06-172021-06\_75.html} 
\end{itemize}

 時刻は6月21日09時15分です。寝る前に、ぽぽひとと柴田収弁護士についてまとめを作成したのですが、起きてからは深澤諭史弁護士の病気、病院、医者というツイートのまとめを作成しました。問題に切り込む糸口になりそうです。これは別のかたちでとりあげます。

 6月16日に作成した小倉秀夫弁護士とモトケンこと矢部善朗弁護士(京都弁護士会)の過去のツイートの記録作成についてご説明をしておきたいと思います。わずか5日ほど前ですが、ツイートの数も多いので、探し出すのも手間が掛かりそうです。

\begin{itemize}
\tightlist
\item
  ./hirono\_hideki2021-06-21\_090913.csv:2021-06-17 09:03:05 ``Amazon
  Photos \url{https://t.co/IdbWoht1zH} 
  2021-06-16\_名誉毀損の刑事告訴に関連したモトケンこと矢部善朗弁護士(京都弁護士会)のツイート''
  \url{https://twitter.com/hirono\_hideki/status/1405315024210456577} 
\item
  ./hirono\_hideki2021-06-21\_090913.csv:2021-06-17 09:15:42
  ``2021-06-16\_名誉毀損の刑事告訴に関連したモトケンこと矢部善朗弁護士(京都弁護士会)のツイート
  - Google フォト \url{https://t.co/ldOJp1LdhS''} 
  \url{https://twitter.com/hirono\_hideki/status/1405318197658611712} 
\end{itemize}

 先にモトケンこと矢部善朗弁護士(京都弁護士会)の方を見つけたのですが、翌日の作成となっていました。弁当を食べに外に出かける前に慌てて作っていたことを思い出しました。ただ時刻は9時3分と9時15分になっています。

 YouTubeの動画の方を探しているのですが、これが簡単に見つかりません。見つけやすくするためにタイトルに共通した接頭辞をつけるなど工夫が必要と思えてきました。

 Twilogから探すこそにしますが、6月16日の16時から18時頃の間とおよその絞り込みは出来ています。まだ5日前だから記憶にあるのですが、これが1年先になると、工夫としないともっと探すのに手間と時間が掛かりそうです。

\begin{itemize}
\tightlist
\item
  奉納\さらば弁護士鉄道・泥棒神社の物語(@hirono\_hideki)/2021年06月16日
  - Twilog \url{https://t.co/9Fr8mNxqSi} 
\end{itemize}

 あると思いこんでいたツイートが見当たりませんでした。YouTubeの方で、直接探します。

\begin{itemize}
\tightlist
\item
  2021年06月16日17時08分25秒の記録_名誉毀損の刑事告訴に関連した小倉秀夫弁護士(東京弁護士)のツイートの記録
  - YouTube \url{https://t.co/RqdHDWCXcX}  0 回視聴•2021/06/16
\end{itemize}

 Twilogの検索で見つかりませんでした。やはりTwitterには未投稿だったようです。この小倉秀夫弁護士の動画はすんなり投稿できたのですが、その次のモトケンこと矢部善朗弁護士(京都弁護士会)の動画は4回投稿に失敗していました。

 今までになかったことなので気がつくのに時間がかかったのですが、パソコンでも、そのmp4ファイルの再生が出来ず、はっきり憶えていないですが、「無効なコンテンツが含まれています」などというエラーが出ていました。

 スクリーンショットの作成を考えたのですが、Amazonプライムフォトだと、日付情報のないファイルとして扱われるのです。写真の並び順ですが、選択できるのは日付順と投稿順の2択のようです。問題なのはファイル名での並び替えが出来ないことです。

 何度か同じご説明をしていると思いますが、スマホやデジカメで撮影したファイルもパソコンで撮ったスクリーンショットも同じjpg形式ですが、スマホやデジカメで撮影したファイルはEXIF情報というものを持っていて、撮影日時という情報も保持しています。

\begin{itemize}
\tightlist
\item
  写真に自動登録される「Exif情報」って何? 知っておくべき安全な使い方 -
  チエネッタ \url{https://t.co/FzZQIAWn5L} 
\end{itemize}

 私が作成するスクリーンショットは、「2021-06-21-015332\_固定されたツイート弁護士 柴田収@DV・モラハラ離婚案件がメイン@themis\_okayama·6月18日DVやモラハラをする人は「感情をコ.jpg」という形式で、日付時間情報が接頭辞になるので、必ず時系列で並びます。

 これまでにいくつか失敗があって修正はしているのですが、プログラムを使った写真のまとめ記事では、最終更新日時で写真が並ぶことがあり、撮影した写真の縦横を変換することは多いのですが、それで見出しと写真の対応に齟齬が生じていました。

\begin{itemize}
\tightlist
\item
  2021-06-16\_名誉毀損の刑事告訴に関連したモトケンこと矢部善朗弁護士(京都弁護士会)のツイート
  - Google フォト \url{https://t.co/JeymW6VBHm} 
\end{itemize}

 Googleフォトのアルバムです。「2021-06-16\_180518_名誉毀損の刑事告訴に関連したモトケンこと矢部善朗弁護士(京都弁護士会)のツイート.jpg」から始まっています。6月16日18時05分18秒が撮影日時です。

 弁護士による民事裁判でもスクリーンショットの画像が印刷されて使われているようですが、スマホでのパソコンの画面の撮影というのは稀にしか見かけないものです。便利なサービスやアプリのあるのでスクリーンショットの方が手軽というのもあるのかもしれません。

 5年ほど前になるのか、スクリーンショットにはURLを含めた方が断然良いという情報も見かけていたのですが、最近見かけるスクリーンショットにURLを含むものはほとんどなく、パソコンよりスマホでの撮影が多いという印象もあります。

 また、GoogleフォトではGoogleにログインしていない状態でも、写真を個別に選択し、ファイル名の情報が見れることは確認していますが、Amazonフォトでは、投稿者のアカウント以外は、ファイル名を見ることが出来ない仕様となっているようです。

 ちょっと試しにやってみたのですが、AmazonPhotosで共有にした写真ファイルを、サインインしていない状態でダウンロードしたところ、元のファイル名と同じ状態でダウンロードが出来ました。複数のファイルのダウンロードはZIPのアーカイブになるはずです。

 スマホでの撮影ですが、標準のカメラではなく、サイズを小さくするアプリを使って撮影しています。1つの写真ファイルの、データ量は「448.6~kB
(448,630 バイト)」といった感じです。サイズは1920x1080です。

 裁判の記録とする場合は、PDFファイルとして保存した方がよいのかもしれません。前にそれらしい情報を少し見かけたことがあるのですが、jpgからの変換は簡単にできるはずなので、必要があればいつでもできるという考えです。

 スクリーンキャストの動画の作成に失敗したモトケンこと矢部善朗弁護士(京都弁護士会)のツイートですが、スマホでの撮影を行ったところいい感じだったので、動画の作成に成功していた小倉秀夫弁護士のツイートも同じく写真撮影で記録しました。

\begin{itemize}
\tightlist
\item
  ./hirono\_hideki2021-06-21\_090913.csv:2021-06-17 10:47:59 ``Amazon
  Photos \url{https://t.co/mcjKmwk7jj} 
  2021-06-17\_名誉毀損の刑事告訴に関連した小倉秀夫弁護士(東京弁護士会)のツイートの記録''
  \url{https://twitter.com/hirono\_hideki/status/1405341421838553088} 
\item
  ./kk\_hirono2021-06-21\_102503.csv:2021-06-21 09:44:21
  ``2021年06月16日17時08分25秒の記録_名誉毀損の刑事告訴に関連した小倉秀夫弁護士(東京弁護士)のツイートの記録
  - YouTube \url{https://t.co/RqdHDWCXcX}  0 回視聴•2021/06/16''
  \url{https://twitter.com/kk\_hirono/status/1406774957837742082} 
\end{itemize}

 まず、AmazonPhotosですが、ページ右上のサインインの下に、四角に丸い点が3つ並んだアイコンのようなものがあって、それをクリックすると「すべてダウンロード」というメニューがありました。91.5MBというサイズですが1分は掛からなかったと思います。

 すでに何度かやったことはあるのですが、AmazonPhotos.zipというファイル名で保存されていました。このzipのアーカイブはWindowsパソコンの標準機能で展開できるはずです。

 どうも先程の検索結果にGoogleフォトが含まれていないことに気が付きました。ブログ記事としても投稿しているはずです。

\begin{itemize}
\tightlist
\item
  2021年06月17日09時28分の登録:
  2021年06月17日の記録:写真資料:2021-06-16\_名誉毀損の刑事告訴に関連したモトケンこと矢部善朗弁護士(京都弁護士会)のツイート
  \url{https://kk2020-09.blogspot.com/2021/06/202106172021-06-16.html} 
\item
  2021年06月17日11時07分の登録:
  2021年06月17日の記録:写真資料:2021-06-17\_名誉毀損の刑事告訴に関連した小倉秀夫弁護士(東京弁護士会)のツイートの記録
  \url{https://kk2020-09.blogspot.com/2021/06/202106172021-06-17.html} 
\end{itemize}

 あとに作成した小倉秀夫弁護士のツイートの記録が「2021年06月17日11時07分の登録」となっていました。弁当を食べに出かける直前の作業でした。

 小倉秀夫弁護士のツイートの記録は139枚ありますが、記事に掲載されている画像・写真は全てGoogleフォトのアルバムのものです。Bloggerの投稿のときにアルバムから指定しています。これだと共有設定の必要はなく共有設定をしていないかもしれません。

 Googleフォトの共有設定がなければ、アルバムの写真をまとめてダウンロードすることもできそうにないですが、必要に応じてAmazonPhotosでやっていただければと思います。

\begin{itemize}
\tightlist
\item
  2021-06-17\_名誉毀損の刑事告訴に関連した小倉秀夫弁護士(東京弁護士会)のツイートの記録
  - Google フォト \url{https://t.co/0YA2YHUP0q} 
\end{itemize}

 共有設定でリンクを取得しました。

 Googleにログインしていない状態ですが、全てダウンロードを実行したところ、「このページは動作していませんvideo-downloads.googleusercontent.com
では現在このリクエストを処理できません。」というエラーが出て実行できません。

 画像(スクリーンショット)のファイルを個別にダウンロードすることは出来ました。

\begin{itemize}
\tightlist
\item
  奉納\危険生物・弁護士脳汚染除去装置\金沢地方検察庁御中\_2020:
  2021年06月17日の記録:写真資料:2021-06-17\_名誉毀損の刑事告訴に関連した小倉秀夫弁護士(東京弁護士会)のツイートの記録
  \url{https://t.co/AGpqIglbru} 
\end{itemize}

 上記の画像と写真をまとめた記事についてご説明をしておきたいと思います。2021-06-16-162301から2021-06-17\_094942まではスクリーンショットの画像ファイルになります。2021-06-17\_095004よりあとのものはスマホで撮影した写真ファイルです。

 スクリーンショットには時間の流れと作業の経過が記録されていますが、深澤諭史弁護士のタイムラインで深澤諭史弁護士にリツイートされた刑裁サイ太のツイートを見たのが始まりでした。

 これだけでも十分に弁護士鉄道の資料として記録に値すると判断したのですが、タイムラインから個別にツイートを開いていくと、小倉秀夫弁護士が刑裁サイ太のツイートに引用ツイートをしていたことが判明したのです。その内容も物凄いものを感じました。

 2021-06-16-191707\_過食B@motaberarenaiyo·1時間まあ色々アレだけど、他に酷い裁判官がゴマンといる中で、1人だけってのがね。.jpg がこのまとめ記事の中では最初に記録した岡口基一裁判官の弾劾訴追決定関係のツイートでした。

 過食弁護士のタイムラインですが、過食弁護士のツイートで朝日新聞社会部のツイートを引用しています。

 小倉秀夫弁護士とモトケンこと矢部善朗弁護士(京都弁護士会)のツイートをスクリーンショットや写真で記録したのは久しぶりですが、面倒もあるのでなかなか実行に踏み切るタイミングがありませんでした。全てのツイートの埋め込みツイートが表示されていることを確認し記録したことになります。

 小倉秀夫弁護士の名誉毀損での刑事責任追求の必要性を再確認したのですが、直後に出てきたのが岡口基一裁判官の弾劾訴追決定のニュースで、小倉秀夫弁護士は岡口基一裁判官と司法修習の同期というだけではなく、以前は支援する弁護団の一人として記者会見にも出ていました。

\begin{itemize}
\tightlist
\item
  2021年06月21日11時40分の登録:
  REGEXP:''岡口''/小倉秀夫(@Hideo\_Ogura)の検索(2015-03-19〜2019-05-14/2021年06月21日11時40分の記録48件)
  \url{https://kk2020-09.blogspot.com/2021/06/regexphideoogura2015-03-192019-05.html} 
\item
  2021年06月21日11時43分の登録:
  REGEXP:''岡口''/小倉秀夫(@chosakukenho)の検索(2020-10-13〜2021-06-19/2021年06月21日11時43分の記録10件)
  \url{https://kk2020-09.blogspot.com/2021/06/regexpchosakukenho2020-10-132021-06.html} 
\end{itemize}

 @Hideo\_Oguraは、更新が止まっている小倉秀夫弁護士のTwitterアカウントですが、本人の話では凍結され更新できなくなっているようです。Twitter社にツイートの削除を求められ、拒絶したまま反論のような記事をブログなどに投稿していました。

 小倉秀夫(@chosakukenho)の方が新たに作成された小倉秀夫弁護士のTwitterアカウントですが、最近はかなりの更新頻度で活発にツイートをしています。このアカウントで岡口基一裁判官に関するツイートは見覚えがなかったのですが、10件というツイートが記録されているようです。

\begin{itemize}
\tightlist
\item
  (05/10) TW chosakukenho(小倉秀夫) 日時:2021-01-04 17:01:00 +0900
  URL:
  \url{https://twitter.com/chosakukenho/status/1346003886868336641\textgreater} {}
  Twitter無視しても1割くらいは採算度外視していますね。昨年は、その枠の多くを岡口案件でとられてしまった感がありますが。
  \url{https://t.co/QZEcXArejr} 
\end{itemize}

 高橋雄一郎弁護士のツイートを引用していますが、今年の1月4日の小倉秀夫弁護士のツイートで、昨年は岡口案件とありますが、2020年は岡口基一裁判官に関して小康状態が続いていたという感覚です。それもあり時々思い出しては調べていました。

 岡口基一裁判官といえば、現在も仙台高裁判事ですが、昨年2020年の5月から6月頃は特に「死刑捏造:
松山事件・尊厳かけた戦いの末に」のことで仙台を舞台にした松山事件に注目していました。それと東京ミネルヴァ法律事務所の問題が勃発したのも昨年中であったように思います。

 岡口基一裁判官の問題ととても良く似た構図に思えたのですが、その東京ミネルヴァ法律事務所の被害者救済に名乗りを上げた弁護団がいて、その中心の弁護士が仙台の弁護士でした。日弁連の会長になった荒中弁護士も仙台の弁護士として注目となっていました。

 日弁連の会長の任期は2年となっていたように思います。たぶん今年の4月が2期目の突入になるはずです。

\begin{itemize}
\item
  東京ミネルヴァ法律事務所破産被害対策全国弁護団 \url{https://t.co/jbkEsgEfqz} 
  弁護団長 新里宏二(仙台弁護士会)
\item
  東京 仙台 新幹線 料金 - Google 検索 \url{https://t.co/iyVfD9iKJT} 
\end{itemize}

 ちょっと調べてみたのですが、自由席でも1万円ちょっとはするようです。片道の料金だと思うので、往復の割引のことまで調べないですが2万円ぐらいにはなりそうです。新幹線なので時間は早いのでしょう。

 余り考えたこともなかったのですが他に調べてみると、東京から仙台は約1時間32分、東京から青森が約3時間という所要時間でした。国道4号線と国道6号線で東京から仙台は何度か行ったことがあるのですが、ずいぶん時間がかかったという記憶はあります。

 仙台駅の駅前に法律事務所があれば別ですが、仙台市内でも移動に時間とお金は掛かりそうです。弁護士がバスを利用するというのも考えにくいですが、東京都内だとさらに混雑が予想されます。

\begin{itemize}
\tightlist
\item
  (10/10) TW chosakukenho(小倉秀夫) 日時:2021-06-19 10:49:37 +0900
  URL:
  \url{https://twitter.com/chosakukenho/status/1406066607311388677\textgreater} {}
  法曹で服装について文句を言われるのって、岡口さんくらいだと思っていた。
  \url{https://t.co/NIKmZ9QJtS} 
\end{itemize}

 見覚えのあるツイートが最後に出てきましたが、6月19日のツイートでした。

\begin{itemize}
\tightlist
\item
  (21/48) TW Hideo\_Ogura(小倉秀夫) 日時:2018-09-12 09:26:00 +0900
  URL:
  \url{https://twitter.com/Hideo\_Ogura/status/1039671498171006976\textgreater} {}
  岡口判事の分限裁判について|小倉秀夫|note(ノート)
  \url{https://t.co/44m18RYxjN} 
\end{itemize}

 岡口基一裁判官の分限裁判というのは2018年のことであったようです。当時は岡口基一裁判官を養護する声も大きく、逆に上司の裁判官らの責任を追求する声もあり、実際に刑事告発という動きもあったかもしれません。

\begin{itemize}
\item
  (29/48) TW Hideo\_Ogura(小倉秀夫) 日時:2018-10-18 21:27:00 +0900
  URL:
  \url{https://twitter.com/Hideo\_Ogura/status/1052898879417462785\textgreater} {}
  岡口判事の代理人として主張書面の作成を分担したというのに、これ以上何をせよと。RT
  @toshi9monsters:
  フェミニズムに色々とやるべきなどと押し付ける表現の自由戦士の方々は、キズナアイやラノベ表紙には執着すれど、岡口裁・・・
  \url{https://t.co/cHhfZMGwBK} 
\item
  (42/48) TW Hideo\_Ogura(小倉秀夫) 日時:2019-04-11 21:34:00 +0900
  URL:
  \url{https://twitter.com/Hideo\_Ogura/status/1116318563411714048\textgreater} {}
  研修所のクラスメートです。RT @kyoshimine:
  岡口裁判官の弁護団、小倉先生と伊藤先生がいるのも、すげぇよな。
\end{itemize}

 クラスメートという言葉も最近は見かけておらず、ずいぶん久しぶりに目にしたように思いました。「弁護士列車にゆられてゆれて〜」という歌詞で、「あゝ上野駅」という曲が頭に浮かんできますが、仙台行きというのも上野駅なのかと考えています。

 暫く前まで岡口基一裁判官の弾劾訴追は立ち消えになる見方が有力という印象もあったのですが、既定のレールが敷設されていたようにも思え、やはり仙台高裁判事という人事は特別なもので、事実上、北の流刑地のような意味合いがあったのかと思えてきます。

\begin{itemize}
\tightlist
\item
  (45/48) TW Hideo\_Ogura(小倉秀夫) 日時:2019-04-17 14:21:00 +0900
  URL:
  \url{https://twitter.com/Hideo\_Ogura/status/1118383941541617664\textgreater} {}
  特定の結論を導くために柔軟な法解釈や事実認定をした例として岡口分限裁判があり、しなかった例としてくだんの無罪判決がある。
\end{itemize}

 小倉秀夫弁護士の@Hideo\_Oguraが凍結となったのはずいぶん前のことと思っていたのですが、平成から令和に移り変わる時期にはまだ健在だったようです。更新が途絶えたのは6月として記憶にあるので、2019年6月だと思います。

\begin{itemize}
\tightlist
\item
  (46/48) TW Hideo\_Ogura(小倉秀夫) 日時:2019-04-21 00:37:00 +0900
  URL:
  \url{https://twitter.com/Hideo\_Ogura/status/1119626210894499840\textgreater} {}
  実際に性犯罪に関する裁判例が裁判所のウェブサイトで公開し、岡口さんがそれに言及したところ、厳重注意を受けたので、もうウェブでは公開できないんじゃないですかね。RT
  @kyoshimine: 小倉先生も「岡口裁判官問題があり、裁判所・・・
  \url{https://t.co/FPi5rKPiC3} 
\end{itemize}

 小倉秀夫弁護士のらしさを感じましたが、これで弁護士を続けているのも事実なのですから問題は深刻に受け止め、対処を講じる必要があると思います。

\begin{itemize}
\tightlist
\item
  (47/48) TW Hideo\_Ogura(小倉秀夫) 日時:2019-04-22 17:18:00 +0900
  URL:
  \url{https://twitter.com/Hideo\_Ogura/status/1120240361056550913\textgreater} {}
  導きたい結論に向かって事実認定をねじ曲げるって、「えっ、なんて岡口分限裁判likeな!」って話にしかなりませんよね。
\end{itemize}

 likeというのは似ているという意味だと思います。データベース操作のSQL文にもLIKEというあいまい検索のようなものがあります。

\begin{itemize}
\item
  【SQL】意外と簡単!これならわかるLIKE句でのあいまい検索 \textbar{}
  侍エンジニアブログ \url{https://t.co/Hzc2B9w8j6} 
\item
  (48/48) TW Hideo\_Ogura(小倉秀夫) 日時:2019-05-14 20:25:00 +0900
  URL:
  \url{https://twitter.com/Hideo\_Ogura/status/1128260131081166850\textgreater} {}
  岡口裁判官があんな目に遭っても何の手も差し伸べなかった人たちですからね。RT
  @kitaguni\_b:
  こんな高裁判決確定されたらシャレにならない。弁護士は業務妨害されても多少は我慢しろと言われるのは、弁護士全体への挑戦状と受け止めた。
\end{itemize}

 そういえば小倉秀夫弁護士は司法修習の任地を新潟として懐かしんだり、良かったと回想するようなツイートを見かけてきましたが、岡口基一裁判官のブログやツイートで新潟の話は見たことがなかったように思います。

 小倉秀夫弁護士は高島章弁護士(新潟県弁護士会)とも司法修習で同じだったと話していたので、同期だと思っていたのですが、修習が半年間重なったような話で、私がデマを流したように咎めるツイートを寄越していました。先程も見かけましたが高島弁護士が1期上とのことです。

\begin{itemize}
\tightlist
\item
  〈〈〈 2021/06/21 17:05:23 Linux Emacs: 〈〈〈
\end{itemize}

\hypertarget{section}{%
\paragraph{}\label{section}}

\hypertarget{ux91d1ux6ca2ux5730ux65b9ux88c1ux5224ux6240}{%
\subsection{金沢地方裁判所}\label{ux91d1ux6ca2ux5730ux65b9ux88c1ux5224ux6240}}

\hypertarget{ux4e09ux5b85ux4fcaux4e00ux90ceux88c1ux5224ux9577ux5c71ux7530ux5fb9ux88c1ux5224ux5b98ux5dddux53e3ux6cf0ux53f8ux88c1ux5224ux5b98}{%
\subsubsection{三宅俊一郎裁判長,山田徹裁判官,川口泰司裁判官}\label{ux4e09ux5b85ux4fcaux4e00ux90ceux88c1ux5224ux9577ux5c71ux7530ux5fb9ux88c1ux5224ux5b98ux5dddux53e3ux6cf0ux53f8ux88c1ux5224ux5b98}}

\hypertarget{ux6afbux4e95ux5149ux653fux5f01ux8b77ux58ebux306eux30c6ux30ecux30d3ux30c9ux30e9ux30deux30a4ux30c1ux30b1ux30a4ux306eux30abux30e9ux30b9ux306bux95a2ux3059ux308bux30c4ux30a4ux30fcux30c8ux304cux304dux3063ux304bux3051ux3067ux9593ux9055ux3044ux306bux6c17ux304cux3064ux3044ux305fux5de6ux966aux5e2dux306eux4f4dux7f6e}{%
\paragraph{櫻井光政弁護士のテレビドラマ「イチケイのカラス」に関するツイートがきっかけで間違いに気がついた左陪席の位置}\label{ux6afbux4e95ux5149ux653fux5f01ux8b77ux58ebux306eux30c6ux30ecux30d3ux30c9ux30e9ux30deux30a4ux30c1ux30b1ux30a4ux306eux30abux30e9ux30b9ux306bux95a2ux3059ux308bux30c4ux30a4ux30fcux30c8ux304cux304dux3063ux304bux3051ux3067ux9593ux9055ux3044ux306bux6c17ux304cux3064ux3044ux305fux5de6ux966aux5e2dux306eux4f4dux7f6e}}

\begin{itemize}
\tightlist
\item
  〉〉〉 Linux Emacs: 2021/05/07 11:04:16 〉〉〉
\end{itemize}

:CATEGORIES: @kanazawabengosi \#金沢弁護士会 @JFBAsns
日本弁護士連合会(日弁連) \#法務省 @MOJ\_HOUMU \#櫻井光政弁護士

〉〉〉 kk\_hironoのリツイート 〉〉〉

\begin{itemize}
\tightlist
\item
  RT
  kk\_hirono(刑事告発・非常上告_金沢地方検察庁御中)|okinahimeji(櫻井光政)
  日時:2021-05-07 11:05/2021/05/04 21:07 URL:
  \url{https://twitter.com/kk\_hirono/status/1390488033070182402} 
  \url{https://twitter.com/okinahimeji/status/1389552302663499778} 
  \textgreater{}
  因みにフジテレビの番組審議会でも、委員の梓沢和幸先生が「裁判官は捜査しないぞ」って、念のため指摘されたそうです。娯楽作品だからいいけどってなったらしいですが。
\end{itemize}

〉〉〉 kk\_hironoのリツイート 〉〉〉

\begin{itemize}
\tightlist
\item
  RT
  kk\_hirono(刑事告発・非常上告_金沢地方検察庁御中)|okinahimeji(櫻井光政)
  日時:2021-05-07 11:05/2021/05/04 18:05 URL:
  \url{https://twitter.com/kk\_hirono/status/1390488046932398081} 
  \url{https://twitter.com/okinahimeji/status/1389506454034337795} 
  \textgreater{}
  みーんなが感動したり面白いと言っているのに茫然自失の君たち、なまじ刑事訴訟法の知識があることが娯楽の邪魔をしている君らは法曹関係者、無言で浮かない顔をしている君は裁判官だな。大丈夫だ。リアルでは左陪席が裁判長をさせられることはないし、捜査もしない。#イチケイのカラス
\end{itemize}

〉〉〉 kk\_hironoのリツイート 〉〉〉

\begin{itemize}
\tightlist
\item
  RT
  kk\_hirono(刑事告発・非常上告_金沢地方検察庁御中)|68fbK1hZxgkjEsc(yt)
  日時:2021-05-07 11:06/2021/05/04 01:51 URL:
  \url{https://twitter.com/kk\_hirono/status/1390488122689949697} 
  \url{https://twitter.com/68fbK1hZxgkjEsc/status/1389261290648915973} 
  \textgreater{}
  イチケイのカラス、ドラマが面白かったので原作を読んだら、違いが大きくて驚いた。どちらも、それぞれの良さがあると感じた。
\end{itemize}

〉〉〉 kk\_hironoのリツイート 〉〉〉

\begin{itemize}
\tightlist
\item
  RT
  kk\_hirono(刑事告発・非常上告_金沢地方検察庁御中)|okinahimeji(櫻井光政)
  日時:2021-05-07 11:06/2021/05/02 10:22 URL:
  \url{https://twitter.com/kk\_hirono/status/1390488226129801216} 
  \url{https://twitter.com/okinahimeji/status/1388665070897352704} 
  \textgreater{} 京都府警のひき逃げ事件捜査 弁護士が異例の立ち会い|NHK
  関西のニュース \url{https://t.co/oac4euEVpI} 
\end{itemize}

 櫻井光政弁護士のタイムラインのツイート4件を告発\市場急配センター殺人未遂事件\金沢地方検察庁・石川県警察御中(@kk\_hirono)でリツイートしました。

 4件目の「京都府警のひき逃げ事件捜査 弁護士が異例の立ち会い|NHK
関西のニュース」というツイートにあるニュースは,一部の弁護士らの話題となっていましたが,Twitterなど全般的な関心は乏しいようで,私もすぐに忘れていました。

 イチケイのカラスというテレビドラマですが,録画予約はしていたのですが,3回目の放送で確認したところ,その放送で予約に失敗していました。

 それというのも21時はずっと前からNHKのNEWS9というニュース番組を毎回予約にしていて,そちらの録画が始まっていました。それにしては1,2回目の録画がそのまま「イチケイのカラス」で成功している様子だったので,不思議には思いました。

 その数日後になりますが,テレビドラマ「イチケイのカラス」の初回放送の録画を作成したのですが,10分ぐらいか観て視聴をやめたままです。線路の踏切で小さい女の子がひかれそうになり,助けた年配の男性がなくなり,被害者の母親が偽証を迫られたというような内容でした。

 裁判の合議体には裁判長の他,左陪席と右陪席があって,右陪席と左陪席の格の違いのものも知っていたのですが,傍聴席から向かって右側が右陪席と考えていました。どうもそれが間違いで,裁判長からみた右側が右陪席となるようです。

 この個人的な勘違いには理由があるのですが,平成4年の傷害・準強姦被告事件の公判で,三宅俊一郎裁判長に向かった右手が川口泰司裁判官でした。

 公判で裁判官の自己紹介はなかったという記憶です。3人の真ん中にいるのが裁判長という認識は当然にありました。なぜ川口泰司裁判官と山田徹裁判官を取り違えずに認識していたのか,今考えると不思議な気もするのですが,判決文の署名の並びで認識した可能性はあると思います。

 私が山田徹裁判官と思っていた人物が山田徹裁判官であることは,もう7,8年ほど前になるのかFacebookの山田徹裁判官の写真で確認しています。正確には山田徹弁護士で,大阪府高槻市で弁護士として法律事務所を開いているという話でした。

 これは個人的な認識としてテレビドラマ「イチケイのカラス」と他生の縁があるのですが,前に女優の黒木華について調べたところ,その大阪府高槻市の出身となっていました。

 大阪府高槻市は国道171号線沿いで,昭和59年には金沢市場輸送の4トン保冷車で,京都市内から神戸方面に向かうのによく通過したという思い出があり,特に印象深いのが,平成3年の10月かあるいは11月,高槻市の市場で馬鈴薯を荷降ろしし,神戸市のポートアイランドに向かった運行です。

\begin{itemize}
\tightlist
\item
  2021年05月07日11時39分の登録:
  「高槻市」を@hirono\_hideki @kk\_hirono @s\_hironoで検索 28件の該当 2021-05-07\_11:39の記録
  \url{https://kk2020-09.blogspot.com/2021/05/hironohidekikkhironoshirono282021-05.html} 
\item
  2021年05月07日11時39分の登録:
  「黒木華」を@hirono\_hideki @kk\_hirono @s\_hironoで検索 36件の該当 2021-05-07\_11:38の記録
  \url{https://kk2020-09.blogspot.com/2021/05/hironohidekikkhironoshirono362021-05\_7.html} 
\item
  2021年05月07日11時40分の登録:
  「(ポートアイランド\textbar 港島)」を@hirono\_hideki @kk\_hirono @s\_hironoで検索 7件の該当 2021-05-07\_11:40の記録
  \url{https://kk2020-09.blogspot.com/2021/05/hironohidekikkhironoshirono72021-05.html} 
\end{itemize}

 まず,「高槻市」の結果をざっとながめていて,記憶の喚起となったのが次の16件目で,ちょうど昨日に思い出しながら視聴していたYouTube動画や,その前の医王山でのニュースのツイートになります。拡張機能で投稿直後に表示されたツイートでした。

2018-06-19 19:11:34
``意識が回復したという兄も、詳細な怪我の程度や回復の状態も不明です。プライバシーが優先されたような石川県内での報道となっていました。すぐに防護柵が設置されたり、安全対策が喫緊の課題とされたところは、昨日の高槻市でのブロック塀の事故の報道とも似ています。''
\url{https://twitter.com/kk\_hirono/status/1009015827142565888} 

〉〉〉 kk\_hironoのリツイート 〉〉〉

\begin{itemize}
\tightlist
\item
  RT
  kk\_hirono(刑事告発・非常上告_金沢地方検察庁御中)|hirono\_hideki(奉納\さらば弁護士鉄道・泥棒神社の物語)
  日時:2021-05-07 11:52/2021/05/06 09:07 URL:
  \url{https://twitter.com/kk\_hirono/status/1390499648297570304} 
  \url{https://twitter.com/hirono\_hideki/status/1390095880204603395} 
  \textgreater{}
  【喧嘩】旭川市立北星中学校の元校長の天下り先を特定!副町長と対決!?抗議で正論を言うと警察を呼ぶ行政!
  - YouTube \url{https://t.co/lBqwJXuLSQ}  627,028 回視聴 •2021/04/25
\end{itemize}

〉〉〉 kk\_hironoのリツイート 〉〉〉

\begin{itemize}
\tightlist
\item
  RT
  kk\_hirono(刑事告発・非常上告_金沢地方検察庁御中)|ishikawa\_tvnews(【石川テレビ】石川さん
  Live News イット! 💙) 日時:2021-05-07 11:52/2021/05/05 13:06 URL:
  \url{https://twitter.com/kk\_hirono/status/1390499660393959426} 
  \url{https://twitter.com/ishikawa\_tvnews/status/1389793674511273985} 
  \textgreater{} 【\#石川のニュース】\#石川テレビ
  母親「探してもらい感謝」・・・10歳と8歳の子供連れた父親が下山途中で行方不明に
  翌朝無事救助 \url{https://t.co/996SuyryL0} 
\end{itemize}

〉〉〉 kk\_hironoのリツイート 〉〉〉

\begin{itemize}
\tightlist
\item
  RT
  kk\_hirono(刑事告発・非常上告_金沢地方検察庁御中)|hirono\_hideki(奉納\さらば弁護士鉄道・泥棒神社の物語)
  日時:2021-05-07 11:52/2021/05/06 09:09 URL:
  \url{https://twitter.com/kk\_hirono/status/1390499687610740740} 
  \url{https://twitter.com/hirono\_hideki/status/1390096462407561221} 
  \textgreater{}
  母親「探してもらい感謝」・・・10歳と8歳の子供連れた父親が下山途中で行方不明に
  翌朝無事救助 \url{https://t.co/QISpMU5nYs} 
\end{itemize}

〉〉〉 kk\_hironoのリツイート 〉〉〉

\begin{itemize}
\tightlist
\item
  RT
  kk\_hirono(刑事告発・非常上告_金沢地方検察庁御中)|hirono\_hideki(奉納\さらば弁護士鉄道・泥棒神社の物語)
  日時:2021-05-07 11:52/2021/05/06 09:12 URL:
  \url{https://twitter.com/kk\_hirono/status/1390499722817798144} 
  \url{https://twitter.com/hirono\_hideki/status/1390097194913435648} 
  \textgreater{}
  目の前が``廃墟''旅館でイメージ悪化・・・人気温泉街の戦い方
  \url{https://t.co/fAOcmz2k3g} 
  会津若松市は「東日本大震災の影響で7年前から客足が減りました。また、福島県ということで、原発の風評被害もありました。5年前に廃業することになりましたが、その後、雪の重みで屋根が倒壊してしまいました」と、
\end{itemize}

〉〉〉 kk\_hironoのリツイート 〉〉〉

\begin{itemize}
\tightlist
\item
  RT
  kk\_hirono(刑事告発・非常上告_金沢地方検察庁御中)|hirono\_hideki(奉納\さらば弁護士鉄道・泥棒神社の物語)
  日時:2021-05-07 11:52/2021/05/06 09:30 URL:
  \url{https://twitter.com/kk\_hirono/status/1390499782901198849} 
  \url{https://twitter.com/hirono\_hideki/status/1390101519513493508} 
  \textgreater{}
  「今以上の炎上騒ぎに・・・」旭川で自称YouTuberの男逮捕 「いじめ報道」巡り10代女性に話を聞こうと迫ったか(STVニュース北海道)
  - Yahoo!ニュース \url{https://t.co/jng2OqT89D}  4/26(月) 19:15配信
\end{itemize}

〉〉〉 kk\_hironoのリツイート 〉〉〉

\begin{itemize}
\tightlist
\item
  RT
  kk\_hirono(刑事告発・非常上告_金沢地方検察庁御中)|bengo4topics(弁護士ドットコムニュース)
  日時:2021-05-07 11:53/2021/05/05 09:43 URL:
  \url{https://twitter.com/kk\_hirono/status/1390499879877677057} 
  \url{https://twitter.com/bengo4topics/status/1389742599112597513} 
  \textgreater{}
  刑事裁判官をテーマにしたフジ系月9ドラマ「イチケイのカラス」。駒沢義男部長のモデルの一人で元裁判官の木谷明弁護士は「刑事裁判の現状は絶望的」として、現実でも入間みちおのような弁護士出身裁判官を増やすべきと語る。
  \url{https://t.co/K2Xw6zEIdY} 
\end{itemize}

〉〉〉 kk\_hironoのリツイート 〉〉〉

\begin{itemize}
\tightlist
\item
  RT
  kk\_hirono(刑事告発・非常上告_金沢地方検察庁御中)|hirono\_hideki(奉納\さらば弁護士鉄道・泥棒神社の物語)
  日時:2021-05-07 11:53/2021/05/06 10:07 URL:
  \url{https://twitter.com/kk\_hirono/status/1390499890527019008} 
  \url{https://twitter.com/hirono\_hideki/status/1390110986242195459} 
  \textgreater{} - そして5人は帰らなかった ① 吾妻連峰・雪山遭難を辿る -
  YouTube \url{https://t.co/a6F7HiTuVw} 
\end{itemize}

 最後のリツイートにある「そして5人は帰らなかった ①
吾妻連峰・雪山遭難を辿る」というYouTube動画は,京都弁護士会で副会長もしていた秋重実弁護士のタイムラインで見かけたものでした。

〉〉〉 kk\_hironoのリツイート 〉〉〉

\begin{itemize}
\tightlist
\item
  RT
  kk\_hirono(刑事告発・非常上告_金沢地方検察庁御中)|hirono\_hideki(奉納\さらば弁護士鉄道・泥棒神社の物語)
  日時:2021-05-07 11:55/2021/05/06 10:55 URL:
  \url{https://twitter.com/kk\_hirono/status/1390500536453980160} 
  \url{https://twitter.com/hirono\_hideki/status/1390123094010843136} 
  \textgreater{} -
  新潟県五頭連峰親子遭難 調査登山 魚止滝~山葵山~松平山~大雪渓手前まで往復 2018年5月21日実施
  - YouTube \url{https://t.co/MDdluiKFYy} 
\end{itemize}

〉〉〉 kk\_hironoのリツイート 〉〉〉

\begin{itemize}
\tightlist
\item
  RT
  kk\_hirono(刑事告発・非常上告_金沢地方検察庁御中)|hirono\_hideki(奉納\さらば弁護士鉄道・泥棒神社の物語)
  日時:2021-05-07 11:55/2021/05/06 11:11 URL:
  \url{https://twitter.com/kk\_hirono/status/1390500563738009600} 
  \url{https://twitter.com/hirono\_hideki/status/1390126925817933828} 
  \textgreater{} スタンレーの魔女(漫画)- マンガペディア
  \url{https://t.co/pA6Kic6s9H} 
\end{itemize}

 部分的に少ししか視聴していないのですが,「新潟県五頭連峰親子遭難 調査登山 魚止滝~山葵山~松平山~大雪渓手前まで往復 2018年5月21日実施」という動画が,再生終了後に出てきました。

 新潟県五頭連峰親子遭難とある遭難事故ですが,石川県小松市の粟津駅の列車接触事故もちょうど同じ頃の報道があったように思います。確か幼い兄弟の弟はその場ですぐに死亡したが,兄は意識不明の後,意識が回復したというニュースで,それで報道も途切れたように思います。

 新潟県五頭連峰親子遭難も確か,捜索隊に発見されるまでかなりの日数があり,それだけ報道の数も多かったと記憶にあります。まるで神隠しのような印象もあったのですが,同じ頃に同じ新潟県の湯沢市でも男児の行方不明で不可解なことがあったというニュースがあったと思います。

\begin{itemize}
\tightlist
\item
  ./hirono\_hideki2021-05-07\_110902.csv:2018-05-19 16:00:08 ``RT
  @key\_ichi: 警察が男児の情報提供求める|NHK 新潟県のニュース
  \url{http://www3.nhk.or.jp/lnews/niigata/20180519/1030003241.html} 
  >18日夜、湯沢町の住宅に小学校の低学年と見られる男の子が訪れたあと、そのまま立ち去って行方がわかっておらず、警察は情報提供を求めています。
  なにこれオカルト \url{http://pic.twitter.com/eTEkSXIulz''} 
  \url{https://twitter.com/hirono\_hideki/status/997733629580070912} 
\end{itemize}

 湯沢市は間違いでした。秋田県湯沢市が菅首相の出身地ということも忘れていましたが,長距離トラックの仕事では,確か国道13号線沿いで,秋田市,盛岡市への通過点としてよく行った場所になります。新潟県の湯沢町も関越道の湯沢インターがあり,よく行っていました。

\begin{itemize}
\item
  ./kk\_hirono2021-05-07\_120904.csv:2020-01-15 16:13:15 ``RT
  @hirono\_hideki: 新潟・五頭連山で遭難した親子とみられる遺体を発見
  \url{http://news.tv-asahi.co.jp/news\_society/articles/000128334.html''} 
  \url{https://twitter.com/kk\_hirono/status/1217343975163514881} 
\item
  ./kk\_hirono2021-05-07\_120904.csv:2020-01-15 16:11:03
  ``新潟県の阿賀野市の辺りは2年ほど前に親子の遭難のニュースがあって、粟津駅の幼い兄弟の列車の事故と時期が近かったのですが、気になるところがあって、周辺の地図もGoogleマップで調べたということがありました。''
  \url{https://twitter.com/kk\_hirono/status/1217343422949756928} 
\item
  ./kk\_hirono2021-05-07\_120904.csv:2018-08-11 17:32:02
  ``どうも新潟県の牛頭連山での遭難した親子の遺体の発見と、石川県小松市のJR粟津駅の幼い兄弟の死傷事故で意識不明だった7歳の兄の意識が回復したというニュースは同じ日の報道だったようです。''
  \url{https://twitter.com/kk\_hirono/status/1028197337863151616} 
\item
  ./kk\_hirono2021-05-07\_120904.csv:2018-08-11 17:22:15 ``RT
  @hirono\_hideki: 新潟・五頭連山で遭難した親子とみられる遺体を発見
  \url{http://news.tv-asahi.co.jp/news\_society/articles/000128334.html''} 
  \url{https://twitter.com/kk\_hirono/status/1028194875257184257} 
\item
  父子の山岳遭難、連絡ミスで初動に遅れ 新潟県警が謝罪:朝日新聞デジタル
  \url{https://t.co/2omv196e6m}  2018年5月11日 0時54分
\item
  不明の親子か、五頭連山で2人の遺体発見 新潟・阿賀野:朝日新聞デジタル
  \url{https://t.co/eY1crkyhwc}  2018年5月29日 14時48分
\end{itemize}

 遭難が5月5日で,6日に下山するという電話連絡があり,遺体が発見されたのが同月29日とのことです。2週間ぐらいと思っていたのですが25日ほど間があったようです。2018年だと思いますが,同じ頃に新潟市西区の女児殺害事件があったと思います。これも連日の報道でした。

\begin{itemize}
\tightlist
\item
  男児2人死傷列車事故で実況見分 現場の線路付近 石川:朝日新聞デジタル
  \url{https://t.co/mDeH4Py3pm} 
  JR西日本金沢支社などによると、2人は25日午後6時45分ごろ、ホームの端から1メートル離れた下り線の本線と待避線の間に倒れた状態で見つかった。
\end{itemize}

 新潟市西区の女児殺害事件ですが,電車の事故と見せかけた特異な事件でした。控訴審に入ってから報道がほとんどなくなったのですが,死刑が求刑され,無期懲役の判決が出ていたように思います。不思議なほどピタリと報道がやんだ事件でもありました。

\begin{quote}
《引用の始まり》
\end{quote}

\begin{quote}
概要[編集]会社員の男K(当時23歳)は仕事帰りに乗っていた軽自動車で、小学2年生の女児Aに意図的に衝突させ、転倒した女児Aを軽自動車の後部座席に乗せた。

Kは電気工事士として働いており、周囲の人らからによると「優しくて明るい子」という評判で[1]、学生時代からアニメ好きの友人らと交流する「オタク」として知られていた[2]。反面、ネットで知り合った未成年者と性交に及ぶなど小児性愛の気質もあった[3]。

Aは「頭が痛い、お母さんに連絡したい」と泣いたが、Kは首を絞めて気絶させた[4]。その後、海岸沿いの広場にAを連れて行ったKは、Aに対して下半身を触れるなどわいせつな行為を行ったが、Aが意識を取り戻したため5分に渡り首を絞め、Aを殺害した。

Aの死体を自宅に持ち帰ったKは、Aの死体への性的暴行を行った後、JR東日本越後線の線路内にAの遺体を遺棄。22時30分頃に走行した普通電車にその遺体を轢過させ[5]、頸部を切断させた。

同年5月14日、KはAへの殺害容疑で逮捕。その後、Kは殺人罪だけでなく、Aへのわいせつ略取、強制わいせつ致死、死体損壊、死体遺棄、電汽車往来危険の罪。別の女児のポルノ画像を所有していた児童ポルノ禁止法違反で起訴された。
\end{quote}

\begin{quote}
《引用の終わり》
\end{quote}

\begin{itemize}
\tightlist
\item
  新潟小2女児殺害事件 -
  Wikipedia \url{https://ja.wikipedia.org/wiki/\%E6\%96\%B0\%E6\%BD\%9F\%E5\%B0\%8F2\%E5\%A5\%B3\%E5\%85\%90\%E6\%AE\%BA\%E5\%AE\%B3\%E4\%BA\%8B\%E4\%BB\%B6n} 
\end{itemize}

\begin{quote}
《引用の始まり》
\end{quote}

\begin{quote}
控訴審・東京高等裁判所[編集]2020年9月24日に東京高等裁判所で控訴審の初公判が第二刑事部(大善文男裁判長)の元で開かれた。検察側は無期懲役とした1審・新潟地裁判決(19年12月)は著しく不当とし、改めて死刑を求めた。弁護側は有期懲役が妥当とした。控訴審で検察側は、被告Kは女児を連れ去ったことの発覚を防ぐため、少なくとも5分以上、首を絞め続けたとし、強固な殺意が明確に認められると主張した。弁護側は、首を絞めたのは気絶させて静かにさせるためで殺意はなかったと反論。無期懲役は重すぎると訴えた。[9]。

2021年3月1日(第2回公判)では、司法解剖を行なった医師が検察側の証人として出廷した。1審は、首を絞めた時間について認定しなかった事に対して、医師は「1審判決には法医学的に誤りがある」として、「少なくとも3分以上、あるいは4分以上首を絞めた」と強調した。
[10]。
\end{quote}

\begin{quote}
《引用の終わり》
\end{quote}

\begin{itemize}
\tightlist
\item
  新潟小2女児殺害事件 -
  Wikipedia \url{https://ja.wikipedia.org/wiki/\%E6\%96\%B0\%E6\%BD\%9F\%E5\%B0\%8F2\%E5\%A5\%B3\%E5\%85\%90\%E6\%AE\%BA\%E5\%AE\%B3\%E4\%BA\%8B\%E4\%BB\%B6n} 
\end{itemize}

 控訴審がどうなっているのか調べたのですが,初めて知ることが多くありました。報道はされていないのか,新潟県内だけで全国ニュースにはなっていないのかもしれないですが,思っていたよりずいぶん悪質な事件だったようです。

 わいせつ目的で女児を気絶させ,その海沿いの広場でのわいせつ行為中に女児が意識を戻したので,再度首を絞めて,今度は死亡させたという事実関係のようです。これは初めて知りました。室内での犯行を疑いもなくイメージしていました。

 自宅に女児の遺体を持ち帰り死体に性的暴行を加えたともあります。遺体を線路に放置し衝突させたことは繰り返しニュースになっていましたが,跳ね飛ばされたのか轢過でどういう状態になったのか不明でしたが,頭部の切断とあります。

 ずいぶんおとなしく真面目そうな被疑者だったのも印象的な事件でしたが,警察の連行で微笑むような表情を見せていたのも印象的な事件でした。わいせつ目的で気絶させるだけで本人は殺意がなかったのかもしれないですが,未必の故意が問題にされることもなかったと思います。

 一審の新潟地裁では「弁護側は殺意とわいせつ行為を否定し、傷害致死罪を主張して結審した{[}7{]}。」とありますが,わいせつ行為まで否定していて,遺体への性的暴行という検察主張の事実と,どのように整合するのか不思議過ぎます。弁護士脳の異常性が事実を無視しているのか・・・。

 「12月4日、Kの犯行が極めて悪質であることを認めながら、遺体を線路に遺棄した行為が殺害行為にはあたらないなど、死刑には慎重さと公平性が重視されるとして、無期懲役の{[}8{]}判決になった」という裁判所の判断のようです。考えて確認すると当然ながら裁判員裁判でした。

 少なくとも逮捕後は,直接被疑者本人の話がない事件だったと思いますが,本人が死刑を受け入れているのか,あわよくば有期懲役悪くて無期懲役と考えているのかよくわからない上,報道も過熱気味でありながら,弁護士が被告人の意向を無視して被告人の立場を悪くしているようにも見えていました。

 遺体を電車に轢過させていなければ,他の同種事件と比較して死刑が求刑されることもなかったと思いますが,弁護士が死刑の方向へと引っ張っているように見える,まるで死神に取り憑かれたような事件で,テレビでも法廷に弁護士の姿がなく,幽霊法廷のように見えたのも印象的でした。

 昨年の9月24日に控訴審の初公判が,今年2021年の3月1日に第2回公判があったとありますが,想像が及んでいなかったところ控訴審は当然ながら東京高裁でした。マスコミの報道も新潟市まで行く必要がなくやりやすくなったはずですが,報道をみていなかったのも気になる現象です。

 たまに見かけることはあって全体の1,2割という気もするところですが,この新潟小2女児殺害事件のWikipediaでは,被疑者・被告人の名前が仮名でKとされています。長い間続いた大きな報道でしたが,被疑者・被告人の名前が思い出せません。少し変わった読み方をする名前だったかもしれません。

\begin{itemize}
\tightlist
\item
  新潟小2女子殺人 小林遼被告のスマホに残るおぞましい検索ワード
  \textbar{} FRIDAYデジタル \url{https://t.co/mHPAwODMY8} 
  『なぎさのふれあい広場』での首絞め行為も「黙らせるため気絶させようと首を絞めた」だけであると主張した。それに関連し「わいせつ行為をしたいと思っていたが、(その時点ではまだ)行為
\end{itemize}

 2019年11月9日の記事ですが,「それに関連し「わいせつ行為をしたいと思っていたが、(その時点ではまだ)行為には及んでおらず、それゆえ殺害の動機はない」と生前のわいせつ行為も否定していた。」とあります。弁護士の入れ知恵だとすれば,これもすごいことでSFの世界のようです。

\begin{itemize}
\tightlist
\item
  なぎさふれあいセンター(ゆうやけこばり) - Google マップ
  \url{https://t.co/8qjiED0Ra5} 
\end{itemize}

 海に沈む夕日の写真がありますが,輪島市からみる夕日と方向が似ています。左斜め方向ですが,同じ奥能登でも宇出津や小木は,海から朝日がのぼり,山に夕陽が沈んでいきます。稀にですが,遠くの山に朝日がのぼるのを見ることもあり,新潟県の方向だと思っています。

 住んでいる家の場所や働いている仕事の場所,特に漁師であれば,毎日のように海を見ているかと思いますが,新潟方面の山がすっきり見えて,朝日がのぼるのは数年に一度見るような景色でした。

 写真をAmazonnPhotoで調べると,昨年2021年11月22日の写真でした。早朝に小木港東一文字堤防に行き,南西の強風のため午前中に引きあげ,イカの駅つくモールに立ち寄り,島根県浜田港のアジの缶詰を買って帰ってきた日のことでした。検索審査会のポスターの写真も夕方にありました。

 時刻は14時04分です。鍋にキャベツとウィンナーを入れ昼食を作っているのですが,検察審査会がらみの個人的な関心として大きなニュースが舞い込んできました。最初はTwitterのトレンドでしたが,探して見つけた京都新聞の記事の内容を見るまで検察審査会とはわかりませんでした。

\begin{itemize}
\tightlist
\item
  〈〈〈 2021/05/07 14:07:14 Linux Emacs: 〈〈〈
\end{itemize}

\hypertarget{ux79c1ux305fux3061ux306fux30d9ux30c6ux30e9ux30f3ux306eux7537ux6027ux5f01ux8b77ux58ebux3042ux308bux3044ux306fux88c1ux5224ux5b98ux305fux3061ux306bux304aux4e16ux8a71ux306bux306aux308bux3068ux540cux6642ux306bux6027ux7684ux306bux643eux53d6ux3055ux308cux308bux3068ux3044ux3046ux4f50ux85e4ux502bux5b50ux5f01ux8b77ux58ebux306eux30c4ux30a4ux30fcux30c8ux3068ux4e09ux5b85ux4fcaux4e00ux90ceux88c1ux5224ux95771}{%
\paragraph{「私たちはベテランの男性弁護士(あるいは裁判官)たちにお世話になると同時に性的に搾取される」という佐藤倫子弁護士のツイートと,三宅俊一郎裁判長(1)}\label{ux79c1ux305fux3061ux306fux30d9ux30c6ux30e9ux30f3ux306eux7537ux6027ux5f01ux8b77ux58ebux3042ux308bux3044ux306fux88c1ux5224ux5b98ux305fux3061ux306bux304aux4e16ux8a71ux306bux306aux308bux3068ux540cux6642ux306bux6027ux7684ux306bux643eux53d6ux3055ux308cux308bux3068ux3044ux3046ux4f50ux85e4ux502bux5b50ux5f01ux8b77ux58ebux306eux30c4ux30a4ux30fcux30c8ux3068ux4e09ux5b85ux4fcaux4e00ux90ceux88c1ux5224ux95771}}

\begin{itemize}
\tightlist
\item
  〉〉〉 Linux Emacs: 2021/05/10 11:04:17 〉〉〉
\end{itemize}

:CATEGORIES: @kanazawabengosi \#金沢弁護士会 @JFBAsns
日本弁護士連合会(日弁連) \#法務省 @MOJ\_HOUMU \#三宅俊一郎裁判長

 さきほど炊飯のスイッチを入れたところで,朝食も食べていないのですが,昨夜は自分で作ったカレーを食べすぎて苦しい思いをしていましいた。珍しく朝に2合を炊いて,昼にも食べたのですが,夕食の残っていた量が多く,食べ過ぎで苦しくなりました。

 千葉の弁護士の強制性交等致傷事件について,坂口靖弁護士のタイムラインとともに取り上げる予定だったのですが,食べ過ぎで気持ち悪く集中できなかったこともあり,坂口靖弁護士のYouTube動画からだったと思いますが,久しぶりに海釣り関連のYouTube動画を視聴していました。

 坂口靖弁護士のタイムラインには,,知らなかったニュースも多く,取り上げたい事柄が多かったのですが,

 時刻は5月11日04時35分になります。4時頃に目が覚めました。寝た時間は0時15分は過ぎていました。2021-05-10
11:10のツイートの直後に電話があり,そのあと出掛けていました。

 坂口靖弁護士のタイムラインのことを書き始めたところで,書き初めのまま中断にしていたことは先程まで気がついておらず,これまで余りやっていないことかと思います。坂口靖弁護士のタイムラインについても取り上げておきたかったのですが,そこに佐藤倫子弁護士のツイートが出ました。

 家に戻ったのは18時に近い時間で,そのあと銭湯に行きAコープ能都店で買い物をして戻ったのですが,その後の夜も記録しておきたい発見というか動きがありました。とりわけ大きいと感じたのがジャーナリストの江川紹子氏のツイートですが,反応はまだ少ししか確認していません。

 次が佐藤倫子弁護士のツイートを目にしてから先程までのスクリーンショットの記録になります。数は多くないと思います。

〉〉〉 kk\_hironoのリツイート 〉〉〉

\begin{itemize}
\tightlist
\item
  RT
  kk\_hirono(刑事告発・非常上告_金沢地方検察庁御中)|s\_hirono(非常上告-最高検察庁御中\_ツイッター)
  日時:2021-05-11 04:50/2021/05/11 04:47 URL:
  \url{https://twitter.com/kk\_hirono/status/1391843077774446592} 
  \url{https://twitter.com/s\_hirono/status/1391842339212070913} 
  \textgreater{}
  2021-05-11-041258\_小倉秀夫@chosakukenho·3時間敗訴した弁護士が地団駄踏んでいる。.jpg
  \url{https://t.co/tKnGfoANw2} 
\end{itemize}

〉〉〉 kk\_hironoのリツイート 〉〉〉

\begin{itemize}
\tightlist
\item
  RT
  kk\_hirono(刑事告発・非常上告_金沢地方検察庁御中)|s\_hirono(非常上告-最高検察庁御中\_ツイッター)
  日時:2021-05-11 04:50/2021/05/11 04:47 URL:
  \url{https://twitter.com/kk\_hirono/status/1391843104613834752} 
  \url{https://twitter.com/s\_hirono/status/1391842266491146240} 
  \textgreater{}
  2021-05-11-040959\_モトケン@motoken\_tw#石川優実さんへの誹謗中傷をやめろ 逆効果にしかならないタグだな。誹謗中傷と批判の区別がつかない人にとっては特.jpg
  \url{https://t.co/Bfkp8XLLBX} 
\end{itemize}

〉〉〉 kk\_hironoのリツイート 〉〉〉

\begin{itemize}
\tightlist
\item
  RT
  kk\_hirono(刑事告発・非常上告_金沢地方検察庁御中)|s\_hirono(非常上告-最高検察庁御中\_ツイッター)
  日時:2021-05-11 04:50/2021/05/11 04:46 URL:
  \url{https://twitter.com/kk\_hirono/status/1391843135987142658} 
  \url{https://twitter.com/s\_hirono/status/1391842193460002816} 
  \textgreater{}
  2021-05-10-204940\_ystk@lawkus·8時間池江璃花子選手への五輪出場辞退要請は誰が行っているのか(鳥海不二夫)https://news。yahoo。co.jpg
  \url{https://t.co/eRuRtHmZ8U} 
\end{itemize}

〉〉〉 kk\_hironoのリツイート 〉〉〉

\begin{itemize}
\tightlist
\item
  RT
  kk\_hirono(刑事告発・非常上告_金沢地方検察庁御中)|s\_hirono(非常上告-最高検察庁御中\_ツイッター)
  日時:2021-05-11 04:50/2021/05/11 04:46 URL:
  \url{https://twitter.com/kk\_hirono/status/1391843172615999490} 
  \url{https://twitter.com/s\_hirono/status/1391842120672022530} 
  \textgreater{}
  2021-05-10-201401\_新説である。やはり、裁判所には「敗訴した弁護士の霊」が多いのだろうか。これは有力な手がかりになりそうだ。われわれは、さらなる手がかりを求めて.jpg
  \url{https://t.co/Fmp64QfhBa} 
\end{itemize}

〉〉〉 kk\_hironoのリツイート 〉〉〉

\begin{itemize}
\tightlist
\item
  RT
  kk\_hirono(刑事告発・非常上告_金沢地方検察庁御中)|s\_hirono(非常上告-最高検察庁御中\_ツイッター)
  日時:2021-05-11 04:50/2021/05/11 04:46 URL:
  \url{https://twitter.com/kk\_hirono/status/1391843195043028994} 
  \url{https://twitter.com/s\_hirono/status/1391842048211181568} 
  \textgreater{}
  2021-05-10-200627\_五輪中止ネット署名、2日で「20万筆」突破・・・森友・黒川に次ぐ「歴代3位」に - 弁護士ドットコム.jpg
  \url{https://t.co/880kJazpt1} 
\end{itemize}

〉〉〉 kk\_hironoのリツイート 〉〉〉

\begin{itemize}
\tightlist
\item
  RT
  kk\_hirono(刑事告発・非常上告_金沢地方検察庁御中)|s\_hirono(非常上告-最高検察庁御中\_ツイッター)
  日時:2021-05-11 04:51/2021/05/11 04:46 URL:
  \url{https://twitter.com/kk\_hirono/status/1391843230841405440} 
  \url{https://twitter.com/s\_hirono/status/1391841975402319873} 
  \textgreater{}
  2021-05-10-195949\_150を超える法廷を持つ。1日の訪問者数が多いときで1万人を超える日本最大級の裁判所であり、「裁判沙汰」になるほどこじれた人々の怨念が吹き溜.jpg
  \url{https://t.co/LqITgh0eQJ} 
\end{itemize}

〉〉〉 kk\_hironoのリツイート 〉〉〉

\begin{itemize}
\tightlist
\item
  RT
  kk\_hirono(刑事告発・非常上告_金沢地方検察庁御中)|s\_hirono(非常上告-最高検察庁御中\_ツイッター)
  日時:2021-05-11 04:51/2021/05/11 04:45 URL:
  \url{https://twitter.com/kk\_hirono/status/1391843253620666371} 
  \url{https://twitter.com/s\_hirono/status/1391841902882791426} 
  \textgreater{}
  2021-05-10-195728\_弁護士落合洋司感染拡大を招く東京(頭狂)オリンピック中止!·3時間寃罪で死んだ人が怨霊になってるんでは。(笑)→弁護士ドットコム: 東京.jpg
  \url{https://t.co/wnKOcOOSK2} 
\end{itemize}

〉〉〉 kk\_hironoのリツイート 〉〉〉

\begin{itemize}
\tightlist
\item
  RT
  kk\_hirono(刑事告発・非常上告_金沢地方検察庁御中)|s\_hirono(非常上告-最高検察庁御中\_ツイッター)
  日時:2021-05-11 04:51/2021/05/11 04:45 URL:
  \url{https://twitter.com/kk\_hirono/status/1391843275531636739} 
  \url{https://twitter.com/s\_hirono/status/1391841830325477376} 
  \textgreater{}
  2021-05-10-195710\_弁護士落合洋司感染拡大を招く東京(頭狂)オリンピック中止!·3時間寃罪で死んだ人が怨霊になってるんでは。(笑)→弁護士ドットコム: 東京.jpg
  \url{https://t.co/pSEj6EO4DJ} 
\end{itemize}

〉〉〉 kk\_hironoのリツイート 〉〉〉

\begin{itemize}
\tightlist
\item
  RT
  kk\_hirono(刑事告発・非常上告_金沢地方検察庁御中)|s\_hirono(非常上告-最高検察庁御中\_ツイッター)
  日時:2021-05-11 04:51/2021/05/11 04:45 URL:
  \url{https://twitter.com/kk\_hirono/status/1391843305734823936} 
  \url{https://twitter.com/s\_hirono/status/1391841757814419456} 
  \textgreater{}
  2021-05-10-195658\_弁護士落合洋司感染拡大を招く東京(頭狂)オリンピック中止!·3時間寃罪で死んだ人が怨霊になってるんでは。(笑)→弁護士ドットコム: 東京.jpg
  \url{https://t.co/CZRaRxB87J} 
\end{itemize}

〉〉〉 kk\_hironoのリツイート 〉〉〉

\begin{itemize}
\tightlist
\item
  RT
  kk\_hirono(刑事告発・非常上告_金沢地方検察庁御中)|s\_hirono(非常上告-最高検察庁御中\_ツイッター)
  日時:2021-05-11 04:51/2021/05/10 19:36 URL:
  \url{https://twitter.com/kk\_hirono/status/1391843344486060032} 
  \url{https://twitter.com/s\_hirono/status/1391703738545491969} 
  \textgreater{}
  2021-05-10-193255\_Shoko Egawa@amneris84むしろ、未来永劫日本は五輪に名乗りを挙げてはいけない、ろくなことにはならない、と子々孫々に伝えてい.jpg
  \url{https://t.co/TfjSjaOTuH} 
\end{itemize}

〉〉〉 kk\_hironoのリツイート 〉〉〉

\begin{itemize}
\tightlist
\item
  RT
  kk\_hirono(刑事告発・非常上告_金沢地方検察庁御中)|s\_hirono(非常上告-最高検察庁御中\_ツイッター)
  日時:2021-05-11 04:51/2021/05/10 19:36 URL:
  \url{https://twitter.com/kk\_hirono/status/1391843388471726083} 
  \url{https://twitter.com/s\_hirono/status/1391703665304604674} 
  \textgreater{}
  2021-05-10-192835\_モトケン@motoken\_tw·58分世の中には(もちろんツイッターにも)、人の嫌がることをして喜ぶという下劣な人間がいるので、そんなことを.jpg
  \url{https://t.co/mURBJPMqhA} 
\end{itemize}

〉〉〉 kk\_hironoのリツイート 〉〉〉

\begin{itemize}
\tightlist
\item
  RT
  kk\_hirono(刑事告発・非常上告_金沢地方検察庁御中)|s\_hirono(非常上告-最高検察庁御中\_ツイッター)
  日時:2021-05-11 04:51/2021/05/10 10:47 URL:
  \url{https://twitter.com/kk\_hirono/status/1391843453718335489} 
  \url{https://twitter.com/s\_hirono/status/1391570459922104320} 
  \textgreater{}
  2021-05-10-094537\_もう20年近く前のことだけど、私たちはベテランの男性弁護士(あるいは裁判官)たちにお世話になると同時に性的に搾取されるというか軽んじられてき.jpg
  \url{https://t.co/PaSK5fs6aY} 
\end{itemize}

〉〉〉 kk\_hironoのリツイート 〉〉〉

\begin{itemize}
\tightlist
\item
  RT
  kk\_hirono(刑事告発・非常上告_金沢地方検察庁御中)|s\_hirono(非常上告-最高検察庁御中\_ツイッター)
  日時:2021-05-11 04:51/2021/05/10 10:46 URL:
  \url{https://twitter.com/kk\_hirono/status/1391843471732797445} 
  \url{https://twitter.com/s\_hirono/status/1391570387142537217} 
  \textgreater{}
  2021-05-10-093705\_佐藤倫子@sato\_\_michiko新人の頃、修習中からお世話になっていた先生と二人で食事にいったらペースに飲み込まれて想定していなかった不.jpg
  \url{https://t.co/snmo4xyJn8} 
\end{itemize}

〉〉〉 kk\_hironoのリツイート 〉〉〉

\begin{itemize}
\tightlist
\item
  RT
  kk\_hirono(刑事告発・非常上告_金沢地方検察庁御中)|s\_hirono(非常上告-最高検察庁御中\_ツイッター)
  日時:2021-05-11 04:52/2021/05/10 10:46 URL:
  \url{https://twitter.com/kk\_hirono/status/1391843553135927296} 
  \url{https://twitter.com/s\_hirono/status/1391570314421628928} 
  \textgreater{}
  2021-05-10-093553\_高橋雄一郎@kamatatylaw·17時間うげぇ,最悪の弁護士だ。俺が引き継いで受任しついでに懲戒請求したい。.jpg
  \url{https://t.co/LegLyiRCxW} 
\end{itemize}

〉〉〉 kk\_hironoのリツイート 〉〉〉

\begin{itemize}
\tightlist
\item
  RT
  kk\_hirono(刑事告発・非常上告_金沢地方検察庁御中)|s\_hirono(非常上告-最高検察庁御中\_ツイッター)
  日時:2021-05-11 04:52/2021/05/10 10:46 URL:
  \url{https://twitter.com/kk\_hirono/status/1391843569703391232} 
  \url{https://twitter.com/s\_hirono/status/1391570241285595136} 
  \textgreater{}
  2021-05-10-093506\_高橋雄一郎@kamatatylaw·1時間コロナ禍の下でオリンピックを強行して関係者の間でクラスターが発生して白人選手が死亡でもしたら完全に.jpg
  \url{https://t.co/Zb33YiMpGC} 
\end{itemize}

〉〉〉 kk\_hironoのリツイート 〉〉〉

\begin{itemize}
\tightlist
\item
  RT
  kk\_hirono(刑事告発・非常上告_金沢地方検察庁御中)|s\_hirono(非常上告-最高検察庁御中\_ツイッター)
  日時:2021-05-11 04:52/2021/05/10 10:45 URL:
  \url{https://twitter.com/kk\_hirono/status/1391843603585007620} 
  \url{https://twitter.com/s\_hirono/status/1391570168237617154} 
  \textgreater{}
  2021-05-10-093327\_高橋雄一郎@kamatatylaw·1時間自分に対するツイートをエゴサして,不快なものを発見したら「名誉感情侵害」とか「名誉権侵害」を振りか.jpg
  \url{https://t.co/Abay6GmJq2} 
\end{itemize}

〉〉〉 kk\_hironoのリツイート 〉〉〉

\begin{itemize}
\tightlist
\item
  RT
  kk\_hirono(刑事告発・非常上告_金沢地方検察庁御中)|s\_hirono(非常上告-最高検察庁御中\_ツイッター)
  日時:2021-05-11 04:52/2021/05/10 09:08 URL:
  \url{https://twitter.com/kk\_hirono/status/1391843661562847232} 
  \url{https://twitter.com/s\_hirono/status/1391545670381621248} 
  \textgreater{}
  2021-05-10-090541\_千葉県弁護士会|法律問題でお困りの際はご相談ください.jpg
  \url{https://t.co/VSzLA8SxD0} 
\end{itemize}

〉〉〉 kk\_hironoのリツイート 〉〉〉

\begin{itemize}
\tightlist
\item
  RT
  kk\_hirono(刑事告発・非常上告_金沢地方検察庁御中)|s\_hirono(非常上告-最高検察庁御中\_ツイッター)
  日時:2021-05-11 04:52/2021/05/10 09:08 URL:
  \url{https://twitter.com/kk\_hirono/status/1391843678184873987} 
  \url{https://twitter.com/s\_hirono/status/1391545597086158848} 
  \textgreater{}
  2021-05-10-085611\_つまらむ@km0bake·9時間一つ一つの攻撃は、名誉毀損までいかなくて違法といえないツイートでも、こころのコップにだんだん水はたまってい.jpg
  \url{https://t.co/XbAS9xHL7z} 
\end{itemize}

〉〉〉 kk\_hironoのリツイート 〉〉〉

\begin{itemize}
\tightlist
\item
  RT
  kk\_hirono(刑事告発・非常上告_金沢地方検察庁御中)|s\_hirono(非常上告-最高検察庁御中\_ツイッター)
  日時:2021-05-11 04:52/2021/05/10 08:42 URL:
  \url{https://twitter.com/kk\_hirono/status/1391843707863797760} 
  \url{https://twitter.com/s\_hirono/status/1391539001052467200} 
  \textgreater{}
  2021-05-10-083932\_うの字を名乗る物さんがリツイート榎本まみ@n\_mototkol·22時間モラの人ってなんですぐ直接会おうとか直接話しさせろって言うんです.jpg
  \url{https://t.co/PTOjy6grW8} 
\end{itemize}

〉〉〉 kk\_hironoのリツイート 〉〉〉

\begin{itemize}
\tightlist
\item
  RT
  kk\_hirono(刑事告発・非常上告_金沢地方検察庁御中)|s\_hirono(非常上告-最高検察庁御中\_ツイッター)
  日時:2021-05-11 04:52/2021/05/10 08:41 URL:
  \url{https://twitter.com/kk\_hirono/status/1391843725161091074} 
  \url{https://twitter.com/s\_hirono/status/1391538928281210885} 
  \textgreater{}
  2021-05-10-083705\_2021/04/28当会会員の起訴に関する会長談話.jpg
  \url{https://t.co/2AAxoKrUjP} 
\end{itemize}

〉〉〉 kk\_hironoのリツイート 〉〉〉

\begin{itemize}
\tightlist
\item
  RT
  kk\_hirono(刑事告発・非常上告_金沢地方検察庁御中)|s\_hirono(非常上告-最高検察庁御中\_ツイッター)
  日時:2021-05-11 04:53/2021/05/10 08:41 URL:
  \url{https://twitter.com/kk\_hirono/status/1391843746996576256} 
  \url{https://twitter.com/s\_hirono/status/1391538855673692163} 
  \textgreater{}
  2021-05-10-083646\_2021/05/07当会会員の再逮捕に関する会長談話.jpg
  \url{https://t.co/NOAZ4zNVKy} 
\end{itemize}

〉〉〉 kk\_hironoのリツイート 〉〉〉

\begin{itemize}
\tightlist
\item
  RT
  kk\_hirono(刑事告発・非常上告_金沢地方検察庁御中)|s\_hirono(非常上告-最高検察庁御中\_ツイッター)
  日時:2021-05-11 04:53/2021/05/10 08:41 URL:
  \url{https://twitter.com/kk\_hirono/status/1391843782870532098} 
  \url{https://twitter.com/s\_hirono/status/1391538783175122945} 
  \textgreater{}
  2021-05-10-083433\_まゆろん筋トレの効果を実感中@mayukotaniguchi弁護士会に電話して、市民のための苦情窓口に繋いでもらって、苦情を言って、本人.jpg
  \url{https://t.co/kwl40zqM75} 
\end{itemize}

〉〉〉 kk\_hironoのリツイート 〉〉〉

\begin{itemize}
\tightlist
\item
  RT
  kk\_hirono(刑事告発・非常上告_金沢地方検察庁御中)|s\_hirono(非常上告-最高検察庁御中\_ツイッター)
  日時:2021-05-11 04:53/2021/05/10 08:40 URL:
  \url{https://twitter.com/kk\_hirono/status/1391843812343906305} 
  \url{https://twitter.com/s\_hirono/status/1391538710634586112} 
  \textgreater{}
  2021-05-10-083216\_固定されたツイート深澤諭史@fukazawas·2020年5月7日まんが 弁護士が教えるウソを見抜く方法 https://amazon。c.jpg
  \url{https://t.co/njcs3Yxzom} 
\end{itemize}

〉〉〉 kk\_hironoのリツイート 〉〉〉

\begin{itemize}
\tightlist
\item
  RT
  kk\_hirono(刑事告発・非常上告_金沢地方検察庁御中)|s\_hirono(非常上告-最高検察庁御中\_ツイッター)
  日時:2021-05-11 04:54/2021/05/10 08:39 URL:
  \url{https://twitter.com/kk\_hirono/status/1391843987065954306} 
  \url{https://twitter.com/s\_hirono/status/1391538418287476736} 
  \textgreater{}
  2021-05-09-201203\_弁護士坂口靖のちゃんねるチャンネル登録刑事事件の受任経験が異常に多い!?弁護士坂口靖が話題のニュースを徹底解説します!人気女子プロレスラーの.jpg
  \url{https://t.co/0dIwA9hwIh} 
\end{itemize}

 24件のリツイートになりましたが,2021-05-09-201203の記録として坂口靖弁護士のYouTube動画を見た辺りまでさかのぼりました。坂口靖弁護士の顔写真は見ていましたが,動画ではかなり違った印象を受けました。

 ブログにおける埋め込みツイートの数が少なくとも24件以上となってしまったので,佐藤倫子弁護士のツイートは別のエントリーとして取り上げたいと思います。レベル4の小見出しがブログへの投稿として対応し,それをエントリーと呼んでいます。

 ずいぶん思い切った告発と思ったのが佐藤倫子弁護士のツイートで,次のステージに移行する端境期のようなものを感じました。それとともに,佐藤倫子弁護士の述懐が自分の経験にも当てはまるように感じ,まず頭に浮かんだのが三宅俊一郎裁判長です。

\begin{itemize}
\tightlist
\item
  〈〈〈 2021/05/11 05:02:19 Linux Emacs: 〈〈〈
\end{itemize}

\hypertarget{ux79c1ux305fux3061ux306fux30d9ux30c6ux30e9ux30f3ux306eux7537ux6027ux5f01ux8b77ux58ebux3042ux308bux3044ux306fux88c1ux5224ux5b98ux305fux3061ux306bux304aux4e16ux8a71ux306bux306aux308bux3068ux540cux6642ux306bux6027ux7684ux306bux643eux53d6ux3055ux308cux308bux3068ux3044ux3046ux4f50ux85e4ux502bux5b50ux5f01ux8b77ux58ebux306eux30c4ux30a4ux30fcux30c8ux3068ux4e09ux5b85ux4fcaux4e00ux90ceux88c1ux5224ux95772}{%
\paragraph{「私たちはベテランの男性弁護士(あるいは裁判官)たちにお世話になると同時に性的に搾取される」という佐藤倫子弁護士のツイートと,三宅俊一郎裁判長(2)}\label{ux79c1ux305fux3061ux306fux30d9ux30c6ux30e9ux30f3ux306eux7537ux6027ux5f01ux8b77ux58ebux3042ux308bux3044ux306fux88c1ux5224ux5b98ux305fux3061ux306bux304aux4e16ux8a71ux306bux306aux308bux3068ux540cux6642ux306bux6027ux7684ux306bux643eux53d6ux3055ux308cux308bux3068ux3044ux3046ux4f50ux85e4ux502bux5b50ux5f01ux8b77ux58ebux306eux30c4ux30a4ux30fcux30c8ux3068ux4e09ux5b85ux4fcaux4e00ux90ceux88c1ux5224ux95772}}

\begin{itemize}
\tightlist
\item
  〉〉〉 Linux Emacs: 2021/05/11 05:04:44 〉〉〉
\end{itemize}

:CATEGORIES: @kanazawabengosi \#金沢弁護士会 @JFBAsns
日本弁護士連合会(日弁連) \#法務省 @MOJ\_HOUMU \#三宅俊一郎裁判長
\#川口泰司裁判官 \#山田徹裁判官

 4時に起きた後,毎日記録しているモトケンこと矢部善朗弁護士(京都弁護士会)と小倉秀夫弁護士のツイートがデータベースへの登録に失敗していたことに気が付きました。5月8日あたりには発生していた様子です。カラムの文字数オーバーでした。300を600に変更しました。

 次が昨日5月10日から本日11日未明の記録の一覧となります。出掛けている時間は長かったのですが,特別な変化や動きを感じた一日だったので,ここで記録とともにご紹介をしておきたいと思います。偶然でありながら関連したような事柄もあります。

\begin{itemize}
\tightlist
\item
  2021年05月10日00時25分の登録:
  2021-05-09の投稿一覧\検察・石川県警察宛記録資料\奉納\危険生物・弁護士脳汚染除去装置\金沢地方検察庁御中:66件
  \url{https://kk2020-09.blogspot.com/2021/05/2021-05-0966.html} 
\item
  2021年05月10日00時25分の登録:
  ツイートの記録資料:\法務検察・石川県警察宛\/小倉秀夫(@chosakukenho)/''2021年05月09日'':2件
  \url{https://kk2020-09.blogspot.com/2021/05/chosakukenho202105092.html} 
\item
  2021年05月10日00時25分の登録:
  ツイートの記録資料:\法務検察・石川県警察宛\/モトケン(@motoken\_tw)/''2021年05月09日'':4件
  \url{https://kk2020-09.blogspot.com/2021/05/motokentw202105094.html} 
\item
  2021年05月10日08時35分の登録:
  \坂本正幸 @sakamotomasayuk\こんなのに弁護士会ってありえないと思うが
  \url{https://kk2020-09.blogspot.com/2021/05/sakamotomasayuk\_10.html} 
\item
  2021年05月10日08時35分の登録:
  \うの字を名乗る?物 @un\_co\_the2nd\「店に伝えて出禁にする」一択弁護士会に私生活まで口出させるのおかしい
  \url{https://kk2020-09.blogspot.com/2021/05/uncothe2nd\_10.html} 
\item
  2021年05月10日08時35分の登録:
  %@mayukotaniguchi まゆろん?️♂️?筋トレの効果を実感中%弁護士会に電話して、市民のための苦情窓口に繋いでもらって、苦情を言って、本人に伝えてくれと言えば、その弁護士に伝わるはずです。
  \url{https://kk2020-09.blogspot.com/2021/05/mayukotaniguchi.html} 
\item
  2021年05月10日08時44分の登録:
  @un\_co\_the2nd(うの字を名乗る?物)のツイート ''.*'' 3239/3239:2021-03-29\_1143〜2021-05-10\_0745 2021年05月10日08時43分の記録
  \url{https://kk2020-09.blogspot.com/2021/05/uncothe2nd323932392021-03-2911432021-05.html} 
\item
  2021年05月10日08時45分の登録:
  @sakamotomasayuk(坂本正幸)のツイート ''.*'' 3250/3250:2021-03-19\_1704〜2021-05-10\_0824 2021年05月10日08時45分の記録
  \url{https://kk2020-09.blogspot.com/2021/05/sakamotomasayuk325032502021-03.html} 
\item
  2021年05月10日08時54分の登録:
  @km0bake(?️つまらむ?️)のツイート ''.*'' 3234/3234:2020-12-16\_2239〜2021-05-09\_2301 2021年05月10日08時54分の記録
  \url{https://kk2020-09.blogspot.com/2021/05/km0bake323432342020-12-1622392021-05.html} 
\item
  2021年05月10日08時55分の登録:
  \?️つまらむ?️ @km0bake\一つ一つの攻撃は、名誉毀損までいかなくて違法といえないツイートでも、こころのコップにだんだん水はたまっていっちゃうんだよ。
  \url{https://kk2020-09.blogspot.com/2021/05/km0bake\_10.html} 
\item
  2021年05月10日08時57分の登録:
  \?️つまらむ?️ @km0bake\池江璃花子氏のツイートは電通の工作では?という疑惑を検証する
  - 社会の独房から
  \url{https://kk2020-09.blogspot.com/2021/05/km0bake\_7.html} 
\item
  2021年05月10日09時07分の登録:
  @lawkus(ystk)のツイート ''.*'' 3240/3240:2021-02-09\_0544〜2021-05-10\_0830 2021年05月10日09時07分の記録
  \url{https://kk2020-09.blogspot.com/2021/05/lawkusystk324032402021-02-0905442021-05.html} 
\item
  2021年05月10日09時07分の登録:
  @uwaaaa(サイ太)のツイート ''.*'' 3238/3238:2020-10-05\_1221〜2021-05-09\_2220 2021年05月10日09時06分の記録
  \url{https://kk2020-09.blogspot.com/2021/05/uwaaaa323832382020-10-0512212021-05.html} 
\item
  2021年05月10日09時16分の登録:
  @imarockcaster42(弁護士 市川 寛)のツイート ''.*'' 1585/1585:2012-10-07\_1821〜2021-05-10\_0014 2021年05月10日09時16分の記録
  \url{https://kk2020-09.blogspot.com/2021/05/imarockcaster42158515852012-10.html} 
\item
  2021年05月10日09時30分の登録:
  @ekinan\_lawyer(えきなんローヤー?)のツイート ''.*'' 3215/3215:2021-04-15\_2057〜2021-05-10\_0924 2021年05月10日09時29分の記録
  \url{https://kk2020-09.blogspot.com/2021/05/ekinanlawyer321532152021-04-1520572021.html} 
\item
  2021年05月10日09時31分の登録:
  @harrier0516osk(向原総合法律事務所 弁護士向原)のツイート ''.*'' 3249/3249:2020-10-06\_1359〜2021-05-10\_0859 2021年05月10日09時30分の記録
  \url{https://kk2020-09.blogspot.com/2021/05/harrier0516osk324932492020-10.html} 
\item
  2021年05月10日09時33分の登録:
  @keita\_adachi(豪弁 足立敬太 @ザンギの極み)のツイート ''.*'' 3234/3234:2021-04-09\_1944〜2021-05-10\_0924 2021年05月10日09時32分の記録
  \url{https://kk2020-09.blogspot.com/2021/05/keitaadachi323432342021-04-0919442021.html} 
\item
  2021年05月10日09時34分の登録:
  @kamatatylaw(高橋雄一郎)のツイート ''.*'' 3229/3229:2020-02-10\_1254〜2021-05-10\_0819 2021年05月10日09時34分の記録
  \url{https://kk2020-09.blogspot.com/2021/05/kamatatylaw322932292020-02-1012542021.html} 
\item
  2021年05月10日09時38分の登録:
  @sato\_\_michiko(§ 佐藤倫子)のツイート ''.*'' 3239/3239:2021-02-05\_1515〜2021-05-10\_0716 2021年05月10日09時38分の記録
  \url{https://kk2020-09.blogspot.com/2021/05/satomichiko323932392021-02-0515152021.html} 
\item
  2021年05月10日09時44分の登録:
  REGEXP:''不本意な関係に持ち込まれたこと''/データベース登録済みツイートの検索:2021-05-08〜2021-05-10/2021年05月10日09時43分の記録:ユーザ・投稿:25/28件
  \url{https://kk2020-09.blogspot.com/2021/05/regexp2021-05-082021-05\_10.html} 
\item
  2021年05月10日09時45分の登録: \§
  佐藤倫子 @sato\_\_michiko\私たちはベテランの男性弁護士(あるいは裁判官)たちにお世話になると同時に性的に搾取されるというか軽んじられてきたと今となって
  \url{https://kk2020-09.blogspot.com/2021/05/satomichiko.html} 
\item
  2021年05月10日09時54分の登録:
  @o2441(スラ弁(弁護士大西洋一))のツイート ''.*'' 3249/3249:2021-02-20\_1859〜2021-05-10\_0908 2021年05月10日09時54分の記録
  \url{https://kk2020-09.blogspot.com/2021/05/o2441324932492021-02-2018592021-05.html} 
\item
  2021年05月10日10時45分の登録:
  REGEXP:''弁護士.*逮捕''/データベース登録済みツイートの検索:2021-05-02〜2021-05-09/2021年05月10日10時44分の記録:ユーザ・投稿:25/48件
  \url{https://kk2020-09.blogspot.com/2021/05/regexp2021-05-022021-05\_10.html} 
\item
  2021年05月10日11時01分の登録:
  @hirono\_hideki(奉納\さらば弁護士鉄道・泥棒神社の物語)のツイート ''.*'' 3241/3241:2021-04-23\_1227〜2021-05-10\_1048 2021年05月10日11時00分の記録
  \url{https://kk2020-09.blogspot.com/2021/05/hironohideki324132412021-04-2312272021.html} 
\item
  2021年05月10日12時41分の登録:
  \浜木綿弁右衛門 @leplusallez\セクハラ現実にある問題だとは(当職は見聞きもしたことないので意外だったけど、この感覚もだめなのだろう。)思うけど、何故「法曹」
  \url{https://kk2020-09.blogspot.com/2021/05/leplusallez\_10.html} 
\item
  2021年05月10日19時28分の登録:
  \モトケン @motoken\_tw\世の中には(もちろんツイッターにも)、人の嫌がることをして喜ぶという下劣な人間がいるので、そんなことをするのはやめて、と言うと、火に油
  \url{https://kk2020-09.blogspot.com/2021/05/motokentw\_10.html} 
\item
  2021年05月10日19時40分の登録:
  REGEXP:''伊藤詩織''/データベース登録済みツイートの検索:2021-04-26〜2021-05-10/2021年05月10日19時40分の記録:ユーザ・投稿:6/16件
  \url{https://kk2020-09.blogspot.com/2021/05/regexp2021-04-262021-05.html} 
\item
  2021年05月10日19時53分の登録:
  REGEXP:''伊藤詩織''/データベース登録済みツイートの検索:2017-10-13〜2021-05-08/2021年05月10日19時41分の記録:ユーザ・投稿:321/2106件
  \url{https://kk2020-09.blogspot.com/2021/05/regexp2017-10-132021-05.html} 
\item
  2021年05月10日19時56分の登録:
  \弁護士落合洋司?感染拡大を招く東京(頭狂)オリンピック中止!? @yjochi\寃罪で死んだ人が怨霊になってるんでは。(笑)→弁護士ドットコム:
  東京地裁の怪異「天井か
  \url{https://kk2020-09.blogspot.com/2021/05/yjochi\_10.html} 
\item
  2021年05月10日20時12分の登録:
  %@stdaux スドー?%理由を聞かれたときには「敗訴した弁護士の霊がラップ音を立てている」とか適当に説明してるんだけど、本当は何なんだろう
  \url{https://kk2020-09.blogspot.com/2021/05/stdaux\_10.html} 
\item
  2021年05月10日20時15分の登録:
  REGEXP:''@fukazawas''/データベース登録済みツイートの検索:2021-05-08〜2021-05-10/2021年05月10日20時15分の記録:ユーザ・投稿:2/4件
  \url{https://kk2020-09.blogspot.com/2021/05/regexpfukazawas2021-05-082021-05.html} 
\item
  2021年05月10日20時18分の登録:
  REGEXP:''三宅雪子''/データベース登録済みツイートの検索:2012-11-12〜2021-02-20/2021年05月10日20時17分の記録:ユーザ・投稿:31/87件
  \url{https://kk2020-09.blogspot.com/2021/05/regexp2012-11-122021-02.html} 
\item
  2021年05月10日20時49分の登録:
  \ystk @lawkus\池江璃花子選手への五輪出場辞退要請は誰が行っているのか(鳥海不二夫)https://
  \url{https://kk2020-09.blogspot.com/2021/05/ystklawkushttps.html} 
\item
  2021年05月10日20時54分の登録:
  \うの字を名乗る?物 @un\_co\_the2nd\毎年繰り返されてきた、子供が学校でもらってきたインフルエンザが回ってきて高齢者が死ぬ、というサイクルは抑えられてたんだよ
  \url{https://kk2020-09.blogspot.com/2021/05/uncothe2nd\_62.html} 
\item
  2021年05月11日00時05分の登録:
  ツイートの記録資料:\法務検察・石川県警察宛\/モトケン(@motoken\_tw)/''2021年05月10日'':1件
  \url{https://kk2020-09.blogspot.com/2021/05/motokentw202105101.html} 
\item
  2021年05月11日00時06分の登録:
  2021-05-10の投稿一覧\検察・石川県警察宛記録資料\奉納\危険生物・弁護士脳汚染除去装置\金沢地方検察庁御中:26件
  \url{https://kk2020-09.blogspot.com/2021/05/2021-05-1026.html} 
\item
  2021年05月11日04時10分の登録:
  \モトケン @motoken\_tw\\#石川優実さんへの誹謗中傷をやめろ
  逆効果にしかならないタグだな。誹謗中傷と批判の区別がつかない人にとっては特に。
  \url{https://kk2020-09.blogspot.com/2021/05/motokentw\_11.html} 
\item
  2021年05月11日04時12分の登録:
  \小倉秀夫 @chosakukenho\敗訴した弁護士が地団駄踏んでいる。
  \url{https://kk2020-09.blogspot.com/2021/05/chosakukenho\_11.html} 
\item
  2021年05月11日04時33分の登録: TWEET:''2021-05-07 03:48〜2021-05-11
  03:50''/モトケン(@motoken\_tw)の検索(2021年05月11日04時33分の記録57件)
  \url{https://kk2020-09.blogspot.com/2021/05/tweet2021-05-07-03482021-05-11.html} 
\item
  2021年05月11日04時33分の登録: TWEET:''2021-05-07 18:16〜2021-05-11
  00:57''/小倉秀夫(@chosakukenho)の検索(2021年05月11日04時32分の記録201件)
  \url{https://kk2020-09.blogspot.com/2021/05/tweet2021-05-07-18162021-05-11.html} 
\end{itemize}

 ku3コマンドでの記録が多くなっていますが,これは指定したアカウントのツイートを13オフセット取得し最大で最新の3200件前後のツイートを取得する自作コマンドです。埋め込みツイートは使っておらず,その分,ツイートの数は多い割にページの読み込みは早いと思います。

 3200件前後という射程になりますが,この射程に入ることを期待したのが千葉の弁護士の強制性交等致傷での逮捕,起訴,再逮捕に関連したツイートです。最初の逮捕のニュースが4月の10日すぎではなかったかと思います。はっきり思い出せないので,調べて確認をしておきます。

 Googleで検索してもほとんどが再逮捕のニュース関連になっており,4月の逮捕のニュースがなかなか見つからない状況となっています。自分のまとめ記事から探したほうが早そうです。

\begin{itemize}
\item
  2021年05月07日04時07分の登録:
  REGEXP:''武田祐介''/データベース登録済みツイートの検索:2021-04-10〜2021-05-07/2021年05月07日04時07分の記録:ユーザ・投稿:5/28件
  \url{https://kk2020-09.blogspot.com/2021/05/regexp2021-04-102021-05.html} 
\item
  (01/28) TW news\_type\_c(NEWS JAPAN R) 日時: 2021-04-10 11:38:36
  +0900 URL:
  \url{https://twitter.com/news\_type\_c/status/1380711783795621891\textgreater} {}
  千葉県警千葉中央署は10日、千葉市中央区中央、弁護士武田祐介容疑者(36)を傷害容疑で緊急逮捕し、強制性交致傷容疑で千葉地検に送検したと発表した。
  発表によると、武田容疑者は8日午後9時30分頃~同45分頃、自宅に知人女性(24)・・・
  \url{https://t.co/RdiCnIruAk} 
\end{itemize}

 4月でも12日から14日あたりを想定していたのですが,4月10日のニュースで逮捕も同じ4月10日だったようです。逮捕された千葉の弁護士の実名になりますが,実名のツイートで私と報道関係以外に記録されているのは,次の1件のリツイートだけです。

\begin{itemize}
\tightlist
\item
  (04/28) RT
  i9wpevKQdGktbmY(シン・へぼろーやー?残高22円?)|JIJOsBizAdv(事情通)
  日時:2021-04-10 16:24:53 +0900/2021-04-10 13:51:00 +0900 URL:
  \url{https://twitter.com/i9wpevKQdGktbmY/status/1380783831645483009} 
  \url{https://twitter.com/JIJOsBizAdv/status/1380745147600670721\textgreater} {}
  武田祐介弁護士(千葉県弁護士会所属。司法修習新63期)は、一橋大学・同法科大学院卒で、武田総合法律事務所代表。\textgreater\textgreater{}
  座右の銘は、「すべての人々に何もかもはできなくとも、誰かに何かはできる」\textgreater{}
  \url{https://t.co/moE0AdK6i2} 
\end{itemize}

 「シン・へぼろーやー?残高22円?」というプロフィールの名前に見覚えはないですが,頻繁にプロフィールの名前を変更するアカウントというのは法クラでは珍しくないと思います。元のツイートが事情通というアカウントです。

 しばらく見なかったアカウントですが,法曹関係者の可能性はあると思いながら特定できるツイートは確認していなかったように思います。もともとブックマークに入れていないのですが,タイムラインの方も長く開いていませんでした。

〉〉〉 kk\_hironoのリツイート 〉〉〉

\begin{itemize}
\tightlist
\item
  RT
  kk\_hirono(刑事告発・非常上告_金沢地方検察庁御中)|JIJOsBizAdv(事情通)
  日時:2021-05-11 05:37/2020/01/04 19:38 URL:
  \url{https://twitter.com/kk\_hirono/status/1391854920882556930} 
  \url{https://twitter.com/JIJOsBizAdv/status/1213409433931239426} 
  \textgreater{} 【謹告】
  当アカウントは、裁判所御用達のデマ野郎によって運営されています。
\end{itemize}

 ヘッダ画像も変わっていないと思いますが,上記のツイートが固定されたツイートとなっており,これは初めて見ました。2020年1月4日のツイートですが,いつから固定されたツイートになっていたのか不明で,少なくとも私は気が付かなかったと思います。

 リツイートですが,令和3年3月31日付告発状の締めくくりあたりで取り上げた海鮮丼の弁護士のツイートが出てきました。

〉〉〉 kk\_hironoのリツイート 〉〉〉

\begin{itemize}
\tightlist
\item
  RT
  kk\_hirono(刑事告発・非常上告_金沢地方検察庁御中)|K\_masafumi(川村真文(弁護士
  大阪)) 日時:2021-05-11 05:42/2021/05/05 10:44 URL:
  \url{https://twitter.com/kk\_hirono/status/1391856200636010497} 
  \url{https://twitter.com/K\_masafumi/status/1389757827279843329} 
  \textgreater{}
  最近、弁護士はクソって認知された感があるけど、そもそも事実を曲げて、法律を駆使して、相手を潰して利益を得ようとする職業なのだから、弁護士は人格者と思う方がおかしいのだよね。
  海外の映画やドラマでも、だいたい強欲の象徴として描かれているわけだし(笑)。
\end{itemize}

 弁護士が海外で強欲の象徴として描かれているとは聞いたことがなかったのですが,弁護士と強欲というのは妙にしっくりくる組み合わせです。

 引用ツイートとして見かけた心の貧困というアカウントのツイートですが,まとめ記事として記録しました。これまでにも時々見かけてきたアカウントですが,いかにも匿名ということもあり,重視はしなかったのですが,法曹関係者の内部告発に近いものがあるのかもしれません。

 意外に利益目的の実名弁護士アカウントのツイートより,このような事情を開かせない匿名アカウントの方が,事実や実態に近いものがあるのではと,さきほど事情通というアカウントの固定ツイートをみてつらつら考えていました。

\begin{itemize}
\tightlist
\item
  2021年05月11日05時45分の登録:
  \心の貧困 @mental\_poverty\イチケイのカラスが最も有害なのは裁判官が頑張れば真実が発見できるという誤りが一般により広がることだと思う。このせいで、呼ばれた
  \url{https://kk2020-09.blogspot.com/2021/05/mentalpoverty\_11.html} 
\end{itemize}

 そういえば,昨夜のイチケイのカラスの放送が録画できるように,ディスクの残量を確保するため視聴しない録画済みのニュースなどを削除しておくつもりだったのですが,昨日の午前中にテレビをつけて実行しようと思いながら忘れていました。失敗している可能性が高そうです。

〉〉〉 kk\_hironoのリツイート 〉〉〉

\begin{itemize}
\tightlist
\item
  RT
  kk\_hirono(刑事告発・非常上告_金沢地方検察庁御中)|silentsmoker99(肛 門 市 場)
  日時:2021-05-11 05:56/2021/05/02 17:26 URL:
  \url{https://twitter.com/kk\_hirono/status/1391859625276821506} 
  \url{https://twitter.com/silentsmoker99/status/1388771964827877376} 
  \textgreater{}
  なお、あの件に関しては無罪になった局長が悲劇のヒロインみたいな感じですが、訴追されてなかったらむしろ上司としての監督責任は免れず、事務次官まで上り詰めることはできなかったんだろうとも思う。
\end{itemize}

 上記のツイートも事情通というアカウントのタイムラインでリツイートとして見かけたものですが,こちらも久しぶりに見かけたアカウントでありながら,法曹関係者の可能性があると考えていました。他の法クラとのツイートのやりとりもごく自然なかたちで見かけたように思います。

〉〉〉 kk\_hironoのリツイート 〉〉〉

\begin{itemize}
\tightlist
\item
  RT
  kk\_hirono(刑事告発・非常上告_金沢地方検察庁御中)|silentsmoker99(肛 門 市 場)
  日時:2021-05-11 05:59/2021/05/11 01:39 URL:
  \url{https://twitter.com/kk\_hirono/status/1391860539626704896} 
  \url{https://twitter.com/silentsmoker99/status/1391794950564507651} 
  \textgreater{}
  完全に任意参加が保障されていればいいと思うよ。でも実際にはそれは無理なんだよな。断れないよ。
  無理矢理「自己紹介」をさせられ、ジジイババアの自慢話を聞かされ、人によっちゃセクハラされ。
  裁判所の飲み会だと不味いメシに不味い酒、その上で金まで徴収される。
  俺は飲み会不要論者。 \url{https://t.co/Fiy7Rxck0w} 
\end{itemize}

〉〉〉 kk\_hironoのリツイート 〉〉〉

\begin{itemize}
\tightlist
\item
  RT kk\_hirono(刑事告発・非常上告_金沢地方検察庁御中)|xmg\_on(It's
  law 1969) 日時:2021-05-11 05:59/2021/05/10 21:51 URL:
  \url{https://twitter.com/kk\_hirono/status/1391860564889018368} 
  \url{https://twitter.com/xmg\_on/status/1391737783841005570} 
  \textgreater{}
  いや、司法修習に、お酒はいらないだろう。宴会カルチャー、もうやめにしませんか?それがセクハラの温床だよ。
  \url{https://t.co/5MFxrrePZj} 
\end{itemize}

〉〉〉 kk\_hironoのリツイート 〉〉〉

\begin{itemize}
\tightlist
\item
  RT
  kk\_hirono(刑事告発・非常上告_金沢地方検察庁御中)|bengoshimentaru(家系弁護士)
  日時:2021-05-11 06:00/2021/05/10 21:47 URL:
  \url{https://twitter.com/kk\_hirono/status/1391860597398982657} 
  \url{https://twitter.com/bengoshimentaru/status/1391736761177415682} 
  \textgreater{} お酒飲めない修習なんて、修習じゃないよ
\end{itemize}

 タイムラインで一番上にあるツイートですが,固定されたツイートではないことを確認しています。司法修習の経験者のようなツイートで,それも他の似たようなアカウントとのやりとりのような流れとなっています。

〉〉〉 kk\_hironoのリツイート 〉〉〉

\begin{itemize}
\tightlist
\item
  RT
  kk\_hirono(刑事告発・非常上告_金沢地方検察庁御中)|silentsmoker99(肛 門 市 場)
  日時:2021-05-11 06:03/2021/05/09 22:30 URL:
  \url{https://twitter.com/kk\_hirono/status/1391861582154530819} 
  \url{https://twitter.com/silentsmoker99/status/1391385120506978308} 
  \textgreater{}
  前にも呟いたことあるけど、誘われてホテルでパンパンしたら後日「無理矢理連れ込まれた」と言われたことあるからね。
  身勝手なやつは男にも女にもいるというのが俺の認識。
\end{itemize}

〉〉〉 kk\_hironoのリツイート 〉〉〉

\begin{itemize}
\tightlist
\item
  RT
  kk\_hirono(刑事告発・非常上告_金沢地方検察庁御中)|silentsmoker99(肛 門 市 場)
  日時:2021-05-11 06:04/2021/05/09 22:24 URL:
  \url{https://twitter.com/kk\_hirono/status/1391861620175867905} 
  \url{https://twitter.com/silentsmoker99/status/1391383596309442567} 
  \textgreater{} ってか本当につけ入られたのかっつー話なんだよな。
  経験者だけど後出しジャンケンする女は結構いる。
\end{itemize}

〉〉〉 kk\_hironoのリツイート 〉〉〉

\begin{itemize}
\tightlist
\item
  RT
  kk\_hirono(刑事告発・非常上告_金沢地方検察庁御中)|big\_lawfirm(企業法務系弁護士その1)
  日時:2021-05-11 06:04/2021/05/09 16:16 URL:
  \url{https://twitter.com/kk\_hirono/status/1391861790644998144} 
  \url{https://twitter.com/big\_lawfirm/status/1391291088992755720} 
  \textgreater{}
  少し問題を整理した方がよくて、袖の中に手を入れて触った裁判官とか胸を触った大学教授とかは紛れもなく許されないと思います。
  他方で修習中にお世話になった先生と新人の頃に食事にいって不本意な関係になった話は(上司部下とか弁護士会の委員会等での上下関係等が無いことが前提ですが)自己責任
  \url{https://t.co/EPj5cjS6tt} 
\end{itemize}

〉〉〉 kk\_hironoのリツイート 〉〉〉

\begin{itemize}
\tightlist
\item
  RT
  kk\_hirono(刑事告発・非常上告_金沢地方検察庁御中)|silentsmoker99(肛 門 市 場)
  日時:2021-05-11 06:05/2021/05/09 22:07 URL:
  \url{https://twitter.com/kk\_hirono/status/1391861915966656512} 
  \url{https://twitter.com/silentsmoker99/status/1391379312805220352} 
  \textgreater{}
  なんとかきょうとか言うせんせい、だいぶメンタルお強そうですねw
\end{itemize}

〉〉〉 kk\_hironoのリツイート 〉〉〉

\begin{itemize}
\tightlist
\item
  RT
  kk\_hirono(刑事告発・非常上告_金沢地方検察庁御中)|silentsmoker99(肛 門 市 場)
  日時:2021-05-11 06:05/2021/05/09 16:22 URL:
  \url{https://twitter.com/kk\_hirono/status/1391861993154310145} 
  \url{https://twitter.com/silentsmoker99/status/1391292515496513537} 
  \textgreater{}
  未熟さにつけ入られた過去の自分を恥じているのか、相手を責めているのか。それによってだいぶ印象は違う。
\end{itemize}

〉〉〉 kk\_hironoのリツイート 〉〉〉

\begin{itemize}
\tightlist
\item
  RT
  kk\_hirono(刑事告発・非常上告_金沢地方検察庁御中)|student\_lawjpn(ほうがくともどき)
  日時:2021-05-11 06:05/2021/05/09 15:15 URL:
  \url{https://twitter.com/kk\_hirono/status/1391862077908652033} 
  \url{https://twitter.com/student\_lawjpn/status/1391275623067176961} 
  \textgreater{}
  成年女性が自由意志で性行為したいと思ってもそれは男性との対等ではない関係と女性の未熟さにつけ込まれたものであり、将来的に後悔することは明らかだし、そんな女性の未熟さにつけこんで性行為する男は許せないので、国家が婚前の女性の性行為を法的に一律禁止して保護すべきですよね。分かります。
\end{itemize}

〉〉〉 kk\_hironoのリツイート 〉〉〉

\begin{itemize}
\tightlist
\item
  RT
  kk\_hirono(刑事告発・非常上告_金沢地方検察庁御中)|silentsmoker99(肛 門 市 場)
  日時:2021-05-11 06:06/2021/05/09 14:54 URL:
  \url{https://twitter.com/kk\_hirono/status/1391862180316737537} 
  \url{https://twitter.com/silentsmoker99/status/1391270379864297474} 
  \textgreater{}
  不同意性交云々の話もそうだったけど、「後からならなんとでも言える」「言ったもん勝ちで言われた方は反論が難しい」というのもまた事実なので、TLに流れてくる過去の告白の全てを鵜呑みにする気はない。
\end{itemize}

〉〉〉 kk\_hironoのリツイート 〉〉〉

\begin{itemize}
\tightlist
\item
  RT
  kk\_hirono(刑事告発・非常上告_金沢地方検察庁御中)|silentsmoker99(肛 門 市 場)
  日時:2021-05-11 06:06/2021/05/09 14:02 URL:
  \url{https://twitter.com/kk\_hirono/status/1391862280300556290} 
  \url{https://twitter.com/silentsmoker99/status/1391257145581047812} 
  \textgreater{}
  話題の件に乗り遅れた理由がわかった。ミュートしてる人ばっかりだったw
\end{itemize}

〉〉〉 kk\_hironoのリツイート 〉〉〉

\begin{itemize}
\tightlist
\item
  RT
  kk\_hirono(刑事告発・非常上告_金沢地方検察庁御中)|nabeteru1Q78(渡辺輝人)
  日時:2021-05-11 06:07/2021/05/08 23:34 URL:
  \url{https://twitter.com/kk\_hirono/status/1391862586426085378} 
  \url{https://twitter.com/nabeteru1Q78/status/1391038688214196225} 
  \textgreater{}
  社会人が自分の名前で政治的な発言したら批判を受けるのは当然ですよ。
  \url{https://t.co/zUOk4CMpfV} 
\end{itemize}

 タイムラインに佐藤倫子弁護士関連のツイートがありました。再捜査要請書_警察庁・石川県警察御中(@kk\_hirono)のアカウントで見ているので,ブロックされて表示されていないアカウントもありそうです。一つ思い出したアカウントがありました。メンタルが強そうとか。

\begin{itemize}
\tightlist
\item
  〈〈〈 2021/05/11 06:11:43 Linux Emacs: 〈〈〈
\end{itemize}

\hypertarget{ux79c1ux305fux3061ux306fux30d9ux30c6ux30e9ux30f3ux306eux7537ux6027ux5f01ux8b77ux58ebux3042ux308bux3044ux306fux88c1ux5224ux5b98ux305fux3061ux306bux304aux4e16ux8a71ux306bux306aux308bux3068ux540cux6642ux306bux6027ux7684ux306bux643eux53d6ux3055ux308cux308bux3068ux3044ux3046ux4f50ux85e4ux502bux5b50ux5f01ux8b77ux58ebux306eux30c4ux30a4ux30fcux30c8ux3068ux4e09ux5b85ux4fcaux4e00ux90ceux88c1ux5224ux95773}{%
\paragraph{「私たちはベテランの男性弁護士(あるいは裁判官)たちにお世話になると同時に性的に搾取される」という佐藤倫子弁護士のツイートと,三宅俊一郎裁判長(3)}\label{ux79c1ux305fux3061ux306fux30d9ux30c6ux30e9ux30f3ux306eux7537ux6027ux5f01ux8b77ux58ebux3042ux308bux3044ux306fux88c1ux5224ux5b98ux305fux3061ux306bux304aux4e16ux8a71ux306bux306aux308bux3068ux540cux6642ux306bux6027ux7684ux306bux643eux53d6ux3055ux308cux308bux3068ux3044ux3046ux4f50ux85e4ux502bux5b50ux5f01ux8b77ux58ebux306eux30c4ux30a4ux30fcux30c8ux3068ux4e09ux5b85ux4fcaux4e00ux90ceux88c1ux5224ux95773}}

\begin{itemize}
\tightlist
\item
  〉〉〉 Linux Emacs: 2021/05/11 06:13:26 〉〉〉
\end{itemize}

:CATEGORIES: @kanazawabengosi \#金沢弁護士会 @JFBAsns
日本弁護士連合会(日弁連) \#法務省 @MOJ\_HOUMU \#三宅俊一郎裁判長
\#川口泰司裁判官 \#山田徹裁判官

\begin{itemize}
\tightlist
\item
  1356:2021-05-11\_05:03:57 \#告発状 \#\#\#\#
  「私たちはベテランの男性弁護士(あるいは裁判官)たちにお世話になると同時に性的に搾取される」という佐藤倫子弁護士のツイートと,三宅俊一郎裁判長(1)
  \url{https://hirono-hideki.hatenadiary.jp/entry/2021/05/11/050354} 
\item
  1357:2021-05-11\_06:13:01 \#告発状 \#\#\#\#
  「私たちはベテランの男性弁護士(あるいは裁判官)たちにお世話になると同時に性的に搾取される」という佐藤倫子弁護士のツイートと,三宅俊一郎裁判長(2)
  \url{https://hirono-hideki.hatenadiary.jp/entry/2021/05/11/061258} 
\end{itemize}

 ブログでは上記エントリーの続きになります。なべきょうこと渡邉恭子弁護士のことかと思ったのですが,Twitterの検索から「なべきょう」でアカウントを見つけました。しばらく見ていなかったのですが,いつのまにか非公開設定になっていました。

 現時点でなべきょう@過眠症というプロフィールの名前になっているTwitterアカウントは,3,257フォロー中,4,535フォロワーとなっています。非公開設定はやったことがないですが,少なくとも相互フォローの間では,通常のツイートのやりとりが出来るようです。

 もう2ヶ月ほど前になるのか,えきなんローヤーというアカウントもしばらく非公開設定にしていました。そう長くはなく10日から2週間程度だったように思います。試験的という説明があったかもしれません。女性弁護士の可能性が高い匿名アカウントです。

\begin{itemize}
\item
  2021年05月11日06時26分の登録:
  REGEXP:''@wata\_nabekyo\_ko''/データベース登録済みツイートの検索:2013-02-05〜2021-05-10/2021年05月11日06時21分の記録:ユーザ・投稿:123/561件
  \url{https://kk2020-09.blogspot.com/2021/05/regexpwatanabekyoko2013-02-052021-05.html} 
\item
  2021年05月11日06時29分の登録:
  REGEXP:''@wata\_nabekyo\_ko''/データベース登録済みツイートの検索:2021-05-04〜2021-05-11/2021年05月11日06時28分の記録:ユーザ・投稿:16/27件
  \url{https://kk2020-09.blogspot.com/2021/05/regexpwatanabekyoko2021-05-042021-05.html} 
\item
  (14/27) TW hirono\_hideki(奉納\さらば弁護士鉄道・泥棒神社の物語)
  日時: 2021-05-06 18:21:55 +0900 URL :
  \url{https://twitter.com/hirono\_hideki/status/1390235368386011138\textgreater} {}
  2021-05-06\_18:20
  奉納\\#危険生物・弁護士脳汚染除去装置\\#金沢地方検察庁御中\_2020:
  \なべきょう@過眠症 @wata\_nabekyo\_ko\返信先:
  @un\_co\_the2ndさん耳とか鼻とかになるともはや・・・
  \url{https://t.co/vzzwMLC13v} 
\end{itemize}

 内容はどうでもよくうの字への返信ツイートとして記録したものでした。5月6日として記録されており,少し思い出すような内容です。

\begin{itemize}
\tightlist
\item
  (19/27) TW hirono\_hideki(奉納\さらば弁護士鉄道・泥棒神社の物語)
  日時: 2021-05-08 15:41:00 +0900 URL :
  \url{https://twitter.com/hirono\_hideki/status/1390919645796864000\textgreater} {}
  2021-05-08\_14:26
  奉納\\#危険生物・弁護士脳汚染除去装置\\#金沢地方検察庁御中\_2020:
  \なべきょう@過眠症 @wata\_nabekyo\_ko\相互フォローの司法試験受験生の方のお父様の弁護士がコロナで亡くなら・・・
  \url{https://t.co/hndYKfi0gm} 
\end{itemize}

 思い出す内容のツイートですが,被害者安藤文さん家族の立場を重ねて考えたツイートでした。オリンピックを敵視する弁護士のツイートは多く見かけますが,将来的な賠償請求の布石としてアピールしているように思えるものも少なくありません。逆にそれが信用を落としそうでもあります。

 1,2二つのツイートで判断することはないですが総合的に考えています。時間の経過ともに記憶が混同する可能性もあり,記録はなるべく正確に残すように心がけています。アカウントそのものを取り違えてしまう可能性もあります。

\begin{itemize}
\item
  (4/100) TW wata\_nabekyo\_ko(なべきょう@過眠症) 日時:
  2021-05-08 13:35 URL :
  \url{https://twitter.com/wata\_nabekyo\_ko/status/1390887936074280964\textgreater} {}
  相互フォローの司法試験受験生の方のお父様の弁護士がコロナで亡くなられて、大変に嘆き悲しんでおられる。\textgreater\textgreater{}
  ・・・こういうの見ても、いや見ながら、オリンピックとか言ってられるの、ほんとわからない。
\item
  (3/100) TW wata\_nabekyo\_ko(なべきょう@過眠症) 日時:
  2021-05-08 13:41 URL :
  \url{https://twitter.com/wata\_nabekyo\_ko/status/1390889616195002369\textgreater} {}
  @jsddg173kongo
  ちょっとそんなふうに今朝までは思ってたけど、相互さんのお父様がコロナで亡くなって「夢や希望なんかより家族が生きてる姿が見たい」って悲しんでおられたのを今朝見てからは意見が変わりました。
\item
  (2/100) TW wata\_nabekyo\_ko(なべきょう@過眠症) 日時:
  2021-05-08 13:58 URL :
  \url{https://twitter.com/wata\_nabekyo\_ko/status/1390893751678033920\textgreater} {}
  そりゃ選手どころか関係者全員恨まれるわ少なくとも私なら恨むよ。\textgreater{}
  亡くなった人、その家族や恋人、それらの無念や恨みが、選手はもちろん関係者全員に向くのは自然だ。\textgreater{}
  それを背負ってまですることかオリンピックは?医療資源を回してもらって、他人の家族を死なせてするスポーツは楽しいか?
\end{itemize}

\begin{lstlisting}
py37_env ❯ tun fukazawas 1
\end{lstlisting}

\begin{itemize}
\tightlist
\item
  RT fukazawas(深澤諭史)|shimadayusuke66(島田雄左) 日時:2021-04-26
  07:41/2021-04-23 18:57 URL:
  \url{https://twitter.com/fukazawas/status/1386450393157169156} 
  \url{https://twitter.com/shimadayusuke66/status/1385533372798173191} 
  \textgreater{}
  弁護士や司法書士の資格取得後、以前は、丁稚奉公したら独立するのスタンダードでした。でも、今は独立に限らず、専門性を高めたり事務所の総合化などに伴い、色んなキャリアプランが増えたように感じます。そう考えると、資格者の役割も多様化してるので、資格取得後のキャリアプランは無限大です。
\end{itemize}

 上記の通り4月26日のリツイートから深澤諭史弁護士のタイムラインの更新が停止していますが,昨年2020年4月頃の深澤諭史弁護士のツイートというのは,コロナ禍を弁護士商売の稼ぎ時と,はしゃぎ喜んでいるように感じられるツイートが多々散見されました。

\begin{lstlisting}
py37_env ❯ d|grep @fukazawas|grep -E '^- 2020年04月.+'|grep -v 'ツイートの記録資料:'|grep -v の検索
\end{lstlisting}

\begin{itemize}
\tightlist
\item
  2020年04月01日17時35分の登録: \papadenka @holmesdenka\返信先:
  ¥\n@fukazawas¥\nさん¥\nどっちかって言うと、なんちゃって神主やな・・・
  \url{http://hirono2014sk.blogspot.com/2020/04/papadenkaholmesdenka-fukazawas.html} 
\item
  2020年04月01日21時43分の登録:
  \深澤諭史 @fukazawas\(・∀・)エイプリルフールネタですが。¥\n(^ω^)実は、コスプレではないのだお。
  \url{http://hirono2014sk.blogspot.com/2020/04/fukazawas.html} 
\item
  2020年04月01日21時50分の登録:
  \深澤諭史 @fukazawas\これからは、公家・貴族弁護士として活動し、相手方には容赦なく朝敵の汚名を着せて戦うことにしたいと思います。¥\n(・∀・)¥\n
  \url{http://hirono2014sk.blogspot.com/2020/04/fukazawas\_1.html} 
\item
  2020年04月01日23時50分の登録: \櫻井光政 @okinahimeji\返信先:
  ¥\n@sansyoub¥\nさん,
  ¥\n@fukazawas¥\nさん¥\n仰る通り,裁判員がなくても証拠開示はできるのですが,裁判員制度
  \url{http://hirono2014sk.blogspot.com/2020/04/okinahimeji-sansyoub-fukazawas.html} 
\item
  2020年04月02日12時51分の登録:
  \深澤諭史 @fukazawas\(#・∀・)コラー!お前らがお肉券とお魚券のことを非難しまくったせいで、これがマスク2枚に化けちゃったじゃないか!¥\n(^ω^)お前も、批
  \url{http://hirono2014sk.blogspot.com/2020/04/fukazawas\_2.html} 
\item
  2020年04月02日12時53分の登録:
  \深澤諭史 @fukazawas\(*・∀・)お肉券とお魚券くらいで、愚策呼ばわりとは、日本人は滑稽だねぇ!¥\n(;^ω^)た、谷岡さん!¥\n(・∀・)エイプリルフールに発表
  \url{http://hirono2014sk.blogspot.com/2020/04/fukazawas\_86.html} 
\item
  2020年04月02日12時54分の登録:
  \深澤諭史 @fukazawas\(・∀・)「マスク2枚の政策は不適切かもしれないけれども、他にいろいろやっているんだから、批判するのはおかしい!」っていう意見に納得しち
  \url{http://hirono2014sk.blogspot.com/2020/04/fukazawas\_46.html} 
\item
  2020年04月02日12時55分の登録:
  \深澤諭史 @fukazawas\こうやって個別に抗議するのも大事だけど、弁護士としては、非弁行為者の代理(紛い)は無視するというのも大事かと。¥\n東京地裁は、最近、本人訴
  \url{http://hirono2014sk.blogspot.com/2020/04/fukazawas\_48.html} 
\item
  2020年04月02日13時02分の登録:
  \深澤諭史 @fukazawas\こういう感じで、個別の非弁業者や非弁事業について、それぞれ事業者や業態(種類)を特定して、日弁連がこまめに声明を出せるというのが理想なん
  \url{http://hirono2014sk.blogspot.com/2020/04/fukazawas\_43.html} 
\item
  2020年04月02日13時02分の登録:
  \深澤諭史 @fukazawas\広島県行政書士会の会長声明に対する抗議文
  \textbar{} 広島弁護士会¥\n(#・∀・)よくやった!!
  \url{http://hirono2014sk.blogspot.com/2020/04/fukazawas\_57.html} 
\item
  2020年04月02日13時07分の登録: \野田隼人 @nodahayato\返信先:
  ¥\n@fukazawas¥\nさん¥\n考査委員は使い捨て要員・・・。
  \url{http://hirono2014sk.blogspot.com/2020/04/nodahayato-fukazawas.html} 
\item
  2020年04月02日13時08分の登録:
  \深澤諭史 @fukazawas\(#・∀・)マスクは1家庭に2枚!咳き込む者が装着し、咳き込んでいない者は後に続け!!賃金保障はしない!満員電車からの後退は認めん!¥\n(
  \url{http://hirono2014sk.blogspot.com/2020/04/fukazawas\_74.html} 
\item
  2020年04月02日13時11分の登録:
  \深澤諭史 @fukazawas\飛行機の安全性と、フィンランドの教育は、このメソッドがすごいうまく行った例。¥\n(・∀・)
  \url{http://hirono2014sk.blogspot.com/2020/04/fukazawas\_79.html} 
\item
  2020年04月02日13時13分の登録:
  \深澤諭史 @fukazawas\(・∀・)日本の政策、制度設計における「きっと、みんな、俺の思うように、自分の利害を投げ打って行動してくれるだろう」という謎の都合の良い
  \url{http://hirono2014sk.blogspot.com/2020/04/fukazawas\_65.html} 
\item
  2020年04月02日13時17分の登録:
  \深澤諭史 @fukazawas\司法試験、これでも延期しないというのであれば、試験監督は、派遣バイトではなくて、司法試験考査委員の先生方が、やるようにすればいいと思う。
  \url{http://hirono2014sk.blogspot.com/2020/04/fukazawas\_4.html} 
\item
  2020年04月02日13時18分の登録:
  \深澤諭史 @fukazawas\テレワーク、本当に生産性上がるな・・・。¥\n問題は、ひたすら黙々と仕事仕事仕事になり、ブラック企業を従業員個人が自主製作してしまうところ。
  \url{http://hirono2014sk.blogspot.com/2020/04/fukazawas\_41.html} 
\item
  2020年04月02日13時34分の登録:
  \深澤諭史 @fukazawas\お願いですから,現金でなくてもいいですから・・,マスクではなくて,お肉券とお魚券をください・・・!
  \#コロナ黙示録
  \url{http://hirono2014sk.blogspot.com/2020/04/fukazawas\_36.html} 
\item
  2020年04月02日17時02分の登録:
  \深澤諭史 @fukazawas\(T∀T)なんで,ここ最近,政府は,冗談だったらつまらないし,本気だったらタチが悪いレベルの政策ばかり連発するのだ・・・。日本人というか
  \url{http://hirono2014sk.blogspot.com/2020/04/fukazawas\_77.html} 
\item
  2020年04月02日17時02分の登録:
  \深澤諭史 @fukazawas\(;・∀・)ちゃんと特典ありますよ!¥\n上級国民:国の要職とか利権とか,桜を見る会招待券とか,国有地9割引購入権とかがもらえます。¥\nそれ以
  \url{http://hirono2014sk.blogspot.com/2020/04/fukazawas\_9.html} 
\item
  2020年04月02日19時28分の登録:
  \深澤諭史 @fukazawas\非弁委員会の開催は用心しないとな。。。¥\n感染がそこで広がったら、非弁業者と非弁提携弁護士が喜ぶだけだ。。。。。
  \url{http://hirono2014sk.blogspot.com/2020/04/fukazawas\_78.html} 
\item
  2020年04月02日21時27分の登録:
  \深澤諭史 @fukazawas\(;・∀・)「公正世界仮説」を壊されるのがたまらなく我慢できない人って、それなりにいます。¥\n私たち法曹は、そうでないと骨身に染みてわかっ
  \url{http://hirono2014sk.blogspot.com/2020/04/fukazawas\_20.html} 
\item
  2020年04月02日21時31分の登録:
  \深澤諭史 @fukazawas\「こういう事件では、弁護士なんかつけなくても訴訟は大丈夫!みんなでネットde真実に目覚めよう!」みたいな話はTwitterでもたまに聞く
  \url{http://hirono2014sk.blogspot.com/2020/04/fukazawastwitter.html} 
\item
  2020年04月02日23時30分の登録:
  \深澤諭史 @fukazawas\¥\n深澤諭史¥\n@fukazawas¥\n·¥\n31分¥\n今年度から、二弁の某委員会の委員長をやる予定です・・・。¥\n休めないお(;ω;)
  \url{http://hirono2014sk.blogspot.com/2020/04/fukazawas-fukazawas-31.html} 
\item
  2020年04月03日12時43分の登録:
  \深澤諭史 @fukazawas\(#・∀・)これは ひどい。¥\n(#^ω^)なるほど,首相主催のパーティーに投資詐欺会社を呼ぶのは「批判はあたらない」が,夜の仕事で生きて
  \url{http://hirono2014sk.blogspot.com/2020/04/fukazawas\_3.html} 
\item
  2020年04月03日13時12分の登録:
  \深澤諭史 @fukazawas\いや、本当に対応しないとまずいですよね。¥\nこのままだと、実務修習地は地獄で、閻魔大王が刑裁実務の指導担当とか、そういうことになってしまい
  \url{http://hirono2014sk.blogspot.com/2020/04/fukazawas\_35.html} 
\item
  2020年04月03日13時14分の登録:
  \深澤諭史 @fukazawas\少額訴訟という制度は「ネットde真実に目覚めたが,この制度を使えば,本人訴訟で簡単に,正しい俺の言い分が通るはずだ」というものではありま
  \url{http://hirono2014sk.blogspot.com/2020/04/fukazawas\_90.html} 
\item
  2020年04月03日13時16分の登録:
  \深澤諭史 @fukazawas\先生,それは「腹痛がするので,自宅で簡単な盲腸手術をしようと思います。手術道具と薬は,ドラッグストアで買えますか?ホムセンがいいでしょう
  \url{http://hirono2014sk.blogspot.com/2020/04/fukazawas\_88.html} 
\item
  2020年04月03日13時20分の登録:
  \深澤諭史 @fukazawas\44歳無職男、亡き母の意向で口座から金を引き出せず激怒
  信用金庫のアクリル板を破壊して逮捕 \textbar{} リアルライブ
  \url{http://hirono2014sk.blogspot.com/2020/04/fukazawas44.html} 
\item
  2020年04月04日01時31分の登録:
  \深澤諭史 @fukazawas\(・∀・)ほんこれ。¥\n(^ω^)受験生という発言力のない人に、一方的に安全圏にいる人が、リスクを負担させるには、正義に悖るお。¥\n引用ツイ
  \url{http://hirono2014sk.blogspot.com/2020/04/fukazawas\_75.html} 
\item
  2020年04月04日15時01分の登録:
  \深澤諭史 @fukazawas\法的助言であれば、アウトです。
  \url{http://hirono2014sk.blogspot.com/2020/04/fukazawas\_11.html} 
\item
  2020年04月04日15時04分の登録:
  \深澤諭史 @fukazawas\相談の標示も実行も非弁行為になると理解しています。
  \url{http://hirono2014sk.blogspot.com/2020/04/fukazawas\_94.html} 
\item
  2020年04月04日15時04分の登録:
  \深澤諭史 @fukazawas\弁護士会に通報!!¥\n相手方に出てきたら、非弁と指摘して徹底無視ですね。
  \url{http://hirono2014sk.blogspot.com/2020/04/fukazawas\_85.html} 
\item
  2020年04月04日15時13分の登録:
  \深澤諭史 @fukazawas\依頼者の言い分を整序しただけの文書を作成し、提出する行為だけですね。¥\n基本的に、賠償関係では関与できることはないと思います。
  \url{http://hirono2014sk.blogspot.com/2020/04/fukazawas\_5.html} 
\item
  2020年04月04日15時18分の登録:
  \深澤諭史 @fukazawas\(・∀・)今の政権が超優秀で、野党が絶対に勝てないのは、これを見て「ひどい!批判だ!」って思ってしまう人たちのリテラシーを、うまく把握し
  \url{http://hirono2014sk.blogspot.com/2020/04/fukazawas\_27.html} 
\item
  2020年04月04日23時21分の登録:
  \深澤諭史 @fukazawas\公正世界仮説のせいだという独自説に1票。¥\n(・∀・)(^ω^)
  \url{http://hirono2014sk.blogspot.com/2020/04/fukazawas1.html} 
\item
  2020年04月04日23時48分の登録:
  \深澤諭史 @fukazawas\浦沢直樹が描いた安倍の似顔絵「アベノマスク」に普通の日本人さんたちが発狂してしまう
  - なんJ政治ネタまとめ
  \url{http://hirono2014sk.blogspot.com/2020/04/fukazawas-j.html} 
\item
  2020年04月04日23時48分の登録:
  \深澤諭史 @fukazawas\ほんと、平成の司法改革は恐ろしいことをしたものだ。。。¥\n(・∀・#)
  \url{http://hirono2014sk.blogspot.com/2020/04/fukazawas\_13.html} 
\item
  2020年04月05日00時19分の登録: \深澤諭史 @fukazawas\¥\n550
  フォロー中¥\n6,853
  フォロワー¥\nフォローしている人にフォロワーはいません¥\n¥\n深澤諭史¥\n10.6万
  ツイート¥\nフォロー¥\n新
  \url{http://hirono2014sk.blogspot.com/2020/04/fukazawas-550-6853-106.html} 
\item
  2020年04月05日00時36分の登録: \黒葛原 歩 @ATsZRA\返信先:
  ¥\n@fukazawas¥\nさん¥\nまさにその通りだと思いますし、残念ながら、今はそれが必要な時期でもあると思います。
  \url{http://hirono2014sk.blogspot.com/2020/04/atszra-fukazawas.html} 
\item
  2020年04月05日01時32分の登録:
  REGEXP:''@fukazawas''/データベース登録済みツイート:2020年04月05日01時13分の記録:ユーザ・投稿:467/7894件
  \url{http://hirono2014sk.blogspot.com/2020/04/regexpfukazawas2020040501134677894.html} 
\item
  2020年04月05日09時34分の登録:
  \深澤諭史 @fukazawas\(;・∀・)とりあえず,文句,ケチ付けることで,「優秀な僕がいい仕事した」アピールしたがる人がいますので・・・。
  \url{http://hirono2014sk.blogspot.com/2020/04/fukazawas\_71.html} 
\item
  2020年04月05日09時35分の登録:
  \深澤諭史 @fukazawas\いや,本当に,司法試験受験生をオモチャにするのは,今すぐやめて欲しい。¥\n平成の司法改革の時も酷かった。¥\n当時関与した人は,「このオモチャ
  \url{http://hirono2014sk.blogspot.com/2020/04/fukazawas\_15.html} 
\item
  2020年04月05日12時52分の登録:
  \深澤諭史 @fukazawas\本当に・・、感染を防ぎたい、という気持ちで胸が一杯ならば、何日でも自粛ができるはずなんじゃ・・。¥\nたとえ、それが、あくまで任意の自粛だか
  \url{http://hirono2014sk.blogspot.com/2020/04/fukazawas\_91.html} 
\item
  2020年04月05日13時00分の登録:
  \深澤諭史 @fukazawas\平成の司法改革では、なぜか一部の学者先生は、「新しい時代に適応できない、落伍した弁護士が世間を恨んでいるだけ!」みたいに、思いたがるけれ
  \url{http://hirono2014sk.blogspot.com/2020/04/fukazawas\_38.html} 
\item
  2020年04月05日13時00分の登録:
  \深澤諭史 @fukazawas\あろうことか、政府を頼ってしまった・・!¥\nもう、お肉券とお魚券、アベノマスクで分かっていたはずなのに、自分以外頼れないこと、政府に期待で
  \url{http://hirono2014sk.blogspot.com/2020/04/fukazawas\_42.html} 
\item
  2020年04月05日13時02分の登録:
  \深澤諭史 @fukazawas\ネットで「こういう判決をもらった」って非法曹がいっているのは,結構な割合で,エア判決だったりします。¥\nいわゆる創作実話のバリエーションで
  \url{http://hirono2014sk.blogspot.com/2020/04/fukazawas\_40.html} 
\item
  2020年04月05日13時15分の登録:
  \深澤諭史 @fukazawas\(;;・∀・)とりあえず,不安になったので,口座を決済性にした。金融機関破綻時に全額保証がつくという。¥\n(;;^ω^)インフレには無力だ
  \url{http://hirono2014sk.blogspot.com/2020/04/fukazawas\_47.html} 
\item
  2020年04月05日13時16分の登録:
  \深澤諭史 @fukazawas\(;;・∀・)とりあえず,不安になったので,口座を決済性にした。金融機関破綻時に全額保証がつくという。¥\n(;;^ω^)インフレには無力だ
  \url{http://hirono2014sk.blogspot.com/2020/04/fukazawas\_23.html} 
\item
  2020年04月05日15時48分の登録:
  \深澤諭史 @fukazawas\(・∀・)これで北朝鮮が「だって、たくさん検査したら、医療崩壊になるんだもん」って反論したら笑える。¥\n(^ω^)泣けるお・・・。
  \url{http://hirono2014sk.blogspot.com/2020/04/fukazawas\_89.html} 
\item
  2020年04月05日19時38分の登録:
  \深澤諭史 @fukazawas\(;・∀・)電車移動は、極力オフピーク、そして全部グリーン車使おう。リスクは全力で回避しないと。。。
  \url{http://hirono2014sk.blogspot.com/2020/04/fukazawas\_25.html} 
\item
  2020年04月05日19時39分の登録:
  \深澤諭史 @fukazawas\お次は「知らないではすまされない
  インターネット利用の心得
  ケーススタディ(2014年、きんざい)」¥\nこれも共著です。金融機関の従業員向
  \url{http://hirono2014sk.blogspot.com/2020/04/fukazawas\_10.html} 
\item
  2020年04月05日22時17分の登録:
  \深澤諭史 @fukazawas\最近だと慰謝料300万円請求中30万円認容、弁護士費用実費約220万円請求中30万円認容というのがありますね。¥\n発信者が弁護士つけて応訴
  \url{http://hirono2014sk.blogspot.com/2020/04/fukazawas\_34.html} 
\item
  2020年04月05日22時31分の登録:
  \深澤諭史 @fukazawas\ちょっと前まで、政府の方針で大喜利するな!って言われていたのに。¥\nまさか、政府が政策で大喜利始めるとは思わなかった・・・・。¥\n(・∀・;
  \url{http://hirono2014sk.blogspot.com/2020/04/fukazawas\_61.html} 
\item
  2020年04月06日11時25分の登録: \野田隼人 @nodahayato\@fukazawas¥\n
  先生,これ,どこか(web上で)に書いてましたっけ?
  \url{http://hirono2014sk.blogspot.com/2020/04/nodahayatofukazawas-web.html} 
\item
  2020年04月06日13時26分の登録: \深澤諭史 @fukazawas\佐々木紀
  \url{http://hirono2014sk.blogspot.com/2020/04/fukazawas\_6.html} 
\item
  2020年04月06日13時44分の登録:
  \深澤諭史 @fukazawas\これは,一部での話しですが,一部のテレビ関係者における,「とにかく人を大事にしない。敬意を持って扱わない。」という傾向は,どうにかならな
  \url{http://hirono2014sk.blogspot.com/2020/04/fukazawas\_30.html} 
\item
  2020年04月06日22時41分の登録:
  \深澤諭史 @fukazawas\なんと、弁護士会の本屋にも並んでいました・・・。驚き・・・!!!!
  \url{http://hirono2014sk.blogspot.com/2020/04/fukazawas\_29.html} 
\item
  2020年04月06日22時47分の登録:
  \向原総合法律事務所 弁護士向原 @harrier0516osk\返信先:
  ¥\n@fukazawas¥\nさん¥\n2日目¥\n「私が事務局長です。よろしく」¥\n颯爽とベンツGLSで出勤
  \url{http://hirono2014sk.blogspot.com/2020/04/harrier0516osk-fukazawas-2-gls.html} 
\item
  2020年04月06日22時58分の登録:
  \深澤諭史 @fukazawas\1日目¥\n(^ω^)なかなかいい感じの事務所に就職決まったお!老舗事務所だし、ボスは有名だし、兄弁は超優秀だお!¥\n破滅まであと99日¥\n\#1
  \url{http://hirono2014sk.blogspot.com/2020/04/fukazawas1-99-1.html} 
\item
  2020年04月06日23時09分の登録:
  \深澤諭史 @fukazawas\いつもの勾留と保釈のときの、罪証隠滅と逃亡のリスクと同じくらい重く判断してくれたなら、既に全閉鎖ですよね(笑)¥\n引用ツイート¥\n
  \url{http://hirono2014sk.blogspot.com/2020/04/fukazawas\_73.html} 
\item
  2020年04月06日23時12分の登録:
  \深澤諭史 @fukazawas\むしろこっちが「何言ってんだこいつ」って反応したくなりますね。¥\n(・∀・;)
  \url{http://hirono2014sk.blogspot.com/2020/04/fukazawas\_56.html} 
\item
  2020年04月06日23時19分の登録:
  \深澤諭史 @fukazawas\(;;・∀・)弁護士法72条の何の要件の問題なのか、よくわからない・・・。
  \url{http://hirono2014sk.blogspot.com/2020/04/fukazawas\_59.html} 
\item
  2020年04月07日13時30分の登録:
  \深澤諭史 @fukazawas\2日目¥\n(^ω^)いよいよ執務開始だお!ボスはとても有名な弁護士らしいお!打ち合わせに同席したけれども,凄い優良企業ばかりだお!早く私も
  \url{http://hirono2014sk.blogspot.com/2020/04/fukazawas\_7.html} 
\item
  2020年04月07日21時46分の登録:
  \深澤諭史 @fukazawas\(・∀・)正直、自粛で一部の人に泣き寝入りを強制するより、
  ¥\n@o2441¥\n
  大西先生が言うように、売却して原資にして補償する方が、フェ
  \url{http://hirono2014sk.blogspot.com/2020/04/fukazawas-o2441.html} 
\item
  2020年04月07日21時48分の登録:
  \深澤諭史 @fukazawas\法クラの皆さん,是非是非是非(・∀・)
  \url{http://hirono2014sk.blogspot.com/2020/04/fukazawas\_72.html} 
\item
  2020年04月07日21時50分の登録:
  \深澤諭史 @fukazawas\刑弁引き取りMLが,すごいことになっている・・・!
  \url{http://hirono2014sk.blogspot.com/2020/04/fukazawas\_54.html} 
\item
  2020年04月08日23時04分の登録:
  \向原総合法律事務所 弁護士向原 @harrier0516osk\返信先:
  ¥\n@fukazawas¥\nさん¥\nお金の問題より、弁護士に対しての扱いの雑さと、キワキワの場面で依
  \url{http://hirono2014sk.blogspot.com/2020/04/harrier0516osk-fukazawas.html} 
\item
  2020年04月09日09時57分の登録:
  \深澤諭史 @fukazawas\ボスは結構なお年だから、役割分担なんだお!¥\n破滅まであと96日¥\n\#100日後に非弁提携で破滅する弁護士
  \url{http://hirono2014sk.blogspot.com/2020/04/fukazawas-100.html} 
\item
  2020年04月09日15時19分の登録:
  \深澤諭史 @fukazawas\(;・∀・)このご時世に,テレビの出演依頼が・・・。
  \url{http://hirono2014sk.blogspot.com/2020/04/fukazawas\_26.html} 
\item
  2020年04月09日18時31分の登録:
  \深澤諭史 @fukazawas\(・∀・)一枚200円の最高級品ですよ,大事にしないと。
  \url{http://hirono2014sk.blogspot.com/2020/04/fukazawas\_98.html} 
\item
  2020年04月09日18時32分の登録:
  \深澤諭史 @fukazawas\(;・∀・)ほんこれ。専門家というか,専門知識を軽視,蔑視,もっといえば敵視する謎の風潮が・・・。¥\n(;^ω^)ぶっちゃけると,法科大学
  \url{http://hirono2014sk.blogspot.com/2020/04/fukazawas\_12.html} 
\item
  2020年04月09日18時34分の登録:
  \深澤諭史 @fukazawas\(・∀・)この一週間で3件受任して,しかも取材やらなんやらあったが,十分対応できている。テレワークのおかげだな。¥\n(;・∀・)さすがに限
  \url{http://hirono2014sk.blogspot.com/2020/04/fukazawas\_17.html} 
\item
  2020年04月09日18時37分の登録:
  \深澤諭史 @fukazawas\(;・∀・)あれ?ツイッターランドによると,韓国には,日本みたいに森羅万象を担当し,立法府の長と首相を兼任する偉大なる指導者がいないから
  \url{http://hirono2014sk.blogspot.com/2020/04/fukazawas\_8.html} 
\item
  2020年04月09日21時35分の登録:
  \深澤諭史 @fukazawas\(・∀・)偉大なる指導者が1枚200円の最高級マスクを2枚も下賜してくださるのに、文句を言う奴は許せないですね。¥\n(^ω^)懐かしい布製
  \url{http://hirono2014sk.blogspot.com/2020/04/fukazawas12002.html} 
\item
  2020年04月10日21時16分の登録:
  \深澤諭史 @fukazawas\事業者の相談に乗っているけれども,本当につらい・・・。¥\n自粛しろ,補償は決めていない,だと,良心を人質にとられ,公衆衛生と従業員の人生を
  \url{http://hirono2014sk.blogspot.com/2020/04/fukazawas\_28.html} 
\item
  2020年04月10日21時17分の登録:
  \深澤諭史 @fukazawas\ぶっちゃけ,年単位の運転資金確保しているけれども,これは,弁護士だから出来ることで,これから先,山のように世間では倒産が出るだろうな・・
  \url{http://hirono2014sk.blogspot.com/2020/04/fukazawas\_49.html} 
\item
  2020年04月10日21時18分の登録:
  \深澤諭史 @fukazawas\(・∀・)私も,華麗に仮差押えを決めたり,訴訟で大勝利したり,あるいは非弁業者を成敗すると,そのタイミングで,GoogleマップとかAm
  \url{http://hirono2014sk.blogspot.com/2020/04/fukazawas\_33.html} 
\item
  2020年04月11日13時14分の登録:
  \深澤諭史 @fukazawas\(*・∀・)土日返上で,顧問業務なう!¥\n(*^ω^)私に出来る,コロナ禍との戦いへの貢献ですお!!
  \url{http://hirono2014sk.blogspot.com/2020/04/fukazawas\_60.html} 
\item
  2020年04月11日13時31分の登録:
  \深澤諭史 @fukazawas\(・∀・)なんか,新型コロナの感染経路がわからないことが問題になっていますが,この国は,一国の宰相の行動とか,税金の使い道の記録が一日で
  \url{http://hirono2014sk.blogspot.com/2020/04/fukazawas\_99.html} 
\item
  2020年04月11日13時57分の登録:
  \深澤諭史 @fukazawas\コロナ禍の影響で、顧問先からたくさんの相談が・・・。¥\nこういう時のための契約なので、がっつり対応します!¥\n(・∀・)
  \url{http://hirono2014sk.blogspot.com/2020/04/fukazawas\_81.html} 
\item
  2020年04月11日14時13分の登録:
  \深澤諭史 @fukazawas\(;・∀・)特定の某プロバイダは、裁判なしでガンガン発信者情報開示に応じているみたいだな。。。¥\n裁判までは大丈夫だと思ったら開示されたと
  \url{http://hirono2014sk.blogspot.com/2020/04/fukazawas\_96.html} 
\item
  2020年04月11日17時55分の登録:
  \深澤諭史 @fukazawas\マスク不足に苦しむ国民の皆様に、
  \#アベノマスク
  2枚を世帯ごとに配布する我々が、無能なわけがない。¥\n配布するマスクは、1枚200円の最
  \url{http://hirono2014sk.blogspot.com/2020/04/fukazawas-2.html} 
\item
  2020年04月11日21時01分の登録:
  \深澤諭史 @fukazawas\【利権】政府様「休業補償はしない。でも和牛の販促に500億投入するわwwwwwwwwww」
  - なんJ政治ネタまとめ
  \url{http://hirono2014sk.blogspot.com/2020/04/fukazawas500-j.html} 
\item
  2020年04月11日21時03分の登録:
  \深澤諭史 @fukazawas\(;・∀・)ぶっちゃけ、日本スゴイに逃げ込むことができれば、どれだけ楽だろうかって、真剣に思っちゃいますけれども・・。¥\n法曹の端くれとし
  \url{http://hirono2014sk.blogspot.com/2020/04/fukazawas\_70.html} 
\item
  2020年04月12日09時15分の登録:
  \深澤諭史 @fukazawas\北先生の事件の時も「訴状は無視すれば大丈夫!」みたいな、都合の良いデマが流行りましたが、この分野も「裁判にならないと開示されないから、そ
  \url{http://hirono2014sk.blogspot.com/2020/04/fukazawas\_52.html} 
\item
  2020年04月12日09時41分の登録:
  \深澤諭史 @fukazawas\ざわ。。。。ざわ。。。。ざわ。。。ざわ。。。
  \url{http://hirono2014sk.blogspot.com/2020/04/fukazawas\_64.html} 
\item
  2020年04月12日21時54分の登録:
  \深澤諭史 @fukazawas\テレワーク、すごい、日曜夜も仕事できる!!!
  \url{http://hirono2014sk.blogspot.com/2020/04/fukazawas\_45.html} 
\item
  2020年04月13日12時43分の登録:
  \深澤諭史 @fukazawas\実際問題として、相談は無料で当たり前、みたいな発想のままの人の事件を受任すると、弁護士もアドバイスや指示を軽視しまくるので、結果は、弁護
  \url{http://hirono2014sk.blogspot.com/2020/04/fukazawas\_55.html} 
\item
  2020年04月13日13時52分の登録:
  \深澤諭史 @fukazawas\向原先生から,貴重なデータをもらうなど。¥\nもう,日弁連は向原先生と委任契約結んで,非弁対策のアドバイザー兼ハンターにするべき。
  \url{http://hirono2014sk.blogspot.com/2020/04/fukazawas\_93.html} 
\item
  2020年04月13日13時53分の登録:
  \深澤諭史 @fukazawas\(#・∀・)早く,いつまでにいくらか,とっとと明らかにして欲しい。¥\n(^ω^)えー,いわば,これまでにない,強い取り組みを,弾力的におこ
  \url{http://hirono2014sk.blogspot.com/2020/04/fukazawas\_32.html} 
\item
  2020年04月13日14時28分の登録:
  \駅員→車掌→裁判官な鉄ヲタ @judgeman246\返信先:
  ¥\n@fukazawas¥\nさん¥\n購入します。
  \url{http://hirono2014sk.blogspot.com/2020/04/judgeman246-fukazawas.html} 
\item
  2020年04月13日14時57分の登録:
  \深澤諭史 @fukazawas\(;・∀・)わざわざ,よりによって,櫻井先生のところを狙うとは,むしろ,相手方の会社に同情を禁じ得ない・・・。
  \url{http://hirono2014sk.blogspot.com/2020/04/fukazawas\_51.html} 
\item
  2020年04月13日18時51分の登録:
  \深澤諭史 @fukazawas\(;・∀・)自分たちが理解できない、気に食わない仕事をする弁護士を、仕事にあぶれている弁護士が金目当てでやっているって思い込むのは、その
  \url{http://hirono2014sk.blogspot.com/2020/04/fukazawas\_21.html} 
\item
  2020年04月13日18時53分の登録:
  \肉弁護士@士業から事業へ @tajima38186887\返信先:
  ¥\n@fukazawas¥\nさん¥\n私は公設で30分で法律相談をまとめ上げるスキルを身に付けましたw
  \url{http://hirono2014sk.blogspot.com/2020/04/tajima38186887-fukazawas-30w.html} 
\item
  2020年04月13日22時35分の登録:
  \深澤諭史 @fukazawas\(・∀・)50枚4900円の転売マスクは叩かれる一方で、1枚200円(50枚1万円)経費別の布マスクが税金で配布される美しい日本!¥\n(#
  \url{http://hirono2014sk.blogspot.com/2020/04/fukazawas\_58.html} 
\item
  2020年04月13日23時06分の登録:
  \深澤諭史 @fukazawas\(#・∀・)人の弱みにつけ込んで、転売マスク売るやつ許せねー!!¥\n(^ω^)とか思っていたら、まさか、税金で、転売マスクを遥かに超えるコ
  \url{http://hirono2014sk.blogspot.com/2020/04/fukazawas\_19.html} 
\item
  2020年04月13日23時08分の登録:
  \佐藤正子 @SATOMasako\¥\n深澤諭史¥\n@fukazawas¥\n·¥\n37分¥\n(#・∀・)人の弱みにつけ込んで、転売マスク売るやつ許せねー!!¥\n(^ω^)とか思って
  \url{http://hirono2014sk.blogspot.com/2020/04/satomasako-fukazawas-37.html} 
\item
  2020年04月13日23時15分の登録:
  \深澤諭史 @fukazawas\(・∀・)私の知っている弁護士業界とは随分違うようだ。。。¥\n(^ω^)だお。
  \url{http://hirono2014sk.blogspot.com/2020/04/fukazawas\_18.html} 
\item
  2020年04月14日08時39分の登録:
  \深澤諭史 @fukazawas\(・∀・)こう言っちゃ怒られそうだけれども。¥\n(^ω^)「モリカケ」「桜」で中央政府がやったことを香川県議会がやって・・¥\n(・∀・)「モ
  \url{http://hirono2014sk.blogspot.com/2020/04/fukazawas\_14.html} 
\item
  2020年04月14日10時59分の登録:
  \深澤諭史 @fukazawas\(・∀・)「国選報酬に不服申し立てをするスタ弁」の話を聞くたびに,スタッフ弁護士というか,この国の弁護士も,まだまだ捨てたもんじゃないな
  \url{http://hirono2014sk.blogspot.com/2020/04/fukazawas\_87.html} 
\item
  2020年04月14日11時03分の登録:
  \深澤諭史 @fukazawas\(・∀・)ガルパンの「戦車道」って「せんしゃみち」って読むの?
  \url{http://hirono2014sk.blogspot.com/2020/04/fukazawas\_80.html} 
\item
  2020年04月14日11時05分の登録:
  \深澤諭史 @fukazawas\適当に独立しちゃった私は,決して,偉そうなこといえないが,本当に独立は慎重に。¥\n勤務弁護士・インハウスから独立弁護士への変化って,司法修
  \url{http://hirono2014sk.blogspot.com/2020/04/fukazawas\_53.html} 
\item
  2020年04月14日11時06分の登録:
  \深澤諭史 @fukazawas\はじめて話す人,しかも予備情報なしで,その人がどういう人か,自分の事業との関係で,どういうリスクがあるか,それを一瞬で判断するスキルは独
  \url{http://hirono2014sk.blogspot.com/2020/04/fukazawas\_44.html} 
\item
  2020年04月15日00時52分の登録:
  \深澤諭史 @fukazawas\そういえば、アソシエイトを「塾生」として採用して、私生活について制限をいろいろ加える法律事務所の話を聞いたことがある。¥\n63期64期の頃
  \url{http://hirono2014sk.blogspot.com/2020/04/fukazawas-6364.html} 
\item
  2020年04月15日09時02分の登録:
  \深澤諭史 @fukazawas\ここ数日の忙しさというか、新件の数が尋常じゃない・・。¥\nこのペースで受任するとパンクしかねないので、色々対策を考えないと・・。
  \url{http://hirono2014sk.blogspot.com/2020/04/fukazawas\_83.html} 
\item
  2020年04月15日13時24分の登録:
  \深澤諭史 @fukazawas\ツイートが本になりました(・∀・)
  \#フォロワーの8割が体験したことなさそうなこと
  \url{http://hirono2014sk.blogspot.com/2020/04/fukazawas-8.html} 
\item
  2020年04月15日13時35分の登録:
  \深澤諭史 @fukazawas\しっかし,コロナ禍のアレコレ見てみると,給与所得者の源泉徴収制度って,国民をチョロい国民にする,政治リテラシーを低下させる,もっといえば
  \url{http://hirono2014sk.blogspot.com/2020/04/fukazawas\_31.html} 
\item
  2020年04月15日16時54分の登録: \深澤諭史 @fukazawas\一足先に
  \#アベノマスク
  をゲットしたんだが,レビューしようかな・・。¥\n正直に客観的にレビューすると,日本を愛する普通の日本人に怒られそ
  \url{http://hirono2014sk.blogspot.com/2020/04/fukazawas\_24.html} 
\item
  2020年04月15日21時28分の登録:
  深澤諭史¥\n@fukazawas¥\n·¥\n8分¥\nほんこれ。¥\nなんだかんだ言っても、弁護士の公共性を痛感する。。。¥\n(;・∀・)
  \url{http://hirono2014sk.blogspot.com/2020/04/fukazawas-8\_15.html} 
\item
  2020年04月15日21時30分の登録:
  \深澤諭史 @fukazawas\ネット上の表現トラブルと依頼者トラブル -
  弁護士 深澤諭史のブログ
  \url{http://hirono2014sk.blogspot.com/2020/04/fukazawas\_76.html} 
\item
  2020年04月15日23時13分の登録:
  \深澤諭史 @fukazawas\これは興味深い。権利濫用は滅多なことでは、極端な請求しない限り認められないので(宇奈月温泉事件レベルとか)。¥\nそのうちブログでレビューす
  \url{http://hirono2014sk.blogspot.com/2020/04/fukazawas\_84.html} 
\item
  2020年04月16日12時42分の登録:
  \深澤諭史 @fukazawas\この文書を起案したり、これに署名することの方が、風俗店でのプレイがバレるより恥ずかしいと思う。¥\n(#・∀・)
  \url{http://hirono2014sk.blogspot.com/2020/04/fukazawas\_16.html} 
\item
  2020年04月16日20時55分の登録:
  \深澤諭史 @fukazawas\(・∀・)ということで、弁護士会での初の委員長(ただし小委員会)です。¥\n(・∀・)外国法事務弁護士資格審査小委員会の委員長を、前回の第二
  \url{http://hirono2014sk.blogspot.com/2020/04/fukazawas\_39.html} 
\item
  2020年04月16日20時58分の登録:
  \深澤諭史 @fukazawas\法クラが指摘する通りになりましたね・・・。
  \url{http://hirono2014sk.blogspot.com/2020/04/fukazawas\_0.html} 
\item
  2020年04月16日21時08分の登録:
  \深澤諭史 @fukazawas\(;・∀・)我々個人事業主は、合計110万円の給付でしのぐことになるのかな。。。
  \url{http://hirono2014sk.blogspot.com/2020/04/fukazawas\_69.html} 
\item
  2020年04月16日21時18分の登録:
  \深澤諭史 @fukazawas\被告人検察官面前調書を証拠請求したら、検察官に不同意にされたので、任意性を争うのか釈明を求めた。
  \#フォロワーの8割が体験したことなさそ
  \url{http://hirono2014sk.blogspot.com/2020/04/fukazawas-8\_16.html} 
\item
  2020年04月16日21時19分の登録:
  \深澤諭史 @fukazawas\市長と一緒にオープンカーでパレードを一時間ほど。
  \#フォロワーの8割が体験したことなさそうなこと
  \url{http://hirono2014sk.blogspot.com/2020/04/fukazawas-8\_50.html} 
\item
  2020年04月17日10時25分の登録:
  \深澤諭史 @fukazawas\ということで,法クラ各位を見習って,どっかに寄付するか・・・。¥\n(・∀・)
  \url{http://hirono2014sk.blogspot.com/2020/04/fukazawas\_50.html} 
\item
  2020年04月17日12時14分の登録:
  \深澤諭史 @fukazawas\『100日後に死ぬワニ』の帯文が波紋・・・「描き下ろし漫画28P」はどこに?「優良誤認表示」との指摘も\textbar 弁護士ドットコムニュース
  \url{http://hirono2014sk.blogspot.com/2020/04/fukazawas10028p.html} 
\item
  2020年04月18日12時31分の登録:
  \深澤諭史 @fukazawas\(#・∀・)「かけがえのない人生を生きる人々」とか「人が大事」とか言いながら、「受験生も新人も発言力ないし、こいつらに負担は全部被せて、
  \url{http://hirono2014sk.blogspot.com/2020/04/fukazawas\_453.html} 
\item
  2020年04月18日19時30分の登録:
  \深澤諭史 @fukazawas\(#・∀・)むしろ、緊急時にリスクを負って急ぎの仕事をするわけだから、税金から填補してもらって、倍額の報酬でやるべき。¥\n(^ω^)診療報
  \url{http://hirono2014sk.blogspot.com/2020/04/fukazawas\_37.html} 
\item
  2020年04月18日20時19分の登録: \自家製パンチェッタ @jikapan\返信先:
  ¥\n@fukazawas¥\nさん¥\n始まりが62くらいとは思いますが、終わりは67くらいまでは厳しかったと感じます。¥\nというか
  \url{http://hirono2014sk.blogspot.com/2020/04/jikapan-fukazawas.html} 
\item
  2020年04月18日20時28分の登録:
  \向原総合法律事務所 弁護士向原 @harrier0516osk\返信先:
  ¥\n@fukazawas¥\nさん¥\n全く同じ観点で見てました。このときに実践していれば、巨万の富が築
  \url{http://hirono2014sk.blogspot.com/2020/04/harrier0516osk-fukazawas\_18.html} 
\item
  2020年04月18日22時23分の登録:
  \深澤諭史 @fukazawas\(・∀・)司法試験は、受験生が「暗記していること」「知っていること」「覚えていること」を逆手にとって全力で、それら受験生を落としにくる試
  \url{http://hirono2014sk.blogspot.com/2020/04/fukazawas\_62.html} 
\item
  2020年04月18日22時44分の登録:
  \深澤諭史 @fukazawas\これは結構深刻な問題。¥\nただ、なぜか、ネットの投稿者は、本人訴訟しちゃう率が多いので、請求がかなりみとめられて、助かっている側面もある。
  \url{http://hirono2014sk.blogspot.com/2020/04/fukazawas\_63.html} 
\item
  2020年04月18日22時50分の登録:
  \深澤諭史 @fukazawas\必要なんですね。他、報道になるレベルで酷い投稿を長期にする、集団を同時に訴える、という方法も。
  \url{http://hirono2014sk.blogspot.com/2020/04/fukazawas\_68.html} 
\item
  2020年04月18日23時14分の登録:
  \ひなた荘の管理人(弁護士) @shinobuhome\返信先:
  ¥\n@fukazawas¥\nさん¥\n彼はどこの世界線の住民なのでしょうか。弁護士って最悪就職せずとも食っていけ
  \url{http://hirono2014sk.blogspot.com/2020/04/shinobuhome-fukazawas.html} 
\item
  2020年04月19日12時12分の登録:
  \深澤諭史 @fukazawas\弁護士業界も空前絶後の不景気がやってきそうだな・・。¥\nまあ、といっても、個人で行える事業の中では屈指の景気に左右されない業種ですが・・・
  \url{http://hirono2014sk.blogspot.com/2020/04/fukazawas\_66.html} 
\item
  2020年04月19日12時39分の登録: \弁護士山下敏雅 :
  子どもの法律ブログ @children\_ymlaw\返信先:
  ¥\n@fukazawas¥\nさん¥\n日弁連に社会科見学に来る子ども達向けボランティアを長年担
  \url{http://hirono2014sk.blogspot.com/2020/04/childrenymlaw-fukazawas.html} 
\item
  2020年04月19日16時51分の登録:
  \深澤諭史 @fukazawas\(;・∀・)音頃さん、改めて読み直してみると、ネタ満載だな。。¥\n某宗教団体までネタにしている。。。¥\n(^ω^)それに気がつくお前も大概だ
  \url{http://hirono2014sk.blogspot.com/2020/04/fukazawas\_498.html} 
\item
  2020年04月19日16時52分の登録:
  \深澤諭史 @fukazawas\いや、まじで、この時期の国選弁護については、危険手当を出すべき。接見一回、また示談のための面談1回について2万円から3万円程度。
  \url{http://hirono2014sk.blogspot.com/2020/04/fukazawas\_307.html} 
\item
  2020年04月19日20時30分の登録:
  \深澤諭史 @fukazawas\本これ。¥\nえきなん先生の言うように大変なことになりつつ。。。」
  \url{http://hirono2014sk.blogspot.com/2020/04/fukazawas\_22.html} 
\item
  2020年04月20日12時34分の登録:
  \深澤諭史 @fukazawas\弁護士費用について誤解を招くような表現をして広告するのは,弁護士業務広告規程違反の問題ががが・・・。¥\n(・∀・;;)
  \url{http://hirono2014sk.blogspot.com/2020/04/fukazawas\_67.html} 
\item
  2020年04月20日21時29分の登録:
  \深澤諭史 @fukazawas\(・∀・)いや、ほんとに、最初の数秒での印象って、裏切られない。
  \url{http://hirono2014sk.blogspot.com/2020/04/fukazawas\_509.html} 
\item
  2020年04月20日21時35分の登録:
  \深澤諭史 @fukazawas\(・∀・)高級酒吹いた。
  \url{http://hirono2014sk.blogspot.com/2020/04/fukazawas\_724.html} 
\item
  2020年04月20日21時38分の登録:
  \深澤諭史 @fukazawas\【笑韓】韓国の新型コロナ対策がマヌケすぎると話題に
  - なんJ政治ネタまとめ \url{http://j-seiji.blog.jp/archiv} 
  \url{http://hirono2014sk.blogspot.com/2020/04/fukazawas-j-httpj-seijiblogjparchiv.html} 
\item
  2020年04月20日22時30分の登録:
  \深澤諭史 @fukazawas\¥\n深澤諭史¥\n10.8万
  ツイート¥\nフォロー¥\n新しいツイートを表示¥\n¥\nフォロー¥\n深澤諭史¥\n@fukazawas¥\n弁護士(第二東京弁護士
  \url{http://hirono2014sk.blogspot.com/2020/04/fukazawas-108-fukazawas.html} 
\item
  2020年04月20日22時31分の登録:
  \深澤諭史 @fukazawas\弁護士でない退職代行業者は、非弁行為の問題もあるけれども、そもそも交渉の相手にされないから、トラブルになったら、自分は知らないって感じで
  \url{http://hirono2014sk.blogspot.com/2020/04/fukazawas\_92.html} 
\item
  2020年04月21日00時56分の登録:
  \深澤諭史 @fukazawas\町弁そのものは確かに潰しはきかない。¥\nただ、企業法務はじめ、ほかの業務するときに、強烈に役に立つ。¥\nと思っています。というか、無茶苦茶実
  \url{http://hirono2014sk.blogspot.com/2020/04/fukazawas\_95.html} 
\item
  2020年04月21日00時58分の登録:
  \深澤諭史 @fukazawas\(;・∀・)国選弁護の現状を連想してしまった。。。。
  \url{http://hirono2014sk.blogspot.com/2020/04/fukazawas\_476.html} 
\item
  2020年04月21日01時01分の登録:
  \向原総合法律事務所 弁護士向原 @harrier0516osk\返信先:
  ¥\n@fukazawas¥\nさん¥\nいきなり名乗らないで話始める人って、どうしてそうなるのだろう?と
  \url{http://hirono2014sk.blogspot.com/2020/04/harrier0516osk-fukazawas\_21.html} 
\item
  2020年04月21日01時03分の登録:
  \深澤諭史 @fukazawas\相手方に代理人がついていないときの弁護士の言動
  - 弁護士 深澤諭史のブログ
  \url{http://hirono2014sk.blogspot.com/2020/04/fukazawas\_137.html} 
\item
  2020年04月21日10時33分の登録:
  \深澤諭史 @fukazawas\1.敵か味方か,客かどうかすらもわからない相手について,瞬時にリスク判定が出来るようになる。¥\n2.身も蓋もない事実や,相談者の嘘に慣れる
  \url{http://hirono2014sk.blogspot.com/2020/04/fukazawas\_381.html} 
\item
  2020年04月21日22時58分の登録:
  %@fukazawas 深澤諭史%(・∀・)Twitterみていると「こいつは,けしからん」「あの界隈はこんな馬鹿だ」みたいなことを,些細なことや影響力のない人の言動でも
  \url{http://hirono2014sk.blogspot.com/2020/04/fukazawas\_964.html} 
\item
  2020年04月21日23時15分の登録:
  \深澤諭史 @fukazawas\あなたも遺留分減殺請求できるかも?相続人間の相続分譲渡についての最高裁判決
  - 弁護士 深澤諭史のブログ http://弁護士
  \url{http://hirono2014sk.blogspot.com/2020/04/fukazawas-http.html} 
\item
  2020年04月21日23時16分の登録:
  \深澤諭史 @fukazawas\「無断転載禁止」とあえて書く意味 - 弁護士
  深澤諭史のブログ http://弁護士.club/archives/10946club/a
  \url{http://hirono2014sk.blogspot.com/2020/04/fukazawas-httpclubarchives10946cluba.html} 
\item
  2020年04月21日23時23分の登録:
  \深澤諭史 @fukazawas\弁護士の自殺率って、平均より結構高いんですよね。¥\n精神衛生には、皆さん、気をつけましょう。¥\n(;・∀・)
  \url{http://hirono2014sk.blogspot.com/2020/04/fukazawas\_97.html} 
\item
  2020年04月23日17時33分の登録:
  \深澤諭史 @fukazawas\やたら挑発的な文言を並べる代理人に遭遇することがあるけれども,かなりの高確率で,ネットで派手に「○○に強い弁護士」って広告している。¥\nお
  \url{http://hirono2014sk.blogspot.com/2020/04/fukazawas\_593.html} 
\item
  2020年04月24日23時06分の登録:
  \深澤諭史 @fukazawas\弁護士に依頼すると喧嘩になるのか? - 弁護士
  深澤諭史のブログ http://弁護士.club/archives/13197club/
  \url{http://hirono2014sk.blogspot.com/2020/04/fukazawas-httpclubarchives13197club.html} 
\item
  2020年04月24日23時23分の登録:
  \向原総合法律事務所 弁護士向原 @harrier0516osk\返信先:
  ¥\n@fukazawas¥\nさん¥\n医テラスができると、病院が破産しまくる未来¥\nまあ医師会は士業団体
  \url{http://hirono2014sk.blogspot.com/2020/04/harrier0516osk-fukazawas\_24.html} 
\item
  2020年04月24日23時42分の登録:
  \深澤諭史 @fukazawas\非弁も犯罪ですが,非弁提携も犯罪です。¥\nということで,非弁提携している弁護士が,その非弁提携の業務の宣伝にTwitterアカウントを使う
  \url{http://hirono2014sk.blogspot.com/2020/04/fukazawas\_662.html} 
\item
  2020年04月24日23時55分の登録:
  \深澤諭史 @fukazawas\朝から余りに酷い非弁・非弁提携を発見した。¥\n証拠は保全したので,所属会に情報提供しよう。¥\nなぜ,非弁と非弁提携が禁じられているのか,その
  \url{http://hirono2014sk.blogspot.com/2020/04/fukazawas\_639.html} 
\item
  2020年04月25日22時21分の登録:
  \深澤諭史 @fukazawas\勤務先が倒産した方々へ、そして経営している会社が倒産した方々へ
  - 弁護士 深澤諭史のブログ http://
  \url{http://hirono2014sk.blogspot.com/2020/04/fukazawas-http\_25.html} 
\item
  2020年04月26日14時27分の登録:
  \深澤諭史 @fukazawas\不適法な主張って、よっぽどですね。¥\n少なくとも、普通にやばい弁護士でもしないですねぇ・・。¥\n(・∀・;)
  \url{http://hirono2014sk.blogspot.com/2020/04/fukazawas\_423.html} 
\item
  2020年04月26日14時41分の登録:
  \深澤諭史 @fukazawas\(・∀・)お詫びとお知らせ。¥\n(^ω^)皆様に大喜利をしていただきましたが、現実の前には完敗でしたので、お詫び致しますお。¥\n(・∀・)ま
  \url{http://hirono2014sk.blogspot.com/2020/04/fukazawas\_287.html} 
\item
  2020年04月26日20時09分の登録:
  \向原総合法律事務所 弁護士向原 @harrier0516osk\返信先:
  ¥\n@harrier0516osk¥\nさん,
  ¥\n@fukazawas¥\nさん¥\n相手方代理人が、当方依
  \url{http://hirono2014sk.blogspot.com/2020/04/harrier0516osk-harrier0516osk-fukazawas.html} 
\item
  2020年04月26日20時14分の登録:
  \向原総合法律事務所 弁護士向原 @harrier0516osk\返信先:
  ¥\n@fukazawas¥\nさん¥\nこの場合の「話し合い」とは「私の言うことを聞け」と同趣旨です
  \url{http://hirono2014sk.blogspot.com/2020/04/harrier0516osk-fukazawas\_26.html} 
\item
  2020年04月26日20時27分の登録:
  \深澤諭史 @fukazawas\モラ夫・DV・ストーカーあるあるですね。パワハラ系の人もあるある。¥\n真意を確かめたいとか、話し合いだの、フェイストーフェイスだのが大好き
  \url{http://hirono2014sk.blogspot.com/2020/04/fukazawasdv.html} 
\item
  2020年04月27日07時31分の登録:
  \深澤諭史 @fukazawas\(#^ω^)億単位の税金かけて、ゴムの長さの違うマスクを配って、挙げ句の果てに使い道は秘密だの、書類は破棄しだの、酷すぎるお!!
  \url{http://hirono2014sk.blogspot.com/2020/04/fukazawas\_430.html} 
\item
  2020年04月27日15時14分の登録:
  \深澤諭史 @fukazawas\「○○な弁護士には依頼したくない!」ってわざわざSNSで発信している人からは,相談や受任を避けるべき典型的な雰囲気をありありと感じてしま
  \url{http://hirono2014sk.blogspot.com/2020/04/fukazawas\_536.html} 
\item
  2020年04月27日23時00分の登録:
  \深澤諭史 @fukazawas\意外なところに、機会はあったりしますよー。¥\n最近だと、ツイートがきっかけになったり(笑)¥\n引用ツイート¥\n
  \url{http://hirono2014sk.blogspot.com/2020/04/fukazawas\_953.html} 
\item
  2020年04月27日23時06分の登録:
  \深澤諭史 @fukazawas\クレーマーに費やすリソースは顧客に - 弁護士
  深澤諭史のブログ
  \url{http://hirono2014sk.blogspot.com/2020/04/fukazawas\_82.html} 
\item
  2020年04月28日13時35分の登録:
  \深澤諭史 @fukazawas\お、虚構新聞の新しい記事か。何かドメイン違うけど。¥\n(・∀・)¥\n引用ツイート¥\n
  \url{http://hirono2014sk.blogspot.com/2020/04/fukazawas\_680.html} 
\item
  2020年04月29日07時57分の登録:
  \深澤諭史 @fukazawas\Zoomで出演していて思ったが、なんかノイズとか、音の途切れが聞こえると、自分が原因ではないか、と、途端に不安になるというのはあるな。¥\n
  \url{http://hirono2014sk.blogspot.com/2020/04/fukazawaszoom.html} 
\item
  2020年04月29日23時18分の登録:
  \深澤諭史 @fukazawas\(・∀・)ほんこれ(・∀・)ほんこれ(・∀・)ほんこれ(・∀・)ほんこれ(・∀・)ほんこれ(・∀・)ほんこれ(・∀・)ほんこれ(・∀・)
  \url{http://hirono2014sk.blogspot.com/2020/04/fukazawas\_961.html} 
\item
  2020年04月29日23時19分の登録:
  \深澤諭史 @fukazawas\勤務先が倒産した方々へ、そして経営している会社が倒産した方々へ
  - 弁護士 深澤諭史のブログ http://弁護士.club
  \url{http://hirono2014sk.blogspot.com/2020/04/fukazawas-httpclub.html} 
\item
  2020年04月30日11時08分の登録:
  \深澤諭史 @fukazawas\弁護士というか法曹をやっていると,識字率と識「文」率は,別物だって思いますよね。仕事ではもちろん,SNSでも,飛んでくるリプライとか読む
  \url{http://hirono2014sk.blogspot.com/2020/04/fukazawas\_376.html} 
\item
  2020年04月30日12時11分の登録:
  \深澤諭史 @fukazawas\今月の売り上げ,セミナーとか講演は吹っ飛ぶが,印税があるので,なんとか生きていけそう・・・。¥\n(・∀・;)
  \url{http://hirono2014sk.blogspot.com/2020/04/fukazawas\_744.html} 
\end{itemize}

 時刻は9時15分です。ご飯を食べて洗濯をしています。外は快晴のようです。昨日の午後は天気が下り坂のように思えたのですが,天気予報は見ていませんでした。

 やはり渡邉恭子弁護士のTwitter非公開設定は,オリンピック中止に関する過激なツイートにあったようですが,一緒に渡辺輝人弁護士に疑問を投げかけるツイートもありました。白血病を克服した池江璃花子選手を標的にしたという共通点もあったようです。

 渡辺輝人弁護士は,白血病を克服した池江璃花子選手を電通と結束させて批判しているようですが,電通といえば数年前の女性社員の自殺の問題があり,過労とかパワハラ,労災絡みで弁護士らが総攻撃と宣伝を繰り広げていたような大手企業でした。

 その数日前に過労自殺とかの第一人者あるはパイオニアのような存在感のある川人博弁護士のインタビュー記事を読んでいたのですが,それも弁護士の肖像というシリーズでした。ずいぶん前からあったようですが,郷原信郎弁護士の検索で出てきたのが始まりのようになりました。

\begin{itemize}
\tightlist
\item
  〈〈〈 2021/05/11 09:29:17 Linux Emacs: 〈〈〈
\end{itemize}

\hypertarget{ux79c1ux305fux3061ux306fux30d9ux30c6ux30e9ux30f3ux306eux7537ux6027ux5f01ux8b77ux58ebux3042ux308bux3044ux306fux88c1ux5224ux5b98ux305fux3061ux306bux304aux4e16ux8a71ux306bux306aux308bux3068ux540cux6642ux306bux6027ux7684ux306bux643eux53d6ux3055ux308cux308bux3068ux3044ux3046ux4f50ux85e4ux502bux5b50ux5f01ux8b77ux58ebux306eux30c4ux30a4ux30fcux30c8ux3068ux4e09ux5b85ux4fcaux4e00ux90ceux88c1ux5224ux95774}{%
\paragraph{「私たちはベテランの男性弁護士(あるいは裁判官)たちにお世話になると同時に性的に搾取される」という佐藤倫子弁護士のツイートと,三宅俊一郎裁判長(4)}\label{ux79c1ux305fux3061ux306fux30d9ux30c6ux30e9ux30f3ux306eux7537ux6027ux5f01ux8b77ux58ebux3042ux308bux3044ux306fux88c1ux5224ux5b98ux305fux3061ux306bux304aux4e16ux8a71ux306bux306aux308bux3068ux540cux6642ux306bux6027ux7684ux306bux643eux53d6ux3055ux308cux308bux3068ux3044ux3046ux4f50ux85e4ux502bux5b50ux5f01ux8b77ux58ebux306eux30c4ux30a4ux30fcux30c8ux3068ux4e09ux5b85ux4fcaux4e00ux90ceux88c1ux5224ux95774}}

\begin{itemize}
\tightlist
\item
  〉〉〉 Linux Emacs: 2021/05/11 09:35:23 〉〉〉
\end{itemize}

:CATEGORIES: @kanazawabengosi \#金沢弁護士会 @JFBAsns
日本弁護士連合会(日弁連) \#法務省 @MOJ\_HOUMU \#三宅俊一郎裁判長
\#川口泰司裁判官 \#山田徹裁判官 \#ジャーナリストの江川紹子氏

\begin{itemize}
\tightlist
\item
  1356:2021-05-11\_05:03:57 \#告発状 \#\#\#\#
  「私たちはベテランの男性弁護士(あるいは裁判官)たちにお世話になると同時に性的に搾取される」という佐藤倫子弁護士のツイートと,三宅俊一郎裁判長(1)
  \url{https://hirono-hideki.hatenadiary.jp/entry/2021/05/11/050354} 
\item
  1357:2021-05-11\_06:13:01 \#告発状 \#\#\#\#
  「私たちはベテランの男性弁護士(あるいは裁判官)たちにお世話になると同時に性的に搾取される」という佐藤倫子弁護士のツイートと,三宅俊一郎裁判長(2)
  \url{https://hirono-hideki.hatenadiary.jp/entry/2021/05/11/061258} 
\item
  1358:2021-05-11\_09:33:45 \#告発状 \#\#\#\#
  「私たちはベテランの男性弁護士(あるいは裁判官)たちにお世話になると同時に性的に搾取される」という佐藤倫子弁護士のツイートと,三宅俊一郎裁判長(3)
  \url{https://hirono-hideki.hatenadiary.jp/entry/2021/05/11/093340} 
\end{itemize}

 上記3件のエントリーの続きになります。

\begin{itemize}
\tightlist
\item
  2021年05月11日06時29分の登録:
  REGEXP:''@wata\_nabekyo\_ko''/データベース登録済みツイートの検索:2021-05-04〜2021-05-11/2021年05月11日06時28分の記録:ユーザ・投稿:16/27件
  \url{https://kk2020-09.blogspot.com/2021/05/regexpwatanabekyoko2021-05-042021-05.html} 
\item
  2021年05月11日09時04分の登録:
  @nabeteru1Q78(渡辺輝人)のツイート ''.*'' 3244/3244:2021-02-11\_1511〜2021-05-11\_0720 2021年05月11日09時04分の記録
  \url{https://kk2020-09.blogspot.com/2021/05/nabeteru1q78324432442021-02-1115112021.html} 
\item
  2021年05月11日09時12分の登録:
  \渡辺輝人 @nabeteru1Q78\こうやって緊急事態宣言と来日を絡められるのも良くない。バッハのために解除したがる菅首相やさざ波の人がいるからである。引用ツイート
  \url{https://kk2020-09.blogspot.com/2021/05/nabeteru1q78\_11.html} 
\item
  2021年05月11日09時14分の登録:
  REGEXP:''渡辺輝人''/データベース登録済みツイートの検索:2021-05-03〜2021-05-11/2021年05月11日09時14分の記録:ユーザ・投稿:9/26件
  \url{https://kk2020-09.blogspot.com/2021/05/regexp2021-05-032021-05\_11.html} 
\item
  2021年05月11日09時18分の登録:
  REGEXP:''#Tokyoインパール2020.*''/データベース登録済みツイートの検索:2018-08-14〜2021-05-11/2021年05月11日09時18分の記録:ユーザ・投稿:14/22件
  \url{https://kk2020-09.blogspot.com/2021/05/regexptokyo20202018-08-142021-05.html} 
\item
  2021年05月11日09時34分の登録:
  REGEXP:''江川紹子''/データベース登録済みツイートの検索:2021-05-08〜2021-05-11/2021年05月11日09時34分の記録:ユーザ・投稿:4/7件
  \url{https://kk2020-09.blogspot.com/2021/05/regexp2021-05-082021-05-1120210511093447.html} 
\end{itemize}

 昨夜,ずいぶん刺激的なジャーナリストの江川紹子氏のツイートを見かけていたのですが,江川紹子をキーワードにtwitterAPI-search-lawList-mydql-add.rb
でまとめ記事を作成してみたところ,トータルでも少ない反応らしいと確認しました。

 江川紹子のキーワードはTwitterAPIの検索でトータル691件でしたが,データベース登録済みツイートの検索:2021-05-08〜2021-05-11/2021年05月11日09時34分の記録:ユーザ・投稿:4/7件,の記事に注目のツイートと関連したものは見当たりませんでした。

\begin{quote}
《引用の始まり》
\end{quote}

\begin{quote}
アカウント名 ツイート数 リツイート数Shoko Egawa(amneris84) 1
0MakotoAkishige(civilista)(akishigemakoto) 0 1芳賀淳(jjjhaga) 0
1弁護士神原元(kambara7) 0 1泥濘ノ魔王獣マガサイケ(k\_sawmen) 0
1Smith目(iy0kahn) 0
1奉納\さらば弁護士鉄道・泥棒神社の物語(hirono\_hideki) 0 1
\end{quote}

\begin{quote}
《引用の終わり》
\end{quote}

\begin{itemize}
\item
  奉納\危険生物・弁護士脳汚染除去装置\金沢地方検察庁御中\_2020:
  REGEXP:''と子々孫々に伝えていくべきでは?→豊田真由子氏''/データベース登録済みツイートの検索:2021-05-09〜2021-05-10/2021年05月11日09時49分の記録:ユーザ・投稿:7/7件 \url{https://kk2020-09.blogspot.com/2021/05/regexp2021-05-092021-05-1020210511094977.htmln} 
\item
  TW amneris84(Shoko Egawa) 日時: 2021/05/09 22:38:30 URL:
  \url{https://twitter.com/amneris84/status/1391387103930445835} 
  \textgreater{}
  むしろ、未来永劫日本は五輪に名乗りを挙げてはいけない、ろくなことにはならない、と子々孫々に伝えていくべきでは?→豊田真由子氏 東京五輪中止なら「未来永劫日本にオリンピックは来ない」
  \url{https://t.co/rVSHfH5aWS} 
\end{itemize}

 冤罪や再審請求,検察批判でも似たような強い調子の発言をしていたジャーナリストの江川紹子氏ですが,これは大きな疑問符とともに気分を害する人も多そうだと思いました。

 このツイートを探すのにジャーナリストの江川紹子氏のタイムラインを遡っていたところ,合理的なのかと思えるジャーナリストの江川紹子氏の安倍元首相批判のツイートがありました。

\begin{itemize}
\tightlist
\item
  TW amneris84(Shoko Egawa) 日時: 2021/05/10 21:32:51 URL:
  \url{https://twitter.com/amneris84/status/1391732970042445832} 
  \textgreater{}
  オリンピックを巡る今の問題は、安倍政権がコロナ禍の深刻さを過小評価し、2年延期ではなく1年延期でできる、それも「完全な形で」出来ると見誤ったのが最大の要因でしょう。2年延期していれば、こんな分断にはなっていなかった。その責任は、例によって誰もとらない。
\end{itemize}

 何かの答えが出たようにも思ったのですが,ジャーナリストの江川紹子氏を満足させられる首相というのは,どういう人物なのか? それに見合った過大評価をしていれば経済的な打撃は大きくなっていたと思いますし,不満も爆発し暴動で無政府状態になっていたという最悪のシナリオも浮かびます。

 再審法改正とか呼びかけながら,再審の問題をまともにとりあげる記事やツイートも最近は見かけないのですが,さきほどは次のような,ジャーナリストの江川紹子氏自身に向けられた指摘への返信ツイートも見かけていました。批判ではないかもしれないので指摘としておきます。

〉〉〉 kk\_hironoのリツイート 〉〉〉

\begin{itemize}
\tightlist
\item
  RT
  kk\_hirono(刑事告発・非常上告_金沢地方検察庁御中)|tk\_andycarlos(アンディ)
  日時:2021-05-11 10:09/2021/05/10 19:50 URL:
  \url{https://twitter.com/kk\_hirono/status/1391923447828156416} 
  \url{https://twitter.com/tk\_andycarlos/status/1391707292769390594} 
  \textgreater{}
  私は江川さんのこと結構好きですけど、どうせなら選手たちに怒りの矛先を向ける人々に物申していただきたかった。選手たちに怒りの矛先が向かうのは異常だ。選手たちは守らないといけない。
  \url{https://t.co/e8K5lLHcne} 
\end{itemize}

〉〉〉 kk\_hironoのリツイート 〉〉〉

\begin{itemize}
\item
  RT
  kk\_hirono(刑事告発・非常上告_金沢地方検察庁御中)|amneris84(Shoko
  Egawa) 日時:2021-05-11 10:09/2021/05/10 20:51 URL:
  \url{https://twitter.com/kk\_hirono/status/1391923504623194114} 
  \url{https://twitter.com/amneris84/status/1391722546945425408} 
  \textgreater{} @tk\_andycarlos
  他人が、ある種の発言をしていないことにクレームをつけるのは、せめて一両日のツイートくらい確認してからにすべきではありませんか。
\item
  2021年05月11日10時11分の登録:
  @amneris84(Shoko Egawa)のツイート ''冤罪'' 8/3248:2020-10-10\_2301〜2021-05-11\_0739 2021年05月11日10時11分の記録
  \url{https://kk2020-09.blogspot.com/2021/05/amneris84shokoegawa832482020-10.html} 
\item
  2021年05月11日10時11分の登録:
  @amneris84(Shoko Egawa)のツイート ''再審'' 12/3248:2020-10-10\_2301〜2021-05-11\_0739 2021年05月11日10時11分の記録
  \url{https://kk2020-09.blogspot.com/2021/05/amneris84shokoegawa1232482020-10.html} 
\item
  2021年05月11日10時12分の登録:
  @amneris84(Shoko Egawa)のツイート ''検察'' 22/3248:2020-10-10\_2301〜2021-05-11\_0739 2021年05月11日10時12分の記録
  \url{https://kk2020-09.blogspot.com/2021/05/amneris84shokoegawa2232482020-10.html} 
\end{itemize}

 ku3コマンドに絞り込み検索の第二引数をつけて実行したものです。昨年の10月10日から本日の5月11日までという範囲で3248件のツイートが検索絞り込みの対象となっています。

 冤罪が8,再審が12,検察が22となっています。これなら軽く目を通すこともできそうです。

\begin{itemize}
\item
  (3228/3248) @amneris84(Shoko
  Egawa)のツイート ''冤罪'' 8/3248:2020-10-10\_2301〜2021-05-11\_0739
  2021年05月11日10時11分の記録\\
  TW amneris84(Shoko Egawa) 日時: 2020-10-12 17:30 URL:
  \url{https://twitter.com/amneris84/status/1315570492590309376} 
  \textgreater{}
  特に、最後の「刑事控訴審とは何のための裁判なのか」以下が重要。日本で冤罪が生まれ、雪冤が難しい最大の問題は裁判所だと思います。人質司法がなくならないのも、裁判所がそれを認めているから。せっかく丁寧な証人尋問で一審無罪になっても、安易にひっくり返す高裁があるから、検察は控訴する
  \url{https://t.co/LDNfr6VmcB} 
\item
  (3055/3248) @amneris84(Shoko
  Egawa)のツイート ''冤罪'' 8/3248:2020-10-10\_2301〜2021-05-11\_0739
  2021年05月11日10時11分の記録\\
  TW amneris84(Shoko Egawa) 日時: 2020-10-29 16:33 URL:
  \url{https://twitter.com/amneris84/status/1321716724262072320} 
  \textgreater{}
  長い時間を奪われ、ようやく名誉回復し、国の責任を追及する手続きに取りかかったところだった。冤罪とはかくも残酷。どうかやすらかに。合掌 →松橋事件で再審無罪の宮田浩喜さん、肺炎で死去 87歳:朝日新聞デジタル
  \url{https://t.co/MBJuLJIbPK} 
\item
  (2620/3248) @amneris84(Shoko
  Egawa)のツイート ''冤罪'' 8/3248:2020-10-10\_2301〜2021-05-11\_0739
  2021年05月11日10時11分の記録\\
  TW amneris84(Shoko Egawa) 日時: 2020-12-05 18:07 URL:
  \url{https://twitter.com/amneris84/status/1335148813946814464} 
  \textgreater{}
  23歳から57歳まで身柄拘束され、30年間以上も死刑確定囚として執行の恐怖にさらされた。雪冤後も、冤罪のために無年金の状態に置かれた問題を訴え続けるなど、冤罪被害と戦う生涯だった。合掌→<速報>免田栄さんが死去 死刑囚として国内初の再審無罪(熊本日日新聞)\\
  \textgreater{} \url{https://t.co/raTRAQRSnI} 
\item
  (2306/3248) @amneris84(Shoko
  Egawa)のツイート ''冤罪'' 8/3248:2020-10-10\_2301〜2021-05-11\_0739
  2021年05月11日10時11分の記録\\
  RT amneris84(Shoko Egawa)|MichikoKameishi(弁護士 亀石倫子)
  日時:2020-12-29 22:16/2020-12-29 17:43 URL:
  \url{https://twitter.com/amneris84/status/1343908741666598913} 
  \url{https://twitter.com/MichikoKameishi/status/1343840045451497474} 
  \textgreater{}
  「警察捜査における取調べ適正化指針」では1日8時間を超える取調べは原則禁止。長時間の取調べで自白を強要し、冤罪を生んだことへの反省から策定された。しかし鹿児島県警は8時間超の取調べを3年間に338件実施。結局、まったく反省していないのだ。\url{https://t.co/yRxoEKAXQR} 
  \url{https://t.co/3nKXtLCHwF} 
\item
  (540/3248) @amneris84(Shoko
  Egawa)のツイート ''冤罪'' 8/3248:2020-10-10\_2301〜2021-05-11\_0739
  2021年05月11日10時11分の記録\\
  TW amneris84(Shoko Egawa) 日時: 2021-04-06 15:54 URL:
  \url{https://twitter.com/amneris84/status/1379326605684662272} 
  \textgreater{}
  (私の視点)冤罪救済の壁 再審事件を放置する裁判所 木谷明:朝日新聞デジタル
  \url{https://t.co/vIfUD5Vs9r} 
\item
  (51/3248) @amneris84(Shoko
  Egawa)のツイート ''冤罪'' 8/3248:2020-10-10\_2301〜2021-05-11\_0739
  2021年05月11日10時11分の記録\\
  TW amneris84(Shoko Egawa) 日時: 2021-05-08 22:04 URL:
  \url{https://twitter.com/amneris84/status/1391016171109974017} 
  \textgreater{}
  半年間の別件による身柄拘束は、後に妥当性が議論にならないか。冤罪懸念だけでなく、出来るだけ早く犯人が最終的に確定するよう考えるのであれば、1つひとつのプロセスは、しっかり吟味することが大事。
\end{itemize}

 再審請求や冤罪で,裁判所を厳しく批判してきたのもジャーナリストの江川紹子氏の特徴です。

 勘違いしていたようですが,上記の引用は「再審」ではなく「冤罪」でした。

\begin{itemize}
\item
  (1384/3248) @amneris84(Shoko
  Egawa)のツイート ''再審'' 12/3248:2020-10-10\_2301〜2021-05-11\_0739
  2021年05月11日10時11分の記録\\
  TW amneris84(Shoko Egawa) 日時: 2021-02-13 08:39 URL:
  \url{https://twitter.com/amneris84/status/1360372967708889088} 
  \textgreater{}
  いったん思い込むと、科学的な証拠で無罪が確定した人も、今なお犯人に見えてしまうのか・・・。だからこそ、被疑者をクロとした捜査結果を別の人がシロにする捜査と、無罪推定の原則は、本当に大事だ→再審無罪の青木さんを「犯人と思う」 法廷で元取調官:朝日新聞デジタル
  \url{https://t.co/lp2pSmKrQQ} 
\item
  (1049/3248) @amneris84(Shoko
  Egawa)のツイート ''再審'' 12/3248:2020-10-10\_2301〜2021-05-11\_0739
  2021年05月11日10時11分の記録\\
  TW amneris84(Shoko Egawa) 日時: 2021-03-03 21:47 URL:
  \url{https://twitter.com/amneris84/status/1367094265139007493} 
  \textgreater{}
  「再審弁護人」鴨志田弁護士、京都に拠点移す「再審法改正のムーブメント起こしたい」(京都新聞)\\
  \textgreater{} \#Yahooニュース\\
  \textgreater{} \url{https://t.co/9oUDuJc5AA} 
\item
  (540/3248) @amneris84(Shoko
  Egawa)のツイート ''再審'' 12/3248:2020-10-10\_2301〜2021-05-11\_0739
  2021年05月11日10時11分の記録\\
  TW amneris84(Shoko Egawa) 日時: 2021-04-06 15:54 URL:
  \url{https://twitter.com/amneris84/status/1379326605684662272} 
  \textgreater{}
  (私の視点)冤罪救済の壁 再審事件を放置する裁判所 木谷明:朝日新聞デジタル
  \url{https://t.co/vIfUD5Vs9r} 
\end{itemize}

 よくみると丁度7ヶ月ほどの観測範囲ですが,やはり少ないと感じました。意外に検察が多めで22件となっています。

\begin{itemize}
\tightlist
\item
  (3091/3248) @amneris84(Shoko
  Egawa)のツイート ''検察'' 22/3248:2020-10-10\_2301〜2021-05-11\_0739
  2021年05月11日10時12分の記録\\
  RT amneris84(Shoko Egawa)|NOSUKE0607(清水 潔) 日時:2020-10-26
  17:06/2020-10-26 12:17 URL:
  \url{https://twitter.com/amneris84/status/1320638012590489600} 
  \url{https://twitter.com/NOSUKE0607/status/1320565268251197440} 
  \textgreater{}
  事件から21年。被害者の名誉が少しでも回復するように祈ってツイートさせて頂きました。正義の実現に関わる警察官、検察官、裁判官の方にもぜひ知って頂きたいのです。\\
  \textgreater{}
  「助けてください」と警察に救いを求めたにもかかわらず、21歳でその命を奪われた女性がいました。どうか忘れないでください。
\end{itemize}

 TwitterAPIで取得したテキストでは分かりづらかったのですが,ジャーナリストの江川紹子氏がリツイートをした清水潔氏のツイートでした。桶川ストーカー殺人事件です。これは意外な発見です。

\begin{itemize}
\tightlist
\item
  〈〈〈 2021/05/11 10:47:23 Linux Emacs: 〈〈〈
\end{itemize}

\hypertarget{ux79c1ux305fux3061ux306fux30d9ux30c6ux30e9ux30f3ux306eux7537ux6027ux5f01ux8b77ux58ebux3042ux308bux3044ux306fux88c1ux5224ux5b98ux305fux3061ux306bux304aux4e16ux8a71ux306bux306aux308bux3068ux540cux6642ux306bux6027ux7684ux306bux643eux53d6ux3055ux308cux308bux3068ux3044ux3046ux4f50ux85e4ux502bux5b50ux5f01ux8b77ux58ebux306eux30c4ux30a4ux30fcux30c8ux3068ux4e09ux5b85ux4fcaux4e00ux90ceux88c1ux5224ux95775}{%
\paragraph{「私たちはベテランの男性弁護士(あるいは裁判官)たちにお世話になると同時に性的に搾取される」という佐藤倫子弁護士のツイートと,三宅俊一郎裁判長(5)}\label{ux79c1ux305fux3061ux306fux30d9ux30c6ux30e9ux30f3ux306eux7537ux6027ux5f01ux8b77ux58ebux3042ux308bux3044ux306fux88c1ux5224ux5b98ux305fux3061ux306bux304aux4e16ux8a71ux306bux306aux308bux3068ux540cux6642ux306bux6027ux7684ux306bux643eux53d6ux3055ux308cux308bux3068ux3044ux3046ux4f50ux85e4ux502bux5b50ux5f01ux8b77ux58ebux306eux30c4ux30a4ux30fcux30c8ux3068ux4e09ux5b85ux4fcaux4e00ux90ceux88c1ux5224ux95775}}

\begin{itemize}
\tightlist
\item
  〉〉〉 Linux Emacs: 2021/05/12 16:35:03 〉〉〉
\end{itemize}

:CATEGORIES: @kanazawabengosi \#金沢弁護士会 @JFBAsns
日本弁護士連合会(日弁連) \#法務省 @MOJ\_HOUMU \#三宅俊一郎裁判長

 それではこのエントリーのメインテーマに設定した佐藤倫子弁護士のツイートをリツイートしたいと思います。たぶん最初に見かけ,最も印象的だったのが見出しにある「もう20年近く前のことだけど、私たちはベテランの男性弁護士(あるいは裁判官)たちに」から始まるツイートですが,その2つ上から。

〉〉〉 kk\_hironoのリツイート 〉〉〉

\begin{itemize}
\tightlist
\item
  RT
  kk\_hirono(刑事告発・非常上告_金沢地方検察庁御中)|sato\_\_michiko(§
  佐藤倫子) 日時:2021-05-12 16:39/2021/05/08 21:37 URL:
  \url{https://twitter.com/kk\_hirono/status/1392383822503710726} 
  \url{https://twitter.com/sato\_\_michiko/status/1391009334201516032} 
  \textgreater{}
  新人の頃、修習中からお世話になっていた先生と二人で食事にいったらペースに飲み込まれて想定していなかった不本意な関係に持ち込まれたこと、当時は不本意とはいえ弁護士同士だから対等だし自分の選択である以上仕方ないと思っていたけど、完全に自分の未熟さに付け入られてのだと、今となっては思う
\end{itemize}

〉〉〉 kk\_hironoのリツイート 〉〉〉

\begin{itemize}
\tightlist
\item
  RT
  kk\_hirono(刑事告発・非常上告_金沢地方検察庁御中)|sato\_\_michiko(§
  佐藤倫子) 日時:2021-05-12 16:39/2021/05/08 21:52 URL:
  \url{https://twitter.com/kk\_hirono/status/1392383835292127232} 
  \url{https://twitter.com/sato\_\_michiko/status/1391013078200815622} 
  \textgreater{}
  修習中に行った旅行で某裁判官に浴衣に袖のなかに手を入れられて腕をさすられたのも本当に気持ち悪かった。修習生のブラジャーを服の上から外すのが得意という弁護士さんもいた。私はされなかったけど。とにかく今思えばかなりセクハラは横行していたし、私たちは感覚を麻痺させて生き延びてきた
\end{itemize}

〉〉〉 kk\_hironoのリツイート 〉〉〉

\begin{itemize}
\tightlist
\item
  RT
  kk\_hirono(刑事告発・非常上告_金沢地方検察庁御中)|sato\_\_michiko(§
  佐藤倫子) 日時:2021-05-12 16:39/2021/05/08 22:03 URL:
  \url{https://twitter.com/kk\_hirono/status/1392383848562917383} 
  \url{https://twitter.com/sato\_\_michiko/status/1391015806226423811} 
  \textgreater{}
  もう20年近く前のことだけど、私たちはベテランの男性弁護士(あるいは裁判官)たちにお世話になると同時に性的に搾取されるというか軽んじられてきたと今となっては分かる。悔しくてヒリヒリする。今の若い人たちにはそんな目に遭って欲しくないと心から思う
\end{itemize}

〉〉〉 kk\_hironoのリツイート 〉〉〉

\begin{itemize}
\tightlist
\item
  RT
  kk\_hirono(刑事告発・非常上告_金沢地方検察庁御中)|sato\_\_michiko(§
  佐藤倫子) 日時:2021-05-12 16:39/2021/05/08 22:18 URL:
  \url{https://twitter.com/kk\_hirono/status/1392383862165016578} 
  \url{https://twitter.com/sato\_\_michiko/status/1391019721789640707} 
  \textgreater{}
  まあまだあるわけなんだけど、今までこういう話を20年近くしてこなかったのは多分自分に落ち度がある、自己責任だ、被害を認めることは自分の弱さを認めることであり弁護士としての能力や資質についてマイナスイメージだという感覚があったのだと思う。今はもう、そんなことは関係ない。怖いものもない
\end{itemize}

〉〉〉 kk\_hironoのリツイート 〉〉〉

\begin{itemize}
\tightlist
\item
  RT
  kk\_hirono(刑事告発・非常上告_金沢地方検察庁御中)|sato\_\_michiko(§
  佐藤倫子) 日時:2021-05-12 16:39/2021/05/08 22:31 URL:
  \url{https://twitter.com/kk\_hirono/status/1392383875125452800} 
  \url{https://twitter.com/sato\_\_michiko/status/1391023035486916613} 
  \textgreater{}
  10代の頃、大学の廊下で教授から振り向きざまに胸を掴まれて「俺が初めてだろ」っていわれたのは本当にショックで流石にそのときは「そんなことだからセクハラオヤジって言われるんですよ!」って大声あげたんだけど、その教授にはすごくお世話にもなって、若い私はどう整理していいのか分からなかった
\end{itemize}

〉〉〉 kk\_hironoのリツイート 〉〉〉

\begin{itemize}
\tightlist
\item
  RT
  kk\_hirono(刑事告発・非常上告_金沢地方検察庁御中)|sato\_\_michiko(§
  佐藤倫子) 日時:2021-05-12 16:39/2021/05/09 07:39 URL:
  \url{https://twitter.com/kk\_hirono/status/1392383903290118146} 
  \url{https://twitter.com/sato\_\_michiko/status/1391160806092382209} 
  \textgreater{}
  身体的な接触以外にも以前ツイートしたような差別的な取り扱いとかもあって、結局大学だろうが司法試験に受かろうがどこにいってもセクハラはあった。女性たちにとっては当たり前すぎて言うまでもないことと思ってたけど、言わないでないことになっても嫌なので、書いてみました
  \url{https://t.co/dFVikzJCHE} 
\end{itemize}

〉〉〉 kk\_hironoのリツイート 〉〉〉

\begin{itemize}
\tightlist
\item
  RT
  kk\_hirono(刑事告発・非常上告_金沢地方検察庁御中)|sato\_\_michiko(§
  佐藤倫子) 日時:2021-05-12 16:39/2021/05/09 08:30 URL:
  \url{https://twitter.com/kk\_hirono/status/1392383952694878211} 
  \url{https://twitter.com/sato\_\_michiko/status/1391173604176322565} 
  \textgreater{}
  ちなみにですが、私は割とはっきり物をいいそうに見える(実際に言えるかは別)タイプだったので「佐藤は危険だから佐藤にはセクハラしないでおこう」という判断をした先生は結構いたのではないかと思います。大人しそうに見えたりして私より酷い目に遭っていた友人たちは多かったと思います
\end{itemize}

 7件のリツイートでしたが,「もう20年近く前のことだけど、私たちはベテランの男性弁護士(あるいは裁判官)たちに」の上に3件,下に4件のツイートがいずれも佐藤倫子弁護士のツイートとして線で結ばれ並んでいます。

 自分のツイートに対する返信ツイートというのは私は余りやらないのでよくわかっていないですが,自分のアカウントのツイートで返信ボタンを開くと,「別のツイートを追加する」という表示があり,選択したツイートと線で結ばれていました。

 私は通常のTwitterのページで投稿することも少ないのですが,Linuxの場合,日本語変換のタイミングがおかしくなるということがあります。タイピングの正確性を要求されるという入力状態なのですが,変換を確定した文字がクリップボードに入るという謎の現象も併発しています。

 特別に取り上げた佐藤倫子弁護士のツイートは,「軽んじられてきたと今となっては分かる。悔しくてヒリヒリする。」という辺りが特に心に響いたというか,三宅俊一郎裁判長の訴訟指揮を思い出しながら重なって見える幻影が遠い昔の記憶と入り混じりました。

 まだ予定している家の中の片付けや裁判資料の整理を行っていないのですが,三宅俊一郎裁判長の公判の公判調書で確認しておきたいことがあって,以前,速記官による記載がないように思ったことなのですが,次のようなやりとりがありました。

 断片的な記憶なのでやりとりともいえないのですが,三宅俊一郎裁判長と向き合い,たぶん何かの質問をきっかけに,私が「あのこ(あるいは,あのおんな),本当に彼氏おったんけ? その彼氏って俺のことじゃないがぁ? いまさらかもしれんけど」などといった発言です。

 その発言に対する反応は,静まり返ったような沈黙を感じたのですが,その次に誰の発言があったのか今は思い出せません。

 そういえば,この公判調書というのもネットで公開したものがあったと思います。

 今,自分のレンタルサーバーにあるpukiwikiで確認したのですが,このページを見つけたのも令和3年3月31日付告発状を提出した前後のことでした。提出後はしばらくの間,pukiwikiの勉強をやり直し,3つのWikiの開設もやったのですが,一長一短があって,満足出来ませんでした。

 自前のサーバでpukiwikiを解説するのが最善で,ローカルのWebサーバ上にも解説をしたのですが,サーバーを使うとなると,通信量やレンタル料金の問題が発生します。お金を掛けるという選択肢はないので,限られた範囲のことしか出来ません。

 少し調べてみましたが次のような専用サーバーがあります。専用とありますがサーバー機器の貸し切りのレンタルのようなものかと思います。

\begin{quote}
《引用の始まり》
\end{quote}

\begin{quote}
CPU4Core 1CPUIntel Xeon E3-1220 v6
3.0GHzメモリ標準8GB最大64GBストレージ標準2本1組最大4本初期費用99,000円~月額料金9,900円~
\end{quote}

\begin{quote}
《引用の終わり》
\end{quote}

\begin{itemize}
\tightlist
\item
  さくらの専用サーバ
  PHY(ホスティングサービス)はさくらインターネット \url{https://server.sakura.ad.jp/?\_ga=2.82760661.2116707625.1620807212-1652753218.1620807212n} 
\end{itemize}

 4つあるプランの一番安いものですが,4Core1CPU,メモリ標準8GB最大64GB,ストレージ標準2本1組最大4本で,初期費用99,000〜,月額料金9,900円〜となっています。私のレンタルサーバーは年間で5千円台です。この違いを見ただけでも推して知るべきと思います。

 いつも使っているLinux自体がサーバーなので,やることはわかっているつもりなのですが,平成15年から3,4年ぐらいだったでしょうか,ドメインを取得して自分のパソコンをサーバーとして稼働させていました。

 パソコン内にも同じ写真があると思います。レンタルサーバーにあったpukiwikiのデータをそのままダウンロードして動かそうとしたのですが,なにやらバージョンの違いがネックになってプラグインが動かず,別の方法で動かしたような覚えもあります。けっこう時間を使いました。

\begin{itemize}
\item
  H04-06-30\_公判調書\_金沢地方裁判所\_01.jpg \url{https://t.co/KVsEXAY8vQ} 
\item
  H04-06-30\_公判調書\_金沢地方裁判所\_02.jpg \url{https://t.co/GbM5jxTTFE} 
\item
  H04-06-30\_公判調書\_金沢地方裁判所\_03.jpg \url{https://t.co/xvh4YDGEn5} 
\item
  H04-06-30\_公判調書\_金沢地方裁判所\_04.jpg \url{https://t.co/XSiLUtEqll} 
\item
  H04-06-30\_公判調書\_金沢地方裁判所\_05.jpg \url{https://t.co/7FSuEmkpht} 
\item
  H04-06-30\_公判調書\_金沢地方裁判所\_06.jpg \url{https://t.co/xG0UczyIH9} 
\item
  H04-06-30\_公判調書\_金沢地方裁判所\_07.jpg \url{https://t.co/GkvDe5PaTd} 
\item
  H04-06-30\_公判調書\_金沢地方裁判所\_08.jpg \url{https://t.co/8fkyZ2q5vp} 
\item
  H04-06-30\_公判調書\_金沢地方裁判所\_09.jpg \url{https://t.co/bpO3RkLNSp} 
\end{itemize}

 写真ファイルがあるディレクトリに移動し,for i in \texttt{ls\ *.jpg};
do ks ``\$i''; sleep 2;
done というコマンドを実行して画像付きツイートをしました。

 時刻は17時42分です。1つずつ画像をクリックして拡大しなければならず,最初は面倒に感じたのですが,番号を見ながら開いていくと,思ったほど手間には感じませんでした。一通り読んだのは何年かぶりになると思いますが,新たな発見に思うことがいくつかあり,気持に余裕が出ているためと思いました。

 どういう形で続きを書いていくか,しばらく考えたいと思います。夕食などはさんで。

\begin{itemize}
\tightlist
\item
  〈〈〈 2021/05/12 17:46:51 Linux Emacs: 〈〈〈
\end{itemize}

\hypertarget{ux4e09ux5b85ux4fcaux4e00ux90ceux88c1ux5224ux9577ux306eux5be9ux7406ux6570ux5e74ux3076ux308aux306eux516cux5224ux8abfux66f8ux306eux8aadux307fux8fbcux307fux3068ux5f01ux8b77ux58ebux30c9ux30c3ux30c8ux30b3ux30e0ux51faux53e3ux7d62ux8a18ux8005ux306etwitterux30bfux30a4ux30e0ux30e9ux30a4ux30f3ux3067ux306eux767aux898b}{%
\paragraph{三宅俊一郎裁判長の審理,数年ぶりの公判調書の読み込みと,弁護士ドットコム出口絢記者のTwitterタイムラインでの発見}\label{ux4e09ux5b85ux4fcaux4e00ux90ceux88c1ux5224ux9577ux306eux5be9ux7406ux6570ux5e74ux3076ux308aux306eux516cux5224ux8abfux66f8ux306eux8aadux307fux8fbcux307fux3068ux5f01ux8b77ux58ebux30c9ux30c3ux30c8ux30b3ux30e0ux51faux53e3ux7d62ux8a18ux8005ux306etwitterux30bfux30a4ux30e0ux30e9ux30a4ux30f3ux3067ux306eux767aux898b}}

\begin{itemize}
\tightlist
\item
  〉〉〉 Linux Emacs: 2021/05/13 05:00:29 〉〉〉
\end{itemize}

:CATEGORIES: @kanazawabengosi \#金沢弁護士会 @JFBAsns
日本弁護士連合会(日弁連) \#法務省 @MOJ\_HOUMU \#弁護士ドットコム
\#三宅俊一郎裁判長

\begin{itemize}
\tightlist
\item
  1362:2021-05-12\_17:47:55 \#告発状 \#\#\#\#
  「私たちはベテランの男性弁護士(あるいは裁判官)たちにお世話になると同時に性的に搾取される」という佐藤倫子弁護士のツイートと,三宅俊一郎裁判長(5)
  \url{https://hirono-hideki.hatenadiary.jp/entry/2021/05/12/174752} 
\end{itemize}

 上記のエントリーの終盤で第2回公判調書のことを取り上げていますが成り行きでした。成り行きでの発見ということになりますが,一区切りつけて投稿し,銭湯と買い物に行って家に戻ったのはだいたい19時20分ぐらいだったと思います。

 ブラウザのブックマークにmというフォルダのアイコンがあるのですが,パソコンの操作を再開してすぐにそれが目にとまり,ファルダの中にある出口絢記者のTwitterのリンクを開きました。時間の経過も含めだいたい次のような流れでした。

\begin{itemize}
\tightlist
\item
  2021年05月13日05時01分の登録:
  @hirono\_hideki(奉納\さらば弁護士鉄道・泥棒神社の物語)のツイート ''.*'' 3233/3233:2021-04-25\_1024〜2021-05-13\_0452 2021年05月13日05時00分の記録
  \url{https://kk2020-09.blogspot.com/2021/05/hironohideki323332332021-04-2510242021.html} 
\end{itemize}

 上記のまとめ記事からリツイートをします。

〉〉〉 kk\_hironoのリツイート 〉〉〉

\begin{itemize}
\tightlist
\item
  RT
  kk\_hirono(刑事告発・非常上告_金沢地方検察庁御中)|hirono\_hideki(奉納\さらば弁護士鉄道・泥棒神社の物語)
  日時:2021-05-13 05:13/2021/05/12 19:40 URL:
  \url{https://twitter.com/kk\_hirono/status/1392573784788398080} 
  \url{https://twitter.com/hirono\_hideki/status/1392429380895342594} 
  \textgreater{} -
  野田聖子の夫は「元暴力団員」と裁判所が認定 約10年間組員として活動
  \textbar{} デイリー新潮 \url{https://t.co/ix4NG16i3x} 
\end{itemize}

〉〉〉 kk\_hironoのリツイート 〉〉〉

\begin{itemize}
\tightlist
\item
  RT
  kk\_hirono(刑事告発・非常上告_金沢地方検察庁御中)|hirono\_hideki(奉納\さらば弁護士鉄道・泥棒神社の物語)
  日時:2021-05-13 05:15/2021/05/12 19:52 URL:
  \url{https://twitter.com/kk\_hirono/status/1392574057636237312} 
  \url{https://twitter.com/hirono\_hideki/status/1392432602041044993} 
  \textgreater{} ▶ ブロックされたツイート%aya\_deguchi(aya
  deguchi)%2021/05/12 13:10:05% \url{https://t.co/rVYW3By8Qd} 
  \textgreater{}
  →果たして明暗を分けたものは何だったのか。実は本誌の裁判では、文信氏の過去を知る「重要な証人」が出廷していたのだ
  \textgreater{} \url{https://t.co/Cv5ixKLNlO}  \#デイリー新潮
\end{itemize}

〉〉〉 kk\_hironoのリツイート 〉〉〉

\begin{itemize}
\tightlist
\item
  RT
  kk\_hirono(刑事告発・非常上告_金沢地方検察庁御中)|hirono\_hideki(奉納\さらば弁護士鉄道・泥棒神社の物語)
  日時:2021-05-13 05:15/2021/05/12 19:56 URL:
  \url{https://twitter.com/kk\_hirono/status/1392574147302068227} 
  \url{https://twitter.com/hirono\_hideki/status/1392433431171133440} 
  \textgreater{} -
  『「初」や「異例」なら記者会見を開きやすい 現役記者に聞く記者会見の開き方やコツ
  Vol.1』 - 弁護士ドットコムタイムズ \url{https://t.co/OluHOw0HK9} 
\end{itemize}

 出口絢記者のタイムラインで見つけた2つの記事です。出口絢記者のTwitterアカウントにはブロックされていて,ブロックされていることを確認しましたが,弁護士ドットコムの記者にブロックされているという現実的意味は大きいという受け止めでいます。目障りならミュートもできるはずです。

 出口絢記者のTwitterアカウントに返信などしたことはなかったと思いますが,リツイートはしていたと思います。その辺りも確認するまとめ記事を作成します。

\begin{itemize}
\tightlist
\item
  2021年05月13日05時22分の登録:
  「(出口絢\textbar{}@aya\_deguchi)」を@hirono\_hideki @kk\_hirono @s\_hironoで検索 188件の該当 2021-05-13\_05:21の記録
  \url{https://kk2020-09.blogspot.com/2021/05/ayadeguchihironohidekikkhironoshirono18.html} 
\end{itemize}

2019-11-30 16:52:08 ``-*-
弁護士ドットコム編集部出口絢記者の記事 弁護人平野敬弁護士(第二東京弁護士会):コインハイブ控訴審、検察官からの質問に「黙秘」貫く''
\url{https://twitter.com/kk\_hirono/status/1200683919831523328} 

 スドーこと平野敬弁護士を取材し記事にしていたのが出口絢記者だったことを思い出しました。どの程度の関係性があるのかわかりませんが,お酒の席での会話というのもありうる話で,この弁護士に限りませんが,弁護士の話を聞き,私のアカウントをブロックした可能性は意識しました。

 妙な関連付けで印象操作するつもりはないのですが,出口絢記者にブロックされていると気がついたのと同じ頃,伊藤詩織さんが性犯罪被害に遭う前,加害者とされた人物と陽気にお酒を酌み交わすような写真を見かけていたように思います。

2020-12-31 03:34:26
``2020-12-31-032611\_aya deguchi@aya\_deguchiブロックされています@aya\_deguchiさんのフォローやツイートの表示はできません。詳細は.jpg
\url{http://pic.twitter.com/Omkwws117n''} 
\url{https://twitter.com/s\_hirono/status/1344351153920770048} 

 ブロックでページ内検索し最初に出てきたツイートです。意外に最近だと思ったのですが,昨年2020年の大晦日となっていました。

\begin{itemize}
\tightlist
\item
  2021年05月13日05時37分の登録:
  「伊藤詩織」を@hirono\_hideki @kk\_hirono @s\_hironoで検索 709件の該当 2021-05-13\_05:36の記録
  \url{https://kk2020-09.blogspot.com/2021/05/hironohidekikkhironoshirono7092021-05.html} 
\end{itemize}

 次の4件のツイートは,「2020-12-」でページ内検索した結果です。

2020-12-07 00:20:10 ``RT @Ohsaworks:
伊藤詩織から届いた訴状を公開したところ、リンク先に韓国から DoS
攻撃(サービス停止攻撃)がありました。 1 秒間に 50
回のアクセスは異様です。 韓国の IP はすべてブロックしました。''
\url{https://twitter.com/hirono\_hideki/status/1335604954728255489} 

2020-12-25 23:08:52 ``RT @BFJNews:
【New】東京地検が12月25日、虚偽告訴と名誉毀損の疑いで書類送検されていた伊藤詩織さんを不起訴処分としました。関係者が明らかにしました。
伊藤さんは取材に「起訴されなくて良かったですけれど、すごく疑問に思った」と複雑な心境を吐露しました(瀬谷健介
@dondon\_01) \url{https://www.buzzfeed.com/jp/kensukeseya/shiori-ito-18''} 
\url{https://twitter.com/hirono\_hideki/status/1342472381005135875} 

2020-12-27 15:25:29 ``2020年12月27日15時25分の実行記録
APIのリミットに達するので8500で処理と中断しました。
twitterAPI-search-lawList-mydql-add.rb''``伊藤詩織''"
ツイート数:14/2199 リツイート数:24/2199 トータル:8500 hirono\_hideki
0/1件 kk\_hirono 0/0件 s\_hirono 0/0件 "
\url{https://twitter.com/hirono\_hideki/status/1343080544167346178} 

2020-12-27 15:37:13 ``-
書類送検された伊藤詩織さん「不起訴」明らかに。彼女が語った複雑な心境(BuzzFeed
Japan) - Yahoo!ニュース
\url{https://news.yahoo.co.jp/articles/eef871cc0f6064a077ea63ee82d072352247f9d0''} 
\url{https://twitter.com/hirono\_hideki/status/1343083497359958018} 

 伊藤詩織さんが性犯罪被害で最初に相談したのが,清水潔氏ということを思い出しページ内検索しました。13件の該当とありますが,右側にある過去記事の一覧が該当することもあります。

2018-02-18 16:48:10
``そのリニアの談合事件も、ネットでは伊藤詩織さんの準強姦事件の被疑者となったジャーナリストの関与が話題になっていました。その伊藤詩織さんに相談を受け、警察の対応の不当性を訴えていたのもジャーナリストの清水潔氏であって、どこか桶川ストーカー殺人事件に似た構造だと見ていました。''
\url{https://twitter.com/kk\_hirono/status/965130809031106560} 

2018-02-23 21:46:14 ``@hasumi29430098
伊藤詩織さんのことしか考えにくいが、これまではなぜか擁護や指示ばかりがツイートでも目についた。ブロックされているジャーナリストの清水潔氏はその代表格。なぜ今夜、発見という気もするが、昼はTwitterのトレンドで銀河鉄道999の再始動の話題を発見。''
\url{https://twitter.com/hirono\_hideki/status/967017759895838720} 

\begin{itemize}
\tightlist
\item
  「伊藤さんから積極的に誘ってきた」山口敬之さん、伊藤詩織さんの主張に反論
  \textbar{} ニコニコニュース \url{https://t.co/MDM2GdEV7i} 
\end{itemize}

 印象的だった写真を探すため,Googleで「伊藤詩織 寿司店」と検索したのですが,画像検索に切り替える前に上記の記事が目に飛び込んできました。それも弁護士ドットコムの記事の転載のようです。「2019/07/12
15:43弁護士ドットコム」とあります。

 「頚椎症で同じ体制を取っていると辛いため、ホテルの部屋の中で座る向きを変えた。その際に、伊藤さんと手が触れた。伊藤さんの方から私の手を握ってきて、引っ張られた。」というのは初めて知った話になるかと思います。

\begin{itemize}
\tightlist
\item
  伊藤詩織さん考察・時系列【2015】4月3日 \textbar{} 伊藤詩織さん考察blog
  \textasciitilde Open Her Black Box\textasciitilde{}
  \url{https://t.co/GAatArSfyo} 
\end{itemize}

 画像検索の結果からリンクを開いたのが上記の記事ですが,写真が沢山掲示された記事で,前に見たと記憶にある写真は単発か2,3枚程度の写真の掲載であったように思います。

 出口絢記者のタイムラインで最初に開いて読んだ記事は次のツイートでした。

\begin{itemize}
\tightlist
\item
  TW aya\_deguchi(aya deguchi) 日時: 2021/05/12 13:10:05 URL:
  \url{https://twitter.com/aya\_deguchi/status/1392331217861632002} 
  \textgreater{}
  →果たして明暗を分けたものは何だったのか。実は本誌の裁判では、文信氏の過去を知る「重要な証人」が出廷していたのだ\\
  \textgreater{} \url{https://t.co/Vimgf7TWNA}  \#デイリー新潮
\end{itemize}

 よく見ると,出口絢記者のツイートの本文に野田聖子という名前はなく,Twitterカードの記事の要約の表示に,野田聖子氏の顔写真と見出しがあります。

\begin{itemize}
\tightlist
\item
  野田聖子の夫は「元暴力団員」と裁判所が認定 約10年間組員として活動
  \textbar{} デイリー新潮 \url{https://t.co/FB1zYtXQH3}  ¥\n 国内 政治
  週刊新潮 2021年5月6日・13日号掲載
\end{itemize}

 野田聖子氏の名前はTwitterのトレンドでも見かけていないのですが,「2021年5月6日・13日号掲載」というだけで記事の配信時刻が見当たりません。

\begin{quote}
《引用の始まり》
\end{quote}

\begin{quote}
2021年05月13日06時21分の実行記録 8500で処理を中断
twitterAPI-search-lawList-mydql-add.rb ``野田聖子'' ツイート数:14/2411
リツイート数:6/2411 トータル:8500hirono\_hideki 1/0件 kk\_hirono
3/1件 s\_hirono 0/0件
\end{quote}

\begin{quote}
《引用の終わり》
\end{quote}

 コマンドの処理結果のメッセージですが,書式を修正し2つのツイートで投稿しました。同じようにスクリプトの変更をしておこうかと思います。

 次のようにスクリプトで出力するメッセージの書式を変更しました。

\begin{lstlisting}
puts "#{create_date}の実行記録:#{'8500で処理を終了' if @total >= 8500} twitterAPI-search-lawList-mydql-add.rb \"#{query}\" ツイート数:#{@tw_count}/#{users.count} リツイート数:#{@rt_count}/#{users.count} トータル:#{@total}"
puts "\"#{query}\"の該当: hirono_hideki #{tweet_count['hirono_hideki']}/#{retweet_count['hirono_hideki']}件 kk_hirono #{tweet_count['kk_hirono']}/#{retweet_count['kk_hirono']}件 s_hirono #{tweet_count['s_hirono']}/#{retweet_count['s_hirono']}件"
\end{lstlisting}

\begin{itemize}
\tightlist
\item
  2021年05月13日06時55分の登録:
  REGEXP:''野田聖子''/データベース登録済みツイートの検索:2015-11-05〜2021-05-13/2021年05月13日06時54分の記録:ユーザ・投稿:24/33件
  \url{https://kk2020-09.blogspot.com/2021/05/regexp2015-11-052021-05.html} 
\end{itemize}

〉〉〉 kk\_hironoのリツイート 〉〉〉

\begin{itemize}
\item
  RT
  kk\_hirono(刑事告発・非常上告_金沢地方検察庁御中)|koneko10291(悠々ママ)
  日時:2021-05-13 07:03/2016/08/18 07:27 URL:
  \url{https://twitter.com/kk\_hirono/status/1392601245173518338} 
  \url{https://twitter.com/koneko10291/status/766038802322173952} 
  \textgreater{} 社会の考え方はそうなんだなぁ
  >障害児を産んだ、その医療費は国民が負担する、ならば一生感謝すべきだ、と。(中略)と慄然(りつぜん)としました。
  特集ワイド:相模原殺傷事件 感じた嫌悪 長男が障害持つ野田聖子衆院議員
  - 毎日新聞 \url{https://t.co/AjwHbnqwf2} 
\item
  特集ワイド:相模原殺傷事件 感じた嫌悪「いつか起きる・・・」 長男が障害持つ野田聖子衆院議員
  \textbar{} 毎日新聞 \url{https://t.co/dvPj6pfK7P} 
\end{itemize}

 大半が有料記事でしたが,「重い障害を持つ長男真輝(まさき)ちゃん(5)を育てながら国政で活動する自民党の野田聖子衆院議員は何を語るのだろう。【構成・吉井理記、写真・内藤絵美】」とありました。見たことのある情報と思ったのですが,重い障害というのは意外です。

 まとめ記事を確認するとモトケンこと矢部善朗弁護士(京都弁護士会)のリツイートとなっていました。(03/33)
RT motoken\_tw(モトケン)|koneko\_1211(悠々ママ(小物、頑張る))
日時:2016-08-18 07:48:00 +0900/2016-08-18 07:27:00 +0900

\begin{itemize}
\item
  (13/33) TW news\_type\_c(NEWS JAPAN) 日時: 2021-05-12 11:42:11
  +0900 URL:
  \url{https://twitter.com/news\_type\_c/status/1392309097517617152\textgreater} {}
  野田聖子の夫は「元暴力団員」と裁判所が認定 約10年間組員として活動
  \textbar{} デイリー新潮 {[}86コメント{]}\textgreater{}
  新聞・テレビが報じない話を記事にする。それこそが週刊誌の真骨頂であるが、痛いところを書かれた相手によっては、法廷で``潔白''を・・・
  \url{https://t.co/0zTUVurADA} 
\item
  \begin{enumerate}
  \def\labelenumi{(\arabic{enumi})}
  \setcounter{enumi}{4}
  \tightlist
  \item
    デイリー新潮(@dailyshincho)さんの返信があるツイート / Twitter
    \url{https://t.co/WRvvSKf35T} 
  \end{enumerate}
\end{itemize}

 今まで未登録だったようですが,デイリー新潮(@dailyshincho)のTwitterアカウントをリストに追加しました。

\begin{itemize}
\item
  \begin{enumerate}
  \def\labelenumi{(\arabic{enumi})}
  \setcounter{enumi}{4}
  \tightlist
  \item
    文春オンライン(@bunshun\_online)さんの返信があるツイート / Twitter
    \url{https://t.co/5mco83yybq} 
  \end{enumerate}
\end{itemize}

 そういえば検索のまとめ記事で余り見かけないと思ったのですが,「おすすめツイート」で見かけた上記の文春オンライン(@bunshun\_online)のTwitterアカウントはリストに登録済みでした。

〉〉〉 kk\_hironoのリツイート 〉〉〉

\begin{itemize}
\tightlist
\item
  RT
  kk\_hirono(刑事告発・非常上告_金沢地方検察庁御中)|bunshun\_online(文春オンライン)
  日時:2021-05-13 07:15/2021/05/13 07:11 URL:
  \url{https://twitter.com/kk\_hirono/status/1392604461726535680} 
  \url{https://twitter.com/bunshun\_online/status/1392603450513387521} 
  \textgreater{} 05月13日 07時点で注目されている写真はこちら
  コブクロ黒田
  ``不倫相手が自殺未遂''報道の「週刊文春」を東京地裁に``差し止め''請求 →
  @bunshun\_online \url{https://t.co/ruUwgG06l9} 
\end{itemize}

 固定ツイートはないようで,タイムラインの一番上に表示されていた上記のツイートですが3分前という表示でした。2,3日前からちらほらとTwitterで見かけていた話題ですが,リンクを開いたり内容は読んでいませんでした。

\begin{quote}
《引用の始まり》
\end{quote}

\begin{quote}
音楽デュオ・コブクロの黒田俊介(44)が5月10日、5月12日発売の「週刊文春」の出版差し止めを請求する「仮処分命令申立書」を東京地方裁判所民事部に提出した。

「週刊文春」では、黒田と不倫関係にあった30代独身女性・A子さんが自殺未遂していたトラブルを取材していた。
\end{quote}

\begin{quote}
《引用の終わり》
\end{quote}

\begin{itemize}
\tightlist
\item
  コブクロ黒田
  ``不倫相手が自殺未遂''報道の「週刊文春」を東京地裁に``差し止め''請求
  \textbar{} 文春オンライン \url{https://bunshun.jp/articles/-/45398n} 
\end{itemize}

 そういえばしばらく前,女性政治家で元検事という肩書もあったと思う山尾志桜里の不倫相手の妻が自殺していたというニュースになるのか話題がネットにありました。元妻のようですが,不倫相手の男性も弁護士でした。倉持麟太郎弁護士。

\begin{itemize}
\tightlist
\item
  山尾志桜里氏は``不倫略奪男''の元妻自死でもダンマリ 自民党入りはご破算(東スポWeb)
  - Yahoo!ニュース \url{https://t.co/yqosFrtn8L}  4/28(水) 11:15配信
\end{itemize}

\begin{quote}
《引用の始まり》
\end{quote}

\begin{quote}
「週刊文春」編集部の取材では、妻子のある黒田はA子さんと2019年秋から不倫関係に陥っていた。黒田が出張の際はA子さんを出張先に呼び、A子さんの家族にもブランド品をプレゼントするなど、親密な関係が続いた。しかし翌年、関係が悪化。精神的に不安定となったA子さんは、それまでのLINEのやり取りなどを黒田の自宅ポストに投函するなどした。女性の言動を「ストーカー」だとみなした黒田の代理人弁護士は、その旨を女性に警告。今年3月には大阪府警豊中警察署にも相談に赴いていた。

 1月にうつ病と診断されていた女性は、「黒田さんは、私のことはもうどうでもいいと思っている」と感じ、4月13日未明、70錠以上の抗うつ薬と睡眠薬を服用し、自殺を図るに至った。
\end{quote}

\begin{quote}
《引用の終わり》
\end{quote}

\begin{itemize}
\tightlist
\item
  コブクロ黒田
  ``不倫相手が自殺未遂''報道の「週刊文春」を東京地裁に``差し止め''請求
  \textbar{} 文春オンライン \url{https://bunshun.jp/articles/-/45398n} 
\end{itemize}

 「女性の言動を「ストーカー」だとみなした黒田の代理人弁護士は、その旨を女性に警告。今年3月には大阪府警豊中警察署にも相談に赴いていた。」,「1月にうつ病と診断されていた女性は」,「4月13日未明、70錠以上の抗うつ薬と睡眠薬を服用し、自殺を図るに至った。」とあります。

 記事を読みすすめると,代理人弁護士のことがいろいろ出てきますが,名前も明かさない弁護士は,小室圭さんの婚約問題でも共通しており,仮面をかぶった幻影のような存在の弁護士に思えてきます。

 今頃になって文春オンラインの記事の見出しにも自殺未遂とあることに気がついたのですが,不倫相手の女性が自殺で亡くなっている報道とばかり思い込んでいました。

2021年05月13日07時36分の実行記録:8500で処理を終了
twitterAPI-search-lawList-mydql-add.rb ``コブクロ'' ツイート数:12/2412
リツイート数:6/2412 トータル:8500``コブクロ''の該当: hirono\_hideki
0/1件 kk\_hirono 1/1件 s\_hirono 0/0件

\begin{itemize}
\tightlist
\item
  (18/33) RT
  1961kumachin(くまちん(弁護士中村元弥))|nmcmnc(春の文筆業)
  日時:2021-05-12 14:03:37 +0900/2021-05-12 14:00:00 +0900 URL:
  \url{https://twitter.com/1961kumachin/status/1392344691878756353} 
  \url{https://twitter.com/nmcmnc/status/1392343975147700226\textgreater} {}
  これも強烈なのにあんまり報じられない。元反社だというだけじゃなく、その立場を利用しているのでは?というとこまで踏み込んでるのに/
  野田聖子の夫は「元暴力団員」と裁判所が認定 約10年間組員として活動
  \url{https://t.co/vUEubNOkZO} 
\end{itemize}

〉〉〉 kk\_hironoのリツイート 〉〉〉

\begin{itemize}
\item
  RT
  kk\_hirono(刑事告発・非常上告_金沢地方検察庁御中)|nmcmnc(春の文筆業)
  日時:2021-05-13 07:40/2019/05/06 14:48 URL:
  \url{https://twitter.com/kk\_hirono/status/1392610684039430146} 
  \url{https://twitter.com/nmcmnc/status/1125276013217234945} 
  \textgreater{}
  (固定ツイート用)お仕事のスケジュールはここになるべく挙げております。
  \url{https://t.co/0GzdLxDiej} 
\item
  \begin{itemize}
  \tightlist
  \item
    (18/33) RT
    1961kumachin(くまちん(弁護士中村元弥))|nmcmnc(春の文筆業)
    日時:2021-05-12 14:03:37 +0900/2021-05-12 14:00:00 +0900 URL:
    \url{https://t.co/7yBmRgYn3n}  \url{https://t.co/kSVPaP3DBE} 
  \end{itemize}
\item
  お仕事のスケジュール - 能町みね子のふつうにっき
  \url{https://t.co/xmxVnt0CO6} 
\end{itemize}

 気になる内容の中村元弥弁護士のリツイートだと思ったのですが,能町みね子という人のTwitterアカウントだったようです。以前はそのままの名前のツイートを何度か見かけていたのですが,新潟県の能町のことを決まって思い出していました。

\begin{itemize}
\tightlist
\item
  能町みね子 - Wikipedia \url{https://t.co/8ujZLVmDIv} 
\end{itemize}

 どんな人なのか何も知らず,昭和の時代のおばさんのような名前というイメージだけあったのですが,名前はペンネームで,エッセイストやライター,コラムニストなどとあります。2007年に性別適合手術をうけ戸籍の性別も女性に変更とあります。

\begin{itemize}
\tightlist
\item
  (19/33) TW hiroya\_nakatani(中谷寛也) 日時: 2021-05-12 14:29:40
  +0900 URL:
  \url{https://twitter.com/hiroya\_nakatani/status/1392351244706930688\textgreater} {}
  1999年の解散時に足を洗っているとすれば、2011年の結婚時にはもう5年以上はたっていることにはなるけどなあ。\textgreater\textgreater{}
  野田聖子の夫は「元暴力団員」と裁判所が認定 約10年間組員として活動
  \url{https://t.co/h5YXefidUS} 
\end{itemize}

 リツイートではなくツイートとしてまとめ記事にあるアカウントなので,リストに登録済みになりますが,見覚えのないプロフィールの名前で,Twitterのページを開くとプロフィールに弁護士とありました。実名の弁護士でしょう。

 ヘッダ画像が野球場のチアガールのような写真,プロフィールのアイコンがトランペットのようなラッパを吹く写真で,タイムラインのツイートを見ましたが,弁護士らしいツイートというのはほとんど見当たらず,よくみる弁護士アカウントとは異色なものを感じました。

\begin{itemize}
\tightlist
\item
  野田聖子の夫は「元暴力団員」と裁判所が認定 約10年間組員として活動
  \textbar{} デイリー新潮
  \url{https://www.dailyshincho.jp/article/2021/05120557/?all=1} 
\end{itemize}

 このデイリー新潮の野田聖子氏の記事ですが,昨夜一通り読んで,三宅俊一郎裁判長の審理の幻影を見るような気分になりました。証人というのは情状証人になると思う母親だけでした。このあと記述を進めていきますが,被告発人岡田進弁護士との間に,事前の打ち合わせは皆無でした。

 次の出口絢記者のタイムラインのツイートで知った記事に移ります。

\begin{itemize}
\tightlist
\item
  金屏風会見で中森明菜は近藤真彦に騙されたのか 仕掛け人が語っていた真相
  \textbar{} デイリー新潮 \url{https://t.co/iv62X5cHy2} 
\end{itemize}

 ブラウザで野田聖子氏のデイリー新潮の記事のページが表示されたタブを終了させようと思いマウスポインターを動かし目線を少し上に向けたタイミングで気がついたのが,上記の中森明菜の記事でアクセスランキングの1位となっていました。

 1989年の大晦日とあるので平成元年ですが,そのような会見があったとはしらず,記事に写真のある中森明菜の隣の席の男性が近藤真彦のようですが,まったく別人の顔つきに思えました。

 「明菜は同7月11日夜、東京・六本木の近藤宅で自殺未遂を図った。その後、姿を消していたとあって、注目の会見だった。」とあります。芸能人同士の男女間のトラブルで中森明菜がひどく傷つき,その後の芸能活動も著しく下火になったような記憶はありました。

 「「私が仕事をしていく上で一番信頼しなきゃいけない人たちを、信頼することが出来なくなってしまった」(明菜)」とありますが,これも平成4年4月1日に傷害・準強姦被告事件がなかったら,被害者安藤文さんの人生がどうなっていたのか,想像するパターンの要素の1つのようです。

 三宅俊一郎裁判長の第2回公判調書は9枚の写真になっていたと思います。文字数は少なく,時間は測っていなかったですが,読むのに10分も掛かったのかという程度の短い時間,寸劇のようなものでした。

\begin{quote}
《引用の始まり》
\end{quote}

\begin{quote}
 当時、明菜は自殺未遂の理由をこう説明した。

「私が仕事をしていく上で一番信頼しなきゃいけない人たちを、信頼することが出来なくなってしまった」(明菜)

 近藤との関係がうまくいっていないことが自殺未遂の大きな理由と思われたが、一方で仕事上の悩みを抱えていたのも事実。

 当時の明菜は超売れっ子。カネのなる木には不純な輩も寄って来る。明菜に甘言を並べ、研音から独立させようと目論む人間もいた。明菜が信頼していたスタッフの悪口を吹き込んだ。これによって明菜は人間不信に陥り、やがてトラブルに発展する。

 では、なぜ近藤宅を自殺未遂の場所に選んだのか。明菜はこう説明した。

「自分が一番信頼できたたった1人の人だったので、近藤さんの部屋に行ってしまって。最初に見つけてほしいなと思って」(明菜)

 明菜の孤独を表す。一方で部屋のカギまで渡した近藤の無自覚と無責任も感じさせる。親しい関係にあったのは疑いようがない。

 会見当時の近藤は25歳。明菜の復帰を祝福する通り一遍の言葉は口にしたものの、その翌1990年に控えた芸能生活10周年のアピールのほうが熱心だった。やはり無責任に映った。
\end{quote}

\begin{quote}
《引用の終わり》
\end{quote}

\begin{itemize}
\tightlist
\item
  金屏風会見で中森明菜は近藤真彦に騙されたのか 仕掛け人が語っていた真相
  \textbar{}
  デイリー新潮 \url{https://www.dailyshincho.jp/article/2021/05101100/?all=1n} 
\end{itemize}

 上記にいくらか広めの範囲の引用をしましたが,「明菜に甘言を並べ、研音から独立させようと目論む人間もいた。明菜が信頼していたスタッフの悪口を吹き込んだ。これによって明菜は人間不信に陥り、やがてトラブルに発展する。」という部分がとりわけ印象的です。

\begin{quote}
《引用の始まり》
\end{quote}

\begin{quote}
では、いつから「明菜は騙された」という金屏風会見の伝説が生まれたのか。新聞・雑誌のすべてを検索システムを使い、確認したところ、2016年2月以降のこと。つい最近だ。

 ある雑誌がこんな匿名の芸能ライターの談話を掲載した。

「明菜はジャニーズサイドから『マッチとの婚約会見だから』と呼ばれたが、現場に来てみると違っていた。つまり騙されたというのが通説になっています」

 以来、この類の話が散見されるようになった。
\end{quote}

\begin{quote}
《引用の終わり》
\end{quote}

\begin{itemize}
\tightlist
\item
  金屏風会見で中森明菜は近藤真彦に騙されたのか 仕掛け人が語っていた真相
  \textbar{}
  デイリー新潮 \url{https://www.dailyshincho.jp/article/2021/05101100/?all=1\&page=2n} 
\end{itemize}

 2ページ目の初めの部分を上記に引用しましたが,「いつから「明菜は騙された」という金屏風会見の伝説が生まれたのか。」ということで,私は知らなかったのですが,こういう話が2016年2月から一人歩きをしていたようです。

 「近藤は1990年代からヒット曲がほとんどない。明菜の会見以降、売れなくなったと言い換えてもいい。近藤は無責任に接してしまった明菜の幻影から、いまだ逃れられていないのではないか。」と,ここで幻影が出てくるとは。

 この記事は「金屏風の伝説が生まれた背景もそこにあるだろう。」で締めくくられていますが,高堀冬彦(たかほり・ふゆひこ) 放送コラムニスト,ジャーナリストとあります。初めて見るお名前ですが,1990年からスポーツ新聞社に入社し記者を続けられているようです。

 確認すると2021年5月10日掲載となっている記事ですが,ちょっとした不思議な巡り合わせを感じた記事でした。同時にネットの情報の多さをあらためて感じましたが,こういう気になった記事を立て続けに読んでいるとそれだけで時間がなくなりそうです。

 寄り道や道草をやったようですが,このあと説明をしていく三宅俊一郎裁判長の審理,刑事裁判の背景を理解していただくには,有用で参考になる情報と思います。

 次こそ出口絢記者のツイートで2つ目の記事のご紹介になりますが,ツイートは2つになっていて,その1つ前になるリツイートというのも気になった内容があるので,一緒に取り上げて書いておきたいと思います。

\begin{itemize}
\item
  TW aya\_deguchi(aya deguchi) 日時: 2021/05/12 11:57:43 URL:
  \url{https://twitter.com/aya\_deguchi/status/1392313006873141253} 
  \textgreater{}
  会見に参加してもクラブ員ではないので受付を担当することはなく諸々興味深かった\\
  \textgreater{}\\
  \textgreater{}
  \textgreater 会見の開催にあたり、弁護士のネームバリューは\\
  \textgreater{}
  影響します。著名な刑事弁護人や、話題のテーマをよく扱っている弁護士からの会見は、基本的に受けることとなります。\\
  \textgreater{} \url{https://t.co/A0j49NikGi} 
\item
  TW aya\_deguchi(aya deguchi) 日時: 2021/05/12 11:53:33 URL:
  \url{https://twitter.com/aya\_deguchi/status/1392311958691454979} 
  \textgreater{}
  司法記者クラブでどんなものなら記者会見が開けるのか、という話面白かった\\
  \textgreater{} \url{https://t.co/A0j49NikGi} 
  \textgreater{}\\
  \textgreater{} \textgreater ーー会見を受け付けてもらえない場合とは\\
  \textgreater{}
  電話を受けた幹事社が「ニュースではない」と判断したものです。弁護士から、よくあるのは「無罪が出そうなので記者会見やりたい」というものです。
\item
  TW bengo4topics(弁護士ドットコムニュース) 日時: 2021/05/12 11:34:17
  URL: \url{https://twitter.com/bengo4topics/status/1392307109212004355} 
  \textgreater{}
  茨城一家殺傷事件に関連して、社長が「被疑者と同姓」という理由で、無関係の三郷市の建設会社が誹謗中傷被害にあっています。\\
  \textgreater{}\\
  \textgreater{}
  いたずら電話は300件ほどあり、いきなり「死ね」と言って切るものもあったそうです。\\
  \textgreater{} \url{https://t.co/tlKfpalBsO} 
\item
  茨城一家殺傷事件「被疑者は社長の息子」デマ拡散 無関係の会社にいたずら電話300件
  - 弁護士ドットコム \url{https://www.bengo4.com/c\_23/n\_13037/} 
\end{itemize}

 まず,上記の弁護士ドットコムの記事ですが,リンクを開いてみると読んでいない内容だと思いました。

 この茨城一家殺傷事件とある事件の被疑者逮捕は,宇出津図書館にある新聞で最初に知ったと思うのですが,ネットでは情報を見かけず数日が過ぎたような気がします。テレビはつけていないので報道はみていません。

 事件発生時はよくテレビをみていたので報道の方もよく見ていました。謎の多い事件とされましたが,雑木林の中に1軒だけあるようなお宅で,田舎にある神社のような風景で,報道内容の不可解さとあいまってミステリーゾーンのようにも思えた印象的な未解決事件でした。

 時刻は10時21分です。しばらく横になってうとうとしていたのですが,物音や告知機の放送ですぐに目が覚めていました。3時25分ぐらいに目が覚めてそのまま起きていたのですが,疲れを感じて眠くなっていました。

※ @kk\_hironoのアカウントがブロックされ,リツイートに失敗したツイート

\begin{itemize}
\tightlist
\item
  TW aya\_deguchi(aya deguchi) 日時:2021/05/12 11:57:43 URL:
  \url{https://twitter.com/aya\_deguchi/status/1392313006873141253} 
  \textgreater{}
  会見に参加してもクラブ員ではないので受付を担当することはなく諸々興味深かった\\
  \textgreater{}\\
  \textgreater{}
  \textgreater 会見の開催にあたり、弁護士のネームバリューは\\
  \textgreater{}
  影響します。著名な刑事弁護人や、話題のテーマをよく扱っている弁護士からの会見は、基本的に受けることとなります。\\
  \textgreater{} \url{https://t.co/A0j49NikGi} 
\end{itemize}

 違ったリンクを開いたのですが,出口絢記者のツイートのリンクだったらしく,「このアカウントの所有者はツイートを表示できるアカウントを制限しているため、このツイートを表示できません。詳細はこちら」という表示でした。

\begin{itemize}
\tightlist
\item
  『「初」や「異例」なら記者会見を開きやすい 現役記者に聞く記者会見の開き方やコツ
  Vol.1』 - 弁護士ドットコムタイムズ
  \url{https://www.bengo4.com/times/articles/293/} 
\end{itemize}

 今度は間違わずに記事のリンクを開けました。昨夜読んでいろいろ参考になった珍しい報道の内幕を垣間見たような記事だったのですが,記事の内容より,第2回公判調書を中心に取り上げると決めた矢先のことで,その出現のタイミングの方が気になりました。

 告発事件の事実関係の本題に分け入るタイミングということです。

\begin{itemize}
\tightlist
\item
  〈〈〈 2021/05/13 10:32:54 Linux Emacs: 〈〈〈
\end{itemize}

\hypertarget{ux4f0aux6771ux4e00ux5ee3ux88c1ux5224ux5b98}{%
\subsubsection{伊東一廣裁判官}\label{ux4f0aux6771ux4e00ux5ee3ux88c1ux5224ux5b98}}

\hypertarget{ux5e744ux670824ux65e5ux5b87ux51faux6d25ux56f3ux66f8ux9928ux3067ux5e73ux621015ux5e74ux306eux5317ux570bux65b0ux805eux7e2eux5c0fux7248ux3092ux307fux3066ux3044ux3066ux6c17ux306bux306aux308bux5224ux6c7aux304cux8907ux6570ux3042ux308aux81eaux5b85ux306bux623bux3063ux3066ux30cdux30c3ux30c8ux3067ux8abfux3079ux305fux4f0aux6771ux4e00ux5ee3ux88c1ux5224ux5b98}{%
\paragraph{2021年4月24日,宇出津図書館で平成15年の北國新聞縮小版をみていて気になる判決が複数あり,自宅に戻ってネットで調べた伊東一廣裁判官}\label{ux5e744ux670824ux65e5ux5b87ux51faux6d25ux56f3ux66f8ux9928ux3067ux5e73ux621015ux5e74ux306eux5317ux570bux65b0ux805eux7e2eux5c0fux7248ux3092ux307fux3066ux3044ux3066ux6c17ux306bux306aux308bux5224ux6c7aux304cux8907ux6570ux3042ux308aux81eaux5b85ux306bux623bux3063ux3066ux30cdux30c3ux30c8ux3067ux8abfux3079ux305fux4f0aux6771ux4e00ux5ee3ux88c1ux5224ux5b98}}

\begin{itemize}
\tightlist
\item
  〉〉〉 Linux Emacs: 2021/04/30 17:26:08 〉〉〉
\end{itemize}

:CATEGORIES: @kanazawabengosi \#金沢弁護士会 @JFBAsns
日本弁護士連合会(日弁連) \#法務省 @MOJ\_HOUMU \#伊東一廣裁判官
\#被告発人木梨松嗣弁護士

 さきほど作成した「2021-04-30\_16:52の記録」ですが,以下に全文を引用します。全部で14件のツイートが記録されていますが,5日前の4月25日10時37分から始まっています。

 宇出津図書館で伊東一廣裁判官の名前を見たのは24日の夕方で,戻ってから伊東一廣裁判官の検索を行ったものと思い込んでいたのですが,実際は翌日の25日午前10時半過ぎから検索を始めていたようです。

\begin{quote}
《引用の始まり》
\end{quote}

\begin{quote}
2021-04-25 10:37:44 ``伊東一廣裁判官(30期)の経歴 \textbar{}
弁護士山中理司のブログ https://t.co/QShst0J29c H14.4.1 ~ H17.3.31
金沢地裁第3部部総括''
https://twitter.com/hirono\_hideki/status/1386132285032861697

2021-04-25 10:41:24
``保護責任者遺棄致死被告事件(金沢地判平成16年06月24日)の判決文 -
無料で法律、判例検索 - とある法律判例の全文検索β
https://t.co/kdxantvszw
¥\n(検察官 磯村建,弁護人 木梨松嗣)(求 刑 懲役5年)平成16年6月24日金沢地方裁判所第三部¥\n裁判長裁判官   伊東一廣¥\n裁判官   田中''
https://twitter.com/hirono\_hideki/status/1386133209092542469

2021-04-25 10:43:47 ``弁護士法人ITJ法律事務所 裁判例集
https://t.co/9eaqD0sQRK
¥\n(検察官 磯村建,弁護人 木梨松嗣)¥\n(求 刑 懲役5年)¥\n平成16年6月24日¥\n金沢地方裁判所第三部¥\n裁判長裁判官   伊東一廣¥\n裁判官   田中智子¥\n裁判官和田将紀''
https://twitter.com/hirono\_hideki/status/1386133806722129921

2021-04-25 11:37:26
``2021-04-25-104220\_(検察官 磯村建,弁護人 木梨松嗣)(求 刑 懲役5年)平成16年6月24日金沢地方裁判所第三部裁判長裁判官   伊東一廣裁判官   田中智.jpg
https://t.co/ZZbCsIo7Lb''
https://twitter.com/s\_hirono/status/1386147309369450500

2021-04-25 11:37:43
``2021-04-25-105633\_鳥毛美範委員,畠山美智子委員,山田賀規委員,伊東一廣委員(オブザーバー)上原卓也裁判官.jpg
https://t.co/cdnQAs1gJd''
https://twitter.com/s\_hirono/status/1386147381909999618

2021-04-25 19:12:55 ``RT @hibi\_tantan24:
「袴田事件」に関わった裁判官の実名:石見勝四、横川敏雄、宮崎梧一、鈴木勝利、安廣文夫、今井功、高井吉央、中西武夫、柏井康央、塩野宜慶、塚本重頼、木下忠良、栗本―央、内山梨枝子、伊東一廣、竹花俊徳、小西秀宣、古田佑紀、中川了滋、津野修¥\nhttp://t.co/pLP1BgDU71''
https://twitter.com/hirono\_hideki/status/1386261936371732485

2021-04-25 21:00:36 ``伊東一廣 - Google 検索
https://t.co/RhTk22tmJG''
https://twitter.com/hirono\_hideki/status/1386289036570992642

2021-04-25 21:02:05
``静岡合同公証役場-静岡県静岡市の電子定款認証をサポートします!
https://t.co/I4EHC3pUaw 指定公証人 山根 薫 公証人    伊東一廣
公証人    藤原俊二 公証人¥\n櫻井達朗 公証人''
https://twitter.com/hirono\_hideki/status/1386289408530259971

2021-04-25 22:40:21
``2021-04-25-190833\_「裁判員制度について」名古屋地方裁判所豊橋支部 支部長 伊東一廣氏 .jpg
https://t.co/drncxLFeHO''
https://twitter.com/s\_hirono/status/1386314138947768321

2021-04-25 22:40:39
``2021-04-25-190851\_「裁判員制度について」名古屋地方裁判所豊橋支部 支部長 伊東一廣氏 .jpg
https://t.co/wi1nagCNEU''
https://twitter.com/s\_hirono/status/1386314211450523652

2021-04-25 22:40:56
``2021-04-25-190950\_講 師:名古屋地方裁判所豊橋支部長伊東一廣氏出身地:長野県茅野市.jpg
https://t.co/EosjCpstxb''
https://twitter.com/s\_hirono/status/1386314283982618636

2021-04-25 22:41:13
``2021-04-25-191001\_講 師:名古屋地方裁判所豊橋支部長伊東一廣氏出身地:長野県茅野市.jpg
https://t.co/mS47rwP0tM''
https://twitter.com/s\_hirono/status/1386314356896325639

2021-04-26 11:41:19
``数日前から,裁判官は記録を,少なくとも私の上申書はほとんど読んでいなかったものと考えていたのですが,本来一番の発見となったのは,このあと集中的に取り上げる,伊東一廣裁判長のことです。これが思わぬ発見となりました。それも用意されていたような今頃の時期です。''
https://twitter.com/kk\_hirono/status/1386510675057410050

2021-04-30 11:24:54 ``静岡地方裁判所 平成5年(わ)141号 判決 -
大判例 https://t.co/UpzpRX67Fe
(裁判長裁判官 鈴木勝利 裁判官 伊東一廣)''
https://twitter.com/hirono\_hideki/status/1387956093297459200
\end{quote}

\begin{quote}
《引用の終わり》
\end{quote}

\begin{itemize}
\tightlist
\item
  奉納\危険生物・弁護士脳汚染除去装置\金沢地方検察庁御中\_2020:
  「伊東一廣」を@hirono\_hideki @kk\_hirono @s\_hironoで検索 14件の該当 2021-04-30\_16:52の記録
  \url{https://kk2020-09.blogspot.com/2021/04/hironohidekikkhironoshirono142021-04.html} 
\end{itemize}

 さきほど気がついたというか思い出したのですが,最も気になった加賀市の保護責任者遺棄致死で懲役4年という判決は,北國新聞縮小版で見かけた記事ではなく,ネットで伊東一廣裁判官を調べたことで出てきた情報でした。

 そして,その加賀市の保護責任者遺棄致死で懲役4年という判決が「平成16年6月24日金沢地方裁判所第三部 裁判長裁判官   伊東一廣」となっていたので,その日,25日の夕方,前日に続き宇出津図書館に行って,平成16年6月の北國新聞縮小版を調べたのでした。

 そして北國新聞縮小版の記事にはなっていないことを確認したのでした。

 他に気になったのは,平成15年の北國新聞縮小版で見た,次の見出しの記事になります。すべて伊東一廣裁判官が判決を出した金沢地方裁判所の刑事裁判になります。

 「珠洲の保険金殺人 二審も無期判決,名高裁金沢」,「松任の長女刺殺 検察側,浦野被告に懲役8年求刑,金沢地裁」,「父親に懲役4年6月 松任の娘殺害金沢地裁判決 承諾殺人退ける」,「加賀の風俗女性強殺未遂 被告に懲役12年 金沢地裁判決 人命軽視甚だしい」

 風俗女性強殺未遂とありますが,求刑が懲役15年,気になったのが「五百万円の慰謝料で被害女性と示談が成立」という点です。無店舗型風俗店(出張ヘルス)の女性の首を締めて現金を強奪とありますが,それ以上の具体的内容は判決から不明です。

 「強盗殺人未遂罪に問われた」ともこの北國新聞の記事にはあるのですが,強盗殺人という罪名は存在しないはずです。今目についた部分に,「失神させた。女性を殺害したと思い込んだ・・・女性の財布から現金三万五千円,バックからネックレスなどを奪って逃げた」とありました。

 「判決理由で伊東裁判長は,パチンコや風俗で借金を重ねたことを「自堕落な生活」と厳しく避難し,「被害女性の精神的苦痛も大きい」と断じた。」とあります。同じ加賀市での事件ですが,保護責任者遺棄致死で懲役4年に似た点があります。

 保護責任者遺棄致死は実の母親が被害者ですが,伊東一廣裁判長が無期懲役判決を出したという珠洲市の保険金目的殺人事件も被害者が実の母親という共通点があります。

 女性を殺害したと思い込んだ,という点は,「父親に懲役4年6月 松任の娘殺害金沢地裁判決 承諾殺人退ける」とも共通性を感じるのですが,実に厳しい断罪で,初めに伊東一廣という裁判長の名前を特別なものと感じたのもこの判決理由でした。

 ただ,後になってネットで調べた「父親に懲役4年6月 松任の娘殺害金沢地裁判決 承諾殺人退ける」の事件内容をみると生き残りは父親だけ,判決では母親が自殺し,悲観した娘も洋弓で自殺を図り,頭に弓が刺さったまま死にきれなかった娘にとどめをさしたという事実認定になっていました。

\begin{itemize}
\tightlist
\item
  1333:2021-04-26\_07:29:01 \#告発状 \#\#\#\#
  被告発人岡田進弁護士が代理人となり過失割合が45%になった,金沢市内で1級1号後遺障害を残した13歳女子の自転車事故
  \url{https://hirono-hideki.hatenadiary.jp/entry/2021/04/26/072859} 
\item
  1334:2021-04-26\_08:43:34 \#告発状 \#\#\#\#
  4月24日と25日の両日,宇出津図書館で調べた平成15年の「金沢市内で1級1号後遺障害を残した13歳女子の自転車事故」,岡田進弁護士が代理人で過失相殺45%
  \url{https://hirono-hideki.hatenadiary.jp/entry/2021/04/26/084332} 
\end{itemize}

 上記2件のエントリーとして取り上げた被告発人岡田進弁護士の民事裁判がきっかけで,平成15年の記事を宇出津図書館の北國新聞縮小版で調べ始めたのがきっかけだったのですが,平成15年というのは新聞の購読をしておらず,外食の店で新聞を手にする程度であったとも思います。

 当時住んでいた羽咋市のアパートで新聞の購読をしていなかったことは明々白々なのですが,宇出津の実家でも母親が新聞の購読はしていなかったように思います。子供の頃から平成9年頃まではずっと新聞の購読をしていて家には決まって新聞があったと思います。

 「松任の娘殺害」も「珠洲市の保険金目的母親殺害事件」もテレビではニュースを見た記憶がないのですが,新聞では記事を読んでいたように思い出しました。

\begin{itemize}
\tightlist
\item
  〈〈〈 2021/04/30 18:26:33 Linux Emacs: 〈〈〈
\end{itemize}

\hypertarget{ux6d0bux5f13ux9283ux3092ux4f7fux3063ux305fux81eaux6bbaux306bux5931ux6557ux3057ux982dux90e8ux306bux77e2ux304cux523aux3055ux3063ux3066ux82e6ux3057ux3080ux5a18ux3092ux6bbaux5bb3ux3057ux305fux7236ux89aaux306bux969cux5bb3ux304cux6b8bux308bux3068ux601dux3063ux3066ux5a18ux3092ux6bbaux3057ux305fux306eux306fux6975ux3081ux3066ux8efdux7387ux3067ux77edux7d61ux7684ux3068ux61f2ux5f794ux5e746ux6708ux3068ux3057ux305fux4f0aux6771ux4e00ux5ee3ux88c1ux5224ux9577}{%
\paragraph{洋弓銃を使った自殺に失敗し頭部に矢が刺さって苦しむ娘を殺害した父親に,「障害が残ると思って娘を殺したのは,極めて軽率で短絡的」と懲役4年6月とした伊東一廣裁判長}\label{ux6d0bux5f13ux9283ux3092ux4f7fux3063ux305fux81eaux6bbaux306bux5931ux6557ux3057ux982dux90e8ux306bux77e2ux304cux523aux3055ux3063ux3066ux82e6ux3057ux3080ux5a18ux3092ux6bbaux5bb3ux3057ux305fux7236ux89aaux306bux969cux5bb3ux304cux6b8bux308bux3068ux601dux3063ux3066ux5a18ux3092ux6bbaux3057ux305fux306eux306fux6975ux3081ux3066ux8efdux7387ux3067ux77edux7d61ux7684ux3068ux61f2ux5f794ux5e746ux6708ux3068ux3057ux305fux4f0aux6771ux4e00ux5ee3ux88c1ux5224ux9577}}

\begin{itemize}
\tightlist
\item
  〉〉〉 Linux Emacs: 2021/04/30 18:31:58 〉〉〉
\end{itemize}

:CATEGORIES: @kanazawabengosi \#金沢弁護士会 @JFBAsns
日本弁護士連合会(日弁連) \#法務省 @MOJ\_HOUMU \#伊東一廣裁判長

 平成15年6月の北國新聞縮小版の記事には,「昨年十一月,自殺を図り死にきれなかった長女=当時(二五)=を刺殺したとして,殺人罪に問われた」とあります。「極めて軽率で短絡的だが,深く反省している」とし,懲役4年6月の判決ですが,求刑は懲役8年とあります。

 「公判では弁護側が「殺害前に娘が承諾していた」と殺人罪よりも刑の軽い承諾殺人罪であると主張。これに対して伊東裁判長は「娘は被告も自殺をすると誤信し殺害を依頼した。真意に基づく承諾とはいえない」と退けた,ともあります。

 続けて記事には,「同被告の弁護人は「主張は認められなかったが,情状面では考慮された」としている」とあります。

 記事には具体的な事実として,「判決によると,五日,被告の妻が自宅で自殺,これを知った娘も自殺を決意,同七日夜,娘は自宅で洋弓銃を使った自殺を試みたが失敗,被告は右頭部に矢が刺さって苦しむ娘の姿を見て「治療しても身体に障害が残ると思い,」などと記載があります。

 部分的に修正を加えていますが,記事には日付の特定と被告の実名があります。記事は北國新聞のテレビ欄をまくった紙面の右側にあります。通常,左側が事件や刑事裁判の報道に多いと思うのですが,その辺りも気になる報道のされかたでした。

 被告人の名前でネットで調べると,いくつか情報があるのですが,娘の殺害が7日午後7時頃,松任署への通報が10日午後5時30分とあります。妻は5日午後7時半頃,自分の胸を包丁で数回刺して自殺とあります。

 11日未明に嘱託殺人の疑いで逮捕とあり,「家族が死んでいる」と松任署に午後5時半頃に通報した後,取調べを受け,容疑が固まった段階で逮捕となってのでしょう。時間は午後8時23分ごろと違っていますが,未明に留置場で逮捕状を示されたのも,私の傷害・準強姦被告事件と似ている状況です。

 午後8時23分ごろというのは平成4年4月1日,私が金沢西警察署に出頭したという時刻で,この時刻はあとで谷内孝志警部補に聞かされました。この出頭にはためらいもあって,被害者安藤文さんが目が見えないと言い出したことで救急車を呼んでもらおうと,即座に金沢西警察署に向かったのでした。

 被害者安藤文さんは,そのあと金沢西警察署から県立中央病院に向かう救急車内で意識を失ったと聞きます。出頭していなければ殺人未遂になっていたと取調の時,谷内孝志警部補には言われていましたし,彼女が死亡していれば殺人罪になっていたかもしれない状況でした。

 私は被害者安藤文さんに「一緒に死ぬか?」とも声を掛けているのですが,彼女がはっきりと強い口調で「いや」と答え,すぐにはっきりと受け答えが出来る状態だったので,私は彼女に,被告発人松平日出男の電話番号を尋ねてもいるのです。その場の機転があったのか,目が見えないと言い出しました。

 金沢西警察署前に車を停めた時,車内に横たわる被害者安藤文さんに,谷内孝志警部補が誰に殴られたと尋ねたところ,彼女は私がいる方向を指差し,続けて殴られた理由を尋ねられたところ,「わからん」と答えたということです。

 これも取調の時,谷内孝志警部補に聞かされたような気もするのですが,被害者安藤文さんの意識が回復した後,谷内孝志警部補が病院で彼女本人に尋問をしたという書面もありました。事件のことは記憶にないあるいは思い出せないという内容であったと思います。

 これは金沢地方裁判所民事A係経由で,原告代理人であった被告発人長谷川紘之弁護士から提出され手元に来た裁判記録の一部にあったように思います。他の民事裁判のことは全くわからないのですが,長谷川紘之弁護士が写しを提出していたのか,かなりの量の記録が私に届きました。

 その書面は2つの綴で,高さはほぼ同じぐらい,脱衣カゴから3センチほど高さがはみ出していたように思います。病院の領収書が多かった他,医師による詳細な診断書も多数含まれ,膣内には精液より多量な体内細胞のようなものが検出されたという記載もあったと強く記憶に残っています。

 令和3年3月31日付告発状には,記述を逡巡したのですが,福井刑務所の満期出所に被告発人大網健二兄弟,関係者KYNの3人と一緒に事前の連絡もなく迎えに来た私の母親は,車が出発してすぐに,被害者安藤文さんが処女であったという話を始めました。これは被告発人安田繁克の供述調書とも矛盾します。

 被害者安藤文さんが処女だったという母親の話は,木梨松嗣弁護士が吹き込んだ可能性が高いと当初より考えていました。他の3人は示し合わせたように重い感じで無言,無反応という態度でした。

 金沢地方裁判所や金沢地方検察庁に提出した書面には記述をしているはずですが,ネットで公開する記述としては初めてのことになると思います。「自殺図った長女を殺した容疑で父親を逮捕・・・石川」という事件の被害者に哀悼の意を捧げるという気持もあって,一線を超えたことになります。

 同時に,伊東一廣裁判長の社会的責任を厳しく追求することも決意としていることを表明します。

\begin{itemize}
\tightlist
\item
  〈〈〈 2021/04/30 19:40:34 Linux Emacs: 〈〈〈
\end{itemize}

\hypertarget{ux77f3ux5dddux770cux8b66ux5bdf}{%
\subsection{石川県警察}\label{ux77f3ux5dddux770cux8b66ux5bdf}}

\hypertarget{ux73e0ux6d32ux8b66ux5bdfux7f72}{%
\subsubsection{珠洲警察署}\label{ux73e0ux6d32ux8b66ux5bdfux7f72}}

\hypertarget{ux30e2ux30c8ux30b1ux30f3ux3053ux3068ux77e2ux90e8ux5584ux6717ux5f01ux8b77ux58ebux4eacux90fdux5f01ux8b77ux58ebux4f1aux306bux5bfeux3059ux308bux540dux8a89ux6bc0ux640dux306eux5211ux4e8bux544aux8a3417ux65e5ux751fux6d3bux4fddux8b77ux62c5ux5f53ux8005ux306eux8a2aux554fux3068ux52a9ux8a00ux624bux52a9ux3051ux7121ux6148ux60b2ux306eux30c4ux30a4ux30fcux30c8}{%
\paragraph{モトケンこと矢部善朗弁護士(京都弁護士会)に対する名誉毀損の刑事告訴:17日生活保護担当者の訪問と,助言,手助け,無慈悲のツイート}\label{ux30e2ux30c8ux30b1ux30f3ux3053ux3068ux77e2ux90e8ux5584ux6717ux5f01ux8b77ux58ebux4eacux90fdux5f01ux8b77ux58ebux4f1aux306bux5bfeux3059ux308bux540dux8a89ux6bc0ux640dux306eux5211ux4e8bux544aux8a3417ux65e5ux751fux6d3bux4fddux8b77ux62c5ux5f53ux8005ux306eux8a2aux554fux3068ux52a9ux8a00ux624bux52a9ux3051ux7121ux6148ux60b2ux306eux30c4ux30a4ux30fcux30c8}}

\begin{itemize}
\tightlist
\item
  〉〉〉 Linux Emacs: 2021/05/18 09:25:39 〉〉〉
\end{itemize}

:CATEGORIES: @kanazawabengosi \#金沢弁護士会 @JFBAsns
日本弁護士連合会(日弁連) \#法務省 @MOJ\_HOUMU
\#モトケンこと矢部善朗弁護士(京都弁護士会)

\begin{itemize}
\item
  TW kk\_hirono(刑事告発・非常上告_金沢地方検察庁御中) 日時:
  2021-05-17 12:43 URL:
  \url{https://twitter.com/kk\_hirono/status/1394136374442348545} 
  \textgreater{} 法務省:検察の在り方検討会議 \url{https://t.co/xK5Zv7g9u2} 
  \textgreater{} 検察の再生に向けて【概要版】{[}PDF:156KB{]}\\
  \textgreater{} 検察の再生に向けて{[}PDF:573KB{]}
\item
  TW kk\_hirono(刑事告発・非常上告_金沢地方検察庁御中) 日時:
  2021-05-17 12:45 URL:
  \url{https://twitter.com/kk\_hirono/status/1394137017047388162} 
  \textgreater{}
  今まで考えたこともなかった資料がPDFファイルとして公開されているようです。これも面白いタイミングで,迷路のような紆余曲折を経た発見となりました。なにかの導きかもしれないので,これはおろそかに出来ないと思います。弁護士鉄道からの脱出に向けて。検察の在り方検討会議です。
\item
  TW kk\_hirono(刑事告発・非常上告_金沢地方検察庁御中) 日時:
  2021-05-17 12:51 URL:
  \url{https://twitter.com/kk\_hirono/status/1394138379399229442} 
  \textgreater{}
  検察の再生に向けて【概要版】{[}PDF:156KB{]},は5ページで,ちょっと安心したのですが,検察の再生に向けて{[}PDF:573KB{]}は不意をつかれ現実に引き戻されたように42ページとなっていました。
\item
  TW kk\_hirono(刑事告発・非常上告_金沢地方検察庁御中) 日時:
  2021-05-17 12:53 URL:
  \url{https://twitter.com/kk\_hirono/status/1394139007919943683} 
  \textgreater{}
  気がつくまで時間がかかったのですが,2つの異なる文書と思っていたのが,同じ「検察の再生に向けて」と題する文書で,概要版と全文に分かれているようです。概要版というのもおそらく初めて目にした言葉に思えます。要約版ならあったように思います。
\item
  TW kk\_hirono(刑事告発・非常上告_金沢地方検察庁御中) 日時:
  2021-05-17 12:55 URL:
  \url{https://twitter.com/kk\_hirono/status/1394139566303367179} 
  \textgreater{}
  全文と概要,あるいは要約という取り扱いの違いは,私が長く頭を悩ませてきた告発状作成での問題で,大きな時間のロスにもなっていました。そういう意味でもこの2つのPDFファイルは参考になりそうです。
\item
  TW kk\_hirono(刑事告発・非常上告_金沢地方検察庁御中) 日時:
  2021-05-17 14:22 URL:
  \url{https://twitter.com/kk\_hirono/status/1394161362809491458} 
  \textgreater{}
  時刻は14時19分です。輪島から生活保護の担当者の訪問があり,銀行に通帳の記帳に行ったり,書面に記入したりで時間がかかっていました。5分程前に帰られたところです。丁度,通帳が繰越になるというタイミングでもあり,余計に時間がかかりました。
\end{itemize}

 上記の2021-05-17
12:55の直後に輪島から生活保護担当者の訪問があったと思うのですが,次の2021-05-17
14:22のツイートにある通り5分程前に帰られたとあります。5分前というのは14時17分頃です。

 途中で車に乗せてもらい銀行へ通帳の記帳に行ったということもあったのですが,ツイートの記録をみると1時間15分ほどの滞在時間があったようです。これまではほとんどが顔見せ程度で,半年ぐらい訪問がなかったこともあったように思います。

 4月に代わったという担当者で2回ほど電話で話していたのですが,1回は私から用事で掛けた電話でした。もう1回は掛かってきた電話だったかもしれないですが,よく憶えていません。

 前年度の担当者は比較的若い感じでしたが,訪問があったのは3,4回,多くて4,5回だと思います。いずれも短い時間の訪問でした。その前の担当者の方が,訪問の回数も多かったように思いますが,事前に訪問の時間を決めることもありました。他は全部,ほとんどがいきなりの訪問です。

 金沢地方検察庁や石川県警察に刑事告発や再捜査の手続きを進めていることは,ずっと前から担当者に話しています。いつ頃から話すようになったのか憶えていませんが,長い間,ほとんど連絡のない時期というのもありました。

 一つの節目は,2013年か2014年だったと思うのですが,それまで近所の同じ小棚木の町内のおばさんが,担当の民生委員として月に一度は顔を出していました。私はほとんど知らない人だったのですが,向こうの人は私の母親のことなどもよく知っている様子でした。

 それが2013年か2014年頃になって,民生委員の担当者がかわる今度は男の人,とその民生委員のおばさんに言われたのですが,それっきり一度も民生委員の訪問というのはなく,名前も何も知らないでいます。

 地元では生活保護を受けることを「民生委員にかかる」ともいうらしいのですが,ネットでも生活保護の話題はよく見かけるものの,民生委員というのは見かけておらず,生活保護との関わりを今から少し調べてみたいと思います。

\begin{quote}
《引用の始まり》
\end{quote}

\begin{quote}
ベストアンサーseka*******seka*******さん

2020/6/4 21:03

民生委員の仕事は主に「障害者世帯」「母子家庭世帯」「高齢者世帯」「生活保護世帯」の「相談」ですね。

あとはケースワーカーから依頼があれば「生活保護不正受給者の身辺調査」も行っていますよ
\end{quote}

\begin{quote}
《引用の終わり》
\end{quote}

\begin{itemize}
\tightlist
\item
  生活保護受給者です。民生委員さんって何ですか。生活保護受給者のために何\ldots{}
  -
  Yahoo!知恵袋 \url{https://detail.chiebukuro.yahoo.co.jp/qa/question\_detail/q10226242445?\_\_ysp=55Sf5rS75L\%2Bd6K235Y\%2BX57Wm6ICFLOawkeeUn\%2BWnlOWToQ\%3D\%3Dn} 
\end{itemize}

 昭和の時代なので今とは制度が違うのかもしれないですが,私は3歳ぐらいで父親を亡くし他に兄弟もなく母子家庭で育っていますが,民生委員のような人の訪問を受けた記憶はなく,事務的な話を聞かれたという憶えもありません。

\begin{itemize}
\item
  生活保護のケースワーカーおよび民生委員について \textbar{} 函館市
  \url{https://www.city.hakodate.hokkaido.jp/citizensvoice/docs/2014031100125/} 
\item
  民生委員・児童委員について |厚生労働省 \url{https://t.co/TxffLecCPc} 
  民生委員制度は、1917(大正
  6)年に岡山県で誕生した「済世顧問制度」を始まりとします。翌1918(大正
  7)年には大阪府で「方面委員制度」が発足し、1928(昭和
  3)年には方面委員制度が全国に普及しました。
\end{itemize}

 昨日は,初対面ということもありましたが,ずいぶんのんびりした様子で,会話が弾みました。ケースワーカーと呼ばれることも確認しましたが,県の職員で,国から委託を受けたかたち,生活保護費のほとんどは国から出ているという話を聞きました。

 前の前の担当者にも質問して聞いていたこともあるのですが,より踏み込んだかたちで確認をしました。その上で,県や国に賠償を求めていくかたちで刑事告発の手続きを進め,これまで準備をしてきたことを話,後には弁護士らに対する求償権や,懲罰的制裁金の話もしました。

 初めの方に金沢地方検察庁からの郵送物と中にあった1枚の書面を見せました。まだ具体的に説明はしていないですが,4月15日に能都郵便局に行って受け取ってきた郵送物です。

 さきほどtun kk\_hirono
144 というコマンドを実行して,昨日のツイートの流れをみたのですが,輪島の生活保護担当者の訪問がある直前に,法務省:検察の在り方検討会議のPDFファイルを見つけダウンロードしていたことはすっかり忘れていました。

 そのタイミングを忘れていただけで,法務省:検察の在り方検討会議のPDFファイルのことはずっと頭にあり,夜のうちに12ページ目まで読んでいます。42ページまであるのでまだ先が長いですが,割と読みやすく,同じような内容が繰り返さえています。

 昨夜も眠った時間がはっきりしなかったのですが,6時過ぎに目が覚めました。前回と同じようにLEDの照明スタンドの強い光が頭の上にある状態で割と長い時,間熟睡していました。ツイートを確認すると寝る前最後のツイートが0時6分となっていました。

 前回と同じように夢を見ながら目が覚めたのですが,今回は長男と別れて数年後に長男が事故で死んだと人に聞かされる夢でした。交通事故ではなく遊んでいてなにかにぶつかったという話であったと思います。

 そんな夢をみてよい目覚めではなかったのですが,しばらくしてそのまま眠り,次に起きたのが8時30分頃だったと思います。多分その後になると思いますが,朝一番で,とても印象的に思えたのが,ここでメインに取り上げる次のモトケンこと矢部善朗弁護士(京都弁護士会)のツイートになります。

※ @kk\_hironoのアカウントがブロックされ,リツイートに失敗したツイート

\begin{itemize}
\tightlist
\item
  TW motoken\_tw(モトケン) 日時:2021/05/17 19:00:06 URL:
  \url{https://twitter.com/motoken\_tw/status/1394231243407953921} 
  \textgreater{}
  この人は、頼まれないと何の助言も手助けもしないみたいだな。\\
  \textgreater{} かなり無慈悲、と言うこともできる。
  \url{https://t.co/31zk56a4tN} 
\end{itemize}

 ちょうどパソコンのウィンドウをモトケンこと矢部善朗弁護士(京都弁護士会)のタイムラインを表示したブラウザに切り替えたタイミングで,ツイートの時刻が14時間から15時間に変わりました。

 APIで取得したツイートをみると昨夜の19時丁度,00分06秒という投稿時刻でした。昨夜も一度は目にしていたツイートなのですが,そのときはさほど気に留めていなかったように思います。

\begin{itemize}
\tightlist
\item
  2021年05月17日06時46分の登録:
  ツイートの記録資料:\法務検察・石川県警察宛\/モトケン(@motoken\_tw)/''2021年05月16日'':12件
  \url{https://kk2020-09.blogspot.com/2021/05/motokentw2021051612.html} 
\item
  2021年05月17日09時58分の登録:
  REGEXP:''再審''/モトケン(@motoken\_tw)の検索(2010-04-17〜2020-11-06/2021年05月17日09時58分の記録51件)
  \url{https://kk2020-09.blogspot.com/2021/05/regexpmotokentw2010-04-172020-11.html} 
\item
  2021年05月17日14時26分の登録:
  \モトケン @motoken\_tw\>一刻も早くストップすべき。具体的には、ツイッタールール(攻撃的な行為/嫌がらせの禁止)に違反していればツイッターの運営がロックまたは
  \url{https://kk2020-09.blogspot.com/2021/05/motokentw\_17.html} 
\item
  2021年05月17日16時56分の登録:
  \モトケン @motoken\_tw\\#石川優実さんへの誹謗中傷をやめろ
  私は誹謗中傷してくるアカウントはブロックするけど、それではあかんのかな?
  \url{https://kk2020-09.blogspot.com/2021/05/motokentw\_83.html} 
\item
  2021年05月17日20時55分の登録:
  \モトケン @motoken\_tw\この人は、頼まれないと何の助言も手助けもしないみたいだな。かなり無慈悲、と言うこともできる。
  \url{https://kk2020-09.blogspot.com/2021/05/motokentw\_45.html} 
\item
  2021年05月18日06時16分の登録:
  ツイートの記録資料:\法務検察・石川県警察宛\/モトケン(@motoken\_tw)/''2021年05月17日'':15件
  \url{https://kk2020-09.blogspot.com/2021/05/motokentw2021051715.html} 
\item
  2021年05月18日06時20分の登録:
  \モトケン @motoken\_tw\少なくとも2000年当時から減り続けているようですけど、新型コロナはいつごろから問題になりましたっけ?常識的に考えて、人は不要なものに
  \url{https://kk2020-09.blogspot.com/2021/05/motokentw\_18.html} 
\item
  2021年05月18日06時24分の登録:
  \モトケン @motoken\_tw\この人は、頼まれないと何の助言も手助けもしないみたいだな。かなり無慈悲、と言うこともできる。
  \url{https://kk2020-09.blogspot.com/2021/05/motokentw\_96.html} 
\end{itemize}

 昨日17日から本日18日にかけデータベースに記録したモトケンこと矢部善朗弁護士(京都弁護士会)のツイートです。さほど気に留めなかったと思いながら,2021年05月17日20時55分の登録:という記録があり,2021年05月18日06時24分の登録:とたぶっています。

 6時過ぎに起きてからは夢の余韻のような状態で,何もせず,そのうち眠りについたと思っていたのですが,06時24分には再び,件のモトケンこと矢部善朗弁護士(京都弁護士会)のツイートを見ていたようです。

 慈悲という言葉も久しぶりに見かけたと思ったのですが,考え事をしながらしばらくして,もう一度モトケンこと矢部善朗弁護士(京都弁護士会)のツイートを読み直すと,無慈悲となっていました。無慈悲はあまりみかけなかったように思うので,3つの感じの並びが新鮮に思えます。

 昨日,エントリーとして取り上げたモトケンこと矢部善朗弁護士(京都弁護士会)のツイートもそうだったのですが,「この人は、頼まれないと何の助言も手助けもしないみたいだな。かなり無慈悲、と言うこともできる。」という文章は,日本語として変わった言葉の並びになっています。

 「かなり無慈悲、と言うこともできる。」という言葉の響きが,そのままモトケンこと矢部善朗弁護士(京都弁護士会)に受けた仕打ち,私の母親と二人の子供の人生に与えた悪影響に思いを馳せ,その諸行無常の結果が現在にあり,未来に向けて厄災除去する必要を改めて感じました。

 最初は「無慈悲」にだけ意識が集中していたのですが,「助言」「手助け」というのも宗教的なものを感じました。観世音菩薩や地蔵菩薩に大乗される一般的な宗教性と,モトケンこと矢部善朗弁護士(京都弁護士会)独自のカルト的宗教性との対比,コントラストになります。

 もともと仏教には伐折羅大将に代表される暴力的攻撃性があると考えているのですが,その攻撃性と自分が正しいと信じて疑わない独善性の強さを,これまでになくくっきり鮮やかにみえたのが,次のモトケンこと矢部善朗弁護士(京都弁護士会)のツイートになります。

\begin{itemize}
\tightlist
\item
  TW motoken\_tw(モトケン) 日時: 2021/05/18 08:27:18 URL:
  \url{https://twitter.com/motoken\_tw/status/1394434381708554241} 
  \textgreater{} @abengers2
  私は、複数の反対派から集中的に罵詈雑言、誹謗中傷、人格攻撃を浴びました。\\
  \textgreater{} 印象ではなく、事実。
\end{itemize}

 最近は余りみかけないですが,印象論というのもモトケンこと矢部善朗弁護士(京都弁護士会)がある種の妖術のように使ってきた呪文のような言葉の1つです。「印象ではなく、事実。」という組み合わせがとても新鮮ですが,事実というのもこの人物には珍しく映りました。

 おそらく私の意見や刑事告訴のことも,「罵詈雑言、誹謗中傷、人格攻撃を浴びました。」というつもりでいるのでしょう。周囲がそのように理解することを期待しているのかもしれません。

\begin{itemize}
\tightlist
\item
  〈〈〈 2021/05/18 11:25:56 Linux Emacs: 〈〈〈
\end{itemize}

\hypertarget{ux30e2ux30c8ux30b1ux30f3ux3053ux3068ux77e2ux90e8ux5584ux6717ux5f01ux8b77ux58ebux4eacux90fdux5f01ux8b77ux58ebux4f1aux306bux5bfeux3059ux308bux540dux8a89ux6bc0ux640dux306eux5211ux4e8bux544aux8a34ux518dux5be9ux6cd5ux3068ux5730ux65b9ux81eaux6cbbux4f53ux672cux4ef6ux544aux767aux4e8bux4ef6ux304cux80fdux767bux753aux306bux4e0eux3048ux308bux5f71ux97ff}{%
\paragraph{モトケンこと矢部善朗弁護士(京都弁護士会)に対する名誉毀損の刑事告訴:再審法と地方自治体,本件告発事件が能登町に与える影響}\label{ux30e2ux30c8ux30b1ux30f3ux3053ux3068ux77e2ux90e8ux5584ux6717ux5f01ux8b77ux58ebux4eacux90fdux5f01ux8b77ux58ebux4f1aux306bux5bfeux3059ux308bux540dux8a89ux6bc0ux640dux306eux5211ux4e8bux544aux8a34ux518dux5be9ux6cd5ux3068ux5730ux65b9ux81eaux6cbbux4f53ux672cux4ef6ux544aux767aux4e8bux4ef6ux304cux80fdux767bux753aux306bux4e0eux3048ux308bux5f71ux97ff}}

\begin{itemize}
\tightlist
\item
  〉〉〉 Linux Emacs: 2021/05/18 11:34:40 〉〉〉
\end{itemize}

:CATEGORIES: @kanazawabengosi \#金沢弁護士会 @JFBAsns
日本弁護士連合会(日弁連) \#法務省 @MOJ\_HOUMU \#再審請求 \#地方自治体
\#モトケンこと矢部善朗弁護士(京都弁護士会)

 見出しのレベル2が石川県警察,レベル3が珠洲警察署となっていますが,「モトケンこと矢部善朗弁護士(京都弁護士会)に対する名誉毀損の刑事告訴:」という接頭辞でシリーズ化しています。テーマの主役,主人公はモトケンこと矢部善朗弁護士(京都弁護士会)です。

 まず,ku3で奉納\さらば弁護士鉄道・泥棒神社の物語(@hirono\_hideki)のまとめを作成しました。最新3,228件のツイートを取得しています。射程の長さとしては十分すぎますが,昨日のツイートの流れを追うのが目的です。

\begin{itemize}
\item
  2021年05月18日11時30分の登録:
  @hirono\_hideki(奉納\さらば弁護士鉄道・泥棒神社の物語)のツイート ''.*'' 3228/3228:2021-05-01\_1311〜2021-05-18\_1126 2021年05月18日11時30分の記録
  \url{https://kk2020-09.blogspot.com/2021/05/hironohideki322832282021-05-0113112021.html} 
\item
  (125/3228)
  @hirono\_hideki(奉納\さらば弁護士鉄道・泥棒神社の物語)のツイート ''.*'' 3228/3228:2021-05-01\_1311〜2021-05-18\_1126
  2021年05月18日11時30分の記録\\
  RT
  hirono\_hideki(奉納\さらば弁護士鉄道・泥棒神社の物語)|nagahito(ながひと)
  日時:2021-05-17 17:24/2021-05-17 12:31 URL:
  \url{https://twitter.com/hirono\_hideki/status/1394207089661071361} 
  \url{https://twitter.com/nagahito/status/1394133449963806722} 
  \textgreater{}
  「再審法改正を」地方動く 49市町村議会、証拠開示など求め意見書
  \url{https://t.co/4NXqyPFI7f}  \#西日本新聞
\item
  わやくその意味や使い方 Weblio辞書 \url{https://t.co/VHsqbOmSxr} 
\end{itemize}

 いきなりですが,ふと,わやくそ,という言葉を思い出したので調べてみました。鳥取弁や岡山弁もみえますが,関西弁というような情報もみえます。昭和の時代,能登の宇出津でよく聞く言葉の1つだったのですが,ふと思い出し,ずいぶん長い間,耳にしていなかったように思いました。

 深層心理から浮かび上がってきたような「わやくそ」という言葉ですが,滅茶苦茶という意味でありながら,風土に溶け込んだ独自のニュアンスの方言になるかと思います。調べてみると,能登より西日本で多いようですが,関東や東北,九州は検索結果の1ページ目に見当たりません。

\begin{itemize}
\tightlist
\item
  広島弁「わや・わやくそ」の意味と使い方事例 - 広島弁一覧表
  \url{https://t.co/zlohe96G8q} 
\end{itemize}

 広島弁というのも意外でしたが,めちゃくちゃ,の他に,むちゃくちゃ,とも紹介されています。無茶苦茶と日本語変換してみましたが,滅茶苦茶と同じように見えます。お茶の茶の字が使われているのも気になるところではありますが,まさにその状態と思っていたのが再審法です。

 再審法改正の方が出現頻度は多いかもしれないですが,そもそも再審法という法律など存在しないはずで,刑事訴訟法にある再審がそれにあたるのかと思いますが,条文の存在を知らない人も多いハズで,再審法という法律が存在すると誤解を与えそうな,いかさま商法を感じさせます。

\begin{itemize}
\tightlist
\item
  日本弁護士連合会:えん罪被害者を一刻も早く救済するために再審法の速やかな改正を求める決議
  \url{https://t.co/yyepOEKn1R} 
  ところが、現行の再審法(刑事訴訟法第4編再審)の規定は、わずか19条しか存在せず、裁判所の裁量に委ねられている点が非常に多いことから、その判断の公正さや適正さが制度的
\end{itemize}

 2019年(令和元年)10月4日という日付のある日弁連の会長声明かと思ったのですが,提案理由という続きがありました。また,これまでみてきた会長声明だと金沢弁護士会などの単位会の場合もそうですが,会長の弁護士の名前があるところ,日本弁護士連合会のみとなっています。

 昨日の,法務省:検察の在り方検討会議に続く発見ですが,提案理由の第一が,「「国家による最大の人権侵害」であるえん罪被害と再審法の意義」という項目名となっています。「全部,警察と検察のせいにする弁護士宣言」という但し書きでもつけてもらいたいところです。

 リツイートの時刻で2021-05-17
17:24頃の発見ですが,これはTwitter検索を実行し出てきたツイートをリツイートしたものです。最初は,ふと思い出して,twitterAPI-search-lawList-mydql-add.rb
を実行したように思います。

 これから「再審法」でtwitterAPI-search-lawList-mydql-add.rb
を実行しますが,リストに登録したアカウントのツイート・リツイートは,コマンドを実行した端末で次のように表示されていきます。

\begin{quote}
《引用の始まり》
\end{quote}

\begin{quote}
10TW kyuuen3tama(国民救援会三多摩総支部事務局)2021/05/18 10:31:33
\url{https://twitter.com/kyuuen3tama/status/1394465651197059076} 「再審法改正を」地方動く 49市町村議会、証拠開示など求め意見書
\url{https://t.co/wQRbUalh0z}  \#西日本新聞
\end{quote}

\begin{quote}
《引用の終わり》
\end{quote}

 最初の検索は「再審」だけだったと思いますが,そのときも同じような「「再審法改正を」地方動く 49市町村議会、証拠開示など求め意見書」を見かけたはずで,これが記事のタイトルと思いますが,\#西日本新聞 とハッシュタグをつけてあります。

 この前調べたところ,twitterAPI-search-lawList-mydql-add.rb
で実行するTwitterAPIの検索では,最大で7日か8日前のツイートまでしか取得できなかったように思いますが,記録としてのデータの追加の他にも,最新の動向を確認するには手っ取り早い手段でもあります。

 一度のコマンドの実行で直近7,8日程度のツイートしか補足できないので,例えば3週間後に同じ検索を実行しても一週間のブランクが出来てしまいます。そういうこともあって,しばらくやっていないと思ったのが「再審」の検索でした。

2021年05月18日12時31分の実行記録: twitterAPI-search-lawList-mydql-add.rb
``再審法'' ツイート数:24/2419 リツイート数:11/2419
トータル:100``再審法''の該当: hirono\_hideki 5/2件 kk\_hirono 10/0件
s\_hirono 1/0件

 トータルで100件という結果でした。もとよりTwitterAPIだけでなく,Twitterの検索自体に精度の低さの問題があることは,自分の経験から度々,指摘をしてきました。なぜか全くヒットしない検索もあります。

2021年05月18日12時40分の実行記録: twitterAPI-search-lawList-mydql-add.rb
``大津 控訴'' ツイート数:0/2419 リツイート数:0/2419 トータル:0``大津
控訴''の該当: hirono\_hideki 0/0件 kk\_hirono 0/0件 s\_hirono 0/0件

 直前に,次のツイートをしているのですが,それでも結果は0です。これはずっと同じです。'大津
控訴'だったのを``大津 控訴''とやっても同じでした。

 引数の記号の組み合わせで完全一致の検索も出来るのですが,「 ``大津
控訴''
」というように半角スペースで区切るだけだと,曖昧検索になるのがTwitterの仕様となっています。例えば「控訴審で特定された茨城県大津港の貨物車」という文章でも,本来はヒットするはずなのです。

\begin{itemize}
\tightlist
\item
  TW hirono\_hideki(奉納\さらば弁護士鉄道・泥棒神社の物語) 日時:
  2021-05-18 12:40 URL:
  \url{https://twitter.com/hirono\_hideki/status/1394498005055639555} 
  \textgreater{} 2021年05月18日12時39分の実行記録:
  twitterAPI-search-lawList-mydql-add.rb ``大津 控訴''
  ツイート数:0/2419 リツイート数:0/2419 トータル:0\\
  \textgreater{} ``大津 控訴''の該当: hirono\_hideki 0/0件 kk\_hirono
  0/0件 s\_hirono 0/0件
\end{itemize}

〉〉〉 kk\_hironoのリツイート 〉〉〉

\begin{itemize}
\tightlist
\item
  RT
  kk\_hirono(刑事告発・非常上告_金沢地方検察庁御中)|MichikoKameishi(弁護士
  亀石倫子) 日時:2021-05-18 12:49/2020/04/02 20:00 URL:
  \url{https://twitter.com/kk\_hirono/status/1394500386329808899} 
  \url{https://twitter.com/MichikoKameishi/status/1245667410515865600} 
  \textgreater{}
  大津地検が控訴する権利を放棄し、西山美香さんの無罪が確定。
  あたりまえだよ。 美香さんに謝罪するべき。 そして国家賠償を。
  \url{https://t.co/CY6sekG7Bh} 
\end{itemize}

〉〉〉 kk\_hironoのリツイート 〉〉〉

\begin{itemize}
\tightlist
\item
  RT
  kk\_hirono(刑事告発・非常上告_金沢地方検察庁御中)|makky\_0528(篠原真紀)
  日時:2021-05-18 12:49/2020/02/26 13:47 URL:
  \url{https://twitter.com/kk\_hirono/status/1394500465354702848} 
  \url{https://twitter.com/makky\_0528/status/1232527654688980993} 
  \textgreater{} 明日14時判決です。 固唾を飲んで待ちます。
  「子供の命守る後押しを」大津いじめ訴訟、27日に控訴審判決(産経新聞)
  \url{https://t.co/iiMOFqXhZY} 
\end{itemize}

〉〉〉 kk\_hironoのリツイート 〉〉〉

\begin{itemize}
\tightlist
\item
  RT
  kk\_hirono(刑事告発・非常上告_金沢地方検察庁御中)|mainichi(毎日新聞)
  日時:2021-05-18 12:50/2020/04/16 07:40 URL:
  \url{https://twitter.com/kk\_hirono/status/1394500698264379396} 
  \url{https://twitter.com/mainichi/status/1250554493965459456} 
  \textgreater{}
  大津園児事故、被告が控訴取り下げ 禁錮4年6月の1審判決が確定
  \url{https://t.co/ApTAeLi5at} 
\end{itemize}

〉〉〉 kk\_hironoのリツイート 〉〉〉

\begin{itemize}
\item
  RT
  kk\_hirono(刑事告発・非常上告_金沢地方検察庁御中)|nhk\_news(NHKニュース)
  日時:2021-05-18 12:50/2020/04/15 18:31 URL:
  \url{https://twitter.com/kk\_hirono/status/1394500710704635904} 
  \url{https://twitter.com/nhk\_news/status/1250355949014925314} 
  \textgreater{} 滋賀 大津 園児死傷事故 被告の実刑確定 控訴取り下げ
  \#nhk\_news \url{https://t.co/T523WjWZv2} 
\item
  \begin{enumerate}
  \def\labelenumi{(\arabic{enumi})}
  \setcounter{enumi}{3}
  \tightlist
  \item
    大津 控訴 - Twitter検索 / Twitter \url{https://t.co/oqPdCA1vim} 
  \end{enumerate}
\end{itemize}

 APIではないブラウザでの通常のTwitter検索だと,「大津
控訴」のヒットが多数あります。

 少し前に次の検索を行っていました。これもちょっと不思議な発見だったのですが,一緒に記録しておきたいと思います。

\begin{itemize}
\tightlist
\item
  (127/3228)
  @hirono\_hideki(奉納\さらば弁護士鉄道・泥棒神社の物語)のツイート ''.*'' 3228/3228:2021-05-01\_1311〜2021-05-18\_1126
  2021年05月18日11時30分の記録\\
  TW hirono\_hideki(奉納\さらば弁護士鉄道・泥棒神社の物語) 日時:
  2021-05-17 17:15 URL:
  \url{https://twitter.com/hirono\_hideki/status/1394205034825482241} 
  \textgreater{} - 黒歴史クリーナー \url{https://t.co/sH631TpRJn} 
\end{itemize}

 検索ではなくTwitterのアプリのページのリンクでした。

〉〉〉 kk\_hironoのリツイート 〉〉〉

\begin{itemize}
\tightlist
\item
  RT
  kk\_hirono(刑事告発・非常上告_金沢地方検察庁御中)|s\_hirono(非常上告-最高検察庁御中\_ツイッター)
  日時:2021-05-18 12:57/2021/05/18 06:26 URL:
  \url{https://twitter.com/kk\_hirono/status/1394502458844487682} 
  \url{https://twitter.com/s\_hirono/status/1394404077220503553} 
  \textgreater{}
  2021-05-17-214800\_DUKEまんごう@nan5o若手匿名果物ですの(╹◡╹)♡ 。わたしのせなかにたたないでー。日本ねむねむ協会会員6号ここ2010年2月からT.jpg
  \url{https://t.co/ueSjA3SDuZ} 
\end{itemize}

〉〉〉 kk\_hironoのリツイート 〉〉〉

\begin{itemize}
\tightlist
\item
  RT
  kk\_hirono(刑事告発・非常上告_金沢地方検察庁御中)|s\_hirono(非常上告-最高検察庁御中\_ツイッター)
  日時:2021-05-18 12:58/2021/05/17 19:32 URL:
  \url{https://twitter.com/kk\_hirono/status/1394502624578215941} 
  \url{https://twitter.com/s\_hirono/status/1394239373390934021} 
  \textgreater{}
  2021-05-17-171529\_DUKEまんごう@nan5o@nan5o ツイ消し中 (ꈨຶꎁꈨຶ)۶'' ꁖʓ╵᷅ժ ̀꒭〜 | 呪われしツイートを一括削除! 黒歴史クリー.jpg
  \url{https://t.co/iIq4iJjMnp} 
\end{itemize}

〉〉〉 kk\_hironoのリツイート 〉〉〉

\begin{itemize}
\tightlist
\item
  RT
  kk\_hirono(刑事告発・非常上告_金沢地方検察庁御中)|s\_hirono(非常上告-最高検察庁御中\_ツイッター)
  日時:2021-05-18 12:58/2021/05/17 19:32 URL:
  \url{https://twitter.com/kk\_hirono/status/1394502660384911363} 
  \url{https://twitter.com/s\_hirono/status/1394239300598865920} 
  \textgreater{}
  2021-05-17-171448\_- (434/434) @nan5o(DUKEまんごう)のツイート ''.*'' 434/434:2021-05-06\_1006〜2021-05.jpg
  \url{https://t.co/ySW5AGcIkW} 
\end{itemize}

 @nan5o(DUKEまんごう)というアカウントで,プロフィールの名前はけっこう変更されているのですが,「まんごう」が基本形かと思います。他には見たことがないのですが,定期的に全てのツイートを削除しているアカウントでした。

\begin{itemize}
\tightlist
\item
  2021年05月18日13時01分の登録:
  「@nan5o」を@hirono\_hideki @kk\_hirono @s\_hironoで検索 227件の該当 2021-05-18\_13:01の記録
  \url{https://kk2020-09.blogspot.com/2021/05/nan5ohironohidekikkhironoshirono2272021.html} 
\end{itemize}

〉〉〉 kk\_hironoのリツイート 〉〉〉

\begin{itemize}
\tightlist
\item
  RT
  kk\_hirono(刑事告発・非常上告_金沢地方検察庁御中)|Toshimitsu\_Dan(だんどじみつ)
  日時:2021-05-18 13:04/2012/04/26 10:48 URL:
  \url{https://twitter.com/kk\_hirono/status/1394504022262894595} 
  \url{https://twitter.com/Toshimitsu\_Dan/status/195328452671442944} 
  \textgreater{}
  無罪とは、心が折れなかった人にたいする神様のご褒美なのです。RT
  @1977taku: @nan5o
  :すごいです。僕も無罪主張したことがありますが・・・あれ,雨がふってきたのかな,頬がぬれてる。
\end{itemize}

2012-04-26 12:50:56 ``RT @Toshimitsu\_Dan:
無罪とは、心が折れなかった人にたいする神様のご褒美なのです。RT
@1977taku: @nan5o
:すごいです。僕も無罪主張したことがありますが・・・あれ,雨がふってきたのかな,頬がぬれてる。''
\url{https://twitter.com/hirono\_hideki/status/195359281464352769} 

2016-05-30 09:03:10
``2016-05-30-090305\_うの字@un\_co\_the2nd@nan5o 奴が捕捉した気になってるツイートを見るにつけセンスがない.jpg
\url{http://pic.twitter.com/YVWGfwTB0w''} 
\url{https://twitter.com/s\_hirono/status/737071819002286080} 

 ようやく見つけましたが,2015年より前と思っていたのが,2016年5月30日でした。しかし,これはスクリーンショットの記録の日時かもしれず,確認が必要です。スクリーンショットであればうの字のツイートも見れるかもしれません。

 上記のツイートは失敗作で,次のツイートにあるスクリーンショットで,うの字のツイートが2016年ということが確認できました。秘本語の環境設定をしていなかったのかと思ったのですが,ブロックされているためログアウトした状態で開いたTwitterだったようです。

 うの字に対する返信で「@un\_co\_the2nd
二つほど速攻でブロックしてます(-\_-)」というツイートをしていたアカウントですが,返信をしたうの字のツイートをみれば,私を名指ししていることが明らかです。

\begin{lstlisting}
py37_env ❯ twilog-serch  敢えてブロッコしないけど
- ./hirono_hideki2021-05-18_130147.csv:2020-09-28 14:03:34 "RT @s_hirono: 2016-05-30-090346_うの字@un_co_the2ndと@nan5o 敢えてブロッコしないけど、金澤p庁御中うっとおしいわ・・・p庁がそんなもんに目をかけると思ってんのか.jpg \url{http://pic.twitter.com/18bZE5TiZs}  " \url{https://twitter.com/hirono_hideki/status/1310445020818759680} 
- ./hirono_hideki2021-05-18_130147.csv:2020-09-28 14:02:27 "RT @hirono_hideki: 奉納\弁護士妖怪大泥棒神社・金沢地方検察庁御中: @un_co_the2nd(うの字)のツイート→ ”敢えてブロッコしないけど、金澤p庁御中うっとおしいわ・・・p庁がそんなもんに目をかけると思ってんのか” \url{http://hirono2016k.blogspot.com/2016/05/uncothe2ndnan5o-p.html?spref=tw}  " \url{https://twitter.com/hirono_hideki/status/1310444739217379328} 
- ./hirono_hideki2021-05-18_130147.csv:2020-09-28 14:01:55 "RT @hirono_hideki: @un_co_the2nd(うの字)のツイート→ ”敢えてブロッコしないけど、金澤p庁御中うっとおしいわ・・・p庁がそんなもんに目をかけると思ってんのか” \url{http://ow.ly/wjU7307TaCV}  " \url{https://twitter.com/hirono_hideki/status/1310444605398163456} 
- ./hirono_hideki2021-05-18_130147.csv:2017-01-11 16:16:02 "@un_co_the2nd(うの字)のツイート→ ”敢えてブロッコしないけど、金澤p庁御中うっとおしいわ・・・p庁がそんなもんに目をかけると思ってんのか” \url{http://ow.ly/wjU7307TaCV}  " \url{https://twitter.com/hirono_hideki/status/819080410365853696} 
- ./hirono_hideki2021-05-18_130147.csv:2016-05-30 09:14:17 "奉納\弁護士妖怪大泥棒神社・金沢地方検察庁御中: @un_co_the2nd(うの字)のツイート→ ”敢えてブロッコしないけど、金澤p庁御中うっとおしいわ・・・p庁がそんなもんに目をかけると思ってんのか” \url{http://hirono2016k.blogspot.com/2016/05/uncothe2ndnan5o-p.html?spref=tw}  " \url{https://twitter.com/hirono_hideki/status/737074618322669568} 
- ./kk_hirono2021-05-18_130904.csv:2017-07-22 19:35:29 "うの字: 敢えてブロッコしないけど、金澤p庁御中うっとおしいわ・・・p庁がそんなもんに目をかけると思ってんのか " \url{https://twitter.com/kk_hirono/status/888709076787412992} 
- ./kk_hirono2021-05-18_130904.csv:2017-07-22 19:34:59 "##### 「敢えてブロッコしないけど、金澤p庁御中うっとおしいわ・・・p庁がそんなもんに目をかけると思ってんのか」という、うの字のツイート " \url{https://twitter.com/kk_hirono/status/888708948492079104} 
- ./kk_hirono2021-05-18_130904.csv:2017-07-22 19:28:13 "2016-05-30-090346_うの字@un_co_the2ndと@nan5o 敢えてブロッコしないけど、金澤p庁御中うっとおしいわ・・・p庁がそんなもんに目をかけると思ってんのか.jpg " \url{https://twitter.com/kk_hirono/status/888707245478891520} 
- ./kk_hirono2021-05-18_130904.csv:2017-05-10 21:24:01 "&gt; 2016-05-30-090346_うの字@un_co_the2ndと@nan5o 敢えてブロッコしないけど、金澤p庁御中うっとおしいわ・・・p庁がそんなもんに目をかけると思ってんのか.jpg \url{http://pic.twitter.com/UAHOGzXAiY}  " \url{https://twitter.com/kk_hirono/status/862282075801395200} 
- ./kk_hirono2021-05-18_130904.csv:2017-05-10 20:52:18 "/git/kk_hirono_2016/a_資料/a_スクリーンショット/2016-05/2016-05-30-090346_うの字@un_co_the2ndと@nan5o 敢えてブロッコしないけど、金澤p庁御中うっとおしいわ・・・p庁がそんなもんに目をかけると思ってんのか.jpg " \url{https://twitter.com/kk_hirono/status/862274090215759876} 
- ./kk_hirono2021-05-18_130904.csv:2016-09-08 10:25:36 "%引用% 敢えてブロッコしないけど、金澤p庁御中うっとおしいわ・・・p庁がそんなもんに目をかけると思ってんのか " \url{https://twitter.com/kk_hirono/status/773693739998916608} 
- ./s_hirono2021-05-18_113546.csv:2017-01-09 03:27:26 "2016-05-30-090346_うの字@un_co_the2ndと@nan5o 敢えてブロッコしないけど、金澤p庁御中うっとおしいわ・・・p庁がそんなもんに目をかけると思ってんのか.jpg \url{http://pic.twitter.com/18bZE5TiZs}  " \url{https://twitter.com/s_hirono/status/818162210341982208} 
\end{lstlisting}

 「金澤p庁御中うっとおしいわ・・・p庁がそんなもんに目をかけると思ってんのか」とありますが,pとは一部の弁護士らがよく使う隠語のようなもので検察を指し,弁護士がB,被害者がV,裁判所や裁判官がJになっていたかと思います。最近でもよく見るのは市川寛弁護士のツイートぐらいです。

※ @kk\_hironoのアカウントがブロックされ,リツイートに失敗したツイート

\begin{itemize}
\tightlist
\item
  TW un\_co\_the2nd(うの字を名乗る💩物) 日時:2021/05/18 09:18:53 URL:
  \url{https://twitter.com/un\_co\_the2nd/status/1394447362068672516} 
  \textgreater{}
  法教育おじさんが頻繁に非モテの心を抉ってくるのは、人権とは気に食わねえものを程よく無視できてこそだという授業なんだよきっと。\\
  \textgreater{}\\
  \textgreater{} 爆裂四散してくれねえかな \url{https://t.co/EbyitxkDQY} 
\end{itemize}

 うの字のタイムラインでみかけたツイートです。アカウント名にも@un\_co\_the2ndということで,セカンドを意味すると思われる表記がありますが,元のうの字のTwitterアカウントは消滅しています。その後,復活あるは転生したのが今のアカウントです。

 プロフィールの名前もうの字を基本形に頻繁に変更していたのですが,最近は固定された状態で,福永活也弁護士の問題に触発されたらしく,「うの字を名乗る物」などとなっています。

 「金澤p庁御中うっとおしいわ・・・p庁がそんなもんに目をかけると思ってんのか」というツイートは消滅したうの字のTwitterアカウントかと思っていたのですが,スクリーンショットで現在のアカウントのツイートと確認しました。

 ツイートのURLを
\url{https://twitter.com/un\_co\_the2nd/status/736910031799484417} 
として見つけましたが,削除されているようです。古い書式のまとめ記事を見つけたのですが,その中にありました。次の記事です。

\begin{itemize}
\tightlist
\item
  奉納\弁護士妖怪大泥棒神社・金沢地方検察庁御中:
  @un\_co\_the2nd(うの字)のツイート→ ''敢えてブロッコしないけど、金澤p庁御中うっとおしいわ・・・p庁がそんなもんに目をかけると思ってんのか''
  \url{https://t.co/gHjbGz6WtE} 
\end{itemize}

〜un\_co\_the2nd(うの字)のツイート〜投稿日時〜2016/05/29
22:20〜ツイートのURL〜
\url{https://twitter.com/un\_co\_the2nd/status/736910031799484417}  〜
\#〜引用〜
敢えてブロッコしないけど、金澤p庁御中うっとおしいわ・・・p庁がそんなもんに目をかけると思ってんのか

 2016年5月29日のツイートだったと確認しました。

\begin{itemize}
\tightlist
\item
  2021年05月17日17時13分の登録:
  @nan5o(DUKEまんごう)のツイート ''.*'' 434/434:2021-05-06\_1006〜2021-05-17\_1648 2021年05月17日17時13分の記録
  \url{https://kk2020-09.blogspot.com/2021/05/nan5oduke4344342021-05-0610062021-05.html} 
\end{itemize}

 特にきっかけはなかったように思うのですが,@nan5o(DUKEまんごう)というアカウントのku3を記録したところ,その最後のツイートに,黒歴史クリーナーとかいう,弁護士業界の一面を象徴する,銀河鉄道999の放送のタイトルにも出てきそうな言葉を発見したのです。

\begin{itemize}
\tightlist
\item
  (434/434)
  @nan5o(DUKEまんごう)のツイート ''.*'' 434/434:2021-05-06\_1006〜2021-05-17\_1648
  2021年05月17日17時13分の記録\\
  TW nan5o(DUKEまんごう) 日時: 2021-05-06 10:06 URL:
  \url{https://twitter.com/nan5o/status/1390110773452558336} 
  \textgreater{} @nan5o ツイ消し中 (ꈨຶꎁꈨຶ)۶" ꁖʓ╵᷅ժ
  ̀꒭\textasciitilde{} \textbar{} 呪われしツイートを一括削除!
  黒歴史クリーナー - \url{https://t.co/8qz6llPl8n}  \#黒歴史クリーナー
\end{itemize}

 一晩経って,いや,2,3分もしないうちに忘れていたように思いますが,「呪われしツイートを一括削除!」ともありました。歴史的な出会いにも感じた,弁護士鉄道の未来を予感させるような言葉でもあります。

\begin{lstlisting}
py37_env ❯ tun fukazawas 1
\end{lstlisting}

\begin{itemize}
\tightlist
\item
  RT fukazawas(深澤諭史)|shimadayusuke66(島田雄左) 日時:2021-04-26
  07:41/2021-04-23 18:57 URL:
  \url{https://twitter.com/fukazawas/status/1386450393157169156} 
  \url{https://twitter.com/shimadayusuke66/status/1385533372798173191} 
  \textgreater{}
  弁護士や司法書士の資格取得後、以前は、丁稚奉公したら独立するのスタンダードでした。でも、今は独立に限らず、専門性を高めたり事務所の総合化などに伴い、色んなキャリアプランが増えたように感じます。そう考えると、資格者の役割も多様化してるので、資格取得後のキャリアプランは無限大です。
\end{itemize}

 深澤諭史弁護士のTwitterアカウントのタイムラインも停止状態が継続していることを確認しましたが,そういえば,昨日は朝に大きな発見として,神奈川県小田原市の村松謙弁護士のTwitterアカウントが非公開設定になっていました。鍵掛けとも呼ばれています。

\begin{quote}
《引用の始まり》
\end{quote}

\begin{quote}
おばけ@km0bake見てはならぬものを見、俺が知りたいと願っていた人を見分けることのできなかったお前ら(眼)は、今後は暗闇のうちにあるだろう。Thiva,
Greece2009年9月からTwitterを利用しています5,455 フォロー中8,090
フォロワーツイートは非公開です@km0bakeさんから承認された場合のみツイートを表示できます。承認をリクエストするには
[フォローする] をクリックします。詳細はこちら
\end{quote}

\begin{quote}
《引用の終わり》
\end{quote}

\begin{itemize}
\item
  \begin{enumerate}
  \def\labelenumi{(\arabic{enumi})}
  \tightlist
  \item
    👁️おばけ👁️さん (@km0bake) / Twitter
    \url{https://twitter.com/km0bake/with\_replies} 
  \end{enumerate}
\end{itemize}

 以前は,Twitterのプロフィールの名前に村松謙弁護士の実名があり,アイコンも顔写真となっていたのですが,そのうち一見して弁護士とはわからないようなアカウントになっていきました。前はプロフィールに法教育とも記載があったように思います。

 Amazonプライムビデオで連続で視聴していた銀河鉄道999では,毎回エンディングの曲がありましたが,「メーテルまた1つ星が消えたよ・・・銀河を流れるように」という歌詞がメロディとともに焼き付いています。

 Twitterにおける法クラ,弁護士アカウントも弁護士鉄道の夜空に,アカウントが削除され星のように消えていくこと,非公開設定でブラックボックス化することは,ずいぶん前から予想していた現象です。

2021年05月18日13時57分の実行記録: twitterAPI-search-lawList-mydql-add.rb
``@km0bake'' ツイート数:33/2419 リツイート数:1/2419
トータル:49``@km0bake''の該当: hirono\_hideki 3/0件 kk\_hirono 4/1件
s\_hirono 3/0件

 twitterAPI-search-lawList-mydql-add.rb
の検索がトータルで49件と少ないので,この前のぽぽひとのような炎上での鍵掛けという可能性はなさそうです。

\begin{itemize}
\tightlist
\item
  2021年05月18日14時02分の登録:
  REGEXP:''@km0bake''/データベース登録済みツイートの検索:2021-04-29〜2021-05-18/2021年05月18日14時02分の記録:ユーザ・投稿:16/38件
  \url{https://kk2020-09.blogspot.com/2021/05/regexpkm0bake2021-04-292021-05.html} 
\end{itemize}

 そういえば昨日の月曜日は,夜にイチケイのカラスの放送があるはずなので,昼のうちに録画機器のディスクスペース確保のため録画一覧からリモコン操作でみない録画済み番組の削除をしておこうと思っていたのですが,午後には思い出すこともなかったように思います。

 現実離れした実際にはありえない内容の法廷ドラマというツイートをいくつか見かけていましたが,ここ数日は「イチケイのカラス」のツイートは見かけていなかったように思います。「九条の大罪」という漫画の場合,初めは多少話題になっていたのですが,忌避されたような感じでした。

 刑事弁護に熱心で業界では一目置かれた印象もある櫻井光政弁護士が監修したというテレビドラマ「イチケイのカラス」ですが,刑事弁護に余り熱心ではなく金儲けを優先する弁護士には,面白くない迷惑な内容となっているのかもしれません。そう思えば,録画しておくべきでした。

 もう何年もまともにテレビの連続ドラマを視聴した憶えがないのですが,1,2回の見逃しはあったかもしれないものの日曜劇場『99.9-刑事専門弁護士-』は,毎回視聴をしていました。最後の方がよくわからずどうなったのか思い出せないのですが,印象に残る場面はいくつかあります。

2021年05月18日14時18分の実行記録: twitterAPI-search-lawList-mydql-add.rb
``イチケイのカラス'' ツイート数:21/2419 リツイート数:13/2419
トータル:4342``イチケイのカラス''の該当: hirono\_hideki 2/0件
kk\_hirono 4/0件 s\_hirono 1/0件

 第二引数を21日に指定してでまとめ記事の作成を始めました。「ユーザ・投稿:34/50件」という結果が出て投稿が終わっていました。

\begin{itemize}
\item
  2021年05月18日14時22分の登録:
  REGEXP:''イチケイのカラス''/データベース登録済みツイートの検索:2021-05-03〜2021-05-18/2021年05月18日14時21分の記録:ユーザ・投稿:34/50件
  \url{https://kk2020-09.blogspot.com/2021/05/regexp2021-05-032021-05\_18.html} 
\item
  (01/50) TW un\_co\_the2nd(うの字を名乗る?物) 日時: 2021-05-03
  21:56:22 +0900 URL:
  \url{https://twitter.com/un\_co\_the2nd/status/1389202170994585602\textgreater} {}
  ドラマ版イチケイのカラス、よくもまあ毎回毎回重大な違法を犯すなあ・・・PとBにやる気がないのって、当該部に係属したら控訴審が第一回戦になるからだな
\item
  (02/50) TW mental\_poverty(心の貧困) 日時: 2021-05-04 11:22:07
  +0900 URL:
  \url{https://twitter.com/mental\_poverty/status/1389404946454384643\textgreater} {}
  イチケイのカラスが最も有害なのは裁判官が頑張れば真実が発見できるという誤りが一般により広がることだと思う。このせいで、呼ばれた裁判員は「え?これしか証拠ないの?」てなるし、無罪判決を真っ白の判断と誤解して批判する人たちが現れる。
\end{itemize}

 上記の2件目のツイートは見覚えのあるものです。リツイートの数が131件として表示されています。刑事告発・非常上告_金沢地方検察庁御中(@kk\_hirono)でブロックされているのかわかりません。

〉〉〉 kk\_hironoのリツイート 〉〉〉

\begin{itemize}
\tightlist
\item
  RT
  kk\_hirono(刑事告発・非常上告_金沢地方検察庁御中)|mental\_poverty(心の貧困)
  日時:2021-05-18 14:27/2021/05/04 11:22 URL:
  \url{https://twitter.com/kk\_hirono/status/1394525084686032897} 
  \url{https://twitter.com/mental\_poverty/status/1389404946454384643} 
  \textgreater{}
  イチケイのカラスが最も有害なのは裁判官が頑張れば真実が発見できるという誤りが一般により広がることだと思う。このせいで、呼ばれた裁判員は「え?これしか証拠ないの?」てなるし、無罪判決を真っ白の判断と誤解して批判する人たちが現れる。
\end{itemize}

 ブロックされていそうに思っていたのですが,リツイートが出来ました。このままプロフィールを見ることもできそうです。

 「主にロースクールの悪口と刑事事件について匿名で好き放題書くためのチラ裏。ツイートは個人の見解で、所属する組織があったとしても無関係です。将来は暇な地方で干されながらネトゲ三昧の生活がしたい。」というプロフィールの内容でした。

〉〉〉 kk\_hironoのリツイート 〉〉〉

\begin{itemize}
\tightlist
\item
  RT
  kk\_hirono(刑事告発・非常上告_金沢地方検察庁御中)|mental\_poverty(心の貧困)
  日時:2021-05-18 14:30/2021/05/18 12:44 URL:
  \url{https://twitter.com/kk\_hirono/status/1394525817133096964} 
  \url{https://twitter.com/mental\_poverty/status/1394499021369090051} 
  \textgreater{}
  起訴状等の匿名化は、審判対象の確定と防御の利益をどう確保するかがポイントになりそう。とりあえず立法の方で議論すること自体は望ましい形と思う。
\end{itemize}

〉〉〉 kk\_hironoのリツイート 〉〉〉

\begin{itemize}
\tightlist
\item
  RT
  kk\_hirono(刑事告発・非常上告_金沢地方検察庁御中)|mental\_poverty(心の貧困)
  日時:2021-05-18 14:30/2021/05/17 21:20 URL:
  \url{https://twitter.com/kk\_hirono/status/1394525827669188608} 
  \url{https://twitter.com/mental\_poverty/status/1394266490099048451} 
  \textgreater{}
  再審開始決定何号でどう審理したんだよというツッコミは無粋なのでしてはならない。というか最近ファンタジーとして割と面白い気がしてきた。
\end{itemize}

 タイムラインに気になる内容のツイートがあったのでリツイートをしました。性犯罪で被害者の匿名化というのはだいぶん前からあったと思うのですが,無実の冤罪の場合,防御が難しくなるという想像もありました。被害者保護の優先でしょうが,弁護士の仕事ぶりもブラックボックス化です。

 そういえば思い入れの強い再審無罪判決を出した大津地裁の大西直樹裁判長ですが,大津市園児死傷事故では,併合審理されたストーカー規制法事件で,実際に出会っていない,対面していない男性に対するストーカー規制法違反の成立を認めていました。

 「再審開始決定何号」の意味がわからないのですが,裁判所では事件番号が割り振られ,今なら令和3年(た)1号などとなるはずですが,「(た)」というのは再審請求を意味する分類で,地方裁判所の刑事事件の一審ならば「(わ)」となっていたように思います。

 4月の年度替わりから始まるのかと今初めて考えたのですが,私の再審請求はすべて「平成○年(た)1号 金沢地方裁判所」でした。名古屋高裁金沢支部では面談で,直接,名古屋高裁金沢支部に再審請求をすることも出来ると言われたのですが,やることはなかったように思います。

 今考えると,親身な人で,名古屋高裁金沢支部の広い部屋で,ごく自然に主のような振る舞いをしているようにみえたので,名古屋高裁金沢支部の裁判長だった可能性がありそうです。当初からその可能性は感じていたのですが,他に気になることがあって優先事項にしませんでした。

 中央に置かれた広い机で,客人のような扱いを受けたのですが,私の立場や事情をよく知っているという様子でした。金沢地方裁判所の建物の2階に,正面玄関の上に,その広い部屋があったのですが,まるで古い映画のセットのような舞台のような部屋でした。

 1階にあった金沢地方裁判所の部屋にも入ったことがあるのですが,机が集まり人の数も多く,雑然とした雰囲気で,ドラマに出てくる新聞記者の集まる新聞社の部屋のようでした。応接室は別にあったのかもしれないですが,名古屋高裁金沢支部の場合は,事務と応接室が一緒になっていたと思います。

 あれは,平成9年の2月か3月のことなので,当時の名古屋高裁金沢支部の裁判長という人物を,いずれ,図書館の北國新聞縮小版で探してみたいと思います。金沢地方検察庁で刑事特別部でゴダイと名乗った人と会って話したのと同じ日かもしれません。部屋ではないところで話したと思います。

 待合所のような広い通路にも思える部屋だったと思うのですが,2人ぐらいで,他に部外者の姿はなかったはずです。一度は,すぐに受理すると言われ,力のこもったような言い方にも聞こえたのですが,当時はその受理の意味がよく理解できていませんでした。最近も受理は見かけていません。

\begin{itemize}
\tightlist
\item
  前原 捷一郎 \textbar{} 裁判官 \textbar{} 新日本法規WEBサイト
  \url{https://t.co/pQFuEZW65E}  ¥\n H12. 9.25 名古屋高裁金沢支部長 ¥\n H11.
  7.12 名古屋高裁金沢支部部総括判事
\end{itemize}

 これまで余り取り上げてこなかったと思いますが,平成11年の安藤健次郎さんに対する傷害事件で,控訴審の名古屋高裁金沢支部で裁判長だったのが,この前原捷一郎裁判長です。名前だけではなく強く印象に残った裁判長でしたが,法廷で何を話していたのか不思議と記憶にありません。

 上記のページに1つだけ見覚えのある顔写真がありましたが,記憶より若く常識がありそうな人物にみえます。頭部が細長く,七福神の置き物のようでした。ドラゴンボールのような漫画でも似たような登場人物がいたような記憶で,私の子供が見れば,即座に指差しをして声をあげそうでした。

 人の容姿をとやかくいうのはよくないし,目的でもないのですが,その裁判自体が異様な悪夢の体験のように記憶され,極端な言い方をすれば,異次元の世界に迷い込んだかのようでした。この前原捷一郎裁判長も私のことをずいぶん驚いた様子で見ていたので,お互いに未知との遭遇だったようです。

\begin{itemize}
\tightlist
\item
  七福神 福禄寿 - Google 検索 \url{https://t.co/b5V6Kc67Br} 
\end{itemize}

 見聞きした憶えがほとんど記憶にないのですが,調べたところ,七福神でも福禄寿という神様のようです。

\begin{itemize}
\tightlist
\item
  前原捷一郎裁判官(22期)の経歴 \textbar{} 弁護士山中理司のブログ
  \url{https://t.co/uhXywlOV8Q}  ¥\n 生年月日 S20.7.15 ¥\n 出身大学 東大 ¥\n
  退官時の年齢 61 歳 ¥\n 叙勲 H28年春・瑞宝重光章 ¥\n H19.7.11
  依願退官 ¥\n H18.2.28 ~ H19.7.10 名古屋高裁2刑部総括
\end{itemize}

 金沢地方裁判所,名古屋高裁金沢支部と平成4年から平成12年まで,法廷で数人の裁判官を見てきたのですが,その中でも特に高齢で,言い方が悪いですが,老い先が短く生存の可能性が低いと考えていたのが,この前原捷一郎裁判官になります。

 2,3日前には司法修習の期が一桁という裁判官を2人見ていて,その一人が被告発人小島裕史裁判長でしたが,22期と出てきたのが驚きで,司法試験の合格が遅かったのかと一瞬考えたのですが,続けて昭和20年7月15日という生年月日がありました。

 昭和11年12月生まれの安藤健次郎さんより9つほど年下だったというのは,衝撃的な驚きです。よくいえば,それだけ裁判長としての貫禄があったということになります。正直なところ七福神のイメージも強かったですが,人間離れした仙人にも思えました。

 法廷で最初に会ったとき,まず思い浮かべたのが自分の長男のことで,この場に長男がいれば,「うわっ」という短い声とともに,腕を上げ指差しをしている光景が浮かびました。実際に,指差しをしたことはなかったと思いますが,当時の私にはそれだけ衝撃的な出会いで,思考の麻痺状態にもありました。

 この平成11年の事件の名古屋高裁金沢支部の控訴審では,小堀秀行弁護士が国選弁護人で,検察官と席を入れ替わるという当時はありえないような出来事もあったのですが,全体としても魔法にかけられていたような控訴審で,裁判の内容がまったくといって記憶にありません。

 今考えると,前原捷一郎裁判長のプロ中のプロのような仕事ぶりで,身動きできないまま判決が出ていたように思います。判決の内容も記憶にないですが,一審の小川賢司裁判官の判決文とは違って,一度も判決文を見ていません。

 令和3年3月31日付告発状でも記述していると思いますが,小川賢司裁判官の判決文,判決書になるのか,この書面は金沢地方検察庁の遠塚さんのやらせで入手しました。入手させられたというのが事実です。そこに遠塚さんの名前があったのですが,実際にあったのは3年後ぐらいかと思います。

 長い間,金沢地方検察庁で遠塚さんに会った日に,収入印紙を買いに行って判決謄本の交付を受けたと思い込んでいたのですが,そうではなかったことを発見し,令和3年3月31日付告発状に思い出せるだけのことを記述したと思います。

 再審請求は平成15年まで,ほぼ年中行事のように毎年繰り返していたと思うのですが,自分で収入印紙を買いに行き,金沢地方検察庁で交付を受けたのもその1回だけだったと思います。本来は必ず必要とされる判決謄本だったと思います。

 かすかに記憶にあるのは,たぶん最初の再審請求のときだと思いますが,金沢刑務所の拘置所の2階で,担当刑務官の嶋さんから,裁判所の方で判決謄本を用意するような連絡を受けたことです。その後も判決謄本を添付するように裁判所からの連絡で指示を受けたことはなかったと思います。

 どのタイミングで金沢刑務所の拘置所から再審請求をしたのか記憶にないのですが,判決が確定しないと再審請求はできず,平成4年の傷害・準強姦被告事件では,判決の確定の知らせと同時に,受刑者とされ拘置所から転房させられました。その日楽しみにしていたアンパンの交付もなく。

 令和3年3月31日付告発状で書いたか,はっきり記憶にないですが,食べ物の物品購入で新たに追加され,初めての注文となったのがそのアンパンでした。日付を特定するだけ記憶がはっきりしませんが,平成6年2月21日頃のことです。

 金沢刑務所の拘置所に来た頃は,アンコが嫌いで,2,3度,ぜんざいや汁粉を食べずに残していたとも記憶にあるのですが,そのうち好んで食べるようになりました。拘置所に来る前は,たい焼きでもパリパリした衣の部分は好きだったのですが,アンコは捨てて食べていました。

 そういえば,平成9年当時の名古屋高裁金沢支部の裁判長,あるいは部総括の裁判官を探していたのでした。もう一息で見つかりそうでした。

\begin{itemize}
\tightlist
\item
  窪田 季夫 \textbar{} 裁判官 \textbar{} 新日本法規WEBサイト
  \url{https://t.co/w0FFDu1vk2}  ¥\n H13.12.24 定年退官 ¥\n H12.
  9.25 金沢家裁所長 ¥\n H11. 7.12 名古屋高裁金沢支部長 ¥\n H
  8.11.12 名古屋高裁金沢支部部総括判事 ¥\n H 6. 4.
  1 津地家裁部総括判事・津簡裁判事
\end{itemize}

 ようやく見つけたようです。「名古屋高裁金沢支部部総括判事」というキーワードを指定した検索結果を開いていたのですが,名前に見覚えのない裁判官が多く,笹本淳子という裁判官の名前はいくらか記憶に残っていました。

 しかし,名古屋高裁金沢支部部総括判事と名古屋高裁金沢支部長の違いもわかりづらく,刑事部と民事部の違いも不明です。

\begin{itemize}
\tightlist
\item
  名古屋高等裁判所金沢支部 平成8年(ネ)120号 判決 - 大判例
  \url{https://t.co/fMRn6jGtkH} 
\end{itemize}

 さきほどから検索でやたらと「(ネ)」という事件番号を見かけていたのですが,控訴人や被控訴人とあるので間違いなく民事裁判になると思います。民事裁判の控訴審判決のニュースを見ることも少ないですが,刑事裁判の控訴審と同じ裁判長の名前を見たことはなかったと思います。

\begin{itemize}
\tightlist
\item
  窪田季夫裁判官(18期)の経歴 \textbar{} 弁護士山中理司のブログ
  \url{https://t.co/vV5vAid0Z9}  ¥\n 生年月日 S11.12.24 ¥\n 出身大学 金沢大
  ¥\n 退官時の年齢 65 歳 ¥\n 叙勲 H19年春・瑞宝中綬章 ¥\n H13.12.24
  定年退官 ¥\n H12.9.25 ~ H13.12.23 金沢家裁所長 ¥\n H11.7.12 ~
  H12.9.24 名古屋高裁金沢支部長
\end{itemize}

 上記のツイートで引用した経歴の1つ下が,「H8.11.12 ~ H11.7.11
名古屋高裁金沢支部民事部部総括」となっていて,「民事部部総括」という表記自体を初めて見たように思います。

 この窪田季夫裁判官ですが,全く初めて見る名前と思いながら気になったのは,金沢での任地が多いのに金沢大学出身となっていること,これは被告発人小島裕史裁判長と共通しており,生年月日の昭和11年12月24日は安藤健次郎さんと一日違いになるかもしれません。たぶん12月25日です。

\begin{itemize}
\tightlist
\item
  小島裕史 \textbar{} 弁護士山中理司のブログ \url{https://t.co/VCjFQctnLK} 
  ¥\n H8.7.15 ~ H10.2.20 秋田地家裁所長 ¥\n H5.1.13 ~ H8.7.14
  名古屋高裁金沢支部長 ¥\n H4.11.20 ~ H5.1.12
  名古屋高裁金沢支部民事部部総括
\end{itemize}

 また目を疑うような情報が出てきたのか,前には気が付かなかった発見になりそうですが,被告発人小島裕史裁判長は平成4年11月20日から平成5年1月12日まで名古屋高裁金沢支部民事部部総括,平成5年1月13日から平成8年7月1日まで名古屋高裁金沢支部長とあります。

\begin{itemize}
\tightlist
\item
  武律 \textbar{} 弁護士山中理司のブログ \url{https://t.co/okMRUja2gA}  ¥\n
  生年月日 S7.4.1 ¥\n 出身大学 京大 ¥\n 退官時の年齢 61 歳 ¥\n 叙勲
  H14年春・勲二等瑞宝章 ¥\n H6.3.1 依願退官 ¥\n H4.11.20 ~ H6.2.28
  大阪高裁2刑判事 ¥\n S63.12.1 ~ H4.11.19
  名古屋高裁金沢支部刑事部部総括
\end{itemize}

 小島裕史裁判長の控訴審判決は平成5年9月7日ですが,濱田武律裁判長と交代したときには,名古屋高裁金沢支部民事部部総括となります。確認しましたが,
上記の通り「S63.12.1 ~ H4.11.19
名古屋高裁金沢支部刑事部部総括」となっていました。

 濱田武律裁判長の司法修習は9期となっていました。同じ日にもう一つ1桁の修習期の裁判官を見ていたと思うのですが,小島裕史裁判長を確認したところ14期でした。生年月日が昭和12年3月10日で,昭和11年12月生まれの安藤健次郎さんとは学年が同じ同級生になるのだと今頃気が付きました。

 昨日見たのか一昨日に見たのか思い出せないですが,自分の目を疑った「出身大学 金沢大」はそのままになっています。金沢大というのは地元の石川県でも余りみかけない表記だと思うのですが,金沢大学に間違いないと思います。

 金沢大学は国立大学と聞いたように思います。私立大学と聞いたことは一度もありません。割合は不明ですが,石川県外からの入学生も多いと考えられます。

 「H5.1.13 ~ H8.7.14
名古屋高裁金沢支部長」とも記載がありますが,ずいぶん長く感じられる裁判長としての任期で,この間には,私の再審請求の抗告棄却の決定もありましたが,被告発人長谷川紘之弁護士の自宅で妻が襲われた強盗事件もありました。

 新聞には金沢市内の円光寺となっていたと記憶に残るのですが,記事には自宅を新築工事中で仮住まいとなっていたようにも記憶に残ります。円光寺は平成9年から11年の間,被告発人大網健二が購入した中古住宅に済んでいたので,よく行くことのある場所でもありました。

 裁判所からは遠いという感覚でしたが,後に被告発人長谷川紘之弁護士の自宅は金沢市内の三十刈ですでに死んでいるとも聞き,富山県の砺波市の有名な大きな神社の息子とも聞きました。4月にアジが極端に釣れなくなった年と記憶にあるのですが,未確認なものの2018年辺りを考えています。

 金沢市内の三十刈は,同じ金沢市内の四十万の近くですが,その辺りは金沢市と野々市市,白山市の境界線がが複雑に入り組んでいます。裁判所とはずいぶん離れた場所で,事情があって通勤をしていたと考える他ありません。法律事務所の住所も裁判所の近くになっていました。

\begin{itemize}
\tightlist
\item
  2021年02月19日14時48分の登録:
  「長谷川紘之弁護士」を@hirono\_hideki @kk\_hirono @s\_hironoで検索
  \url{https://kk2020-09.blogspot.com/2021/02/hironohidekikkhironoshirono\_86.html} 
\end{itemize}

 2月19日に作成したまとめ記事がありましたが,この時期だとスクリプトではなく,ワンライナーのコマンドとして実行していた時期になるかもしれません。

 現在とまとめ記事の書式は違いがなさそうでしたが,「法律事務所」でページ内検索をしても,被告発人長谷川紘之弁護士の法律事務所名が見つかりませんでした。住所や番地まであることを期待したのですが,残念です。

\begin{itemize}
\tightlist
\item
  2021年05月18日17時22分の登録:
  「長谷川紘之」を過去のはてなダイアリーの記事から検索
  \url{https://kk2020-09.blogspot.com/2021/05/blog-post\_62.html} 
\end{itemize}

 にわかに信じがたくプログラム的なバグなのかと疑いたくなるのですが,上記のはてなダイアリーのまとめ記事に,「法律事務所」のページ内検索の結果はありませんでした。

20111118:{[}link{]}
\emph{1321608027}{[}再審請求の資料{]}平成6年7月5日付民事訴状(原告訴訟代理人長谷川紘之弁護士作成)(写真
6 枚) - 廣野秀樹 - Picasa ウェブ アルバム

\begin{itemize}
\tightlist
\item
  hatena-diary\_20111118 -
  告発\金沢地方検察庁\最高検察庁\法務省\石川県警察御中
  \url{https://t.co/Kuw1x75deN} 
\end{itemize}

 タイトルを見てもさっぱりですが,上記の記事にPicasa ウェブ
アルバムの裁判資料の写真の一覧がありました。よくまとめられていると思いましたが,被告発人長谷川紘之弁護士が代理人となった判決に,法律事務所の名前も住所もありませんでした。当然といえば当然ですが。

\begin{itemize}
\tightlist
\item
  長谷川紘之 法律事務所 金沢 - Google 検索 \url{https://t.co/NxaRpoYZDj} 
\end{itemize}

 東京ではなかったかと思いますが長谷川紘之弁護士と同姓同名の弁護士がいたことを思い出しました。

\begin{itemize}
\tightlist
\item
  長谷川紘之法律事務所(石川県) \textbar{} いい相続 -
  相続の無料相談と相続に強い専門家紹介 \url{https://t.co/16hVsgPMES}  所在地:
  ¥\n 石川県金沢市大手町15-32
\end{itemize}

 被告発人長谷川紘之弁護士は死んでいるはずで,金沢弁護士会のホームページの弁護士検索でも登録がないことを確認したように思います。古い情報がそのまま残っている可能性もありますが,敦賀弁護士のように法律事務所名がそのまま残っていることもありました。

\begin{itemize}
\tightlist
\item
  長谷川法律事務所 - Google マップ \url{https://t.co/7wfgEAzYaz}  ¥\n
  〒920-0912 石川県金沢市大手町15−32 アイビーガーデン大手町
\end{itemize}

 ここ1年以内にGoogleマップのストリートビューで見たような建物ですが,長谷川法律事務所という名前が残っているのは意外に感じました。法律事務所の看板なので当然,所属の弁護士はいるのでしょう。Googleマップにも長谷川法律事務所の下に「弁護士」と表示があります。

 一応と思い,ストリートビューを表示させると,ビルは工事中で,道路の向かいには見たことのない大きく新しい建物がありました。「KKRホテル金沢」というようです。

\begin{itemize}
\tightlist
\item
  長谷川・東法律事務所 - 石川県金沢市 - 弁護士ドットコム
  \url{https://t.co/nJv7OmS49u}  ¥\n 所在地 ¥\n 〒 920-0912 ¥\n 石川県
  金沢市大手町15-32 アイビーガーデン大手町102
\end{itemize}

 長谷川法律事務所と同じ住所,金沢市大手町15-32ですが,これはアイビーガーデン大手町という建物の住所のようです。住居と一緒になった貸事務所と思われますが,同じ建物に長谷川法律事務所と長谷川・東法律事務所という2つの法律事務所があるとは,都心でも考えにくところです。

\begin{itemize}
\tightlist
\item
  東 祐紀弁護士(長谷川・東法律事務所) - 石川県金沢市 -
  弁護士ドットコム \url{https://t.co/hyc16RRUTp} 
\end{itemize}

 東祐紀弁護士という見たことのあるような弁護士の名前が出てきたのですが,上記の弁護士ドットコムのページには顔写真があり,若い頃の写真を使う弁護士も中にはいますが,20代にもみえるような若い感じの弁護士です。

 さきほどから薄々,4月2日頃に調べた金沢弁護士会の4人の副会長の一人かもしれないと思っているのですが,この弁護士ドットコムのページは,そのときに見た記憶がありません。顔写真は前に見たような気もするのですが,4月ではないと思います。

 東祐紀弁護士の弁護士ドットコムのページには,2011年弁護士登録,経歴・技能に冤罪弁護経験とあり,学歴が2008年3月大阪大学法学部卒業,2010年3月神戸大学法科大学院卒業とあります。金沢から大阪や神戸の大学に行く人はいると思いますが,関西から移り住んだ可能性もありそうです。

 弁護士登録から10年で金沢弁護士会の副会長になるとは,あれですが,前年度の金沢弁護士会の会長だった金沢合同法律事務所の宮西香弁護士も,ずいぶん前に副会長の経験があると経歴に見たように思います。

 金沢弁護士会の今年度の会長,4人の副会長のことは4月の1日か2日の夜以来,一度も調べていないのですが,金沢合同法律事務所の所属弁護士と野田政仁弁護士の法律事務所の所属弁護士がそれぞれ副会長になっていたことは,割合はっきりと憶えています。

\begin{itemize}
\tightlist
\item
  ホーム - 七尾市立朝日小学校 \url{https://t.co/hlQML1AduT}  ¥\n 投稿日時 :
  2020/06/02  校長先生 ¥\n
   6年生は、金沢の弁護士 東 祐紀氏(長谷川・東法律事務所)を講師にお迎えし、体育館でいじめ予防教室を行いました。 会場は6年生が考えて準備してくれました。最上級生として考え工夫していると感
\end{itemize}

 七尾市内に朝日小学校があるともしらないですが,朝日という地名も心当たりがなく,調べてみないとわかりません。東祐紀弁護士が朝日小学校の卒業生だったという可能性もありますが,断定できるような情報ではないと考えます。

 Googleマップで調べるとJR徳田駅の近くでした。はっきり断定できる記憶ではないですが,昭和の国鉄時代も七尾駅の次の停車駅が徳田駅となっていたように思います。急行列車が停車する駅ではなかったとも思います。

 東祐紀弁護士をGoogleで検索しても思いの外,情報が少ないのですが,長谷川法律事務所を長谷川・東法律事務所と変更しているとすれば,被告発人長谷川紘之弁護士が亡くなっていることを前提として,故人の弁護士の名前と自分の名前を一緒にするというのは,初めて見るケースです。

 今のところ長谷川・東法律事務所には東祐紀弁護士1人の所属しか確認できないのですが,2人以上の弁護士が在籍しているものと考えるのが普通で,会話のネタにはなりそうですが,そうすると被告発人長谷川紘之弁護士の名前を相手に印象づけることにもなりそうです。

 普通に考えれば,これも被告発人長谷川紘之弁護士が亡くなっていることを前提としてですが,師匠として世話になった故人を偲び,思慕の念あるいは尊敬の念を込めて,名前を残しているものとなるでしょう。

 今年度の金沢弁護士会の副会長4人の布陣というのも気になっていたのですが,被告発人弁護士らの正統性を内外に印象づける狙いがあるようにも思えていました。令和3年3月31日付告発状ぐらいは読んでいるのでしょうか。

 \url{http://p2223.nsk.ne.jp/kibougaoka/com/pdf/2014/yakuin.pdf} 
に「社会福祉法人 希望が丘 理事、評議員、監事名簿」の監事という記載がありました。

 評議員として粟田真人弁護士という名前もありますが,理事長が里見秀幸という名前で,理事が6人,監事が2人,評議員が8人となっています。社会福祉法人 希望が丘というのは,よくありそうな名称ですが,医療福祉施設の可能性がありそうと思っています。

\begin{itemize}
\tightlist
\item
  社会福祉法人 希望が丘 \url{https://t.co/TTWGzeDCSF}  ¥\n
  金沢城北ライオンズクラブ様より焼きたてのたこ焼きとふくら焼きを寄贈していただきました。
\end{itemize}

 このライオンズクラブというのは,10年ほど前になるかと思いますが,Googleで被告発人木梨松嗣弁護士の写真を探し始めた当初に,会合に出席する木梨松嗣弁護士を見かけていたように思います。あるいは通常の検索で,ホームページに写真があっただけかもしれません。

 住所が金沢市の小池町となっていますが,聞いたことのない地名です。

\begin{itemize}
\tightlist
\item
  希望が丘 - Google マップ \url{https://t.co/TIwC2kZEJz} 
\end{itemize}

 聞いたことのない金沢市小池町という住所だと思ったら,森本の山奥のような場所が出てきました。そのまえに児童施設という情報も見かけているのですが,児童養護施設とは違いがありそうです。

 被告発人大網健二が被告発人木梨松嗣弁護士とビジネスパートナーのような関係と話していた平成11年2月頃,金沢市木越のメロン幼稚園の前身として,被告発人大網健二に車で連れて行かれた古い木造の建物も児童養護施設のような雰囲気のある森本の場所でした。

 現在は違った場所を走っている国道159号線から,さほど離れておらず,周囲に大きな建物のない住宅地の一角に,その玄関が印象的な木造の建物はありました。メロン幼稚園と同じ幼稚園だったのかと思いますが,園児の姿は記憶にありません。

 とても懐かしさのある木造の建物でしたが,大きな玄関から広い廊下があったように思います。木造の平屋建てで,建物のかたちや雰囲気は,子供の頃,テレビでみていたアニメ,タイガーマスクの児童養護施設のようでした。昭和40年代後半のことです。

 自治体と能登町について当初取り上げるつもりだったのですが,能登町についてはイカの駅つくモールのことで,また別の機会に取り上げたいと思います。

 再審のことより,被告発人長谷川紘之弁護士や被告発人木梨松嗣弁護士のことを集中的に取り上げたくなったので,手短に切り上げたいと思いますが,再審法改正のことで自治体のことが出ていました。出処は大崎事件と同じ西日本新聞でした。

\begin{itemize}
\tightlist
\item
  「再審法改正を」地方動く 49市町村議会、証拠開示など求め意見書|【西日本新聞me】
  \url{https://t.co/woPWUssr4n} 
  全国で少なくとも49市町村議会が速やかな法改正を国に求める意見書を可決した。
\end{itemize}

 この「全国で少なくとも49市町村議会が速やかな法改正を国に求める意見書を可決した。」の意味がよくわからないのですが,法改正を国に求める意見書を可決した,というのも初めて見かけるようなニュースです。

 町議会の可決で,全国的な批判にさらされ,未開人のような意見が出たのも,私の生まれ住む能登町で,イカの駅つくモールでの巨大イカのモニュメントが発端です。

 能登町と再審法改正との組み合わせというか接点が出てくるとは夢にも思わなかったのですが,梨木作次郎弁護士に同じく未開人のような評価を受け,殺人罪での少年事件の無罪判決という弁護士鉄道の金字塔のような足跡を残されたのも,お隣りの珠洲市蛸島町,蛸島事件になります。

 なお,先程も前年度の金沢弁護士会の会長,宮西香弁護士の所属として取り上げた金沢合同法律事務所が,その梨木作次郎弁護士の開所になるという歴史があり,蛸島町は能登鉄道の終着駅,蛸島駅がありました。弁護士鉄道と能登鉄道は,私の母親の晩年の人生そのものでもあります。

\begin{itemize}
\tightlist
\item
  〈〈〈 2021/05/18 19:08:34 Linux Emacs: 〈〈〈
\end{itemize}

\hypertarget{ux5e746ux67086ux65e5ux306eux5915ux65b9ux306bux8aadux3093ux3060ux30e2ux30c8ux30b1ux30f3ux3053ux3068ux77e2ux90e8ux5584ux6717ux5f01ux8b77ux58ebux4eacux90fdux5f01ux8b77ux58ebux4f1aux306eux30c4ux30a4ux30fcux30c8ux672cux8ceaux7684ux306bux8003ux3048ux76f4ux3059ux304dux3063ux304bux3051ux3068ux306aux3063ux305fux77e2ux90e8ux5584ux6717ux5f01ux8b77ux58ebux306eux5f71ux97ffux3068ux8cacux4efb}{%
\paragraph{2021年6月6日の夕方に読んだモトケンこと矢部善朗弁護士(京都弁護士会)のツイート、本質的に考え直すきっかけとなった矢部善朗弁護士の影響と責任}\label{ux5e746ux67086ux65e5ux306eux5915ux65b9ux306bux8aadux3093ux3060ux30e2ux30c8ux30b1ux30f3ux3053ux3068ux77e2ux90e8ux5584ux6717ux5f01ux8b77ux58ebux4eacux90fdux5f01ux8b77ux58ebux4f1aux306eux30c4ux30a4ux30fcux30c8ux672cux8ceaux7684ux306bux8003ux3048ux76f4ux3059ux304dux3063ux304bux3051ux3068ux306aux3063ux305fux77e2ux90e8ux5584ux6717ux5f01ux8b77ux58ebux306eux5f71ux97ffux3068ux8cacux4efb}}

\begin{itemize}
\tightlist
\item
  〉〉〉 Linux Emacs: 2021/06/07 06:13:07 〉〉〉
\end{itemize}

:CATEGORIES: @kanazawabengosi \#金沢弁護士会 @JFBAsns
日本弁護士連合会(日弁連) \#法務省 @MOJ\_HOUMU
\#モトケンこと矢部善朗弁護士(京都弁護士会)

\begin{itemize}
\tightlist
\item
  1406:2021-06-07\_06:15:55 \#告発状 \#\#\#\#
  弁護士鉄道の歴史探訪・奥能登紀行編:「連続テレビ小説まれ」の舞台となった輪島市大沢と、「ほうたろう」という妖怪の風呂敷アイコンのTwitterアカウントのこと
  \url{https://hirono-hideki.hatenadiary.jp/entry/2021/06/07/061553} 
\end{itemize}

 内容が変わりますが作業は上記のエントリーの続きになります。それでも1つ書き忘れていたことがあって、まだ調べていないのですが、妖怪ウォッチのコマさんの発見です。ほうたろうというTwitterアカウントの発見は2015年11月17日になっていました。

\begin{itemize}
\tightlist
\item
  奉納\さらば弁護士鉄道・泥棒神社の物語(@hirono\_hideki)/「コマさん」の検索結果
  - Twilog \url{https://t.co/YgUbpY1qPL} 
\end{itemize}

 思ったより間があったのですが、2015年12月20日に妖怪ウォッチのコマさんに触れたツイートが3件ありました。泥棒のような風呂敷を担いでいたのも、昨日の発見でしたが、すっかり忘れていました。妖怪ウォッチの話題もずいぶん久しく見かけていません。

 なお、妖怪ウォッチのコマさんは、神社の狛犬が家出をして妖怪になったような話になっていたと記憶にあります。理屈っぽいところがあって弁護士をイメージして作られたキャラクターなのかと考えることもありました。

\begin{quote}
《引用の始まり》
\end{quote}

\begin{quote}
田舎にある神社のこま犬に取り憑いていたものの、その神社が取り壊された為に都会にやってきた妖怪。都会での生活に馴染もうと頑張っている。驚いたり、興奮すると「もんげー!!」と口にしてしまう。大好物はソフトクリーム。
\end{quote}

\begin{quote}
《引用の終わり》
\end{quote}

\begin{itemize}
\tightlist
\item
  妖怪ウォッチ|キャラクター
  \url{https://www.tv-tokyo.co.jp/anime/youkai-watch/sp/chara/index\_2.html} 
\end{itemize}

 確認のため調べてみました。記憶とはちょっとストーリーが違っていました。色違いのコマじろう、という弟のキャラクラーがいることも忘れていました。早朝5時30分頃のテレビで数は少ないですが何度かみたことがありました。日曜日です。今はどんな放送をやっているのかもしりません。

 そういえば、私はモトケンこと矢部善朗弁護士(京都弁護士会)のことを、「弁護士妖怪モトケン大王」と名指しし、こちらから訴えてやろうかという対応を受けました。その前に「化け物人間」ともつけていました。ストレートに思いをぶつけました。弁護士としての危険性に対する防御です。

 警察が相手にするわけがないというのがモトケンこと矢部善朗弁護士(京都弁護士会)の基本的な見込みのようでした。化け物人間弁護士妖怪モトケン大王とまで名指しするには、それだけの理由と呪縛のような悪影響があったわけですが、今はそれを紐解くときです。

 昨日の夕方のモトケンこと矢部善朗弁護士(京都弁護士会)のツイートを取り上げる前、そのしばらく前にスクリーンショットとして見かけていた2015年のモトケンこと矢部善朗弁護士(京都弁護士会)のツイートの内容というのも衝撃のものがありました。そちらからご紹介したいと思います。

 2015年11月のフォルダに本来探していたモトケンこと矢部善朗弁護士(京都弁護士会)のツイートのスクリーンショットがなかったのですが、11月17日にほうたろうのツイートが出てくる間に、いくつか発見があって、思っていた以上にひどい内容だったので、そちらから先にご紹介します。

〉〉〉 kk\_hironoのリツイート 〉〉〉

\begin{itemize}
\tightlist
\item
  RT
  kk\_hirono(刑事告発・非常上告_金沢地方検察庁御中)|s\_hirono(非常上告-最高検察庁御中\_ツイッター)
  日時:2021-06-07 06:52/2021/06/07 06:51 URL:
  \url{https://twitter.com/kk\_hirono/status/1401658278287855617} 
  \url{https://twitter.com/s\_hirono/status/1401657932769497089} 
  \textgreater{}
  2015-11-15-191253\_モトケンさんはTwitterを使っています: ''議論する価値のない人はブロックすることがある。相手が残念がるかどうかは関係ない。.jpg
  \url{https://t.co/0aeinWAU9l} 
\end{itemize}

〉〉〉 kk\_hironoのリツイート 〉〉〉

\begin{itemize}
\tightlist
\item
  RT
  kk\_hirono(刑事告発・非常上告_金沢地方検察庁御中)|s\_hirono(非常上告-最高検察庁御中\_ツイッター)
  日時:2021-06-07 06:52/2021/06/07 06:51 URL:
  \url{https://twitter.com/kk\_hirono/status/1401658298529579015} 
  \url{https://twitter.com/s\_hirono/status/1401657919637135360} 
  \textgreater{}
  2015-11-11-205246\_モトケンさんはTwitterを使っています: ''ツイッターは、頭のいい人にとっては頭がいいことを証明する場になるが、頭が悪い人にとっては頭が.jpg
  \url{https://t.co/AzK6AYhhgH} 
\end{itemize}

〉〉〉 kk\_hironoのリツイート 〉〉〉

\begin{itemize}
\tightlist
\item
  RT
  kk\_hirono(刑事告発・非常上告_金沢地方検察庁御中)|s\_hirono(非常上告-最高検察庁御中\_ツイッター)
  日時:2021-06-07 06:52/2021/06/07 06:51 URL:
  \url{https://twitter.com/kk\_hirono/status/1401658312190353408} 
  \url{https://twitter.com/s\_hirono/status/1401657906794139652} 
  \textgreater{}
  2015-11-09-205559\_モトケンさんはTwitterを使っています: ''明確な脅迫だと思うなら告訴とか告発したらどうかな。明確なら受理してれるよ。.jpg
  \url{https://t.co/oXENH0O55c} 
\end{itemize}

〉〉〉 kk\_hironoのリツイート 〉〉〉

\begin{itemize}
\tightlist
\item
  RT
  kk\_hirono(刑事告発・非常上告_金沢地方検察庁御中)|s\_hirono(非常上告-最高検察庁御中\_ツイッター)
  日時:2021-06-07 06:52/2021/06/07 06:50 URL:
  \url{https://twitter.com/kk\_hirono/status/1401658332952219651} 
  \url{https://twitter.com/s\_hirono/status/1401657893439479809} 
  \textgreater{}
  2015-11-08-104237\_モトケンさんはTwitterを使っています: ''@s\_hirono 嘘つき''.jpg
  \url{https://t.co/bnLjtuyyrp} 
\end{itemize}

〉〉〉 kk\_hironoのリツイート 〉〉〉

\begin{itemize}
\tightlist
\item
  RT
  kk\_hirono(刑事告発・非常上告_金沢地方検察庁御中)|s\_hirono(非常上告-最高検察庁御中\_ツイッター)
  日時:2021-06-07 06:52/2021/06/07 06:50 URL:
  \url{https://twitter.com/kk\_hirono/status/1401658347108007941} 
  \url{https://twitter.com/s\_hirono/status/1401657880562913283} 
  \textgreater{}
  2015-11-08-104118\_モトケンさんはTwitterを使っています: ''@hirono\_hideki そういう話は告訴が受理されてからの話だな。''.jpg
  \url{https://t.co/gTDMvhNSeW} 
\end{itemize}

〉〉〉 kk\_hironoのリツイート 〉〉〉

\begin{itemize}
\tightlist
\item
  RT
  kk\_hirono(刑事告発・非常上告_金沢地方検察庁御中)|s\_hirono(非常上告-最高検察庁御中\_ツイッター)
  日時:2021-06-07 06:52/2021/06/07 06:50 URL:
  \url{https://twitter.com/kk\_hirono/status/1401658375440527361} 
  \url{https://twitter.com/s\_hirono/status/1401657867569033219} 
  \textgreater{}
  2015-11-08-103940\_モトケンさんはTwitterを使っています: ''ずーっと片思いだった相手に、思わせぶりな言葉をかけてしまったかもしれないw'' .jpg
  \url{https://t.co/eCxYs0Esa1} 
\end{itemize}

〉〉〉 kk\_hironoのリツイート 〉〉〉

\begin{itemize}
\tightlist
\item
  RT
  kk\_hirono(刑事告発・非常上告_金沢地方検察庁御中)|s\_hirono(非常上告-最高検察庁御中\_ツイッター)
  日時:2021-06-07 06:52/2021/06/07 06:50 URL:
  \url{https://twitter.com/kk\_hirono/status/1401658389613060099} 
  \url{https://twitter.com/s\_hirono/status/1401657853958492162} 
  \textgreater{}
  2015-11-08-103753\_モトケンさんはTwitterを使っています: ''被害妄想の人に絡まれるとかなり大変。まあ、経緯を知っている人が大勢いるので、分が悪いのは彼の.jpg
  \url{https://t.co/VmHhcJA1oq} 
\end{itemize}

〉〉〉 kk\_hironoのリツイート 〉〉〉

\begin{itemize}
\tightlist
\item
  RT
  kk\_hirono(刑事告発・非常上告_金沢地方検察庁御中)|s\_hirono(非常上告-最高検察庁御中\_ツイッター)
  日時:2021-06-07 06:52/2021/06/07 06:50 URL:
  \url{https://twitter.com/kk\_hirono/status/1401658404213383171} 
  \url{https://twitter.com/s\_hirono/status/1401657839764996097} 
  \textgreater{}
  2015-11-08-103607\_モトケンさんはTwitterを使っています: ''。@s\_hirono 確認するから係長の名前を明らかにしてくれないかな。警察は異動が多いから.jpg
  \url{https://t.co/G6XicesHH5} 
\end{itemize}

〉〉〉 kk\_hironoのリツイート 〉〉〉

\begin{itemize}
\tightlist
\item
  RT
  kk\_hirono(刑事告発・非常上告_金沢地方検察庁御中)|s\_hirono(非常上告-最高検察庁御中\_ツイッター)
  日時:2021-06-07 06:53/2021/06/07 06:50 URL:
  \url{https://twitter.com/kk\_hirono/status/1401658420856377348} 
  \url{https://twitter.com/s\_hirono/status/1401657826850656260} 
  \textgreater{}
  2015-11-08-103422\_モトケンさんはTwitterを使っています: ''。@hirono\_hideki ←自覚のある人がここに1人。告訴するすると言いつつもう数年。.jpg
  \url{https://t.co/QApF0CYwRZ} 
\end{itemize}

〉〉〉 kk\_hironoのリツイート 〉〉〉

\begin{itemize}
\tightlist
\item
  RT
  kk\_hirono(刑事告発・非常上告_金沢地方検察庁御中)|s\_hirono(非常上告-最高検察庁御中\_ツイッター)
  日時:2021-06-07 06:53/2021/06/07 06:50 URL:
  \url{https://twitter.com/kk\_hirono/status/1401658444810100739} 
  \url{https://twitter.com/s\_hirono/status/1401657813353512961} 
  \textgreater{}
  2015-11-08-103241\_モトケンさんはTwitterを使っています: ''最近、というかしばらく前から、ストーカーに付きまとわれている気分がしているw 自覚のある人が.jpg
  \url{https://t.co/udwAzdzctL} 
\end{itemize}

〉〉〉 kk\_hironoのリツイート 〉〉〉

\begin{itemize}
\tightlist
\item
  RT
  kk\_hirono(刑事告発・非常上告_金沢地方検察庁御中)|s\_hirono(非常上告-最高検察庁御中\_ツイッター)
  日時:2021-06-07 06:53/2021/06/07 06:50 URL:
  \url{https://twitter.com/kk\_hirono/status/1401658459506962432} 
  \url{https://twitter.com/s\_hirono/status/1401657800359497728} 
  \textgreater{}
  2015-11-08-103145\_モトケンさんはTwitterを使っています: ''最近、というかしばらく前から、ストーカーに付きまとわれている気分がしているw 自覚のある人が.jpg
  \url{https://t.co/L7xJRtHliO} 
\end{itemize}

〉〉〉 kk\_hironoのリツイート 〉〉〉

\begin{itemize}
\tightlist
\item
  RT
  kk\_hirono(刑事告発・非常上告_金沢地方検察庁御中)|s\_hirono(非常上告-最高検察庁御中\_ツイッター)
  日時:2021-06-07 06:53/2021/06/07 06:50 URL:
  \url{https://twitter.com/kk\_hirono/status/1401658475466268673} 
  \url{https://twitter.com/s\_hirono/status/1401657787428474880} 
  \textgreater{}
  2015-11-08-102559\_モトケンさんはTwitterを使っています: ''やはり、日本の刑事司法の最大の問題点は裁判所だな。''.jpg
  \url{https://t.co/IwYeELV9YR} 
\end{itemize}

〉〉〉 kk\_hironoのリツイート 〉〉〉

\begin{itemize}
\tightlist
\item
  RT
  kk\_hirono(刑事告発・非常上告_金沢地方検察庁御中)|s\_hirono(非常上告-最高検察庁御中\_ツイッター)
  日時:2021-06-07 06:53/2021/06/07 06:50 URL:
  \url{https://twitter.com/kk\_hirono/status/1401658491077464065} 
  \url{https://twitter.com/s\_hirono/status/1401657774711341056} 
  \textgreater{}
  2015-11-08-102457\_モトケンさんはTwitterを使っています: ''@imaloser15 ありがとうございます。一番検察寄りの部だったみたいで、余計に嬉しいで.jpg
  \url{https://t.co/BCjGCgr1et} 
\end{itemize}

〉〉〉 kk\_hironoのリツイート 〉〉〉

\begin{itemize}
\tightlist
\item
  RT
  kk\_hirono(刑事告発・非常上告_金沢地方検察庁御中)|s\_hirono(非常上告-最高検察庁御中\_ツイッター)
  日時:2021-06-07 06:53/2021/06/07 06:50 URL:
  \url{https://twitter.com/kk\_hirono/status/1401658509003956230} 
  \url{https://twitter.com/s\_hirono/status/1401657761935413251} 
  \textgreater{}
  2015-11-08-102223\_モトケンさんはTwitterを使っています: ''@fukazawas ある意味、事件に恵まれました。ひどい起訴とひどい一審判決。''.jpg
  \url{https://t.co/WJmIMj1EL3} 
\end{itemize}

〉〉〉 kk\_hironoのリツイート 〉〉〉

\begin{itemize}
\tightlist
\item
  RT
  kk\_hirono(刑事告発・非常上告_金沢地方検察庁御中)|s\_hirono(非常上告-最高検察庁御中\_ツイッター)
  日時:2021-06-07 06:53/2021/06/07 06:50 URL:
  \url{https://twitter.com/kk\_hirono/status/1401658533238644740} 
  \url{https://twitter.com/s\_hirono/status/1401657748903763969} 
  \textgreater{}
  2015-11-08-102106\_モトケンさんはTwitterを使っています: ''今日は控訴審で一部無罪がとれた。贅沢言うと少し不満。しかし、裁判官というのは有罪判決を書くプ.jpg
  \url{https://t.co/S8stWVjKVl} 
\end{itemize}

〉〉〉 kk\_hironoのリツイート 〉〉〉

\begin{itemize}
\tightlist
\item
  RT
  kk\_hirono(刑事告発・非常上告_金沢地方検察庁御中)|s\_hirono(非常上告-最高検察庁御中\_ツイッター)
  日時:2021-06-07 06:53/2021/06/07 06:50 URL:
  \url{https://twitter.com/kk\_hirono/status/1401658551064436743} 
  \url{https://twitter.com/s\_hirono/status/1401657736106897410} 
  \textgreater{}
  2015-11-08-102039\_モトケンさんはTwitterを使っています: ''今日は控訴審で一部無罪がとれた。贅沢言うと少し不満。しかし、裁判官というのは有罪判決を書くプ.jpg
  \url{https://t.co/eyCdgTHXSR} 
\end{itemize}

〉〉〉 kk\_hironoのリツイート 〉〉〉

\begin{itemize}
\tightlist
\item
  RT
  kk\_hirono(刑事告発・非常上告_金沢地方検察庁御中)|s\_hirono(非常上告-最高検察庁御中\_ツイッター)
  日時:2021-06-07 06:53/2021/06/07 06:50 URL:
  \url{https://twitter.com/kk\_hirono/status/1401658567950622720} 
  \url{https://twitter.com/s\_hirono/status/1401657723410784257} 
  \textgreater{}
  2015-11-07-205747\_廣野秀樹\さらば弁護士鉄道・泥棒神社の夜さんはTwitterを使っています: ''。@motoken\_tw あなたことモトケンこと矢部善朗弁護.jpg
  \url{https://t.co/HuMgpFYaGS} 
\end{itemize}

〉〉〉 kk\_hironoのリツイート 〉〉〉

\begin{itemize}
\tightlist
\item
  RT
  kk\_hirono(刑事告発・非常上告_金沢地方検察庁御中)|s\_hirono(非常上告-最高検察庁御中\_ツイッター)
  日時:2021-06-07 06:53/2021/06/07 06:50 URL:
  \url{https://twitter.com/kk\_hirono/status/1401658583259914243} 
  \url{https://twitter.com/s\_hirono/status/1401657709905076227} 
  \textgreater{}
  2015-11-07-205356\_モトケンさんはTwitterを使っています: ''法律に違反しなければ何をやってもいい、と考えている人は、大抵法律に違反する。幸運にも法律に違.jpg
  \url{https://t.co/NYuhzgC9ee} 
\end{itemize}

〉〉〉 kk\_hironoのリツイート 〉〉〉

\begin{itemize}
\tightlist
\item
  RT
  kk\_hirono(刑事告発・非常上告_金沢地方検察庁御中)|s\_hirono(非常上告-最高検察庁御中\_ツイッター)
  日時:2021-06-07 06:53/2021/06/06 17:05 URL:
  \url{https://twitter.com/kk\_hirono/status/1401658598556532737} 
  \url{https://twitter.com/s\_hirono/status/1401450200489549827} 
  \textgreater{}
  2021-06-06-170420\_教皇ノースライム@noooooooorth個人的には弁護士会はまずは会員の具体的な利益のために活動するべきだと思う。個人だろうが会社だろうが.jpg
  \url{https://t.co/PodUA1en5f} 
\end{itemize}

〉〉〉 kk\_hironoのリツイート 〉〉〉

\begin{itemize}
\tightlist
\item
  RT
  kk\_hirono(刑事告発・非常上告_金沢地方検察庁御中)|s\_hirono(非常上告-最高検察庁御中\_ツイッター)
  日時:2021-06-07 06:53/2021/06/06 17:05 URL:
  \url{https://twitter.com/kk\_hirono/status/1401658614595559425} 
  \url{https://twitter.com/s\_hirono/status/1401450128255250433} 
  \textgreater{}
  2021-06-06-154809\_ほうたろう@lawyerhotaro無料相談なんてやめなされ。占い師なんて30分で2万円ですよ。少なくともみんな占い師以上の情報は提供できる.jpg
  \url{https://t.co/4vtkJFOFV5} 
\end{itemize}

〉〉〉 kk\_hironoのリツイート 〉〉〉

\begin{itemize}
\tightlist
\item
  RT
  kk\_hirono(刑事告発・非常上告_金沢地方検察庁御中)|s\_hirono(非常上告-最高検察庁御中\_ツイッター)
  日時:2021-06-07 06:53/2021/06/06 17:05 URL:
  \url{https://twitter.com/kk\_hirono/status/1401658642772893697} 
  \url{https://twitter.com/s\_hirono/status/1401450055689670660} 
  \textgreater{}
  2021-06-06-154742\_ほうたろう@lawyerhotaro無料相談なんてやめなされ。占い師なんて30分で2万円ですよ。少なくともみんな占い師以上の情報は提供できる.jpg
  \url{https://t.co/FhHxMtcEEe} 
\end{itemize}

 ほうたろうの2つのスクリーンショットはリツイートを忘れていたようですが、北周士弁護士のスクリーンショットはさらに異様な出現に思えました。

 昨日は2015年の8月12日を起点にほうたろうのスクリーンショットを探し、その過程でモトケンこと矢部善朗弁護士(京都弁護士会)の強烈なツイートのスクリーンショットを発見したのです。余り記憶にない内容でしたが、かなり衝撃的でした。その頃、その程度はけっこう普通だったとも思えます。

 繰り返して調べたのですがスクリーンショットが見つかりません。「汚物」とあったのが特に印象的だったので、そちらからツイートは探し出せそうです。

\begin{itemize}
\tightlist
\item
  ``汚物'' (from:motoken\_tw) - Twitter検索 / Twitter
  \url{https://twitter.com/search?lang=ja\&q=\%22\%E6\%B1\%9A\%E7\%89\%A9\%22\%20(from\%3Amotoken\_tw)\&src=typed\_query} 
\end{itemize}

 ツイート自体は上記の検索で見つけることができたのですが、データベースに未登録のツイートとなっていました。また、ツイートの表示のされ方が、昨日見かけたスクリーンショットとは違った印象を受けました。

〉〉〉 kk\_hironoのリツイート 〉〉〉

\begin{itemize}
\tightlist
\item
  RT
  kk\_hirono(刑事告発・非常上告_金沢地方検察庁御中)|t\_iori(高千穂
  伊織) 日時:2021-06-07 07:21/2015/09/02 10:42 URL:
  \url{https://twitter.com/kk\_hirono/status/1401665660661297157} 
  \url{https://twitter.com/t\_iori/status/638889808287281152} 
  \textgreater{}
  ツイッターが生まれる前からこのくらいの捏造こじつけやらかしてたから、モトケンさんも小倉弁護士専用サイト建てたんだと思っていた
  RT @motoken\_tw 小倉秀夫弁護士がデマッターになったのはいつからだろう?
  \url{https://t.co/m6gyWmmVfT} 
\end{itemize}

※ @kk\_hironoのアカウントがブロックされ,リツイートに失敗したツイート

\begin{itemize}
\tightlist
\item
  TW motoken\_tw(モトケン) 日時:2013/07/29 18:51:05 URL:
  \url{https://twitter.com/motoken\_tw/status/361785943897812992} 
  \textgreater{} @bwpotato
  あの人の場合は、土俵から出て行ってから、自分で勝手にいくつも土俵を作って、そこから汚物を投げてくるような人なので、対抗できる発信力がないと泣き寝入りをすることになります。
\end{itemize}

 ちょっと違ったものを読み込んだのですが、次が探していた内容のモトケンこと矢部善朗弁護士(京都弁護士会)のツイートです。昨日、スクリーンショットで最初に見たときは、文字制限ギリギリの長文のツイートに見えました。改行も入って見えました。

※ @kk\_hironoのアカウントがブロックされ,リツイートに失敗したツイート

\begin{itemize}
\tightlist
\item
  TW motoken\_tw(モトケン) 日時:2015/09/02 10:45:00 URL:
  \url{https://twitter.com/motoken\_tw/status/638890336232697856} 
  \textgreater{} @t\_iori
  あれは本家の雰囲気が荒れないための汚物専用ブログ。
\end{itemize}

 モトケンこと矢部善朗弁護士(京都弁護士会)が小倉秀夫弁護士のことを汚物と名指ししているのは余り記憶になかったのですが、長い間、熾烈なバトルのようなものを繰り返していたことはよく憶えています。

 「あれは本家の雰囲気が荒れないための汚物専用ブログ。」というツイートの内容ですが、これはモトケンこと矢部善朗弁護士(京都弁護士会)の過去のブログのことを知らないとさっぱりわからないと思います。「小倉ウォッチ」というようなブログ名でした。もう少しひねりがあったかもしれません。

\begin{itemize}
\tightlist
\item
  元検弁護士のつぶやき \url{http://www.yabelab.net/blog/} 
\end{itemize}

 2009年6月7日に停止し、レイアウトが崩れリンクは無効が多く、今は残骸のようになっている上記のモトケンこと矢部善朗弁護士(京都弁護士会)のブログですが、メニューに「小倉ヲチ」というリンクが残っています。これが小倉秀夫弁護士を批判する専用ブログとなっていました。

 Googleで「小倉ヲチ」を検索しても1ページにそれらしいのがあったのは、次の私のツイートだけでしたが、日本語や英語以外の言語で表示されました。

\begin{itemize}
\tightlist
\item
  TW kk\_hirono(刑事告発・非常上告_金沢地方検察庁御中) 日時:
  2015/09/03 00:55:01 URL:
  \url{https://twitter.com/kk\_hirono/status/639104248467185665} 
  \textgreater{}
  ここでモトケンこと矢部善朗弁護士(京都弁護士会)が本家というのもメインブログである「元検弁護士のつぶやき」のことかと思いますが、そちらの方でも小倉秀夫弁護士の批判をやっていて、それを見るだけでも十分すぎたので、小倉ヲチのブログの方はほとんど見ていませんでした。
\end{itemize}

 「ヲチ」というのはモトケンこと矢部善朗弁護士(京都弁護士会)独自のもので、2008年当時他では見ない表現というか言葉でしたが、ウォッチのことらしく、要は観察のことだと理解していました。

 小倉秀夫弁護士に限らず、モトケンこと矢部善朗弁護士(京都弁護士会)の槍玉となった批判の対象は、原因がよくわからないことが多いのですが、それがモトケンこと矢部善朗弁護士(京都弁護士会)の弁護士商売上の技術であり、職人芸のようなものと次第に捉えるようになっていきました。

 そういう打算とは別に時折、本気のようなツイートも見かけるのですが、昨日の夕方見かけたモトケンこと矢部善朗弁護士(京都弁護士会)のツイートは、まさにその類でした。午前中にも関連したツイートを1つ見かけていたのですが、そのときは気に掛けずにいました。

\begin{itemize}
\item
  TW motoken\_tw(モトケン) 日時: 2021/06/06 09:45:18 URL:
  \url{https://twitter.com/motoken\_tw/status/1401339378471948295} 
  \textgreater{}
  私はこういう客を見たら直ちに110番して警察官に来てもらいます。\\
  \textgreater{}
  このご時世で、アゴマスクで唾を吐き飛ばしながらわめき散らす輩は悪質な営業妨害かつ他の客に対する感染危険行為。
  \url{https://t.co/apKASyHbvP} 
\item
  TW motoken\_tw(モトケン) 日時: 2021/06/06 08:52:43 URL:
  \url{https://twitter.com/motoken\_tw/status/1401326146923290629} 
  \textgreater{} お店側の対応を支持します。\\
  \textgreater{}\\
  \textgreater{}
  亀戸の弁当店「キッチンDIVE」あごマスク男、菓子折持って意気消沈で来店(スポーツ報知)\\
  \textgreater{} \#Yahooニュース\\
  \textgreater{} \url{https://t.co/kwmg8P1t1b} 
  \textgreater{} \url{https://t.co/6YRKtsNYzQ} 
\end{itemize}

 昨日の8時52分のツイートを朝に見かけていて、夕方になってそのツイートを引用した9時45分のツイートを見かけていたようです。

 リンクの記事を読んで、そのあとTwitter検索でいくつか情報を見たのですが、厳しすぎるような印象を受けました。他にモトケンこと矢部善朗弁護士(京都弁護士会)のツイートで思い出したことがあり、埋め込みツイートの数を増やすので、続きは別のエントリーを作成するかたちとします。

\begin{itemize}
\tightlist
\item
  〈〈〈 2021/06/07 08:05:07 Linux Emacs: 〈〈〈
\end{itemize}

\hypertarget{ux4e80ux6238ux306eux5f01ux5f53ux5e97ux30adux30c3ux30c1ux30f3uxff44uxff49uxff56uxff45ux3042ux3054ux30deux30b9ux30afux7537ux83d3ux5b50ux6298ux6301ux3063ux3066ux610fux6c17ux6d88ux6c88ux3067ux6765ux5e97ux3068ux3044ux3046ux554fux984cux306bux5bfeux3059ux308bux30e2ux30c8ux30b1ux30f3ux3053ux3068ux77e2ux90e8ux5584ux6717ux5f01ux8b77ux58ebux4eacux90fdux5f01ux8b77ux58ebux4f1aux306eux53b3ux3057ux3055}{%
\paragraph{「亀戸の弁当店「キッチンDIVE」あごマスク男、菓子折持って意気消沈で来店」という問題に対するモトケンこと矢部善朗弁護士(京都弁護士会)の厳しさ}\label{ux4e80ux6238ux306eux5f01ux5f53ux5e97ux30adux30c3ux30c1ux30f3uxff44uxff49uxff56uxff45ux3042ux3054ux30deux30b9ux30afux7537ux83d3ux5b50ux6298ux6301ux3063ux3066ux610fux6c17ux6d88ux6c88ux3067ux6765ux5e97ux3068ux3044ux3046ux554fux984cux306bux5bfeux3059ux308bux30e2ux30c8ux30b1ux30f3ux3053ux3068ux77e2ux90e8ux5584ux6717ux5f01ux8b77ux58ebux4eacux90fdux5f01ux8b77ux58ebux4f1aux306eux53b3ux3057ux3055}}

\begin{itemize}
\tightlist
\item
  〉〉〉 Linux Emacs: 2021/06/07 08:18:26 〉〉〉
\end{itemize}

:CATEGORIES: @kanazawabengosi \#金沢弁護士会 @JFBAsns
日本弁護士連合会(日弁連) \#法務省 @MOJ\_HOUMU
\#モトケンこと矢部善朗弁護士(京都弁護士会)

\begin{itemize}
\tightlist
\item
  1407:2021-06-07\_08:05:41 \#告発状 \#\#\#\#
  2021年6月6日の夕方に読んだモトケンこと矢部善朗弁護士(京都弁護士会)のツイート、本質的に考え直すきっかけとなった矢部善朗弁護士の影響と責任
  \url{https://hirono-hideki.hatenadiary.jp/entry/2021/06/07/080538} 
\end{itemize}

 内容は上記のエントリーの続きになりますが、まずはモトケンこと矢部善朗弁護士(京都弁護士会)の2つのツイートを次に再掲しておきます。

\begin{itemize}
\item
  TW motoken\_tw(モトケン) 日時: 2021/06/06 09:45:18 URL:
  \url{https://twitter.com/motoken\_tw/status/1401339378471948295} 
  \textgreater{}
  私はこういう客を見たら直ちに110番して警察官に来てもらいます。\\
  \textgreater{}
  このご時世で、アゴマスクで唾を吐き飛ばしながらわめき散らす輩は悪質な営業妨害かつ他の客に対する感染危険行為。
  \url{https://t.co/apKASyHbvP} 
\item
  TW motoken\_tw(モトケン) 日時: 2021/06/06 08:52:43 URL:
  \url{https://twitter.com/motoken\_tw/status/1401326146923290629} 
  \textgreater{} お店側の対応を支持します。\\
  \textgreater{}\\
  \textgreater{}
  亀戸の弁当店「キッチンDIVE」あごマスク男、菓子折持って意気消沈で来店(スポーツ報知)\\
  \textgreater{} \#Yahooニュース\\
  \textgreater{} \url{https://t.co/kwmg8P1t1b} 
  \textgreater{} \url{https://t.co/6YRKtsNYzQ} 
\end{itemize}

 昨日の夕方、「キッチンDIVE」をTwitter検索したのですが、話題は余り見かけませんでした。ニュースのようなかたちにはなっているツイートがあって、動画もあったので視聴はしましたが、短いニュースで、いくつかの条件が揃ってニュースになったのかと思いました。

 さきほど「あごマスク男」をTwitterAPIで検索しましたが、これも乏しい数でした。今一度、ブラウザのTwitterで検索をしてみたいと思います。

〉〉〉 kk\_hironoのリツイート 〉〉〉

\begin{itemize}
\tightlist
\item
  RT
  kk\_hirono(刑事告発・非常上告_金沢地方検察庁御中)|divemamuru(キッチンDIVE
  24時間営業中 御徒町店プレオープン中) 日時:2021-06-07
  08:28/2021/06/01 17:07 URL:
  \url{https://twitter.com/kk\_hirono/status/1401682391878361089} 
  \url{https://twitter.com/divemamuru/status/1399638734018994178} 
  \textgreater{}
  緊急事態宣言中の深夜に泥酔しマスクもせず来店、スタッフに因縁をつけ暴言を吐く。
  スタッフの退勤時間を聞き出そうとし、更には殺ってやるからな!と脅迫。
  終いにはスタッフに現金を投げつける暴行を働いた2名の人物を絶対に許しません。
  既に警察署に被害届を提出しました。 \url{https://t.co/whGcmmrSaD} 
\end{itemize}

 被害に遭ったというお店のツイートのようです。被害に遭ったというお店は亀戸のお店とはなっていました。いくつかチェーン店があり、人気のようですが、この問題で初めて知ったように思います。200円から弁当があるとか、1kgの弁当に挑戦者求むという情報がありました。

〉〉〉 kk\_hironoのリツイート 〉〉〉

\begin{itemize}
\tightlist
\item
  RT
  kk\_hirono(刑事告発・非常上告_金沢地方検察庁御中)|Yamada888888888(バロン😈という名の日本人)
  日時:2021-06-07 08:42/2021/06/05 16:37 URL:
  \url{https://twitter.com/kk\_hirono/status/1401686025215246339} 
  \url{https://twitter.com/Yamada888888888/status/1401080697721294851} 
  \textgreater{} こんにちは こういうの本当に腹が立つ😈💦
  クレーマーがいるとお客さんがお店に来なくなるからマジ損害賠償ものなんだよね。
  将来の顧客\&売上を逃した分も含めて3億円くらい請求しよう😈✨
  この人達はお金持ちらしいよ☺️ニコ \#あごマスク客 \#あごマスク
  \#あごマスク男 \url{https://t.co/jVbwcHytMH} 
\end{itemize}

 「あごマスク男」のTwitter検索で動画を選択すると、上記のツイート1件だけが出てきました。YouTubeではなく、テレビ放送の録画ともはっきりしないのですが、初めてみた内容で、客のものすごい怒りようが伝わり、警察を呼べと完全に開き直っています。

 昨日の夕方に見かけた情報では泥酔とあったのですが、酒に酔っている様子には思えないというのが正直なところです。場所が離れているためもあり、店の人の声が聞きづらいですが、動画での再生が始まる前から、客が完全にブチ切れる、経過があったらしいことはうかがえます。

 あらためて店の側にも羹に懲りて膾を吹くという過剰な対応が、1つの誘引となっていたようにも思えるのですが、昨日か一昨日にもモトケンこと矢部善朗弁護士(京都弁護士会)のツイートで、羹に言及したツイートを見かけていたところでした。

\begin{itemize}
\tightlist
\item
  硬貨を投げる「あごマスク客」に店員が「私、奴隷じゃない」 弁当屋「カスハラ」騒動の意味
  - 弁護士ドットコム \url{https://t.co/QieEGDooAi}  ¥\n 2021年06月05日
  10時04分
\end{itemize}

 6月5日あるので一昨日ですが、初めて見出しをみたような弁護士ドットコムの記事です。ただ、「私、奴隷じゃない」という部分は昨日の夕方に見かけていた憶えがあります。

 記事を読み始める前に初めにある画像をみて思ったのですが、これはテレビの映像ではなく、この弁当店がネットで中継しているという動画で編集されたデザインが盛り込まれているようです。

\begin{quote}
《引用の始まり》
\end{quote}

\begin{quote}
200円弁当など、安さと量に定評のある24時間営業の弁当屋「キッチンDIVE」(東京・亀戸)で、来店した男性2人組が店員を怒鳴ったり、硬貨を投げつけたりする動画が話題だ。店側は警察に被害届を提出。示談にも応じないと公言している。

昨今は客からの暴言や暴行など、カスタマーハラスメント(カスハラ)にも注目が集まっている。

店主の伊藤慶さんは、「クレーマーや面倒くさい人は客じゃない。放っておいたら優良顧客が逃げてしまう」と語り、断固たる態度をとるべきと話す。迷惑客への対応を聞いた。(編集部・園田昌也)
\end{quote}

\begin{quote}
《引用の終わり》
\end{quote}

\begin{itemize}
\tightlist
\item
  硬貨を投げる「あごマスク客」に店員が「私、奴隷じゃない」 弁当屋「カスハラ」騒動の意味
  - 弁護士ドットコム \url{https://www.bengo4.com/c\_18/n\_13152/} 
\end{itemize}

 上記の引用部分に店主の声として「クレーマーや面倒くさい人は客じゃない。放っておいたら優良顧客が逃げてしまう」とありますが、これは1年ほど前まで、けっこう深澤諭史弁護士のツイートとして見かけていた話に酷似しています。最近は見かけないですが、クレーマーのまとめがあったはず。

\begin{itemize}
\tightlist
\item
  2021年05月02日12時26分の登録:
  REGEXP:''クレーマー''/深澤諭史(@fukazawas)の検索(2015-02-04〜2021-02-19/2021年05月02日12時26分の記録80件)
  \url{https://kk2020-09.blogspot.com/2021/05/regexpfukazawas2015-02-042021-02.html} 
\end{itemize}

 最新版ということになりそうですが、一月と5日ほど前の5月2日記録したまとめ記事がありました。きっかけとなるようなツイートは記憶にないですが、直近に出ている深澤諭史弁護士のツイートがどんなのか気になります。いちいち正確には記憶していられません。

 ストーカーもこのクレーマーというキーワードと同じですが、深澤諭史弁護士のツイートをみていると、弁護士の社会的危険性をまざまざと目の当たりにします。昨日辺りに見かけていたコメントにも弁当屋の件で、最悪、ストーカー殺人事件に発展しかねない、と言ったご意見がありました。

\begin{itemize}
\tightlist
\item
  (05/80) RT fukazawas(深澤諭史)|uwaaaa(サイ太) 日時:2016-11-05
  10:45:00 +0900/2016-11-05 10:40:00 +0900 URL:
  \url{https://twitter.com/fukazawas/status/794717310753718272} 
  \url{https://twitter.com/uwaaaa/status/794715897554272256\textgreater} {}
  行政訴訟の実務本、行政クレーマーへの対処法を書いておくべき。
\end{itemize}

 行政、自治体に弁護士に対する責任追及のあり方を考えてもらいたいというのも私の重要なコンセプトの1つとなっています。身近なところから能登町、金沢市、石川県になります。青果物の配達で市場急配センターが中心となる金沢中央卸売市場は、金沢市の管理になるという情報も早くから知っています。

\begin{itemize}
\tightlist
\item
  (06/80) RT fukazawas(深澤諭史)|stdaux(スドー) 日時:2016-11-08
  21:41:00 +0900/2016-11-08 20:00:00 +0900 URL:
  \url{https://twitter.com/fukazawas/status/795969486306164736} 
  \url{https://twitter.com/stdaux/status/795943980361883648\textgreater} {}
  「法律職は定形的だから人工知能に仕事奪われるんじゃないの」という話をされるたびに「おう、示談金のために被疑者の親類友人に連絡しまくって罵倒を浴びながら借金する機能や、意味不明な相手方クレーマーを宥めながら生活の悩みを聞いてやったりして落ち着かせる機能を早く実装してくれ」と思う
\end{itemize}

 深澤諭史弁護士がスドーこと平野敬弁護士のツイートをリツイートしているのは珍しくも思ったのですが、参考になるエントリーがあると思います。ご紹介しておきます。

\begin{itemize}
\tightlist
\item
  15:2019-12-30\_22:49:51 *
  コインハイブ事件の無罪判決にみる平野敬弁護士(第二東京弁護士会)の警察、検察批判のスタンス
  \url{https://hirono-hideki.hatenadiary.jp/entry/2019/12/30/224945} 
\item
  181:2020-02-08\_12:32:23 *
  「Twitterやってない裁判官にツイッターワールドの空気感を言語化して伝えるのに苦労してる。」という深澤諭史弁護士のリツイートの先は、明示のないコインハイブ事件の逆転有罪判決
  \url{https://hirono-hideki.hatenadiary.jp/entry/2020/02/08/123221} 
\end{itemize}

 エントリーナンバーが15番、181番として出てきました。15番が2019年12月30日、181番が2020年2月8日となっているので、時期と流れがわかりやすく感じました。

\begin{itemize}
\tightlist
\item
  (09/80) TW fukazawas(深澤諭史) 日時:2017-03-01 21:45:00 +0900
  URL:
  \url{https://twitter.com/fukazawas/status/836920206689607680\textgreater} {}
  工作教室の開催者の嘆き「無料開催の参加者はクレーマーだらけ。ゴミの片づけをする人も少ない。お金をいただくときは、こんなことはなかったのに・・・」
  - Togetterまとめ \url{https://t.co/ZN0vLNdtXL} 
  @togetter\_jpさんから\textgreater\textgreater{} 法律相談と同じ傾向が?
\end{itemize}

 無料にこだわり槍玉にあげて敵視するのも深澤諭史弁護士の特徴ですが、こんなかたちで無料とクレーマーがセットで出てくるのは意外なメニューでした。弁護士食堂です。

\begin{itemize}
\tightlist
\item
  (30/80) TW fukazawas(深澤諭史) 日時:2018-11-13 12:25:00 +0900
  URL:
  \url{https://twitter.com/fukazawas/status/1062184534966362112\textgreater} {}
  私はこれを徹底してる。\textgreater{}
  クレーマーに時間を費やすことは、他の顧客への重大な背信行為であり、それこそ、クレームをつけるべき行為だと思ってる。
  \url{https://t.co/rac4gslEH8} 
\end{itemize}

 それらしいツイートが出てきましたが、探しているのはストーカーとクレーマーがセットになった深澤諭史弁護士のツイートで、繰り返し深澤諭史弁護士自身によってリツイートされていたと思うのですが、悪魔合体のような社会的害悪と社会的危険性を感じる深澤諭史弁護士のツイートでした。

 さっそく次にあるツイートとして出てきました。編集的にこじらせた人とありますが、ストーカーを医者の治療に直接結びつけるのも深澤諭史弁護士の特徴です。

\begin{itemize}
\item
  (31/80) TW fukazawas(深澤諭史) 日時:2019-01-13 10:39:00 +0900
  URL:
  \url{https://twitter.com/fukazawas/status/1084263760489996288\textgreater} {}
  DV、ストーカー、クレーマー的な行動をしていて、そこで介入した弁護士に法的措置を取られて責任を取らされ、そういうことができなくなって、弁護士全般に逆恨みの憎悪を燃やしている人は一定程度いる。\textgreater{}
  偏執的にこじらせた人がいることと、まと・・・ \url{https://t.co/3rJWlx58cA} 
\item
  (33/80) TW fukazawas(深澤諭史) 日時:2019-02-18 09:45:00 +0900
  URL:
  \url{https://twitter.com/fukazawas/status/1097295949523611648\textgreater} {}
  そうです。クレーマー、DV加害者、ストーカーに早めに厳正対処することは、自分の身を守るためでもありますが、逆説的ですが、加害者のためでもあります。\textgreater{}
  早めに加害の芽を摘み取ることで、彼らも重罪を犯さずに済むからです。
  \url{https://t.co/9sbPFpvxWN} 
\end{itemize}

 ちょっと忘れていましたが、「クレーマー、DV加害者、ストーカー」ということで、DV加害者も一緒になって出てきました。

\begin{itemize}
\item
  (34/80) TW fukazawas(深澤諭史) 日時:2019-03-25 11:02:00 +0900
  URL:
  \url{https://twitter.com/fukazawas/status/1109999004169850881\textgreater} {}
  \url{https://t.co/uoA2YhEcGX\textgreater} {}
  クレーマー対応を弁護士業務としてやることがあるけれども、これは真理ですね。\textgreater{}
  クレーマー排除は、権利ではなくて義務。\textgreater{}
  勤務先とほかの顧客に対する重大な義務だから、クレーマーに従うことは、重大な背信であるという発想の転換が大事(・∀・)
\item
  (35/80) TW fukazawas(深澤諭史) 日時:2019-03-25 11:57:00 +0900
  URL:
  \url{https://twitter.com/fukazawas/status/1110012701206470656\textgreater} {}
  1人のクレーマーを納得させるエネルギーで、10人20人の顧客を感動させることができます。ほんと。\textgreater{}
  (・∀・)(^ω^) \url{https://t.co/cFecB605Mc} 
\end{itemize}

 弁護士鉄道の記録資料として珠玉のような深澤諭史弁護士のツイートが目白押しですが、先に進めます。

\begin{itemize}
\item
  (63/80) TW fukazawas(深澤諭史) 日時:2019-12-18 16:30:00 +0900
  URL:
  \url{https://twitter.com/fukazawas/status/1207201352641732608\textgreater} {}
  クレーマーを全員出禁にしたらまともな客が3倍以上・・・店長に称賛集まる
  \#ldnews \url{https://t.co/tLS4Zv7c1A\textgreater} {}
  顧問先にも強調しているが、「クレーマー1名を納得させるエネルギーで10名の顧客を感動させられる。そっちにエネルギーを注いだ方が得。」
\item
  (67/80) TW fukazawas(深澤諭史) 日時:2020-06-12 12:05:00 +0900
  URL:
  \url{https://twitter.com/fukazawas/status/1271277420889337863\textgreater} {}
  DV加害者とかストーカーとか、モンスタークレーマーなんかは、その典型だったりしますね。\textgreater{}
  別に弁護士に敵対的な感情(妄想)を抱く人の数は、他の職業よりも特別に多いとは思わないですが、偏執的、攻撃性の高いカテゴリの人にそういう感情を持・・・
  \url{https://t.co/pWBIgSFEQz} 
\item
  (80/80) TW fukazawas(深澤諭史) 日時:2021-02-19 12:17:00 +0900
  URL:
  \url{https://twitter.com/fukazawas/status/1362602197549076484\textgreater} {}
  ですよね。\textgreater{}
  (・∀・)弁護士は、クレーマー、ストーカー代行業でもないですし。
  \url{https://t.co/n59GIordqr} 
\end{itemize}

 弁護士ドットコムの記事の続きを読むことにします。埋め込みツイートもかなり使ってしまいました。

\begin{quote}
《引用の始まり》
\end{quote}

\begin{quote}
この時間帯、勤務していた店員は男女1名ずつ。キッチンDIVEでは、コロナ以前はセルフの電子レンジを置いていたが、時間帯によっては客の列ができていたため、密集を避けるために撤去していた。客の求めに対し、男性店員はレンジは置いていないと答えた。

その後、商品のキャンセルなどをめぐり、2人組が「お前、態度悪すぎるぞ」などと語気を強めはじめ、「だからそんな(弁当屋の)仕事してんねん、アホ」「俺、お前の年収、一カ月で儲けてるから、クズ」などと罵倒。店員側も反論したことで、2人組がヒートアップし、大声で怒鳴ったり、お金を投げつけたりする行為に及んだ。
\end{quote}

\begin{quote}
《引用の終わり》
\end{quote}

\begin{itemize}
\tightlist
\item
  硬貨を投げる「あごマスク客」に店員が「私、奴隷じゃない」 弁当屋「カスハラ」騒動の意味
  - 弁護士ドットコム \url{https://www.bengo4.com/c\_18/n\_13152/} 
\end{itemize}

\begin{quote}
《引用の始まり》
\end{quote}

\begin{quote}
映像によると、2人組のうちひとりは接客業を自称し、「こんな口のききかたするか、客に向かって。客じゃないってこと? するんかって、お前」などと怒鳴りつけている。

これに対し、女性店員が「お客さんじゃないんで、出て行ってもらえますか」と応じたところ、男性は返品時に受け取った硬貨を投げつけ、「金払ったら客だろうが、コラ、オイ」などと声を荒げている。

女性店員はひるむことなく、「お客さんじゃないです」「(お金)払ってもらってないです。投げただけです」と返している。

伊藤さんは、「店員には、場合によっては客を出禁にして良いと伝えてある。お客さんの代わりはいるけど、スタッフの代わりはいない。スタッフを採用するほうが難しいし、コストがかかる」と強調する。
\end{quote}

\begin{quote}
《引用の終わり》
\end{quote}

\begin{itemize}
\tightlist
\item
  硬貨を投げる「あごマスク客」に店員が「私、奴隷じゃない」 弁当屋「カスハラ」騒動の意味
  - 弁護士ドットコム \url{https://www.bengo4.com/c\_18/n\_13152/} 
\end{itemize}

 上記の「2人組のうちひとりは接客業を自称し、」という部分をみて、昨日、飲食業の経営をしているような人という情報を見かけていたことを思い出しました。まったく逆の立場でお客に対する不満を募らせることもあったのかと想像しますが、記事の伝え方に方向性があるようにも感じました。

\begin{quote}
《引用の始まり》
\end{quote}

\begin{quote}
中には店員の対応に疑問を持つ人もいるだろう。しかし、店員なら侮辱に耐えなくてはならないのだろうか。「お前の給料はここで買ったお金で払われてんだろ」となじられた女性店員は「私、奴隷じゃないです」と反論している。怒鳴り、侮辱し、物を投げつけたほうに非があるのは明らかだ。
\end{quote}

\begin{quote}
《引用の終わり》
\end{quote}

\begin{itemize}
\tightlist
\item
  硬貨を投げる「あごマスク客」に店員が「私、奴隷じゃない」 弁当屋「カスハラ」騒動の意味
  - 弁護士ドットコム \url{https://www.bengo4.com/c\_18/n\_13152/} 
\end{itemize}

 記事を読んでいるうちに忘れていましたが、「私、奴隷じゃない」という言葉が出た辺りの説明が上記の引用部分にありました。風俗嬢を守ると言いながら、最果ての救いがない底辺の仕事をしているように物語る人物もいて、過去に弁護士との仲良し関係を印象づける発言もありました。

 この弁護士ドットコムの記事を読み始めた後になるのか、北周士弁護士が前に飲食店の無断キャンセルが社会問題になっているときに、専用の救済窓口のようなものを開設していたことがありました。今もそのままあるのかもしれないですが、他のと一緒で、見かけなくなっています。

2021年06月07日10時02分の実行記録: twitterAPI-search-lawList-mydql-add.rb
``無断キャンセル'' ツイート数:2/2453 リツイート数:1/2453
トータル:4509``無断キャンセル''の該当: hirono\_hideki 0/0件 kk\_hirono
1/0件 s\_hirono 0/0件

 検索をすると新型コロナウイルスの予防接種に関する無断キャンセルのツイートばかり出てきましたが、意外に少なく4509件でストップしました。

\begin{itemize}
\item
  2021年06月07日05時04分の登録:
  REGEXP:''私選弁護人太田圭一先生、連絡して下さい''/データベース登録済みツイートの検索:2021-06-06〜2021-06-06/2021年06月07日05時04分の記録:ユーザ・投稿:8/8件
  \url{https://kk2020-09.blogspot.com/2021/06/regexp2021-06-062021-06-0620210607050488.html} 
\item
  2021年06月07日05時31分の登録:
  \弁護士落合洋司?感染拡大を招く東京(頭狂)オリンピック中止!? @yjochi\恐怖と絶望と多種多様なウイルスを世界に拡散するだけだわね。
  \url{https://kk2020-09.blogspot.com/2021/06/yjochi\_7.html} 
\item
  2021年06月07日05時32分の登録:
  \弁護士落合洋司?感染拡大を招く東京(頭狂)オリンピック中止!? @yjochi\自分たちさえスポーツができて金が儲かれば、他人が死んでもいいって、最低の奴らだよね。
  \url{https://kk2020-09.blogspot.com/2021/06/yjochi\_12.html} 
\item
  2021年06月07日07時12分の登録:
  REGEXP:''汚物''/モトケン(@motoken\_tw)の検索(2015-10-25〜2021-03-17/2021年06月07日07時12分の記録9件)
  \url{https://kk2020-09.blogspot.com/2021/06/regexpmotokentw2015-10-252021-03.html} 
\item
  2021年06月07日07時16分の登録: %@motoken\_tw モトケン%@t\_iori¥\n
  あれは本家の雰囲気が荒れないための汚物専用ブログ。
  \url{https://kk2020-09.blogspot.com/2021/06/motokentwtiorin.html} 
\item
  2021年06月07日08時09分の登録:
  REGEXP:''羹''/モトケン(@motoken\_tw)の検索(2012-10-23〜2020-08-20/2021年06月07日08時09分の記録3件)
  \url{https://kk2020-09.blogspot.com/2021/06/regexpmotokentw2012-10-232020-08.html} 
\item
  2021年06月07日08時11分の登録:
  REGEXP:''羹''/モトケン(@motoken\_tw)の検索(2010-05-14〜2020-08-20/2021年06月07日08時11分の記録8件)
  \url{https://kk2020-09.blogspot.com/2021/06/regexpmotokentw2010-05-142020-08.html} 
\item
  2021年06月07日08時11分の登録:
  REGEXP:''汚物''/モトケン(@motoken\_tw)の検索(2013-07-29〜2021-03-17/2021年06月07日08時11分の記録11件)
  \url{https://kk2020-09.blogspot.com/2021/06/regexpmotokentw2013-07-292021-03.html} 
\item
  2021年06月07日08時16分の登録:
  REGEXP:''あごマスク男''/データベース登録済みツイートの検索:2021-06-04〜2021-06-06/2021年06月07日08時16分の記録:ユーザ・投稿:2/2件
  \url{https://kk2020-09.blogspot.com/2021/06/regexp2021-06-042021-06-0620210607081622.html} 
\item
  2021年06月07日10時04分の登録:
  REGEXP:''無断キャンセル''/データベース登録済みツイートの検索:2014-08-30〜2021-06-07/2021年06月07日10時04分の記録:ユーザ・投稿:9/14件
  \url{https://kk2020-09.blogspot.com/2021/06/regexp2014-08-302021-06.html} 
\item
  (01/14) TW @fukazawas(深澤諭史) 日時: 2014-08-30 10:21:00 +0900
  URL:
  \url{https://twitter.com/fukazawas/status/505525655388168192\textgreater} {}
  弁護士があえて無料相談を受ける理由は,いろいろあるが,一つに,\textgreater\textgreater{}
  「相談料を気にして相談せずに放置すると取り返しがつかなくなる可能性の高い事案」\textgreater\textgreater{}
  というものがある。\textgreater{}
  無料相談を無断キャンセルする人たちは,そのあたりは分かっているのだろうか。
\end{itemize}

 ちょっと想定外の深澤諭史弁護士のツイートがでてきましたが、無料相談の無断キャンセルとあります。深澤諭史弁護士の態度で、うさんくさい、時間の無駄になると思ってやめた、相手にする価値なしと判断された可能性というのもあるのかもしれません。

 4月の終わりあたりからTwitterの更新がなくなり、10日ほど前かに再開はしたものの一日に1,2件ほどのリツイートがあるだけとなっている下火状態の深澤諭史弁護士のタイムラインですが、以前は最大級の弁護士ハザードの危険信号が鳴り響いていたものだと改めて思います。

\begin{itemize}
\tightlist
\item
  (08/14) TW @lawkus(ystk) 日時: 2019-11-12 05:57:00 +0900 URL:
  \url{https://twitter.com/lawkus/status/1193996090577539072\textgreater} {}
  「居酒屋無断キャンセル、偽計業務妨害容疑で59歳男逮捕」という見出しの産経の記事が流れてきた。中身を読むと偽名で団体予約をしてからばっくれた、他店でも同じ偽名で同様事案が発生している、というもの。単なる無断キャンセルで検挙されるはずもなく、産経のはフェイクニュース級の釣り見出し。
\end{itemize}

 三浦義隆弁護士のツイートですが、人生の落とし穴に誘引するような引力を感じることがしばしばです。弁護士にすればそれが獲物でもあるのかと考えさせられ、勉強もさせてもらったようには思えます。陥穽というほどではないと思いますが、男女関係でもトラブルを招き寄せる発言と感じてきました。

 過去に記録したものがあったと思いますが、処女という言葉に、三浦義隆弁護士の本質が現れているとも感じたツイートがあります。

\begin{itemize}
\tightlist
\item
  2018年02月08日20時06分の登録:
  REGEXP:''処女''/ystk(@lawkus)の検索(2011-09-19〜2017-06-11/2018年02月08日20時06分の記録62件)
  \url{http://hirono2014sk.blogspot.com/2018/02/regexpystklawkus2011-09-192017-06.html} 
\end{itemize}

 2018年に作成したまとめ記事が1つあるだけでしたが、当時のものは書式に問題があるので、新規に作り直すことにします。

\begin{itemize}
\tightlist
\item
  ``処女'' (from:lawkus) - Twitter検索 / Twitter
  \url{https://twitter.com/search?lang=ja\&q=\%22\%E5\%87\%A6\%E5\%A5\%B3\%22\%20(from\%3Alawkus)\&src=typed\_query} 
\end{itemize}

 数えたところ38件のツイートがあって、2019年より後のツイートを2件ほどデータベースに追加しました。このTwitterの検索は精度が悪く、出てこないツイートもありそうです。問題になっているのも見たことはないのですが、精度が改善されたという話も見ていません。

\begin{itemize}
\tightlist
\item
  2021年06月07日10時30分の登録:
  REGEXP:''処女''/ystk(@lawkus)の検索(2011-09-19〜2019-06-19/2021年06月07日10時30分の記録64件)
  \url{https://kk2020-09.blogspot.com/2021/06/regexpystklawkus2011-09-192019-06.html} 
\end{itemize}

 64件となっていますが、あるいはリツイートが含まれているかもしれません。ただ、2018年02月08日20時06分の記録62件から2つ増えた2021年06月07日10時30分の記録64件なので、先程の追加で帳尻は合います。

 ご紹介は次のツイートだけで十分かと思います。

\begin{itemize}
\tightlist
\item
  (04/64) TW lawkus(ystk) 日時:2012-01-21 21:59:00 +0900 URL:
  \url{https://twitter.com/lawkus/status/160708205662584832\textgreater} {}
  処女信仰とか気持ち悪いと思う一方,処女に多少の価値は認めていて,「処女ともやった上で〈処女なんて面倒臭いだけだよ〉とか言える俺」に価値を見出すというのが,最大公約数的な立場なのではないかと思うね。
\end{itemize}

 無断キャンセルのツイートに戻ります。さっそくでしたがまた、三浦義隆弁護士のツイートができました。

\begin{itemize}
\tightlist
\item
  (09/14) TW @lawkus(ystk) 日時: 2019-11-12 06:15:00 +0900 URL:
  \url{https://twitter.com/lawkus/status/1194000775216685056\textgreater} {}
  ほら、産経が釣り見出し書くから、早速「居酒屋を無断キャンセルすると逮捕される」と勘違いする人が出現してる。
  \url{https://t.co/nqOKtkv7ma} 
\end{itemize}

 私は以前、小倉秀夫弁護士に対しては返信のツイートで、「みんな不幸になれ、金持って来い!」とTwitterで叫んでいるようにみえると言ったことがありました。最近は以前のような刺激的な挑発とも思えるツイートというのは見かけなくなっています。その分、私としても平静を保っていられます。

\begin{itemize}
\tightlist
\item
  (14/14) TW @kk\_hirono(刑事告発・非常上告_金沢地方検察庁御中)
  日時: 2021-06-07 10:00:52 +0900 URL:
  \url{https://twitter.com/kk\_hirono/status/1401705687483179008\textgreater} {}
  この弁護士ドットコムの記事を読み始めた後になるのか、北周士弁護士が前に飲食店の無断キャンセルが社会問題になっているときに、専用の救済窓口のようなものを開設していたことがありました。今もそのままあるのかもしれないですが、他のと一緒で、見かけなくなっています。
\end{itemize}

 最後の14件目がさきほどの私のツイートでしたが、それと関係した内容のツイートは見当たりませんでした。

〉〉〉 kk\_hironoのリツイート 〉〉〉

\begin{itemize}
\tightlist
\item
  RT
  kk\_hirono(刑事告発・非常上告_金沢地方検察庁御中)|s\_hirono(非常上告-最高検察庁御中\_ツイッター)
  日時:2021-06-07 10:46/2019/12/18 09:07 URL:
  \url{https://twitter.com/kk\_hirono/status/1401717105431515136} 
  \url{https://twitter.com/s\_hirono/status/1207089974455095296} 
  \textgreater{}
  2019-12-06\_085028_テレビの画面・スッキリ・「無断キャンセル」問題にくわしい 北周士弁護士.jpg
  \url{https://t.co/8XXJMFVVHs} 
\end{itemize}

 記憶になかったのですが、北周士弁護士が紹介されている番組の画面を撮影した写真が出てきました。

〉〉〉 kk\_hironoのリツイート 〉〉〉

\begin{itemize}
\tightlist
\item
  RT
  kk\_hirono(刑事告発・非常上告_金沢地方検察庁御中)|noooooooorth(教皇ノースライム)
  日時:2021-06-07 10:50/2019/08/01 10:56 URL:
  \url{https://twitter.com/kk\_hirono/status/1401718101805129733} 
  \url{https://twitter.com/noooooooorth/status/1156745575053664256} 
  \textgreater{}
  飲食店「無断キャンセル」で弁護士が料金回収する新サービス登場・・・撲滅につながる?詳細を聞いた
  - FNNプライムオンライン \url{https://t.co/MAmxHQ3IJz}  \#FNN
  ノーキャンドットコムについて取材を受けました。
\end{itemize}

〉〉〉 kk\_hironoのリツイート 〉〉〉

\begin{itemize}
\tightlist
\item
  RT
  kk\_hirono(刑事告発・非常上告_金沢地方検察庁御中)|noooooooorth(教皇ノースライム)
  日時:2021-06-07 10:50/2020/01/20 21:55 URL:
  \url{https://twitter.com/kk\_hirono/status/1401718173129351168} 
  \url{https://twitter.com/noooooooorth/status/1219242099242962945} 
  \textgreater{}
  TBSのNスタから旅館の無断キャンセルについての取材を受けました。
  \url{https://t.co/SrtfBMKqR9} 
\end{itemize}

〉〉〉 kk\_hironoのリツイート 〉〉〉

\begin{itemize}
\tightlist
\item
  RT
  kk\_hirono(刑事告発・非常上告_金沢地方検察庁御中)|noooooooorth(教皇ノースライム)
  日時:2021-06-07 10:50/2020/01/18 16:45 URL:
  \url{https://twitter.com/kk\_hirono/status/1401718264414179328} 
  \url{https://twitter.com/noooooooorth/status/1218439270181900289} 
  \textgreater{}
  正月の宿泊、無断キャンセルで被害100万円超・・・同じ名前で5施設予約
  : 国内 : ニュース : 読売新聞オンライン \url{https://t.co/RNLQQqflax} 
  これはかなり悪質ですなぁ・・・。
\end{itemize}

〉〉〉 kk\_hironoのリツイート 〉〉〉

\begin{itemize}
\item
  RT
  kk\_hirono(刑事告発・非常上告_金沢地方検察庁御中)|noooooooorth(教皇ノースライム)
  日時:2021-06-07 10:50/2019/08/30 08:35 URL:
  \url{https://twitter.com/kk\_hirono/status/1401718272702091272} 
  \url{https://twitter.com/noooooooorth/status/1167219145038938112} 
  \textgreater{}
  昨日のスーパーJチャンネルでノーキャンドットコムについての取材を受けました。動画も上がってますね。
  無断キャンセル客は許さない!飲食店依頼で訴訟も \url{https://t.co/xPqk8lmA6b} 
\item
  \begin{enumerate}
  \def\labelenumi{(\arabic{enumi})}
  \setcounter{enumi}{9}
  \tightlist
  \item
    ``無断キャンセル'' (from:noooooooorth) - Twitter検索 / Twitter
    \url{https://t.co/qvGdRqJfc4} 
  \end{enumerate}
\end{itemize}

 また4つ埋め込みツイートを使ってしまいました。数が増えすぎると読み込みに時間がかかったり表示されないことがあります。画面にNスタとなっていましたが、スタジオに見えました。これまで北周士弁護士のテレビのスタジオ出演というのは観ていなかったと思います。

 「ノーキャンドットコム」ですが、すっかり忘れていたもので、最初はキャンペーンの意味なのかと思ったのですが、キャンセルを意味するのでしょう。

\begin{itemize}
\tightlist
\item
  2021年06月07日10時54分の登録:
  REGEXP:''ノーキャンドットコム''/データベース登録済みツイートの検索:2019-12-06〜2021-06-07/2021年06月07日10時54分の記録:ユーザ・投稿:3/4件
  \url{https://kk2020-09.blogspot.com/2021/06/regexp2019-12-062021-06-0720210607105434.html} 
\end{itemize}

 北周士弁護士の場合、根は真面目で頭のいい人だと思っているのですが、北周士弁護士に限らず頭の良いはずの弁護士が、放射能汚染で誕生したというゴジラをもじったベン・ゴシラというアイコンで、社会に汚染をばらまくツイートを日夜繰り返しているのに、危機感はないのかと不思議に思っていました。

 時刻は11時08分です。さきほどモトケンこと矢部善朗弁護士(京都弁護士会)のタイムラインを見に行ったところ、トレンドに、また弁護士がからんでいそうなニュースを発見しました。トヨタ社長と、一転和解というキーワードがあります。リンクを開くとお馴染のトヨタ自動車の社長の名前が出てきました。

 弁当屋の問題もまだ調べて記述しておきたい物足りなさがあるのですが、時間は待ってくれないので先を急ぎたいと思います。

\begin{itemize}
\tightlist
\item
  〈〈〈 2021/06/07 11:12:21 Linux Emacs: 〈〈〈
\end{itemize}

\hypertarget{section-1}{%
\paragraph{}\label{section-1}}

\hypertarget{ux91d1ux6ca2ux4e2dux8b66ux5bdfux7f72}{%
\subsubsection{金沢中警察署}\label{ux91d1ux6ca2ux4e2dux8b66ux5bdfux7f72}}

\hypertarget{ux5922ux306bux51faux3066ux304dux305fux7a81ux7136ux306eux89e3ux96c7ux901aux77e5ux5e73ux621021ux5e74ux5f53ux6642ux306eux91d1ux6ca2ux4e2dux8b66ux5bdfux7f72ux3068ux306eux95a2ux4fc2ux6027ux3046ux306eux5b57ux3068ux3044ux3046ux533fux540dux5f01ux8b77ux58ebux306eux30c4ux30a4ux30fcux30c8ux306eux5f71ux97ffux304bux3082ux3057ux308cux306aux3044ux5922}{%
\paragraph{夢に出てきた突然の解雇通知、平成21年当時の金沢中警察署との関係性、うの字という匿名弁護士のツイートの影響かもしれない夢}\label{ux5922ux306bux51faux3066ux304dux305fux7a81ux7136ux306eux89e3ux96c7ux901aux77e5ux5e73ux621021ux5e74ux5f53ux6642ux306eux91d1ux6ca2ux4e2dux8b66ux5bdfux7f72ux3068ux306eux95a2ux4fc2ux6027ux3046ux306eux5b57ux3068ux3044ux3046ux533fux540dux5f01ux8b77ux58ebux306eux30c4ux30a4ux30fcux30c8ux306eux5f71ux97ffux304bux3082ux3057ux308cux306aux3044ux5922}}

\begin{itemize}
\tightlist
\item
  〉〉〉 Linux Emacs: 2021/06/13 10:50:59 〉〉〉
\end{itemize}

:CATEGORIES: @kanazawabengosi \#金沢弁護士会 @JFBAsns
日本弁護士連合会(日弁連) \#法務省 @MOJ\_HOUMU \#金沢中警察署 \#うの字

 金沢中警察署をテーマにエントリーを作成するのは今回が初めてになります。まだ中途段階なのですが金沢弁護士会をテーマに3人で900万円の国選弁護報酬を請求し最高裁まで争ったという鳥取の強盗殺人事件について取り上げていたところでした。

\begin{itemize}
\tightlist
\item
  1417:2021-06-12\_15:17:40 \#告発状 \#\#\#\#
  鳥取が舞台の弁護士劇場「「無罪を勝ち取った国選弁護人3人が法テラスに対する報酬請求」の引用ツイート」に思う思惑と打算の弁護士本線
  \url{https://hirono-hideki.hatenadiary.jp/entry/2021/06/12/151737} 
\end{itemize}

 昨日6月12日の午後のことですが、拘置所で購入して読んだ山陰本線の本を探したが見つからず、目的以外の発見がけっこうな数ありました。また、やりたい作業が増えたことになりますが、その分、予定していた記述を減らして作業における時間の割当を考慮しなければならないことになりました。

 昨日の発見として、平成4年11月10日を指定した初公判の期日の書面があったのですが、午前10時半からとなっていました。夕方に姿を見たという感覚があった濱田武律裁判長でしたが、被害者安藤文さんの兄が法廷で暴れたのも、この11月10日の初公判だった可能性が高く思えてきました。

 私の場合、昼食時間を挟んだ公判というのは一度も経験がなく、拘置所に戻ってから昼食となっていました。被害者安藤文さんの兄が暴れたのは退廷のときですが、今思えば、濱田武律裁判長の「心神喪失なら無罪じゃないか」という言葉が、被害者安藤文さんの兄を刺激していた可能性がありそうです。

 思えば、この濱田武律裁判長の公判の記録、他に公判調書などとなっているものが全く見つからないのですが、被告発人木梨松嗣弁護士や被告発人長谷川紘之弁護士が意図的に除外した可能性がありそうです。

 他に昨日は、重複した書面もいくつか見つけたのですが、保存状態がずいぶんと違っていました。新たな発見に思えたものが1つあって、山口成良金沢大学教授が証人として出廷したときの公判調書のようなものです。まだ軽くしか目を通していないので詳しくはわかっていません。

 私の記憶では最後の法廷での審理というか判決前の公判となっているのですが、そのときも安藤健次郎さんと被害者安藤文さんの母親の姿があり、母親は見たのも初めてでしたが、出入り口から入った傍聴席の一番手前に二人並んで座っていました。裁判官から見ると法廷の一番奥になります。

 他のことに気を取られ、しっかり推移を記録してこなかったのが悔やまれる被告発人小島裕史裁判長の控訴審ですが、そういう当時のことを思い浮かべながら目にしたのが、昨夜のうの字のタイムラインでの3,4つ連続したようなツイートでした。

\begin{itemize}
\item
  〉〉〉 アカウント(@un\_co\_the2nd)は,@kk\_hironoをブロックしています。リツイートできませんでした。
  〉〉〉 ¥\n ¥\n \url{https://t.co/oIfyAweuT6} 
\item
  〉〉〉 アカウント(@un\_co\_the2nd)は,@kk\_hironoをブロックしています。リツイートできませんでした。
  〉〉〉 ¥\n ¥\n \url{https://t.co/2W7NArydW4} 
\item
  〉〉〉 アカウント(@un\_co\_the2nd)は,@kk\_hironoをブロックしています。リツイートできませんでした。
  〉〉〉 ¥\n ¥\n \url{https://t.co/DeGxBIoqW0} 
\item
  〉〉〉 アカウント(@un\_co\_the2nd)は,@kk\_hironoをブロックしています。リツイートできませんでした。
  〉〉〉 ¥\n ¥\n \url{https://t.co/UgFimqteEj} 
\end{itemize}

※ @kk\_hironoのアカウントがブロックされ,リツイートに失敗したツイート

\begin{itemize}
\tightlist
\item
  TW un\_co\_the2nd(うの字を名乗る💩物) 日時:2021/06/12 19:17:16 URL:
  \url{https://twitter.com/un\_co\_the2nd/status/1403657648822571010} 
  \textgreater{}
  ビジネス界でブイブイ言わせて年に何千万円もらうのもいいけど、人間のクソみてえな有様をいっぱい見るのもバッヂあってこその道なので、そういう方向に進む優秀な人がいても・・・いいと思うの・・・・・・おいでよクソみてえな町弁業へ・・・
\end{itemize}

※ @kk\_hironoのアカウントがブロックされ,リツイートに失敗したツイート

\begin{itemize}
\tightlist
\item
  TW un\_co\_the2nd(うの字を名乗る💩物) 日時:2021/06/12 19:20:01 URL:
  \url{https://twitter.com/un\_co\_the2nd/status/1403658340098473993} 
  \textgreater{}
  人間の醜さを煮詰めたようなクソみてえな事件、面白い(語弊)よ・・・
\end{itemize}

※ @kk\_hironoのアカウントがブロックされ,リツイートに失敗したツイート

\begin{itemize}
\tightlist
\item
  TW un\_co\_the2nd(うの字を名乗る💩物) 日時:2021/06/12 19:26:15 URL:
  \url{https://twitter.com/un\_co\_the2nd/status/1403659907350491137} 
  \textgreater{}
  守秘義務的に、そんなクソみてえなクソ事件のどこがどう面白いのかの説明ができねえ・・・カネに勝てない・・・
\end{itemize}

※ @kk\_hironoのアカウントがブロックされ,リツイートに失敗したツイート

\begin{itemize}
\tightlist
\item
  TW un\_co\_the2nd(うの字を名乗る💩物) 日時:2021/06/12 19:25:09 URL:
  \url{https://twitter.com/un\_co\_the2nd/status/1403659629771427844} 
  \textgreater{} 起きて半畳寝て一畳、天下取っても二合半ぞ・・・
  \url{https://t.co/ElzcWe9gZg} 
\end{itemize}

 夢のことですが、仕事をしていた工場でタイムカードを探していて見つからず、社長のような人物が近づいてきて、解雇だと言われ、理由を尋ねるとネットでなんでも公開していることだと言われました。

 法的手段に訴えるという裁判の話をしても相手にされなかったのですが、これから警察に相談に行くと告げると、後ろをずっとついてきて、つきまとうように翻意を求め何度も繰り返してきました。

 ネットでも書いたことがあると思うのですが、仕事で厳しい指導をし、工場の責任者のような人に注意され、さらに自分の心配をしろと強く言われ、そのときに聞いたのが警察が私のことで会社に話を聞きにしたということでした。金沢中警察署とは聞かなかったかもしれないですが、経緯がありました。

 それが平成20年の年末になるのではと思うのですが、そのあとリーマンショックで仕事が減り、翌年の平成21年3月15日で仕事をやめることになったのです。この3月15日というのは軽四の車検が切れる日で、その日に合わせた仕事納めで、給料の締切日までは有給扱いにしてもらっていました。

 能登半島地震の前の年になるので、平成18年の10月ですが、被害者安藤文さんの自宅を訪問したことで、金沢中警察署に連れて行かれ、ストーカー扱いされ誓約書を書かされたということがありました。そのとき担当したのが山出警部補です。

 山出警部補ですが、その時点で2年前からと言っていたかもしれません。確認はしませんが、はてなダイアリーには記憶の新しい段階で詳しい記述があるはずです。毎月1回はとも言っていたかもしれないですが、被害者安藤文さんの自宅に行き、安藤健次郎さんと膝を突き合わせて話していると言っていました。

 あとで考えれば、捜査をやりやすくするために外形的な事実通り、私をストーカーという前提にしたのかもしれないですが、私は妥協の余地がないほど頭に来ていました。そういう警察に対する不満もはてなダイアリーでぶつけていたのかもしれません。

 弁護士に魔法に掛けられ踊らされていたと思うのが、警察に対する不信感と不満です。その弁護士の正体について、わかりやすく感じた1人がうの字というTwitterアカウントになります。衝撃的な存在ですが、法クラという仲間内では違和感なく受け入れられ、打ち解けているようです。

 人気者とも思えるうの字ですが、特に注目してきたのが坂本正幸弁護士との関係性です。昨日、今日といくつかツイートのやりとりをみたのですが、以前にも増して仲の良い師弟関係のようなものを感じました。他の法クラもそうですが、リアルな付き合いがありそうです。

 どこのタイムラインだったかはっきりしませんが、それらしい内容の中村元弥弁護士のツイートをみかけました。いわぽんこと岩田圭只弁護士のタイムラインだったように思えてきました。まとめ記事を作成しているので、そちらで確認してみます。

\begin{itemize}
\item
  奉納\危険生物・弁護士脳汚染除去装置\金沢地方検察庁御中\_2020:
  \いわぽん @yiwapon\うんこを投げつけあうような仕事をするのは、とてもつらい。
  \url{https://t.co/99pULkDReh} 
\item
  (5/100) RT yiwapon(いわぽん)|un\_co\_the2nd(うの字を名乗る💩物)
  日時:2021-06-13 01:33/2021-06-12 19:17 URL:
  \url{https://twitter.com/yiwapon/status/1403752232261013510} 
  \url{https://twitter.com/un\_co\_the2nd/status/1403657648822571010\textgreater} {}
  ビジネス界でブイブイ言わせて年に何千万円もらうのもいいけど、人間のクソみてえな有様をいっぱい見るのもバッヂあってこその道なので、そういう方向に進む優秀な人がいても・・・いいと思うの・・・・・・おいでよクソみてえな町弁業へ・・・
\item
  (9/100) RT
  yiwapon(いわぽん)|1961kumachin(くまちん(弁護士中村元弥))
  日時:2021-06-12 14:34/2021-06-12 14:08 URL:
  \url{https://twitter.com/yiwapon/status/1403586546360492034} 
  \url{https://twitter.com/1961kumachin/status/1403579841597739012\textgreater} {}
  飲み友だちが増えるだけで、仕事は全然拡張されません。
  \url{https://t.co/7950bQEua7} 
\end{itemize}

 やはりいわぽんこと岩田圭只弁護士のタイムラインに、岩田圭只弁護士のリツイートがありました。

 山出警部補とは、2009年(平成21年)3月15日の夜遅くに宇出津に戻ってからしばらくして金沢中警察署に電話をして話したことがありました。仕事先に警察が来たと聞いたことを話すと、「わしはそんな汚いことせん」などと言っていました。相談に来いとも言っていたように思います。

 その後になるのではと思うのですが、金沢中警察署に電話をしたとき山出警部補は留置場の係をしていると言われました。電話を取り次いでもらったのですが、たぶんその前に刑事課長を名乗る人が電話口に出て話をしました。会話の内容は記憶にないですが、YouTube動画として音声を公開していました。

 他にも複数、電話での会話を録音してYouTubeに公開していた時期で、私が会話を録音し公開していることも知っていたはずです。録音はピタリとやめたのですが、取り調べの可視化を声高に叫ぶ弁護士やジャーナリストの江川紹子氏にツイートで紹介し、相手にされなかったことで無意味に思いやめました。

 金沢中警察署の刑事課長の会話を録音したYouTubeを見つけたのは、1,2年前と思いますが、金沢地方検察庁の田中さんであったか、その人との会話のYouTube動画も同時に再生したように思います。時期もはっきり記憶にないのですが、公開をしたのは長くて半年ぐらいの期間で、2012年辺りかと思います。

\begin{itemize}
\tightlist
\item
  〈〈〈 2021/06/13 12:26:27 Linux Emacs: 〈〈〈
\end{itemize}

\hypertarget{ux30a6ux30f3ux30b3ux306eux5370ux306eux3046ux306eux5b57ux3068ux3044ux3046ux533fux540dux5f01ux8b77ux58ebtwitterux30a2ux30abux30a6ux30f3ux30c8ux3068ux5742ux672cux6b63ux5e78ux5f01ux8b77ux58ebux6cd5ux653fux5927ux5b66ux6559ux6388ux3068ux306eux95a2ux4fc2ux6027ux95a2ux7a7aux9023ux7d61ux6a4bux306eux6bcdux5b50ux81eaux6bbaux548cux6b4cux5c71ux306eux4e0dux53efux89e3ux306aux4e8bux4ef6ux304bux3089ux306eux3064ux306aux304cux308a}{%
\paragraph{ウンコの印のうの字という匿名弁護士Twitterアカウントと、坂本正幸弁護士(法政大学教授)との関係性、関空連絡橋の母子自殺、和歌山の不可解な事件からのつながり}\label{ux30a6ux30f3ux30b3ux306eux5370ux306eux3046ux306eux5b57ux3068ux3044ux3046ux533fux540dux5f01ux8b77ux58ebtwitterux30a2ux30abux30a6ux30f3ux30c8ux3068ux5742ux672cux6b63ux5e78ux5f01ux8b77ux58ebux6cd5ux653fux5927ux5b66ux6559ux6388ux3068ux306eux95a2ux4fc2ux6027ux95a2ux7a7aux9023ux7d61ux6a4bux306eux6bcdux5b50ux81eaux6bbaux548cux6b4cux5c71ux306eux4e0dux53efux89e3ux306aux4e8bux4ef6ux304bux3089ux306eux3064ux306aux304cux308a}}

\begin{itemize}
\tightlist
\item
  〉〉〉 Linux Emacs: 2021/06/13 12:32:35 〉〉〉
\end{itemize}

:CATEGORIES: @kanazawabengosi \#金沢弁護士会 @JFBAsns
日本弁護士連合会(日弁連) \#法務省 @MOJ\_HOUMU \#うの字
\#坂本正幸弁護士 \#再審請求

 きっかけは「再審」の検索だったのですが、処理の出力に気になるツイートを見かけ、投稿が終わるとすぐに記事を閲覧しました。探したツイートの内容は、フィクションの読書感想文のようなものだったのですが、だいぶん下の方にあり、その前に和歌山と関連したツイートを見かけてきました。

\begin{itemize}
\item
  奉納\危険生物・弁護士脳汚染除去装置\金沢地方検察庁御中\_2020:
  REGEXP:''再審''/データベース登録済みツイートの検索:2021-06-07〜2021-06-13/2021年06月13日08時16分の記録:ユーザ・投稿:61/108件
  \url{https://t.co/pmDF7GBHTW} 
\item
  奉納\危険生物・弁護士脳汚染除去装置\金沢地方検察庁御中\_2020:
  REGEXP:''@un\_co\_the2nd''/データベース登録済みツイートの検索:2021-05-30〜2021-06-13/2021年06月13日09時48分の記録:ユーザ・投稿:47/131件
  \url{https://t.co/smuA26N6Pz} 
\item
  (061/108) RT
  @tk84yuki(?高橋ユキ?)|red0101a(赤石晋一郎/ジャーナリスト)
  日時:2021-06-10 20:16:34 +0900/2021-06-10 14:24:00 +0900 URL:
  \url{https://twitter.com/tk84yuki/status/1402947794827059202} 
  \url{https://twitter.com/red0101a/status/1402859143795404803\textgreater} {}
  再審請求がニュースになっていた \#和歌山カレー事件。痛ましい話が飛び込んできた。敏腕記者によると、下記死亡事件、林真須美氏の家族であるという情報が浮上しているという。不審死なのか、心中なのかー。憶測が憶測を呼んでいるという\textgreater\textgreater{}
  \url{https://t.co/5XboYSHjDb} 
\end{itemize}

 今になって、リツイートしたアカウントを確認したのですが、元霞っ子クラブの女性記者でした。山口県の山奥の近隣住民殺害事件で本を出しているということもありました。高橋ユキという名前はちょくちょくと見かけているのですが、余り調べていない人物でもあります。

\begin{itemize}
\tightlist
\item
  高橋ユキが裁判傍聴した実績がすごい!プロフィールや著書の内容とは?【おしゃべりオジサンとヤバイ女】
  \textbar{} Free Talk \url{https://t.co/2hjpn3IcDz} 
\end{itemize}

 Googleで画像検索するとずいぶんたくさんの顔写真がありましたが、はっきり見たことがあると記憶にある人物ではなさそうです。霞っ子クラブのこともずいぶん前のこととして記憶にあるのですが、はっきりした時期は憶えておらず、2009年3月より前であったような気もします。

 範囲選択のできないページですが、裁判の傍聴は2005年から、1500回以上の傍聴とあります。ピンとこない数ですが、それより裁判が月曜から金曜の10時から12時と13時から16時というのが具体的な数値としてより気になりました。

 自分の経験としても16時から17時の間に公判をしていたような記憶があるのですが、18時頃までというのも以前、一度ネットに情報を見かけたような記憶がありました。午前9時30分からの公判というのも1度はあったような気もするのですが、高橋ユキさんの話では一日5時間になります。

 前にも書いていますが、金沢刑務所の拘置所の規模というのはよくわかっていて、名古屋高裁金沢支部の控訴審と最高裁への上告審の被告人もいたので、数は減りますが、平日5時間を3人の裁判官に割り振ると、1に当たり何時間の公判ができるのかと先読みで考えてしまいます。

 言い換えれば、私の個人的な経験のみの見通しになるのですが、統計情報のようなものは見たことがなく、問題にされているのも見ていないです。

 最初のときは軽くツイートに目を通しただけで、すぐに関空連絡橋の検索に移ったのですが、報道の記事を読んで、ただ同じ和歌山というだけで関係づけられた憶測なのかと思っていました。それが「敏腕記者」による情報とありました。

 長男となっていたのか記憶は定かでないですが、和歌山毒カレー事件の子供のことはテレビの特集でみたことがありました。結婚したとか結婚するとかいう話でアパートの一室が撮影されていたような記憶もあります。新しい家族という話であったかもしれないですが、うる憶えです。

 年配の再婚相手でないかぎり16歳の長女がいるということはない年齢と思います。どうもこの16歳の長女の実名だけが出ているようなのですが、父親から虐待を受けた可能性があるという報道になっていました。交通事故以外ではありえないような診断名になっていたのも不思議でした。

\begin{itemize}
\item
  \begin{enumerate}
  \def\labelenumi{(\arabic{enumi})}
  \setcounter{enumi}{5}
  \tightlist
  \item
    和歌山 カレー 長女 - Twitter検索 / Twitter \url{https://t.co/S7xNTHFHSF} 
  \end{enumerate}
\end{itemize}

 どうも本当の話のようです。和歌山毒カレー事件の長男のことしか頭になかったのですが、長女がいたらしく、その長女が関空連絡橋から4歳の娘と飛び降り自殺をしたということでした。和歌山毒カレー事件の再審請求は数日前のニュースでしたが、再審請求に受理というのも初めてみていました。

 大きな事件で長年に渡り話題になっていたので、断片的な情報は多くみかけていた和歌山毒カレー事件ですが、長女に16歳の長女がいたというのは、けっこうな年齢になりそうです。昔、親の因果がなんとかという言葉があったように思いますが、身につまされて考えさせられる事件です。

 今回、再審請求をしたという弁護士の名前も見かけていましたが、記憶では香川県の弁護士で懲戒処分で問題にされていました。元裁判官でもあったように思うのですが、改めて調べることはしていませんでした。

 もともと和歌山毒カレー事件は、安田好弘弁護士がやっていたと思うのですが、ネットで公開されていた再審請求書を読んだこともありました。10年ほど前になりそうですが、これもはっきりした時期は記憶にありません。

\begin{itemize}
\tightlist
\item
  【独白】「長女と孫が亡くなった」和歌山カレー事件の林健治さん 16歳孫が変死後、長女が自殺か〈dot.〉(AERA
  dot.) - Yahoo!ニュース \url{https://t.co/XdbHVNoCIO} 
  (AERAdot.編集部 今西憲之)
\end{itemize}

 記事の最後に見覚えのある記者の名前がありました。一年に1,2回、同じように最後に気がつく形で記事を読んでいるように思いますが、Twitterのアカウントもあったように思います。ヘッダ画像が東日本大震災のシンボル的な被害の建物の写真となっていました。

〉〉〉 kk\_hironoのリツイート 〉〉〉

\begin{itemize}
\tightlist
\item
  RT
  kk\_hirono(刑事告発・非常上告_金沢地方検察庁御中)|maido\_osaka(imanishi)
  日時:2021-06-13 13:55/2021/06/13 12:48 URL:
  \url{https://twitter.com/kk\_hirono/status/1403939177163415554} 
  \url{https://twitter.com/maido\_osaka/status/1403922303042326528} 
  \textgreater{}
  【独白】「長女と孫が亡くなった」和歌山カレー事件の林健治さん 16歳孫が変死後、長女が自殺か〈dot.〉(AERA
  dot.) \#Yahooニュース \url{https://t.co/1shOfjknWs} 
  \#カレー事件 \#関西空港 \#和歌山カレー事件
\end{itemize}

 アイコンの方は以前見たときと同じと思いますが、ヘッダ画像が初めて見るもので著書の表紙の写真となっているようです。プロフィールに「ブログは、商売繁盛でささもってこい!」とありました。記憶にある印象的なフレーズでした。

 Twitterのプロフィールにあるリンクを開くと、「このサイトにアクセスできません imanishinoriyuki.jp
にタイプミスがないか確認してください。」というエラーが出ていました。

\begin{itemize}
\item
  \begin{enumerate}
  \def\labelenumi{(\arabic{enumi})}
  \setcounter{enumi}{5}
  \tightlist
  \item
    和歌山カレー事件の林健治さん - Twitter検索 / Twitter
    \url{https://t.co/zza31i9D72} 
  \end{enumerate}
\end{itemize}

 Twitterのトレンドに出てきましたが、「食べ物・トレンド」とあります。よく見ると毒カレーとはなっておらず、和歌山カレー事件とだけあります。ただのカレー事件で死刑判決が出るはずもないのですが、あまりに有名な事件なのでカレー事件だけでも十分伝わるという考えなのかもしれません。

 ライターの文章が読みづらい、わかりづらい、下手というツイートがずいぶんな数目立ちます。今西憲之氏は個人的にずいぶん前から知っている記者で名前も覚えやすく、それ以上に東日本大震災の爆破されたような建物の写真と、商売繁盛笹もってこい、のフレーズが強烈に印象的でした。

 パソコンの時計をみると14時13分でした。テレビでも報道をやっているのかと思ったのですが、昼食がまだでどうしたものかと考える時間が長くなっています。

 14時18分です。テレビをつけるとゴルフをやっていて、理解するのに5秒ほど掛かったのですが、日曜日だと思い出しました。テレビ金沢にチャンネルを変えたのですが、やたらとCMが長く、そこまで言って委員会NPでは、女性天皇の問題を取り上げている様子でした。

 忘れていた午後のポツンと一軒家の再放送を確認したのですが、放送はなかったです。一月ほど前になりますが、奥能登の輪島市の町野が出たという放送を見逃し、再放送に注意していたのですが、それもたびたび忘れていました。

 もうずいぶん長く、テレビニュース以外の情報番組あるいは報道番組で、再審請求関連はみていないのですが、最後に見たのも銭湯で、木谷明弁護士を特集した放送でした。バラエティのような番組ではいくつか過去の事件をみていました。

 今西憲之氏の記事では、再審請求の前日の事件で自殺した長女らは再審請求のことをしらなかったはずという話でした。改めて再審請求が受理されたという話に覚える違和感ですが、告訴告発でよく問題になるハードルの不受理ですが、再審請求では不受理というのを聞いた覚えがありません。

 昨日の夕方の銭湯のテレビで、能登地方は夕方から雨と聞いていたのですが、ネットの天気予報をみると雨が降るのは22時からとなっていました。その雨も明日の朝の9時台までとなっています。

 関空連絡橋での飛び降りのニュースは、ネットで見出しだけ見かけたいたのですが、他に集中していることがあって思考の中断を避けたく、リンクを開かず記事を読んでいませんでした。知った直後に作成され投稿されたのが、次のうの字のアカウント名でのまとめ記事でした。

\begin{itemize}
\tightlist
\item
  奉納\危険生物・弁護士脳汚染除去装置\金沢地方検察庁御中\_2020:
  REGEXP:''@un\_co\_the2nd''/データベース登録済みツイートの検索:2021-05-30〜2021-06-13/2021年06月13日09時48分の記録:ユーザ・投稿:47/131件
  \url{https://kk2020-09.blogspot.com/2021/06/regexpuncothe2nd2021-05-302021-06.html} 
\end{itemize}

 坂本正幸弁護士のツイートが4件該当していることを確認しページ検索をしたのですが、その4件目だったと思います。それもかなり考えさせられる内容だったのですが、他のことに気を取られたこともあり、思い出せずにいます。短い感じのツイートではありました。

※ @kk\_hironoのアカウントがブロックされ,リツイートに失敗したツイート

\begin{itemize}
\tightlist
\item
  TW sakamotomasayuk(坂本正幸) 日時:2021/06/10 15:57:20 URL:
  \url{https://twitter.com/sakamotomasayuk/status/1402882556219953155} 
  \textgreater{} @un\_co\_the2nd ワニの裁判にしましょうか
\end{itemize}

 ワニの裁判という寓話のような物語でもあるのかと思い検索をしたのですが、「100日後に死ぬワニ」に関連したものが多く、他は見当たりませんでした。うの字のツイートが返信先として埋め込みツイートに表示されていたのですが、それが次のうの字のツイートになります。

※ @kk\_hironoのアカウントがブロックされ,リツイートに失敗したツイート

\begin{itemize}
\tightlist
\item
  TW un\_co\_the2nd(うの字を名乗る💩物) 日時:2021/06/10 15:06:26 URL:
  \url{https://twitter.com/un\_co\_the2nd/status/1402869747063033857} 
  \textgreater{}
  証拠がないからこのままだと敗訴ですよって言うと、この国の法律がおかしいと宣う。証拠がないと敗訴は、煮えたぎった湯に手を突っ込んで決める方式でも取っていなければどの国でも同じっすよ。
\end{itemize}

 うの字が噛み砕いた言い回しで「煮えたぎった湯に手を突っ込んで決める方式」というのは盟神探湯のことと思います。以前法クラのツイートで何度か見かけていましたが、だいぶん前から知っていた言葉で、しかし読み方がなかなか憶えられずにいました。「くがたち」と読むようです。

 裁判を儀式的に皮肉ったものと思えるのですが、一部の弁護士の真理や信念になっているとも思える用例ばかりでした。

\begin{itemize}
\tightlist
\item
  2021年06月13日14時52分の登録:
  REGEXP:''盟神探湯''/データベース登録済みツイートの検索:2010-08-25〜2021-06-13/2021年06月13日14時51分の記録:ユーザ・投稿:34/52件
  \url{https://kk2020-09.blogspot.com/2021/06/regexp2010-08-252021-06.html} 
\end{itemize}

 もう一つ法クラが好む似たような言葉があったと思いながら思い出せずにいたのですが、「蠱毒」を発見しました。法クラというより深澤諭史弁護士がよく使っていたという記憶です。

\begin{itemize}
\item
  関空沖 親子とみられる2人死亡 連絡橋から飛び降りたか|NHK
  関西のニュース \url{https://t.co/J208ks9VZV} 
  9日午後4時ごろ、関西空港の連絡橋を車で通行していた男性から、「道路に停車している車から人が出て落ちたように見えた」と警察に通報がありました。
\item
  毒物カレー事件で再審請求 「第三者の犯行」と主張(共同通信) -
  Yahoo!ニュース \url{https://t.co/zpENamsFSW} 
  弁護人を務める生田暉雄弁護士は9日、和歌山地裁に新たに再審請求を申し立てたと明らかにした。5月31日付。「第三者による犯行は明白で、林死刑囚は無罪」と主張している。
\end{itemize}

 今西憲之氏は前日としていたようですが、同じ6月9日だったようです。それも再審請求は5月31日付で、生田暉雄弁護士が再審請求を明らかにしたのが6月9日とあります。

 和歌山カレー事件については、結構前から冤罪と確信するというようなツイートを弁護士の間でも見かけていました。細かくはっきりした記憶はないですが、すぐに名前を思い出せない女性弁護士のツイートは最近のものとして記憶にあります。

 亀石倫子弁護士でした。Twitterで検索をしましたが、けっこうな数出ているようです。以下にリツイートをやります。

〉〉〉 kk\_hironoのリツイート 〉〉〉

\begin{itemize}
\tightlist
\item
  RT
  kk\_hirono(刑事告発・非常上告_金沢地方検察庁御中)|MichikoKameishi(弁護士
  亀石倫子) 日時:2021-06-13 15:53/2020/03/29 00:49 URL:
  \url{https://twitter.com/kk\_hirono/status/1403968780724629504} 
  \url{https://twitter.com/MichikoKameishi/status/1243928322918084611} 
  \textgreater{}
  22年前に和歌山で起きた毒物カレー事件。カレー鍋のヒ素と林真須美死刑囚の自宅にあったヒ素は別のものだという鑑定結果が出ている。有罪であることに合理的な疑いが生じている以上、「疑わしきは被告人の利益に」が原則。裁判のやり直しを認めない大阪高裁決定は極めて不当。\url{https://t.co/E77twiaCA9} 
\end{itemize}

〉〉〉 kk\_hironoのリツイート 〉〉〉

\begin{itemize}
\tightlist
\item
  RT
  kk\_hirono(刑事告発・非常上告_金沢地方検察庁御中)|MichikoKameishi(弁護士
  亀石倫子) 日時:2021-06-13 15:53/2021/04/29 19:27 URL:
  \url{https://twitter.com/kk\_hirono/status/1403968875620835328} 
  \url{https://twitter.com/MichikoKameishi/status/1387715245808906243} 
  \textgreater{}
  和歌山カレー事件の犯人視報道によって林真須美さんが犯人だと信じて疑わなかった過去の自分を紀州のドンファン事件報道を見て苦々しく思い出す。何も変わってない。無罪推定もへったくれもない。直接証拠もないのにあいつは悪い奴だ、犯人に違いないと信じさせる報道ばかり。
  \url{https://t.co/oayUEtTzhm} 
\end{itemize}

〉〉〉 kk\_hironoのリツイート 〉〉〉

\begin{itemize}
\tightlist
\item
  RT
  kk\_hirono(刑事告発・非常上告_金沢地方検察庁御中)|MichikoKameishi(弁護士
  亀石倫子) 日時:2021-06-13 15:54/2020/03/29 01:00 URL:
  \url{https://twitter.com/kk\_hirono/status/1403968936459210754} 
  \url{https://twitter.com/MichikoKameishi/status/1243931036653735936} 
  \textgreater{}
  私が刑事弁護士になろうと思ったのは、和歌山カレー事件の弁護人との出会いがきっかけだった。それまでは、林真須美死刑囚が犯人に決まっていると思い込んでいた。彼女の言い分を知らず、証拠も見ず、裁判を傍聴したこともなく、ただ報道を鵜呑みにしていた。でもそれは大きな間違いだった。
  \url{https://t.co/dWEp1PXg3R} 
\end{itemize}

〉〉〉 kk\_hironoのリツイート 〉〉〉

\begin{itemize}
\tightlist
\item
  RT
  kk\_hirono(刑事告発・非常上告_金沢地方検察庁御中)|MichikoKameishi(弁護士
  亀石倫子) 日時:2021-06-13 15:54/2018/07/26 14:36 URL:
  \url{https://twitter.com/kk\_hirono/status/1403968986547593219} 
  \url{https://twitter.com/MichikoKameishi/status/1022354905833455617} 
  \textgreater{}
  「当時、ヒ素は白アリ駆除のほか、殺鼠剤や農薬、みかんの減酸剤としても需要があり、和歌山市内だけでも『同一の工場が同一の原料を用いて同一の時期に製造した亜ヒ酸』が大量に出回っていた。
  中井鑑定を以て、林真須美を犯人と特定することはできない」
  \url{https://t.co/uPIlQjE5tV} 
\end{itemize}

〉〉〉 kk\_hironoのリツイート 〉〉〉

\begin{itemize}
\tightlist
\item
  RT
  kk\_hirono(刑事告発・非常上告_金沢地方検察庁御中)|MichikoKameishi(弁護士
  亀石倫子) 日時:2021-06-13 15:54/2017/12/25 15:21 URL:
  \url{https://twitter.com/kk\_hirono/status/1403969094047600647} 
  \url{https://twitter.com/MichikoKameishi/status/945177617807716352} 
  \textgreater{}
  死刑囚の母を持つ息子、被害者の母・元恋人の思い 事件の涙 Human
  Crossroads
  今回番組では1998年の「和歌山・毒物カレー事件」、2007年名古屋市で起きた「闇サイト殺人事件」を取り上げます。
  \url{https://t.co/PZEWs6WaEq} 
\end{itemize}

 最近のものは3月29日と4月29日でした。ずいぶん前になりますが、ネットで誹謗中傷されたと怒っていた時期があり、テレビの報道ステーションの出演とも重なっていたように思います。確信とまではなかったですが、「、ただ報道を鵜呑みにしていた。でもそれは大きな間違いだった。」とあります。

 亀石倫子弁護士のタイムラインは午前7時53分のツイートが最終更新となっています。今の所、今西憲之氏のアエラというのかその記事以外は報道をみていないので、知らないでいる可能性というのもあるのかもしれません。

 どんなニュースだったのか思い出せないですが、しばらく前にも似たことがあって、最初に独占ニュースのようなものがでて、それからぽつぽつと他の報道が出ていました。主要なメディアは一通り揃ったかもしれません。それほど深くは考えておらず、記憶が薄くぼんやりになっています。

 前に見かけていたものですが、亀石倫子弁護士は冤罪事件とされ、再審請求中である大崎事件について、次のツイートも行っていました。

\begin{itemize}
\tightlist
\item
  TW MichikoKameishi(弁護士 亀石倫子) 日時: 2021/06/11 11:08:13 URL:
  \url{https://twitter.com/MichikoKameishi/status/1403172186811031557} 
  \textgreater{}
  大崎事件第4次再審請求、鑑定人澤野医師の尋問終わる。鴨志田祐美弁護士「どこにもほころびがなかった。どこにも破綻がない、積み残しがない、すべての疑問にこれほどクリアに答えられる場を弁護士17年やっていて初めて見た」原口アヤ子さんは来週94歳。一日も早く再審開始を\\
  \textgreater{} \url{https://t.co/iqK4btKOWX} 
\end{itemize}

 同志のような関係性も強く感じる鴨志田裕美弁護士の発言と記事を紹介しただけのようなツイートですが、地元鹿児島県大崎市における影響というのをずっと考えていました。

 鹿児島地検の検事正に問題性の指摘も受けニュース記事となっていたのですが、ずっと前に亡くなっている2人の人物を死体遺棄事件の真犯人として名指しするのが、今回の再審請求の主張の骨子となっているわけです。

 大挙して消えた西日本新聞の大崎事件の特集記事にも、その2人の人物が一月ほど連日のように志布志警察署に呼び出されて事情聴取を受けていたという話がありました。その遺族の取材はしていたようでしたが、記事の内容がよく思い出せません。

 3日ほど前になりますが、西日本新聞と大崎事件の組み合わせで検索すると、西日本新聞のサイトに大崎事件関連の記事の一覧のようなものがありました。

\begin{itemize}
\item
  まとめ「大崎事件」|【西日本新聞me】 \url{https://t.co/S2nGPk5Gzv} 
\item
  大崎事件「事故死」鑑定医、9日に証人尋問|【西日本新聞me】
  \url{https://t.co/SDYpgvp8fO}  残り 1717文字 ¥\n 有料会員限定
\end{itemize}

 「被害者を軽トラックの荷台に乗せるイメージ」という説明文がありますが、ずいぶん乱暴に怪我人を扱うイラストとなっています。

 3ページに分かれた記事の一覧に目を通しましたが、それらしい記事は見当たりませんでした。ツイートとして記録には残しているはずですが、すぐに見つかるかはわかりません。

 次の記事だったと思います。連載の1つの記事という記憶だったのですが、似たようなタイトルの記事が2つありました。

\begin{itemize}
\item
  ./kk\_hirono2021-06-13\_162503.csv:2020-03-05 16:10:01
  ``[source:]「おやじは疑われていた」 検証・大崎事件(8)|【西日本新聞ニュース】
  \url{https://www.nishinippon.co.jp/item/n/587707/''} 
  \url{https://twitter.com/kk\_hirono/status/1235462556086464512} 
\item
  ./kk\_hirono2021-06-13\_162503.csv:2020-03-05 15:58:25
  ``[source:]「私はおやじを信じる」 検証・大崎事件(9)|【西日本新聞ニュース】
  \url{https://www.nishinippon.co.jp/item/n/588781/''} 
  \url{https://twitter.com/kk\_hirono/status/1235459636565696513} 
\item
  おやじは疑われていた 大崎事件 - Google 検索 \url{https://t.co/tM5QqN5JWL} 
\end{itemize}

 それらしい痕跡は私のツイート1件のみでした。壮大なマジックショーを見たような気分です。

 それらしい痕跡は私のツイート1件のみでした。壮大なマジックショーを見たような気分です。

〉〉〉 kk\_hironoのリツイート 〉〉〉

\begin{itemize}
\tightlist
\item
  RT
  kk\_hirono(刑事告発・非常上告_金沢地方検察庁御中)|hirono\_hideki(奉納\さらば弁護士鉄道・泥棒神社の物語)
  日時:2021-06-13 16:44/2020/02/28 17:29 URL:
  \url{https://twitter.com/kk\_hirono/status/1403981587830636547} 
  \url{https://twitter.com/hirono\_hideki/status/1233308291117137920} 
  \textgreater{}
  「おやじは疑われていた」 検証・大崎事件(8)|【西日本新聞ニュース】
  \url{https://t.co/2lx881Lgd6}  2020/2/28 6:00 西日本新聞 社会面
\end{itemize}

 Googleの検索結果からリンクを開くとペンガル語と日本語の選択を迫られ、日本語を選択すると違った日本語に変わっていました。リクエストパラメーターで言語設定があったのですが、その部分を削除すると通常の日本語での表示ができました。

〉〉〉 kk\_hironoのリツイート 〉〉〉

\begin{itemize}
\tightlist
\item
  RT
  kk\_hirono(刑事告発・非常上告_金沢地方検察庁御中)|s\_hirono(非常上告-最高検察庁御中\_ツイッター)
  日時:2021-06-13 16:49/2021/06/13 16:47 URL:
  \url{https://twitter.com/kk\_hirono/status/1403982730166034435} 
  \url{https://twitter.com/s\_hirono/status/1403982277340663811} 
  \textgreater{}
  2021-06-13-164325\_奉納\さらば弁護士鉄道・泥棒神社の物語@hirono\_hideki「おやじは疑われていた」 検証・大崎事件(8)|【西日本新聞ニュース】 h.jpg
  \url{https://t.co/r8isulhclD} 
\end{itemize}

〉〉〉 kk\_hironoのリツイート 〉〉〉

\begin{itemize}
\tightlist
\item
  RT
  kk\_hirono(刑事告発・非常上告_金沢地方検察庁御中)|s\_hirono(非常上告-最高検察庁御中\_ツイッター)
  日時:2021-06-13 16:49/2021/06/13 16:46 URL:
  \url{https://twitter.com/kk\_hirono/status/1403982768233537536} 
  \url{https://twitter.com/s\_hirono/status/1403982204712013826} 
  \textgreater{}
  2021-06-13-164238\_納妻護士鉄道・@hirono\_hideki(おやじは嫌少年いいた検証・大崎()8)(【日本日本新聞】】ps ps https:// nish.jpg
  \url{https://t.co/cxOkc5d6jl} 
\end{itemize}

 なにかのいたずらのような現象ですが、令和3年3月31日付告発状の作成の終盤でも趙誠峰レストランというのが出てきました。そのときも失恋レストランという古い曲のことを思い出していたのですが、鹿児島と長崎が舞台の映画トラック野郎で夏目雅子と恋人役で出ていた歌手のヒット曲でした。

 大崎事件の弁護団の報道をみていると、壮大なスケールの映画をみているような気分になるのですが、鹿児島の大崎事件の発生も1979年10月とあるので、そのトラック野郎の映画のロケや公開に近い時期になりそうです。

\begin{itemize}
\tightlist
\item
  トラック野郎・男一匹桃次郎 - Wikipedia \url{https://t.co/H9hVadgO9U}  ¥\n
  『トラック野郎・男一匹桃次郎』(トラックやろう・おとこいっぴきももじろう)は、1977年(昭和52年)12月24日公開の日本映画。菅原文太主演、東映製作・配給による「トラック野郎シリーズ」第6弾。
\end{itemize}

 長崎というのは私の思い違いだったらしく佐賀県唐津市とおもわれる唐津くんちとありました。この唐津市の魚市場も長距離トラックの仕事で行ったことがあったのですが、古くて大きな魚市場だったと印象に残っています。2回行ったような気もするのですが、1回は中西運輸商の大型保冷車だったと思います。

 トラック野郎の映画はサブタイトルと内容が一致して記憶にあるものはほとんどないのですが、全作品はみておらず、いずれも昭和59年以降にレンタルビデオで視聴しました。金沢でレンタルビデオ店ができたばかりの頃で、1本のレンタル料金が1500円だったと記憶にあります。たぶん2泊3日ぐらいです。

 レンタルビデオで視聴するまでドラック野郎の映画の内容は知らなかったはずですが、夏目雅子という女優は、能都中学校の頃、日曜日の夜に孫悟空をモデルにした実写のドラマがあり、本来は男性のはずだったと思う三蔵法師役を女性の夏目雅子が演じていたことでも印象的でした。

 その数年後には亡くなったというニュースになった夏目雅子ですが、若い女性の病死というのは初めて聞くニュースで、ずいぶん不思議なことがあるものだという感想が、医療知識の不足と相まって、記憶に残りました。

\begin{itemize}
\tightlist
\item
  夏目雅子 - Wikipedia \url{https://t.co/H9ciB3sj4w}  夏目 雅子(なつめ
  まさこ、1957年(昭和32年)12月17日 -
  1985年(昭和60年)9月11日{[}1{]})は、日本の女優。
\end{itemize}

 初めてではないように思いますが、改めて調べてみると、昭和60年9月11日に27歳没とありました。

 「1978年、『西遊記』で三蔵法師役を演じ、人気を得る。」とあります。忘れていたのですが、映画「二百三高地」では、小学校の先生のような役柄で、映画の最後の場面では子供を引率して花見をしている場面が印象的でした。設定は金沢となっていたはずです。本当にあったような話でした。

 大崎事件がどれほど前の事件なのか身をもって実感できるのですが、今頃になって事故死だったと断定するところに弁護士鉄道の歴史と時空の超越を感じます。前は凶器のタオルが供述調書と整合しないとかいう話で、事故死の可能性は予備的主張にとどまっていたように思います。

 今回の再審が裁判所に認められれば、他界した2人が怪我人を乱暴に軽トラックの荷台に投げ込み、家まで送ったら死んでいたので、牛小屋の堆肥に埋めたということが、歴史的な事実として公認されることになりそうです。2年前になるのか最高裁が破棄をしていなければ、今頃は別のかたちで終わっていた事件かもしれません。

 勝手に人の人生を塗り替える可能性があると考えると、弁護士鉄道には脅威を感じます。すでに共犯者は3人ほど知的障害者の烙印をおされ、2人になるのかずっと前に自殺したという話もあります。

 尤も再審請求には前回と同じ主張が許されないという制約があるので、事故死以外の主張はとりにくいのかもしれません。最初から事故死の線で調べるなり再審請求をやっていればよさそうなものですが、元々事実など二の次で、弁護士鉄道の宣伝として利用されてきたというところが大きいように思えます。

 生田暉雄弁護士ですが、懲戒処分が最多の9回となっていました。元裁判官というのも少し記憶にあったのですが、22年間も裁判官をしていたとのことです。

\begin{itemize}
\item
  生田暉雄 - Wikiwand
  \url{https://www.wikiwand.com/ja/\%E7\%94\%9F\%E7\%94\%B0\%E6\%9A\%89\%E9\%9B\%84} 
\item
  TW uwaaaa(サイ太) 日時: 2021/06/13 17:49:55 URL:
  \url{https://twitter.com/uwaaaa/status/1403998053363716098} 
  \textgreater{} 「勾留」不服申し立てで弁護士に報酬への懸念(1/2ページ)
  - 産経ニュース \url{https://t.co/0wJC8x0DlQ} 
\item
  TW uwaaaa(サイ太) 日時: 2021/06/13 17:50:28 URL:
  \url{https://twitter.com/uwaaaa/status/1403998190869774338} 
  \textgreater{}
  全件準抗告キャンペーンは効果あったっていう話じゃありませんでしたっけ?
\end{itemize}

 刑裁サイ太のタイムラインが2つツイートで更新されていました。幅広い方向性に向けた問題提起として2,3日前に多くのメンション付きツイートをやっていたので、その影響があって更新されなくなったのかと気にかけていました。うの字はほぼ影響がなかったみたいです。

 時刻は18時29分です。昼食は15時ぐらいになって「よこすか海軍カレーラーメン」というカップ麺で済ませていたのですが、夕食もどうしたものかと思案しています。

\begin{itemize}
\tightlist
\item
  「勾留」不服申し立てで弁護士に報酬への懸念(1/2ページ) - 産経ニュース
  \url{https://t.co/GgLSCHDDVe}  ¥\n 2021/6/13 12:00 ¥\n 森西 勇太
\end{itemize}

 刑裁サイ太が紹介した産経ニュースの記事は本日6月13日の12時00分が配信時刻となっていました。なぜ日曜日に平日に発表されたようなニュースが配信されているのかタイミングに疑問を持ったのですが、関空連絡橋の母娘心中は、大阪が地元で大きなニュースになっていそうです。少なくとも私は弁護士商売と関連付けて考えています。

〉〉〉 kk\_hironoのリツイート 〉〉〉

\begin{itemize}
\item
  RT
  kk\_hirono(刑事告発・非常上告_金沢地方検察庁御中)|NOSUKE0607(清水
  潔) 日時:2021-06-13 18:40/2021/06/13 17:22 URL:
  \url{https://twitter.com/kk\_hirono/status/1404010660132921347} 
  \url{https://twitter.com/NOSUKE0607/status/1403991063103111171} 
  \textgreater{}
  和歌山カレー事件取材で何度か顔を見た事があるが、まさか関空連絡橋なら落ちたのがあの娘だったとは。事件なのか事故なのか、謎が謎を呼びそうだ。
  林眞須美死刑囚、``関空連絡橋から長女が飛び降り''
  \url{https://t.co/j9tljDWF39} 
\item
  林眞須美死刑囚、``関空連絡橋から長女が飛び降り''一報への肉声|NEWSポストセブン
  \url{https://t.co/U6qYpstfDu}  2021.06.13 16:00  NEWSポストセブン
\end{itemize}

 アエラ以外に初めて見る報道ですが、今日の16時が配信時刻となっています。ここでも生田暉雄弁護士の発言として「第三者による犯行は明白」とあります。社会に向けさしたる根拠も示さず余りに唐突で挑戦的、その影響の大きな親族間には衝撃が走り、大きな動揺となった可能性もあるのかもしれないです。

 個人的に弁護士と関わることの危険性をさらに大きく考えさせられたのですが、関空連絡橋を死に場所に選び4歳の女児を道連れにしたというのも、深く問題を考えさせられるところです。

 時刻は19時02分です。今見たうの字のタイムラインに「甲斐中辰夫東京地検次席検事(当時)」という懐かしい名前を見ましたが、時期は思い出せないものの最初に名前を知ったのが金沢地方検察庁の検事正だったと思います。その後、最高裁判事になっていましたが、ずいぶん長く名前を見ていませんでした。

 時刻は6月14日13時50分です。考えさせられるニュースや話題が沢山ありすぎ多岐亡羊の状態でした。大きな道標のような発見となったのは「司法制度改革20年の呪い」というツイートです。4月の終わりから本人の新規ツイートは初めてかと思われます。相手に向けた挨拶みたいなのは前に1つありました。

\begin{itemize}
\tightlist
\item
  〈〈〈 2021/06/14 13:55:30 Linux Emacs: 〈〈〈
\end{itemize}

\hypertarget{ux548cux6b4cux5c71ux30abux30ecux30fcux4e8bux4ef6ux5bb6ux65cfux306eux5fc3ux4e2dux4e8bux4ef6ux6df1ux6fa4ux8aedux53f2ux5f01ux8b77ux58ebux306eux53f8ux6cd5ux5236ux5ea6ux6539ux9769uxff12uxff10ux5e74ux306eux546aux3044ux3068ux3044ux3046ux30c4ux30a4ux30fcux30c8ux8133ux6a5fux80fdux969cux5bb3ux306eux5c11ux5973ux306eux5bb6ux65cfux304cux88abux5bb3ux306bux906dux3063ux305fux5f01ux8b77ux58ebux4e8bux4ef6ux304bux3089ux8003ux5bdfux3059ux308bux5f01ux8b77ux58ebux9244ux9053ux306eux5b9fux9332ux30c9ux30adux30e5ux30e1ux30f3ux30c8}{%
\paragraph{和歌山カレー事件家族の心中事件、深澤諭史弁護士の「司法制度改革20年の呪い」というツイート、脳機能障害の少女の家族が被害に遭った弁護士事件から考察する弁護士鉄道の実録ドキュメント}\label{ux548cux6b4cux5c71ux30abux30ecux30fcux4e8bux4ef6ux5bb6ux65cfux306eux5fc3ux4e2dux4e8bux4ef6ux6df1ux6fa4ux8aedux53f2ux5f01ux8b77ux58ebux306eux53f8ux6cd5ux5236ux5ea6ux6539ux9769uxff12uxff10ux5e74ux306eux546aux3044ux3068ux3044ux3046ux30c4ux30a4ux30fcux30c8ux8133ux6a5fux80fdux969cux5bb3ux306eux5c11ux5973ux306eux5bb6ux65cfux304cux88abux5bb3ux306bux906dux3063ux305fux5f01ux8b77ux58ebux4e8bux4ef6ux304bux3089ux8003ux5bdfux3059ux308bux5f01ux8b77ux58ebux9244ux9053ux306eux5b9fux9332ux30c9ux30adux30e5ux30e1ux30f3ux30c8}}

\begin{itemize}
\tightlist
\item
  〉〉〉 Linux Emacs: 2021/06/14 15:04:48 〉〉〉
\end{itemize}

:CATEGORIES: @kanazawabengosi \#金沢弁護士会 @JFBAsns
日本弁護士連合会(日弁連) \#法務省 @MOJ\_HOUMU \#深澤諭史弁護士

 まず、和歌山カレー事件ですが、本日は今のところ新しいニュースがないようです。昨日の夕方になりますか、謎に思っていた再審請求の受理について、最高裁で継続中の再審請求とは別に、別の弁護士によって申立があったということで、それを裁判所が受理する判断をしたということでした。

 和歌山カレー事件死刑囚の家族の心中事件は、今のところテレビで報道は見ておらず、週刊誌のようなウェブサイトのニュース以外は、ネットでも報道をみていません。NHK、朝日新聞、産経新聞、毎日新聞、読売新聞などです。

 これまでネットの弁護士らが強い反応を示してきた虐待、DVなどの要素もある事件ですが、検索で記録した弁護士らの反応というのも少ないもので、意識的に避けているとしか思えないぐらいです。弁護士にはなにかと不都合がある事件なのかと考え、調べながら記録をしています。

 最初に報道を知ったのは昨日かあるいは一昨日になると思うのですが、和歌山カレー事件死刑囚の家族が16歳長女の虐待死事件と、関空連絡橋からの飛び降り自殺をしたことで、和歌山カレー事件について新たに知ったことがいくつかありました。冤罪の可能性をうかがわせる事実関係になります。

 この和歌山カレー事件の刑事裁判のニュースというのは控訴審になってから見たのですが、一審で黙秘を貫き通し死刑判決が出て、控訴審では一転して自ら積極的に供述を始めたという話でした。弁護士の存在というのも余り見えてこない事件でしたが、再審請求では安田好弘弁護士が関与したはずです。

 正直なところ冤罪とは思えない事件で、黙秘したのがなによりの証拠と疑いもなく考えていたのですが、弁護士らが手抜きの弁護活動をするとともに疑惑をアシストし、再審請求へという道筋を人生のレールとして敷設した、これも1つの弁護士鉄道の物語に思えてきました。

 このあと別のかたちで取り上げご紹介したいところですが、1998年(平成10年)11月18日とある和歌山弁護士会の提言・要請なる書面も発見するに至っています。

\begin{itemize}
\tightlist
\item
  産経新聞大阪本社 編集室 御 中|弁護士会の意見|和歌山弁護士会
  \url{http://www.wakaben.or.jp/opinion/requests/199811\_kare4.html} 
\end{itemize}

 次にこれまでのTwitterのまとめた記録です。

\begin{itemize}
\tightlist
\item
  2021年06月13日18時09分の登録:
  REGEXP:''和歌山.*長女''/データベース登録済みツイートの検索:2021-06-10〜2021-06-13/2021年06月13日18時08分の記録:ユーザ・投稿:10/18件
  \url{https://kk2020-09.blogspot.com/2021/06/regexp2021-06-102021-06.html} 
\item
  2021年06月14日06時16分の登録:
  REGEXP:''和歌山.*事件''/データベース登録済みツイートの検索:2009-11-05〜2021-06-14/2021年06月14日06時13分の記録:ユーザ・投稿:65/152件
  \url{https://kk2020-09.blogspot.com/2021/06/regexp2009-11-052021-06.html} 
\item
  2021年06月14日06時19分の登録:
  Ex-REGEXP:''和歌山.*事件''/データベース登録済みツイートの検索:2009-11-05〜2021-06-14/2021年06月14日06時17分の記録:ユーザ・投稿:62/92件
  \url{https://kk2020-09.blogspot.com/2021/06/ex-regexp2009-11-052021-06.html} 
\item
  2021年06月14日06時20分の登録:
  Ex-REGEXP:''和歌山.*事件''/データベース登録済みツイートの検索:2021-06-10〜2021-06-14/2021年06月14日06時20分の記録:ユーザ・投稿:19/30件
  \url{https://kk2020-09.blogspot.com/2021/06/ex-regexp2021-06-102021-06.html} 
\item
  2021年06月14日06時21分の登録:
  Ex-REGEXP:''和歌山.*事件''/データベース登録済みツイートの検索:2021-06-13〜2021-06-14/2021年06月14日06時21分の記録:ユーザ・投稿:14/19件
  \url{https://kk2020-09.blogspot.com/2021/06/ex-regexp2021-06-132021-06.html} 
\end{itemize}

 Ex-REGEXP:とあるのは私の3つのTwitterアカウントのツイートを除外したものになります。

\begin{quote}
《引用の始まり》
\end{quote}

\begin{quote}
アカウント名 ツイート数 リツイート数okumuraosaka(okumuraosaka) 1
0石井孝明(Ishii Takaaki)(ishiitakaaki) 1
0園田昌也@赤羽(chew\_bacca1987) 1 0NEWS JAPAN(NEWS\_JAPAN\_S) 5
0びーちゃん(eeyy888777) 0 1?高橋ユキ?(tk84yuki) 1
0ルート66(元ルパン3世)(Route66\_LP3) 0
1神庭亮介(kamba\_ryosuke) 2 0清水 潔(NOSUKE0607) 1
0青木 俊(AokiTonko) 0 1???₅₄(hKodama) 1
0弁護士 山中理司(yamanaka\_osaka) 0 1ギタ弁(guitar\_ben) 0
1藤井誠二(seijifujii1965) 1 0
\end{quote}

\begin{quote}
《引用の終わり》
\end{quote}

\begin{itemize}
\tightlist
\item
  奉納\危険生物・弁護士脳汚染除去装置\金沢地方検察庁御中\_2020:
  Ex-REGEXP:''和歌山.*事件''/データベース登録済みツイートの検索:2021-06-13〜2021-06-14/2021年06月14日06時21分の記録:ユーザ・投稿:14/19件
  \url{https://kk2020-09.blogspot.com/2021/06/ex-regexp2021-06-132021-06.html} 
\end{itemize}

 お馴染の法クラのTwitterアカウントの名前がほとんど見えません。

 次に深澤諭史弁護士のツイートです。次の前エントリーの終わりに触れていますが、深澤諭史弁護士本人のツイートとしては4月の終わり以来、一月半ぶりになります。途中、挨拶返しのようなツイートは1件確認しておりますが、声明的なものはずいぶん久々です。

2021年06月14日15時51分の実行記録:8500で処理を終了
twitterAPI-search-lawList-mydql-add.rb ``和歌山 長女''
ツイート数:24/2472 リツイート数:7/2472 トータル:8500``和歌山
長女''の該当: hirono\_hideki 2/0件 kk\_hirono 6/2件 s\_hirono 0/0件

\begin{itemize}
\tightlist
\item
  2021年06月14日12時54分の登録:
  \深澤諭史 @fukazawas\司法制度改革20年の呪い
  \url{https://yiwapon.net/archives/10721} 
  「司法制度改革に関していえば、制度を活かす
  \url{https://kk2020-09.blogspot.com/2021/06/fukazawas-httpsyiwaponnetarchives10721.html} 
\item
  2021年06月14日15時53分の登録:
  REGEXP:''和歌山.*長女''/データベース登録済みツイートの検索:2021-06-13〜2021-06-14/2021年06月14日15時52分の記録:ユーザ・投稿:12/23件
  \url{https://kk2020-09.blogspot.com/2021/06/regexp2021-06-132021-06\_14.html} 
\end{itemize}

 弁護士のツイートはわずかですが、いずれも週刊誌の記事の紹介のようなものばかりです。

 次に深澤諭史弁護士のツイートです。

\begin{itemize}
\tightlist
\item
  (1/100) TW fukazawas(深澤諭史) 日時: 2021-06-14 12:09 URL:
  \url{https://twitter.com/fukazawas/status/1404274860314877953\textgreater} {}
  司法制度改革20年の呪い \url{https://t.co/jBq2cdcCLc\textgreater} {}
  「司法制度改革に関していえば、制度を活かすものは疑いもなく人だみたいなことを高らかに謳ってたくせに、司法修習生の貸与制にみられるような人を大事にしない制度を・・・
  \url{https://t.co/xC11tR6Be6} 
\end{itemize}

\begin{quote}
《引用の始まり》
\end{quote}

\begin{quote}
20年の歩みと歪んだ「この国のかたち」司法の人的及び制度的基盤が充実したかというと、ただ人が増えた20年間であった。その他の感想はない。

内閣機能の強化は果たされたのだろうが、内閣を構成する肝心のヘッドやメンバーがアレだと、内閣主導の旗の下で人事を引っ掻き回したり思い付きで施策をブチ上げるという病理的な現象も生じる。どの政党が政権を取ろうと懸念は同じであるが、このところはほとんど何も歯止めが効かないという心細さを味わい続けている。

結局、行政改革でも司法制度改革でも「この国のかたち」にこだわった割には、バランスを損なってずいぶん歪んだ「かたち」になってしまった。しかも、むしろそれは改革を主導した人たちが元々意図したものとすら思えてくることがある。強い内閣、弱い司法、そして規制緩和と三つ揃えば、好き勝手できる層もいるのである。しかし、そのような在り方の国は、長い目で見れば不公正に対する人々の怒りによりいずれ爆発し、滅亡するだろう。

いや、もう、我が身はどうでもいいからこんな国は一度滅亡の憂き目を見るべきで、その後に次世代が然るべき新時代を拓けばいいというくらいの思いが浮かぶことすらある。
\end{quote}

\begin{quote}
《引用の終わり》
\end{quote}

\begin{itemize}
\tightlist
\item
  司法制度改革20年の呪い -- BLOG@yiwapon
  \url{https://yiwapon.net/archives/10721} 
\end{itemize}

 上記の引用部分は「20年の歩みと歪んだ「この国のかたち」」という見出しで始まり、「我が身はどうでもいいからこんな国は一度滅亡の憂き目を見るべきで、その後に次世代が然るべき新時代を拓けばいいというくらいの思いが浮かぶことすら」と締めくくられ、次の「改革と人間観」に続いています。

 実名がページに見当たりませんが、いわぽんこと岩田圭只弁護士のブログかと思います。北海道帯広市の弁護士で所属は釧路弁護士会となっていたかもしれません。前年度になるのか日弁連の役員にもなっていました。理事であったのか70名ほどの1人だったと思います。

 司法制度改革への痛烈な批判のようですが、ずっと前から同じ方向制で、今回はより激しく過激になっているとは感じました。鬱屈した不満も溜まり、あるいは法律事務所の経営や生活が危機的な状況として差し迫っているのかもしれません。

 管理人について、という自己紹介にも「デビュー作は『成仏理論に関する考察』」とありますが、この成仏理論というのが運動のスローガン、旗印のようになっていました。それでも最近は見かけることが少なくなっている法クラの「成仏」です。

 「d\textbar grep 成仏\textbar wc
-l」というコマンドを実行すると結果が278件でした。掲載してご紹介しようと思ったのですが、数が多すぎます。基本的なセットがその成仏理論の発言者とされる高橋宏志という人物で、法クラにはピロシとも揶揄され怨嗟の的となっています。

\begin{itemize}
\tightlist
\item
  (社説)司法改革20年 未完の歩み つなぐ使命:朝日新聞デジタル
  \url{https://t.co/Vpj1fsE2h5}  ¥\n 2021年6月13日 5時00分
\end{itemize}

 いわぽんこと岩田圭只弁護士がブログの記事で引き合いにだしたのが上記の朝日新聞デジタルの記事で、小さめの写真があり、今朝から2,3度見かけていた写真なのですが、この記事で小泉総理大臣と横に並ぶのが、同じく法クラに目の敵にされた佐藤幸治氏だとわかりました。

\begin{itemize}
\tightlist
\item
  (社説)司法改革20年 未完の歩み つなぐ使命:朝日新聞デジタル
  \url{https://t.co/nUiRfIXH22} 
  司法制度改革審議会長の佐藤幸治・京都大名誉教授(右)から意見書を受け取る小泉純一郎首相(いずれも当時)=2001年6月12日、首相官邸
\end{itemize}

 どちらかといえば、成仏理論を唱えたとして高橋宏志氏の方が批判にさらされるのを見ることが多かったのですが、深澤諭史弁護士は佐藤幸治氏の方により重い責任があると考えている様子で、まさに文字通りの戦犯扱いという印象です。

 これまで見てきた深澤諭史弁護士の数々のツイートでも、とりわけ印象深く心に突き刺さるのが、これから新たな記録を行う脳機能障害の少女にまつわる深澤諭史弁護士のツイートです。弁護士鉄道の夜空にひときわ強く輝く危険な大凶星のようなもので、人間社会に強く警鐘を鳴らしておきたいところです。

\begin{lstlisting}
base ❯ d|grep 脳機能障害
\end{lstlisting}

\begin{itemize}
\tightlist
\item
  2018年04月10日02時39分の登録:
  REGEXP:''脳機能障害.*少女''/データベース登録済みツイート:2018年04月10日02時38分の記録:ユーザ・投稿:5/23件
  \url{http://hirono2014sk.blogspot.com/2018/04/regexp201804100238523.html} 
\item
  2018年09月08日15時54分の登録:
  REGEXP:''脳機能障害.*少女''/データベース登録済みツイート:2018年09月08日15時54分の記録:ユーザ・投稿:5/24件
  \url{http://hirono2014sk.blogspot.com/2018/09/regexp201809081554524.html} 
\item
  2018年09月08日16時08分の登録:
  REGEXP:''脳機能障害.*少女''/データベース登録済みツイート:2018年09月08日16時08分の記録:ユーザ・投稿:6/35件
  \url{http://hirono2014sk.blogspot.com/2018/09/regexp201809081608635.html} 
\item
  2019年06月21日14時39分の登録:
  %@yukihirosasamo ささもひょんⁿ@赤腹魔王%さっきのと同一人物ですか・・・・・・。開いた口が塞がらないよ・・・・・・。¥\n脳機能障害を負った少女の一家から着服
  「示談不成立」とウソ
  \url{http://hirono2014sk.blogspot.com/2019/06/yukihirosasamon.html} 
\item
  2021年02月18日22時20分の登録:
  %@uwaaaa サイ太%高次脳機能障害,体幹機能障害という重篤な障害を負い,今も原告が介護しているという話で,ようやく認容額が175万円ですからね。「不貞慰謝料の相場は
  \url{https://kk2020-09.blogspot.com/2021/02/uwaaaa\_18.html} 
\item
  2021年06月06日00時19分の登録:
  「脳機能障害.+少女」を@hirono\_hideki @kk\_hirono @s\_hironoで検索 94件の該当 2021-06-06\_00:19の記録
  \url{https://kk2020-09.blogspot.com/2021/06/hironohidekikkhironoshirono942021-06.html} 
\end{itemize}

 深澤諭史弁護士のツイートはかなり見つけづらくなっているのですが、「脳機能障害」と「佐藤幸治」という深澤諭史弁護士のツイート・リツイートに限定したまとめ記事を作成し、そこから探し出したいと思います。

\begin{itemize}
\item
  2021年06月14日16時39分の登録:
  REGEXP:''佐藤幸治''/深澤諭史(@fukazawas)の検索(2013-01-31〜2021-04-22/2021年06月14日16時39分の記録107件)
  \url{https://kk2020-09.blogspot.com/2021/06/regexpfukazawas2013-01-312021-04.html} 
\item
  2021年06月14日16時39分の登録:
  REGEXP:''脳機能障害''/深澤諭史(@fukazawas)の検索(2015-07-22〜2017-04-20/2021年06月14日16時39分の記録2件)
  \url{https://kk2020-09.blogspot.com/2021/06/regexpfukazawas2015-07-222017-04.html} 
\item
  (1/2) RT
  fukazawas(深澤諭史)|yukihirosasamo(ささもたん@赤腹魔王)
  日時:2015-07-22 18:48:00 +0900/2015-07-22 18:46:00 +0900 URL:
  \url{https://twitter.com/fukazawas/status/623791767637225472} 
  \url{https://twitter.com/yukihirosasamo/status/623791245568012288\textgreater} {}
  さっきのと同一人物ですか・・・・・・。開いた口が塞がらないよ・・・・・・。\textgreater{}
  脳機能障害を負った少女の一家から着服 「示談不成立」とウソ \#ldnews
  \url{http://t.co/BW3bqtjmAW} 
  target="\_blank"\textgreater \url{http://t.co/BW3bqtjmAW} 
\end{itemize}

 脳機能障害の深澤諭史弁護士のツイートのまとめは2件でしたが、もう1件が岡口基一裁判官のツイートのリツイートとなっていました。Twitter社に永久凍結された岡口基一裁判官のアカウントです。

\begin{itemize}
\tightlist
\item
  (008/107) TW fukazawas(深澤諭史) 日時:2014-10-28 14:37:00 +0900
  URL:
  \url{https://twitter.com/fukazawas/status/526970889359732737\textgreater} {}
  医療の世界も,国民に開かれ信用されるように,大手術には,くじ引きで選ばれた「医療員」が手術に参加するようにしてはどうか。\textgreater{}
  専門知識の無い一般国民の感覚で内臓を切り裂いて,よりよき医療を実現しよう。\textgreater{}
  患者第一号は,類似の制度導入に尽力された佐藤幸治先生にお願いしよう。\textgreater{}
  (・∀・)
\end{itemize}

 最近は裁判員制度を批判しあるいは皮肉る弁護士のツイートを不思議なほど見かけなくなっているのですが、さきほどいわぽんこと岩田圭只弁護士が引き合いに出していた朝日新聞デジタルの記事に、その裁判員制度に関する記述があってので、やはり引用してご紹介しておきたいと思います。

\begin{itemize}
\item
  (社説)司法改革20年 未完の歩み つなぐ使命:朝日新聞デジタル
  \url{https://t.co/nUiRfIXH22} 
  司法制度改革審議会長の佐藤幸治・京都大名誉教授(右)から意見書を受け取る小泉純一郎首相(いずれも当時)=2001年6月12日、首相官邸
\item
  (社説)司法改革20年 未完の歩み つなぐ使命:朝日新聞デジタル
  \url{https://t.co/Vpj1fsE2h5} 
\end{itemize}

 短い方が、ブラウザのTwitter拡張機能で取得したページタイトルですが、別の方法で取得したページタイトルは、「司法制度改革審議会長の佐藤幸治・京都大名誉教授(右)から意見書を受け取る小泉純一郎首相(いずれも当時)=2001年6月12日、首相官邸」という部分が追加されていました。

 ブラウザのTwitter拡張機能から投稿した直後に、中村元弥弁護士のツイートが出てイチケイのカラスが見えたのですが、反射的にウィンドウを閉じてしまいました。けっこうやることで反射的に閉じるボタンをクリックしてしまうのですが、今回はそのツイートを探してみます。

〉〉〉 kk\_hironoのリツイート 〉〉〉

\begin{itemize}
\tightlist
\item
  RT
  kk\_hirono(刑事告発・非常上告_金沢地方検察庁御中)|1961kumachin(くまちん(弁護士中村元弥))
  日時:2021-06-14 17:05/2021/06/14 01:25 URL:
  \url{https://twitter.com/kk\_hirono/status/1404349291485749252} 
  \url{https://twitter.com/1961kumachin/status/1404112799773519874} 
  \textgreater{} 来週のNNNドキュメントは 「イチケイの見習い裁判官」
  全国放送!!
\end{itemize}

〉〉〉 kk\_hironoのリツイート 〉〉〉

\begin{itemize}
\tightlist
\item
  RT
  kk\_hirono(刑事告発・非常上告_金沢地方検察庁御中)|lawyernakahara(弁護士
  中原潤一) 日時:2021-06-14 17:06/2021/06/14 00:28 URL:
  \url{https://twitter.com/kk\_hirono/status/1404349448914800643} 
  \url{https://twitter.com/lawyernakahara/status/1404098413570842624} 
  \textgreater{}
  控訴審の国選が回ってこない方でも、依頼者に控訴するかどうかのアドバイスをするためには控訴審の知識は不可欠です。これを一回見ておけば、アドバイスができるようになりますよ!
  \url{https://t.co/qk0jtQsiyP} 
\end{itemize}

〉〉〉 kk\_hironoのリツイート 〉〉〉

\begin{itemize}
\tightlist
\item
  RT
  kk\_hirono(刑事告発・非常上告_金沢地方検察庁御中)|k\_nextgen(K-Ben
  NextGen) 日時:2021-06-14 17:06/2021/06/13 23:34 URL:
  \url{https://twitter.com/kk\_hirono/status/1404349469022253056} 
  \url{https://twitter.com/k\_nextgen/status/1404084720506347520} 
  \textgreater{}
  控訴審なんてちゃんと勉強したことない、控訴審が初めて、そもそも控訴審って何をすれば良いの、配転されたら困る!という方々向けに、基礎の基礎をお伝えする研修です。
  修習生、若手弁護士の皆様、カモン!(全仏オープン決勝で興奮している講師より)
  \url{https://t.co/liTJRAPtnX} 
\end{itemize}

 最初、中村元弥弁護士のタイムラインをだいぶん遡ってイチケイとあるツイートを見つけたのですが、もっと文字数が多いツイートだったように思え、ホームのタイムラインを探したのですが見つからず、中村元弥弁護士のタイムラインに戻ったところで、下に2つ並んでいるツイートに着目しました。

 ホームのタイムラインを遡ることは余りなかったのですが、意外に早く最後のツイートに行き着きました。途中に、寺内貫太郎一家の集合写真があってツイートに訃報のニュースがありました。日付を確認しなかったのですが、トレンドには入っていないようです。

\begin{quote}
《引用の始まり》
\end{quote}

\begin{quote}
06月14日 16時29分

「この木なんの木」の歌い出しで知られるCMソングや都はるみさんの「北の宿から」などを手がけた作曲家で、俳優としても活躍した小林亜星さんが先月30日に心不全のため亡くなりました。88歳でした。
\end{quote}

\begin{quote}
《引用の終わり》
\end{quote}

\begin{itemize}
\tightlist
\item
  作曲家の小林亜星さん死去 88歳 俳優としても活躍|NHK 首都圏のニュース
  \url{https://www3.nhk.or.jp/shutoken-news/20210614/1000065744.html} 
\end{itemize}

 本日16時29分というニュースでした。2年ほど前、どんたく宇出津店で聴いたことのない子供向けのサンタクロースの歌が繰り返し流れていて、家に戻ってから調べたのですが、作曲が小林亜生となっていました。

 福井刑務所では希望者だけのクリスマス会に出て、幼稚園か保育園児の合唱の曲を聴いたこともあったのですが、クリスマスソングは時期になるとスーパーでもいろいろ流れるのに、聞き覚えのない曲なのが不思議でした。

\begin{itemize}
\item
  あわてんぼうのサンタクロース (歌詞付) - YouTube
  \url{https://t.co/0VHkwUCtXp}  ¥\n 10,456,301 回視聴•2016/12/19
\item
  あわてんぼうのサンタクロースとは - コトバンク \url{https://t.co/BHwBHkI2to} 
  唱歌。昭和46年(1971)発表。吉岡治作詞、小林亜星作曲。子供向けのクリスマス曲として知られる。
\end{itemize}

 前にも調べて確認していたと思うのですが、昭和46年発売とあります。私は昭和39年11月生まれなので7歳ぐらいですが、聴いた覚えがないので不思議に思っていました。特徴的なクリスマスソングだったので耳にするとすぐに気になったのですが、最初は外国の曲と思っていました。

 寺内貫太郎一家というテレビドラマもよく憶えているのですが、前に調べたところ別のドラマと記憶が混同していて多分別のドラマの挿入曲となっていたのが「昭和枯れすすき」でした。考えてみると、男女が心中をする場面の曲でしたが、自殺というのはニュースでも余り見かけない時代だったと思います。

\begin{itemize}
\tightlist
\item
  昭和枯れすゝき - Wikipedia \url{https://t.co/HWTvoIjkR9} 
  1974年7月21日に発売した当初はレコードの売れ行きが伸び悩んでいたが、同年10月16日から放送開始された『時間ですよ昭和元年』(TBS系列)の挿入歌として、細川俊之演じる十郎と大楠道代演じる菊との居酒屋の場面に効果的に使われたことにより、
\end{itemize}

 「時間ですよ昭和元年」というドラマのタイトルは前回調べたときも記憶になかったのですが、女湯が出てくる場面の動画があって、それらしいドラマがあったことを思い出しました。それまでは長い間、「昭和枯れすすき」を寺内貫太郎一家の挿入曲と思い込んでいました。

 寺内貫太郎一家は、別に「林檎殺人事件」という曲が大ヒットしたことをよく憶えていたのですが、そちらは中学生の頃の記憶だと思います。

\begin{itemize}
\item
  (010/107) TW fukazawas(深澤諭史) 日時:2015-04-19 19:04:00 +0900
  URL:
  \url{https://twitter.com/fukazawas/status/589731208059166721\textgreater} {}
  佐藤幸治教授に足りないもの \url{http://t.co/LenMr7XwAQ} 
  target="\_blank"\textgreater \url{http://t.co/LenMr7XwAQ\textgreater} {}
  「制度設計者として無責任のそしりを免れない。弁護士会など他人に責任転嫁しているうちは、「世間知らずが祭り上げられて浮かれたあげく、晩節を汚した」との評価を受け続けるだろう。」
\item
  佐藤幸治教授に足りないもの: 花水木法律事務所 \url{https://t.co/8sFXtaCEB2} 
  ¥\n 2011年6月 9日 (木)
\end{itemize}

 深澤諭史弁護士のツイートにあるのはカッコ書きで全文引用のようです。花水木法律事務所というブログはこれまでに何度かリンクを開いてみているのですが、どんな弁護士が所属しているのか気になることもなく調べていませんでした。

 よくみるとページの左上に、小さい文字ですが「弁護士小林正啓と弁護士櫻井美幸のブログです」とありました。どちらも見覚えのない弁護士の名前です。

〉〉〉 kk\_hironoのリツイート 〉〉〉

\begin{itemize}
\tightlist
\item
  RT
  kk\_hirono(刑事告発・非常上告_金沢地方検察庁御中)|lawyerkobayashi(小林正啓)
  日時:2021-06-14 17:45/2015/11/02 21:14 URL:
  \url{https://twitter.com/kk\_hirono/status/1404359403604627457} 
  \url{https://twitter.com/lawyerkobayashi/status/661154531028434944} 
  \textgreater{} 完全自動運転自動車とトロッコ問題について \#BLOGOS
  \url{https://t.co/KlWxM3tO9h} 
\end{itemize}

 それらしきTwitterアカウントを見つけましたが、2015年11月2日で更新が止まっています。気になったのはプロフィールに「大阪市北区」とあったことです。

 和歌山カレー事件関係の報道がないかとテレビをつけたのですが、ミヤネ屋の速報として大阪府北区天満のカラオケパブでの20代女性殺害事件が中継されていました。天満というだけで気になったのですが、有名な天神橋筋が天満の近くというのは初めて知ったことで意外でした。

 大阪府北区天満が気になっていたのは、扱いは小さいが内容が大きい弁護士の不祥事関連のツイートを隣接した時期に2つ見かけていたことがあったからです。奥村徹弁護士の法律事務所の住所も大阪府北区天満となっていたように思います。

\begin{itemize}
\tightlist
\item
  死亡したのはカラオケパブの「25歳の女性オーナー」と確認 胸には刺し傷、首には複数の切り傷
  \url{https://t.co/fEo2fe9UMi} 
\end{itemize}

 殺害された女性と思われる顔写真がありますが、テレビでは20代の女性オーナーということで気になっていました。店の名前からTwitterのアカウントも見つけていたのですが、ミヤネ屋の放送で、店の営業状況を逐一ツイートで知らせているという話があり、最後の更新の話もありました。

\begin{itemize}
\tightlist
\item
  カラオケパブ - Twitter検索 / Twitter \url{https://t.co/57ygG9rUtB} 
\end{itemize}

 タイムラインにあるツイートの画像で思い出したのですが、アザラシのごまちゃんは漫画のキャラクターで、それもずいぶん前、昭和の終わりか平成の初めだったと思います。アザラシのごまちゃんというのはすぐに思い出していたのですが、漫画のことはイラストを見るまで浮かびませんでした。

\begin{itemize}
\tightlist
\item
  少年アシベ - Wikipedia \url{https://t.co/7Vv0Ukputr} 
  1988年から1994年まで『週刊ヤングジャンプ』(集英社)に連載、単行本全8巻。1990年にはOVAが製作、1991年にはテレビアニメ『少年アシベ』、1992年には『少年アシベ2』が放送。
\end{itemize}

 テレビアニメというのは記憶にないのですが、昭和63年から平成6年の連載で、週刊ヤングジャンプとありました。長距離トラックの仕事をした頃は、ビックコミックやモーニングとあわせ、薄い週刊誌の漫画としてよく買ったり、コンビニで立ち読みをして読んでいました。

〉〉〉 kk\_hironoのリツイート 〉〉〉

\begin{itemize}
\tightlist
\item
  RT
  kk\_hirono(刑事告発・非常上告_金沢地方検察庁御中)|mainichijpnews(毎日新聞ニュース)
  日時:2021-06-14 18:37/2021/06/14 18:27 URL:
  \url{https://twitter.com/kk\_hirono/status/1404372399361970180} 
  \url{https://twitter.com/mainichijpnews/status/1404370003642646536} 
  \textgreater{}
  死亡女性は25歳のカラオケパブ経営者 大阪・天神橋の殺人事件
  \url{https://t.co/md3UfUB3nv} 
\end{itemize}

 Googleマップで確認していないのですが、天満駅近くの事件が「大阪・天神橋の殺人事件」という見出しででてきました。ごちそうさんというNHK連続テレビ小説の印象が強いのですが、天神橋の商店街で、帰還兵となった肉屋の息子が精神的病を抱えている場面が強く印象に残り、歴史の重みを感じました。

\begin{itemize}
\item
  カラオケパブごまちゃんオーナーはまゆたろう25歳?犯人はストーカー?凶器はまだ見つからず│ShutterStrike
  \url{https://t.co/0cDY07usTl} 
  今回の凄惨な事件現場となってしまったのは、カラオケパブごまちゃんというお店です。
  ¥\n ¥\n 住所は大阪府大阪市北区天神橋4丁目10−5 第2JKビル 5階。
\item
  〒530-0041 大阪府大阪市北区天神橋4丁目10−5 - Google マップ
  \url{https://t.co/Op0z94Q4aQ} 
\end{itemize}

 天神橋筋というのは歴史のあるアーケード街とばかり思い込むイメージがあったのですが、側に扇町公園という広い公園があるのも意外でした。昭和の時代の演歌の歌詞に水の都といわれた大阪なので、川沿いというのもイメージにあったのですが、大川という聞いたことのない川からも離れているようです。

 大阪市内は昭和59年に金沢市場輸送の4トン保冷車で集中的によく行った時期があり、見覚えのある地名も多いのですが、北区に隣接して福島区があるのが意外で、尤もよく行った大阪本場市場があるのが福島区でした。前に調べたとき大阪拘置所もその近くにあったように思います。

\begin{itemize}
\tightlist
\item
  大阪市中央卸売市場 本場 - Google マップ \url{https://t.co/8zhWx3Ydlw} 
\end{itemize}

 平成4年の1月にも行ったと思う大阪の本場市場ですが、Googleマップで見ると、クールファイブの歌の歌詞にもあった中之島や、大阪地方検察庁、大阪地方裁判所ともかなり近いということがわかりました。当時はまだかなり古い町並みや建物が残っていたと思います。

 平成3年11月の半ば過ぎから平成4年2月の初めに集中的に行ったのが和歌山県のかつらぎ町の農協で、ミカンやイヨカンを積みに行ったのですが、ほとんどの場合、行き荷は馬鈴薯で卸先は、大阪府の茨木市となっていたと思う北部市場、大阪市福島区の大阪本場市場のいずれかでした。

 かつらぎ農協は集中的に行ったというより集中的に市場急配センターの配車で行かされたのですが、手積み手降ろしの仕事を他の運転手が嫌がったため集中的に回ってきた仕事でした。以前はただ懐かしいというだけの感慨だったのですが、何か縁があったのかとも考えるようになっており、記録しています。

\begin{itemize}
\tightlist
\item
  スナックとパブの違いは?|Q\&A相談室|バイトル \textbar{}
  バイトルマガジン BOMS(ボムス) \url{https://t.co/A4nacbwBWC} 
\end{itemize}

 余り考えたことはなかったのですが、パブというのはお酒を提供するお店ということがわかり、カラオケを楽しむコンセプトを主体としたのがカラオケパブということでした。スナックにもカラオケはつきものと思いますが、愛好会のような性格を感じます。カラオケボックスに近いものを考えていました。

 刑事訴訟法あるいは刑事訴訟法愛好会なるものがあったことを思い出します。刑裁サイ太が会員であったことは確認していないですが、殺人事件の実名報道があると、思いっきり皮肉ってマスコミを罵倒していたという印象が色濃くあります。

 個人的な哀悼の気持ちを込めて弁護士鉄道の歴史に記録を残しておきたいと思います。テーマは「刑裁サイ太@uwaaaaVSマスコミ」。マスコミ報道について改めて考えさせられる機会となっているのも和歌山カレー事件の親族の末路です。

\begin{itemize}
\tightlist
\item
  2021年06月14日20時24分の登録:
  REGEXP:''マスコミ''/サイ太(@uwaaaa)の検索(2012-02-20〜2021-06-03/2021年06月14日20時24分の記録103件)
  \url{https://kk2020-09.blogspot.com/2021/06/regexpuwaaaa2012-02-202021-06.html} 
\item
  2021年06月14日20時24分の登録:
  REGEXP:''佐藤幸治''/サイ太(@uwaaaa)の検索(2017-01-20〜2017-07-14/2021年06月14日20時24分の記録2件)
  \url{https://kk2020-09.blogspot.com/2021/06/regexpuwaaaa2017-01-202017-07.html} 
\end{itemize}

 20時24分という同時刻になっていますが、投稿は''佐藤幸治''が先、''マスコミ''が後です。

\begin{itemize}
\item
  カラオケパブごまちゃんの被害者に捧げたい弁護士鉄道の記録→ 奉納\危険生物・弁護士脳汚染除去装置\金沢地方検察庁御中\_2020:
  REGEXP:''マスコミ''/サイ太(@uwaaaa)の検索(2012-02-20〜2021-06-03/2021年06月14日20時24分の記録103件)
  \url{https://t.co/Hrk14WZ7ne} 
\item
  @nhk\_news -
  カラオケパブごまちゃんの被害者に捧げたい弁護士鉄道の記録→ 奉納\危険生物・弁護士脳汚染除去装置\金沢地方検察庁御中\_2020:
  REGEXP:''マスコミ''/サイ太(@uwaaaa)の検索(2012-02-20〜2021-06-03/2021年06月14日20時24分の記録103件)
  \url{https://t.co/Hrk14WZ7ne} 
\item
  @nikkei -
  カラオケパブごまちゃんの被害者に捧げたい弁護士鉄道の記録→ 奉納\危険生物・弁護士脳汚染除去装置\金沢地方検察庁御中\_2020:
  REGEXP:''マスコミ''/サイ太(@uwaaaa)の検索(2012-02-20〜2021-06-03/2021年06月14日20時24分の記録103件)
  \url{https://t.co/Hrk14WZ7ne} 
\item
  @WSJJapan -
  カラオケパブごまちゃんの被害者に捧げたい弁護士鉄道の記録→ 奉納\危険生物・弁護士脳汚染除去装置\金沢地方検察庁御中\_2020:
  REGEXP:''マスコミ''/サイ太(@uwaaaa)の検索(2012-02-20〜2021-06-03/2021年06月14日20時24分の記録103件)
  \url{https://t.co/Hrk14WZ7ne} 
\item
  @ReutersJapan -
  カラオケパブごまちゃんの被害者に捧げたい弁護士鉄道の記録→ 奉納\危険生物・弁護士脳汚染除去装置\金沢地方検察庁御中\_2020:
  REGEXP:''マスコミ''/サイ太(@uwaaaa)の検索(2012-02-20〜2021-06-03/2021年06月14日20時24分の記録103件)
  \url{https://t.co/Hrk14WZ7ne} 
\item
  @asahi -
  カラオケパブごまちゃんの被害者に捧げたい弁護士鉄道の記録→ 奉納\危険生物・弁護士脳汚染除去装置\金沢地方検察庁御中\_2020:
  REGEXP:''マスコミ''/サイ太(@uwaaaa)の検索(2012-02-20〜2021-06-03/2021年06月14日20時24分の記録103件)
  \url{https://t.co/Hrk14WZ7ne} 
\item
  @Yomiuri\_Online -
  カラオケパブごまちゃんの被害者に捧げたい弁護士鉄道の記録→ 奉納\危険生物・弁護士脳汚染除去装置\金沢地方検察庁御中\_2020:
  REGEXP:''マスコミ''/サイ太(@uwaaaa)の検索(2012-02-20〜2021-06-03/2021年06月14日20時24分の記録103件)
  \url{https://t.co/Hrk14WZ7ne} 
\item
  @yamada\_official -
  カラオケパブごまちゃんの被害者に捧げたい弁護士鉄道の記録→ 奉納\危険生物・弁護士脳汚染除去装置\金沢地方検察庁御中\_2020:
  REGEXP:''マスコミ''/サイ太(@uwaaaa)の検索(2012-02-20〜2021-06-03/2021年06月14日20時24分の記録103件)
  \url{https://t.co/Hrk14WZ7ne} 
\item
  @biccameraE -
  カラオケパブごまちゃんの被害者に捧げたい弁護士鉄道の記録→ 奉納\危険生物・弁護士脳汚染除去装置\金沢地方検察庁御中\_2020:
  REGEXP:''マスコミ''/サイ太(@uwaaaa)の検索(2012-02-20〜2021-06-03/2021年06月14日20時24分の記録103件)
  \url{https://t.co/Hrk14WZ7ne} 
\item
  @vaam\_official -
  カラオケパブごまちゃんの被害者に捧げたい弁護士鉄道の記録→ 奉納\危険生物・弁護士脳汚染除去装置\金沢地方検察庁御中\_2020:
  REGEXP:''マスコミ''/サイ太(@uwaaaa)の検索(2012-02-20〜2021-06-03/2021年06月14日20時24分の記録103件)
  \url{https://t.co/Hrk14WZ7ne} 
\item
  @meiji\_s\_genki -
  カラオケパブごまちゃんの被害者に捧げたい弁護士鉄道の記録→ 奉納\危険生物・弁護士脳汚染除去装置\金沢地方検察庁御中\_2020:
  REGEXP:''マスコミ''/サイ太(@uwaaaa)の検索(2012-02-20〜2021-06-03/2021年06月14日20時24分の記録103件)
  \url{https://t.co/Hrk14WZ7ne} 
\item
  @essential\_jp -
  カラオケパブごまちゃんの被害者に捧げたい弁護士鉄道の記録→ 奉納\危険生物・弁護士脳汚染除去装置\金沢地方検察庁御中\_2020:
  REGEXP:''マスコミ''/サイ太(@uwaaaa)の検索(2012-02-20〜2021-06-03/2021年06月14日20時24分の記録103件)
  \url{https://t.co/Hrk14WZ7ne} 
\item
  @nivea\_promo\_jp -
  カラオケパブごまちゃんの被害者に捧げたい弁護士鉄道の記録→ 奉納\危険生物・弁護士脳汚染除去装置\金沢地方検察庁御中\_2020:
  REGEXP:''マスコミ''/サイ太(@uwaaaa)の検索(2012-02-20〜2021-06-03/2021年06月14日20時24分の記録103件)
  \url{https://t.co/Hrk14WZ7ne} 
\item
  @SatoMasahisa -
  カラオケパブごまちゃんの被害者に捧げたい弁護士鉄道の記録→ 奉納\危険生物・弁護士脳汚染除去装置\金沢地方検察庁御中\_2020:
  REGEXP:''マスコミ''/サイ太(@uwaaaa)の検索(2012-02-20〜2021-06-03/2021年06月14日20時24分の記録103件)
  \url{https://t.co/Hrk14WZ7ne} 
\item
  @miosugita -
  カラオケパブごまちゃんの被害者に捧げたい弁護士鉄道の記録→ 奉納\危険生物・弁護士脳汚染除去装置\金沢地方検察庁御中\_2020:
  REGEXP:''マスコミ''/サイ太(@uwaaaa)の検索(2012-02-20〜2021-06-03/2021年06月14日20時24分の記録103件)
  \url{https://t.co/Hrk14WZ7ne} 
\item
  @YoshikoSakurai -
  カラオケパブごまちゃんの被害者に捧げたい弁護士鉄道の記録→ 奉納\危険生物・弁護士脳汚染除去装置\金沢地方検察庁御中\_2020:
  REGEXP:''マスコミ''/サイ太(@uwaaaa)の検索(2012-02-20〜2021-06-03/2021年06月14日20時24分の記録103件)
  \url{https://t.co/Hrk14WZ7ne} 
\item
  @kitamuraharuo -
  カラオケパブごまちゃんの被害者に捧げたい弁護士鉄道の記録→ 奉納\危険生物・弁護士脳汚染除去装置\金沢地方検察庁御中\_2020:
  REGEXP:''マスコミ''/サイ太(@uwaaaa)の検索(2012-02-20〜2021-06-03/2021年06月14日20時24分の記録103件)
  \url{https://t.co/Hrk14WZ7ne} 
\item
  @takenoma -
  カラオケパブごまちゃんの被害者に捧げたい弁護士鉄道の記録→ 奉納\危険生物・弁護士脳汚染除去装置\金沢地方検察庁御中\_2020:
  REGEXP:''マスコミ''/サイ太(@uwaaaa)の検索(2012-02-20〜2021-06-03/2021年06月14日20時24分の記録103件)
  \url{https://t.co/Hrk14WZ7ne} 
\item
  @YoichiTakahashi -
  カラオケパブごまちゃんの被害者に捧げたい弁護士鉄道の記録→ 奉納\危険生物・弁護士脳汚染除去装置\金沢地方検察庁御中\_2020:
  REGEXP:''マスコミ''/サイ太(@uwaaaa)の検索(2012-02-20〜2021-06-03/2021年06月14日20時24分の記録103件)
  \url{https://t.co/Hrk14WZ7ne} 
\item
  @sanrin\_hikiyose -
  カラオケパブごまちゃんの被害者に捧げたい弁護士鉄道の記録→ 奉納\危険生物・弁護士脳汚染除去装置\金沢地方検察庁御中\_2020:
  REGEXP:''マスコミ''/サイ太(@uwaaaa)の検索(2012-02-20〜2021-06-03/2021年06月14日20時24分の記録103件)
  \url{https://t.co/Hrk14WZ7ne} 
\item
  @primaham\_info -
  カラオケパブごまちゃんの被害者に捧げたい弁護士鉄道の記録→ 奉納\危険生物・弁護士脳汚染除去装置\金沢地方検察庁御中\_2020:
  REGEXP:''マスコミ''/サイ太(@uwaaaa)の検索(2012-02-20〜2021-06-03/2021年06月14日20時24分の記録103件)
  \url{https://t.co/Hrk14WZ7ne} 
\item
  @yamazakipan\_cp -
  カラオケパブごまちゃんの被害者に捧げたい弁護士鉄道の記録→ 奉納\危険生物・弁護士脳汚染除去装置\金沢地方検察庁御中\_2020:
  REGEXP:''マスコミ''/サイ太(@uwaaaa)の検索(2012-02-20〜2021-06-03/2021年06月14日20時24分の記録103件)
  \url{https://t.co/Hrk14WZ7ne} 
\item
  @asahi -
  カラオケパブごまちゃんの被害者に捧げたい弁護士鉄道の記録→ 奉納\危険生物・弁護士脳汚染除去装置\金沢地方検察庁御中\_2020:
  REGEXP:''マスコミ''/サイ太(@uwaaaa)の検索(2012-02-20〜2021-06-03/2021年06月14日20時24分の記録103件)
  \url{https://t.co/Hrk14WZ7ne} 
\item
  @abcabcabc999666 -
  カラオケパブごまちゃんの被害者に捧げたい弁護士鉄道の記録→ 奉納\危険生物・弁護士脳汚染除去装置\金沢地方検察庁御中\_2020:
  REGEXP:''マスコミ''/サイ太(@uwaaaa)の検索(2012-02-20〜2021-06-03/2021年06月14日20時24分の記録103件)
  \url{https://t.co/Hrk14WZ7ne} 
\item
  @himaben1st -
  カラオケパブごまちゃんの被害者に捧げたい弁護士鉄道の記録→ 奉納\危険生物・弁護士脳汚染除去装置\金沢地方検察庁御中\_2020:
  REGEXP:''マスコミ''/サイ太(@uwaaaa)の検索(2012-02-20〜2021-06-03/2021年06月14日20時24分の記録103件)
  \url{https://t.co/Hrk14WZ7ne} 
\item
  @hdm1987 -
  カラオケパブごまちゃんの被害者に捧げたい弁護士鉄道の記録→ 奉納\危険生物・弁護士脳汚染除去装置\金沢地方検察庁御中\_2020:
  REGEXP:''マスコミ''/サイ太(@uwaaaa)の検索(2012-02-20〜2021-06-03/2021年06月14日20時24分の記録103件)
  \url{https://t.co/Hrk14WZ7ne} 
\item
  @osakashakai -
  カラオケパブごまちゃんの被害者に捧げたい弁護士鉄道の記録→ 奉納\危険生物・弁護士脳汚染除去装置\金沢地方検察庁御中\_2020:
  REGEXP:''マスコミ''/サイ太(@uwaaaa)の検索(2012-02-20〜2021-06-03/2021年06月14日20時24分の記録103件)
  \url{https://t.co/Hrk14WZ7ne} 
\item
  @news\_ytv -
  カラオケパブごまちゃんの被害者に捧げたい弁護士鉄道の記録→ 奉納\危険生物・弁護士脳汚染除去装置\金沢地方検察庁御中\_2020:
  REGEXP:''マスコミ''/サイ太(@uwaaaa)の検索(2012-02-20〜2021-06-03/2021年06月14日20時24分の記録103件)
  \url{https://t.co/Hrk14WZ7ne} 
\item
  @JCC\_NEWS -
  カラオケパブごまちゃんの被害者に捧げたい弁護士鉄道の記録→ 奉納\危険生物・弁護士脳汚染除去装置\金沢地方検察庁御中\_2020:
  REGEXP:''マスコミ''/サイ太(@uwaaaa)の検索(2012-02-20〜2021-06-03/2021年06月14日20時24分の記録103件)
  \url{https://t.co/Hrk14WZ7ne} 
\item
  @nhk\_bknews -
  カラオケパブごまちゃんの被害者に捧げたい弁護士鉄道の記録→ 奉納\危険生物・弁護士脳汚染除去装置\金沢地方検察庁御中\_2020:
  REGEXP:''マスコミ''/サイ太(@uwaaaa)の検索(2012-02-20〜2021-06-03/2021年06月14日20時24分の記録103件)
  \url{https://t.co/Hrk14WZ7ne} 
\item
  @news24ntv -
  カラオケパブごまちゃんの被害者に捧げたい弁護士鉄道の記録→ 奉納\危険生物・弁護士脳汚染除去装置\金沢地方検察庁御中\_2020:
  REGEXP:''マスコミ''/サイ太(@uwaaaa)の検索(2012-02-20〜2021-06-03/2021年06月14日20時24分の記録103件)
  \url{https://t.co/Hrk14WZ7ne} 
\item
  @matomedane -
  カラオケパブごまちゃんの被害者に捧げたい弁護士鉄道の記録→ 奉納\危険生物・弁護士脳汚染除去装置\金沢地方検察庁御中\_2020:
  REGEXP:''マスコミ''/サイ太(@uwaaaa)の検索(2012-02-20〜2021-06-03/2021年06月14日20時24分の記録103件)
  \url{https://t.co/Hrk14WZ7ne} 
\item
  @oshakaibu -
  カラオケパブごまちゃんの被害者に捧げたい弁護士鉄道の記録→ 奉納\危険生物・弁護士脳汚染除去装置\金沢地方検察庁御中\_2020:
  REGEXP:''マスコミ''/サイ太(@uwaaaa)の検索(2012-02-20〜2021-06-03/2021年06月14日20時24分の記録103件)
  \url{https://t.co/Hrk14WZ7ne} 
\item
  @kantele\_news -
  カラオケパブごまちゃんの被害者に捧げたい弁護士鉄道の記録→ 奉納\危険生物・弁護士脳汚染除去装置\金沢地方検察庁御中\_2020:
  REGEXP:''マスコミ''/サイ太(@uwaaaa)の検索(2012-02-20〜2021-06-03/2021年06月14日20時24分の記録103件)
  \url{https://t.co/Hrk14WZ7ne} 
\item
  @FNN\_News -
  カラオケパブごまちゃんの被害者に捧げたい弁護士鉄道の記録→ 奉納\危険生物・弁護士脳汚染除去装置\金沢地方検察庁御中\_2020:
  REGEXP:''マスコミ''/サイ太(@uwaaaa)の検索(2012-02-20〜2021-06-03/2021年06月14日20時24分の記録103件)
  \url{https://t.co/Hrk14WZ7ne} 
\item
  @kubo\_law\_office -
  カラオケパブごまちゃんの被害者に捧げたい弁護士鉄道の記録→ 奉納\危険生物・弁護士脳汚染除去装置\金沢地方検察庁御中\_2020:
  REGEXP:''マスコミ''/サイ太(@uwaaaa)の検索(2012-02-20〜2021-06-03/2021年06月14日20時24分の記録103件)
  \url{https://t.co/Hrk14WZ7ne} 
\item
  @sekiguchi\_shiga -
  カラオケパブごまちゃんの被害者に捧げたい弁護士鉄道の記録→ 奉納\危険生物・弁護士脳汚染除去装置\金沢地方検察庁御中\_2020:
  REGEXP:''マスコミ''/サイ太(@uwaaaa)の検索(2012-02-20〜2021-06-03/2021年06月14日20時24分の記録103件)
  \url{https://t.co/Hrk14WZ7ne} 
\item
  @ngn\_tty\_law -
  カラオケパブごまちゃんの被害者に捧げたい弁護士鉄道の記録→ 奉納\危険生物・弁護士脳汚染除去装置\金沢地方検察庁御中\_2020:
  REGEXP:''マスコミ''/サイ太(@uwaaaa)の検索(2012-02-20〜2021-06-03/2021年06月14日20時24分の記録103件)
  \url{https://t.co/Hrk14WZ7ne} 
\item
  @hiroben\_ -
  カラオケパブごまちゃんの被害者に捧げたい弁護士鉄道の記録→ 奉納\危険生物・弁護士脳汚染除去装置\金沢地方検察庁御中\_2020:
  REGEXP:''マスコミ''/サイ太(@uwaaaa)の検索(2012-02-20〜2021-06-03/2021年06月14日20時24分の記録103件)
  \url{https://t.co/Hrk14WZ7ne} 
\item
  @aiben\_info -
  カラオケパブごまちゃんの被害者に捧げたい弁護士鉄道の記録→ 奉納\危険生物・弁護士脳汚染除去装置\金沢地方検察庁御中\_2020:
  REGEXP:''マスコミ''/サイ太(@uwaaaa)の検索(2012-02-20〜2021-06-03/2021年06月14日20時24分の記録103件)
  \url{https://t.co/Hrk14WZ7ne} 
\item
  @TobenMedia -
  カラオケパブごまちゃんの被害者に捧げたい弁護士鉄道の記録→ 奉納\危険生物・弁護士脳汚染除去装置\金沢地方検察庁御中\_2020:
  REGEXP:''マスコミ''/サイ太(@uwaaaa)の検索(2012-02-20〜2021-06-03/2021年06月14日20時24分の記録103件)
  \url{https://t.co/Hrk14WZ7ne} 
\item
  @niben\_net -
  カラオケパブごまちゃんの被害者に捧げたい弁護士鉄道の記録→ 奉納\危険生物・弁護士脳汚染除去装置\金沢地方検察庁御中\_2020:
  REGEXP:''マスコミ''/サイ太(@uwaaaa)の検索(2012-02-20〜2021-06-03/2021年06月14日20時24分の記録103件)
  \url{https://t.co/Hrk14WZ7ne} 
\item
  @kuma\_bar\_assoc -
  カラオケパブごまちゃんの被害者に捧げたい弁護士鉄道の記録→ 奉納\危険生物・弁護士脳汚染除去装置\金沢地方検察庁御中\_2020:
  REGEXP:''マスコミ''/サイ太(@uwaaaa)の検索(2012-02-20〜2021-06-03/2021年06月14日20時24分の記録103件)
  \url{https://t.co/Hrk14WZ7ne} 
\item
  @ehimeben -
  カラオケパブごまちゃんの被害者に捧げたい弁護士鉄道の記録→ 奉納\危険生物・弁護士脳汚染除去装置\金沢地方検察庁御中\_2020:
  REGEXP:''マスコミ''/サイ太(@uwaaaa)の検索(2012-02-20〜2021-06-03/2021年06月14日20時24分の記録103件)
  \url{https://t.co/Hrk14WZ7ne} 
\item
  @shigaben -
  カラオケパブごまちゃんの被害者に捧げたい弁護士鉄道の記録→ 奉納\危険生物・弁護士脳汚染除去装置\金沢地方検察庁御中\_2020:
  REGEXP:''マスコミ''/サイ太(@uwaaaa)の検索(2012-02-20〜2021-06-03/2021年06月14日20時24分の記録103件)
  \url{https://t.co/Hrk14WZ7ne} 
\item
  @kanaben\_info -
  カラオケパブごまちゃんの被害者に捧げたい弁護士鉄道の記録→ 奉納\危険生物・弁護士脳汚染除去装置\金沢地方検察庁御中\_2020:
  REGEXP:''マスコミ''/サイ太(@uwaaaa)の検索(2012-02-20〜2021-06-03/2021年06月14日20時24分の記録103件)
  \url{https://t.co/Hrk14WZ7ne} 
\item
  @tomoyasuoyama -
  カラオケパブごまちゃんの被害者に捧げたい弁護士鉄道の記録→ 奉納\危険生物・弁護士脳汚染除去装置\金沢地方検察庁御中\_2020:
  REGEXP:''マスコミ''/サイ太(@uwaaaa)の検索(2012-02-20〜2021-06-03/2021年06月14日20時24分の記録103件)
  \url{https://t.co/Hrk14WZ7ne} 
\item
  @JFBAsaigai -
  カラオケパブごまちゃんの被害者に捧げたい弁護士鉄道の記録→ 奉納\危険生物・弁護士脳汚染除去装置\金沢地方検察庁御中\_2020:
  REGEXP:''マスコミ''/サイ太(@uwaaaa)の検索(2012-02-20〜2021-06-03/2021年06月14日20時24分の記録103件)
  \url{https://t.co/Hrk14WZ7ne} 
\item
  @kanazawabengosi -
  カラオケパブごまちゃんの被害者に捧げたい弁護士鉄道の記録→ 奉納\危険生物・弁護士脳汚染除去装置\金沢地方検察庁御中\_2020:
  REGEXP:''マスコミ''/サイ太(@uwaaaa)の検索(2012-02-20〜2021-06-03/2021年06月14日20時24分の記録103件)
  \url{https://t.co/Hrk14WZ7ne} 
\item
  @bengoshikaikysb -
  カラオケパブごまちゃんの被害者に捧げたい弁護士鉄道の記録→ 奉納\危険生物・弁護士脳汚染除去装置\金沢地方検察庁御中\_2020:
  REGEXP:''マスコミ''/サイ太(@uwaaaa)の検索(2012-02-20〜2021-06-03/2021年06月14日20時24分の記録103件)
  \url{https://t.co/Hrk14WZ7ne} 
\item
  @sagakenben -
  カラオケパブごまちゃんの被害者に捧げたい弁護士鉄道の記録→ 奉納\危険生物・弁護士脳汚染除去装置\金沢地方検察庁御中\_2020:
  REGEXP:''マスコミ''/サイ太(@uwaaaa)の検索(2012-02-20〜2021-06-03/2021年06月14日20時24分の記録103件)
  \url{https://t.co/Hrk14WZ7ne} 
\item
  @chunichi\_denhen -
  カラオケパブごまちゃんの被害者に捧げたい弁護士鉄道の記録→ 奉納\危険生物・弁護士脳汚染除去装置\金沢地方検察庁御中\_2020:
  REGEXP:''マスコミ''/サイ太(@uwaaaa)の検索(2012-02-20〜2021-06-03/2021年06月14日20時24分の記録103件)
  \url{https://t.co/Hrk14WZ7ne} 
\item
  東京新聞(TOKYO Web)さん (@tokyo\_shimbun) / Twitter
  \url{https://t.co/rxwYvSwPsB}  -
  カラオケパブごまちゃんの被害者に捧げたい弁護士鉄道の記録→ :''マスコミ''/サイ太(@uwaaaa)の検索(2012-02-20〜2021-06-03/2021年06月14日20時24分の記録103件)
  \url{https://t.co/Hrk14WZ7ne} 
\item
  東京新聞政治部さん (@tokyoseijibu) / Twitter \url{https://t.co/EcG5w0L9P3}  -
  カラオケパブごまちゃんの被害者に捧げたい弁護士鉄道の記録→ :''マスコミ''/サイ太(@uwaaaa)の検索(2012-02-20〜2021-06-03/2021年06月14日20時24分の記録103件)
  \url{https://t.co/Hrk14WZ7ne} 
\item
  東京新聞編集局さん (@tokyonewsroom) / Twitter \url{https://t.co/WZrQ6mU3UA} 
  -
  カラオケパブごまちゃんの被害者に捧げたい弁護士鉄道の記録→ :''マスコミ''/サイ太(@uwaaaa)の検索(2012-02-20〜2021-06-03/2021年06月14日20時24分の記録103件)
  \url{https://t.co/Hrk14WZ7ne} 
\item
  literaさん (@litera\_web) / Twitter \url{https://t.co/5pUcFcVd1c}  -
  カラオケパブごまちゃんの被害者に捧げたい弁護士鉄道の記録→ :''マスコミ''/サイ太(@uwaaaa)の検索(2012-02-20〜2021-06-03/2021年06月14日20時24分の記録103件)
  \url{https://t.co/Hrk14WZ7ne} 
\item
  文春オンラインさん (@bunshun\_online) / Twitter
  \url{https://t.co/13PgMdXB1M}  -
  カラオケパブごまちゃんの被害者に捧げたい弁護士鉄道の記録→ :''マスコミ''/サイ太(@uwaaaa)の検索(2012-02-20〜2021-06-03/2021年06月14日20時24分の記録103件)
  \url{https://t.co/Hrk14WZ7ne} 
\item
  週刊文春さん (@shukan\_bunshun) / Twitter \url{https://t.co/rFgoHNr21F}  -
  カラオケパブごまちゃんの被害者に捧げたい弁護士鉄道の記録→ :''マスコミ''/サイ太(@uwaaaa)の検索(2012-02-20〜2021-06-03/2021年06月14日20時24分の記録103件)
  \url{https://t.co/Hrk14WZ7ne} 
\item
  共同通信公式さん (@kyodo\_official) / Twitter \url{https://t.co/P7Nxqgcyz3} 
  -
  カラオケパブごまちゃんの被害者に捧げたい弁護士鉄道の記録→ :''マスコミ''/サイ太(@uwaaaa)の検索(2012-02-20〜2021-06-03/2021年06月14日20時24分の記録103件)
  \url{https://t.co/Hrk14WZ7ne} 
\item
  NHKニュースさん (@nhk\_news) / Twitter \url{https://t.co/vQoBrCUvSb}  -
  カラオケパブごまちゃんの被害者に捧げたい弁護士鉄道の記録→ :''マスコミ''/サイ太(@uwaaaa)の検索(2012-02-20〜2021-06-03/2021年06月14日20時24分の記録103件)
  \url{https://t.co/Hrk14WZ7ne} 
\item
  首相官邸(災害・危機管理情報)さん (@Kantei\_Saigai) / Twitter
  \url{https://t.co/Vzy3wllQ24}  -
  カラオケパブごまちゃんの被害者に捧げたい弁護士鉄道の記録→ :''マスコミ''/サイ太(@uwaaaa)の検索(2012-02-20〜2021-06-03/2021年06月14日20時24分の記録103件)
  \url{https://t.co/Hrk14WZ7ne} 
\item
  防衛省・自衛隊さん (@ModJapan\_jp) / Twitter \url{https://t.co/l8KFyXEcVb}  -
  カラオケパブごまちゃんの被害者に捧げたい弁護士鉄道の記録→ :''マスコミ''/サイ太(@uwaaaa)の検索(2012-02-20〜2021-06-03/2021年06月14日20時24分の記録103件)
  \url{https://t.co/Hrk14WZ7ne} 
\item
  厚生労働省さん (@MHLWitter) / Twitter \url{https://t.co/1BxRfxggjN}  -
  カラオケパブごまちゃんの被害者に捧げたい弁護士鉄道の記録→ :''マスコミ''/サイ太(@uwaaaa)の検索(2012-02-20〜2021-06-03/2021年06月14日20時24分の記録103件)
  \url{https://t.co/Hrk14WZ7ne} 
\item
  首相官邸さん (@kantei) / Twitter \url{https://t.co/SvaOrwX2nw}  -
  カラオケパブごまちゃんの被害者に捧げたい弁護士鉄道の記録→ :''マスコミ''/サイ太(@uwaaaa)の検索(2012-02-20〜2021-06-03/2021年06月14日20時24分の記録103件)
  \url{https://t.co/Hrk14WZ7ne} 
\item
  内閣官房さん (@Naikakukanbo) / Twitter \url{https://t.co/KuTW0xUDLd}  -
  カラオケパブごまちゃんの被害者に捧げたい弁護士鉄道の記録→ :''マスコミ''/サイ太(@uwaaaa)の検索(2012-02-20〜2021-06-03/2021年06月14日20時24分の記録103件)
  \url{https://t.co/Hrk14WZ7ne} 
\item
  総務省さん (@MIC\_JAPAN) / Twitter \url{https://t.co/2J0Fmdf2le}  -
  カラオケパブごまちゃんの被害者に捧げたい弁護士鉄道の記録→ :''マスコミ''/サイ太(@uwaaaa)の検索(2012-02-20〜2021-06-03/2021年06月14日20時24分の記録103件)
  \url{https://t.co/Hrk14WZ7ne} 
\item
  法務省さん (@MOJ\_HOUMU) / Twitter \url{https://t.co/b7aL48LWCQ}  -
  カラオケパブごまちゃんの被害者に捧げたい弁護士鉄道の記録→ :''マスコミ''/サイ太(@uwaaaa)の検索(2012-02-20〜2021-06-03/2021年06月14日20時24分の記録103件)
  \url{https://t.co/Hrk14WZ7ne} 
\item
  観光庁(Japan Tourism Agency)さん (@Kanko\_Jpn) / Twitter
  \url{https://t.co/gaZdxm8N2k}  -
  カラオケパブごまちゃんの被害者に捧げたい弁護士鉄道の記録→ :''マスコミ''/サイ太(@uwaaaa)の検索(2012-02-20〜2021-06-03/2021年06月14日20時24分の記録103件)
  \url{https://t.co/Hrk14WZ7ne} 
\item
  NHK盛岡放送局さん (@nhk\_morioka) / Twitter \url{https://t.co/f2nXv9ZbE8}  -
  カラオケパブごまちゃんの被害者に捧げたい弁護士鉄道の記録→ :''マスコミ''/サイ太(@uwaaaa)の検索(2012-02-20〜2021-06-03/2021年06月14日20時24分の記録103件)
  \url{https://t.co/Hrk14WZ7ne} 
\item
  岩手日報さん (@iwatenippo) / Twitter \url{https://t.co/bJ9jSDi6pW}  -
  カラオケパブごまちゃんの被害者に捧げたい弁護士鉄道の記録→ :''マスコミ''/サイ太(@uwaaaa)の検索(2012-02-20〜2021-06-03/2021年06月14日20時24分の記録103件)
  \url{https://t.co/Hrk14WZ7ne} 
\item
  岩手県広聴広報課さん (@pref\_iwate) / Twitter \url{https://t.co/kwaSixZijc} 
  -
  カラオケパブごまちゃんの被害者に捧げたい弁護士鉄道の記録→ :''マスコミ''/サイ太(@uwaaaa)の検索(2012-02-20〜2021-06-03/2021年06月14日20時24分の記録103件)
  \url{https://t.co/Hrk14WZ7ne} 
\item
  IBC岩手放送さん (@IBC\_online) / Twitter \url{https://t.co/q86aG9AwrY}  -
  カラオケパブごまちゃんの被害者に捧げたい弁護士鉄道の記録→ :''マスコミ''/サイ太(@uwaaaa)の検索(2012-02-20〜2021-06-03/2021年06月14日20時24分の記録103件)
  \url{https://t.co/Hrk14WZ7ne} 
\item
  立命館大学放送局(衣笠キャンパス)さん (@RBC\_KIC) / Twitter
  \url{https://t.co/1e2m9n2hvV}  -
  カラオケパブごまちゃんの被害者に捧げたい弁護士鉄道の記録→ :''マスコミ''/サイ太(@uwaaaa)の検索(2012-02-20〜2021-06-03/2021年06月14日20時24分の記録103件)
  \url{https://t.co/Hrk14WZ7ne} 
\item
  立命館大学放送局(びわこ・くさつキャンパス)さん (@RBC\_BKC) / Twitter
  \url{https://t.co/BnoUuEKA1g}  -
  カラオケパブごまちゃんの被害者に捧げたい弁護士鉄道の記録→ :''マスコミ''/サイ太(@uwaaaa)の検索(2012-02-20〜2021-06-03/2021年06月14日20時24分の記録103件)
  \url{https://t.co/Hrk14WZ7ne} 
\item
  同志社学生放送局(京田辺キャンパス)さん (@DSBtnb) / Twitter
  \url{https://t.co/yi16L3oH2O}  -
  カラオケパブごまちゃんの被害者に捧げたい弁護士鉄道の記録→ :''マスコミ''/サイ太(@uwaaaa)の検索(2012-02-20〜2021-06-03/2021年06月14日20時24分の記録103件)
  \url{https://t.co/Hrk14WZ7ne} 
\item
  KUBS 京都大学放送局さん (@kubs\_kyodai) / Twitter
  \url{https://t.co/OvWZsPRdn7}  -
  カラオケパブごまちゃんの被害者に捧げたい弁護士鉄道の記録→ :''マスコミ''/サイ太(@uwaaaa)の検索(2012-02-20〜2021-06-03/2021年06月14日20時24分の記録103件)
  \url{https://t.co/Hrk14WZ7ne} 
\item
  京都橘大学放送研究部さん (@tachibana\_ho\_so) / Twitter
  \url{https://t.co/AJYXPZasZU}  -
  カラオケパブごまちゃんの被害者に捧げたい弁護士鉄道の記録→ :''マスコミ''/サイ太(@uwaaaa)の検索(2012-02-20〜2021-06-03/2021年06月14日20時24分の記録103件)
  \url{https://t.co/Hrk14WZ7ne} 
\item
  日本大学法学部放送研究会さん (@nbf\_j) / Twitter
  \url{https://t.co/FYH7OJ2bci}  -
  カラオケパブごまちゃんの被害者に捧げたい弁護士鉄道の記録→ :''マスコミ''/サイ太(@uwaaaa)の検索(2012-02-20〜2021-06-03/2021年06月14日20時24分の記録103件)
  \url{https://t.co/Hrk14WZ7ne} 
\item
  UTBC - 東京大学放送研究会さん (@toudaihouken) / Twitter
  \url{https://t.co/hOOYLQhG0J}  -
  カラオケパブごまちゃんの被害者に捧げたい弁護士鉄道の記録→ :''マスコミ''/サイ太(@uwaaaa)の検索(2012-02-20〜2021-06-03/2021年06月14日20時24分の記録103件)
  \url{https://t.co/Hrk14WZ7ne} 
\item
  上智大学放送研究会 (SBC)さん (@sbc\_sophia) / Twitter
  \url{https://t.co/NyzONtF9tE}  -
  カラオケパブごまちゃんの被害者に捧げたい弁護士鉄道の記録→ :''マスコミ''/サイ太(@uwaaaa)の検索(2012-02-20〜2021-06-03/2021年06月14日20時24分の記録103件)
  \url{https://t.co/Hrk14WZ7ne} 
\item
  法政大学放送研究会MediaWaveさん (@HoseiMediaWave) / Twitter
  \url{https://t.co/gkrxZrcMvt}  -
  カラオケパブごまちゃんの被害者に捧げたい弁護士鉄道の記録→ :''マスコミ''/サイ太(@uwaaaa)の検索(2012-02-20〜2021-06-03/2021年06月14日20時24分の記録103件)
  \url{https://t.co/Hrk14WZ7ne} 
\item
  放送集団オケアノスさん (@ocanos) / Twitter \url{https://t.co/aKzzP6nFmH}  -
  カラオケパブごまちゃんの被害者に捧げたい弁護士鉄道の記録→ :''マスコミ''/サイ太(@uwaaaa)の検索(2012-02-20〜2021-06-03/2021年06月14日20時24分の記録103件)
  \url{https://t.co/Hrk14WZ7ne} 
\item
  京都女子大学放送研究会さん (@k\_w\_b\_c) / Twitter
  \url{https://t.co/E23Us6LDCs}  -
  カラオケパブごまちゃんの被害者に捧げたい弁護士鉄道の記録→ :''マスコミ''/サイ太(@uwaaaa)の検索(2012-02-20〜2021-06-03/2021年06月14日20時24分の記録103件)
  \url{https://t.co/Hrk14WZ7ne} 
\item
  福岡県警察本部広報課さん (@fukkei\_koho) / Twitter
  \url{https://t.co/higAkjg7BZ}  -
  カラオケパブごまちゃんの被害者に捧げたい弁護士鉄道の記録→ :''マスコミ''/サイ太(@uwaaaa)の検索(2012-02-20〜2021-06-03/2021年06月14日20時24分の記録103件)
  \url{https://t.co/Hrk14WZ7ne} 
\item
  福岡県庁さん (@Pref\_Fukuoka) / Twitter \url{https://t.co/GZ0loc3vOP}  -
  カラオケパブごまちゃんの被害者に捧げたい弁護士鉄道の記録→ :''マスコミ''/サイ太(@uwaaaa)の検索(2012-02-20〜2021-06-03/2021年06月14日20時24分の記録103件)
  \url{https://t.co/Hrk14WZ7ne} 
\item
  福岡市広報戦略室さん (@Fukuokacity\_pr) / Twitter
  \url{https://t.co/X92kNkoeP0}  -
  カラオケパブごまちゃんの被害者に捧げたい弁護士鉄道の記録→ :''マスコミ''/サイ太(@uwaaaa)の検索(2012-02-20〜2021-06-03/2021年06月14日20時24分の記録103件)
  \url{https://t.co/Hrk14WZ7ne} 
\item
  佐賀県警察さん (@goroukun\_spp) / Twitter \url{https://t.co/8D0Am4wvKr}  -
  カラオケパブごまちゃんの被害者に捧げたい弁護士鉄道の記録→ :''マスコミ''/サイ太(@uwaaaa)の検索(2012-02-20〜2021-06-03/2021年06月14日20時24分の記録103件)
  \url{https://t.co/Hrk14WZ7ne} 
\item
  大分県警察さん (@oita\_police) / Twitter \url{https://t.co/ZeuQn6kBBc}  -
  カラオケパブごまちゃんの被害者に捧げたい弁護士鉄道の記録→ :''マスコミ''/サイ太(@uwaaaa)の検索(2012-02-20〜2021-06-03/2021年06月14日20時24分の記録103件)
  \url{https://t.co/Hrk14WZ7ne} 
\item
  大分県さん (@oitapref) / Twitter \url{https://t.co/koQ5RBnsfv}  -
  カラオケパブごまちゃんの被害者に捧げたい弁護士鉄道の記録→ :''マスコミ''/サイ太(@uwaaaa)の検索(2012-02-20〜2021-06-03/2021年06月14日20時24分の記録103件)
  \url{https://t.co/Hrk14WZ7ne} 
\item
  大分市さん (@OitaCity\_PR) / Twitter \url{https://t.co/qHcnFNJphG}  -
  カラオケパブごまちゃんの被害者に捧げたい弁護士鉄道の記録→ :''マスコミ''/サイ太(@uwaaaa)の検索(2012-02-20〜2021-06-03/2021年06月14日20時24分の記録103件)
  \url{https://t.co/Hrk14WZ7ne} 
\item
  熊本県警察本部さん (@yuppi\_KK) / Twitter \url{https://t.co/Ri2GnttshV}  -
  カラオケパブごまちゃんの被害者に捧げたい弁護士鉄道の記録→ :''マスコミ''/サイ太(@uwaaaa)の検索(2012-02-20〜2021-06-03/2021年06月14日20時24分の記録103件)
  \url{https://t.co/Hrk14WZ7ne} 
\item
  熊本市さん (@kumamotocity\_) / Twitter \url{https://t.co/oe8CTA03Jl}  -
  カラオケパブごまちゃんの被害者に捧げたい弁護士鉄道の記録→ :''マスコミ''/サイ太(@uwaaaa)の検索(2012-02-20〜2021-06-03/2021年06月14日20時24分の記録103件)
  \url{https://t.co/Hrk14WZ7ne} 
\item
  熊本市長 大西一史さん (@K\_Onishi) / Twitter \url{https://t.co/LSF36nXvfm}  -
  カラオケパブごまちゃんの被害者に捧げたい弁護士鉄道の記録→ :''マスコミ''/サイ太(@uwaaaa)の検索(2012-02-20〜2021-06-03/2021年06月14日20時24分の記録103件)
  \url{https://t.co/Hrk14WZ7ne} 
\item
  熊本日日新聞社さん (@KUMANICHIs) / Twitter \url{https://t.co/lG3941rDtf}  -
  カラオケパブごまちゃんの被害者に捧げたい弁護士鉄道の記録→ :''マスコミ''/サイ太(@uwaaaa)の検索(2012-02-20〜2021-06-03/2021年06月14日20時24分の記録103件)
  \url{https://t.co/Hrk14WZ7ne} 
\item
  島根県警察安全安心情報(公式)さん (@mikopi\_SP) / Twitter
  \url{https://t.co/GokxlaXNZg}  -
  カラオケパブごまちゃんの被害者に捧げたい弁護士鉄道の記録→ :''マスコミ''/サイ太(@uwaaaa)の検索(2012-02-20〜2021-06-03/2021年06月14日20時24分の記録103件)
  \url{https://t.co/Hrk14WZ7ne} 
\item
  青森県警察本部さん (@AomoriPolice) / Twitter \url{https://t.co/MbxbFSioJr}  -
  カラオケパブごまちゃんの被害者に捧げたい弁護士鉄道の記録→ :''マスコミ''/サイ太(@uwaaaa)の検索(2012-02-20〜2021-06-03/2021年06月14日20時24分の記録103件)
  \url{https://t.co/Hrk14WZ7ne} 
\item
  松江市防災情報さん (@bousai\_matsue) / Twitter \url{https://t.co/7r5HcFcBRZ} 
  -
  カラオケパブごまちゃんの被害者に捧げたい弁護士鉄道の記録→ :''マスコミ''/サイ太(@uwaaaa)の検索(2012-02-20〜2021-06-03/2021年06月14日20時24分の記録103件)
  \url{https://t.co/Hrk14WZ7ne} 
\item
  神奈川県警察本部刑事部捜査第二課さん (@KPP\_souni) / Twitter
  \url{https://t.co/LdAvrP7zww}  -
  カラオケパブごまちゃんの被害者に捧げたい弁護士鉄道の記録→ :''マスコミ''/サイ太(@uwaaaa)の検索(2012-02-20〜2021-06-03/2021年06月14日20時24分の記録103件)
  \url{https://t.co/Hrk14WZ7ne} 
\item
  神奈川県警察本部犯罪抑止対策室さん (@KPP\_yokushi) / Twitter
  \url{https://t.co/NdCnuSkzce}  -
  カラオケパブごまちゃんの被害者に捧げたい弁護士鉄道の記録→ :''マスコミ''/サイ太(@uwaaaa)の検索(2012-02-20〜2021-06-03/2021年06月14日20時24分の記録103件)
  \url{https://t.co/Hrk14WZ7ne} 
\item
  茨城県警察本部(公式)さん (@ibarakipolice) / Twitter
  \url{https://t.co/yD9ywvgQEg}  -
  カラオケパブごまちゃんの被害者に捧げたい弁護士鉄道の記録→ :''マスコミ''/サイ太(@uwaaaa)の検索(2012-02-20〜2021-06-03/2021年06月14日20時24分の記録103件)
  \url{https://t.co/Hrk14WZ7ne} 
\item
  茨城県さん (@Ibaraki\_Kouhou) / Twitter \url{https://t.co/nMdHU6f2yd}  -
  カラオケパブごまちゃんの被害者に捧げたい弁護士鉄道の記録→ :''マスコミ''/サイ太(@uwaaaa)の検索(2012-02-20〜2021-06-03/2021年06月14日20時24分の記録103件)
  \url{https://t.co/Hrk14WZ7ne} 
\item
  茨城新聞社さん (@ibarakishimbun) / Twitter \url{https://t.co/sFNC0T4bE5}  -
  カラオケパブごまちゃんの被害者に捧げたい弁護士鉄道の記録→ :''マスコミ''/サイ太(@uwaaaa)の検索(2012-02-20〜2021-06-03/2021年06月14日20時24分の記録103件)
  \url{https://t.co/Hrk14WZ7ne} 
\item
  茨城県鹿嶋市(公式)さん (@kashima\_city) / Twitter
  \url{https://t.co/FhbqPzSib8}  -
  カラオケパブごまちゃんの被害者に捧げたい弁護士鉄道の記録→ :''マスコミ''/サイ太(@uwaaaa)の検索(2012-02-20〜2021-06-03/2021年06月14日20時24分の記録103件)
  \url{https://t.co/Hrk14WZ7ne} 
\item
  神栖市観光協会さん (@kamisu\_kanko) / Twitter \url{https://t.co/7QoHz1c62v} 
  -
  カラオケパブごまちゃんの被害者に捧げたい弁護士鉄道の記録→ :''マスコミ''/サイ太(@uwaaaa)の検索(2012-02-20〜2021-06-03/2021年06月14日20時24分の記録103件)
  \url{https://t.co/Hrk14WZ7ne} 
\item
  水戸市さん (@kouhou\_mito) / Twitter \url{https://t.co/h46G2TWYOk}  -
  カラオケパブごまちゃんの被害者に捧げたい弁護士鉄道の記録→ :''マスコミ''/サイ太(@uwaaaa)の検索(2012-02-20〜2021-06-03/2021年06月14日20時24分の記録103件)
  \url{https://t.co/Hrk14WZ7ne} 
\item
  土浦市(公式)さん (@tsuchiura\_city) / Twitter
  \url{https://t.co/jvS9l3SJkX}  -
  カラオケパブごまちゃんの被害者に捧げたい弁護士鉄道の記録→ :''マスコミ''/サイ太(@uwaaaa)の検索(2012-02-20〜2021-06-03/2021年06月14日20時24分の記録103件)
  \url{https://t.co/Hrk14WZ7ne} 
\item
  龍ケ崎市(茨城県)さん (@ryugasaki\_city) / Twitter
  \url{https://t.co/MJpQPWT5ng}  -
  カラオケパブごまちゃんの被害者に捧げたい弁護士鉄道の記録→ :''マスコミ''/サイ太(@uwaaaa)の検索(2012-02-20〜2021-06-03/2021年06月14日20時24分の記録103件)
  \url{https://t.co/Hrk14WZ7ne} 
\item
  牛久市(公式)さん (@ushiku\_city) / Twitter \url{https://t.co/2gWtEhiVeh}  -
  カラオケパブごまちゃんの被害者に捧げたい弁護士鉄道の記録→ :''マスコミ''/サイ太(@uwaaaa)の検索(2012-02-20〜2021-06-03/2021年06月14日20時24分の記録103件)
  \url{https://t.co/Hrk14WZ7ne} 
\item
  つくばみらい市さん (@tsukubamirai\_c) / Twitter
  \url{https://t.co/tZqTfJpX25}  -
  カラオケパブごまちゃんの被害者に捧げたい弁護士鉄道の記録→ :''マスコミ''/サイ太(@uwaaaa)の検索(2012-02-20〜2021-06-03/2021年06月14日20時24分の記録103件)
  \url{https://t.co/Hrk14WZ7ne} 
\item
  千葉県警察さん (@Chibakenkei) / Twitter \url{https://t.co/EaGtt2iSOa}  -
  カラオケパブごまちゃんの被害者に捧げたい弁護士鉄道の記録→ :''マスコミ''/サイ太(@uwaaaa)の検索(2012-02-20〜2021-06-03/2021年06月14日20時24分の記録103件)
  \url{https://t.co/Hrk14WZ7ne} 
\item
  千葉日報さん (@chibanippo) / Twitter \url{https://t.co/2g7ZunORPB}  -
  カラオケパブごまちゃんの被害者に捧げたい弁護士鉄道の記録→ :''マスコミ''/サイ太(@uwaaaa)の検索(2012-02-20〜2021-06-03/2021年06月14日20時24分の記録103件)
  \url{https://t.co/Hrk14WZ7ne} 
\item
  神戸新聞さん (@kobeshinbun) / Twitter \url{https://t.co/BPnv3B8eSF}  -
  カラオケパブごまちゃんの被害者に捧げたい弁護士鉄道の記録→ :''マスコミ''/サイ太(@uwaaaa)の検索(2012-02-20〜2021-06-03/2021年06月14日20時24分の記録103件)
  \url{https://t.co/Hrk14WZ7ne} 
\end{itemize}

〉〉〉 kk\_hironoのリツイート 〉〉〉

\begin{itemize}
\tightlist
\item
  RT
  kk\_hirono(刑事告発・非常上告_金沢地方検察庁御中)|hirono\_hideki(奉納\さらば弁護士鉄道・泥棒神社の物語)
  日時:2021-06-14 21:49/2021/06/14 21:47 URL:
  \url{https://twitter.com/kk\_hirono/status/1404420795774824452} 
  \url{https://twitter.com/hirono\_hideki/status/1404420142675628042} 
  \textgreater{} ニューズウィーク日本版さん (@Newsweek\_JAPAN) / Twitter
  \url{https://t.co/F0npu71ZqQ}  -
  カラオケパブごまちゃんの被害者に捧げたい弁護士鉄道の記録→ :''マスコミ''/サイ太(@uwaaaa)の検索(2012-02-20〜2021-06-03/2021年06月14日20時24分の記録103件)
  \url{https://t.co/f1G8wTFFzx} 
\end{itemize}

〉〉〉 kk\_hironoのリツイート 〉〉〉

\begin{itemize}
\tightlist
\item
  RT
  kk\_hirono(刑事告発・非常上告_金沢地方検察庁御中)|hirono\_hideki(奉納\さらば弁護士鉄道・泥棒神社の物語)
  日時:2021-06-14 21:49/2021/06/14 21:46 URL:
  \url{https://twitter.com/kk\_hirono/status/1404420803496579076} 
  \url{https://twitter.com/hirono\_hideki/status/1404420061670957064} 
  \textgreater{} ロイターさん (@ReutersJapan) / Twitter
  \url{https://t.co/2wYMQnWAO3}  -
  カラオケパブごまちゃんの被害者に捧げたい弁護士鉄道の記録→ :''マスコミ''/サイ太(@uwaaaa)の検索(2012-02-20〜2021-06-03/2021年06月14日20時24分の記録103件)
  \url{https://t.co/f1G8wTFFzx} 
\end{itemize}

〉〉〉 kk\_hironoのリツイート 〉〉〉

\begin{itemize}
\tightlist
\item
  RT
  kk\_hirono(刑事告発・非常上告_金沢地方検察庁御中)|hirono\_hideki(奉納\さらば弁護士鉄道・泥棒神社の物語)
  日時:2021-06-14 21:49/2021/06/14 21:46 URL:
  \url{https://twitter.com/kk\_hirono/status/1404420819875336198} 
  \url{https://twitter.com/hirono\_hideki/status/1404420003433041930} 
  \textgreater{} ウォール・ストリート・ジャーナル日本版さん (@WSJJapan)
  / Twitter \url{https://t.co/nFnbGurO9u}  -
  カラオケパブごまちゃんの被害者に捧げたい弁護士鉄道の記録→ :''マスコミ''/サイ太(@uwaaaa)の検索(2012-02-20〜2021-06-03/2021年06月14日20時24分の記録103件)
  \url{https://t.co/f1G8wTFFzx} 
\end{itemize}

〉〉〉 kk\_hironoのリツイート 〉〉〉

\begin{itemize}
\tightlist
\item
  RT
  kk\_hirono(刑事告発・非常上告_金沢地方検察庁御中)|hirono\_hideki(奉納\さらば弁護士鉄道・泥棒神社の物語)
  日時:2021-06-14 21:49/2021/06/14 21:46 URL:
  \url{https://twitter.com/kk\_hirono/status/1404420829241217032} 
  \url{https://twitter.com/hirono\_hideki/status/1404419954657566723} 
  \textgreater{} 日本経済新聞 電子版(日経電子版)さん (@nikkei) /
  Twitter \url{https://t.co/cBzanJRTub}  -
  カラオケパブごまちゃんの被害者に捧げたい弁護士鉄道の記録→ :''マスコミ''/サイ太(@uwaaaa)の検索(2012-02-20〜2021-06-03/2021年06月14日20時24分の記録103件)
  \url{https://t.co/f1G8wTFFzx} 
\end{itemize}

〉〉〉 kk\_hironoのリツイート 〉〉〉

\begin{itemize}
\tightlist
\item
  RT
  kk\_hirono(刑事告発・非常上告_金沢地方検察庁御中)|hirono\_hideki(奉納\さらば弁護士鉄道・泥棒神社の物語)
  日時:2021-06-14 21:49/2021/06/14 21:46 URL:
  \url{https://twitter.com/kk\_hirono/status/1404420843568910345} 
  \url{https://twitter.com/hirono\_hideki/status/1404419896549597189} 
  \textgreater{} 毎日新聞さん (@mainichi) / Twitter
  \url{https://t.co/h1ygLebE5q}  -
  カラオケパブごまちゃんの被害者に捧げたい弁護士鉄道の記録→ :''マスコミ''/サイ太(@uwaaaa)の検索(2012-02-20〜2021-06-03/2021年06月14日20時24分の記録103件)
  \url{https://t.co/f1G8wTFFzx} 
\end{itemize}

〉〉〉 kk\_hironoのリツイート 〉〉〉

\begin{itemize}
\tightlist
\item
  RT
  kk\_hirono(刑事告発・非常上告_金沢地方検察庁御中)|hirono\_hideki(奉納\さらば弁護士鉄道・泥棒神社の物語)
  日時:2021-06-14 21:49/2021/06/14 21:45 URL:
  \url{https://twitter.com/kk\_hirono/status/1404420852230152196} 
  \url{https://twitter.com/hirono\_hideki/status/1404419850747867145} 
  \textgreater{} 朝日新聞(asahi shimbun)さん (@asahi) / Twitter
  \url{https://t.co/759qwK6GAq}  -
  カラオケパブごまちゃんの被害者に捧げたい弁護士鉄道の記録→ :''マスコミ''/サイ太(@uwaaaa)の検索(2012-02-20〜2021-06-03/2021年06月14日20時24分の記録103件)
  \url{https://t.co/f1G8wTFFzx} 
\end{itemize}

〉〉〉 kk\_hironoのリツイート 〉〉〉

\begin{itemize}
\tightlist
\item
  RT
  kk\_hirono(刑事告発・非常上告_金沢地方検察庁御中)|hirono\_hideki(奉納\さらば弁護士鉄道・泥棒神社の物語)
  日時:2021-06-14 21:50/2021/06/14 21:45 URL:
  \url{https://twitter.com/kk\_hirono/status/1404420861268938758} 
  \url{https://twitter.com/hirono\_hideki/status/1404419807324217346} 
  \textgreater{} 読売新聞オンラインさん (@Yomiuri\_Online) / Twitter
  \url{https://t.co/Mxn2jGoaJS}  -
  カラオケパブごまちゃんの被害者に捧げたい弁護士鉄道の記録→ :''マスコミ''/サイ太(@uwaaaa)の検索(2012-02-20〜2021-06-03/2021年06月14日20時24分の記録103件)
  \url{https://t.co/f1G8wTFFzx} 
\end{itemize}

〉〉〉 kk\_hironoのリツイート 〉〉〉

\begin{itemize}
\tightlist
\item
  RT
  kk\_hirono(刑事告発・非常上告_金沢地方検察庁御中)|hirono\_hideki(奉納\さらば弁護士鉄道・泥棒神社の物語)
  日時:2021-06-14 21:50/2021/06/14 21:45 URL:
  \url{https://twitter.com/kk\_hirono/status/1404420873008816130} 
  \url{https://twitter.com/hirono\_hideki/status/1404419764777209861} 
  \textgreater{} 産経ニュースさん (@Sankei\_news) / Twitter
  \url{https://t.co/PtdvGaGYTV}  -
  カラオケパブごまちゃんの被害者に捧げたい弁護士鉄道の記録→ :''マスコミ''/サイ太(@uwaaaa)の検索(2012-02-20〜2021-06-03/2021年06月14日20時24分の記録103件)
  \url{https://t.co/f1G8wTFFzx} 
\end{itemize}

〉〉〉 kk\_hironoのリツイート 〉〉〉

\begin{itemize}
\tightlist
\item
  RT
  kk\_hirono(刑事告発・非常上告_金沢地方検察庁御中)|hirono\_hideki(奉納\さらば弁護士鉄道・泥棒神社の物語)
  日時:2021-06-14 21:50/2021/06/14 21:43 URL:
  \url{https://twitter.com/kk\_hirono/status/1404420884543152129} 
  \url{https://twitter.com/hirono\_hideki/status/1404419270117773315} 
  \textgreater{} 長野県軽井沢町(公式)さん (@karuizawatown) / Twitter
  \url{https://t.co/G77TpWfeyy}  -
  カラオケパブごまちゃんの被害者に捧げたい弁護士鉄道の記録→ :''マスコミ''/サイ太(@uwaaaa)の検索(2012-02-20〜2021-06-03/2021年06月14日20時24分の記録103件)
  \url{https://t.co/f1G8wTFFzx} 
\end{itemize}

〉〉〉 kk\_hironoのリツイート 〉〉〉

\begin{itemize}
\tightlist
\item
  RT
  kk\_hirono(刑事告発・非常上告_金沢地方検察庁御中)|hirono\_hideki(奉納\さらば弁護士鉄道・泥棒神社の物語)
  日時:2021-06-14 21:50/2021/06/14 21:43 URL:
  \url{https://twitter.com/kk\_hirono/status/1404420893921529857} 
  \url{https://twitter.com/hirono\_hideki/status/1404419216753582080} 
  \textgreater{} 軽井沢新聞社さん (@vignette01) / Twitter
  \url{https://t.co/zu4pMVLqVe}  -
  カラオケパブごまちゃんの被害者に捧げたい弁護士鉄道の記録→ :''マスコミ''/サイ太(@uwaaaa)の検索(2012-02-20〜2021-06-03/2021年06月14日20時24分の記録103件)
  \url{https://t.co/f1G8wTFFzx} 
\end{itemize}

〉〉〉 kk\_hironoのリツイート 〉〉〉

\begin{itemize}
\tightlist
\item
  RT
  kk\_hirono(刑事告発・非常上告_金沢地方検察庁御中)|hirono\_hideki(奉納\さらば弁護士鉄道・泥棒神社の物語)
  日時:2021-06-14 21:50/2021/06/14 21:43 URL:
  \url{https://twitter.com/kk\_hirono/status/1404420904600313857} 
  \url{https://twitter.com/hirono\_hideki/status/1404419165067251718} 
  \textgreater{} 南房総市広報さん (@minamiboso\_koho) / Twitter
  \url{https://t.co/L5a0sYQaWp}  -
  カラオケパブごまちゃんの被害者に捧げたい弁護士鉄道の記録→ :''マスコミ''/サイ太(@uwaaaa)の検索(2012-02-20〜2021-06-03/2021年06月14日20時24分の記録103件)
  \url{https://t.co/f1G8wTFFzx} 
\end{itemize}

〉〉〉 kk\_hironoのリツイート 〉〉〉

\begin{itemize}
\tightlist
\item
  RT
  kk\_hirono(刑事告発・非常上告_金沢地方検察庁御中)|hirono\_hideki(奉納\さらば弁護士鉄道・泥棒神社の物語)
  日時:2021-06-14 21:50/2021/06/14 21:43 URL:
  \url{https://twitter.com/kk\_hirono/status/1404420915161559042} 
  \url{https://twitter.com/hirono\_hideki/status/1404419118074187781} 
  \textgreater{} 房日新聞社さん (@Bonichi1948) / Twitter
  \url{https://t.co/Yo1SLBc992}  -
  カラオケパブごまちゃんの被害者に捧げたい弁護士鉄道の記録→ :''マスコミ''/サイ太(@uwaaaa)の検索(2012-02-20〜2021-06-03/2021年06月14日20時24分の記録103件)
  \url{https://t.co/f1G8wTFFzx} 
\end{itemize}

〉〉〉 kk\_hironoのリツイート 〉〉〉

\begin{itemize}
\tightlist
\item
  RT
  kk\_hirono(刑事告発・非常上告_金沢地方検察庁御中)|hirono\_hideki(奉納\さらば弁護士鉄道・泥棒神社の物語)
  日時:2021-06-14 21:50/2021/06/14 21:42 URL:
  \url{https://twitter.com/kk\_hirono/status/1404420927941529603} 
  \url{https://twitter.com/hirono\_hideki/status/1404419005129973761} 
  \textgreater{} PR TIMESさん (@PRTIMES\_JP) / Twitter
  \url{https://t.co/FOmVcoodeV}  -
  カラオケパブごまちゃんの被害者に捧げたい弁護士鉄道の記録→ :''マスコミ''/サイ太(@uwaaaa)の検索(2012-02-20〜2021-06-03/2021年06月14日20時24分の記録103件)
  \url{https://t.co/f1G8wTFFzx} 
\end{itemize}

〉〉〉 kk\_hironoのリツイート 〉〉〉

\begin{itemize}
\tightlist
\item
  RT
  kk\_hirono(刑事告発・非常上告_金沢地方検察庁御中)|hirono\_hideki(奉納\さらば弁護士鉄道・泥棒神社の物語)
  日時:2021-06-14 21:50/2021/06/14 21:42 URL:
  \url{https://twitter.com/kk\_hirono/status/1404420937433317381} 
  \url{https://twitter.com/hirono\_hideki/status/1404418952617369607} 
  \textgreater{} 弁護士 金田万作さん (@mansakukanada) / Twitter
  \url{https://t.co/x0IXlaUDKE}  -
  カラオケパブごまちゃんの被害者に捧げたい弁護士鉄道の記録→ :''マスコミ''/サイ太(@uwaaaa)の検索(2012-02-20〜2021-06-03/2021年06月14日20時24分の記録103件)
  \url{https://t.co/f1G8wTFFzx} 
\end{itemize}

〉〉〉 kk\_hironoのリツイート 〉〉〉

\begin{itemize}
\tightlist
\item
  RT
  kk\_hirono(刑事告発・非常上告_金沢地方検察庁御中)|hirono\_hideki(奉納\さらば弁護士鉄道・泥棒神社の物語)
  日時:2021-06-14 21:50/2021/06/14 21:42 URL:
  \url{https://twitter.com/kk\_hirono/status/1404420948204294146} 
  \url{https://twitter.com/hirono\_hideki/status/1404418909009178630} 
  \textgreater{} ルート66(元ルパン3世)さん (@Route66\_LP3) /
  Twitter \url{https://t.co/V1VCbvH3pk}  -
  カラオケパブごまちゃんの被害者に捧げたい弁護士鉄道の記録→ :''マスコミ''/サイ太(@uwaaaa)の検索(2012-02-20〜2021-06-03/2021年06月14日20時24分の記録103件)
  \url{https://t.co/f1G8wTFFzx} 
\end{itemize}

〉〉〉 kk\_hironoのリツイート 〉〉〉

\begin{itemize}
\tightlist
\item
  RT
  kk\_hirono(刑事告発・非常上告_金沢地方検察庁御中)|hirono\_hideki(奉納\さらば弁護士鉄道・泥棒神社の物語)
  日時:2021-06-14 21:50/2021/06/14 21:41 URL:
  \url{https://twitter.com/kk\_hirono/status/1404420959612719108} 
  \url{https://twitter.com/hirono\_hideki/status/1404418843053727744} 
  \textgreater{} 判例時報・編集部さん (@hanreijiho) / Twitter
  \url{https://t.co/2O4lRmeF0W}  -
  カラオケパブごまちゃんの被害者に捧げたい弁護士鉄道の記録→ :''マスコミ''/サイ太(@uwaaaa)の検索(2012-02-20〜2021-06-03/2021年06月14日20時24分の記録103件)
  \url{https://t.co/f1G8wTFFzx} 
\end{itemize}

〉〉〉 kk\_hironoのリツイート 〉〉〉

\begin{itemize}
\tightlist
\item
  RT
  kk\_hirono(刑事告発・非常上告_金沢地方検察庁御中)|hirono\_hideki(奉納\さらば弁護士鉄道・泥棒神社の物語)
  日時:2021-06-14 21:50/2021/06/14 21:41 URL:
  \url{https://twitter.com/kk\_hirono/status/1404420972531261442} 
  \url{https://twitter.com/hirono\_hideki/status/1404418739295031299} 
  \textgreater{} 某役所の中の人の1人さん (@judgeinjp) / Twitter
  \url{https://t.co/75SoSDK2lO}  -
  カラオケパブごまちゃんの被害者に捧げたい弁護士鉄道の記録→ :''マスコミ''/サイ太(@uwaaaa)の検索(2012-02-20〜2021-06-03/2021年06月14日20時24分の記録103件)
  \url{https://t.co/f1G8wTFFzx} 
\end{itemize}

〉〉〉 kk\_hironoのリツイート 〉〉〉

\begin{itemize}
\tightlist
\item
  RT
  kk\_hirono(刑事告発・非常上告_金沢地方検察庁御中)|hirono\_hideki(奉納\さらば弁護士鉄道・泥棒神社の物語)
  日時:2021-06-14 21:50/2021/06/14 21:41 URL:
  \url{https://twitter.com/kk\_hirono/status/1404420983851601920} 
  \url{https://twitter.com/hirono\_hideki/status/1404418685238865920} 
  \textgreater{} Jはお前なんだよさん (@tako\_kora\_) / Twitter
  \url{https://t.co/4YwEAd9CeF}  -
  カラオケパブごまちゃんの被害者に捧げたい弁護士鉄道の記録→ :''マスコミ''/サイ太(@uwaaaa)の検索(2012-02-20〜2021-06-03/2021年06月14日20時24分の記録103件)
  \url{https://t.co/f1G8wTFFzx} 
\end{itemize}

〉〉〉 kk\_hironoのリツイート 〉〉〉

\begin{itemize}
\tightlist
\item
  RT
  kk\_hirono(刑事告発・非常上告_金沢地方検察庁御中)|hirono\_hideki(奉納\さらば弁護士鉄道・泥棒神社の物語)
  日時:2021-06-14 21:50/2021/06/14 21:41 URL:
  \url{https://twitter.com/kk\_hirono/status/1404420994610073610} 
  \url{https://twitter.com/hirono\_hideki/status/1404418629395877896} 
  \textgreater{} とある裁判所書記官さん (@saibanshokikan) / Twitter
  \url{https://t.co/39bZh4mtlj}  -
  カラオケパブごまちゃんの被害者に捧げたい弁護士鉄道の記録→ :''マスコミ''/サイ太(@uwaaaa)の検索(2012-02-20〜2021-06-03/2021年06月14日20時24分の記録103件)
  \url{https://t.co/f1G8wTFFzx} 
\end{itemize}

〉〉〉 kk\_hironoのリツイート 〉〉〉

\begin{itemize}
\tightlist
\item
  RT
  kk\_hirono(刑事告発・非常上告_金沢地方検察庁御中)|hirono\_hideki(奉納\さらば弁護士鉄道・泥棒神社の物語)
  日時:2021-06-14 21:50/2021/06/14 21:40 URL:
  \url{https://twitter.com/kk\_hirono/status/1404421005548740617} 
  \url{https://twitter.com/hirono\_hideki/status/1404418585888317450} 
  \textgreater{} 匿名裁判官さん (@courts\_jp) / Twitter
  \url{https://t.co/cbDqTAEgl1}  -
  カラオケパブごまちゃんの被害者に捧げたい弁護士鉄道の記録→ :''マスコミ''/サイ太(@uwaaaa)の検索(2012-02-20〜2021-06-03/2021年06月14日20時24分の記録103件)
  \url{https://t.co/f1G8wTFFzx} 
\end{itemize}

〉〉〉 kk\_hironoのリツイート 〉〉〉

\begin{itemize}
\tightlist
\item
  RT
  kk\_hirono(刑事告発・非常上告_金沢地方検察庁御中)|hirono\_hideki(奉納\さらば弁護士鉄道・泥棒神社の物語)
  日時:2021-06-14 21:50/2021/06/14 21:40 URL:
  \url{https://twitter.com/kk\_hirono/status/1404421017854910472} 
  \url{https://twitter.com/hirono\_hideki/status/1404418538144624640} 
  \textgreater{} 弁護士 山中理司さん (@yamanaka\_osaka) / Twitter
  \url{https://t.co/7LlDuU6Je5}  -
  カラオケパブごまちゃんの被害者に捧げたい弁護士鉄道の記録→ :''マスコミ''/サイ太(@uwaaaa)の検索(2012-02-20〜2021-06-03/2021年06月14日20時24分の記録103件)
  \url{https://t.co/f1G8wTFFzx} 
\end{itemize}

〉〉〉 kk\_hironoのリツイート 〉〉〉

\begin{itemize}
\tightlist
\item
  RT
  kk\_hirono(刑事告発・非常上告_金沢地方検察庁御中)|hirono\_hideki(奉納\さらば弁護士鉄道・泥棒神社の物語)
  日時:2021-06-14 21:50/2021/06/14 21:40 URL:
  \url{https://twitter.com/kk\_hirono/status/1404421030324539392} 
  \url{https://twitter.com/hirono\_hideki/status/1404418494217691137} 
  \textgreater{} 井田良さん (@idaprof) / Twitter \url{https://t.co/fKC2L28A5v} 
  -
  カラオケパブごまちゃんの被害者に捧げたい弁護士鉄道の記録→ :''マスコミ''/サイ太(@uwaaaa)の検索(2012-02-20〜2021-06-03/2021年06月14日20時24分の記録103件)
  \url{https://t.co/f1G8wTFFzx} 
\end{itemize}

〉〉〉 kk\_hironoのリツイート 〉〉〉

\begin{itemize}
\item
  RT
  kk\_hirono(刑事告発・非常上告_金沢地方検察庁御中)|hirono\_hideki(奉納\さらば弁護士鉄道・泥棒神社の物語)
  日時:2021-06-14 21:50/2021/06/14 21:40 URL:
  \url{https://twitter.com/kk\_hirono/status/1404421039874998280} 
  \url{https://twitter.com/hirono\_hideki/status/1404418439700041728} 
  \textgreater{} 有斐閣雑誌編集部さん (@Jurist\_Hogaku) / Twitter
  \url{https://t.co/JYQbKH82KH}  -
  カラオケパブごまちゃんの被害者に捧げたい弁護士鉄道の記録→ :''マスコミ''/サイ太(@uwaaaa)の検索(2012-02-20〜2021-06-03/2021年06月14日20時24分の記録103件)
  \url{https://t.co/f1G8wTFFzx} 
\item
  〈〈〈 2021/06/14 21:51:36 Linux Emacs: 〈〈〈
\end{itemize}

\hypertarget{ux548cux6b4cux5c71ux30abux30ecux30fcux4e8bux4ef6ux5bb6ux65cfux306eux5fc3ux4e2dux4e8bux4ef6ux6df1ux6fa4ux8aedux53f2ux5f01ux8b77ux58ebux306eux53f8ux6cd5ux5236ux5ea6ux6539ux9769uxff12uxff10ux5e74ux306eux546aux3044ux3068ux3044ux3046ux30c4ux30a4ux30fcux30c8ux8133ux6a5fux80fdux969cux5bb3ux306eux5c11ux5973ux306eux5bb6ux65cfux304cux88abux5bb3ux306bux906dux3063ux305fux5f01ux8b77ux58ebux4e8bux4ef6ux304bux3089ux8003ux5bdfux3059ux308bux5f01ux8b77ux58ebux9244ux9053ux306eux5b9fux9332ux30c9ux30adux30e5ux30e1ux30f3ux30c82}{%
\paragraph{和歌山カレー事件家族の心中事件、深澤諭史弁護士の「司法制度改革20年の呪い」というツイート、脳機能障害の少女の家族が被害に遭った弁護士事件から考察する弁護士鉄道の実録ドキュメント(2)}\label{ux548cux6b4cux5c71ux30abux30ecux30fcux4e8bux4ef6ux5bb6ux65cfux306eux5fc3ux4e2dux4e8bux4ef6ux6df1ux6fa4ux8aedux53f2ux5f01ux8b77ux58ebux306eux53f8ux6cd5ux5236ux5ea6ux6539ux9769uxff12uxff10ux5e74ux306eux546aux3044ux3068ux3044ux3046ux30c4ux30a4ux30fcux30c8ux8133ux6a5fux80fdux969cux5bb3ux306eux5c11ux5973ux306eux5bb6ux65cfux304cux88abux5bb3ux306bux906dux3063ux305fux5f01ux8b77ux58ebux4e8bux4ef6ux304bux3089ux8003ux5bdfux3059ux308bux5f01ux8b77ux58ebux9244ux9053ux306eux5b9fux9332ux30c9ux30adux30e5ux30e1ux30f3ux30c82}}

\begin{itemize}
\tightlist
\item
  〉〉〉 Linux Emacs: 2021/06/14 21:54:03 〉〉〉
\end{itemize}

:CATEGORIES: @kanazawabengosi \#金沢弁護士会 @JFBAsns
日本弁護士連合会(日弁連) \#法務省 @MOJ\_HOUMU \#深澤諭史弁護士
\#刑裁サイ太 \#金沢中警察署

\begin{itemize}
\tightlist
\item
  \#告発状 \#\#\#\#
  和歌山カレー事件家族の心中事件、深澤諭史弁護士の「司法制度改革20年の呪い」というツイート、脳機能障害の少女の家族が被害に遭った弁護士事件から考察する弁護士鉄道の実録ドキュメント
  - 告発\金沢地方検察庁\最高検察庁\法務省\石川県警察御中2020
  \url{https://t.co/SescZXAyD0} 
\end{itemize}

 埋め込みツイートをやりすぎたと思ったのですが、大丈夫だったようです。

 時刻は10時48分です。和歌山カレー事件の報道に進展はみられませんが、この問題に対する弁護士の反応が皆無に等しく、完全スルーと言ってもよさそうです。冤罪の可能性が高いと見る機運がTwitterの検索では多いのですが、だとすれば歴史的な報道被害で、家族の人生がまるごと台無しにされたわけです。

 弁護士の仕事について気になったツイートをいくつか見かけているのでご紹介しておきたいと思います。1つは昨日か一昨日に深澤諭史弁護士のタイムラインで見かけた深澤諭史弁護士のリツイートです。

\begin{itemize}
\tightlist
\item
  RT fukazawas(深澤諭史)|shoheikoyanolaw(古家野 彰平)
  日時:2021-06-14 17:57/2021-06-14 17:28 URL:
  \url{https://twitter.com/fukazawas/status/1404362311192649731} 
  \url{https://twitter.com/shoheikoyanolaw/status/1404355038529474563} 
  \textgreater{}
  弁護士に紛争案件の処理を依頼するメリットって、経済的利益に目が向きがちなんですが、①相手方と直接交渉するストレスや手間を回避したり、②親しい人に相談すると重くなる話を弁護士に託すことで心を軽くしたりと、紛争で暗転した身の回りの風景を正常化させる効果もあったりするんですよね。
\end{itemize}

 少し調べて意味を確認したのですが、セラピストに近い感じです。また、弁護士は占い師に似ていると思えることもありますが、法律相談より占い師の方が時間あたりの費用が高いとぼやく弁護士のツイートというのはこれまで何度か見かけてきました。

 次に大阪のパチパチ弁護士のタイムラインで本日見かけたツイートです。ブックマークには入れていないアカウントですが、昨日の天満のカラオケパブの事件のこともあったので、見かけたアカウントのタイムラインを遡ってみました。

〉〉〉 kk\_hironoのリツイート 〉〉〉

\begin{itemize}
\tightlist
\item
  RT
  kk\_hirono(刑事告発・非常上告_金沢地方検察庁御中)|obpmb3fN93mQI9i(大阪名物パチパチ弁護士)
  日時:2021-06-15 11:16/2021/06/14 09:19 URL:
  \url{https://twitter.com/kk\_hirono/status/1404623724716851203} 
  \url{https://twitter.com/obpmb3fN93mQI9i/status/1404231990052757505} 
  \textgreater{}
  いじめとか性的搾取、児童虐待など「子供が声を上げにくい」被害の防止のため、マンガを誰か作ってくれへんかな。
  なんか強引にでもスーパー弁護士が出てきてハッピーエンドにできるやつがええ。
  原爆教育で「はだしのゲン」を学校の図書館に置くのが効果的やったやろ。それと同じ狙いや。
\end{itemize}

〉〉〉 kk\_hironoのリツイート 〉〉〉

\begin{itemize}
\tightlist
\item
  RT
  kk\_hirono(刑事告発・非常上告_金沢地方検察庁御中)|obpmb3fN93mQI9i(大阪名物パチパチ弁護士)
  日時:2021-06-15 11:16/2021/06/14 09:39 URL:
  \url{https://twitter.com/kk\_hirono/status/1404623771038666759} 
  \url{https://twitter.com/obpmb3fN93mQI9i/status/1404237112522600451} 
  \textgreater{}
  なお、想定している主人公とストーリーは、例えば「パチパチ弁護士マンだ!覚悟しろ!発信者特定請求ビーム!損害賠償パンチ!国家賠償コンボアタック!」とかいって解決するような感じ。1話完結で12話くらい。
\end{itemize}

〉〉〉 kk\_hironoのリツイート 〉〉〉

\begin{itemize}
\tightlist
\item
  RT
  kk\_hirono(刑事告発・非常上告_金沢地方検察庁御中)|noooooooorth(教皇ノースライム)
  日時:2021-06-15 11:16/2021/06/14 07:58 URL:
  \url{https://twitter.com/kk\_hirono/status/1404623838286016527} 
  \url{https://twitter.com/noooooooorth/status/1404211676157673475} 
  \textgreater{}
  個人的には弁護士の仕事は「法律を中心とした知識と技術で問題解決or問題発生の予防を提供してお金をいただく仕事」と表現したい。
\end{itemize}

〉〉〉 kk\_hironoのリツイート 〉〉〉

\begin{itemize}
\item
  RT
  kk\_hirono(刑事告発・非常上告_金沢地方検察庁御中)|obpmb3fN93mQI9i(大阪名物パチパチ弁護士)
  日時:2021-06-15 11:17/2021/06/12 21:42 URL:
  \url{https://twitter.com/kk\_hirono/status/1404624027956637701} 
  \url{https://twitter.com/obpmb3fN93mQI9i/status/1403694220402065410} 
  \textgreater{}
  追い込まれると、生きるために、何だってするようになってまうねんなあ。。。
  \url{https://t.co/EpsYueN1Fa} 
\item
  〉〉〉 アカウント(@O59K2dPQH59QEJx)は,@kk\_hironoをブロックしています。リツイートできませんでした。
  〉〉〉 ¥\n ¥\n \url{https://t.co/X0PZCDE3y7} 
\end{itemize}

※ @kk\_hironoのアカウントがブロックされ,リツイートに失敗したツイート

\begin{itemize}
\tightlist
\item
  TW O59K2dPQH59QEJx(ピピピーッ) 日時:2021/06/12 20:48:08 URL:
  \url{https://twitter.com/O59K2dPQH59QEJx/status/1403680514561634310} 
  \textgreater{} 金に困った貧民の道徳観など、猿未満。\\
  \textgreater{} むしろ、仲間の死を悲しむ猿のほうが文化資本が高い。\\
  \textgreater{}\\
  \textgreater{}
  後先考えず、「金を引っ張ること」しか考えていない貧民って、マジでおるんよね。\\
  \textgreater{}
  騙し、色恋を使い、友情やら血縁やらでごまかし、とにかく現預金を奪い使うことしか考えとらんクズ。
\end{itemize}

 なかには自覚することなく依頼者や家族、親族の窮地を加速し倍加する弁護士というのもいそうですが、もともと弁護士の仕事の中に含まれた危険性と思います。本来手当が必要なはずですが、和歌山カレー事件をスルーする弁護士を見ていると、改善というのは望めないと思います。

 福井刑務所では三省録というノートに反省文のようなことを書かされていたのですが、福井女子中学生殺人事件の逆転有罪判決があった頃で、警察に批判を向けるだけの弁護士に反省点などあるのかと考えたことをよく憶えています。

 私の中では、福井女子中学生殺人事件の逆転有罪判決を出したのが被告発人小島裕史裁判長だったということだけではなく、同じ頃にネットで長文の再審請求を読んだのも和歌山カレー事件との交わりのある共通点でした。

 今はネットでほとんど情報を見かけない福井女子中学生殺人事件ですが、冤罪の可能性が高いとされ、実際、再審開始の決定も出ています。名古屋高裁金沢支部が再審開始の決定を出し、名古屋高裁本庁が検察の抗告を受け破棄しているはずです。

\begin{quote}
《引用の始まり》
\end{quote}

\begin{quote}
大阪名物パチパチ弁護士@obpmb3fN93mQI9i氏名非公表、身バレ厳禁(といっても結構バレてる)の弁護士。安全ピントラブル対策弁護団員。痴漢なくなれ!https://twitter.com/anzenpinbengoda

50期中盤、年齢40代、旧試組。パチパチ本音言いまくります。炎上上等!あと愛妻家。長男長女(6歳以下)の育児と仕事の両立を目指し日々奮闘。togetter.com/li/13611032018年2月からTwitterを利用しています6,908
フォロー中1万 フォロワー
\end{quote}

\begin{quote}
《引用の終わり》
\end{quote}

\begin{itemize}
\tightlist
\item
  大阪名物パチパチ弁護士さん (@obpmb3fN93mQI9i) / Twitter
  \url{https://twitter.com/obpmb3fN93mQI9i} 
\end{itemize}

 大阪名物パチパチ弁護士さん (@obpmb3fN93mQI9i)
のプロフィールですが、「安全ピントラブル対策弁護団員」とあり、モトケンこと矢部善朗弁護士(京都弁護士会)も関わっていました。依頼の実績はゼロとしか思えないのですが、数年前からずっと続いています。

 昨日は、Googleマップで大阪市内の地図をみて、いくつか発見というか気づきがあったのですが、天満は北区とあるので、中心部から離れたところというイメージがそれまでありました。同じく長距離トラックの仕事でよく行った福島区もずいぶん中心部に近いのだと気が付きました。

 2つのエントリーの2つ目にに分かれていますが、一番のメインに取り上げておきたかったのが、脳機能障害の少女の家族の話で、その家族の人生を狂わせた弁護士も大阪の弁護士になっていました。ただ、新潟県燕市の辺りでも仕事をしてそこでも悪事をやって発覚したようです。

 昨日の続きになりますが、「佐藤幸治」をキーワードにまとめた深澤諭史弁護士のツイートから、煎じて飲ませたいという深澤諭史弁護士のツイートを探します。

\begin{itemize}
\item
  奉納\危険生物・弁護士脳汚染除去装置\金沢地方検察庁御中\_2020:
  REGEXP:''佐藤幸治''/深澤諭史(@fukazawas)の検索(2013-01-31〜2021-04-22/2021年06月14日16時39分の記録107件)
  \url{https://t.co/9aV2KKPOLR} 
\item
  (011/107) TW fukazawas(深澤諭史) 日時:2015-06-28 12:22:00 +0900
  URL:
  \url{https://twitter.com/fukazawas/status/614997243226009600\textgreater} {}
  平成の司法改革,佐藤幸治先生や高橋宏志先生らが夢見た,新しい公選弁護のあり方ですね。\textgreater{}
  (#・∀・)両先生もさぞやお喜びでしょう。 \url{https://t.co/TkcXuB5njC} 
\end{itemize}

 探していた深澤諭史弁護士のツイートを見つけました。このあと時間を確認しますが、次のツイートになっていました。

\begin{itemize}
\tightlist
\item
  (012/107) TW fukazawas(深澤諭史) 日時:2015-07-22 18:49:00 +0900
  URL:
  \url{https://twitter.com/fukazawas/status/623791961539919873\textgreater} {}
  >RT\textgreater{}
  自由競争とか,市場原理とか,無邪気にいっている先生,特に佐藤幸治先生とかには,百回くらい読んで頂きたい記事ですね。\textgreater{}
  (#・∀・)
\end{itemize}

 感覚としてもう2年ぐらいは見かけなくなっていると思いますが、「>RT」というスタイルの深澤諭史弁護士のツイートは以前よく見かけたもので、タイムラインにツイートがあるときは、一つ前のツイートを指しているのだと理解できますが、検索で関連性を掴むことは不可能に近いかもしれません。

 千や二千のツイートしかないアカウントであれば、時間を掛けてタイムラインを遡れば、そのうち見つかるかもしれないですが、深澤諭史弁護士のアカウントのツイート数は、2021年6月15日の今現在で12.8万件となっています。

 そのための記録ということもあるのですが、これまでに何度かスクリーンショットを記憶しています。

〉〉〉 kk\_hironoのリツイート 〉〉〉

\begin{itemize}
\tightlist
\item
  RT
  kk\_hirono(刑事告発・非常上告_金沢地方検察庁御中)|s\_hirono(非常上告-最高検察庁御中\_ツイッター)
  日時:2021-06-15 12:00/2020/11/28 19:47 URL:
  \url{https://twitter.com/kk\_hirono/status/1404634906219597827} 
  \url{https://twitter.com/s\_hirono/status/1332637112999395335} 
  \textgreater{}
  2015-07-22-192255\_深澤諭史@fukazawas>RT自由競争とか,市場原理とか,無邪気にいっている先生,特に佐藤幸治先生とかには,百回くらい読んで頂きたい記.jpg
  \url{https://t.co/hGllTTn042} 
\end{itemize}

〉〉〉 kk\_hironoのリツイート 〉〉〉

\begin{itemize}
\tightlist
\item
  RT
  kk\_hirono(刑事告発・非常上告_金沢地方検察庁御中)|s\_hirono(非常上告-最高検察庁御中\_ツイッター)
  日時:2021-06-15 12:00/2020/11/28 19:46 URL:
  \url{https://twitter.com/kk\_hirono/status/1404634941296496640} 
  \url{https://twitter.com/s\_hirono/status/1332637099661541378} 
  \textgreater{}
  2015-07-22-192154\_自由競争とか,市場原理とか,無邪気にいっている先生,特に佐藤幸治先生とかには,百回くらい読んで頂きたい記事ですね。.jpg
  \url{https://t.co/w3YQWYuQMK} 
\end{itemize}

〉〉〉 kk\_hironoのリツイート 〉〉〉

\begin{itemize}
\tightlist
\item
  RT
  kk\_hirono(刑事告発・非常上告_金沢地方検察庁御中)|s\_hirono(非常上告-最高検察庁御中\_ツイッター)
  日時:2021-06-15 12:00/2020/10/17 21:39 URL:
  \url{https://twitter.com/kk\_hirono/status/1404634997105954819} 
  \url{https://twitter.com/s\_hirono/status/1317445235866177539} 
  \textgreater{}
  2020-10-17-213945\_深澤諭史@fukazawas>RT自由競争とか,市場原理とか,無邪気にいっている先生,特に佐藤幸治先生とかには,百回くらい読んで頂きたい記事.jpg
  \url{https://t.co/AC58lwwQf7} 
\end{itemize}

〉〉〉 kk\_hironoのリツイート 〉〉〉

\begin{itemize}
\tightlist
\item
  RT
  kk\_hirono(刑事告発・非常上告_金沢地方検察庁御中)|s\_hirono(非常上告-最高検察庁御中\_ツイッター)
  日時:2021-06-15 12:01/2020/08/18 15:18 URL:
  \url{https://twitter.com/kk\_hirono/status/1404635029259522055} 
  \url{https://twitter.com/s\_hirono/status/1295605915530309632} 
  \textgreater{}
  2020-08-18-151441\_深澤諭史@fukazawas>RT自由競争とか,市場原理とか,無邪気にいっている先生,特に佐藤幸治先生とかには,百回くらい読んで頂きたい記事.jpg
  \url{https://t.co/OmMK7bKP8h} 
\end{itemize}

〉〉〉 kk\_hironoのリツイート 〉〉〉

\begin{itemize}
\tightlist
\item
  RT
  kk\_hirono(刑事告発・非常上告_金沢地方検察庁御中)|s\_hirono(非常上告-最高検察庁御中\_ツイッター)
  日時:2021-06-15 12:02/2020/08/18 09:43 URL:
  \url{https://twitter.com/kk\_hirono/status/1404635476200345605} 
  \url{https://twitter.com/s\_hirono/status/1295521745294655488} 
  \textgreater{}
  2020-08-17-200234\_深澤諭史@fukazawas>RT自由競争とか,市場原理とか,無邪気にいっている先生,特に佐藤幸治先生とかには,百回くらい読んで頂きたい記事.jpg
  \url{https://t.co/gaHgT0mXQ7} 
\end{itemize}

〉〉〉 kk\_hironoのリツイート 〉〉〉

\begin{itemize}
\tightlist
\item
  RT
  kk\_hirono(刑事告発・非常上告_金沢地方検察庁御中)|s\_hirono(非常上告-最高検察庁御中\_ツイッター)
  日時:2021-06-15 12:02/2020/08/18 09:43 URL:
  \url{https://twitter.com/kk\_hirono/status/1404635512032227334} 
  \url{https://twitter.com/s\_hirono/status/1295521672527716353} 
  \textgreater{}
  2020-08-17-200104\_深澤諭史@fukazawas>RT自由競争とか,市場原理とか,無邪気にいっている先生,特に佐藤幸治先生とかには,百回くらい読んで頂きたい記事.jpg
  \url{https://t.co/o94VQQvcAd} 
\end{itemize}

〉〉〉 kk\_hironoのリツイート 〉〉〉

\begin{itemize}
\tightlist
\item
  RT
  kk\_hirono(刑事告発・非常上告_金沢地方検察庁御中)|s\_hirono(非常上告-最高検察庁御中\_ツイッター)
  日時:2021-06-15 12:03/2019/06/21 14:47 URL:
  \url{https://twitter.com/kk\_hirono/status/1404635548170407940} 
  \url{https://twitter.com/s\_hirono/status/1141945768741236741} 
  \textgreater{}
  2019-06-21-143938\_深澤諭史さんのツイート: ''>RT 自由競争とか,市場原理とか,無邪気にいっている先生,特に佐藤幸治先生とかには,百回くらい読んで頂きたい記事ですね。 (.jpg
  \url{https://t.co/n5PofrOAaP} 
\end{itemize}

〉〉〉 kk\_hironoのリツイート 〉〉〉

\begin{itemize}
\tightlist
\item
  RT
  kk\_hirono(刑事告発・非常上告_金沢地方検察庁御中)|s\_hirono(非常上告-最高検察庁御中\_ツイッター)
  日時:2021-06-15 12:03/2018/04/11 00:27 URL:
  \url{https://twitter.com/kk\_hirono/status/1404635609080107016} 
  \url{https://twitter.com/s\_hirono/status/983728242874920960} 
  \textgreater{}
  2018-04-10-024958\_自由競争とか,市場原理とか,無邪気にいっている先生,特に佐藤幸治先生とかには,百回くらい読んで頂きたい記事ですね。 (#・∀・).jpg
  \url{https://t.co/5x09bkh1aM} 
\end{itemize}

〉〉〉 kk\_hironoのリツイート 〉〉〉

\begin{itemize}
\tightlist
\item
  RT
  kk\_hirono(刑事告発・非常上告_金沢地方検察庁御中)|s\_hirono(非常上告-最高検察庁御中\_ツイッター)
  日時:2021-06-15 12:03/2018/04/10 02:53 URL:
  \url{https://twitter.com/kk\_hirono/status/1404635656198922243} 
  \url{https://twitter.com/s\_hirono/status/983402595703111681} 
  \textgreater{}
  2018-04-10-024958\_自由競争とか,市場原理とか,無邪気にいっている先生,特に佐藤幸治先生とかには,百回くらい読んで頂きたい記事ですね。 (#・∀・).jpg
  \url{https://t.co/IhmY9TNyZJ} 
\end{itemize}

〉〉〉 kk\_hironoのリツイート 〉〉〉

\begin{itemize}
\tightlist
\item
  RT
  kk\_hirono(刑事告発・非常上告_金沢地方検察庁御中)|s\_hirono(非常上告-最高検察庁御中\_ツイッター)
  日時:2021-06-15 12:03/2017/10/09 15:59 URL:
  \url{https://twitter.com/kk\_hirono/status/1404635696384528391} 
  \url{https://twitter.com/s\_hirono/status/917283465053167617} 
  \textgreater{}
  2017-10-09-155950\_深澤諭史 > >RT 自由競争とか,市場原理とか,無邪気にいっている先生,特に佐藤幸治先生とかには,百回くらい読んで頂きたい記事ですね。 (.jpg
  \url{https://t.co/bQ9loiP7l7} 
\end{itemize}

〉〉〉 kk\_hironoのリツイート 〉〉〉

\begin{itemize}
\tightlist
\item
  RT
  kk\_hirono(刑事告発・非常上告_金沢地方検察庁御中)|s\_hirono(非常上告-最高検察庁御中\_ツイッター)
  日時:2021-06-15 12:03/2016/11/06 07:23 URL:
  \url{https://twitter.com/kk\_hirono/status/1404635726956752897} 
  \url{https://twitter.com/s\_hirono/status/795028695270363136} 
  \textgreater{}
  2015-07-22-192255\_深澤諭史@fukazawas>RT自由競争とか,市場原理とか,無邪気にいっている先生,特に佐藤幸治先生とかには,百回くらい読んで頂きたい記.jpg
  \url{https://t.co/FJw6NCsrQh} 
\end{itemize}

〉〉〉 kk\_hironoのリツイート 〉〉〉

\begin{itemize}
\tightlist
\item
  RT
  kk\_hirono(刑事告発・非常上告_金沢地方検察庁御中)|s\_hirono(非常上告-最高検察庁御中\_ツイッター)
  日時:2021-06-15 12:03/2016/11/06 07:23 URL:
  \url{https://twitter.com/kk\_hirono/status/1404635755926802432} 
  \url{https://twitter.com/s\_hirono/status/795028678237298688} 
  \textgreater{}
  2015-07-22-192154\_自由競争とか,市場原理とか,無邪気にいっている先生,特に佐藤幸治先生とかには,百回くらい読んで頂きたい記事ですね。.jpg
  \url{https://t.co/L0jDLxasQ0} 
\end{itemize}

〉〉〉 kk\_hironoのリツイート 〉〉〉

\begin{itemize}
\tightlist
\item
  RT
  kk\_hirono(刑事告発・非常上告_金沢地方検察庁御中)|s\_hirono(非常上告-最高検察庁御中\_ツイッター)
  日時:2021-06-15 12:04/2016/11/04 10:49 URL:
  \url{https://twitter.com/kk\_hirono/status/1404635785408614408} 
  \url{https://twitter.com/s\_hirono/status/794355931756662784} 
  \textgreater{}
  2016-11-04-104942\_自由競争とか,市場原理とか,無邪気にいっている先生,特に佐藤幸治先生とかには,百回くらい読んで頂きたい記事ですね。 (#・∀・).jpg
  \url{https://t.co/7lA9RNJryM} 
\end{itemize}

〉〉〉 kk\_hironoのリツイート 〉〉〉

\begin{itemize}
\tightlist
\item
  RT
  kk\_hirono(刑事告発・非常上告_金沢地方検察庁御中)|s\_hirono(非常上告-最高検察庁御中\_ツイッター)
  日時:2021-06-15 12:04/2016/11/04 10:29 URL:
  \url{https://twitter.com/kk\_hirono/status/1404635818451341313} 
  \url{https://twitter.com/s\_hirono/status/794350728256974849} 
  \textgreater{}
  2015-07-22-192255\_深澤諭史@fukazawas>RT自由競争とか,市場原理とか,無邪気にいっている先生,特に佐藤幸治先生とかには,百回くらい読んで頂きたい記.jpg
  \url{https://t.co/Tt8D8lkbkX} 
\end{itemize}

〉〉〉 kk\_hironoのリツイート 〉〉〉

\begin{itemize}
\tightlist
\item
  RT
  kk\_hirono(刑事告発・非常上告_金沢地方検察庁御中)|s\_hirono(非常上告-最高検察庁御中\_ツイッター)
  日時:2021-06-15 12:04/2016/11/04 10:29 URL:
  \url{https://twitter.com/kk\_hirono/status/1404635848247705606} 
  \url{https://twitter.com/s\_hirono/status/794350711723073537} 
  \textgreater{}
  2015-07-22-192154\_自由競争とか,市場原理とか,無邪気にいっている先生,特に佐藤幸治先生とかには,百回くらい読んで頂きたい記事ですね。.jpg
  \url{https://t.co/yV2VCCgcR9} 
\end{itemize}

〉〉〉 kk\_hironoのリツイート 〉〉〉

\begin{itemize}
\tightlist
\item
  RT
  kk\_hirono(刑事告発・非常上告_金沢地方検察庁御中)|s\_hirono(非常上告-最高検察庁御中\_ツイッター)
  日時:2021-06-15 12:04/2016/04/26 12:33 URL:
  \url{https://twitter.com/kk\_hirono/status/1404635884071182339} 
  \url{https://twitter.com/s\_hirono/status/724803666121621510} 
  \textgreater{}
  2016-04-26-123350\_深澤諭史 @fukazawas  2015年7月22日>RT自由競争とか,市場原理とか,無邪気にいっている先生,特に佐藤幸治先生とかには,.jpg
  \url{https://t.co/BiFtBwMHHX} 
\end{itemize}

〉〉〉 kk\_hironoのリツイート 〉〉〉

\begin{itemize}
\tightlist
\item
  RT
  kk\_hirono(刑事告発・非常上告_金沢地方検察庁御中)|s\_hirono(非常上告-最高検察庁御中\_ツイッター)
  日時:2021-06-15 12:04/2015/07/22 19:23 URL:
  \url{https://twitter.com/kk\_hirono/status/1404635914748403718} 
  \url{https://twitter.com/s\_hirono/status/623800402090176514} 
  \textgreater{}
  2015-07-22-192255\_深澤諭史@fukazawas>RT自由競争とか,市場原理とか,無邪気にいっている先生,特に佐藤幸治先生とかには,百回くらい読んで頂きたい記.jpg
  \url{http://t.co/VtSkKUknbQ} 
\end{itemize}

〉〉〉 kk\_hironoのリツイート 〉〉〉

\begin{itemize}
\tightlist
\item
  RT
  kk\_hirono(刑事告発・非常上告_金沢地方検察庁御中)|s\_hirono(非常上告-最高検察庁御中\_ツイッター)
  日時:2021-06-15 12:04/2015/07/22 19:21 URL:
  \url{https://twitter.com/kk\_hirono/status/1404635952765542409} 
  \url{https://twitter.com/s\_hirono/status/623800146728370176} 
  \textgreater{}
  2015-07-22-192154\_自由競争とか,市場原理とか,無邪気にいっている先生,特に佐藤幸治先生とかには,百回くらい読んで頂きたい記事ですね。.jpg
  \url{http://t.co/Yjv22BMfLb} 
\end{itemize}

 時刻は15時07分です。昼は外で海鮮丼を食べてきたのですが、量が多くお腹がいっぱいになりました。そのあと宇出津新港に買い物に行っています。メインとして深澤諭史弁護士のツイートは記録の作業ができましたが、その脳機能障害の少女の家族と弁護士についても稿を改め記録しておきたいと思います。

\begin{itemize}
\tightlist
\item
  〈〈〈 2021/06/15 15:11:23 Linux Emacs: 〈〈〈
\end{itemize}

\hypertarget{ux548cux6b4cux5c71ux30abux30ecux30fcux4e8bux4ef6ux5bb6ux65cfux306eux5fc3ux4e2dux4e8bux4ef6ux6df1ux6fa4ux8aedux53f2ux5f01ux8b77ux58ebux306eux53f8ux6cd5ux5236ux5ea6ux6539ux9769uxff12uxff10ux5e74ux306eux546aux3044ux3068ux3044ux3046ux30c4ux30a4ux30fcux30c8ux8133ux6a5fux80fdux969cux5bb3ux306eux5c11ux5973ux306eux5bb6ux65cfux304cux88abux5bb3ux306bux906dux3063ux305fux5f01ux8b77ux58ebux4e8bux4ef6ux304bux3089ux8003ux5bdfux3059ux308bux5f01ux8b77ux58ebux9244ux9053ux306eux5b9fux9332ux30c9ux30adux30e5ux30e1ux30f3ux30c83}{%
\paragraph{和歌山カレー事件家族の心中事件、深澤諭史弁護士の「司法制度改革20年の呪い」というツイート、脳機能障害の少女の家族が被害に遭った弁護士事件から考察する弁護士鉄道の実録ドキュメント(3)}\label{ux548cux6b4cux5c71ux30abux30ecux30fcux4e8bux4ef6ux5bb6ux65cfux306eux5fc3ux4e2dux4e8bux4ef6ux6df1ux6fa4ux8aedux53f2ux5f01ux8b77ux58ebux306eux53f8ux6cd5ux5236ux5ea6ux6539ux9769uxff12uxff10ux5e74ux306eux546aux3044ux3068ux3044ux3046ux30c4ux30a4ux30fcux30c8ux8133ux6a5fux80fdux969cux5bb3ux306eux5c11ux5973ux306eux5bb6ux65cfux304cux88abux5bb3ux306bux906dux3063ux305fux5f01ux8b77ux58ebux4e8bux4ef6ux304bux3089ux8003ux5bdfux3059ux308bux5f01ux8b77ux58ebux9244ux9053ux306eux5b9fux9332ux30c9ux30adux30e5ux30e1ux30f3ux30c83}}

\begin{itemize}
\tightlist
\item
  〉〉〉 Linux Emacs: 2021/06/15 15:14:10 〉〉〉
\end{itemize}

:CATEGORIES: @kanazawabengosi \#金沢弁護士会 @JFBAsns
日本弁護士連合会(日弁連) \#法務省 @MOJ\_HOUMU \#深澤諭史弁護士
\#弁護士の犯罪 \#司法制度改革

 海鮮丼を食べ、宇出津新港のどんたく宇出津店で買い物をして家に戻った後のことになりますが、ツイートのリンクを辿ると意外な発見がありました。ドラえもんにどこでもドアというのがあったように思いますが、弁護士の新しい世界が開かれたので、ご紹介しておきたいと思います。

\begin{itemize}
\tightlist
\item
  都 行志/Miyako Kojiさん (@Miyako\_Koji) / Twitter
  \url{https://twitter.com/Miyako\_Koji} 
\end{itemize}

 忘れていたのですが、リンクを逆に辿ると上記の都行志弁護士のタイムラインが始まりでした。関連性のないツイートも一緒にご紹介しておきますが、同じ都行志弁護士のタイムラインにあるものです。

 メインとして取り上げておきたかったのは、都行志弁護士がリツイートをしたまゆろんという鳥取の女性弁護士のツイートから始まります。

〉〉〉 kk\_hironoのリツイート 〉〉〉

\begin{itemize}
\tightlist
\item
  RT
  kk\_hirono(刑事告発・非常上告_金沢地方検察庁御中)|kikugawa\_ben(菊川弁護士/青梅&新宿の弁護士)
  日時:2021-06-15 15:22/2021/06/15 12:01 URL:
  \url{https://twitter.com/kk\_hirono/status/1404685785383661569} 
  \url{https://twitter.com/kikugawa\_ben/status/1404635068681711620} 
  \textgreater{} 福永、また訴訟提起してしました\^{}\^{};
  高橋先生のツイートをRTしたことが不法行為らしいです。
\end{itemize}

〉〉〉 kk\_hironoのリツイート 〉〉〉

\begin{itemize}
\tightlist
\item
  RT
  kk\_hirono(刑事告発・非常上告_金沢地方検察庁御中)|take\_\textbf{five(中村剛(take-five))
  日時:2021-06-15 15:23/2021/06/15 10:18 URL:
  \url{https://twitter.com/kk\_hirono/status/1404685893823172610} 
  \url{https://twitter.com/take} }\_five/status/1404609187905040386\\
  \textgreater{}
  ところ構わずケンカを売りまくって、やっぱこちらが諦めるからなかったことにしてねと言われても、ところ構わずケンカを売ったという事実はなくならないので、なかったことにはならないよね。ケンカを仕掛けるからには、ボロッカスに負ける可能性も当然想定しなきゃいけないよね。
\end{itemize}

〉〉〉 kk\_hironoのリツイート 〉〉〉

\begin{itemize}
\tightlist
\item
  RT
  kk\_hirono(刑事告発・非常上告_金沢地方検察庁御中)|mayukotaniguchi(まゆろん🏋️‍♂️💪筋トレの効果を実感中)
  日時:2021-06-15 15:23/2021/06/15 03:23 URL:
  \url{https://twitter.com/kk\_hirono/status/1404685936705769472} 
  \url{https://twitter.com/mayukotaniguchi/status/1404504734229139457} 
  \textgreater{} それは感じる。 強くなければ弁護士はできない。
  優しくなければ弁護士たる資格がない。 \url{https://t.co/b0QO1woPOy} 
\end{itemize}

〉〉〉 kk\_hironoのリツイート 〉〉〉

\begin{itemize}
\tightlist
\item
  RT
  kk\_hirono(刑事告発・非常上告_金沢地方検察庁御中)|John35408507(Alvin@企業法務系弁護士)
  日時:2021-06-15 15:23/2021/06/14 21:42 URL:
  \url{https://twitter.com/kk\_hirono/status/1404686005148340227} 
  \url{https://twitter.com/John35408507/status/1404419005968834561} 
  \textgreater{}
  結局、肉体的・精神的なタフネスが弁護士として成功するための必須の条件であるという結論に至っている
  そして、これは弁護士になった以降に努力して伸ばせるものでもないような気がしている
\end{itemize}

〉〉〉 kk\_hironoのリツイート 〉〉〉

\begin{itemize}
\tightlist
\item
  RT
  kk\_hirono(刑事告発・非常上告_金沢地方検察庁御中)|machiben\_global(町田弁太郎)
  日時:2021-06-15 15:24/2021/06/14 21:48 URL:
  \url{https://twitter.com/kk\_hirono/status/1404686131782770691} 
  \url{https://twitter.com/machiben\_global/status/1404420560965181440} 
  \textgreater{} @John35408507 成功するには、
  金銭欲と客を食い物に出来る図太さ
  時間と自分と自分の財布を大事にして、それ以外に一切価値を認めない、そういう熱い気持ち
  が大事かと
\end{itemize}

〉〉〉 kk\_hironoのリツイート 〉〉〉

\begin{itemize}
\tightlist
\item
  RT
  kk\_hirono(刑事告発・非常上告_金沢地方検察庁御中)|John35408507(Alvin@企業法務系弁護士)
  日時:2021-06-15 15:24/2021/06/14 22:11 URL:
  \url{https://twitter.com/kk\_hirono/status/1404686229073928193} 
  \url{https://twitter.com/John35408507/status/1404426158985277444} 
  \textgreater{} @machiben\_global
  そこまでは申し上げません。そのレベルの方の成功のもっと手前の成功です。
\end{itemize}

〉〉〉 kk\_hironoのリツイート 〉〉〉

\begin{itemize}
\tightlist
\item
  RT
  kk\_hirono(刑事告発・非常上告_金沢地方検察庁御中)|machiben\_global(町田弁太郎)
  日時:2021-06-15 15:24/2021/06/15 13:14 URL:
  \url{https://twitter.com/kk\_hirono/status/1404686326427836417} 
  \url{https://twitter.com/machiben\_global/status/1404653454417547264} 
  \textgreater{} 他人事として割り切れるドライさ。
  他人事として客観視・俯瞰視出来ることはクライアントにとっても有益。傍目八目。
  民事裁判官にとって必要な能力は、予断と偏見。これがないと事件が停滞する。
  地方行脚している出来ない裁判官は、明らかに予断力と偏見力が欠けている。
  刑事裁判官も同じか。 \url{https://t.co/zIIptKDtGS} 
\end{itemize}

〉〉〉 kk\_hironoのリツイート 〉〉〉

\begin{itemize}
\tightlist
\item
  RT
  kk\_hirono(刑事告発・非常上告_金沢地方検察庁御中)|leHMxNYV0XhsBdS(通りすがりの者)
  日時:2021-06-15 15:25/2021/06/12 21:48 URL:
  \url{https://twitter.com/kk\_hirono/status/1404686397961740289} 
  \url{https://twitter.com/leHMxNYV0XhsBdS/status/1403695586851790857} 
  \textgreater{}
  司法試験はある程度地頭と才能は必要。とは思いますが、弁護士として成功するために必要なものは、分野別でしょうが、また別にもあるように思います。
  鈍感力とか、いくら叩かれても折れない鋼のメンタルといったものも、その一つだとは思いますけれど。
\end{itemize}

〉〉〉 kk\_hironoのリツイート 〉〉〉

\begin{itemize}
\tightlist
\item
  RT
  kk\_hirono(刑事告発・非常上告_金沢地方検察庁御中)|kouya7977(林宏弥)
  日時:2021-06-15 15:26/2021/06/13 07:40 URL:
  \url{https://twitter.com/kk\_hirono/status/1404686712874311680} 
  \url{https://twitter.com/kouya7977/status/1403844607536820224} 
  \textgreater{}
  私は、司法試験、一定の地頭必要だとは思いますが(東大京大レベルとかいうわけではない、そこまで高い必要ない)、一番大切、重要な要素は、ともすれば繰り返し繰り返しですあきちゃうこと諦めてしまいたくなることあると思いますが、それらに負けず継続して「努力」できる力だと思っています。
  \url{https://t.co/nUqgbYRvlV} 
\end{itemize}

〉〉〉 kk\_hironoのリツイート 〉〉〉

\begin{itemize}
\tightlist
\item
  RT
  kk\_hirono(刑事告発・非常上告_金沢地方検察庁御中)|kouya7977(林宏弥)
  日時:2021-06-15 15:26/2021/06/13 16:09 URL:
  \url{https://twitter.com/kk\_hirono/status/1404686813499998208} 
  \url{https://twitter.com/kouya7977/status/1403972891205529601} 
  \textgreater{}
  司法試験をゲーム攻略に例えてる方もいらっしゃいますが、私は違うと思っています。なぜなら、はじめの段階は別として司法試験の勉強は繰り返し繰り返しで、ゲームをするときのようにいつまでも心踊るものではないから‥‥
  それでも、継続して努力できるかどうかだと思う
\end{itemize}

〉〉〉 kk\_hironoのリツイート 〉〉〉

\begin{itemize}
\tightlist
\item
  RT
  kk\_hirono(刑事告発・非常上告_金沢地方検察庁御中)|leHMxNYV0XhsBdS(通りすがりの者)
  日時:2021-06-15 15:27/2021/06/13 17:51 URL:
  \url{https://twitter.com/kk\_hirono/status/1404686917816393729} 
  \url{https://twitter.com/leHMxNYV0XhsBdS/status/1403998528267902979} 
  \textgreater{} @kouya7977
  継続して努力することも、大切なことかもしれませんね。
  私は、目の前の相手の挙措から感情を読み取ることと、そういった感情を抱くことになった理由を察知する能力も、同じように大切かもしれないと思っています。
  街弁的感覚が強すぎるのかもしれませんが。
\end{itemize}

〉〉〉 kk\_hironoのリツイート 〉〉〉

\begin{itemize}
\tightlist
\item
  RT
  kk\_hirono(刑事告発・非常上告_金沢地方検察庁御中)|kouya7977(林宏弥)
  日時:2021-06-15 15:27/2021/06/13 18:16 URL:
  \url{https://twitter.com/kk\_hirono/status/1404687003485044736} 
  \url{https://twitter.com/kouya7977/status/1404004760718434306} 
  \textgreater{} @leHMxNYV0XhsBdS
  私が、努力が一番重要といったのは司法試験に受かるためにはということです。先生のツイートに対応してませんでしたね;;
  実務家として成功するためには先生が挙げられてることなど他の要素も必要になってくると思ってます。
\end{itemize}

〉〉〉 kk\_hironoのリツイート 〉〉〉

\begin{itemize}
\item
  RT
  kk\_hirono(刑事告発・非常上告_金沢地方検察庁御中)|leHMxNYV0XhsBdS(通りすがりの者)
  日時:2021-06-15 15:27/2021/06/13 20:39 URL:
  \url{https://twitter.com/kk\_hirono/status/1404687087668924422} 
  \url{https://twitter.com/leHMxNYV0XhsBdS/status/1404040729777754114} 
  \textgreater{} @kouya7977
  どう見てもそのように読むべきツイートでしたね、すみません。
  努力できることは才能の一部、でも努力すれば何でもできるというわけではない、しかし不断の努力なしでは成し遂げられないこともある。
  難しいですね・・・
\item
  〉〉〉 アカウント(@Miyako\_Koji)は,@kk\_hironoをブロックしています。リツイートできませんでした。
  〉〉〉 ¥\n ¥\n \url{https://t.co/gzejKin78J} 
\end{itemize}

※ @kk\_hironoのアカウントがブロックされ,リツイートに失敗したツイート

\begin{itemize}
\tightlist
\item
  TW Miyako\_Koji(都 行志/Miyako Koji) 日時:2021/06/15 12:51:04
  URL: \url{https://twitter.com/Miyako\_Koji/status/1404647618941046791} 
  \textgreater{}
  私も福永活也氏から、次から次に訴訟が追加されてきていますが、皆様のご支援を受けて、戦い抜きたいと思っています。カンパをして下さった皆様、弁護団の先生方、本当にありがとうございます。皆様のご支援がなければ、この圧力に屈していたかもしれません。
\end{itemize}

 感想を一言でいえば「弁護士劇場」です。都行志弁護士から始まりましたが、もともと群馬県高崎市に事務所を開いていた弁護士で、最初にTwitterアカウントを見かけたときから高崎市だったように思います。

 詳しいこと正確なことは知らないのですが、群馬県の高崎市の付近で思い出すのが国定忠治で、昭和40年代はよく見聞きする機会がありました。母親の影響ということはなく母親がテレビ以外にレコードを聴いているのもラジオを聴いているのも見たことは一度もありません。テレビだけです。芝居も同じく。

 「赤城の山も今宵かぎり」で始まる名ゼリフがあったと思います。それとは別になるのかと思いますが、「瞼の母」という似たような世界観の古い曲があって、これは拘置所のラジオ放送でたびたび聴いていました。「浪曲十八番」というラジオ番組の中だったかもしれません。

 国定忠治は江戸時代末期のヤクザの親分で抗争があって、そこから名台詞も出たのだと思うのですが、この令和の時代に勃発しているのが、この都行志弁護士と福永活也弁護士のしのぎをけずるような訴訟合戦で、投げ銭よろしく、クラウドファンディングで支援する人も多数いるようです。

 福永活也弁護士については、出版されている本のタイトルを見ただけでも物語や小説の登場人物と思えるのですが、現実に活動する弁護士で、かなりの数、言動を問題視され弁護士会に懲戒請求も受けているようですが、少なくとも今のところは本業である弁護士活動に支障なくやっているようです。

\begin{itemize}
\tightlist
\item
  日本一稼ぐ弁護士の仕事術 \textbar{} 福永 活也 \textbar 本 \textbar{}
  通販 \textbar{} Amazon \url{https://t.co/qYCS6mCd94}  2019/7/12
\end{itemize}

 今年の1月に図書館の取り寄せで、「石川県奥能登でおきた蛸島学童殺人事件裁判記録」と2冊一緒に借りて読んでいます。期待したほど具体的な記述は見られなかったのですが、東日本大震災で多数の信頼を得ながら実績を積み重ね、本のタイトルにあるような成功をおさめたようです。

 大成功をやっかんだTwitterの弁護士らが、不当に評価を貶めたという趣旨になっていたようにも思うのですが、この福永活也弁護士はやたらと自分のツイートを削除するという行動癖があり、なかなか正確と思われる情報源に辿り着けないということがありました。

 昨日あたりも、目を疑うような内容のツイートのスクリーンショットをみかけたのですが、日付を見るとずいぶん前のもので、本当に実在したツイートなのかという実感がわきませんでした。なにかとトラブルや軋轢を推奨し、活性化しているように思えるのですが、弁護士の仕事を増やしているのかと。

※ @kk\_hironoのアカウントがブロックされ,リツイートに失敗したツイート

\begin{itemize}
\tightlist
\item
  TW kamatatylaw(高橋雄一郎) 日時:2021/06/14 07:40:45 URL:
  \url{https://twitter.com/kamatatylaw/status/1404207139552260099} 
  \textgreater{}
  訴えの変更でツイが追加されると一面クリアで次の面に進むみたいなやつですかね。
  \url{https://t.co/YAX1QJqG5M} 
\end{itemize}

 上記の高橋雄一郎弁護士のツイートに、福永活也弁護士のツイートのスクリーンショットがあります。2020年9月21日のツイートとなっています。たかがTwitterなのでインベーダ感覚、所詮架空のゲームなどとあります。1年経っていないようですが、リツイートした相手を訴えたような話が出ています。

\begin{itemize}
\tightlist
\item
  TW kamatatylaw(高橋雄一郎) 日時: 2021/06/14 07:28:58 URL:
  \url{https://twitter.com/kamatatylaw/status/1404204172099031046} 
  \textgreater{}
  確かに,自分がたまたま弁護士だったから,おっ,来た来た!みたいな「インベーダーゲーム感覚」で楽しめるのだけど,弁護士に金払って訴訟代理させてたりすると,ツイが追加される毎に暗い気持ちになるんだろう。このお遊びは弁護士相手だけにしてもらいたいね。
\end{itemize}

 以前、モトケンこと矢部善朗弁護士(京都弁護士会)と小倉秀夫弁護士の長い対立関係のときにも感じていたのですが、いわゆる八百長試合のようなもので、弁護士の仕事を増やし、ゲームの課金感覚でお金を落としてくれる相手を求めているだけのように思えるのです。

 次の2つのツイートも高橋雄一郎弁護士のタイムラインで、高橋雄一郎弁護士のリツイートですが、しばらく前に見ていた考えさせられるツイートです。2つ目のツイートには弁護士ドットコムの引用ツイートがありますが、その引用ツイートのタイムラインをみるとずいぶん批判を受けていて潮目を感じました。

〉〉〉 kk\_hironoのリツイート 〉〉〉

\begin{itemize}
\tightlist
\item
  RT
  kk\_hirono(刑事告発・非常上告_金沢地方検察庁御中)|obpmb3fN93mQI9i(大阪名物パチパチ弁護士)
  日時:2021-06-15 16:40/2021/06/08 13:33 URL:
  \url{https://twitter.com/kk\_hirono/status/1404705352327725057} 
  \url{https://twitter.com/obpmb3fN93mQI9i/status/1402121651345530897} 
  \textgreater{}
  面会交流を巡る紛争や調整に、一時間あたり2万円の報酬が損益分岐点の平均と思える弁護士をフル活用できるはずあらへん。受任から解決まで何時間かかると思ってるんや。
  なんか別の代理人制度か資格を作ったほうがええと思う。
\end{itemize}

※ @kk\_hironoのアカウントがブロックされ,リツイートに失敗したツイート

\begin{itemize}
\tightlist
\item
  TW jmjhjmwtad(7286) 日時:2021/06/08 14:55:43 URL:
  \url{https://twitter.com/jmjhjmwtad/status/1402142275719766018} 
  \textgreater{}
  ドットコムはメディアの要件満たしてないと言われても仕方がない。メディア名乗りたいんであれば、最低限の中立性と、いわゆる一般人を揶揄しないのは最低限ちゃう。\\
  \textgreater{} 大丈夫かね、こんなタイトルの連載して。
  \url{https://t.co/OS3T087VoA} 
\end{itemize}

〉〉〉 kk\_hironoのリツイート 〉〉〉

\begin{itemize}
\tightlist
\item
  RT
  kk\_hirono(刑事告発・非常上告_金沢地方検察庁御中)|bengo4topics(弁護士ドットコムニュース)
  日時:2021-06-15 16:41/2021/06/08 10:01 URL:
  \url{https://twitter.com/kk\_hirono/status/1404705625120993280} 
  \url{https://twitter.com/bengo4topics/status/1402068130151600153} 
  \textgreater{}
  弁護士歴33年の大貫憲介弁護士による新コラム「モラ夫バスター弁護日誌」が始まります。
  大貫弁護士が目の当たりにした離婚事件を元に、モラハラ夫とは何か、その呪縛から逃れるためにはどうすればいいのか、連載で読み解いていきます。
  \url{https://t.co/NYFazoPriP} 
\end{itemize}

 時刻は17時05分です。行き当たりで見繕った品揃えという意味合いもあるのですが、これから取り上げる弁護士の問題と比較対象にしていただければと思います。

 次も行き当たりで、Googleの検索から始めて行きます。

\begin{itemize}
\item
  脳機能障害の少女 - Google 検索 \url{https://t.co/RTQKaPpZKu} 
\item
  脳機能障害の少女 弁護士 - Google 検索 \url{https://t.co/TnBC3y2eQ8} 
\item
  脳機能障害 少女 弁護士 - Google 検索 \url{https://t.co/gw9Y7tIxYk} 
\item
  脳機能障害 少女 家族 弁護士 - Google 検索 \url{https://t.co/ZVMSGoL3DM} 
\item
  大阪 少女 家族 弁護士 横領 - Google 検索 \url{https://t.co/qYev4qD3vb} 
\item
  産経ニュース \url{https://t.co/CCkVUxwJov}  404 Not Found ¥\n
  お探しのページは見つかりませんでした。 ¥\n
  ページが削除されたか移動した可能性があります。
\item
  大阪 少女 家族 弁護士 横領 - Google 検索 \url{https://t.co/qYev4qD3vb} 
  弁護士による多額着服・詐取事件の構図。大阪弁護士会(左上)所属の弁護士、\hspace{0pt}久保田昇被告による犯行は平成21年春から今春まで実に6年にも及んだ。大阪\hspace{0pt}地検特捜部が弁護・・・
\item
  少女 家族 久保田昇 - Google 検索 \url{https://t.co/oDaRWaIR4e} 
\item
  5億円ネコババ弁護士の〝裏の顔〟 脳障害少女の家族も食い物に はぎ取られた「弱者の味方」の仮面
  - 自動ニュース作成G \url{https://t.co/74OqAdrw5R}  2015-07-30 13:12:52
\item
  奉納\危険生物・弁護士脳汚染除去装置\金沢地方検察庁御中\_2020:
  「脳機能障害.+少女」を@hirono\_hideki @kk\_hirono @s\_hironoで検索 94件の該当 2021-06-06\_00:19の記録
  \url{https://t.co/5KrjXqFOWf} 
\item
  脳機能障害を負った少女の一家から着服 「示談不成立」とウソ -
  ライブドアニュース \url{https://t.co/wWWo2cRhIF}  ¥\n
  脳機能障害を負った少女の一家を標的にした、着服弁護士について報じている
  ¥\n 「示談不成立」と嘘をつき、5千万円以上の示談金を着服していたという
  ¥\n 母親は、弁護士への不信感から賠償請求に
\item
  脳機能障害を負った少女の一家から着服 「示談不成立」とウソ -
  ライブドアニュース \url{https://t.co/wWWo2cRhIF} 
  母親は、弁護士への不信感から賠償請求に踏み出せないことを明かした ¥\n
  提供社の都合により、削除されました。 ¥\n 概要のみ掲載しております。
\item
  久保田昇 判決 - Google 検索 \url{https://t.co/uO7amvDINI} 
\item
  5億円着服・詐欺で弁護士に懲役11年判決 大阪地裁 - 産経ニュース
  \url{https://t.co/5nwi5EVDBW} 
  村越一浩裁判長は「弁護士の立場を悪用して被害者を裏切り、信頼を失墜させた」として懲役11年(求刑懲役13年)を言い渡した。
\item
  実刑判決の久保田昇弁護士を除名に 大阪弁護士会 -- 弁護士自治を考える会
  \url{https://t.co/IRZN842M3L} 
  大阪弁護士会は白井弁護士と久保田弁護士の2名を今期除名しました。 ¥\n
  2名の大阪弁護士会の弁護士が刑務所にお入りになったということです ¥\n
  それでも大坂弁護士会としての責任はないと考え会長が辞任
\item
  白井裕之弁護士(大阪)懲戒処分の要旨 -- 弁護士自治を考える会
  \url{https://t.co/svnhbNcQmm} 
  判決文偽造で逮捕され起訴になり、弁護士会が懲戒処分をしなくても有罪になれば弁護士資格は喪失しますから弁護士会は処分を出さない場合があります。
\item
  白井裕之弁護士(大阪)懲戒処分の要旨 -- 弁護士自治を考える会
  \url{https://t.co/svnhbNcQmm} 
  不動産の売却代金として被告名義の口座に振り込まれた約2901万円のうち約2806万円を、33回にわたって引き出して着服したとされる。
\item
  白井裕之弁護士(大阪)懲戒処分の要旨 -- 弁護士自治を考える会
  \url{https://t.co/svnhbNcQmm} 
  29日付で、産経ニュースは「未提訴放置を隠蔽→判決偽造 元弁護士に懲役3年判決 大阪地裁、2800万円着服も認定」として、以下の記事を配信した。
\item
  白井裕之弁護士(大阪)懲戒処分の要旨 -- 弁護士自治を考える会
  \url{https://t.co/svnhbNcQmm} 
  事務所 大阪市北区西天満2     白井裕之法律事務所
\item
  ここまで地に落ちるとは! :弁護士 田沢剛 {[}マイベストプロ神奈川{]}
  \url{https://t.co/szSoFzpiNw} 
  一連の捜査で、交通事故で脳障害を負った10代の少女一家が受け取るはずだった示談金約5400万円まで着服する無慈悲ぶりも判明。
\item
  弁護士 田沢剛 {[}マイベストプロ神奈川{]} \url{https://t.co/szSoFzpiNw} 
  「ひとりでは重い荷物も二人三脚なら軽くすることができる」。事務所のホームページに柔和な笑顔で登場し、依頼を呼びかけていた久保田被告。しかし、実際に手に入れたかったのは、依頼人の苦しみの解決でなく、「カネ」だった。
\item
  自転車弁護士 : 鳶の羽 \url{https://t.co/hwdKl6UZcS}  ケイエヌ法律事務所久保田
  昇 弁護士 ¥\n 所属弁護士会    大阪弁護士会 ¥\n
  登録年度・登録番号 1983年/No.18509 ¥\n 事務所名      
  ケイエヌ法律事務所 ¥\n 所属弁護士数    4 名
\item
  自転車弁護士 : 鳶の羽 \url{https://t.co/hwdKl6UZcS}  ¥\n 弁護士情報
      渡邉 四郎(男性) ¥\n           久保田 昇(男性)
  ¥\n           西井 耕平(男性) ¥\n      
      久保田 惇(男性) ¥\n 所在地 大阪府大阪市中央区谷町9-2-14
  中田東海ビル4階
\end{itemize}

 新潟県燕市ではなく新潟市の建設会社でした。削除されたニュース記事もあり、脳機能障害の少女や家族では情報を得ることができない状態でした。さすがに久保田昇という名前を見つけると、ある程度の情報は収集することが出来ました。

 確認しておきたかった久保田昇弁護士の法律事務所は、大阪府大阪市中央区谷町9-2-14
中田東海ビル4階となっていましたが、4人の所属弁護士がいる法律事務所とは意外で、個人経営かワンマン経営の法律事務所でなければできない横領事件かという思い込みがありました。

\begin{itemize}
\tightlist
\item
  弁護士 田沢剛 {[}マイベストプロ神奈川{]} \url{https://t.co/szSoFzpiNw} 
  (1)新潟市の建設会社(2)大阪府岸和田市の建設会社(3)大阪市内で幼稚園を運営する学校法人(4)娘の交通事故の示談金請求業務を委任した女性。犯行時期は平成21年春から今春まで、実に6年間にも及んでいた。
\end{itemize}

 「犯行時期は平成21年春から今春まで、実に6年間にも及んでいた。」にある今春は2015年のことと思われます。「供託金など約3億5千万円が返還されず、同社は久保田被告を相手取り、25年に大阪地裁に提訴したのだ。」が端緒のようですが、これがなければ娘の交通事故の示談金は未発覚だったかも。

 知っていてGoogleで検索しても見つけづらい情報ですが、これは岡山の事件、福岡の事件も同じです。福岡では弁護士会の監督責任が問われていましたが、問題なしとされていました。久保田昇弁護士の件も含め、いずれも新聞の全国ニュースやテレビの全国ニュースにもなっていなかったと思われます。

\begin{itemize}
\item
  岡山 弁護士 横領 - Google 検索 \url{https://t.co/1Jl24nBzdc} 
\item
  岡山弁護士巨額横領事件 - Wikipedia \url{https://t.co/I9WHVuAwGY} 
  その後の調べで、2006年から2012年まで交通事故・医療過誤・遺産相続等の民事訴訟に伴う賠償金や刑事訴訟の保釈金について22件計9億円の横領をしていたことが判明し{[}2{]}、業務上横領罪で追起訴された{[}3{]}{[}4{]}。
\item
  岡山弁護士巨額横領事件 - Wikipedia \url{https://t.co/I9WHVuAwGY} 
  被害者の多くは交通事故の遺族や重度の後遺障害を負った人々。その状況を知りながら生活に不可欠な賠償金、保険金を横領し、さらなる窮地に追い込んだ」として被告人に懲役14年(求刑懲役15年)が言い渡された{[}2{]}{[}5{]}{[}6{]}。
\item
  福岡 弁護士 巨額詐欺 - Google 検索 \url{https://t.co/w2FJhWJX24} 
\item
  高橋浩文元弁護士(福岡)懲役14年の判決・巨額詐欺事件 --
  弁護士自治を考える会 \url{https://t.co/SURWUKitmk} 
  詐欺と業務上横領の罪に問われた元弁護士高橋浩文被告(51)の判決が11日、福岡地裁であり、野島秀夫裁判官は懲役14年(求刑懲役15年)を言い渡した。
\end{itemize}

 しっかりしたキーワードの組み合わせがないとネットで出てこない情報となっているようです。出てくる情報の数も年々減っているのかもしれません。検索結果に出てきて存在しないという産経新聞の記事もありましたが、そのうち傑作結果自体が消えていくのかもしれません。弁護士鉄道の幻影のようです。

 そういえば、昼にどんたく宇出津店に買い物に行ったとき、トイレから出てくると、通路にある高田屋の服に、大きく「塩対応」とプリントされてものがあったことを思い出しました。警察の塩対応というのも弁護士の暗躍が大きく影響していそうです。それも弁護士鉄道の幻影なのかもしれません。

 締めくくりに「消費者被害」をキーワードにした深澤諭史弁護士のツイートをご紹介したいと思います。個人的な感想ですが、決まってテレビアニメ「マジンガーZ」のテーマ曲が頭に浮かんできて、歌詞が「スーパー弁護士ふかざわFZゼット!!!」に置き換わります。

\begin{itemize}
\item
  マジンガーZ OP - YouTube \url{https://t.co/xbBk74SW8z} 
\item
  2021年06月06日11時44分の登録:
  REGEXP:''消費者被害''/深澤諭史(@fukazawas)の検索(2013-07-17〜2021-03-13/2021年06月06日11時43分の記録66件)
  \url{https://kk2020-09.blogspot.com/2021/06/regexpfukazawas2013-07-172021-03.html} 
\item
  (01/66) TW fukazawas(深澤諭史) 日時:2013-07-17 13:08:00 +0900
  URL:
  \url{https://twitter.com/fukazawas/status/357350987356319745\textgreater} {}
  いい大人に2年ないし3年の時間と数百万の費用を費やさせておいて,5年で消えるとか,もはや消費者被害レベル/『ローの弁解「きょきょ教育効果はごっご5年で消える」(笑)』
  \url{http://t.co/U9mKHo6QHs} 
  target="\_blank"\textgreater \url{http://t.co/U9mKHo6QHs} 
\item
  (05/66) TW fukazawas(深澤諭史) 日時:2014-11-14 15:56:00 +0900
  URL:
  \url{https://twitter.com/fukazawas/status/533151373349507072\textgreater} {}
  非弁は業際問題ではなく消費者被害。\textgreater\textgreater{}
  \#エアリプ
\item
  (06/66) TW fukazawas(深澤諭史) 日時:2015-02-08 11:18:00 +0900
  URL:
  \url{https://twitter.com/fukazawas/status/564246906482597888\textgreater} {}
  7.弁護士による弁護士自治の「放棄」\textgreater{}
  8.民事においては,消費者被害に取り組む弁護士は全滅。\textgreater{}
  9.刑事においては,捜査当局と闘う弁護士は全滅。\textgreater{}
  10.19世紀型資本主義と,中世ならぬ「中生代」な刑事司法の実現へ。
\item
  (11/66) TW fukazawas(深澤諭史) 日時:2016-10-02 15:36:00 +0900
  URL:
  \url{https://twitter.com/fukazawas/status/782469189759971328\textgreater} {}
  >RT\textgreater{}
  非弁対策は、消費者被害対策でもあり、人権活動でもあると思います。\textgreater{}
  (・∀・)
\item
  (20/66) TW fukazawas(深澤諭史) 日時:2017-06-22 10:31:00 +0900
  URL:
  \url{https://twitter.com/fukazawas/status/877700611398959104\textgreater} {}
  しっかし,最近,弁護士法違反の摘発に警察がかなり積極的だなぁ。\textgreater{}
  よいことではある。\textgreater{}
  非弁は,弁護士の職域という観点からばかり捉えがちだが,実際の案件に触れると,むしろ「消費者被害」の側面が非常に大きい。被害者は弁護士ではなくて依頼者。\textgreater{}
  (#・∀・)
\item
  (34/66) TW fukazawas(深澤諭史) 日時:2018-03-05 10:21:00 +0900
  URL:
  \url{https://twitter.com/fukazawas/status/970469256071491585\textgreater} {}
  非弁って結局,消費者被害なんですよね。
  これは,弁護士自身についても留意が必要なんですが,法律事務サービスって,類型的に「悪徳」をやりやすいという問題が・・・。
\item
  (57/66) TW fukazawas(深澤諭史) 日時:2019-07-17 12:54:00 +0900
  URL:
  \url{https://twitter.com/fukazawas/status/1151339393283547141\textgreater} {}
  ほんこれ。非弁護士取締委員会ずーーっとやっているが,その本質は消費者被害であることを,ただただひたすら痛感する(・∀・;;;;)
  \url{https://t.co/kHoPHE2ubI} 
\item
  (63/66) TW fukazawas(深澤諭史) 日時:2020-03-09 00:01:00 +0900
  URL:
  \url{https://twitter.com/fukazawas/status/1236668271463546881\textgreater} {}
  (・∀・)非弁問題については,弁護士の職域の問題ではなくて,消費者被害,健全な法の支配の確立という観点から論じるべきであると思っています。\textgreater{}
  (^ω^)そのための「具体的な」方策を策定して,実行するべきであります。\textgreater{}
  \#日弁連会長選挙 \url{https://t.co/gR9sX61KrZ} 
\item
  (64/66) TW fukazawas(深澤諭史) 日時:2020-04-18 11:50:00 +0900
  URL:
  \url{https://twitter.com/fukazawas/status/1251342357670191104\textgreater} {}
  平成の司法改革では、結局、市民にとって使いやすい、頼りがいのある司法は一切実現できず、消費者被害が増え、非弁業者が弁護士の信用を換金しただけに終わった。\textgreater{}
  現場も知らずに妄想だけたくましくした学者先生に関わらせたのがそもそもの間違い・・・
  \url{https://t.co/pJTVEUXMUu} 
\item
  (66/66) TW fukazawas(深澤諭史) 日時:2021-03-13 16:56:31 +0900
  URL:
  \url{https://twitter.com/fukazawas/status/1370644930842947587\textgreater} {}
  「非弁は消費者被害。非弁提携はその共犯。」\textgreater{}
  この命題は、常に全力で補強されていきますね。。。\textgreater{}
  (・∀・;)
\end{itemize}

 なお、深澤諭史弁護士をZと思う所以ですが、Zは最後のアルファベット文字すなわち「最終進化系の弁護士」あるいは「弁護士の最終進化系」の姿が深澤諭史弁護士に感じられます。突き抜けた進化は滅亡の印とも考えます。

\begin{itemize}
\tightlist
\item
  〈〈〈 2021/06/15 18:30:03 Linux Emacs: 〈〈〈
\end{itemize}

\hypertarget{ux540cux30586ux670813ux65e5ux306eux9577ux5973ux3068ux5b6bux304cux4ea1ux304fux306aux3063ux305fux3068ux3044ux3046ux548cux6b4cux5c71ux30abux30ecux30fcux4e8bux4ef6ux306eux5831ux9053ux3068ux5831ux916cux76eeux7684ux306eux6e96ux6297ux544aux304cux4e71ux767aux3055ux308cux308bux6050ux308cux3082ux61f8ux5ff5ux3055ux308cux308bux3068ux3044ux3046ux7523ux7d4cux65b0ux805eux306eux8a18ux4e8bux5f01ux8b77ux58ebux3089ux306eux53cdux5fdcux306eux5927ux304dux3055ux306eux9055ux3044}{%
\paragraph{同じ6月13日の「長女と孫が亡くなった」という和歌山カレー事件の報道と、「報酬目的の準抗告が乱発される恐れも懸念される。」という産経新聞の記事、弁護士らの反応の大きさの違い}\label{ux540cux30586ux670813ux65e5ux306eux9577ux5973ux3068ux5b6bux304cux4ea1ux304fux306aux3063ux305fux3068ux3044ux3046ux548cux6b4cux5c71ux30abux30ecux30fcux4e8bux4ef6ux306eux5831ux9053ux3068ux5831ux916cux76eeux7684ux306eux6e96ux6297ux544aux304cux4e71ux767aux3055ux308cux308bux6050ux308cux3082ux61f8ux5ff5ux3055ux308cux308bux3068ux3044ux3046ux7523ux7d4cux65b0ux805eux306eux8a18ux4e8bux5f01ux8b77ux58ebux3089ux306eux53cdux5fdcux306eux5927ux304dux3055ux306eux9055ux3044}}

\begin{itemize}
\tightlist
\item
  〉〉〉 Linux Emacs: 2021/06/16 11:34:46 〉〉〉
\end{itemize}

:CATEGORIES: @kanazawabengosi \#金沢弁護士会 @JFBAsns
日本弁護士連合会(日弁連) \#法務省 @MOJ\_HOUMU \#準抗告 \#刑事弁護
\#再審請求 \#冤罪

 まず、産経新聞の記事です。確認したところ配信時刻が2021年6月13日12時00分となっていました。

\begin{quote}
《引用の始まり》
\end{quote}

\begin{quote}
警察などの捜査機関に逮捕された容疑者の身体を拘束する裁判所の「勾留」処分に不服を申し立てた国選弁護人に対し、大阪弁護士会は4月から、1件当たり最大4万円の報酬を支払う制度の運用を始めた。弁護士会がこうした申し立てに報酬を支払うことは全国的にも珍しい。容疑者が罪を認めない否認事件の勾留期間が長引く傾向にある中、いわゆる「人質司法」のあり方に一石を投じる一方で、報酬目的の準抗告が乱発される恐れも懸念される。

4万円の成功報酬検察官は刑事訴訟法上の規定に従い、逮捕、送検された容疑者の身柄を拘束し続けるため、裁判所に勾留を請求することができる。

罪を犯したとされる相当の理由があることや、罪証隠滅や逃亡の恐れがあることが要件。延長が認められれば、送検後の勾留決定から最大20日間にわたり、強制的な取り調べが可能だ。

同法は、捜査機関が犯罪を調べる際、逮捕や勾留といった強制捜査ではなく、原則相手の同意に基づく任意捜査を求めている。容疑者の弁護人は、逮捕後の勾留が不当であれば、釈放を求める準抗告ができる。

大阪弁護士会は、容疑者の経済的事情などを背景に国が選んだ国選弁護人がこうした準抗告を申し立てた場合に1万円、それが認められるとさらに3万円の報酬を支払う制度を4月に開始した。

同会は会員が納める会費を積み立て、1500万円の予算を確保。申請は1人5回まで可能で、起訴後の保釈請求については報酬の対象にならない。
\end{quote}

\begin{quote}
《引用の終わり》
\end{quote}

\begin{itemize}
\tightlist
\item
  「勾留」不服申し立てで弁護士に報酬への懸念(1/2ページ) - 産経ニュース
  \url{https://www.sankei.com/article/20210613-ATTWEMUVSJL2ZPPADVH44JZNNI/} 
\end{itemize}

\begin{quote}
《引用の始まり》
\end{quote}

\begin{quote}
こうした現状について、大阪弁護士会刑事弁護委員会副委員長の森直也弁護士は、準抗告に取り組む弁護人側の意識の低さを指摘。「事件によっては弁護人側でさえ、『勾留されてもいいのでは』と考える人もいる」と話し、準抗告に対する意識の改善を図るためにも、今回の制度創設に至ったと明かす。

愛知県弁護士会が一昨年に、試験的に同様の報酬制度を設けた例はあったが、本格的に始めたのは大阪弁護士会が初という。
\end{quote}

\begin{quote}
《引用の終わり》
\end{quote}

\begin{itemize}
\tightlist
\item
  「勾留」不服申し立てで弁護士に報酬への懸念(2/2ページ) - 産経ニュース
  \url{https://www.sankei.com/article/20210613-ATTWEMUVSJL2ZPPADVH44JZNNI/2/} 
\end{itemize}

\begin{quote}
《引用の始まり》
\end{quote}

\begin{quote}
検察内部も冷ややかな視線を向けている。ある検察幹部は「必要に応じて容疑者の勾留を請求している。準抗告が増えたところで、こちらは従来通りの対応を続ける」と話す。

準抗告が乱発されれば、それに伴って釈放の増加も予想される。ただ別の検察幹部は、裁判所が報酬目的の準抗告と判断し、逆効果になる可能性もあると指摘。その上で、「弁護士会側のある種のキャンペーンに、裁判所がなびくことはあってはならない」と力説した。(森西勇太)
\end{quote}

\begin{quote}
《引用の終わり》
\end{quote}

\begin{itemize}
\tightlist
\item
  「勾留」不服申し立てで弁護士に報酬への懸念(2/2ページ) - 産経ニュース
  \url{https://www.sankei.com/article/20210613-ATTWEMUVSJL2ZPPADVH44JZNNI/2/} 
\end{itemize}

 改めて記事を読み返したのですが、時期を特定できる情報は「大阪弁護士会は4月から、1件当たり最大4万円の報酬を支払う制度の運用を始めた。」ぐらいで、6月から制度の運用を初めたとありますが、6月13日の配信記事あるいは配信ニュースとなっています。

 次に最初に見かけた和歌山カレー事件関連のニュースです。日付は同じ6月13日に間違いないと思いますが、時刻を確認していないか私が忘れています。

〉〉〉 kk\_hironoのリツイート 〉〉〉

\begin{itemize}
\item
  RT
  kk\_hirono(刑事告発・非常上告_金沢地方検察庁御中)|take\_\textbf{five(中村剛(take-five))
  日時:2021-06-16 11:45/2021/06/15 22:15 URL:
  \url{https://twitter.com/kk\_hirono/status/1404993525175906307} 
  \url{https://twitter.com/take} }\_five/status/1404789632009523205\\
  \textgreater{}
  性犯罪で処罰の間隙をなくすためには、年齢を問わずあらゆる性行為を犯罪化し、後は警察に任せるということだろう。これなら間隙はなくなる。しかし、とんでもないディストピアになる。そのため、正当な性行為と不当な性行為の線引をしないといけないが、何が正当かは人によって異なるので難しい。
\item
  【独白】「長女と孫が亡くなった」和歌山カレー事件の林健治さん 16歳孫が変死後、長女が自殺か〈dot.〉(AERA
  dot.) - Yahoo!ニュース \url{https://t.co/SqXvod4N2k}  ¥\n 6/13(日)
  10:13配信
\end{itemize}

 ブラウザの拡張機能を使ったTwitter投稿の直後に、ホームの最新ツイートとして中村剛弁護士のツイートが出てきました。確認しないまま小さいウィンドウを消してしまいましたが、リツイートだと思います。

 性犯罪の処罰範囲の問題でディストピアという言葉が出てきましたが、長年に渡る再審請求の問題の一つの結末が、まさにディストピアと言えるような6月13日のアエラの記事でした。ユートピアの反対語がディストピアだったと思います。これまでも弁護士のツイートとして見かけてきました。

〉〉〉 kk\_hironoのリツイート 〉〉〉

\begin{itemize}
\tightlist
\item
  RT
  kk\_hirono(刑事告発・非常上告_金沢地方検察庁御中)|matimura(田丁木寸)
  日時:2021-06-16 18:19/2021/06/16 17:48 URL:
  \url{https://twitter.com/kk\_hirono/status/1405092720905916416} 
  \url{https://twitter.com/matimura/status/1405084895165095936} 
  \textgreater{} 将来に禍根を残す裁判官訴追。
  これまでの訴追請求・弾劾事例と比べて、著しい不均衡があるだけでなく、裁判官の市民的自由という点は勿論、法的紛争に関するSNS上での議論制約という点でも、さらには裁判例の公開への制約という点でも悪しき先例を残す決定だ。
  \url{https://t.co/oIw2zIkpXT} 
\end{itemize}

〉〉〉 kk\_hironoのリツイート 〉〉〉

\begin{itemize}
\tightlist
\item
  RT
  kk\_hirono(刑事告発・非常上告_金沢地方検察庁御中)|asahi\_tokyo(朝日新聞コブク郎)
  日時:2021-06-16 18:19/2021/06/16 17:37 URL:
  \url{https://twitter.com/kk\_hirono/status/1405092750731579394} 
  \url{https://twitter.com/asahi\_tokyo/status/1405082070401970177} 
  \textgreater{} 仙台高裁判事の罷免を審理へ SNSで不適切投稿
  \url{https://t.co/WALYDFzVEP} 
  仙台高裁の岡口基一判事は、SNS上で不適切な投稿をしたとして、最高裁から2度の戒告処分を受けていました。
  国会の裁判官訴追委員会は16日、裁判官弾劾裁判所に罷免(免職)を求めて訴追すると決めました。
\end{itemize}

 時刻は6月19日11時58分です。2,3日中断をしていました。

 リツイートしていたmatimura(田丁木寸) 日時:2021-06-16
18:19/2021/06/16
17:48が中断のきっかけでしたが、このツイートで初めて岡口基一裁判官の弾劾裁判所への訴追決定のニュースを知ったのでした。それも名誉毀損に関する資料のまとめの作成を終えた直後の発見でした。

 和歌山に関しては次のような経緯でまとめ記事を作成しています。

\begin{itemize}
\item
  2021年04月30日15時28分の登録: \弁護士
  亀石倫子 @MichikoKameishi\和歌山カレー事件の犯人視報道によって林真須美さんが犯人だと信じて疑わなかった過去の自分を紀州のドンファン事件報道を見
  \url{https://kk2020-09.blogspot.com/2021/04/michikokameishi\_30.html} 
\item
  2021年05月01日09時37分の登録:
  \奥村徹弁護士 @okumuraosaka\警察を持ち上げる  ドン・ファン元妻逮捕の全舞台裏 警察庁が和歌山に送り込んだ「切り札官僚」(東スポWeb)\#Yahooニュー
  \url{https://kk2020-09.blogspot.com/2021/05/okumuraosakawebyahoo.html} 
\item
  2021年05月08日18時33分の登録:
  \くまえもん @cure\_kumaemon\紀州といえば、和歌山毒物カレー事件も後味の悪い裁判だったな。弁護団は再審目指して頑張っているようだけど。
  \url{https://kk2020-09.blogspot.com/2021/05/curekumaemon\_90.html} 
\item
  2021年06月05日19時55分の登録:
  \りっぴぃ @rippy08\かつてB肝弁護団で署名集めしたときとか,和歌山の先生方にどれだけ世話になったよ
  \url{https://kk2020-09.blogspot.com/2021/06/rippy08b.html} 
\item
  2021年06月13日18時09分の登録:
  REGEXP:''和歌山.*長女''/データベース登録済みツイートの検索:2021-06-10〜2021-06-13/2021年06月13日18時08分の記録:ユーザ・投稿:10/18件
  \url{https://kk2020-09.blogspot.com/2021/06/regexp2021-06-102021-06.html} 
\item
  2021年06月14日06時16分の登録:
  REGEXP:''和歌山.*事件''/データベース登録済みツイートの検索:2009-11-05〜2021-06-14/2021年06月14日06時13分の記録:ユーザ・投稿:65/152件
  \url{https://kk2020-09.blogspot.com/2021/06/regexp2009-11-052021-06.html} 
\item
  2021年06月14日06時19分の登録:
  Ex-REGEXP:''和歌山.*事件''/データベース登録済みツイートの検索:2009-11-05〜2021-06-14/2021年06月14日06時17分の記録:ユーザ・投稿:62/92件
  \url{https://kk2020-09.blogspot.com/2021/06/ex-regexp2009-11-052021-06.html} 
\item
  2021年06月14日06時20分の登録:
  Ex-REGEXP:''和歌山.*事件''/データベース登録済みツイートの検索:2021-06-10〜2021-06-14/2021年06月14日06時20分の記録:ユーザ・投稿:19/30件
  \url{https://kk2020-09.blogspot.com/2021/06/ex-regexp2021-06-102021-06.html} 
\item
  2021年06月14日06時21分の登録:
  Ex-REGEXP:''和歌山.*事件''/データベース登録済みツイートの検索:2021-06-13〜2021-06-14/2021年06月14日06時21分の記録:ユーザ・投稿:14/19件
  \url{https://kk2020-09.blogspot.com/2021/06/ex-regexp2021-06-132021-06.html} 
\item
  2021年06月14日15時53分の登録:
  REGEXP:''和歌山.*長女''/データベース登録済みツイートの検索:2021-06-13〜2021-06-14/2021年06月14日15時52分の記録:ユーザ・投稿:12/23件
  \url{https://kk2020-09.blogspot.com/2021/06/regexp2021-06-132021-06\_14.html} 
\item
  2021年06月16日00時02分の登録:
  @un\_co\_the2nd(うの字を名乗る?物)のツイート ''(和歌山|再審|長女)'' 2/3218:2021-05-03\_1133〜2021-06-15\_2006 2021年06月16日00時02分の記録
  \url{https://kk2020-09.blogspot.com/2021/06/uncothe2nd232182021-05-0311332021-06.html} 
\item
  2021年06月16日00時03分の登録:
  @uwaaaa(サイ太)のツイート ''(和歌山|再審|長女)'' 5/3218:2020-11-24\_1316〜2021-06-15\_2010 2021年06月16日00時03分の記録
  \url{https://kk2020-09.blogspot.com/2021/06/uwaaaa532182020-11-2413162021-06.html} 
\item
  2021年06月16日00時04分の登録:
  @motoken\_tw(モトケン)のツイート ''(和歌山|再審|長女)'' 0/3249:2021-03-14\_1414〜2021-06-15\_2233 2021年06月16日00時04分の記録
  \url{https://kk2020-09.blogspot.com/2021/06/motokentw032492021-03-1414142021-06.html} 
\item
  2021年06月16日00時04分の登録:
  @lawkus(ystk)のツイート ''(和歌山|再審|長女)'' 6/3228:2021-02-24\_1731〜2021-06-15\_0655 2021年06月16日00時04分の記録
  \url{https://kk2020-09.blogspot.com/2021/06/lawkusystk632282021-02-2417312021-06.html} 
\item
  2021年06月16日00時06分の登録:
  @chosakukenho(小倉秀夫)のツイート ''(和歌山|再審|長女)'' 1/3246:2021-03-24\_0031〜2021-06-15\_2256 2021年06月16日00時06分の記録
  \url{https://kk2020-09.blogspot.com/2021/06/chosakukenho132462021-03-2400312021-06.html} 
\item
  2021年06月16日00時14分の登録:
  @amneris84(Shoko Egawa)のツイート ''(和歌山|再審|長女)'' 17/3245:2020-11-23\_2318〜2021-06-15\_1353 2021年06月16日00時14分の記録
  \url{https://kk2020-09.blogspot.com/2021/06/amneris84shokoegawa1732452020-11.html} 
\item
  2021年06月16日00時30分の登録: REGEXP:''和歌山.*カレー''/Shoko
  Egawa(@amneris84)の検索(2012-08-16〜2013-06-13/2021年06月16日00時30分の記録5件)
  \url{https://kk2020-09.blogspot.com/2021/06/regexpshoko-egawaamneris842012-08.html} 
\item
  2021年06月16日00時31分の登録:
  REGEXP:''和歌山.*カレー''/モトケン(@motoken\_tw)の検索(2010-12-10〜2012-08-16/2021年06月16日00時31分の記録2件)
  \url{https://kk2020-09.blogspot.com/2021/06/regexpmotokentw2010-12-102012-08.html} 
\item
  2021年06月16日10時26分の登録:
  Ex-REGEXP:''和歌山.*事件''/データベース登録済みツイートの検索:2021-06-14〜2021-06-15/2021年06月16日10時26分の記録:ユーザ・投稿:4/10件
  \url{https://kk2020-09.blogspot.com/2021/06/ex-regexp2021-06-142021-06.html} 
\item
  2021年06月16日10時41分の登録:
  REGEXP:''和歌山.*長女''/データベース登録済みツイートの検索:2021-06-14〜2021-06-16/2021年06月16日10時40分の記録:ユーザ・投稿:5/23件
  \url{https://kk2020-09.blogspot.com/2021/06/regexp2021-06-142021-06\_16.html} 
\item
  2021年06月17日14時59分の登録:
  REGEXP:''和歌山.*長女''/データベース登録済みツイートの検索:2021-06-16〜2021-06-17/2021年06月17日14時58分の記録:ユーザ・投稿:4/12件
  \url{https://kk2020-09.blogspot.com/2021/06/regexp2021-06-162021-06\_17.html} 
\item
  2021年06月17日16時28分の登録:
  REGEXP:''和歌山.*事件''/データベース登録済みツイートの検索:2021-06-15〜2021-06-17/2021年06月17日16時28分の記録:ユーザ・投稿:9/34件
  \url{https://kk2020-09.blogspot.com/2021/06/regexp2021-06-152021-06.html} 
\item
  2021年06月17日16時30分の登録:
  Ex-REGEXP:''和歌山.*事件''/データベース登録済みツイートの検索:2021-06-15〜2021-06-17/2021年06月17日16時30分の記録:ユーザ・投稿:7/13件
  \url{https://kk2020-09.blogspot.com/2021/06/ex-regexp2021-06-152021-06.html} 
\item
  2021年06月18日21時38分の登録:
  REGEXP:''和歌山カレー事件.*週刊文春''/データベース登録済みツイートの検索:2021-06-17〜2021-06-17/2021年06月18日21時38分の記録:ユーザ・投稿:4/4件
  \url{https://kk2020-09.blogspot.com/2021/06/regexp2021-06-172021-06-1720210618213844.html} 
\item
  2021年06月19日11時17分の登録:
  REGEXP:''和歌山.*(事件|長女)''/データベース登録済みツイートの検索:2021-06-17〜2021-06-19/2021年06月19日11時17分の記録:ユーザ・投稿:7/28件
  \url{https://kk2020-09.blogspot.com/2021/06/regexp2021-06-172021-06\_19.html} 
\item
  2021年06月19日11時19分の登録:
  REGEXP:''和歌山.*(事件|長女)''/データベース登録済みツイートの検索:2021-06-10〜2021-06-19/2021年06月19日11時18分の記録:ユーザ・投稿:31/171件
  \url{https://kk2020-09.blogspot.com/2021/06/regexp2021-06-102021-06\_19.html} 
\item
  2021年06月19日11時21分の登録:
  Ex-REGEXP:''和歌山.*(事件|長女)''/データベース登録済みツイートの検索:2021-06-10〜2021-06-19/2021年06月19日11時19分の記録:ユーザ・投稿:29/58件
  \url{https://kk2020-09.blogspot.com/2021/06/ex-regexp2021-06-102021-06\_19.html} 
\item
  (01/58) TW @NEWS\_JAPAN\_S(NEWS JAPAN) 日時: 2021-06-10 05:00:30
  +0900 URL:
  \url{https://twitter.com/NEWS\_JAPAN\_S/status/1402717257868922880\textgreater} {}
  【和歌山】毒物カレー事件で再審請求
  「第三者の犯行」と主張\textgreater{}
  1998年に和歌山市で4人が死亡した毒物カレー事件で、殺人罪などで2009年に死刑が確定した林真須美死刑囚(59)の弁護人を務める生田暉雄弁護士は9日、和歌山地裁に新たに・・・
  \url{https://t.co/Tz2JfUFHAW} 
\end{itemize}

 6月9日のツイートとしては記録されていなかったようです。その6月9日に新たな再審請求の第一報があったという話が多いのですが、午前中なのか午後だったのか確認をしておきたいと思っていました。

\begin{itemize}
\tightlist
\item
  (04/58) TW @NEWS\_JAPAN\_S(NEWS JAPAN) 日時: 2021-06-10 15:01:04
  +0900 URL:
  \url{https://twitter.com/NEWS\_JAPAN\_S/status/1402868398200168452\textgreater} {}
  娘が意識も呼吸もなく黒いものを吐いている、と和歌山で通報の母(30代)と次女が関空の橋から飛び降り死亡\textgreater{}
  関空橋の転落死 30代母親と次女か 和歌山で死亡の長女?にあざ9日午後4時ごろ、大阪府泉佐野市の関西国際空港連絡橋を車で走行し・・・
  \url{https://t.co/54QVdXYmPO} 
\end{itemize}

 同じ@NEWS\_JAPAN\_S(NEWS
JAPAN)というニュースサイトのようなアカウントですが、2021-06-10
15:01:04が少なくとも私の観測では、第一報となる長女と娘の三女の関空連絡橋からの飛び降り自殺、そして和歌山市内の自宅での自殺した長女の16歳長女の不審死に関連するツイートになるようです。

\begin{itemize}
\tightlist
\item
  (11/58) TW @sonoda\_hisashi(園田寿) 日時: 2021-06-10 22:35:58
  +0900 URL:
  \url{https://twitter.com/sonoda\_hisashi/status/1402982876950368257\textgreater} {}
  おおー!\textgreater{}
  和歌山毒物カレー事件の林真須美死刑囚が再審請求、受理される(読売新聞オンライン)\textgreater{}
  \#Yahooニュース\textgreater{} \url{https://t.co/bDC4i5QNLa} 
\end{itemize}

 この園田寿弁護士のツイートも気になっていたのですが、園田寿弁護士は刑事裁判に造詣が深い上、関西の大学の教授でもあったと思います。単純な考えですが、和歌山カレー事件に関する情報も多く、知識や経験もあったと考えられますが、遅れたタイミングでのツイートとは感じました。

\begin{itemize}
\tightlist
\item
  園田寿 - Wikipedia \url{https://t.co/EAuc9jXkP6}  園田
  寿(そのだ・ひさし、1952年-
  )は、日本の法学者、甲南大学名誉教授、弁護士。元関西大学教授、元甲南大学教授。専門は刑法。
\end{itemize}

\begin{quote}
《引用の始まり》
\end{quote}

\begin{quote}
教授方針[編集]``甲南の刑法の特徴は、現実に根ざした理論の構築と展開を目指していることです。過去から積み上げられてきた犯罪に対処するためのルールが、最先端の犯罪でどこまで通用するのか。人間の哲学観、世界観などに裏打ちされる面と、二者択一の論理を追うシビアな部分を併せ持つ刑法の世界で、そんな新たな問題を見つめる視点・感覚を教授していきます。''
(甲南大学法科大学院教員紹介HPより)
\end{quote}

\begin{quote}
《引用の終わり》
\end{quote}

\begin{itemize}
\tightlist
\item
  園田寿 - Wikipedia
  \url{https://ja.wikipedia.org/wiki/\%E5\%9C\%92\%E7\%94\%B0\%E5\%AF\%BF} 
\end{itemize}

 上記に甲南大学の教授方針という部分を引用しました。教育方針というのは昔から見かけますが、教授方針という言葉は初めて目にしたと思います。私が和歌山カレー事件の沿革や社会的影響に着目するのと似たようなものを感じました。私の場合はマスコミの報道と黙秘権の行使に重点を置いています。

\begin{itemize}
\tightlist
\item
  (12/58) TW @okumuraosaka(okumuraosaka) 日時: 2021-06-11 07:36:32
  +0900 URL:
  \url{https://twitter.com/okumuraosaka/status/1403118914939199492\textgreater} {}
  再審請求しており、同地裁、大阪高裁で棄却され、最高裁に特別抗告中。今回の再審請求は別の弁護人が行っており、和歌山毒物カレー事件の林真須美死刑囚、再審請求し受理される
  : 社会 : ニュース : 読売新聞オンライン \url{https://t.co/9I5DCGOzzD} 
\end{itemize}

 読売新聞の記事なので同じだと思いますが、他の記事を読んだ後、読売新聞の記事を読んで初めてなぜ再審請求の受理となっているのか理解が出来ました。中には再審請求が通ったと勘違いするツイートも散見されましたが、従前の最高裁に継続中の再審請求とは別に再審請求が受理されたということでした。

 再審請求が最高裁ということは特別抗告中のはずですが、それが集結する前に新たな再審請求を起こしたわけで、おそらくは前代未聞と思われますが、これを重視した弁護士のツイートというのも全く見かけていません。新たな弁護人は別の弁護士で、それも懲戒処分9回の記録を持つ生田暉雄弁護士です。

 この和歌山カレー事件の再審請求は、大崎事件のように第3次再審請求というような表現を見たことがないので、最初の再審請求がまだ終結していないとも考えられるのですが、10年ほど前に安田好弘弁護士が再審請求を起こしたということで話題となっていました。その後、立ち消えのように再審請求に関する主張や情報は見かけなくなっていました。

\begin{quote}
《引用の始まり》
\end{quote}

\begin{quote}
死刑囚・林は2020年9月27日時点で[17]死刑囚として大阪拘置所に収監されている[18]一方、2009年7月22日付で和歌山地裁に再審を請求した[47][48]。林とその弁護団は「無罪を言い渡すべき新たな証拠」として「祭り会場に残された紙コップのヒ素が自宅から発見されたものとは異なる。京都大学の研究者の鑑定からも『事件当時のヒ素の鑑定方法は問題がある』ことは明らかだ」と主張した[49]。

2014年(平成26年)3月、林は支援者の釜ヶ崎地域合同労働組合委員長・北大阪合同労働組合執行委員長・稲垣浩と養子縁組している[50]。この養子縁組は、本人との定期的な面会を行うためと見られる[要出典]。

2009年7月の再審請求は和歌山地裁(浅見健次郎裁判長)の2017年(平成29年)3月29日付決定により棄却され[49]、これを不服とした林は2017年4月3日までに大阪高裁に即時抗告した[51][52]。しかし大阪高裁(樋口裕晃裁判長)は2020年(令和2年)3月24日付で死刑囚・林の即時抗告を棄却する決定を出した[53]ため、林はこれを不服として同年4月8日付で最高裁に特別抗告を行った[54]。一方、特別抗告中の2021年5月には「事件は第三者による犯行」として和歌山地裁に第2次再審請求を行い、同月31日付で受理された[注
9][20]。
\end{quote}

\begin{quote}
《引用の終わり》
\end{quote}

\begin{itemize}
\tightlist
\item
  和歌山毒物カレー事件 - Wikipedia
  \url{https://ja.wikipedia.org/wiki/\%E5\%92\%8C\%E6\%AD\%8C\%E5\%B1\%B1\%E6\%AF\%92\%E7\%89\%A9\%E3\%82\%AB\%E3\%83\%AC\%E3\%83\%BC\%E4\%BA\%8B\%E4\%BB\%B6\#\%E5\%86\%8D\%E5\%AF\%A9\%E8\%AB\%8B\%E6\%B1\%82} 
\end{itemize}

 先日一通り読んだWikipediaで確認しましたが、再審請求は2009年7月22日付で、最高裁への特別抗告が昨年の2020年4月8日付とあります。同じ理由で再審請求が出来ないというのも再審請求ならではの特徴ですが、今回の新たな再審請求は「事件は第三者による犯行」を理由としているようです。

 23年経ってから「事件は第三者による犯行」を証明できるのか疑問ですが、この和歌山カレー事件のえん罪の可能性というのはかなり多くの人が信じ、納得しているということが最近になってよくわかってきました。それまでも情報は見かけていましたが、6月9日以降に調べたこと、知ったことが多くあります。

\begin{itemize}
\item
  (15/58) TW @okumuraosaka(okumuraosaka) 日時: 2021-06-13 11:25:31
  +0900 URL:
  \url{https://twitter.com/okumuraosaka/status/1403901313981714436\textgreater} {}
  【独自】「長女と孫が亡くなった」和歌山カレー事件の 16歳孫が変死後、長女が自殺か
  \url{https://t.co/Py4U28oQnJ}  @dot\_asahi\_pubより
\item
  (25/58) TW @NOSUKE0607(清水 潔) 日時: 2021-06-13 17:22:08 +0900
  URL:
  \url{https://twitter.com/NOSUKE0607/status/1403991063103111171\textgreater} {}
  和歌山カレー事件取材で何度か顔を見た事があるが、まさか関空連絡橋なら落ちたのがあの娘だったとは。事件なのか事故なのか、謎が謎を呼びそうだ。\textgreater\textgreater{}
  林眞須美死刑囚、``関空連絡橋から長女が飛び降り''
  \url{https://t.co/j9tljDWF39} 
\end{itemize}

 和歌山カレー事件は和歌山市園部地区にマスコミが終結したことでも異例な経過を辿った事件でしたが、桶川ストーカー殺人事件を独占的に取材し解決したと印象にある清水潔氏が和歌山カレー事件と接点があったというのも意外な発見でした。

 前にも書いたと思いますが、図書館の取り寄せで借りた桶川ストーカー殺人事件の清水潔氏の本は、カラオケボックスでの取材の場面あたりで読むのを中断したまま返却しています。他に集中したいことがあったというのも理由の1つですが、読んでおきたい本とは考えております。

\begin{itemize}
\tightlist
\item
  (29/58) TW @hKodama(???₅₄) 日時: 2021-06-13 21:24:15 +0900 URL:
  \url{https://twitter.com/hKodama/status/1404051991119097856\textgreater} {}
  1998年(平成10年)11月18日\textgreater\textgreater{}
  大阪弁護士会 会長 久保井 一匡\textgreater{}
  和歌山弁護士会 会長 田中 昭彦\textgreater\textgreater{}
   申入書\textgreater{}
  貴社大阪本社10月30日朝刊のコラム「産経抄」は、和歌山のヒ素保険金事件で逮捕された夫婦の弁護人の活動について・・・
  \url{https://t.co/XHJDe7QsRx} 
\end{itemize}

 上記のツイートもけっこう大きな発見に思いました。まさに弁護士鉄道の記録資料です。

 アイコンもプロフィールの名前も割と最近になってリニューアルされたと感じていたアカウントですが、ブックマークに入れた記憶はないものの、割とよく見かけてきたアカウントで、国道54号線の標識がトレードマークのようになっています。プロフィールに広島市から島根県松江市に至るとありました。

 昭和60年頃、中国自動車道の三次インターには何度か乗り降りしているのですが、トラックステーションがあったという記憶もあります。一度、その広島県三好市から国道54号線で島根県松江市に向かったこともありました。

\begin{itemize}
\tightlist
\item
  (45/58) TW @shukan\_bunshun(週刊文春) 日時: 2021-06-16 16:03:31
  +0900 URL:
  \url{https://twitter.com/shukan\_bunshun/status/1405058441580748801\textgreater} {}
  4歳娘と無理心中 林眞須美長女(37)の地獄 和歌山カレー事件から23年目の悲劇\textgreater{}
  \#週刊文春\textgreater{} \url{https://t.co/8fJ9lgAXIV} 
\end{itemize}

 上記の「4歳娘と無理心中 林眞須美長女(37)の地獄 和歌山カレー事件から23年目の悲劇」という見出しの週刊文春の記事も気になっているのですが、リンクを開くと完全な有料記事となっている様子で、週刊文春の記事としては初めて目にした公開部分もない完全有料記事に思えました。

〉〉〉 kk\_hironoのリツイート 〉〉〉

\begin{itemize}
\tightlist
\item
  RT
  kk\_hirono(刑事告発・非常上告_金沢地方検察庁御中)|shukan\_bunshun(週刊文春)
  日時:2021-06-19 13:27/2021/06/16 16:03 URL:
  \url{https://twitter.com/kk\_hirono/status/1406106355547578371} 
  \url{https://twitter.com/shukan\_bunshun/status/1405058441580748801} 
  \textgreater{}
  4歳娘と無理心中 林眞須美長女(37)の地獄 和歌山カレー事件から23年目の悲劇
  \#週刊文春 \url{https://t.co/8fJ9lgAXIV} 
\end{itemize}

 私の告発\市場急配センター殺人未遂事件\金沢地方検察庁・石川県警察御中(@kk\_hirono)のアカウントのリツイートで、リツイートの数が60になったところですが、不思議なほど少なく思えます。もちろん理由はわかりません。

\begin{itemize}
\tightlist
\item
  (46/58) TW @FNN\_News(FNNプライムオンライン) 日時: 2021-06-16
  19:37:17 +0900 URL:
  \url{https://twitter.com/FNN\_News/status/1405112236436189184\textgreater} {}
  和歌山カレー事件 林眞須美死刑囚の新たな「再審請求」
  \#FNNプライムオンライン \#関西テレビ \url{https://t.co/M9SWXFczhv} 
\end{itemize}

\begin{quote}
《引用の始まり》
\end{quote}

\begin{quote}
こうした中、香川県の弁護士が、この再審請求とは別に新たな再審請求を行い、和歌山地方裁判所に受理されたと明らかにしました。

弁護士は「当時の裁判資料を見直した結果、林死刑囚以外の第三者による犯行を示す``新たな証拠''が見つかった。無罪なのは明白だ」と主張しています。
\end{quote}

\begin{quote}
《引用の終わり》
\end{quote}

\begin{itemize}
\tightlist
\item
  和歌山カレー事件 林眞須美死刑囚の新たな「再審請求」
  \url{https://www.fnn.jp/articles/-/197274} 
\end{itemize}

 確認のつもりで記事のリンクを開いたのですが、いくつか見ていた生田暉雄弁護士の記者会見とは趣が違うようで、「香川県の弁護士が」「弁護士は」とあるだけで、生田暉雄弁護士の名前は見当たりません。FNNプライムオンラインとあるので、これは本来テレビ向けのニュースなのかと思えてきました。

\begin{itemize}
\tightlist
\item
  (48/58) TW @MichikoKameishi(弁護士 亀石倫子) 日時: 2021-06-17
  05:16:02 +0900 URL:
  \url{https://twitter.com/MichikoKameishi/status/1405257881390260225\textgreater} {}
  「お母さんはやっていないと信じても、世間はそうは思ってくれない」「事件から離れたい。私は幸せになりたい」\textgreater{}
  4歳娘と無理心中 林眞須美長女(37)の地獄 和歌山カレー事件から23年目の悲劇
  \#週刊文春 \url{https://t.co/5beCQCuxzM} 
\end{itemize}

 亀石倫子弁護士のツイートが出てきましたが、カッコ書きになっているのは、記事からの引用部分になるのでしょう。特別な有料記事になっているのも皮肉な気がします。「事件から離れたい。私は幸せになりたい」という部分ぐらいは見せて、続きは有料というのであれば、わかるようには思います。

 ネットの記事の有料化というのも時代の流れかと思っていますが、今まで無償で読めていたことが不思議に思えたり、報道の公益性で差別化があるのは疑問に思えたり、なのですが、報道する側の独自の判断で、視聴者が選択するのも同じというのが基本的な考えです。

\begin{itemize}
\tightlist
\item
  (55/58) TW @harrier0516osk(向原総合法律事務所 弁護士向原) 日時:
  2021-06-18 09:18:46 +0900 URL:
  \url{https://twitter.com/harrier0516osk/status/1405681358878375936\textgreater} {}
  @bengoshimentaru
  個人的にはそんなにいただいたら嬉しすぎます!笑\textgreater{}
  どの程度の費用が適正かは事案によるかと思いますが、和歌山事件のような事案になると数年越しですもんね。
\end{itemize}

 上記の向原栄大朗弁護士のツイートにある和歌山事件というのは和歌山カレー事件ではなく、紀州のドンファン事件のことかと思います。逮捕されたというニュースから随分経ちますが、起訴されたというニュースはまだ見ていないかもしれません。暫くの間は弁護士のツイートもそこそも反応がありました。

 紀州のドンファン事件で最初に名乗りを上げたのが佐藤大和弁護士でしたが、すっかり名前を見かけなくなりました。芸能人の権利を守る会を立ち上げていましたが、向原栄大朗弁護士もその主要メンバーで、愛媛の農業アイドル自殺問題では、テレビの記者会見に姿があり、極力映らないように意識している姿が印象的でした。

\begin{itemize}
\item
  佐藤大和(代表弁護士・社会保険労務士) \textbar{} レイ法律事務所
  \url{https://t.co/XUjvpktQkl} 
\item
  エンターテインメント・芸能法務 \textbar{} レイ法律事務所
  \url{https://t.co/I1YiwSQXaU} 
\end{itemize}

 他にも検索をしているのですが、佐藤大和弁護士が主催をし、ネットで大々的な宣伝活動に見えたエンターテインメントの会、なんとか協会になっていたような気もするのですが、自然消滅したのか情報が見当たらなくなっています。安井飛鳥弁護士も名を連ねていたと思います。最近、法クラとして問題視されています。

\begin{itemize}
\tightlist
\item
  (58/58) TW @shukan\_bunshun(週刊文春) 日時: 2021-06-19 07:24:11
  +0900 URL:
  \url{https://twitter.com/shukan\_bunshun/status/1406014910832660480\textgreater} {}
  4歳娘と無理心中 林眞須美長女(37)の地獄 和歌山カレー事件から23年目の悲劇
  \textbar{} 週刊文春 電子版 \#週刊文春\textgreater{}
  \url{https://t.co/UtXASxFJZZ} 
\end{itemize}

 最終のツイートとして記録されたのが本日6月19日7時24分の上記の週刊文春のツイートですが、リンクを開くとやはり「<定期購読プラン>スタンダードプラン
2,200円 / 月(税込)年間プラン 22,000円 /
年(税込)」などという表示が出ました。

 少数のリツイートはあるようですが、それもネット記事の紹介の類で、和歌山カレー事件に言及した弁護士のツイートというのは、皆無に近い状態といって過言はないと思います。無関心とは意味が違うのかもしれないですが、無関心というのも相当数の割合でいそうです。

 次にお約束の準抗告ですが、これは弁護士の反応が入れ食い状態だったようです。

 昨年12月1日以降の準抗告に関する弁護士のツイートのまとめ記事を弁護士鉄道の記録資料としてご紹介したいと思います。

\begin{itemize}
\tightlist
\item
  2020年12月01日21時25分の登録:
  \TーTAKA @TGN54\うおおおおおお、保釈請求却下決定からの準抗告通ったあああああああ!!!!!
  \url{http://kk2020-09.blogspot.com/2020/12/ttakatgn54.html} 
\item
  2020年12月27日10時57分の登録: \弁護士 市川
  寛 @imarockcaster42\逮捕状は夜中でも出すけど、勾留に対する準抗告は翌日に回すってのは、通らないと思いますよ。その日のうちに帰れるかどうか
  \url{http://kk2020-09.blogspot.com/2020/12/imarockcaster42\_13.html} 
\item
  2020年12月28日22時38分の登録:
  REGEXP:''準抗告''/データベース登録済みツイートの検索:2020-12-27〜2020-12-28/2020年12月28日22時36分の記録:ユーザ・投稿:60/94件
  \url{http://kk2020-09.blogspot.com/2020/12/regexp2020-12-272020-12.html} 
\item
  2020年12月28日22時56分の登録:
  REGEXP:''準抗告''/データベース登録済みツイート:2020年12月28日22時42分の記録:ユーザ・投稿:248/913件
  \url{http://kk2020-09.blogspot.com/2020/12/regexp202012282242248913.html} 
\item
  2020年12月30日10時40分の登録: %@tomo\_law\_ たろう
  teacher%話題の勾留の準抗告の話し、結局裁判所側は「逮捕・勾留された奴のために何で裁判所が犠牲になるんだ」という差別思考がスタートに
  \url{http://kk2020-09.blogspot.com/2020/12/tomolaw-teacher.html} 
\item
  2021年01月06日22時33分の登録:
  \北白川 @GUv4i6\地方の地裁の所長って、次の異動で高裁部総括とかだからバリバリの実務家である。よって、準抗告対応するし、家事審判官や調停官だったりする。ただ、なん
  \url{http://kk2020-09.blogspot.com/2021/01/guv4i6\_6.html} 
\item
  2021年01月15日08時44分の登録:
  \弁護士戸舘圭之オフィシャル/とってぃ/袴田事件弁護団 @todateyoshiyuki\あとは、弁護人にとっても逮捕準抗告での準備内容がそのまま勾留請求却下のための弁護
  \url{http://kk2020-09.blogspot.com/2021/01/todateyoshiyuki\_88.html} 
\item
  2021年03月06日23時35分の登録:
  \ピピピーッ @O59K2dPQH59QEJx\たとえば実名アカが田舎の特定可能な裁判体に関して、「準抗告棄却ワロタ。やはり身分保障がしっかりされた裁判官様のご判断は浮世
  \url{https://kk2020-09.blogspot.com/2021/03/o59k2dpqh59qejx\_6.html} 
\item
  2021年03月14日18時22分の登録: \弁護士
  中村憲昭 @nakanori930\「特に釈放を急ぐ特別な理由はありますか?」って、いつから休みの日は準抗告の当否判断しないのが原則になったんだろう?最初から?
  \url{https://kk2020-09.blogspot.com/2021/03/nakanori930\_14.html} 
\item
  2021年04月09日12時14分の登録:
  \北白川 @GUv4i6\勾留理由開示制度が,これを受けて上訴(準抗告等)するために拘禁理由を開示するための制度(刑訴法354条参照)とすると,準抗告できるだけの勾留要件
  \url{https://kk2020-09.blogspot.com/2021/04/guv4i6\_9.html} 
\item
  2021年05月02日13時02分の登録:
  \北白川 @GUv4i6\今日は、最近、準抗告とおらんし、ええことないから、神社にお祓いにいく
  \url{https://kk2020-09.blogspot.com/2021/05/guv4i6.html} 
\item
  2021年05月12日00時20分の登録: \弁護士
  中原潤一 @lawyernakahara\延長するなと意見書を提出するも満額の延長。準抗告も棄却。弁護人は勾留の翌日にVに連絡を取って,その翌日にはお会いして取
  \url{https://kk2020-09.blogspot.com/2021/05/lawyernakaharav.html} 
\item
  2021年05月28日21時24分の登録:
  \弁護士樋詰哲朗 @hizumelaw\弊所の勤務弁護士が初めての国選で勾留に対する準抗告が認められました。でかした!と褒めてあげた。
  \url{https://kk2020-09.blogspot.com/2021/05/hizumelaw\_28.html} 
\item
  2021年06月13日19時02分の登録:
  \靴下わんこ @kutusita\_wanko\弁護活動に対する敵視の歴史は検察だけじゃなく。結果無罪にしたけれども、無用な公訴提起を招来したのは、準抗告申立や黙秘権行使と
  \url{https://kk2020-09.blogspot.com/2021/06/kutusitawanko.html} 
\item
  2021年06月14日05時54分の登録: \§
  佐藤倫子 @sato\_\_michiko\通る準抗告がどれだけ瞬発力と手間を要すると思っているのか。金のためなら準抗告なんかせずにダラダラ接見だけしてますわ
  \url{https://kk2020-09.blogspot.com/2021/06/satomichiko.html} 
\item
  2021年06月14日05時56分の登録:
  \泥濘大魔王サイケ @k\_sawmen\検察幹部が「準抗告は逆効果」って言うってことは、準抗告しまくったほうがいいってことですね
  \url{https://kk2020-09.blogspot.com/2021/06/ksawmen\_23.html} 
\item
  2021年06月14日11時36分の登録:
  \TーTAKA @TGN54\ニュースに関連して「準抗告」でツイートを検索したら、準抗告通った旨の報告ツイートがいくつも出てきて、結構通しているのだな、と思った次第。どん
  \url{https://kk2020-09.blogspot.com/2021/06/ttakatgn54.html} 
\item
  2021年06月14日11時37分の登録:
  \魚占い @sakanauranai\報酬目的の準抗告?! \#なんか見た
  \url{https://kk2020-09.blogspot.com/2021/06/sakanauranai\_17.html} 
\item
  2021年06月16日11時08分の登録:
  REGEXP:''準抗告''/データベース登録済みツイートの検索:2020-12-28〜2021-06-15/2021年06月16日11時03分の記録:ユーザ・投稿:147/283件
  \url{https://kk2020-09.blogspot.com/2021/06/regexp2020-12-282021-06.html} 
\item
  2021年06月16日11時22分の登録:
  REGEXP:''準抗告''/データベース登録済みツイートの検索:2021-06-09〜2021-06-15/2021年06月16日11時17分の記録:ユーザ・投稿:111/189件
  \url{https://kk2020-09.blogspot.com/2021/06/regexp2021-06-092021-06.html} 
\item
  2021年06月18日04時53分の登録:
  REGEXP:''準抗告''/データベース登録済みツイートの検索:2021-06-15〜2021-06-17/2021年06月18日04時53分の記録:ユーザ・投稿:17/25件
  \url{https://kk2020-09.blogspot.com/2021/06/regexp2021-06-152021-06\_18.html} 
\end{itemize}

 まとめ記事の期間指定というのはいかようにもできるのですが、期間を絞って数を減らすのは、埋め込みツイートでの表示が見やすいというのが一つの理由です。全体的な俯瞰には適さないかもしれないですが、埋め込みツイートの表示が見やすく関連したツイートも同時に表示されることがあります。

\begin{itemize}
\tightlist
\item
  2021年06月19日11時54分の登録:
  REGEXP:''#岡口基一判事弾劾訴追に反対する法曹の会''/データベース登録済みツイートの検索:2021-06-17〜2021-06-18/2021年06月19日11時52分の記録:ユーザ・投稿:32/40件
  \url{https://kk2020-09.blogspot.com/2021/06/regexp2021-06-172021-06\_75.html} 
\item
  2021年06月19日11時55分の登録:
  \ピピピーッ @O59K2dPQH59QEJx\「強烈な被害者意識を持ち、それを生き甲斐にしているヤツ」から離れろ。そうすれば、現預金はたまり、幸せにもなれる。
  \url{https://kk2020-09.blogspot.com/2021/06/o59k2dpqh59qejx\_19.html} 
\item
  2021年06月19日12時12分の登録:
  REGEXP:''現預金''/データベース登録済みツイートの検索:2017-11-03〜2021-06-19/2021年06月19日12時09分の記録:ユーザ・投稿:82/199件
  \url{https://kk2020-09.blogspot.com/2021/06/regexp2017-11-032021-06.html} 
\item
  2021年06月19日14時47分の登録:
  REGEXP:''準抗告''/データベース登録済みツイートの検索:2021-06-17〜2021-06-19/2021年06月19日14時46分の記録:ユーザ・投稿:19/44件
  \url{https://kk2020-09.blogspot.com/2021/06/regexp2021-06-172021-06\_25.html} 
\item
  2021年06月19日14時48分の登録:
  REGEXP:''準抗告''/データベース登録済みツイートの検索:2021-06-09〜2021-06-09/2021年06月19日14時47分の記録:ユーザ・投稿:7/8件
  \url{https://kk2020-09.blogspot.com/2021/06/regexp2021-06-092021-06-0920210619144778.html} 
\item
  2021年06月19日14時51分の登録:
  REGEXP:''準抗告''/データベース登録済みツイートの検索:2021-06-13〜2021-06-13/2021年06月19日14時48分の記録:ユーザ・投稿:73/110件
  \url{https://kk2020-09.blogspot.com/2021/06/regexp2021-06-132021-06\_19.html} 
\end{itemize}

 2021-06-09〜2021-06-15/2021年06月16日11時17分の記録:ユーザ・投稿:111/189件のまとめ記事で、何度やっても埋め込みツイートの表示が出来なかったので、範囲を狭めて作り直しをしました。6月9日は勘違いだったと思います。6月13日だと思うのですが、これから開いて確認します。

\begin{itemize}
\item
  (001/110) TW @dben82716007(D弁) 日時: 2021-06-13 13:13:54 +0900
  URL:
  \url{https://twitter.com/dben82716007/status/1403928591444811782\textgreater} {}
  ある検察幹部は「必要に応じて容疑者の勾留を請求している。準抗告が増えたところで、こちらは従来通りの対応を続ける」\textgreater\textgreater{}
  ですって!
\item
  「勾留」不服申し立てで弁護士に報酬への懸念(1/2ページ) - 産経ニュース
  \url{https://t.co/GgLSCHDDVe}  ¥\n 2021/6/13 12:00 ¥\n 森西 勇太
\end{itemize}

 2021/6/13
12:00という時刻を見てこれは前にも取り上げていると思ったのですが、「大阪弁護士会は4月から、」というのがポイントとなる記事でした。

 細かい正確なところは抜きにして、逮捕から48時間あるいは72時間で、引き続き身柄を拘束して取り調べを行うのに必要なのが勾留請求で、それに待ったをかけあわよくば勾留取り消しとさせるのが弁護士による準抗告という手続きという私の理解です。なお、勾留理由開示の方は、このところ情報を全く見かけなくなっています。

 東電OL殺害事件も弁護士が被疑者に否認あるいは黙秘を徹底させ、初動捜査を撹乱し妨害したというのが私の見立てになりますが、法律上も逮捕の要件というのは緩やかなもので、相当の嫌疑があるから逮捕で身柄を拘束し、引き続き勾留で取り調べをするはずですが、弁護士というのはその辺りの事情は知らず、お構いなしに難癖をつけ、ときに被疑者の立場を致命的に悪くさせているという印象があります。

 一昔前は被疑者段階での弁護活動というのも余りなかったはずで、そのうち当番弁護というのが出てきましたが、当番弁護でえん罪の嫌疑が晴れたという話も聞いたことがなく、事実無根のえん罪であれば勾留中の取り調べというのは、身の潔白を晴らす機会でもあるはずです。これは身にしみて経験しているところであり、それを遠のかせたのも弁護士が闇雲に仕掛ける警察への負担だという認識です。

 被疑者段階だけではなく被告人となった一審でも黙秘を貫いたとされる和歌山カレー事件の死刑囚ですが、本人も身に覚えがないからこそ強気の態度があったのかと想像したり、無罪での賠償金の獲得を目指したのかとも想像するところがあるのですが、平成10年当時であれば、えん罪の無罪で一攫千金という風潮はまだ少なからず社会に残っていたという気がします。

 和歌山カレー事件の死刑囚とロス疑惑の三浦和義氏との間に親交があり、マスコミ相手の損害賠償の指南あるいは伝授のようなものがあったということも数日前に知ったのですが、無実であれば、マスコミ対応も裏目となって、死刑囚という立場になったように思えてきます。

 弁護士も黙秘を後押ししたようですが、疑惑に丁寧に向き合っていたならば、早い段階で嫌疑は晴れていたのかもしれないという情報が最近になって目立つのですが、そこまで注目したのも長女の家族の心中や変死という契機があったからで、それが6月9日の発生で6月13日のネットの報道となっていたのです。

 今のところテレビではまったく報道を見ていませんが、警察の対応に対する疑問や批判というのも見かけていません。

 和歌山カレー事件は平成10年7月の事件ということですが、私は金沢市北安江の借家に住み、関係者KYNの配管工事の仕事をしていた頃で、パソコンや法律の勉強と再審請求の書面作成もやっていて、テレビは意識的に見ないようにしていた時期でした。2階の部屋にもテレビは置いていませんでした。

 それでもテレビで大きな報道が長く続いた事件であったことはよく憶えています。報道の加熱ぶりでも異様な事件だったと思いますが、えん罪という話が出てきたのも死刑囚となった後のことになるのかと思われます。弁護士にすれば再審請求にのせるのが既定路線だったのかと思えるところです。

 数年前からえん罪や再審請求に対するマスコミの対応自体に変化が感じられるのですが、ニュースとしての結果の報道、バラエティ番組での再現ドラマ以外は、ほとんど取り上げられることがないという気がしています。

 私が知らないだけの契機があったのかとも気になるのですが、数年前までは再審請求やえん罪の取り組みに熱心だったジャーナリストの江川紹子氏も、最近はほとんどその関係の発言が見られず、和歌山カレー事件においても言及は見られません。

\begin{itemize}
\tightlist
\item
  和歌山 カレー (from:amneris84) - Twitter検索 / Twitter
  \url{https://t.co/O3Il7TriqW} 
\end{itemize}

 前にもやっているように思いながら実行したのですが、まったく意外な結果が出てきました。この機会にジャーナリストの江川紹子氏と再審請求、えん罪について個別のエントリーとして取り上げておきたいと思います。

\begin{itemize}
\tightlist
\item
  〈〈〈 2021/06/19 15:57:03 Linux Emacs: 〈〈〈
\end{itemize}

\hypertarget{twitterux691cux7d22ux3067ux8abfux3079ux3066ux307fux308bux30682013ux5e74ux3067ux9014ux5207ux308cux3066ux3044ux305fux30b8ux30e3ux30fcux30caux30eaux30b9ux30c8ux306eux6c5fux5dddux7d39ux5b50ux6c0fux306eux548cux6b4cux5c71ux30abux30ecux30fcux4e8bux4ef6ux306bux95a2ux3059ux308bux30c4ux30a4ux30fcux30c8}{%
\paragraph{Twitter検索で調べてみると、2013年で途切れていたジャーナリストの江川紹子氏の和歌山カレー事件に関するツイート}\label{twitterux691cux7d22ux3067ux8abfux3079ux3066ux307fux308bux30682013ux5e74ux3067ux9014ux5207ux308cux3066ux3044ux305fux30b8ux30e3ux30fcux30caux30eaux30b9ux30c8ux306eux6c5fux5dddux7d39ux5b50ux6c0fux306eux548cux6b4cux5c71ux30abux30ecux30fcux4e8bux4ef6ux306bux95a2ux3059ux308bux30c4ux30a4ux30fcux30c8}}

\begin{itemize}
\tightlist
\item
  〉〉〉 Linux Emacs: 2021/06/19 16:06:28 〉〉〉
\end{itemize}

:CATEGORIES: @kanazawabengosi \#金沢弁護士会 @JFBAsns
日本弁護士連合会(日弁連) \#法務省 @MOJ\_HOUMU
\#ジャーナリストの江川紹子氏 \#再審請求 \#えん罪

\begin{itemize}
\tightlist
\item
  和歌山 カレー (from:amneris84) - Twitter検索 / Twitter
  \url{https://t.co/O3Il7TriqW} 
\end{itemize}

〉〉〉 kk\_hironoのリツイート 〉〉〉

\begin{itemize}
\tightlist
\item
  RT
  kk\_hirono(刑事告発・非常上告_金沢地方検察庁御中)|amneris84(Shoko
  Egawa) 日時:2021-06-19 16:07/2013/06/13 20:03 URL:
  \url{https://twitter.com/kk\_hirono/status/1406146585222737928} 
  \url{https://twitter.com/amneris84/status/345134223826485248} 
  \textgreater{}
  わらしの過去ツイなど →能阿弥被告は常習的に証拠偽造をくり返す 和歌山カレー事件の捜査でも?
  - どこへ行く、日本。 \url{http://t.co/ZJlnuSUgZh} 
\end{itemize}

〉〉〉 kk\_hironoのリツイート 〉〉〉

\begin{itemize}
\tightlist
\item
  RT
  kk\_hirono(刑事告発・非常上告_金沢地方検察庁御中)|amneris84(Shoko
  Egawa) 日時:2021-06-19 16:07/2013/05/08 21:29 URL:
  \url{https://twitter.com/kk\_hirono/status/1406146598854283268} 
  \url{https://twitter.com/amneris84/status/332109983267303424} 
  \textgreater{}
  3)能阿弥被告は、和歌山カレー事件の捜査にも関わり、4通の鑑定書作成に関与したほか、問題のヒ素が付着していたとされる紙コップを科警研に鑑定嘱託する際に運んだり保管したりしていた。この事件は死刑判決が確定しているが、弁護団は再審を請求する中で、現場で発見された紙コップは黄色で
\end{itemize}

〉〉〉 kk\_hironoのリツイート 〉〉〉

\begin{itemize}
\tightlist
\item
  RT
  kk\_hirono(刑事告発・非常上告_金沢地方検察庁御中)|amneris84(Shoko
  Egawa) 日時:2021-06-19 16:07/2012/08/17 08:54 URL:
  \url{https://twitter.com/kk\_hirono/status/1406146611076427779} 
  \url{https://twitter.com/amneris84/status/236249712611704832} 
  \textgreater{}
  今日の読売新聞によれば、鑑定結果を捏造した和歌山県警科捜研の主任研究員は、カレー事件の捜査にも携わっていた。〈鑑定はすべて2人以上で実施。県警は、捏造の可能性は低いとしながらも、念のため調査する〉と。調査は第三者が行うべきだにゃ
\end{itemize}

〉〉〉 kk\_hironoのリツイート 〉〉〉

\begin{itemize}
\tightlist
\item
  RT
  kk\_hirono(刑事告発・非常上告_金沢地方検察庁御中)|amneris84(Shoko
  Egawa) 日時:2021-06-19 16:07/2013/05/08 17:34 URL:
  \url{https://twitter.com/kk\_hirono/status/1406146622677938178} 
  \url{https://twitter.com/amneris84/status/332050825876754432} 
  \textgreater{} 関係あります RT @buvery
  この人、カレー事件とは関係ありませんか。RT:
  元科捜研主任、データ捏造認める 和歌山地裁 - 47NEWS
  \url{http://t.co/BjmGIyDpcw}  ・・・
\end{itemize}

〉〉〉 kk\_hironoのリツイート 〉〉〉

\begin{itemize}
\tightlist
\item
  RT
  kk\_hirono(刑事告発・非常上告_金沢地方検察庁御中)|amneris84(Shoko
  Egawa) 日時:2021-06-19 16:07/2012/08/16 13:43 URL:
  \url{https://twitter.com/kk\_hirono/status/1406146634291941382} 
  \url{https://twitter.com/amneris84/status/235959977896992768} 
  \textgreater{}
  カレー事件の一審からの弁護人は「鑑定嘱託の日付がバックデートされているなど、和歌山県警科捜研を巡る問題はいくつもあった。とてもいい加減でルーズだった」と。再審で、ヒ素が検出された紙コップは別物と弁護団は主張。「問題のコップを科警研などあちこちに運んだのも、和歌山県警科捜研」と
\end{itemize}

〉〉〉 kk\_hironoのリツイート 〉〉〉

\begin{itemize}
\tightlist
\item
  RT
  kk\_hirono(刑事告発・非常上告_金沢地方検察庁御中)|amneris84(Shoko
  Egawa) 日時:2021-06-19 16:07/2012/08/17 09:30 URL:
  \url{https://twitter.com/kk\_hirono/status/1406146644861620224} 
  \url{https://twitter.com/amneris84/status/236258607581241344} 
  \textgreater{} この顔文字かわゆいにゃ RT @r\_i\_k\_i\_y\_a (。☉\_☉。)b
  RT @amneris84:
  今日の読売新聞によれば、鑑定結果を捏造した和歌山県警科捜研の主任研究員は、カレー事件の捜査にも携わっていた。
\end{itemize}

〉〉〉 kk\_hironoのリツイート 〉〉〉

\begin{itemize}
\tightlist
\item
  RT
  kk\_hirono(刑事告発・非常上告_金沢地方検察庁御中)|amneris84(Shoko
  Egawa) 日時:2021-06-19 16:07/2012/08/16 13:22 URL:
  \url{https://twitter.com/kk\_hirono/status/1406146658786680838} 
  \url{https://twitter.com/amneris84/status/235954691622711297} 
  \textgreater{} 事件は14年前。問題の主任研究員は49歳・・・ RT @badjoe2
  おいおい、和歌山と言えば、例の毒カレー事件、まさか、証拠資料をねつ造していないだろうな。マスコミは当時の勤務状態を確認した方がいい。
\end{itemize}

 Twitter検索に出ていたジャーナリストの江川紹子氏のツイートを全てリツイートしたのですが、7件というのは少なすぎる気がします。ジャーナリストの江川紹子氏はTwilogに登録していたように思うので、そちらでも調べておきたいと思います。

\begin{itemize}
\tightlist
\item
  Shoko Egawa(@amneris84)/「和歌山 カレー」の検索結果 - Twilog
  \url{https://t.co/ie3FykE0hq} 
\end{itemize}

 上記のTwilogの検索でも結果は7件のツイートでした。

 能阿弥被告という名前を見たのも初めてのように思うのですが、能楽と世阿弥を組み合わせたような変わった名前で、能楽の舞台などと聞いたことがありますが、和歌山カレー事件が舞台の演目のように思えてきます。ジャーナリストの江川紹子氏の関心もこの科捜研主任に集中しているようです。

 奈良、京都、和歌山は、サスペンスとの呼ばれた2時間ドラマの舞台としてみることが多かったとも思うのですが、推理小説のイメージが重なるのは十津川警部と同じ十津川という地名がその辺りにあったからかもしれません。和歌山県だったようにも思うのですが、確認しておきます。

\begin{itemize}
\tightlist
\item
  十津川村 - Wikipedia \url{https://t.co/GcazxHCcYt} 
  十津川村(とつかわむら)は、奈良県の最南端に位置する村。面積は奈良県で一番大きい。紀伊半島の内陸にある山村で、三重県や和歌山県と県境を接する。
\end{itemize}

 私の予想は外れ奈良県でしたが、和歌山県とも三重県とも県境になるとあります。ジャーナリストの江川紹子氏が強いこだわりを見せていたのが、名張事件で、名張は三重県だったと思いますが、奈良県に近いという話もあったと記憶します。

\begin{itemize}
\tightlist
\item
  名張市、三重県 から 奈良県吉野郡十津川村 - Google マップ
  \url{https://t.co/PYd7p4T1Vf} 
\end{itemize}

 調べてみると116kmから140kmとずいぶん離れていました。十津川村の方が周辺に大きな市町村が見当たらず、ずいぶん辺境に見えます。好きなジャンルではないのですが、十津川といえば、十津川警部の推理小説が有名で、なにかとよく見かけた時期がありました。

\begin{itemize}
\tightlist
\item
  十津川警部シリーズ (渡瀬恒彦) - Wikipedia \url{https://t.co/Ex1PNhJFsW} 
  『西村京太郎サスペンス
  十津川警部シリーズ』(にしむらきょうたろうサスペンス
  とつがわけいぶシリーズ)は、1992年から2015年までTBS系で放送された刑事ドラマシリーズ。全54回。主演は渡瀬恒彦。
\end{itemize}

 十津川警部が渡瀬恒彦というイメージは余りなかったので、放送も余り観ていなかったのかと思いますが、個人的に渡瀬恒彦はタクシー運転手のシリーズやお宮さんというイメージが強く残っています。元気そうだったのに急な訃報で驚いたということもありました。

\begin{itemize}
\tightlist
\item
  渡瀬恒彦 - Wikipedia \url{https://t.co/jR1pUNRbpB} 
\end{itemize}

 渡瀬恒彦が大原麗子と結婚していたというのも意外に思ったのですが、「おみやさん」というシリーズだったようです。2002年-2016年とあるだけで放送回数もわからないですが、未解決事件を操作する係で、迷宮入りから呼ばれていた言葉という説明がドラマにあったように憶えています。

\begin{itemize}
\tightlist
\item
  おみやさん - Wikipedia \url{https://t.co/YHbcOdIQZ9} 
  『おみやさん』は、テレビ朝日系「木曜ミステリー」枠(木曜 20:00 -
  20:54)で放送された刑事ドラマシリーズ。主演は渡瀬恒彦。 ¥\n ¥\n
  石ノ森章太郎の漫画『草壁署迷宮課おみやさん』を原作としている。
\end{itemize}

 前に見かけていたように思ったのですが、それでも石ノ森章太郎という漫画家と刑事ドラマというのは全く意外な組み合わせです。石ノ森章太郎は宮城県石巻市で、長距離トラックの仕事をしていた頃、資料館のような建物の前を通りかかり、漫画家が猪苗代湖の野口英世のように扱われているのは、とても珍しく感じました。

\begin{itemize}
\tightlist
\item
  石ノ森章太郎 - Wikipedia \url{https://t.co/wi287sIdk8}  石ノ森
  章太郎(いしのもり
  しょうたろう、1938年〈昭和13年〉1月25日{[}2{]}{[}3{]} -
  1998年〈平成10年〉1月28日{[}3{]})は、日本の漫画家、特撮作品原作者。本名は小野寺
  章太郎(おのでら しょうたろう){[}3{]}。
\end{itemize}

 名前を見かけなくなって久しく、昭和13年生まれということでまず驚いたのですが、平成10年に60歳で亡くなっていたというのも驚きました。安藤健次郎さんを基準に考えることが多いのですが、安藤健次郎さんは昭和11年12月生まれです。

\begin{itemize}
\tightlist
\item
  宮城県登米郡石森町 - Google マップ \url{https://t.co/3Y9eYw9ZPW} 
\end{itemize}

 石巻市からはけっこう離れた場所で、岩手県一関市の方が近いようですが、その一関市から石巻市の国道というのも金沢市場輸送の長距離トラックの仕事ではよく通った道で、青森市から翌日積みで石巻に向かうときは、ほとんどこの道でした。

 金沢市場輸送で一番多かったのが宮城県の塩釜市と石巻市の鮮魚の定期便で、塩釜市と石巻市の周辺はよく通行していたのですが、松山事件というのもそこあたりで発生した弁護士鉄道の歴史的な冤罪事件でした。死刑囚から再審で無罪になったはずですが、ほとんど見かけることのない刑事事件、刑事裁判となっています。

\begin{itemize}
\tightlist
\item
  松山事件 - Wikipedia \url{https://t.co/cGmfEPUpgv} 
  松山事件(まつやまじけん)は、1955年(昭和30年)10月18日に、宮城県志田郡松山町(現大崎市)にて発生した放火殺人事件と、それに伴った冤罪事件である。四大死刑冤罪事件の一つ(免田事件、財田川事件、松山事件、島田事件)。日本弁護士連合会が支
\end{itemize}

 やはり四大死刑冤罪事件の一つでしたが、免田事件以外はネットの情報も乏しく詳細がわからずにいました。「死刑捏造:
松山事件・尊厳かけた戦いの末に」という本の存在を知って、図書館の取り寄せで読んだのですが、目からウロコが落ちるような感じで弁護士鉄道の歴史的風景が視界に広がりました。

〉〉〉 kk\_hironoのリツイート 〉〉〉

\begin{itemize}
\tightlist
\item
  RT
  kk\_hirono(刑事告発・非常上告_金沢地方検察庁御中)|amneris84(Shoko
  Egawa) 日時:2021-06-19 17:14/2013/12/08 13:46 URL:
  \url{https://twitter.com/kk\_hirono/status/1406163357955022852} 
  \url{https://twitter.com/amneris84/status/409544613506994176} 
  \textgreater{}
  大出教授「死刑再審4事件松山事件までは戦後処理として行われた。その後は間違っていないということになっている」
\end{itemize}

 ジャーナリストの江川紹子氏のTwilogで検索結果が1件だけありました。

\begin{itemize}
\tightlist
\item
  Shoko Egawa(@amneris84)/「名張」の検索結果 - Twilog
  \url{https://t.co/Ftrjdhi1q4} 
\end{itemize}

 予想通り沢山の検索結果のツイートが出てきましたが、「名張毒ぶどう酒事件」というのが多いようです。一審で無罪判決が出ていることでえん罪という説得力があるという向きが多いようですが、マスコミの面前で犯行を認めたとか、真犯人の可能性が高い話もいくつか出ています。

 本人が名古屋拘置所の刑務官に、自分は事件の犯人でえん罪というのは周りが勝手に騒いでいるだけ、と話していたとテレビでやっていたのですが、本人の話というのは具体的に伝わらず、ジャーナリストの江川紹子氏などが本尊を祭り上げた儀式をやっているように思えるのが、この「名張毒ぶどう酒事件」の特徴でした。

 情報の伝わり方にも違和感を覚えるところがあり、気になって調べたところ、一審で無罪判決が出た後、2年以上社会で生活をしながら、無実であれば弁護士とも打ち合わせた出来たはずなのに、控訴審で逆転の有罪判決となっていて、弁護士が何をやっていて、後になって騒ぎ出したのか不思議に思える弁護士現象でした。

 ジャーナリストの江川紹子氏の本というのは一冊も読んだことがないのですが、ネットの記事というのはけっこうな数読んでいます。「死刑捏造:
松山事件・尊厳かけた戦いの末に」のような客観的な状況の描写があるとも思えないのですが、オウム真理教事件の本に関しては取材を含め、今でも高い評価を受けているようです。

〉〉〉 kk\_hironoのリツイート 〉〉〉

\begin{itemize}
\tightlist
\item
  RT
  kk\_hirono(刑事告発・非常上告_金沢地方検察庁御中)|amneris84(Shoko
  Egawa) 日時:2021-06-19 17:35/2021/04/06 15:54 URL:
  \url{https://twitter.com/kk\_hirono/status/1406168670871588864} 
  \url{https://twitter.com/amneris84/status/1379326605684662272} 
  \textgreater{}
  (私の視点)冤罪救済の壁 再審事件を放置する裁判所 木谷明:朝日新聞デジタル
  \url{https://t.co/vIfUD5Vs9r} 
\end{itemize}

〉〉〉 kk\_hironoのリツイート 〉〉〉

\begin{itemize}
\tightlist
\item
  RT
  kk\_hirono(刑事告発・非常上告_金沢地方検察庁御中)|amneris84(Shoko
  Egawa) 日時:2021-06-19 17:35/2021/03/29 10:07 URL:
  \url{https://twitter.com/kk\_hirono/status/1406168730350931968} 
  \url{https://twitter.com/amneris84/status/1376340276453597188} 
  \textgreater{}
  名張毒ぶどう酒事件60年、10次再審請求異議審「新証拠」の評価が焦点(産経新聞)
  \#Yahooニュース \url{https://t.co/vtnmzZQRsE} 
\end{itemize}

〉〉〉 kk\_hironoのリツイート 〉〉〉

\begin{itemize}
\tightlist
\item
  RT
  kk\_hirono(刑事告発・非常上告_金沢地方検察庁御中)|amneris84(Shoko
  Egawa) 日時:2021-06-19 17:35/2021/03/03 21:47 URL:
  \url{https://twitter.com/kk\_hirono/status/1406168833614704642} 
  \url{https://twitter.com/amneris84/status/1367094265139007493} 
  \textgreater{}
  「再審弁護人」鴨志田弁護士、京都に拠点移す「再審法改正のムーブメント起こしたい」(京都新聞)
  \#Yahooニュース \url{https://t.co/9oUDuJc5AA} 
\end{itemize}

〉〉〉 kk\_hironoのリツイート 〉〉〉

\begin{itemize}
\item
  RT
  kk\_hirono(刑事告発・非常上告_金沢地方検察庁御中)|amneris84(Shoko
  Egawa) 日時:2021-06-19 17:35/2021/02/13 08:39 URL:
  \url{https://twitter.com/kk\_hirono/status/1406168871552241665} 
  \url{https://twitter.com/amneris84/status/1360372967708889088} 
  \textgreater{}
  いったん思い込むと、科学的な証拠で無罪が確定した人も、今なお犯人に見えてしまうのか・・・。だからこそ、被疑者をクロとした捜査結果を別の人がシロにする捜査と、無罪推定の原則は、本当に大事だ→再審無罪の青木さんを「犯人と思う」 法廷で元取調官:朝日新聞デジタル
  \url{https://t.co/lp2pSmKrQQ} 
\item
  Shoko Egawa(@amneris84)/「再審」の検索結果 - Twilog
  \url{https://t.co/qObWK25NJW} 
\end{itemize}

 今年に入って再審をキーワードに含むジャーナリストの江川紹子氏のツイートが4件あって、他に2件、市川寛弁護士の袴田事件に関するツイートのリツイートがありました。名張毒ぶどう酒事件60年とあります。弁護士以外が悪い刑事裁判として後世に語り継がれていくのでしょう。

 本来、弁護士鉄道の語り部という役割が大きかったと思うジャーナリストの江川紹子氏で、検察批判というのも社会に与えた影響が相当大きかったのではと思います。

 先日、久しぶりにジャーナリストの江川紹子氏の持ち味というか、切れ味を感じさせるジャーナリストの江川紹子氏のツイートを見かけているのですが、岡口基一裁判官に関するツイートでした。

〉〉〉 kk\_hironoのリツイート 〉〉〉

\begin{itemize}
\tightlist
\item
  RT
  kk\_hirono(刑事告発・非常上告_金沢地方検察庁御中)|amneris84(Shoko
  Egawa) 日時:2021-06-19 17:48/2021/06/18 22:42 URL:
  \url{https://twitter.com/kk\_hirono/status/1406171913622224905} 
  \url{https://twitter.com/amneris84/status/1405883521596489731} 
  \textgreater{}
  開催ありきではなく。花火大会を愛する人々の命と健康を考えた措置どこかの組織と違って・・・
  →「隅田川花火大会」2年連続の中止 コロナ収束見通せず 東京 \textbar{}
  NHKニュース \url{https://t.co/51Vk5fDYR6} 
\end{itemize}

〉〉〉 kk\_hironoのリツイート 〉〉〉

\begin{itemize}
\tightlist
\item
  RT
  kk\_hirono(刑事告発・非常上告_金沢地方検察庁御中)|amneris84(Shoko
  Egawa) 日時:2021-06-19 17:48/2021/06/18 21:35 URL:
  \url{https://twitter.com/kk\_hirono/status/1406171998904995842} 
  \url{https://twitter.com/amneris84/status/1405866646154059787} 
  \textgreater{}
  金曜日に行われることが恒例化していた首相記者会見が、昨日行われたのは、この判決の後の会見、という状況を避けたかったんだろな・・・
  →河井元法相に懲役3年の実刑判決 参院選の買収事件 東京地裁|NHK
  首都圏のニュース \url{https://t.co/rwCa5jSSxl} 
\end{itemize}

〉〉〉 kk\_hironoのリツイート 〉〉〉

\begin{itemize}
\tightlist
\item
  RT
  kk\_hirono(刑事告発・非常上告_金沢地方検察庁御中)|amneris84(Shoko
  Egawa) 日時:2021-06-19 17:48/2021/06/18 14:37 URL:
  \url{https://twitter.com/kk\_hirono/status/1406172046925516800} 
  \url{https://twitter.com/amneris84/status/1405761538414157828} 
  \textgreater{}
  実刑判決が出ました →河井元法相に懲役3年判決 参院選買収事件で東京地裁:
  日本経済新聞 \url{https://t.co/r6ku5kx8YF} 
\end{itemize}

〉〉〉 kk\_hironoのリツイート 〉〉〉

\begin{itemize}
\tightlist
\item
  RT
  kk\_hirono(刑事告発・非常上告_金沢地方検察庁御中)|amneris84(Shoko
  Egawa) 日時:2021-06-19 17:49/2021/06/16 22:32 URL:
  \url{https://twitter.com/kk\_hirono/status/1406172308226527236} 
  \url{https://twitter.com/amneris84/status/1405156409495277577} 
  \textgreater{}
  この方の表現ぶりが適切とは思いませんが、これで罷免とかって、いくらなんでも行きすぎと思いますよ。→仙台高裁判事の罷免を審理へ SNSで不適切投稿(朝日新聞デジタル)
  \url{https://t.co/JLWnbFq4lf} 
\end{itemize}

 他にもジャーナリストの江川紹子氏のタイムラインで見かけて気になっていたジャーナリストの江川紹子氏のツイートをリツイートしました。適切な政権批判をしているつもりなのだと思うのですが、上っ面を引っ掻き回すだけの印象が拭えず、それでも見識に疑いを持たれることもなく社会的な支持や理解は相当受けていそうです。

〉〉〉 kk\_hironoのリツイート 〉〉〉

\begin{itemize}
\tightlist
\item
  RT
  kk\_hirono(刑事告発・非常上告_金沢地方検察庁御中)|1961kumachin(くまちん(弁護士中村元弥))
  日時:2021-06-19 17:59/2021/05/29 09:52 URL:
  \url{https://twitter.com/kk\_hirono/status/1406174876499517441} 
  \url{https://twitter.com/1961kumachin/status/1398442138702991363} 
  \textgreater{}
  そんなに警察官や検察官が信頼できるなら、冤罪は生まれない
  \url{https://t.co/rBoGlRCZIy} 
\end{itemize}

〉〉〉 kk\_hironoのリツイート 〉〉〉

\begin{itemize}
\tightlist
\item
  RT
  kk\_hirono(刑事告発・非常上告_金沢地方検察庁御中)|amneris84(Shoko
  Egawa) 日時:2021-06-19 18:00/2021/05/19 23:33 URL:
  \url{https://twitter.com/kk\_hirono/status/1406175046452666374} 
  \url{https://twitter.com/amneris84/status/1395024907814150146} 
  \textgreater{}
  冤罪事件など、被疑者・被告人の側から取材するケースはあるので、そういう手法を否定するものではないが、かなり驚きました。ここまで本件被疑者に寄り添う理由は、なるべく早く番組で知らせて欲しい、と希望。
\end{itemize}

〉〉〉 kk\_hironoのリツイート 〉〉〉

\begin{itemize}
\tightlist
\item
  RT
  kk\_hirono(刑事告発・非常上告_金沢地方検察庁御中)|amneris84(Shoko
  Egawa) 日時:2021-06-19 18:01/2021/05/19 23:23 URL:
  \url{https://twitter.com/kk\_hirono/status/1406175177843482624} 
  \url{https://twitter.com/amneris84/status/1395022371153580036} 
  \textgreater{}
  愛知のリコール不正事件さっきテレビを見たら、テレ朝系の局のカメラが、田中孝博容疑者が宿泊しているホテルで前夜に逮捕を前提にしたインタビューを行い、早朝の逮捕の時も室内で田中容疑者の後ろからカメラを回している映像に呆気に取られた。そこまでべったりしんねり密になる関係って何なの⁈
\end{itemize}

〉〉〉 kk\_hironoのリツイート 〉〉〉

\begin{itemize}
\tightlist
\item
  RT
  kk\_hirono(刑事告発・非常上告_金沢地方検察庁御中)|amneris84(Shoko
  Egawa) 日時:2021-06-19 18:01/2021/05/08 22:04 URL:
  \url{https://twitter.com/kk\_hirono/status/1406175304737951754} 
  \url{https://twitter.com/amneris84/status/1391016171109974017} 
  \textgreater{}
  半年間の別件による身柄拘束は、後に妥当性が議論にならないか。冤罪懸念だけでなく、出来るだけ早く犯人が最終的に確定するよう考えるのであれば、1つひとつのプロセスは、しっかり吟味することが大事。
\end{itemize}

〉〉〉 kk\_hironoのリツイート 〉〉〉

\begin{itemize}
\tightlist
\item
  RT
  kk\_hirono(刑事告発・非常上告_金沢地方検察庁御中)|amneris84(Shoko
  Egawa) 日時:2021-06-19 18:01/2021/05/08 21:35 URL:
  \url{https://twitter.com/kk\_hirono/status/1406175402372993028} 
  \url{https://twitter.com/amneris84/status/1391008886862192646} 
  \textgreater{}
  茨城一家殺傷事件の捜査が気になる。昨年11月に殺人予備で家宅捜索→殺予の証拠なく、三郷市火災予防条例違反で逮捕→消防法違反で起訴→初公判期日設定→初公判前日に警察手帳の記章偽造容疑で逮捕→消防法違反の公判が5月11日に設定→5月7日に茨城夫妻への殺人容疑で逮捕→ようやく可視化対象に
\end{itemize}

〉〉〉 kk\_hironoのリツイート 〉〉〉

\begin{itemize}
\tightlist
\item
  RT
  kk\_hirono(刑事告発・非常上告_金沢地方検察庁御中)|amneris84(Shoko
  Egawa) 日時:2021-06-19 18:02/2021/05/08 21:44 URL:
  \url{https://twitter.com/kk\_hirono/status/1406175457112776705} 
  \url{https://twitter.com/amneris84/status/1391011181914050568} 
  \textgreater{}
  「第二のサカキバラ」のように騒いでいるメディアもあるが、ここは冷静に事実が明らかになるのを待ちたい。新聞は、かつてイケイケで捜査情報流していた読売などがとりわけ慎重に、捜査側の見立てとは逆の情報も含めて報じている。かなりの変化だと思う。
\end{itemize}

〉〉〉 kk\_hironoのリツイート 〉〉〉

\begin{itemize}
\tightlist
\item
  RT
  kk\_hirono(刑事告発・非常上告_金沢地方検察庁御中)|MichikoKameishi(弁護士
  亀石倫子) 日時:2021-06-19 18:02/2020/12/29 17:43 URL:
  \url{https://twitter.com/kk\_hirono/status/1406175629926539268} 
  \url{https://twitter.com/MichikoKameishi/status/1343840045451497474} 
  \textgreater{}
  「警察捜査における取調べ適正化指針」では1日8時間を超える取調べは原則禁止。長時間の取調べで自白を強要し、冤罪を生んだことへの反省から策定された。しかし鹿児島県警は8時間超の取調べを3年間に338件実施。結局、まったく反省していないのだ。\url{https://t.co/yRxoEKAXQR} 
  \url{https://t.co/3nKXtLCHwF} 
\end{itemize}

〉〉〉 kk\_hironoのリツイート 〉〉〉

\begin{itemize}
\tightlist
\item
  RT
  kk\_hirono(刑事告発・非常上告_金沢地方検察庁御中)|amneris84(Shoko
  Egawa) 日時:2021-06-19 18:03/2020/12/15 22:46 URL:
  \url{https://twitter.com/kk\_hirono/status/1406175705050664960} 
  \url{https://twitter.com/amneris84/status/1338842986600226819} 
  \textgreater{}
  冤罪で無期懲役とされた服役中も、余命宣告された今も、「できることをやる」と。→再審で無罪、命の残りは2カ月 「真実に誠実」語る決意:朝日新聞デジタル
  \url{https://t.co/EZW7H0bxHG} 
\end{itemize}

〉〉〉 kk\_hironoのリツイート 〉〉〉

\begin{itemize}
\tightlist
\item
  RT
  kk\_hirono(刑事告発・非常上告_金沢地方検察庁御中)|amneris84(Shoko
  Egawa) 日時:2021-06-19 18:03/2020/12/05 18:07 URL:
  \url{https://twitter.com/kk\_hirono/status/1406175760180645894} 
  \url{https://twitter.com/amneris84/status/1335148813946814464} 
  \textgreater{}
  23歳から57歳まで身柄拘束され、30年間以上も死刑確定囚として執行の恐怖にさらされた。雪冤後も、冤罪のために無年金の状態に置かれた問題を訴え続けるなど、冤罪被害と戦う生涯だった。合掌→<速報>免田栄さんが死去 死刑囚として国内初の再審無罪(熊本日日新聞)
  \url{https://t.co/raTRAQRSnI} 
\end{itemize}

〉〉〉 kk\_hironoのリツイート 〉〉〉

\begin{itemize}
\tightlist
\item
  RT
  kk\_hirono(刑事告発・非常上告_金沢地方検察庁御中)|amneris84(Shoko
  Egawa) 日時:2021-06-19 18:03/2020/10/29 16:33 URL:
  \url{https://twitter.com/kk\_hirono/status/1406175839541022725} 
  \url{https://twitter.com/amneris84/status/1321716724262072320} 
  \textgreater{}
  長い時間を奪われ、ようやく名誉回復し、国の責任を追及する手続きに取りかかったところだった。冤罪とはかくも残酷。どうかやすらかに。合掌 →松橋事件で再審無罪の宮田浩喜さん、肺炎で死去 87歳:朝日新聞デジタル
  \url{https://t.co/MBJuLJIbPK} 
\end{itemize}

〉〉〉 kk\_hironoのリツイート 〉〉〉

\begin{itemize}
\tightlist
\item
  RT
  kk\_hirono(刑事告発・非常上告_金沢地方検察庁御中)|amneris84(Shoko
  Egawa) 日時:2021-06-19 18:03/2020/10/12 17:30 URL:
  \url{https://twitter.com/kk\_hirono/status/1406175922122723333} 
  \url{https://twitter.com/amneris84/status/1315570492590309376} 
  \textgreater{}
  特に、最後の「刑事控訴審とは何のための裁判なのか」以下が重要。日本で冤罪が生まれ、雪冤が難しい最大の問題は裁判所だと思います。人質司法がなくならないのも、裁判所がそれを認めているから。せっかく丁寧な証人尋問で一審無罪になっても、安易にひっくり返す高裁があるから、検察は控訴する
  \url{https://t.co/LDNfr6VmcB} 
\end{itemize}

〉〉〉 kk\_hironoのリツイート 〉〉〉

\begin{itemize}
\item
  RT
  kk\_hirono(刑事告発・非常上告_金沢地方検察庁御中)|amneris84(Shoko
  Egawa) 日時:2021-06-19 18:04/2020/09/21 10:49 URL:
  \url{https://twitter.com/kk\_hirono/status/1406176021271900162} 
  \url{https://twitter.com/amneris84/status/1307859531007848448} 
  \textgreater{}
  大阪地検特捜部検事による証拠改ざんが発覚して10年。東京新聞(左)は冤罪被害者の村木厚子さんのインタビューで事件の本質を振り返り、改善した点と課題を挙げる。読売新聞(右)は、検察の変化をクローズアップ。「改革は続く」とする林真琴検事総長のインタビュー
  \url{https://t.co/zKHBc92o1s} 
\item
  Shoko Egawa(@amneris84)/「冤罪」の検索結果 - Twilog
  \url{https://t.co/5wIZ8SYa5Q} 
\end{itemize}

 今日は6月19日の土曜日で、前の日曜日になりますが、和歌山カレー事件の関係でテレビをつけていたところ、Mr.サンデーで木谷明弁護士の特集をし、無罪判決30件あるいは30件以上を出した裁判官として紹介していました。物語の世界をみるように思ったのですが、イチケイのカラスというドラマの最終回を盛り立てるコーナーだったようです。

 木谷明弁護士の裁判官時代の無罪判決は埼玉県で多かったという話をみましたが、その多くが高野隆弁護士の手掛けた刑事弁護であったという話で、高野隆弁護士の話としてPDFファイルをたまたま見かけたという記憶です。木谷明弁護士のことを悪く言う人は見かけたことがないですが、高野隆弁護士が絡むだけでも単純な話には思えません。

〉〉〉 kk\_hironoのリツイート 〉〉〉

\begin{itemize}
\item
  RT
  kk\_hirono(刑事告発・非常上告_金沢地方検察庁御中)|amneris84(Shoko
  Egawa) 日時:2021-06-19 18:17/2019/04/23 20:31 URL:
  \url{https://twitter.com/kk\_hirono/status/1406179283635081220} 
  \url{https://twitter.com/amneris84/status/1120651473317031936} 
  \textgreater{} 弁護人は、TBS報道を否定 →TBS NEWSの報道について -
  刑事裁判を考える:高野隆@ブログ \url{https://t.co/Bb9y0zFBPE} 
\item
  刑事裁判を考える:高野隆@ブログ:TBS NEWSの報道について
  \url{https://t.co/yfp1PYC3PX} 
\item
  刑事裁判を考える:高野隆@ブログ:「PDF謄写」の開始を求める申し入れ
  \url{https://t.co/qUJN6T0VMC} 
  証拠開示のデジタル化を実現する会(共同代表 後藤貞人、高野隆)は、6月3日付で、東京地検と大阪地検あてに検察官手持ち証拠をPDFデータで開示することを求める申し入れを行いました。
\end{itemize}

 実際、
証拠開示のデジタル化を実現する、のを遠のかせている代表が、高野隆弁護士と後藤貞人弁護士に思えてならないのですが、とてつもない負担を検察実務に掛けそうです。裁判員裁判の歴史的記録を達成したこの2人の弁護士ですが、どれだけ他の被告人の審理に影響を与えたのかと単純に考えました。

 twitterAPI-search-lawList-mydql-add\_for10exit.rb
高野隆 、というコマンドを実行したところ、高野隆弁護士と甲斐中辰夫という名前が一緒に出てくるツイートがあったことを思い出しました。

\begin{itemize}
\tightlist
\item
  〈〈〈 2021/06/19 18:31:43 Linux Emacs: 〈〈〈
\end{itemize}

\hypertarget{ux5e73ux62105ux5e74ux5f53ux6642ux91d1ux6ca2ux5730ux65b9ux691cux5bdfux5e81ux306eux691cux4e8bux6b63ux3060ux3063ux305fux7532ux6590ux4e2dux8fb0ux592bux5f01ux8b77ux58eb2006ux5e7410ux6708ux306eux91d1ux6ca2ux5730ux65b9ux691cux5bdfux5e81ux304cux5973ux4f53ux76dbux308aux5bb4ux4f1aux3068ux3044ux3046ux30d5ux30eaux30fcux30b8ux30e3ux30fcux30caux30eaux30b9ux30c8ux53e4ux5dddux5229ux660eux6c0fux306eux8a18ux4e8b}{%
\paragraph{平成5年当時金沢地方検察庁の検事正だった甲斐中辰夫弁護士、2006年10月の金沢地方検察庁が「女体盛り宴会」というフリージャーナリスト古川利明氏の記事}\label{ux5e73ux62105ux5e74ux5f53ux6642ux91d1ux6ca2ux5730ux65b9ux691cux5bdfux5e81ux306eux691cux4e8bux6b63ux3060ux3063ux305fux7532ux6590ux4e2dux8fb0ux592bux5f01ux8b77ux58eb2006ux5e7410ux6708ux306eux91d1ux6ca2ux5730ux65b9ux691cux5bdfux5e81ux304cux5973ux4f53ux76dbux308aux5bb4ux4f1aux3068ux3044ux3046ux30d5ux30eaux30fcux30b8ux30e3ux30fcux30caux30eaux30b9ux30c8ux53e4ux5dddux5229ux660eux6c0fux306eux8a18ux4e8b}}

\begin{itemize}
\tightlist
\item
  〉〉〉 Linux Emacs: 2021/06/20 09:02:20 〉〉〉
\end{itemize}

:CATEGORIES: @kanazawabengosi \#金沢弁護士会 @JFBAsns
日本弁護士連合会(日弁連) \#法務省 @MOJ\_HOUMU \#金沢地方検察庁
\#深澤諭史弁護士 \#高野隆弁護士

\begin{quote}
《引用の始まり》
\end{quote}

\begin{quote}
#甲斐中辰夫、オマエ、まだ、最高裁の判事でおったんか。「四国タイムズ」社長の川上道大のオッサンが名誉毀損で起訴された事件で、最高裁で上告が棄却されたのとほぼ同時のタイミングで、オマエ、最高裁に天下っとるそうやないか。あの事件は、起訴した当時の高松地検検事正の佐藤勝が、堀井茂・香川県弁護士会会長の就任パーティーの席上、「弁護士の顔を立てて、川上を起訴してやった」って酒の勢いで漏らしとるんだよな。川上のオッサン、言っとったで、例の片山津温泉での「コンパニオン付き女体盛り宴会」のときの金沢地検の検事正は、オマエだったと。三井環のオッサンの上告審では、一緒に机を並べて仕事をしている古田佑紀と一緒に、裁判体に入れてもらうよう、長官にちゃんと今からアタマ下げておけ。古田も宇都宮地検の検事正でチョーカツを使い込んでおるからな。腐れ法務・検察のメンツに賭けて、ぬあんとしてでも、「上告棄却」に持っていかなアカンからな。
\end{quote}

\begin{quote}
《引用の終わり》
\end{quote}

\begin{itemize}
\tightlist
\item
  大阪高裁、「三井環口封じ逮捕事件」公判の審理打ち切りの仰天情報! :
  古川利明の同時代ウォッチング \url{https://toshiaki.exblog.jp/4570869/} 
\end{itemize}

 Googleで「金沢地方検察庁 女体盛り」で検索したのですが、見つかったのは次の自分のブログ記事ぐらいでした。古川利明という記者の名前がわかったので、そこから調べると見覚えのある「古川利明の同時代ウォッチング」というブログを発見。ページ内検索では見つからなかったですが、2006年10月のアーカイブを指定し見つけました。

\begin{itemize}
\tightlist
\item
  hatena-diary\_20061031 -
  告発\金沢地方検察庁\最高検察庁\法務省\石川県警察御中
  \url{https://hirono-hideki.hatenablog.com/entry/2006/10/31/000000} 
\end{itemize}

 古川利明というとても懐かしく感じる名前だったので、改めて調べてみたのですが、意外なことに年齢が近く昭和40年生まれとなっていました。ずいぶん長い間、お名前をみることがなかったと思うので、まとめ記事を作成しました。これからリンクを開いて内容を確認します。

\begin{itemize}
\tightlist
\item
  2021年06月20日09時08分の登録:
  「古川利明」を@hirono\_hideki @kk\_hirono @s\_hironoで検索 5件の該当 2021-06-20\_09:08の記録
  \url{https://kk2020-09.blogspot.com/2021/06/hironohidekikkhironoshirono52021-06.html} 
\end{itemize}

\begin{quote}
《引用の始まり》
\end{quote}

\begin{quote}
2010-05-09 01:13:38 ``古川利明の同時代ウォッチング :
資本主義にある本質としての http://toshiaki.exblog.jp/5722777/
三井環のオッサンの口封じデッチ上げタイホで、''``贈''``の渡真利の公判を担当した大島忠郁がちゃんと大阪高検刑事部長に出世しとるやんか!(引用)''
https://twitter.com/hirono\_hideki/status/13616338338

2010-05-09 14:43:18 ``古川利明の同時代ウォッチング :
大阪高裁、「三井環口封じ逮 http://toshiaki.exblog.jp/4570869/
例の片山津温泉での「コンパニオン付き女体盛り宴会」のときの金沢地検の検事正は、オマエだったと。(引用:こういうのもありました。)''
https://twitter.com/hirono\_hideki/status/13651935631

2020-12-31 12:10:22 ``RT @hirono\_hideki:
古川利明の同時代ウォッチング : 大阪高裁、「三井環口封じ逮
http://toshiaki.exblog.jp/4570869/
例の片山津温泉での「コンパニオン付き女体盛り宴会」のときの金沢地検の検事正は、オマエだったと。(引用:こういうのもありました。)''
https://twitter.com/kk\_hirono/status/1344480991402856448

2021-06-20 08:08:24 ``古川利明 - Wikiwand https://t.co/u6hGyLQc5f
古川 利明(ふるかわ としあき、1965年[1] -
)は、日本のジャーナリスト。''
https://twitter.com/hirono\_hideki/status/1406388426551611392

2021-06-20 08:12:36 ``古川利明の同時代ウォッチング
https://t.co/NPXmYiS0WY''
https://twitter.com/hirono\_hideki/status/1406389481465282562
\end{quote}

\begin{quote}
《引用の終わり》
\end{quote}

\begin{itemize}
\tightlist
\item
  奉納\危険生物・弁護士脳汚染除去装置\金沢地方検察庁御中\_2020:
  「古川利明」を@hirono\_hideki @kk\_hirono @s\_hironoで検索 5件の該当 2021-06-20\_09:08の記録
  \url{https://kk2020-09.blogspot.com/2021/06/hironohidekikkhironoshirono52021-06.html} 
\end{itemize}

 昨年の大晦日である2020-12-31
12:10:22のツイートというのは全く記憶にないと思ったのですが、よくみるとリツイートになっていました。

 「例の片山津温泉での「コンパニオン付き女体盛り宴会」のときの金沢地検の検事正は、オマエだったと」という内容のツイートをリツイートした経緯が思い出せないですが、2010年5月9日のツイートを2020年12月31日にリツイートしたようです。

\begin{itemize}
\tightlist
\item
  刑事告発・非常上告_金沢地方検察庁御中(@kk\_hirono)/2020年12月31日 -
  Twilog \url{https://t.co/Sx5u2bWMcJ} 
\end{itemize}

 金沢地方検察庁の歴代の検事正について調べていたようですが、気になるツイートの記録を発見しました。

\begin{itemize}
\tightlist
\item
  検察庁にまだいたら、今頃、検事正になるのを楽しみに、でかい執務室で大威張りで過ごす日々だっただろうけど。今や、なか卯で/落合洋司弁護士
  \textbar{} 告発-金沢地方検察庁御中\_2013-WordPress
  \url{https://t.co/4UFQizpRTL} 
\end{itemize}

※ @kk\_hironoのアカウントがブロックされ,リツイートに失敗したツイート

\begin{itemize}
\tightlist
\item
  TW
  yjochi(弁護士落合洋司🌸感染拡大を招く東京(頭狂)オリンピック中止!🌸)
  日時:2013/06/25 20:04:18 URL:
  \url{https://twitter.com/yjochi/status/349483180736266241} 
  \textgreater{}
  検察庁にまだいたら、今頃、検事正になるのを楽しみに、でかい執務室で大威張りで過ごす日々だっただろうけど。今や、なか卯で、すだちおろしうどん食べ、美空ひばりを聞きながら、書類に囲まれ、SNSで次々と弾(?)を発射する毎日。人生って不思議なものだ。笑
\end{itemize}

 大晦日の落合洋司弁護士(東京弁護士会)のツイートなのかと思ったのですが、2013年6月25日20時04分でした。

 落合洋司弁護士(東京弁護士会)はブログなどでよく徳島地検にいた頃の回想を書いていましたが、大阪地検の勢力図のような話や三井環氏の問題にも間接的に自身の経験を交えた発言をしていたように記憶にあります。当時は大阪と東京の裁判所で、刑事裁判の量刑の相場が違うという話もありました。

 「#甲斐中辰夫、オマエ、まだ、最高裁の判事でおったんか。「四国タイムズ」社長の川上道大のオッサンが名誉毀損で起訴された事件で、」という2006年の古川利明氏の記事ですが、生々しい内容で四国が出てきました。

 さきほどは香川県を思い浮かべていたと思ったのですが、記事を読み直すと、「起訴した当時の高松地検検事正の佐藤勝が、堀井茂・香川県弁護士会会長の就任パーティーの席上、「弁護士の顔を立てて、川上を起訴してやった」って酒の勢いで漏らしとるんだよな」とありました。

 この香川県で思い出すのが和歌山カレー事件の再審請求で話題となった生田暉雄弁護士です。愛媛県の弁護士会にも所属していた時期があるという情報を見かけたような気もするのですが、弁護士会の懲戒処分が9回というのは尋常ではない記録です。

\begin{quote}
《引用の始まり》
\end{quote}

\begin{quote}
経歴[編集]1941年、兵庫県神戸市生まれ。関西大学卒業後、高校教師や市役所職員として働きながら司法試験に合格、修習生時代は青法協に所属していた[2]。

1970年、裁判官任官後、1987年に大阪高等裁判所判事に。1992年に退官し弁護士になるまで、裁判官としての勤務は22年間にわたる。

弁護士としては、忌避されがちな難事案を積極的に引き受け、高知白バイ事件、愛媛教科書裁判などの弁護を担う。自らの著書にて裁判所の実態を告発する一方、裁判を「主権実現の手段」と位置付け、市民の手に取り戻す術を提案している[3]。

その一方、2020年までに9回の懲戒を受けた経験があり、当時所属していた香川県弁護士会の会長は現役最多記録ではないかとしている[4]。当人は「弁護士会からのやっかみ等で、弁護士の懲戒ということをやられ」[2]たと述べている。

林眞須美の弁護人を務め、2021年5月31日、新たに和歌山地裁に再審請求を申し立てたことが、同年6月9日に明らかにされた[5][6]。
\end{quote}

\begin{quote}
《引用の終わり》
\end{quote}

\begin{itemize}
\tightlist
\item
  生田暉雄 - Wikipedia
  \url{https://ja.wikipedia.org/wiki/\%E7\%94\%9F\%E7\%94\%B0\%E6\%9A\%89\%E9\%9B\%84} 
\end{itemize}

 「弁護士会からのやっかみ等で、弁護士の懲戒ということをやられ」[2]という脚注のような部分を開くと、「a
b 最高裁のウラ金 2011年1月20日 生田暉雄」というリンクが出てきました。

 最高裁の裏金という話はほとんど聞いた覚えはないのですが、北海道警察や検察庁の裏金は、問題が盛り上がりを見せた時期があり、それが古川利明氏の記事にある2006年10月当時の情勢であったと思います。

 ネットの情報に金沢地方検察庁の裏金問題を見かけたこともあったのですが、検察事務官が内部告発をしたとかするという話であったと記憶します。ただ、大きく取り上げられるようなことはなく、立ち消えしたような感じがあったと、少なくとも私の記憶には残っています。

\begin{itemize}
\tightlist
\item
  最高裁のウラ金 « 魚の目:魚住 昭 責任総編集 ウェブマガジン
  \url{https://t.co/XHgdL64Uj9} 
\end{itemize}

 時刻は10時12分です。上記の記事の「裁判官の飲み会」という辺りまで読んでいるのですが、さきほど変な場所に、魚の頭が落ちていました。包丁で調理されたような切り口で、頭の形はウミタナゴの特徴がありました。

 テレビなどで鳥が空から小魚を落とすという話は聞いたことがあったのですが、鳥以外には考えられないものの気になります。生ごみは大きな編み籠に入っているので、ゴミ袋を漁って取り出すということはあり得ないと思います。

 魚の目というページを読んでいる途中だったということに気がついたのですが、ずいぶん久しぶりに見かけたと思うページで、魚住昭という編集者の名前から「魚の目」なっているのかと気が付きました。

\begin{itemize}
\tightlist
\item
  最高裁のウラ金 « 魚の目:魚住 昭 責任総編集 ウェブマガジン
  \url{https://t.co/XHgdL64Uj9} 
  ところが徳島地裁では、その前にラジオ商事件とか、森永ドライ砒素ミルクの事件なんかがあって、ほかの事件が全部止まって、ロッカーに何本ももうほとんど判決を書くだけの事件がたまっているのです。
\end{itemize}

 時刻は11時48分です。長い文章でしたが読み終えました。

\begin{quote}
《引用の始まり》
\end{quote}

\begin{quote}
生田暉雄氏のプロフィール・1970年  裁判官任官・1987年  大阪高等裁判所判事・1992年  退官(裁判官歴22年)・同年、弁護士登録(香川県弁護士会所属)・現在・・・裁判は主権実現の手段であるとの考えのもとに、東京、宇都宮、愛媛の教科書裁判に関与している。また、最高裁の「やらせタウンミーティング」違法訴訟、国民投票法違憲訴訟を提訴すべく、準備中

(編集者注・これは生田氏の講演内容をまとめたものです。JR東日本労組のご協力に感謝します)
\end{quote}

\begin{quote}
《引用の終わり》
\end{quote}

\begin{itemize}
\tightlist
\item
  最高裁のウラ金 « 魚の目:魚住 昭 責任総編集 ウェブマガジン
  \url{http://uonome.jp/read/1048} 
\end{itemize}

 上記に文末にあった「生田暉雄氏のプロフィール」という部分を引用しましたが、昭和45年に裁判官に任官、平成4年に退官とあります。

 どこで読んだのか思い出せないですが、今朝に読んだ中で一時期、東京の弁護士会に移り、香川に戻ったというのがありました。プロフィールには「同年、弁護士登録(香川県弁護士会所属)」とあるだけなので、愛媛県の弁護士会というのはなさそうです。

 懲戒処分が9回という生田暉雄弁護士の記録ですが、これは全て香川県弁護士会での懲戒処分という可能性が高いと思えてきました。地域的な事情や人間関係という絡みもいろいろとありそうには思えます。

 香川県弁護士会としても日本一の懲戒処分回数の弁護士を記録したいという思いがあったとは思えないですが、何度か業務停止も経験しているようなのによく復活し弁護士業を継続できているものと思います。

 9回の懲戒処分とありますが、少し内容を見たのは2,3件だと思います。探せば他にも見つかるのかもしれないですが、他の弁護士のやっかみで9回も続くとは考えにくく、水面下での熾烈な闘争のようなものが長年続いているのかとも、想像したくなります。

 甲斐中辰夫弁護士について書き始めたのですが、最高裁判事の頃は何かと名前を見かけました。よい評判というのは見かけなかったのですが、それこそやっかみのようなかたちで反感を買うことがあるいはあったのかもしれません。

 Wikipediaによると金沢地方検察庁の検事正とあるのは平成5年で、平成6年には違っていました。検事正の任期が1年というのも極端に短いですが、そのわずか一年ほどの間のことで、片山津温泉での女体盛り宴会の話が出たようです。

 今では見かけることのない「女体盛り」ですが、20年、30年前は割とよく週刊誌などで見かけたような記憶があります。今朝他に「わかめ酒」というのも思い出したのですが、どちらもずいぶん長い間、目にすることのない言葉でした。

 最高裁判事の息子と言っていたKKさんですが、石川県警察の警察官として加賀市の山代温泉かその近くの派出所に勤務しているとき、事故処理かなにかで、客として行ったことのあるソープ嬢と鉢合わせになり、困ったという話をしていたことがありました。

 金沢市場輸送と市場急配センターの竹沢俊寿会長の友人という最高裁判事でしたが、竹沢俊寿会長は中央大学の出身で空手部にいたという話を何度かしていました。平成5年頃になるのか、たまたま何かで読んだのですが、その時点で私立大学出身の最高裁判事はいないという話でした。

\begin{quote}
《引用の始まり》
\end{quote}

\begin{quote}
1958年 兵庫県立豊岡高等学校卒業[2]1962年
中央大学法学部法律学科卒業[2]1964年 司法修習生1966年 検事任官1982年
内閣調査官1985年 東京地方検察庁検事1987年 法務大臣官房営繕課長1990年
千葉地方検察庁次席検事1992年 最高検察庁検事1993年
金沢地方検察庁検事正1994年 東京地方検察庁次席検事
\end{quote}

\begin{quote}
《引用の終わり》
\end{quote}

\begin{itemize}
\tightlist
\item
  甲斐中辰夫 - Wikipedia
  \url{https://ja.wikipedia.org/wiki/\%E7\%94\%B2\%E6\%96\%90\%E4\%B8\%AD\%E8\%BE\%B0\%E5\%A4\%AB} 
\end{itemize}

 もしやと思って確認したのですが、甲斐中辰夫弁護士のWikipediaに、「1962年
中央大学法学部法律学科卒業」とありました。自分が生まれる2年前に大学を卒業というだけで、すごく昔の人に思えたのですが、昭和15年1月2日生まれとあります。

 平成6年の1994年が東京地方裁判所次席検事でした。平成5年が金沢地方検察庁検事正ですが、個人的には栄転に思えます。名前をちらほらと見かけるようになった最高裁判事が、2002年10月7日ですが、その次の退官が2010年1月1日とあるので、7年3ヶ月ほどでしょう。

 最高裁判事の任期というのは、これまで問題にされたのも見たことがなく、深く考えたこともなかったのですが、7年というのはずいぶん長く感じます。元号に直すと、平成14年から平成22年になります。

 そもそも金沢地方検察庁と名古屋高検金沢支部との関係性自体がほとんどわからないのですが、一審で不本意な判決が出た場合、上級庁と協議というのはお決まりとはなっていました。控訴審の段階で一審の地検と協議というのは聞いたことのない話ですが、普通にある話のようにも思えてきました。

 令和3年3月31日付告発状では、名古屋高検金沢支部の松浦由紀夫検事のことを取り上げたと思いますが、そういえば、石川県能登町とあったのも、香川県高松市でした。高松地検の検事正就任というニュースであったように思います。

\begin{itemize}
\tightlist
\item
  告発に至る経緯・奥能登の蛸島事件と,弁護士列車編\#\#\#\#
  秋田県に1億6480万円の賠償を命じた弁護士刺殺事件と蛸島事件との共通性,落合洋司弁護士(東京弁護士会)の反応
  - 告発\金沢地方検察庁\最高検察庁\法務省\石川県警察御中2020
  \url{https://t.co/OzQl3NEQRb} 
\end{itemize}

 検索で見つかった上記の記事に松浦由紀夫検事の記載がありましたが、「高松」とページ内検索しても該当なしでした。

\begin{itemize}
\tightlist
\item
  松浦由紀夫 検事正 - Google 検索 \url{https://t.co/GZcJeXeBcg} 
  最も的確な検索結果を表示するために、上の 21
  件と似たページは除外されています。 ¥\n
  検索結果をすべて表示するには、ここから再検索してください。
\end{itemize}

\begin{quote}
《引用の始まり》
\end{quote}

\begin{quote}
2010-07-21 12:05:40
``平成4年の事件の控訴審では、名古屋高検金沢支部から答弁書のようなものが拘置所に送られて来ました。たぶん「答弁書」という表題になっていたと思います。探せば見つかるかも知れないです。検察官の名前は松浦由紀夫(漢字違っているかも)。平成11年の事件ではありませんでした。\ldots{}''
https://twitter.com/hirono\_hideki/status/19044672811

2014-08-07 10:33:33
``当時の私には本気を疑うような内容だったので、ざっと目を通した程度で、読み返して深く考えることもなかったように思いますが、一審の有罪判決を当然とするような内容であったとは思います。松浦由紀夫とかいう検事でしたが、これは控訴審の判決書にも名前が書かれているはずです。''
https://twitter.com/kk\_hirono/status/497193798074507264
\end{quote}

\begin{quote}
《引用の終わり》
\end{quote}

\begin{itemize}
\tightlist
\item
  奉納\危険生物・弁護士脳汚染除去装置\金沢地方検察庁御中\_2020:
  「松浦由紀夫」を@hirono\_hideki @kk\_hirono @s\_hironoで検索 2件の該当 2021-06-20\_12:38の記録
  \url{https://kk2020-09.blogspot.com/2021/06/hironohidekikkhironoshirono22021-06\_20.html} 
\end{itemize}

 納得がいかないのですが該当が2件のみでした。由紀夫という名前の漢字が間違っているのかとも思えてきたのですが、不思議な現象です。

〉〉〉 kk\_hironoのリツイート 〉〉〉

\begin{itemize}
\item
  RT
  kk\_hirono(刑事告発・非常上告_金沢地方検察庁御中)|s\_hirono(非常上告-最高検察庁御中\_ツイッター)
  日時:2021-06-20 12:45/2021/02/08 12:12 URL:
  \url{https://twitter.com/kk\_hirono/status/1406458184101863433} 
  \url{https://twitter.com/s\_hirono/status/1358614716927549440} 
  \textgreater{}
  2021-02-08-112902\_これは、高松地検の松浦由記夫検事正が、先月28日に着任会見したときの抱負の記事である(6月29日付産経新聞)。松浦検事正は石川県能登町の出身.jpg
  \url{https://t.co/FSAdZILHhE} 
\item
  非常上告-最高検察庁御中\_ツイッター(@s\_hirono)/「松浦」の検索結果 -
  Twilog \url{https://t.co/mmB6JXXvRv} 
\end{itemize}

 やはり名前の漢字を間違えていたようです。2021-02-08-112902というスクリーンショットですが、記事を読み直すと、金沢大学文学部卒業後に司法試験合格とありました。金沢大学の文学部というのは余り聞いた覚えがないですが、どこのキャンパスになるのか気になります。

\begin{itemize}
\tightlist
\item
  四国タイムズ|Shikoku
  Times ニュース 2007年7月号 (平成19年7月5日発行)
  \url{https://t.co/9OnJhMI6u7} 
  高松地検の松浦由記夫検事正が、先月28日に着任会見したときの抱負の記事である(6月29日付産経新聞)。松浦検事正は石川県能登町の出身で、金沢大学文学部卒業後に司法試験に合格したという。
\end{itemize}

 今頃になって気がついたのですが、四国タイムズの記事となっていました。さきほども見かけた四国タイムズですが、単純に新聞社なのかと思っていたところ、地域情報誌のような気もしてきました。なんとかタイムズという新聞社は、全国で他にも見かけたような憶えはありました。

\begin{itemize}
\tightlist
\item
  日本タイムズ - Wikipedia \url{https://t.co/4uYzXqsQMq} 
  日本タイムズ(にっぽんタイムズ)は東京都に本社を置く日本タイムズ社が発行する新聞。2016年3月までの旧称は四国タイムズ。
\end{itemize}

 日本タイムズというのは見覚えがないのですが、四国タイムズが日本タイムズに名称変更になったというのも驚きです。香川県といえば高松市だったと思いますが、政界を揺るがすきっかけとなった皇民党事件の街宣活動のニュースを見たという記憶があります。

\begin{itemize}
\tightlist
\item
  松浦 由記夫弁護士(松浦法律事務所)に法律相談 -
  石川県金沢市、北陸鉄道石川線 野町駅 \textbar{} Legalus
  \url{https://t.co/Elp4pgB6Vg}  〒920-0944 石川県金沢市三口新町1-1-5 ¥\n
  弁護士情報を修正・追加
\end{itemize}

 なぞの住宅地が出てくる住所ですが、そのままページが残っているようです。

 時刻は14時31分です。最高裁判所の判事の出身大学のことですが、講談社現代新書の「最高裁判所」という本に書いてあったのかもしれません。この講談社現代新書というシリーズは値段が安かったこともあり、拘置所にいるときよく購入して読んでいました。

 最初にこのエントリーでは深澤諭史弁護士のことを一緒に取り上げるつもりで、カテゴリーに入れたのですが、思い当たるのは昨夜のリツイートとその直前の岡山の弁護士のツイートのことで、甲斐中辰夫弁護士との関連はないと思います。

 Twitterの検索で「和歌山 事件」として新しい情報が出ていないか見ていたのですが、被告人の自宅が放火されたという話を思い出しました。前に検索しても少ししか情報が見つからなかったのですが、起訴され弁護士が精神障害を争ったという話の断片を見かけたように思います。

 もう一度調べてみたのですが、林真須美死刑囚の自宅を放火した犯人が逮捕されたという情報はありませんでした。放火されたのが2000年2月16日となっていました。

\begin{itemize}
\tightlist
\item
  〈〈〈 2021/06/20 14:58:17 Linux Emacs: 〈〈〈
\end{itemize}

\hypertarget{ux5f01ux8b77ux58ebux3063ux3066ux30afux30ecux30fcux30deux30fcux30b9ux30c8ux30fcux30abux30fcuxff44uxff56ux52a0ux5bb3ux8005ux7b49ux304bux3089ux8077ux696dux5168ux4f53ux3068ux3057ux3066ux9006ux6068ux307fux8cb7ux3044ux3084ux3059ux3044ux3067ux3059ux304bux3089ux306dux3068ux3044ux3046ux6df1ux6fa4ux8aedux53f2ux5f01ux8b77ux58ebux306eux30c4ux30a4ux30fcux30c8ux306bux307fux308bux5f01ux8b77ux58ebux306eux793eux4f1aux7684ux6709ux5bb3ux6027ux3068ux5371ux967aux6027}{%
\paragraph{「弁護士って、クレーマー、ストーカー、DV加害者等から、職業全体として、逆恨み買いやすいですからね。」という深澤諭史弁護士のツイートにみる弁護士の社会的有害性と危険性}\label{ux5f01ux8b77ux58ebux3063ux3066ux30afux30ecux30fcux30deux30fcux30b9ux30c8ux30fcux30abux30fcuxff44uxff56ux52a0ux5bb3ux8005ux7b49ux304bux3089ux8077ux696dux5168ux4f53ux3068ux3057ux3066ux9006ux6068ux307fux8cb7ux3044ux3084ux3059ux3044ux3067ux3059ux304bux3089ux306dux3068ux3044ux3046ux6df1ux6fa4ux8aedux53f2ux5f01ux8b77ux58ebux306eux30c4ux30a4ux30fcux30c8ux306bux307fux308bux5f01ux8b77ux58ebux306eux793eux4f1aux7684ux6709ux5bb3ux6027ux3068ux5371ux967aux6027}}

\begin{itemize}
\tightlist
\item
  〉〉〉 Linux Emacs: 2021/06/20 15:05:44 〉〉〉
\end{itemize}

:CATEGORIES: @kanazawabengosi \#金沢弁護士会 @JFBAsns
日本弁護士連合会(日弁連) \#法務省 @MOJ\_HOUMU \#深澤諭史弁護士
\#ストーカー

\begin{itemize}
\item
  RT
  fukazawas(深澤諭史)|themis\_okayama(弁護士 柴田収@DV・モラハラ離婚案件がメイン)
  日時:2021-06-19 16:51/2021-06-18 19:47 URL:
  \url{https://twitter.com/fukazawas/status/1406157675910418432} 
  \url{https://twitter.com/themis\_okayama/status/1405839691702702083} 
  \textgreater{}
  DV・モラハラをする人って暴力(身体的なものだけでなく精神的・経済的・社会的暴力も含む)を使って家庭内でやりたい放題するんだけど、弁護士が間に入って調停や訴訟に引きずり出したら、それまで使っていた暴力が全然通用しなくなるのか、めちゃくちゃストレスを感じるようです。
\item
  RT fukazawas(深澤諭史)|fukazawas(深澤諭史) 日時:2021-06-19
  16:52/2020-12-06 20:31 URL:
  \url{https://twitter.com/fukazawas/status/1406157842248200195} 
  \url{https://twitter.com/fukazawas/status/1335547372999299073} 
  \textgreater{}
  弁護士って、クレーマー、ストーカー、DV加害者等から、職業全体として、逆恨み買いやすいですからね。\\
  \textgreater{}
  仕事柄、最初の思い通りにならない「敵」になることが多いので。\\
  \textgreater{} (・∀・;) \url{https://t.co/vmrr1hZkBL} 
\end{itemize}

 深澤諭史弁護士が昨年2020年12月6日の自分のツイートを昨日6月19日16時52分にリツイートしていたわけですが、私が深澤諭史弁護士のタイムラインで気がついたのは、夜の遅い時間になっていたように思います。

 なぜ今頃、このタイミングでリツイートなのかと怪訝に思ったのですが、一つ前のツイートをみると柴田収弁護士のツイートのリツイートでした。思い出すまで数秒間時間が掛かったのですが、まもなく、ぽぽひとで知られた岡山市の弁護士だったと思い出しました。

 昭和60年は岡山市によく行ったのですが、国道2号線で広島方面からの通過が多く、それとは別に、岡山の佐川急便に行く仕事がありました。これはほぼ例外なく日帰りの仕事だったと思います。早朝に荷降ろしをして金沢にとんぼ返りをしていました。

 今は道路事情も全く変わっていると思いますが、当時は兵庫県との県境辺りに山陽自動車道があったものの20kmか30kmぐらいの短い区間で、その山陽自動車道を降りてから岡山ブルーハイウェイという有料道路に乗って岡山市内に入っていました。

 その岡山市内に入る手前であったように思うのですが、左折して玉野市方面に向かい、ゆるい上り坂の左手に岡山の佐川急便があったという記憶です。他と比較して大きくない店舗でした。

 昭和60年には2,3度、同じ中西運輸商の仕事で四国に行っているのですが、瀬戸大橋が開通する前で、宇高連絡船で四国に渡っていました。宇野港だったと思いますが、山の方から急な坂道を降りてフェリー乗り場に向かったのが印象に残っています。

\begin{itemize}
\tightlist
\item
  宇高連絡船 展示コーナー 宇高連絡船愛好會 - Google マップ
  \url{https://t.co/Oy5IxCw6DC} 
\end{itemize}

 Googleマップで確認すると宇野港は玉野市でした。岡山市内から宇野港に行った道路というのも記憶にないのですが、岡山の港というのも全く見かけた記憶がなかったのが、テレビの事件のニュースで岡山港を見て、珍しく思ったということもありました。

 中西運輸商の仕事で最初に四国に行ったのが香川県の坂出市だったと思うのですが、善通寺市というのも記憶にあります。愛媛県になりますが香川県の県境に近い川之江市の大きな工場からティッシュペーパーかトイレットペーパーを運んだという記憶もあります。

 それとは別に高知県の須崎市に荷物を運んだこともありました。いずれも宇高連絡船だったと思います。当時はまだお遍路さんのことを知らなかったと思うのですが、何か宗教的な独特の雰囲気は感じていました。

 金沢市場輸送の大型車でも四国へは何度か行っているのですが、その頃には瀬戸大橋が開通し、筍やスイカを運んだ徳島市は、神戸から淡路島までフェリーで渡り、あとは国道を走っていました。有料道路になっていたような気もしますが、余り憶えていません。

\begin{itemize}
\tightlist
\item
  北島町 - Google マップ \url{https://t.co/y46Oo0mWRf}  徳島県板野郡
\end{itemize}

 徳島市内に近かったと記憶にあるのですが、徳島県板野郡北島町となっていました。たぶんここだと思うのですが、昭和62年の4月に筍を積んで松山市に行き、夕方に近い午後に人参を積んだと記憶にあります。東京行きの荷物で、世田谷の市場もありました。

 他に高松市内の大きなデパート、たぶん三越百貨店と思うのですが、そこで荷降ろしをし、次に向かったのも徳島県板野郡北島町であったように思います。畑の中のようなところで一泊したのですが、妻子を同行させていて、妻が保冷車の荷台で寝ると言い出したこともよく憶えています。

 昭和60年当時の中西運輸商では4トン車に乗務していましたが、メインであった佐川急便の仕事でも九州よりは広島便が多かったかもしれません。広島で荷降ろしをして九州に帰り荷を積みに向かうこともあったのですが、ほとんどは国道2号線で福山市に向かっていました。

 広島市内から帰り荷を積んだという記憶はなく、国道2号線で福山市に向かうと、帰りは同じ国道2号線で岡山市内を通り、姫路市内かその手前のたつの市から姫路バイパスに乗っていたという記憶です。

 姫路バイパスからは播但有料道路で、中国自動車道の福崎インターから名神高速に向かう場合と、そのまま直進して国道9号線に出て京都府の福知山市から国道27号線に入って、舞鶴市内を通過して、福井県の敦賀インターから北陸自動車道に乗ることが多かったかもしれません。

 舞鶴経由の方が確実に名神高速の渋滞は避けられ、時間もさほど変わらなかったように思います。国道9号線の分岐点は和田山という地名になっていたように思います。中一日の九州方面の運行も同じ国道9号線を走っていました。

 岡山市内というのも前出の佐川急便以外に荷降ろしがほとんどなく、国道を通過するだけだったのですが、一度だけ岡山駅の駅前を通過した記憶があります。記憶がはっきりしませんが、岡山県津山市から岡山市内に出たこともあったような気がします。

 岡山市といえば、近年のネット上の情報で、最初はテレビで知ったようにも思うのですが、用水路の転落事故が突出しているという話があります。なかなか対策しきれない事情もあるのかと思うのですが、地理的な問題もあるのでしょう。

〉〉〉 kk\_hironoのリツイート 〉〉〉

\begin{itemize}
\item
  RT
  kk\_hirono(刑事告発・非常上告_金沢地方検察庁御中)|K\_masafumi(川村真文(弁護士
  大阪)) 日時:2021-06-20 16:16/2021/06/20 14:11 URL:
  \url{https://twitter.com/kk\_hirono/status/1406511202532880387} 
  \url{https://twitter.com/K\_masafumi/status/1406479677758218242} 
  \textgreater{}
  連れ去り側の戦略は、相手を殴って、高葛藤だから会わせられないというやつ。
\item
  「異常と思わないのが異常」県警本部長も絶句・・・なぜ?
  ``岡山特有''の用水路転落死亡事故 \textbar{} Cyclist
  \url{https://t.co/EoUBZsakck}  2016/03/25 06:00
\item
  3年で転落死79人の岡山「人食い用水路」
  事故多発理由や対策例を岡山市に直撃! \textbar{} くるまのニュース
  \url{https://t.co/kn7AXK7p5u}  2019.12.05
\end{itemize}

 久しぶりに見かけたのですが、海鮮丼アイコンの実名弁護士のツイートが出てきました。一度目は、またしても反射的にウィンドウを閉じてしまったのですが、二度目の投稿でホームのタイムラインを遡るとすぐに見つかりました。

 もともと「ぽぽひと」というのは謎の妖怪のようなイメージがあったのですが、在りし日は深澤諭史弁護士のタイムラインでよくツイートを見かけていたものです。

 d\textbar grep @popohito\textbar wc
-l というコマンドの実行結果が468件ありました。全部というのは多すぎるので、昨年2020年12月1日以降のものに限定し、次に掲載しご紹介します。

\begin{itemize}
\tightlist
\item
  2020年12月01日11時26分の登録:
  \ぽぽひと@wordタケノコ粒あん党 @popohito\看護師や医療事務職も使命感で対応してくれるから手当てを増やす必要はないと???
  \url{http://kk2020-09.blogspot.com/2020/12/wordpopohito.html} 
\item
  2020年12月01日11時34分の登録:
  \ぽぽひと@wordタケノコ粒あん党 @popohito\6歳と4歳って留守番任せるにはどうなんだろ。うちも6歳と4歳だが6歳が重度自閉症なので留守番は到底無理なんだが。
  \url{http://kk2020-09.blogspot.com/2020/12/wordpopohito\_1.html} 
\item
  2020年12月05日03時10分の登録:
  \ぽぽひと@wordタケノコ粒あん党 @popohito\犯罪被害者の考え方を知らない刑事弁護人がダメなのはもちろんだけど、逆に被疑者被告人の心理を知らな過ぎる被害者代理
  \url{http://kk2020-09.blogspot.com/2020/12/wordpopohito\_5.html} 
\item
  2020年12月06日17時27分の登録:
  \ぽぽひと@wordタケノコ粒あん党 @popohito\高収入を得るのに必要なのは自分のスキルを磨くことよりも成長産業の企業に就職することってのに似てるな
  \url{http://kk2020-09.blogspot.com/2020/12/wordpopohito\_6.html} 
\item
  2020年12月06日17時28分の登録:
  \ぽぽひと@wordタケノコ粒あん党 @popohito\逆にあの訴状を見ることにより、現預金を稼ぐために必要なのは起案能力をコツコツと磨く地道な努力よりも、鉱脈を掘り当
  \url{http://kk2020-09.blogspot.com/2020/12/wordpopohito\_7.html} 
\item
  2020年12月06日21時07分の登録:
  @popohito(ぽぽひと@wordタケノコ粒あん党)のツイート ''.*'' 3224/3224:2020-10-27\_1106〜2020-12-06\_2036 2020年12月06日21時07分の記録
  \url{http://kk2020-09.blogspot.com/2020/12/popohitoword322432242020-10-2711062020.html} 
\item
  2020年12月08日11時18分の登録: \蛇毒 @bigbrother939\返信先:
  @popohitoさん,
  @fukazawasさん一番バランスいいのは1年半ではないかと思っています(1年半世代)。
  \url{http://kk2020-09.blogspot.com/2020/12/bigbrother939-popohito-fukazawas11.html} 
\item
  2020年12月10日09時03分の登録:
  \ぽぽひと@wordタケノコ粒あん党 @popohito\中国韓国には堂々と出てこいというけどロシアには言わないのだろうか。引用ツイート
  \url{http://kk2020-09.blogspot.com/2020/12/wordpopohito\_10.html} 
\item
  2020年12月13日18時13分の登録:
  \ぽぽひと@wordタケノコ粒あん党 @popohito\蓋を開けてみればパチンコは感染に全然影響なかったんだけどさ、春先にパチンコ店バッシングしていた人でパチンコ店に謝
  \url{http://kk2020-09.blogspot.com/2020/12/wordpopohito\_13.html} 
\item
  2020年12月15日10時42分の登録:
  \ぽぽひと@wordタケノコ粒あん党 @popohito\権力による自由の規制は必要最小限度であるべきって基本を忘れてはならないと思う。自身もDV被害者支援の端にこっそり
  \url{http://kk2020-09.blogspot.com/2020/12/wordpopohito\_15.html} 
\item
  2020年12月17日12時41分の登録:
  \ぽぽひと@wordタケノコ粒あん党 @popohito\残業代は「長く働いた方がトク」な制度ではなく「長く働かせたら損」な制度なんだけどね。従業員の労働を管理するのは使
  \url{http://kk2020-09.blogspot.com/2020/12/wordpopohito\_17.html} 
\item
  2020年12月22日07時14分の登録:
  \ぽぽひと@wordタケノコ粒あん党 @popohito\性犯罪の無罪判決がことごとくひっくり返った理由は①フラワーデモの効果で裁判所が動いた②地裁裁判官の判断がポンコツ
  \url{http://kk2020-09.blogspot.com/2020/12/wordpopohito\_22.html} 
\item
  2020年12月26日14時03分の登録:
  \ぽぽひと@wordタケノコ粒あん党 @popohito\DV夫のところに子供を置いて別居してしまった\#ジブリで学ぶ法務
  \url{http://kk2020-09.blogspot.com/2020/12/wordpopohito\_26.html} 
\item
  2021年01月04日20時16分の登録:
  \ぽぽひと@wordタケノコ粒あん党 @popohito\アメリカの巨大ローファームが利益を出せるくらいの高収益案件を発注してから言ってくれ。
  \url{http://kk2020-09.blogspot.com/2021/01/wordpopohito.html} 
\item
  2021年01月04日20時16分の登録:
  \ぽぽひと@wordタケノコ粒あん党 @popohito\弁護士を「金で動く」って批判する人は、「食べる人の笑顔が料理人にとっての1番の喜びでしょ!」って言って高級料理を
  \url{http://kk2020-09.blogspot.com/2021/01/wordpopohito\_4.html} 
\item
  2021年01月16日16時11分の登録:
  %@popohito ぽぽひと@wordタケノコ粒あん党%日弁連って「弁護士もコロナで困っています。だからこういう制度を作って助けてください。」って政府に陳情しなければな
  \url{http://kk2020-09.blogspot.com/2021/01/popohitoword.html} 
\item
  2021年01月18日06時38分の登録:
  @popohito(ぽぽひと@wordタケノコ粒あん党)のツイート ''.*'' 3230/3230:2020-12-23\_0738〜2021-01-17\_1655 2021年01月18日06時38分の記録
  \url{http://kk2020-09.blogspot.com/2021/01/popohitoword323032302020-12-2307382021.html} 
\item
  2021年01月21日14時50分の登録:
  \ぽぽひと@wordタケノコ粒あん党 @popohito\黒光りするアレは普通に別の場所に移動して生きていきそうです。ってか、むしろ生物が腐敗してご飯が増えるかも。
  \url{http://kk2020-09.blogspot.com/2021/01/wordpopohito\_21.html} 
\item
  2021年01月26日21時16分の登録:
  \ぽぽひと@wordタケノコ粒あん党 @popohito\これは生活保護とかに進むべきではなかろうか。あと,債務があったら自己破産で。
  \url{http://kk2020-09.blogspot.com/2021/01/wordpopohito\_26.html} 
\item
  2021年01月28日18時09分の登録:
  \ぽぽひと@wordタケノコ粒あん党 @popohito\返信先:
  @All\_of\_Me2018さんあれば大都市だからできる戦略だと思います。100人中90人から10点を
  \url{http://kk2020-09.blogspot.com/2021/01/wordpopohito-allofme2018.html} 
\item
  2021年01月28日18時10分の登録:
  \ぽぽひと@wordタケノコ粒あん党 @popohito\そりゃ、依頼者や潜在的依頼者には到底見せられない闇を吐き出したいからですよ。。。
  \url{http://kk2020-09.blogspot.com/2021/01/wordpopohito\_28.html} 
\item
  2021年01月28日19時13分の登録: \野田隼人 @nodahayato\返信先:
  @popohitoさん事務局とあわせてアクセスする人数が10人以上いますか?
  \url{http://kk2020-09.blogspot.com/2021/01/nodahayato-popohito.html} 
\item
  2021年01月28日19時17分の登録:
  \ぽぽひと@wordタケノコ粒あん党 @popohito\【ゆる募】法律事務所におけるサーバーの内容と費用の相場みなさん,だいたいどんなサーバーを導入してどれくらいの費用
  \url{http://kk2020-09.blogspot.com/2021/01/wordpopohito\_71.html} 
\item
  2021年01月28日19時19分の登録:
  \ぽぽひと@wordタケノコ粒あん党 @popohito\それくらいかかるんですねえ。なら、月額2万円の6年払いはそれほど高過ぎるわけではないのか。。。
  \url{http://kk2020-09.blogspot.com/2021/01/wordpopohito\_12.html} 
\item
  2021年01月29日09時52分の登録:
  \ぽぽひと@wordタケノコ粒あん党 @popohito\ヤバいツイ弁はだいたい実名垢という当職の仮説を補強するサンプルがまた一つ増えた。なお、実名垢だったらヤバいという
  \url{http://kk2020-09.blogspot.com/2021/01/wordpopohito\_29.html} 
\item
  2021年02月03日14時17分の登録:
  \ぽぽひと@wordタケノコ粒あん党 @popohito\このお医者さんの収入の源である健康保険は貧民たちによって支えられているのじゃよ。
  \url{http://kk2020-09.blogspot.com/2021/02/wordpopohito.html} 
\item
  2021年02月05日10時46分の登録:
  \ぽぽひと@wordタケノコ粒あん党 @popohito\今後、ワクチンを輸入する際に謎の中抜き企業が登場し、それを野党が追及したら、政権支持者が「野党はワクチン普及を妨
  \url{http://kk2020-09.blogspot.com/2021/02/wordpopohito\_5.html} 
\item
  2021年02月08日21時03分の登録:
  \ぽぽひと@wordタケノコ粒あん党 @popohito\黙秘が最強の防御であることは今でこそ弁護士の常識になっているけど、ほんの10年くらいまでは黙秘するとかえって不利
  \url{http://kk2020-09.blogspot.com/2021/02/wordpopohito\_8.html} 
\item
  2021年02月12日21時14分の登録:
  %@popohito ぽぽひと@?%当職がやったら絶対に懲戒必至なことを会務に熱心な相手方代理人がやったので依頼者が懲戒請求をしていて綱紀にかかっている最中なのだが,その
  \url{https://kk2020-09.blogspot.com/2021/02/popohito.html} 
\item
  2021年02月17日22時22分の登録:
  \ぽぽひと@ダイエッター @popohito\鬼滅の刃の遊郭編が子供に説明しにくいのって、「人身売買の歴史がー」とか「女性の搾取がー」とかみたいな高尚なじゃなくて、単に性
  \url{https://kk2020-09.blogspot.com/2021/02/popohito\_61.html} 
\item
  2021年02月17日22時22分の登録:
  \ぽぽひと@ダイエッター @popohito\当職は背信的悪意者からの譲受人の子孫です!
  \url{https://kk2020-09.blogspot.com/2021/02/popohito\_17.html} 
\item
  2021年02月22日23時15分の登録:
  \ぽぽひと@ダイエッター @popohito\着手金30万円で3年間徹底的にやりまくった上に成功報酬も実費も免除したのに本人から着手金返せと紛議を申し立てられている案件,
  \url{https://kk2020-09.blogspot.com/2021/02/popohito\_22.html} 
\item
  2021年02月24日14時15分の登録:
  \ぽぽひと@ダイエッター @popohito\「相談者・依頼者には弁護士の質はわからない。広告による集客は情報格差を利用して邪道。」という人が同じ口で「一生懸命仕事をした
  \url{https://kk2020-09.blogspot.com/2021/02/popohito\_24.html} 
\item
  2021年02月27日14時57分の登録:
  \ぽぽひと@ダイエッター @popohito\すげえ、令和になってもセクハラの場面でいやよいやよも好きのうち理論をぶち上げる人がいることに驚く。昭和かよ。
  \url{https://kk2020-09.blogspot.com/2021/02/popohito\_27.html} 
\item
  2021年03月01日22時27分の登録:
  \めしだ@法教育おじさん @r\_messy\返信先: @hdm1987さん,
  @popohitoさんWordは消毒だ〜(強弁)
  \url{https://kk2020-09.blogspot.com/2021/03/rmessy-hdm1987-popohitoword.html} 
\item
  2021年03月03日23時50分の登録:
  \ぽぽひと@ダイエッター @popohito\ツイッターの情報収集もなかなか馬鹿にならなくてね。この前,面会交流案件の非監護親依頼者からイソ弁に,「学校での面会交流ができ
  \url{https://kk2020-09.blogspot.com/2021/03/popohito.html} 
\item
  2021年03月07日19時51分の登録:
  %@popohito ぽぽひと@ダイエッター%中の人がインスタグラマーなのか?
  \url{https://kk2020-09.blogspot.com/2021/03/popohito\_7.html} 
\item
  2021年03月10日20時03分の登録:
  \ぽぽひと@ダイエッター @popohito\ポジショントークとして明らかに法的に誤った情報を(おそらく意図的に)市民向けに発信する弁護士のことはめちゃくちゃ軽蔑する。ガ
  \url{https://kk2020-09.blogspot.com/2021/03/popohito\_10.html} 
\item
  2021年03月10日20時06分の登録:
  \ぽぽひと@ダイエッター @popohito\【悲報】長男氏の4月からの進学先の教員に事件の相手方が2人いることが発覚。
  \url{https://kk2020-09.blogspot.com/2021/03/popohito\_97.html} 
\item
  2021年03月10日23時26分の登録:
  %@popohito ぽぽひと@ダイエッター%人の心を95\%ほど失っているので、「よし、これで面会拒否事由がまた一つ増えた」と思ってしまう。
  \url{https://kk2020-09.blogspot.com/2021/03/popohito\_56.html} 
\item
  2021年03月12日18時38分の登録:
  %@popohito ぽぽひと@ダイエッター%弁護士で意外とあるのが、当事者本人が役所に何らかの申請をしてダメって言われたことについて、弁護士が色々と根拠資料を揃えて交渉
  \url{https://kk2020-09.blogspot.com/2021/03/popohito\_12.html} 
\item
  2021年03月19日10時02分の登録:
  \ぽぽひと@ダイエッター @popohito\専門家が箸にも棒にもかからないクソデマをばら撒きまくって、市民に「専門家の間でもいろんな見解が分かれているんだな」って誤解さ
  \url{https://kk2020-09.blogspot.com/2021/03/popohito\_19.html} 
\item
  2021年03月22日10時14分の登録:
  \ぽぽひと@ダイエッター @popohito\「弁護士が金目当てで◯◯事件をやっている」と言われる◯◯事件、だいたい儲からない案件です。
  \#意外にこれ知られてないんですけ
  \url{https://kk2020-09.blogspot.com/2021/03/popohito\_22.html} 
\item
  2021年04月01日15時26分の登録:
  \ぽぽひと@ダイエッター @popohito\弁護士アカウントでヤバいアカウントはだいたい実名なんだけどね。プロの目から見たらヤバいこと言っていても、実名だから素人には信
  \url{https://kk2020-09.blogspot.com/2021/04/popohito.html} 
\item
  2021年04月06日13時20分の登録:
  \ぽぽひと@ダイエッター @popohito\今の家裁実務だと面会って最終的には認められることが大半なんだけど,最終局面にたどり着くための時間が死ぬほどかかるのが非監護親
  \url{https://kk2020-09.blogspot.com/2021/04/popohito\_6.html} 
\item
  2021年04月08日21時50分の登録:
  \ぽぽひと@ダイエッター @popohito\返信先:
  @un\_co\_the2ndさん野田先生だったかが以前「それはそちらがお決めになることですので」と回答するというツイ
  \url{https://kk2020-09.blogspot.com/2021/04/popohito-uncothe2nd.html} 
\item
  2021年04月14日16時30分の登録:
  \ぽぽひと@ダイエッター @popohito\今日は奮発して鰻を食べるフバ~!
  \#弁護士を一行でイラつかせる選手権
  \url{https://kk2020-09.blogspot.com/2021/04/popohito\_14.html} 
\item
  2021年04月17日15時15分の登録:
  \うの字を名乗る?物 @un\_co\_the2nd\返信先:
  @popohitoさんいくさ・・・します?
  \url{https://kk2020-09.blogspot.com/2021/04/uncothe2nd-popohito.html} 
\item
  2021年06月10日00時38分の登録: \坂本正幸 @sakamotomasayuk\返信先:
  @popohitoさんいきなり重くなるのでなにか間に段階作れないかと思う
  \url{https://kk2020-09.blogspot.com/2021/06/sakamotomasayuk-popohito.html} 
\end{itemize}

 2021年04月17日15時15分の登録といううの字のツイートが記憶に残りますが、ほどなくぽぽひとのTwitterアカウントは非公開になっていたように思います。そういえば、3日ほど前、非公開設定になっていた村松謙弁護士のTwitterアカウントが削除されたのか消滅していました。

 時刻は16時45分です。いつの間にか17時近くなっていたので驚いたのですが、外は日差しがあって日が高く感じられます。しばらくうの字のタイムラインをみていたのですが、香川県の関連で思い出すことがありました。

\begin{itemize}
\tightlist
\item
  TW un\_co\_the2nd(うの字を名乗る💩物) 日時: 2021/06/20 09:59:28
  URL: \url{https://twitter.com/un\_co\_the2nd/status/1406416377074786304} 
  \textgreater{} 誰か〜閲覧してきてけろー \url{https://t.co/P72oDi76K9} 
\end{itemize}

〉〉〉 kk\_hironoのリツイート 〉〉〉

\begin{itemize}
\tightlist
\item
  RT
  kk\_hirono(刑事告発・非常上告_金沢地方検察庁御中)|MasatoshiAdachi(Masatoshi
  Adachi ../.. 足立昌聰) 日時:2021-06-20 16:48/2021/06/16 00:26 URL:
  \url{https://twitter.com/kk\_hirono/status/1406519208691724289} 
  \url{https://twitter.com/MasatoshiAdachi/status/1404822600891568128} 
  \textgreater{}
  主張書面自体を見ていないので、要約が不正確なおそれはあるが、県が本当に「幸福追求権などは基本的人権とはいえず」と主張しているのだとすると、正気の沙汰ではない。
  >ゲーム条例訴訟 「依存症は予防が必要」 原告主張に県反論 地裁口答弁論 /香川
  \textbar{} 毎日新聞 \url{https://t.co/0mVxHVIGog} 
\end{itemize}

 プロフィールを見て弁護士とあるのが意外に感じたのですがリストには登録済みのアカウントでした。足立昌聰弁護士になりますが、これまで数回見ているTwitterの名前で、共謀罪に反対していた関東学院大学の教授の息子になるのかと思いました。

 これまで余り見かけてこなかった弁護士アカウントですが、ツイートのリツイートが3,645件、182件の引用ツイート、3,055件のいいねとなっています。弁護士のツイートとしては久しぶりにずいぶん反応が多いと感じました。

\begin{itemize}
\tightlist
\item
  足立昌勝 - Wikiwand \url{https://t.co/EivuVdcIBT}  足立 昌勝(あだち
  まさかつ、1943年4月30日 -
  )は、日本の法学者。関東学院大学名誉教授。専攻は近代刑法成立史。東京都出身{[}1{]}。
\end{itemize}

 もうずいぶんと長く名前を見かけたことがなかったので、もっと古い時代の人でずっと前に亡くなっているのかと思っていたのですが、昭和18年生まれで存命のようです。平成20年頃になるのか共謀罪の反対で、今の世田谷区長と同じぐらいネットで名前を見かけていました。

 1963年4月
中央大学法学部法律学科入学とありますが、さきほどの甲斐中辰夫元最高裁判事は、1962年卒業となっていたように思います。

 足立昌聰弁護士はプロフィールに久留米の高校卒業となっていたので、親子という可能性はなさそうです。世田谷区長の名前は保坂展人氏でした。暫く前は、拡張機能のツイートの直後によくツイートが出ていたのですが、最近は頻度が低くなっているような気がします。

 さきほどぽぽひとの2020年12月1日以降の記事の記録をみたのですが、実子誘拐や連れ去りというキーワードは見当たらなかったように思います。今も検索すれば見つかると思うのですが、とある記事でぽぽひとの正体が岡山市の実名弁護士として暴露されていました。

\begin{itemize}
\item
  \begin{enumerate}
  \def\labelenumi{(\arabic{enumi})}
  \tightlist
  \item
    ぽぽひと@悪徳の栄えさん (@popohito) / Twitter
    \url{https://t.co/c9vQcAkEtY}  ¥\n ブロックされています ¥\n
    @popohitoさんのフォローやツイートの表示はできません。詳細はこちら
  \end{enumerate}
\end{itemize}

 さきほどアカウントが存在するのか確認しておこうと思いながら忘れていました。Googleで「ぽぽひと 正体」という検索に出てきたのですが、検索結果の1ページ目に、あの記事の見出しは見当たりません。

\begin{quote}
《引用の始まり》
\end{quote}

\begin{quote}
@popohito新62 地方在住
フォローリクエストは法曹、司法修習生、司法試験受験生、法律事務所事務職員、裁判所・検察庁職員のみ受け付けます。2015年5月からTwitterを利用しています2,038
フォロー中3,744 フォロワー
\end{quote}

\begin{quote}
《引用の終わり》
\end{quote}

\begin{itemize}
\tightlist
\item
  ぽぽひと@悪徳の栄えさん (@popohito) / Twitter
  \url{https://twitter.com/popohito} 
\end{itemize}

 今のところ検察庁職員というTwitterアカウントは見ていないと思うのですが、弁護士と検察庁職員の個人的な関係やつきあいというのもこれまで余り考えたことがありませんでした。ただ、最近、女性の検察事務官が暴力団の手先になっていたという話は見かけました。

 忘れていた落合洋司弁護士(東京弁護士会)の古いツイートだったと思います。

\begin{quote}
《引用の始まり》
\end{quote}

\begin{quote}
4月15日付の本サイト記事において、ツイッターで子の連れ去りを公言していた「ぽぽひと(@popohito)」というアカウントの正体は、柴田収(しばた
しゅう)という懲戒歴のある弁護士であることを指摘しました。その翌日「ぽぽひと」弁護士はツイッターにおいて、本サイトの指摘内容を認めました。
\end{quote}

\begin{quote}
《引用の終わり》
\end{quote}

\begin{itemize}
\tightlist
\item
  「ぽぽひと」こと柴田収弁護士の「セカンドハラスメント」 \textbar{}
  弁護士の倫理について考える \url{https://legal-ethics.info/1004/} 
\end{itemize}

 上記の引用部分に4月15日付とありますが、記事は4月19日となっています。見出しが変わっただけかと思ったのですが、そうでもなさそうです。ちょうど、昨日辺りに見かけていたサイトと思ったのですが、見覚えのある見出しが人気記事ラインキングにありました。次の記事です。

\begin{itemize}
\item
  弁護士の篠田奈保子氏、自分への懲戒請求を抑え込むことに成功 \textbar{}
  弁護士の倫理について考える \url{https://t.co/BvNSgZFpkJ}  2021年6月14日
\item
  連れ去り指南を公言する岡山テミス法律事務所の柴田収弁護士 \textbar{}
  弁護士の倫理について考える \url{https://t.co/ujzVlZznBa} 
\end{itemize}

 10項目ある小見出しの1番目に「ツイッターで「ぽぽひと」を名乗る弁護士の正体」とありますが、これがもともと記事の見出しになっていたように思います。

 今日は6月20日で、本当に4月15日なのかと思ったのですが、それというのも冒頭に「橋本崇戴棋士の引退のきっかけにもなった、「子供の連れ去り」が社会問題化しています。」とあるからです。もっと最近のことと思っていたのですが、けっこう経っているのだと気が付きました。

\begin{itemize}
\tightlist
\item
  弁護士の倫理について考える \textbar{}
  法曹倫理や士業の自治制度、報酬制度などについて考えるサイトです。お問い合わせはlegalethicsinfo@gmail.comへどうぞ。
  \url{https://t.co/EYQfbqxzsh} 
\end{itemize}

 どういうサイトなのか気になっていたのですが、記事の一覧のような上記のページを開くと、足立啓太弁護士の実名が入った記事もありました。

\begin{itemize}
\item
  母親を中傷した足立敬太弁護士(旭川)と「公正推論」の心理 \textbar{}
  弁護士の倫理について考える \url{https://t.co/VfVR004kMr} 
\item
  横領と手形偽造で懲戒された東京の坂本昌史弁護士(社会福祉法人俊真会理事長)
  \textbar{} 弁護士の倫理について考える \url{https://t.co/kUPjy871mi} 
  2019年11月1日
\end{itemize}

 これまで余り見た憶えのないサイトだと思ったので、最初の記事を探したのですが、上記の2019年11月1日の記事に行き着きました。これが最初の投稿になるのかはっきりしませんが、途中でサイト名が変わっていない限り、余り見かけていなかったように思います。

\begin{itemize}
\tightlist
\item
  2021年06月20日16時38分の登録:
  \リーチ一発ツモ裏1 @luckymangan\自分も横浜市民ですが、発送予定日がこの日であって、接種予定日は未定なんですよね・・・
  \url{https://kk2020-09.blogspot.com/2021/06/luckymangan\_20.html} 
\item
  2021年06月20日16時38分の登録:
  \うの字を名乗る?物 @un\_co\_the2nd\返信先:
  @ebiben2008さんナメられたら殺す!!!ですな
  \url{https://kk2020-09.blogspot.com/2021/06/uncothe2nd-ebiben2008.html} 
\item
  2021年06月20日16時40分の登録:
  \うの字を名乗る?物 @un\_co\_the2nd\一矢報いる、という語のチョイスからして子供は紛争の道具なんですね感じゃよ
  \url{https://kk2020-09.blogspot.com/2021/06/uncothe2nd\_20.html} 
\item
  2021年06月20日18時21分の登録:
  「弁護士の倫理について考える」を@hirono\_hideki @kk\_hirono @s\_hironoで検索 26件の該当 2021-06-20\_18:21の記録
  \url{https://kk2020-09.blogspot.com/2021/06/hironohidekikkhironoshirono262021-06.html} 
\item
  2021年06月20日18時24分の登録:
  REGEXP:''弁護士の倫理について考える''/データベース登録済みツイートの検索:2021-03-16〜2021-06-20/2021年06月20日18時24分の記録:ユーザ・投稿:3/26件
  \url{https://kk2020-09.blogspot.com/2021/06/regexp2021-03-162021-06.html} 
\end{itemize}

\begin{quote}
《引用の始まり》
\end{quote}

\begin{quote}
2021-03-16 20:59:44
``奈良の悪徳離婚弁護士・西村香苗の「親子を引き離すお仕事」 \textbar{}
弁護士の倫理について考える https://legal-ethics.info/506/
2021年3月2日''
https://twitter.com/hirono\_hideki/status/1371793299850207237

2021-04-17 16:57:06 ``RT @PoDJY2QWOE3dMqv:
連れ去り指南を公言する岡山テミス法律事務所の柴田収弁護士 \textbar{}
弁護士の倫理について考える https://t.co/eusoejZAsa''
https://twitter.com/hirono\_hideki/status/1383328654407602182

2021-04-17 16:57:41 ``RT @yukimamafx:
\textgreater 相手方の不倫現場を押さえ、その場で相手を脅して離婚届などに署名させるなどしたことが、懲戒の理由¥\n¥\nここまでやってくれるなんて神弁護士やん。岡山で何かあったら依頼しまくろ!¥\n¥\n連れ去り指南を公言する岡山テミス法律事務所の柴田収弁護士
\textbar{} 弁護士の倫理について考える https://t.co/kMtDoLYuOi''
https://twitter.com/hirono\_hideki/status/1383328801128521728

2021-04-17 16:58:11 ``RT @SHhBQ3aRmNjfI9l:
連れ去り指南を公言する岡山テミス法律事務所の柴田収弁護士 \textbar{}
弁護士の倫理について考える https://t.co/WtjG7XIO5t
¥\n※弁護事務所が犯罪集団。いや犯罪集団が弁護事務所経営を許されるという凄い国家。''
https://twitter.com/hirono\_hideki/status/1383328925451882496

2021-04-17 17:23:35 ``- 離婚事件を扱う「オススメ弁護士」19選
\textbar{} 弁護士の倫理について考える https://t.co/nbMM1I19yw''
https://twitter.com/hirono\_hideki/status/1383335316799463425

2021-04-17 17:28:21 ``-
なぜ弁護士会役員には犯罪者が多い? 横領や詐欺、盗撮など10年で21件
\textbar{} 弁護士の倫理について考える https://t.co/XalODtvJ7k''
https://twitter.com/hirono\_hideki/status/1383336517175701508

2021-04-21 14:23:12 ``-
「ぽぽひと」こと柴田収弁護士の「セカンドハラスメント」 \textbar{}
弁護士の倫理について考える https://t.co/DhlmDkriT4''
https://twitter.com/hirono\_hideki/status/1384739472642756608

2021-06-19 22:53:31
``弁護士の篠田奈保子氏、自分への懲戒請求を抑え込むことに成功 \textbar{}
弁護士の倫理について考える https://t.co/ZLEaNKzQ9Q''
https://twitter.com/hirono\_hideki/status/1406248783839592451
\end{quote}

\begin{quote}
《引用の終わり》
\end{quote}

\begin{itemize}
\tightlist
\item
  奉納\危険生物・弁護士脳汚染除去装置\金沢地方検察庁御中\_2020:
  「弁護士の倫理について考える」を@hirono\_hideki @kk\_hirono @s\_hironoで検索 26件の該当 2021-06-20\_18:21の記録
  \url{https://kk2020-09.blogspot.com/2021/06/hironohidekikkhironoshirono262021-06.html} 
\end{itemize}

 最初に出てきたのが、「奈良の悪徳離婚弁護士・西村香苗の「親子を引き離すお仕事」」という見出しの記事ですが、この記事で最初に知った弁護士の名前ではなかったと思います。なにかで話題となっていました。

 たしか顔写真も出ていて、悪人には見えなかったですが、指南したメールの内容がネットに公開されていました。ご送信という話だったかもしれません。しばらくしてすっかり忘れていましたが、いちおう、確認はしておきたいと思います。

〉〉〉 kk\_hironoのリツイート 〉〉〉

\begin{itemize}
\tightlist
\item
  RT
  kk\_hirono(刑事告発・非常上告_金沢地方検察庁御中)|TKO86407825(TKO)
  日時:2021-06-20 18:40/2021/03/01 23:55 URL:
  \url{https://twitter.com/kk\_hirono/status/1406547450429534208} 
  \url{https://twitter.com/TKO86407825/status/1366401674567315458} 
  \textgreater{}
  相手弁護士西村香苗が依頼人ではなく誤って私に送信してきたメール
  一生に一度の娘の小学校入学式に「別居親を参加させる義務なんてないですし!」と楽しそうに父娘のきずなを切り裂いて調停する元奈良弁護士会会長
  \url{https://t.co/WnyYx7BRzL} 
\end{itemize}

 3月1日という日付のツイートですが、すっかり忘れていたことに気が付きました。弁護士が依頼人ではなく対立関係の相手方に誤ってメールを送信したとあります。

\begin{itemize}
\tightlist
\item
  弁護士紹介 \textbar{} [奈良弁護士会所属] きずな西大寺法律事務所
  \url{https://t.co/sfc4xFIBvs} 
\end{itemize}

 写真にある西村香苗弁護士ですが、たぶん洋服なのですが、これが法被に見えて酒蔵の女社長のように見えてしまいます。人当たりの良さそうな人で、もともと営業向きなのかとも思えてきます。

 西大寺というのは奈良市内の住所で地名のようですが、「きずな西大寺法律事務所」というのは、何度も読み返したくなるような奥行きとインパクトを感じる命名です。きずなとお寺の名前をそのままくっつけたようにみえるためでしょうか。

\begin{itemize}
\tightlist
\item
  奉納\危険生物・弁護士脳汚染除去装置\金沢地方検察庁御中\_2020:
  REGEXP:''弁護士の倫理について考える''/データベース登録済みツイートの検索:2021-03-16〜2021-06-20/2021年06月20日18時24分の記録:ユーザ・投稿:3/26件
  \url{https://t.co/SUloSfCuQ1} 
\end{itemize}

 私の2つのアカウント以外は、足立敬太弁護士のツイートがあるだけだったのですが、それが予想とは違った内容でした。これで悪徳弁護士というのは行き過ぎで、正直、逆効果で弁護士を利するのではというのが、正直な感想です。

\begin{itemize}
\item
  (04/26) TW @keita\_adachi(豪弁 足立敬太 @ザンギの極み) 日時:
  2021-04-20 20:24:24 +0900 URL:
  \url{https://twitter.com/keita\_adachi/status/1384467983326580740\textgreater} {}
  こんなのが目に入った\textgreater\textgreater{}
  原口未緒弁護士が、子供を利用して金を脅し取る手口をTV告白 \textbar{}
  弁護士の倫理について考える \url{https://t.co/7lG9GPP7rZ} 
\item
  (05/26) TW @keita\_adachi(豪弁 足立敬太 @ザンギの極み) 日時:
  2021-04-20 20:28:03 +0900 URL:
  \url{https://twitter.com/keita\_adachi/status/1384468902998990854\textgreater} {}
  むむむ\textgreater\textgreater{}
  「子と会いたければ金払え」原口未緒弁護士の手口に懲戒請求 \textbar{}
  弁護士の倫理について考える \url{https://t.co/jzj7CjGiTe} 
\end{itemize}

 足立敬太弁護士もこれは行けると思ってツイートしたのかという考えも浮かぶのですが、この2つのツイートが、弁護士界隈で浮かび上がった、というのも不可思議な現象のように思えます。あとは自分を名指しで批判されても徹底スルーという姿勢なのでしょう。

 リンクの記事に出てくる原口未緒弁護士は、画像と動画の静止画をみて、もしかしてあの女性弁護士なのではと思ったのですが、自分のツイートの検索結果で別の女性弁護士と確認できました。印象に残るエピソードの女性弁護士でしたが、名前はすぐに忘れていました。

\begin{quote}
《引用の始まり》
\end{quote}

\begin{quote}
2019-04-19 17:29:28
``原口未緒弁護士(激レア)は被告人と結婚、離婚でバツ3に!学歴や経歴は?
https://newsnachricht.net/featuredperson/haraguchimio''
https://twitter.com/hirono\_hideki/status/1119156043265662976
\end{quote}

\begin{quote}
《引用の終わり》
\end{quote}

\begin{itemize}
\item
  奉納\危険生物・弁護士脳汚染除去装置\金沢地方検察庁御中\_2020:
  「原口未緒」を@hirono\_hideki @kk\_hirono @s\_hironoで検索 1件の該当 2021-06-20\_18:57の記録
  \url{https://kk2020-09.blogspot.com/2021/06/hironohidekikkhironoshirono12021-06.html} 
\item
  原口未緒 弁護士法人未緒法律事務所 \textbar{} 日経 私の道しるべ
  \url{https://t.co/VRQPIapLOF} 
\end{itemize}

 率直に、仏様のようにみえる女性弁護士ですが、勘違いしたという女性弁護士がまた強烈なキャラクターでした。名前はまるっきり憶えておらず、銀座のワンフロワーで女2人男1人の独立した名前の法律事務所があって、いずれも映画に出てくるような名前の法律事務所でした。

 最初のきっかけとして記憶にあるのは、小金井ストーカー事件の被害者女性の代理人弁護士で、男女の弁護士がいた女性弁護士の方ですが、その女性弁護士について調べると、法律事務所の住所から3人の弁護士が出てきたのです。

 ページにある動画を視聴したのですが、YouTubeの動画ではなく、映像がきれいすぎたので気になったのですが、まるで法律事務所のホームページのようなページが、日本経済新聞社のページなのだと、一番下にさり気なくある表示で気が付きました。

\begin{itemize}
\item
  「レアーナ」(玉里友香弁護士)「ケレス」「ウラノス」!?ワンフロアに3つの法律事務所、どういう意味があるのか?
  -- 弁護士自治を考える会 \url{https://t.co/4sKdId4AhN} 
\item
  よくも悪くも有名な玉里友香弁護士(第一東京)に業務停止2月の懲戒処分 ところで銀座の事務所には3つの法律事務所が存在していますが、これはどういう事なんですかね?
  -- 鎌倉九郎 \url{https://t.co/W0TvEagr43} 
\end{itemize}

 思えば、玉里友香弁護士の名前だけではなく、懲戒処分のことも忘れていましたが、「レアーナ」という法律事務所の名前は、シャンプーの商品名ではないかという説を見かけたように思い出しました。上記のページに「ウラノスとケレスはギリシャ・ローマ神話の神々の名前で天体の名前」とあります。

\begin{itemize}
\tightlist
\item
  中傷サイトの削除依頼 発信者情報の特定 検索結果削除
  関連検索キーワード削除 損害賠償請求等 : MIKATA法律事務所
  \url{https://t.co/3VnXjy4tze} 
\end{itemize}

 上記のページにある玉里友香弁護士の写真ですが、今みると、さすがにこれは肌色のシャツを下に着ているのではと思えてきました。素肌からこの格好だと、ちょっと体を動かしただけで乳首が見えるはずです。とにかく物凄い女性弁護士だと、この写真を見て思いました。

 g\textbar grep 玉里友香
の検索結果がなかったので、どうなのかと思っていたのですが、ブログの記事が見つかりました。Googleの検索結果です。

\begin{itemize}
\tightlist
\item
  **
  初めて知ったと思われる玉里友香弁護士,東京銀座のフロアにレアーナ法律事務所,高田沙代子弁護士のケレス法律事務所,まだ調べていないウラノス法律事務所があるという,ケレスは準惑星とも
  - 告発\金沢地方検察庁\最高検察庁\法務省\石川県警察御中2020
  \url{https://t.co/3RMMiHHXIX} 
\end{itemize}

 2020年9月9日の記事ですが、軽く目を通したところ、ギリシア神話のテミスが出てきて、()に「法を司る神」とあります。そういえば、本日、ぽぽひとについて調べ始めたところ、テミス法律事務所というのを見たようで、多少気になっていたのです。

\begin{itemize}
\tightlist
\item
  テミス法律事務所 - Google 検索 \url{https://t.co/asxPVlGR1A} 
\end{itemize}

 全国にテミスと名のつく法律事務所がいくつかあることがわかったのですが、ぽぽひとの正体とも呼ばれた岡山の実名弁護士は、やはり、「岡山テミス法律事務所」となっていました。自分が法律の神様という意識で弁護士の仕事をしているのかと考えさせられます。

\begin{itemize}
\tightlist
\item
  お客様の声_岡山テミス法律事務所様(テレワーク編) - YouTube
  \url{https://t.co/BASO2tQS2E} 
\end{itemize}

 離婚問題などとはまったく無関係でしたが、写真で見ていたのと同じ柴田収弁護士が語りだしていました。印象の異なる別の顔写真もあったので、多少気になっていたのですが、映像だと写真情報よりよくわかると思いました。

 コマンドの出力をパイプでつないでtacコマンドで逆順に表示させたのですが、ぽぽひとの記録は2017年10月から始まっていて、その2017年10月だけで次の記録がありました。

\begin{itemize}
\tightlist
\item
  2017年10月24日22時22分の登録:
  \ぽぽひと@死して屍動けばゾンビ @popohito\依頼者がどうでといい話をしているときは眠気が襲ってくるけれども、法律上重要な点に話題が移った瞬間に
  \url{http://hirono2014sk.blogspot.com/2017/10/popohito\_24.html} 
\item
  2017年10月20日11時04分の登録:
  \ぽぽひと@死して屍動けばゾンビ @popohito\¥\n
  \url{http://hirono2014sk.blogspot.com/2017/10/popohito\_20.html} 
\item
  2017年10月19日11時37分の登録:
  \ぽぽひと@死して屍動けばゾンビ @popohito\アディーレは審査請求して執行停止を申し立てろって文句つける人の感覚、マジで理解不能
  \url{http://hirono2014sk.blogspot.com/2017/10/popohito\_10.html} 
\item
  2017年10月18日00時28分の登録:
  \ぽぽひと@死して屍動けばゾンビ @popohito\なにより、水戸黄門の印籠のような効果は実際のところないですからねえ
  \url{http://hirono2014sk.blogspot.com/2017/10/popohito\_18.html} 
\item
  2017年10月16日00時29分の登録:
  \ぽぽひと@死して屍動けばゾンビ @popohito\弁護士が違法行為をするのは重いってのなら、事務員に昼休み中に電話応対させてる事務所は軒並み業務停止にするか
  \url{http://hirono2014sk.blogspot.com/2017/10/popohito\_16.html} 
\item
  2017年10月14日23時07分の登録:
  \ぽぽひと@死して屍動けばゾンビ @popohito\アディーレって着手金無料や激安が多く、着手金を返還してもそれほどダメージなさそうな印象なんだけど
  \url{http://hirono2014sk.blogspot.com/2017/10/popohito\_23.html} 
\item
  2017年10月14日23時03分の登録:
  \ぽぽひと@死して屍動けばゾンビ @popohito\当地のアディーレ支店の入ってるビル、新聞沙汰になるような事件が起きたら退去せにゃならんという条項が賃貸借契約に入ってるそうな
  \url{http://hirono2014sk.blogspot.com/2017/10/popohito\_19.html} 
\item
  2017年10月14日20時06分の登録:
  %@popohito ぽぽひと@死して屍動けばゾンビ%訴訟の勝ち負けは弁護士の腕よりは事件の筋によって決まるのだが、弁護士が早期に介入や助言をすれば事件の筋自体が変わる
  \url{http://hirono2014sk.blogspot.com/2017/10/popohito\_14.html} 
\item
  2017年10月09日09時56分の登録:
  \ぽぽひと@死して屍動けばゾンビ @popohito\弁護士費用10万円を渋った翌日に出会い系サイトに100万円を突っ込む相談者に頭抱え
  \url{http://hirono2014sk.blogspot.com/2017/10/popohito\_9.html} 
\item
  2017年10月06日00時32分の登録:
  %@popohito ぽぽひと@死して屍動けばゾンビ%今年の人権大会の犯罪被害者支援関係でも、会内で意見が大きく割れて激しい議論が巻き起こりそうなセンシティブな決議が出る予定なんですか
  \url{http://hirono2014sk.blogspot.com/2017/10/popohito\_6.html} 
\item
  2017年10月06日00時29分の登録:
  %@popohito ぽぽひと@死して屍動けばゾンビ%なお、私は個人的には死刑に否定的ですし、仮に明日の会の主張のような決議を人権大会で採択したらやっぱり反対します
  \url{http://hirono2014sk.blogspot.com/2017/10/popohito.html} 
\end{itemize}

 Twitterアカウントのプロフィールの名前が、ぽぽひとが基本で、後ろの部分が幾度か変更されていたことは記憶にあったのですが、もっとも印象的だった「ぽぽひと@死して屍動けばゾンビ」が最初のものとして記録されていたようです。

 ゾンビというのも最近は見かけない言葉ですが、ホラー映画の影響が大きく一時期は長年に渡り、よく見かけた言葉でした。

 そのぽぽひとの正体とされる柴田収弁護士の生年や年齢を確認していないか、確認しても忘れているのですが、この「死して屍」の意味を知る年代はかなり幅が限られてくるように思います。私の記憶では昭和50年代前半ですが、日曜日の昼過ぎに「大江戸捜査網」というテレビドラマがありました。

 現在の記憶のままですが、「ご下命いかにても果たすべし、なお、死して屍ひろうものなし」であったと思います。隠密という言葉も長く見かけたことがなく、最近はほとんどテレビをつけていないものの、テレビで時代劇をみることはほとんどなくなっていました。

 隠密同心というのもあったように思います。スパイという言葉自体、最近は余り見かけず、昔とは意味合いも違っているように思えるのですが、洋画のジャンルでもスパイというのがけっこうあったと思います。日本語に訳せば諜報になるのかもしれません。工作員というのも見かけなくなりました。

\begin{itemize}
\tightlist
\item
  松方弘樹・・・大江戸捜査網 絶唱!夜霧に燃えた女郎花 - YouTube
  \url{https://t.co/FoaJv4vYPC} 
\end{itemize}

 初めと終わりだけで途中を飛ばして視聴したのですが、オープニングにあると思っていた決めセリフが、終わりに近い悪人退治の場面にありました。「隠密同心心得の条」から始まっていたようなことも思い出しました。

\begin{itemize}
\tightlist
\item
  【ドラマOP/ED】桃太郎侍第1話 最終話OP ED 高橋英樹主演 - YouTube
  \url{https://t.co/IQqfH3HLyZ} 
\end{itemize}

 数年前に桃太郎侍の決め台詞の場面がないかとYouTubeで探したことがあったのですが、著作権の関係なのか動画が見つからず弁護士が暗躍しているように想像したことがありました。三波春夫の歌声を聴いたのも久しぶりですが、余り聞き覚えのない曲だと思いました。

\begin{itemize}
\tightlist
\item
  口上(一つ、人の世生き血をすすり・・・二つ、不埒な悪行三昧・・・三つ醜い浮き世の鬼を、退治てくれよう桃太郎!!)桃太郎侍
  - YouTube \url{https://t.co/zjwgxQNoQt}  1,717 回視聴•2021/04/19 ¥\n  ¥\n
  10 ¥\n  ¥\n 6 ¥\n  ¥\n 共有 ¥\n  ¥\n 保存 ¥\n 
\end{itemize}

 映像はおろかセリフの音声もありませんでした。

 時刻は21時48分です。YouTubeで、宇出津や鵜川の町並みの動画をいくつか視聴していました。

 テレビ桃太郎侍の決め台詞の場面の動画は今回も見つからなかったのですが、その他の桃太郎侍や、その前の大江戸捜査網の動画を見ていても、時代の移り変わりというものを大きく感じ、そのあとに見たのが宇出津の町並みでした。これもかなり変わったと、実際に目にするより細かいところで感じました。

\begin{itemize}
\tightlist
\item
  TW fukazawas(深澤諭史) 日時: 2020/12/06 20:31:21 URL:
  \url{https://twitter.com/fukazawas/status/1335547372999299073} 
  \textgreater{}
  弁護士って、クレーマー、ストーカー、DV加害者等から、職業全体として、逆恨み買いやすいですからね。\\
  \textgreater{}
  仕事柄、最初の思い通りにならない「敵」になることが多いので。\\
  \textgreater{} (・∀・;) \url{https://t.co/vmrr1hZkBL} 
\end{itemize}

 深澤諭史弁護士のタイムラインで、深澤諭史弁護士自身のリツイートがそのままとなっていたツイートです。

 これほどの馬鹿が弁護士として未だに存在するのかというのが、まず率直な意見です。

 テミスというギリシア神話の法律の神々が出てきましたが、深澤諭史弁護士も自身を神のように祀るような写真をツイートで公開していたことがありました。高貴で検索すれば見つけられるツイートだと思いますが、そのような家系というか血筋があるような話でした。

 必ずあると思っていた、「ajx-user-mysql-REGEXP\_blogger\_hirono2014sk.rb
fukazawas ``高貴'' `1017-01-01/3000-01-01'」の結果がゼロでした。

\begin{itemize}
\tightlist
\item
  奉納\危険生物・弁護士脳汚染除去装置\金沢地方検察庁御中\_2020:
  REGEXP:''貴''/深澤諭史(@fukazawas)の検索(2013-10-26〜2021-06-12/2021年06月20日22時01分の記録169件)
  \url{https://t.co/baS6jfrEUU} 
\end{itemize}

 何度ブラウザの再起動やページの再読込みを繰り返しても埋め込みツイートの表示が出来ずにいます。埋め込みツイートで画像が見れないと内容が確認できないツイートというのもあって、これが典型的な例だと思っていましたが、それにしても表示が出来ません。

 ブラウザを変更し、さらに再読込を何度か繰り返したところ、ようやく埋め込みツイートが表示され、画像を目視したことで特定できたのが次の深澤諭史弁護士のツイートになります。

※ @kk\_hironoのアカウントがブロックされ,リツイートに失敗したツイート

\begin{itemize}
\tightlist
\item
  TW fukazawas(深澤諭史) 日時:2020/04/01 12:43:04 URL:
  \url{https://twitter.com/fukazawas/status/1245194948556685316} 
  \textgreater{} 【ご報告】\\
  \textgreater{} 本日付で、朝廷より官位を賜りました。\\
  \textgreater{} 今まで黙っていましたが、さる方の末裔でございました。\\
  \textgreater{}
  これからは、公家・貴族弁護士として活動し、相手方には容赦なく朝敵の汚名を着せて戦うことにしたいと思います。\\
  \textgreater{} (・∀・) \url{https://t.co/C7xT5ukW6m} 
\end{itemize}

 これなら以前にスクリーンショットの記録もしていると思いますが、神主の格好をした深澤諭史弁護士の写真で、関連のツイートにはコスプレではなく、ツイートに事実が1つ含まれているともありました。

 「弁護士って、クレーマー、ストーカー、DV加害者等から、職業全体として、逆恨み買いやすいですからね。」とツイートで言い切るぐらいなので、弁護士に人生や生活を狂わされ、救済の道を模索している私の立場では、迷惑この上ないのですが、これも現実として納得感は大きい弁護士業界の実態です。

 この深澤諭史弁護士のツイートのためにどれほど心をかき乱され、時間を無駄に浪費させられてきたのか、そのことを改めて明らかにしておきたいところです。性根が変わらないことも改めて確認した次第です。

\begin{itemize}
\tightlist
\item
  〈〈〈 2021/06/20 22:30:07 Linux Emacs: 〈〈〈
\end{itemize}

\hypertarget{section-2}{%
\paragraph{}\label{section-2}}

\hypertarget{ux91d1ux6ca2ux897fux8b66ux5bdfux7f72}{%
\subsubsection{金沢西警察署}\label{ux91d1ux6ca2ux897fux8b66ux5bdfux7f72}}

\hypertarget{ux5b89ux85e4ux5065ux6b21ux90ceux3055ux3093ux3068ux5317ux5468ux58ebux5f01ux8b77ux58ebux4e8cux3064ux306eux5f01ux8b77ux58ebux9244ux90532021-05-21_ux5317ux5468ux58ebux5f01ux8b77ux58ebux306etwitterux30bfux30a4ux30e0ux30e9ux30a4ux30f3ux3068ux3044ux3046ux8a18ux9332ux306eux4f5cux6210}{%
\paragraph{安藤健次郎さんと北周士弁護士,二つの弁護士鉄道:2021-05-21\_北周士弁護士のTwitterタイムライン,という記録の作成}\label{ux5b89ux85e4ux5065ux6b21ux90ceux3055ux3093ux3068ux5317ux5468ux58ebux5f01ux8b77ux58ebux4e8cux3064ux306eux5f01ux8b77ux58ebux9244ux90532021-05-21_ux5317ux5468ux58ebux5f01ux8b77ux58ebux306etwitterux30bfux30a4ux30e0ux30e9ux30a4ux30f3ux3068ux3044ux3046ux8a18ux9332ux306eux4f5cux6210}}

\begin{itemize}
\tightlist
\item
  〉〉〉 Linux Emacs: 2021/05/21 10:31:34 〉〉〉
\end{itemize}

:CATEGORIES: @kanazawabengosi \#金沢弁護士会 @JFBAsns
\#日本弁護士連合会(日弁連) \#法務省 @MOJ\_HOUMU

 まず,2021年05月19日22時45分の記録,とした金沢弁護士会のTwitterから始まるこれまでのデータベースへの記録をご紹介したいと思います。

 この流れの中で,北周士弁護士のタイムラインでの一連のツイートという大きな発見がありました。

\begin{itemize}
\tightlist
\item
  2021年05月19日22時45分の登録:
  @kanazawabengosi(金沢弁護士会)のツイート ''.*'' 15/15:2020-04-15\_1209〜2021-02-10\_1549 2021年05月19日22時45分の記録
  \url{https://kk2020-09.blogspot.com/2021/05/kanazawabengosi15152020-04-1512092021.html} 
\item
  2021年05月19日22時46分の登録:
  @JFBAsns(日本弁護士連合会(日弁連))のツイート ''.*'' 1665/1665:2015-10-01\_1241〜2021-05-19\_1506 2021年05月19日22時46分の記録
  \url{https://kk2020-09.blogspot.com/2021/05/jfbasns166516652015-10-0112412021-05.html} 
\item
  2021年05月20日07時24分の登録:
  ツイートの記録資料:\法務検察・石川県警察宛\/モトケン(@motoken\_tw)/''2021年05月19日'':42件
  \url{https://kk2020-09.blogspot.com/2021/05/motokentw2021051942.html} 
\item
  2021年05月20日07時25分の登録:
  2021-05-19の投稿一覧\検察・石川県警察宛記録資料\奉納\危険生物・弁護士脳汚染除去装置\金沢地方検察庁御中:32件
  \url{https://kk2020-09.blogspot.com/2021/05/2021-05-1932.html} 
\item
  2021年05月20日07時25分の登録:
  ツイートの記録資料:\法務検察・石川県警察宛\/小倉秀夫(@chosakukenho)/''2021年05月19日'':18件
  \url{https://kk2020-09.blogspot.com/2021/05/chosakukenho2021051918.html} 
\item
  2021年05月20日07時27分の登録:
  \モトケン @motoken\_tw\例えば、手製の毒薬で人を殺した殺人事件の犯罪報道で、毒薬の作り方を詳細に報道してもいいのか、ということはよくよく考えた方がいいと思うけ
  \url{https://kk2020-09.blogspot.com/2021/05/motokentw\_20.html} 
\item
  2021年05月20日08時25分の登録:
  \4代目 @4thlawyer\少年非行は重大事件も含めて激減しているんだぜ。幽遊白書とかスラムダンクの主人公みたいな不良少年はもう鑑別所には滅多にいないんだぜ。罪を犯し
  \url{https://kk2020-09.blogspot.com/2021/05/44thlawyer.html} 
\item
  2021年05月20日09時09分の登録: twitterAPI-search-lawList-mydql-add.rb
  再審 \textbar{} tee -a s.txt
  \url{https://kk2020-09.blogspot.com/2021/05/twitterapi-search-lawlist-mydql-addrb.html} 
\item
  2021年05月20日09時20分の登録:
  kk2020-09\_ajx-all-user-mysql-REGEXP\_blogger.rb ``再審'' ``2021-05-13
  09:01/2021-05-20 09:01'' \textbar{} tee s.txt
  \url{https://kk2020-09.blogspot.com/2021/05/kk2020-09ajx-all-user-mysql.html} 
\item
  2021年05月20日09時56分の登録: 2021年05月20日09時39分の実行記録:
  twitterAPI-search-lawList-mydql-add.rb ``@fukazawas''
  ツイート数:6/2420 リツイート数:14/2420 トータル:51
  \url{https://kk2020-09.blogspot.com/2021/05/202105200939-twitterapi-search-lawlist.html} 
\item
  2021年05月20日09時58分の登録:
  REGEXP:''@fukazawas''/データベース登録済みツイートの検索:2021-04-30〜2021-05-19/2021年05月20日09時57分の記録:ユーザ・投稿:14/217件
  \url{https://kk2020-09.blogspot.com/2021/05/regexpfukazawas2021-04-302021-05.html} 
\item
  2021年05月20日09時59分の登録:
  ターミナルの記録:REGEXP:''@fukazawas''/データベース登録済みツイートの検索:2021-04-30〜2021-05-19/2021年05月20日09時57分の記録:ユーザ・投稿:14/217件
  \url{https://kk2020-09.blogspot.com/2021/05/regexpfukazawas2021-04-302021-05\_20.html} 
\item
  2021年05月20日10時40分の登録:
  2019年5月の写真:九十九湾・小木港とも旗祭り・青柏祭・羽咋市内・氣多大社
  \url{https://kk2020-09.blogspot.com/2021/05/20195.html} 
\item
  2021年05月20日12時21分の登録:
  2021-05-20_福井刑務所 ノート雑記帳 平成6年4月18日から4月20日
  \url{https://kk2020-09.blogspot.com/2021/05/2021-05-206418420.html} 
\item
  2021年05月20日13時49分の登録:
  REGEXP:''旭川.*少女''/データベース登録済みツイートの検索:2021-04-29〜2021-05-20/2021年05月20日13時48分の記録:ユーザ・投稿:14/245件
  \url{https://kk2020-09.blogspot.com/2021/05/regexp2021-04-292021-05.html} 
\item
  2021年05月20日14時28分の登録:
  \モトケン @motoken\_tw\それ以外の問題があるかないかは、実際の報道次第。記者の能力の低さが窺われるので、問題のない記事にはなりそうもない気がしますけどね。
  \url{https://kk2020-09.blogspot.com/2021/05/motokentw\_61.html} 
\item
  2021年05月20日15時20分の登録:
  \うの字を名乗る?物 @un\_co\_the2nd\なお弊所では費用対効果の観点から法テラス利用は致しません
  \url{https://kk2020-09.blogspot.com/2021/05/uncothe2nd\_20.html} 
\item
  2021年05月20日18時11分の登録:
  \いわぽん @yiwapon\法テラスの業務必携が情報公開制度で開示を受けられたんだとすると、契約弁護士には教えないという今の扱い自体全く意味が分からん。ていうかバカにし
  \url{https://kk2020-09.blogspot.com/2021/05/yiwapon\_20.html} 
\item
  2021年05月20日18時38分の登録:
  REGEXP:''深澤先生''/データベース登録済みツイートの検索:2021-05-13〜2021-05-19/2021年05月20日18時37分の記録:ユーザ・投稿:11/12件
  \url{https://kk2020-09.blogspot.com/2021/05/regexp2021-05-132021-05\_46.html} 
\item
  2021年05月20日18時40分の登録:
  kk2020-09\_ajx-all-user-mysql-REGEXP\_blogger.rb ``深澤先生''
  ``2021-05-13 18:36/2021-05-20 18:36'' \textbar{} tee s.txt
  \url{https://kk2020-09.blogspot.com/2021/05/kk2020-09ajx-all-user-mysql\_20.html} 
\item
  2021年05月20日18時43分の登録:
  %@harrier0516osk 向原総合法律事務所 弁護士向原%本人訴訟支援が流行っていた時代、非弁問題ってまだそんなにポピュラーではありませんでした。深澤先生の功績は非常に大きい。
  \url{https://kk2020-09.blogspot.com/2021/05/harrier0516osk\_20.html} 
\item
  2021年05月20日18時49分の登録: %@okinahimeji 櫻井光政%返信先:
  ¥\n@fukazawas¥\nさん¥\n本人訴訟支援を売りにしていた某行政書士の訴状を思い出します。
  \url{https://kk2020-09.blogspot.com/2021/05/okinahimeji-nfukazawasnn.html} 
\item
  2021年05月20日18時49分の登録:
  %@fukazawas 深澤諭史%自分で答弁書書くとか、非弁護士に答弁書書いてもらうとか、やめた方がいいのに。¥\nオウンゴールばかり目にしている。¥\n(・∀・)
  \url{https://kk2020-09.blogspot.com/2021/05/fukazawasnn.html} 
\item
  2021年05月20日18時55分の登録:
  \ふて寝べん @hirune\_b\「政治問題化」「政治利用」という言葉が好きな人に限って、それらの言葉の意味を説明できない件。
  \url{https://kk2020-09.blogspot.com/2021/05/hiruneb\_20.html} 
\item
  2021年05月20日18時55分の登録:
  %@hirune\_b ふて寝べん%深澤先生、しばらくツイートが停止しているけれど大丈夫なのでしょうか?¥\n大丈夫なら良いのですが。
  \url{https://kk2020-09.blogspot.com/2021/05/hirunebn\_20.html} 
\item
  2021年05月20日18時55分の登録:
  %@hirune\_b ふて寝べん%深澤先生、しばらくツイートが停止しているけれど大丈夫なのでしょうか?¥\n大丈夫なら良いのですが。
  \url{https://kk2020-09.blogspot.com/2021/05/hirunebn.html} 
\item
  2021年05月20日18時56分の登録:
  REGEXP:''オウンゴール''/データベース登録済みツイートの検索:2011-07-07〜2021-05-16/2021年05月20日18時52分の記録:ユーザ・投稿:95/190件
  \url{https://kk2020-09.blogspot.com/2021/05/regexp2011-07-072021-05.html} 
\item
  2021年05月20日20時01分の登録:
  REGEXP:''深澤弁護士''/データベース登録済みツイートの検索:2021-05-11〜2021-05-12/2021年05月20日20時01分の記録:ユーザ・投稿:1/6件
  \url{https://kk2020-09.blogspot.com/2021/05/regexp2021-05-112021-05-1220210520200116.html} 
\item
  2021年05月20日20時04分の登録: 2021年05月20日20時03分の実行記録:
  twitterAPI-search-lawList-mydql-add.rb ``深澤弁護士''
  \url{https://kk2020-09.blogspot.com/2021/05/202105202003-twitterapi-search-lawlist.html} 
\item
  2021年05月20日20時09分の登録: \都 行志/Miyako
  Koji @Miyako\_Koji\ガッキー結婚か。熱狂的ガッキーファンの友人のメンタル面が心配やわ\ldots{}
  \url{https://kk2020-09.blogspot.com/2021/05/miyako-kojimiyakokoji\_20.html} 
\item
  2021年05月20日20時11分の登録: \都 行志/Miyako
  Koji @Miyako\_Koji\群馬の事務所で雇っていた事務員さん2名は本当に優秀だったので、事務所移転の際に事務員さんに辞めてもらわざるを
  \url{https://kk2020-09.blogspot.com/2021/05/miyako-kojimiyakokoji2.html} 
\item
  2021年05月20日20時11分の登録: \都 行志/Miyako
  Koji @Miyako\_Koji\地方だと、採用面での法律事務所のブランド、競争力がまだまだあって、「えっ、うちなんかでいいんですか?」と言っ
  \url{https://kk2020-09.blogspot.com/2021/05/miyako-kojimiyakokoji\_41.html} 
\item
  2021年05月20日20時12分の登録:
  \?鳩屋? @haya\_rt\パンパンのわたしさんにはパンが一つでは足りないのである
  \url{https://kk2020-09.blogspot.com/2021/05/hayart\_20.html} 
\item
  2021年05月20日20時12分の登録: \野田隼人 Atty. NODA
  Hayato @nodahayato\とりあえず「パンパン」の隠語としての意味を調べてから一度、再考していただきたい。
  \url{https://kk2020-09.blogspot.com/2021/05/atty-noda-hayatonodahayato\_20.html} 
\item
  2021年05月20日20時14分の登録:
  \井垣孝之 @igaki\朝から大谷翔平選手の試合を見てますが、メジャーの観戦者はもうほぼ誰もマスクしてないですね。チームはマドン監督と一部のコーチと通訳の水原さんくらい
  \url{https://kk2020-09.blogspot.com/2021/05/igaki\_20.html} 
\item
  2021年05月20日20時15分の登録:
  \大村真司@攻めと実績の弁護士\&中小企業診断士in広島 @Ohmura\_LAW\日本の刑事司法は50年遅れてる、というより昭和の時代から全然進歩していない(だからむしろ差
  \url{https://kk2020-09.blogspot.com/2021/05/blog-post\_11.html} 
\item
  2021年05月20日20時15分の登録:
  \大村真司@攻めと実績の弁護士\&中小企業診断士in広島 @Ohmura\_LAW\BSではぐれ刑事純情派をやってたのに気付いて、懐かしくてみたんだが・・・人情部分はともかく、見
  \url{https://kk2020-09.blogspot.com/2021/05/blog-post\_20.html} 
\item
  2021年05月20日20時20分の登録:
  @Ohmura\_LAW(大村真司@攻めと実績の弁護士&中小企業診断士in広島)のツイート ''.*'' 3223/3223:2013-10-24\_1139〜2021-05-20\_1931 2021年05月20日20時20分の記録
  \url{https://kk2020-09.blogspot.com/2021/05/ohmuralawin322332232013-10-2411392021.html} 
\item
  2021年05月20日20時21分の登録:
  @bubblinkace(あわわ)のツイート ''.*'' 3225/3225:2020-12-11\_2035〜2021-05-20\_2012 2021年05月20日20時21分の記録
  \url{https://kk2020-09.blogspot.com/2021/05/bubblinkace322532252020-12-1120352021.html} 
\item
  2021年05月20日20時22分の登録:
  \HRK @hKodama\「自由と正義」の巻頭の『ひと筆』。個人的に最高に自由で傑作だと思うのは、私淑する先生による2007年4月号の記事なので、みんな機会があったら読
  \url{https://kk2020-09.blogspot.com/2021/05/hrkhkodama20074.html} 
\item
  2021年05月20日20時23分の登録:
  @hKodama(HRK)のツイート ''.*'' 3225/3225:2020-09-10\_2335〜2021-05-20\_1912 2021年05月20日20時23分の記録
  \url{https://kk2020-09.blogspot.com/2021/05/hkodamahrk322532252020-09-1023352021-05.html} 
\item
  2021年05月20日20時26分の登録:
  \そらまめ @sollamame\当事者/証人尋問予定者には、事前に尋問手続に関するわかり易いメモを渡しておくと良い。当職は圓道先生の著書にあった「証人尋問心得」の魔改造
  \url{https://kk2020-09.blogspot.com/2021/05/sollamame\_20.html} 
\item
  2021年05月20日20時27分の登録:
  %@1961kumachin くまちん(弁護士中村元弥)%7月30日の日弁連ライブ実務研修「法律事務所向け情報セキュリティ対策」は、元警察官でサイバー関係を担当していた弁護士が講師だそうな
  \url{https://kk2020-09.blogspot.com/2021/05/1961kumachin30.html} 
\item
  2021年05月20日20時28分の登録: \弁護士
  中原潤一 @lawyernakahara\こちらの件,本日罪名が2段階落ちた上で,不起訴。うん。知ってた。こうなることは5月3日から知ってた。こうなるってことは
  \url{https://kk2020-09.blogspot.com/2021/05/lawyernakahara\_20.html} 
\item
  2021年05月20日20時29分の登録:
  \くまちん(弁護士中村元弥) @1961kumachin\確かに探しにくいけど山中ブログに載っているな。もっともこれは改訂前のもので、今年改訂されて本文だけで260頁に増
  \url{https://kk2020-09.blogspot.com/2021/05/1961kumachin\_20.html} 
\item
  2021年05月20日20時35分の登録:
  \坂野真一 @sakanosi\乗っていた京阪電車が人身事故の影響で、遅れた。20時26分の事故だと車内放送で言っていた。20時10分に私は乗ったので、その時、事故に遭っ
  \url{https://kk2020-09.blogspot.com/2021/05/sakanosi20262010.html} 
\item
  2021年05月20日21時10分の登録:
  REGEXP:''坂野真一''/データベース登録済みツイートの検索:2016-03-16〜2021-05-20/2021年05月20日21時10分の記録:ユーザ・投稿:15/19件
  \url{https://kk2020-09.blogspot.com/2021/05/regexp2016-03-162021-05.html} 
\item
  2021年05月20日21時13分の登録:
  REGEXP:''@sakanosi''/データベース登録済みツイートの検索:2012-02-27〜2021-05-20/2021年05月20日21時12分の記録:ユーザ・投稿:24/71件
  \url{https://kk2020-09.blogspot.com/2021/05/regexpsakanosi2012-02-272021-05.html} 
\item
  2021年05月20日21時18分の登録:
  \小倉秀夫 @chosakukenho\NHKニュース等での犯罪報道がなされているときに、そのれを前提とするコメントをツイートするにあたって、独自取材しなければ相当性が認
  \url{https://kk2020-09.blogspot.com/2021/05/chosakukenhonhk.html} 
\item
  2021年05月20日21時23分の登録: \山口貴士
  aka無駄に感じが悪いヤマベン @otakulawyer\明らかに伊藤和子弁護士の言葉が過ぎた事案について大弁護団を組み、学者を何人も巻き込んで最高裁までよく
  \url{https://kk2020-09.blogspot.com/2021/05/akaotakulawyer\_20.html} 
\item
  2021年05月20日21時36分の登録: \弁護士
  伊藤祐貴@台東区・上野御徒町 @itoyukilaw\「不適切な保育」初の全国調査
  340件以上確認 厚労省対策検討 \textbar{} NHKニュース
  \url{https://kk2020-09.blogspot.com/2021/05/itoyukilaw-340-nhk.html} 
\item
  2021年05月21日04時28分の登録:
  ツイートの記録資料:\法務検察・石川県警察宛\/小倉秀夫(@chosakukenho)/''2021年05月20日'':8件
  \url{https://kk2020-09.blogspot.com/2021/05/chosakukenho202105208.html} 
\item
  2021年05月21日04時28分の登録:
  ツイートの記録資料:\法務検察・石川県警察宛\/モトケン(@motoken\_tw)/''2021年05月20日'':16件
  \url{https://kk2020-09.blogspot.com/2021/05/motokentw2021052016.html} 
\item
  2021年05月21日04時29分の登録:
  2021-05-20の投稿一覧\検察・石川県警察宛記録資料\奉納\危険生物・弁護士脳汚染除去装置\金沢地方検察庁御中:48件
  \url{https://kk2020-09.blogspot.com/2021/05/2021-05-2048.html} 
\item
  2021年05月21日04時29分の登録:
  %@toshifumi\_Ando 安藤 俊文%2021-05-20の投稿一覧\検察・石川県警察宛記録資料\奉納\危険生物・弁護士脳汚染除去装置\金沢地方検察庁御中:48件¥\n
  \url{https://kk2020-09.blogspot.com/2021/05/toshifumiando2021-05-2048n.html} 
\item
  2021年05月21日04時42分の登録:
  \安藤 俊文 @toshifumi\_Ando\二階氏側近、1億5千万円に「根堀り葉堀り踏み込むな」(朝日新聞デジタル)
  \url{https://kk2020-09.blogspot.com/2021/05/toshifumiando15.html} 
\item
  2021年05月21日04時42分の登録:
  \安藤 俊文 @toshifumi\_Ando\え?じゃあ誰が?元選対委員長の甘利氏「1ミクロンも関わらず」 1・5億円提供(産経新聞)
  \url{https://kk2020-09.blogspot.com/2021/05/toshifumiando.html} 
\item
  2021年05月21日04時43分の登録:
  \郷原信郎【長いものには巻かれない・権力と戦う弁護士】 @nobuogohara\「大阪での医療崩壊、コロナ死者急増の原因は、維新橋下府政・市政での公立病院削減」という指
  \url{https://kk2020-09.blogspot.com/2021/05/nobuogohara\_21.html} 
\item
  2021年05月21日04時48分の登録:
  \教皇ノースライム @noooooooorth\助けを求められたら弁護士として色々できる事案であっても助けを求められなければそもそも活動することができない。
  \url{https://kk2020-09.blogspot.com/2021/05/noooooooorth\_21.html} 
\item
  2021年05月21日04時49分の登録:
  \教皇ノースライム @noooooooorth\木村響子さんの訴訟の件について、代理人として活動している弊所清水弁護士のコメントが日経に掲載されまあいた。
  \url{https://kk2020-09.blogspot.com/2021/05/noooooooorth\_35.html} 
\item
  2021年05月21日05時09分の登録:
  REGEXP:''木村響子さん''/データベース登録済みツイートの検索:2021-05-17〜2021-05-20/2021年05月21日05時08分の記録:ユーザ・投稿:13/21件
  \url{https://kk2020-09.blogspot.com/2021/05/regexp2021-05-172021-05\_21.html} 
\item
  2021年05月21日05時10分の登録:
  「判決」を@hirono\_hideki @kk\_hirono @s\_hironoで検索 11785件の該当 2021-05-21\_05:09の記録
  \url{https://kk2020-09.blogspot.com/2021/05/hironohidekikkhironoshirono117852021-05.html} 
\item
  2021年05月21日05時11分の登録:
  「保険金」を@hirono\_hideki @kk\_hirono @s\_hironoで検索 291件の該当 2021-05-21\_05:11の記録
  \url{https://kk2020-09.blogspot.com/2021/05/hironohidekikkhironoshirono2912021-05.html} 
\item
  2021年05月21日05時14分の登録:
  \教皇ノースライム @noooooooorth\新垣結衣と星野源が結婚を発表「互いに支え合い豊かな時間を積み重ねていけたら」【コメント全文】(オリコン)
  \url{https://kk2020-09.blogspot.com/2021/05/noooooooorth\_61.html} 
\item
  2021年05月21日05時15分の登録:
  \教皇ノースライム @noooooooorth\宗教って現代においても「救い」としては本当にすごいというか結局これしかないみたいなところがあると思うんですが、自分だけが救
  \url{https://kk2020-09.blogspot.com/2021/05/noooooooorth\_25.html} 
\item
  2021年05月21日05時16分の登録:
  \教皇ノースライム @noooooooorth\自ら助かろうとしていない人を助けることはできないんですが、本当に追い詰められている人は助けを求めることすらできなくなってい
  \url{https://kk2020-09.blogspot.com/2021/05/noooooooorth\_24.html} 
\item
  2021年05月21日05時26分の登録:
  \教皇ノースライム @noooooooorth\弊所武内による法律事務所アルシエンのコンセプト解説【パートナーを目指すかずっとイソ弁のままでいるか】を公開しました。アルシ
  \url{https://kk2020-09.blogspot.com/2021/05/noooooooorth\_10.html} 
\item
  2021年05月21日05時26分の登録:
  \井垣孝之 @igaki\まあこんなこと書いてはいますが、弁護士の場合、年収数百万円は誤差の範疇だし、環境によっては数年で余裕で年間1000万円以上の差はつくので、目先の
  \url{https://kk2020-09.blogspot.com/2021/05/igaki1000.html} 
\item
  2021年05月21日05時30分の登録:
  %@noooooooorth 教皇ノースライム%弊所武内による法律事務所アルシエンのコンセプト解説【しなやかな解決力の意味】を公開しました。アルシエンのキャッチフレーズの一つについて具体的にはどのような意味なのかを解説しています。
  \url{https://kk2020-09.blogspot.com/2021/05/noooooooorth\_91.html} 
\item
  2021年05月21日05時45分の登録:
  %@noooooooorth 教皇ノースライム%弁護士が「商売下手の職人」でいられるのも「高い参入障壁」と「弁護士法の各種規制」のお陰ですよ。
  \url{https://kk2020-09.blogspot.com/2021/05/noooooooorth\_20.html} 
\item
  2021年05月21日05時47分の登録:
  %@noooooooorth 教皇ノースライム%お客さんの「欲」を満たすのがビジネスの基本だけど自分自身が「欲」に飲まれてしまうといけない。
  \url{https://kk2020-09.blogspot.com/2021/05/noooooooorth\_47.html} 
\item
  2021年05月21日05時51分の登録:
  \井垣孝之 @igaki\・ブラックに働いてそこそこ稼ぎたい→四大・そこそこ働いて稼ぎを安定させたい→新興大手・ホワイトに働いて稼ぎを安定させたい→インハウス・バリバリ働
  \url{https://kk2020-09.blogspot.com/2021/05/igaki\_21.html} 
\item
  2021年05月21日05時57分の登録:
  %@noooooooorth 教皇ノースライム%700人ちょいミュートしてるんで全く見えていない人が結構いるのだが、それでもずーっと私にリプライを飛ばしているっぽい人が何
  \url{https://kk2020-09.blogspot.com/2021/05/noooooooorth700.html} 
\item
  2021年05月21日06時01分の登録:
  「@noooooooorth」を@hirono\_hideki @kk\_hirono @s\_hironoで検索 1049件の該当 2021-05-21\_06:01の記録
  \url{https://kk2020-09.blogspot.com/2021/05/noooooooorthhironohidekikkhironoshirono.html} 
\item
  2021年05月21日06時40分の登録:
  2021-05-21\_北周士弁護士のTwitterタイムライン
  \url{https://kk2020-09.blogspot.com/2021/05/2021-05-21twitter.html} 
\item
  2021年05月21日10時23分の登録:
  \小倉秀夫 @chosakukenho\被告がガッキーだったとしても同じように論評するのだろうか。
  \url{https://kk2020-09.blogspot.com/2021/05/chosakukenho\_29.html} 
\item
  2021年05月21日10時24分の登録:
  \サイ太 @uwaaaa\こうやって伊藤和子氏の判決を論評すると「女性だから」とか言われるんですけど,この件で仮に被告がおぐりんだったとしても(本件では被告代理人だったみ
  \url{https://kk2020-09.blogspot.com/2021/05/uwaaaa\_21.html} 
\end{itemize}

 F12でツイートしながら頭で数えていた数とは10以上違いがあるのですが,xsel
-b\textbar wc
-l の結果は78となっていました。このコマンドはクリップボードの中にあるテキストの行数をカウントしたものです。

 2021-05-21\_北周士弁護士のTwitterタイムライン
\url{https://kk2020-09.blogspot.com/2021/05/2021-05-21twitter.html} 
がスクリーンショットの画像ファイルをまとめた記事になります。

 Bloggerのブログへの投稿はほとんどAPIを使ってプログラムで処理しているのですが,上記の画像ファイルのまとめは,ブラウザで行っています。いったん編集画面で画像ファイルを読み込み,URLを抜き出してHTMLを編集しコピペしたHTMLのテキストを編集画面に貼り付けています。

 LANG=c ls -l `which
photo-blogger-souce-make.rb というコマンドを実行すると,May 14
22:24 となっていました。ppコマンドと一緒に使っていたものを再編集したのですが,その最終更新時刻が5月14日22時24分という意味です。

 ppコマンドでは,同時に画像ファイル・写真ファイルをデータベースに登録し,ブログの記事の写真のある場所へアンカー付きのURLも登録していました。なので次のような方法ですぐに写真を探せます。

\begin{lstlisting}
py37_env ❯ pp -p|grep 木梨松嗣
\end{lstlisting}

\begin{itemize}
\tightlist
\item
  2018-10-02-223700\_木梨松嗣 - Twitter検索.jpg
  \url{http://hirono2014sk.blogspot.com/2018/10/2018100410332018-09-25-1743362018-10-03.html\#20181002223700} 
\item
  2018-10-02-223818\_木梨松嗣 - Google 検索.jpg
  \url{http://hirono2014sk.blogspot.com/2018/10/2018100410332018-09-25-1743362018-10-03.html\#20181002223818} 
\item
  2018-10-02-224142\_木梨松嗣 - Google 検索.jpg
  \url{http://hirono2014sk.blogspot.com/2018/10/2018100410332018-09-25-1743362018-10-03.html\#20181002224142} 
\item
  2018-10-02-224208\_木梨松嗣 - Google 検索.jpg
  \url{http://hirono2014sk.blogspot.com/2018/10/2018100410332018-09-25-1743362018-10-03.html\#20181002224208} 
\item
  2018-10-02\_204711_テレビの画面・金沢市長選立候補予定者説明会 金沢市選挙管理委員会 木梨松嗣委員長.jpg
  \url{http://hirono2014sk.blogspot.com/2018/10/2018100806342018-09-242151552018-10.html\#20181002204711} 
\item
  2018-11-13\_115034_テレビの画面・木梨松嗣委員長 「市民の期待に応え''地域経済力''をうかんなく発揮してほしい」 山野之義金沢市長.jpg
  \url{http://hirono2014sk.blogspot.com/2018/11/2018111407582018-11-120850182018-11.html\#20181113115034} 
\item
  2018-11-13\_115035_テレビの画面・木梨松嗣委員長 「市民の期待に応え''地域経済力''をうかんなく発揮してほしい」 山野之義金沢市長.jpg
  \url{http://hirono2014sk.blogspot.com/2018/11/2018111407582018-11-120850182018-11.html\#20181113115035} 
\item
  2018-11-13\_115042_テレビの画面・木梨松嗣委員長 「市民の期待に応え''地域経済力''をうかんなく発揮してほしい」 山野之義金沢市長.jpg
  \url{http://hirono2014sk.blogspot.com/2018/11/2018111407582018-11-120850182018-11.html\#20181113115042} 
\item
  2018-11-13\_115055_テレビの画面・木梨松嗣委員長 「市民の期待に応え''地域経済力''をうかんなく発揮してほしい」 山野之義金沢市長.jpg
  \url{http://hirono2014sk.blogspot.com/2018/11/2018111407582018-11-120850182018-11.html\#20181113115055} 
\item
  2019-04-23\_115250_テレビの画面・任期 5月2日からの4年間 木梨松嗣弁護士.jpg
  \url{http://hirono2014sk.blogspot.com/2019/04/2019043016552019-04-212204422019-04.html\#20190423115250} 
\item
  2019-04-23\_205134_テレビの画面・金沢市議会議員選挙 38人に当選証書 木梨松嗣弁護士.jpg
  \url{http://hirono2014sk.blogspot.com/2019/04/2019043016552019-04-212204422019-04.html\#20190423205134} 
\item
  2019-04-23\_205149_テレビの画面・金沢市議会議員選挙 38人に当選証書 木梨松嗣弁護士.jpg
  \url{http://hirono2014sk.blogspot.com/2019/04/2019043016552019-04-212204422019-04.html\#20190423205149} 
\item
  2019-04-23\_205150_テレビの画面・金沢市議会議員選挙 38人に当選証書 木梨松嗣弁護士.jpg
  \url{http://hirono2014sk.blogspot.com/2019/04/2019043016552019-04-212204422019-04.html\#20190423205150} 
\item
  2019-04-23\_205152_テレビの画面・金沢市議会議員選挙 38人に当選証書 木梨松嗣弁護士.jpg
  \url{http://hirono2014sk.blogspot.com/2019/04/2019043016552019-04-212204422019-04.html\#20190423205152} 
\item
  2019-04-23\_205153_テレビの画面・金沢市議会議員選挙 38人に当選証書 金沢市選挙管理委員会 木梨松嗣委員長.jpg
  \url{http://hirono2014sk.blogspot.com/2019/04/2019043016552019-04-212204422019-04.html\#20190423205153} 
\item
  2019-04-23\_205156_テレビの画面・金沢市議会議員選挙 38人に当選証書 木梨松嗣弁護士.jpg
  \url{http://hirono2014sk.blogspot.com/2019/04/2019043016552019-04-212204422019-04.html\#20190423205156} 
\item
  2019-04-23\_205159_テレビの画面・金沢市議会議員選挙 38人に当選証書 木梨松嗣弁護士.jpg
  \url{http://hirono2014sk.blogspot.com/2019/04/2019043016552019-04-212204422019-04.html\#20190423205159} 
\item
  2019-05-10-123149\_小倉秀夫さんのツイート: ''私には関係のないことです。RT @hirono\_hideki: 。@Hideo\_Ogura 一つ言って起きますが、私が木梨松嗣.jpg
  \url{http://hirono2014sk.blogspot.com/2019/05/2019051323432019-05-08-1547222019-05-13.html\#20190510123149} 
\end{itemize}

 作業効率の威力を感じました。今回の,北周士弁護士のタイムラインのスクリーンショットのまとめは,これまでと大きな違いが1つあって,Bloggerへの画像ファイルのアップロードではなく,Googleフォトに作成したアルバムの画像ファイルを使いました。

 もともとppコマンドを使わなくなったのは,Bloggerのブログへの画像・写真ファイルのアップロードが,Googleの1つのアルバムに紐付けられているらしく,アップロード時にそのアルバムの上限に達したというエラーが出たからです。

 今朝は4時20分過ぎに目が覚めて,1時間ほど起きていたつもりだったのですが,2021年05月21日06時40分の登録:
2021-05-21\_北周士弁護士のTwitterタイムライン,となっていたので7時前後までは起きていたようです。

 このところ早い時間に目が覚めてそのままずっと起きていることが多かったのですが,一息ついたら眠くなり,そのまま寝ていて起きたのが10時過ぎでした。こういう遅めの時間に起きるのも久しぶりでしたが,時間の流れが早く感じます。

 やはりデータベースへの登録はしておいた方がよいと思いました。のちのちの作業効率が断然違うのは明らかです。

\begin{itemize}
\tightlist
\item
  〈〈〈 2021/05/21 11:18:51 Linux Emacs: 〈〈〈
\end{itemize}

\hypertarget{twitterux3067ux30d6ux30edux30c3ux30afux3055ux308cux7d9aux3051ux3066ux3044ux308bux3053ux3068ux3092ux78baux8a8dux3057ux305fux61b2ux6cd5ux30acux30fcux30ebux306eux8457ux8005ux5927ux5cf6ux7fa9ux5247ux5f01ux8b77ux58ebux306eux30bfux30a4ux30e0ux30e9ux30a4ux30f3ux3067ux898bux304bux3051ux305fux90fdux306fux5f01ux8b77ux58ebux3092ux3064ux3051ux305aux3068ux3044ux3046ux30cbux30e5ux30fcux30b9}{%
\paragraph{Twitterでブロックされ続けていることを確認した憲法ガールの著者,大島義則弁護士のタイムラインで見かけた「都は、弁護士をつけず」というニュース}\label{twitterux3067ux30d6ux30edux30c3ux30afux3055ux308cux7d9aux3051ux3066ux3044ux308bux3053ux3068ux3092ux78baux8a8dux3057ux305fux61b2ux6cd5ux30acux30fcux30ebux306eux8457ux8005ux5927ux5cf6ux7fa9ux5247ux5f01ux8b77ux58ebux306eux30bfux30a4ux30e0ux30e9ux30a4ux30f3ux3067ux898bux304bux3051ux305fux90fdux306fux5f01ux8b77ux58ebux3092ux3064ux3051ux305aux3068ux3044ux3046ux30cbux30e5ux30fcux30b9}}

\begin{itemize}
\tightlist
\item
  〉〉〉 Linux Emacs: 2021/05/24 10:31:43 〉〉〉
\end{itemize}

:CATEGORIES: @kanazawabengosi \#金沢弁護士会 @JFBAsns
\#日本弁護士連合会(日弁連) \#法務省
@MOJ\_HOUMU,金沢西警察署,石川県警察,憲法

\begin{itemize}
\tightlist
\item
  TW kk\_hirono(刑事告発・非常上告_金沢地方検察庁御中) 日時:
  2021-05-21 11:18 URL:
  \url{https://twitter.com/kk\_hirono/status/1395564741167222790} 
  \textgreater{} - 〈〈〈 2021/05/21 11:18:51 Linux Emacs: 〈〈〈
\end{itemize}

 上記の5月21日11時18分のツイート以来の再開になります。この間,家の中でとても大きな発見があり,デジカメでの写真撮影からブログ記事としての投稿まで作業を続けていました。ようやく一息ついたところで目にしたのが,このタイトルにある大島義則弁護士のタイムラインです。

 大島義則氏が弁護士であることを確認したのですが,憲法と行政法の学者というイメージが強くあり,弁護士としては情報を見かけることも少なかったように思うのですが,いざ確認の為調べてみると,所属の法律事務所が長谷川法律事務所となっていました。

〉〉〉 kk\_hironoのリツイート 〉〉〉

\begin{itemize}
\tightlist
\item
  RT
  kk\_hirono(刑事告発・非常上告_金沢地方検察庁御中)|babel0101(anonymity)
  日時:2021-05-24 10:38/2021/05/24 01:21 URL:
  \url{https://twitter.com/kk\_hirono/status/1396641829840297986} 
  \url{https://twitter.com/babel0101/status/1396501511404875776} 
  \textgreater{} ん、乗っ取られてるか、あれ。
\end{itemize}

〉〉〉 kk\_hironoのリツイート 〉〉〉

\begin{itemize}
\tightlist
\item
  RT
  kk\_hirono(刑事告発・非常上告_金沢地方検察庁御中)|mizuno\_law(水野泰孝
  Yasutaka Mizuno) 日時:2021-05-24 10:40/2021/05/23 18:38 URL:
  \url{https://twitter.com/kk\_hirono/status/1396642203716362245} 
  \url{https://twitter.com/mizuno\_law/status/1396400082279309312} 
  \textgreater{}
  行訴法には、平成16年改正で創設された後、ほぼ死文化している釈明処分の特則の規定(23条の2)があります。私が知る限り、これまで数件の税務訴訟で使われただけ。この制度はうまく使いこなせれば原告側の大きな武器になる可能性があり、日弁連の会員サイトでも活用のための書式等を用意しています。
\end{itemize}

〉〉〉 kk\_hironoのリツイート 〉〉〉

\begin{itemize}
\tightlist
\item
  RT
  kk\_hirono(刑事告発・非常上告_金沢地方検察庁御中)|simuken2016(健🍣 弁護士兼マンガ家)
  日時:2021-05-24 10:40/2021/05/23 18:13 URL:
  \url{https://twitter.com/kk\_hirono/status/1396642285094199296} 
  \url{https://twitter.com/simuken2016/status/1396393718156955649} 
  \textgreater{}
  そういえば先生より受験生時代いきなり呼び出されて寿司ご馳走になり帰り際に「そういえば今日ビール券もらったのよ。ビールを家じゃ飲まないから使って」っと頂いたこともあったな🥸
   ブル弁凄い・・・イケメン・・・ まだそのレベルには至っていない🕵️‍♂️💸
\end{itemize}

〉〉〉 kk\_hironoのリツイート 〉〉〉

\begin{itemize}
\tightlist
\item
  RT
  kk\_hirono(刑事告発・非常上告_金沢地方検察庁御中)|babel0101(anonymity)
  日時:2021-05-24 10:41/2021/05/23 18:09 URL:
  \url{https://twitter.com/kk\_hirono/status/1396642407307902977} 
  \url{https://twitter.com/babel0101/status/1396392785155031040} 
  \textgreater{}
  処分違憲否定論をとるかはともかく、実務的な行政法規適用のプラクティスはうまく説明する見解だと思います。
\end{itemize}

〉〉〉 kk\_hironoのリツイート 〉〉〉

\begin{itemize}
\tightlist
\item
  RT
  kk\_hirono(刑事告発・非常上告_金沢地方検察庁御中)|babel0101(anonymity)
  日時:2021-05-24 10:41/2021/05/23 18:07 URL:
  \url{https://twitter.com/kk\_hirono/status/1396642484055281664} 
  \url{https://twitter.com/babel0101/status/1396392357528948738} 
  \textgreater{} 石川健治=神橋一彦=土井真一=中川丈久「公法訴訟第 20
  回 〔座 談会〕『公法訴訟』論の可能性(1) ――連載終了にあたって」法学教室
  391 号 106 頁〔中川丈久発言部分〕。
\end{itemize}

〉〉〉 kk\_hironoのリツイート 〉〉〉

\begin{itemize}
\tightlist
\item
  RT
  kk\_hirono(刑事告発・非常上告_金沢地方検察庁御中)|babel0101(anonymity)
  日時:2021-05-24 10:41/2021/05/23 18:05 URL:
  \url{https://twitter.com/kk\_hirono/status/1396642527264972805} 
  \url{https://twitter.com/babel0101/status/1396391869899100170} 
  \textgreater{} 45Ⅱの憲法適合性の審査義務はないかぁ。
\end{itemize}

〉〉〉 kk\_hironoのリツイート 〉〉〉

\begin{itemize}
\tightlist
\item
  RT
  kk\_hirono(刑事告発・非常上告_金沢地方検察庁御中)|idleness\_venomy(venomy)
  日時:2021-05-24 10:41/2021/05/23 17:40 URL:
  \url{https://twitter.com/kk\_hirono/status/1396642573750468618} 
  \url{https://twitter.com/idleness\_venomy/status/1396385483853242368} 
  \textgreater{} \textgreater 都は、弁護士をつけず
  弁護士業界に対する挑戦。 @CoronaKaKensyo \#note
  \url{https://t.co/Np5q6UKvLh} 
\end{itemize}

〉〉〉 kk\_hironoのリツイート 〉〉〉

\begin{itemize}
\tightlist
\item
  RT
  kk\_hirono(刑事告発・非常上告_金沢地方検察庁御中)|tomok70270102(tomoya
  kotera(弁護士)) 日時:2021-05-24 10:42/2021/05/23 16:42 URL:
  \url{https://twitter.com/kk\_hirono/status/1396642770798841858} 
  \url{https://twitter.com/tomok70270102/status/1396370869467844616} 
  \textgreater{}
  友人が所属しているパートナー10名程度の事務所。売上1億超の人から1000万程度の人までいるが取り分は完全に平等で1人3000万程度になるらしい。
  それでもパートナー同士お互いの実力を認めているし儲からないけど社会的意義のある活動もあるので揉めないとのこと。凄い方々もいるものだなと。
\end{itemize}

〉〉〉 kk\_hironoのリツイート 〉〉〉

\begin{itemize}
\item
  RT
  kk\_hirono(刑事告発・非常上告_金沢地方検察庁御中)|ZeLo\_Nomura(野村諭/法律事務所ZeLo・外国法共同事業)
  日時:2021-05-24 10:43/2021/05/23 10:52 URL:
  \url{https://twitter.com/kk\_hirono/status/1396642919134625792} 
  \url{https://twitter.com/ZeLo\_Nomura/status/1396282750450114561} 
  \textgreater{}
  法律事務所の若い弁護士の潜在的地位というのは、カテゴリカルにも違うし、事務所によっても違いますね。古典的な個人事務所に近ければ、案件はボスが取ってきて、若手は案件に関われて学べて、固定の報酬をもらう。中期的に学んだスキルで、自分だけで案件回せるようになり自分のお客をとる。
\item
  〉〉〉 アカウント(@nodahayato)は,@kk\_hironoをブロックしています。リツイートできませんでした。
  〉〉〉 ¥\n ¥\n \url{https://t.co/4JsWKdXmub} 
\end{itemize}

※ @kk\_hironoのアカウントがブロックされ,リツイートに失敗したツイート

\begin{itemize}
\tightlist
\item
  TW nodahayato(野田隼人 Atty. NODA Hayato) 日時:2021/05/23 20:49:37
  URL: \url{https://twitter.com/nodahayato/status/1396433131742269445} 
  \textgreater{}
  設問について情報公開請求をすべきなのではないかしら・・・(大阪の方を眺めつつ
  \url{https://t.co/qMFXJ9w33R} 
\end{itemize}

 上記に大島義則弁護士のタイムラインにあったツイートをリツイートしました。奉納\さらば弁護士鉄道・泥棒神社の物語(@hirono\_hideki)ではブロックされている大島義則弁護士のTwitterアカントですが,刑事告発・非常上告_金沢地方検察庁御中(@kk\_hirono)ではされていませんでした。

 リストに新規登録のアカウントが2つありました。主に憲法や行政法について専門性の高いツイートとしてご紹介しました。今回の名宛人は,金沢西警察署と石川県警察を指定しました。憲法問題として参考にしていただきたいと思います。

\begin{quote}
《引用の始まり》
\end{quote}

\begin{quote}
その中で、松田典浩裁判長は都に対し、緊急事態宣言の終了3日前に時短命令を出した根拠や、時短命令の対象が32店舗(うちGDが26店舗)に絞り込まれた理由について、具体的に説明するよう求めた。 都は、弁護士をつけず、松下博之訟務担当部長ら4人の指定代理人が出廷した。 次回期日は7月9日に決定。その1週間前までに、東京都は改めて命令の正当性について主張書面を提出することとなった。 筆者(楊井)は、弁論期日で傍聴席から取材した。
\end{quote}

\begin{quote}
《引用の終わり》
\end{quote}

\begin{itemize}
\tightlist
\item
  東京地裁、都に時短命令を発した根拠の更なる説明求める グローバルダイニング提訴の第1回期日で|コロナ禍検証プロジェクト|note
  \url{https://note.com/verify\_corona\_ka/n/n0e31b9219fa8} 
\end{itemize}

 上記の引用のため記事を軽く読み直したのですが,「筆者(楊井)は、弁論期日で傍聴席から取材した。」という部分は,最初にざっと読んだとき気が付かなかったようです。

\begin{quote}
《引用の始まり》
\end{quote}

\begin{quote}
コロナ禍を多角的に調査・検証し、一日も早く正常な社会を取り戻すための戦略を考えます。ご支援いただきますよう、お願いいたします。
(事務局:NPO法人日本公共利益研究所/プロジェクト主宰:弁護士
楊井人文)3 フォロー422 フォロワー
\end{quote}

\begin{quote}
《引用の終わり》
\end{quote}

\begin{itemize}
\tightlist
\item
  コロナ禍検証プロジェクト|note \url{https://note.com/verify\_corona\_ka} 
\end{itemize}

 報道の記者のグループかと思ったのですが,プロジェクト主宰が楊井人文弁護士となっています。楊枝と同じ感じということはわかったのですが,「よう」以外の読み方がわかりません。どこかで見かけた名前のようにも思いますが,念の為確認しておきます。

\begin{itemize}
\tightlist
\item
  2021年05月24日10時57分の登録:
  「楊井人文」を@hirono\_hideki @kk\_hirono @s\_hironoで検索 11件の該当 2021-05-24\_10:57の記録
  \url{https://kk2020-09.blogspot.com/2021/05/hironohidekikkhironoshirono112021-05\_24.html} 
\end{itemize}

 いくつか該当があったのでまとめ記事にしました。

2015-07-10 22:17:18
``憲法学者の圧倒的多数が「違憲」と判断している中、長谷部氏らを"「違憲」派"と評することへの違和感はともかく
→【安保報道】朝毎東は「違憲」派学者重用、読産日は憲法学者に触れずー6月の新聞1面を分析(楊井人文)
-- Y!ニュース・・・
\url{https://hirono2015k.wordpress.com/2015/07/10/\%e6\%86\%b2\%e6\%b3\%95\%e5\%ad\%a6\%e8\%80\%85\%e3\%81\%ae\%e5\%9c\%a7\%e5\%80\%92\%e7\%9a\%84\%e5\%a4\%9a\%e6\%95\%b0\%e3\%81\%8c\%e3\%80\%8c\%e9\%81\%95\%e6\%86\%b2\%e3\%80\%8d\%e3\%81\%a8\%e5\%88\%a4\%e6\%96\%ad\%e3\%81\%97\%e3\%81\%a6\%e3\%81\%84\%e3\%82\%8b''} 
\url{https://twitter.com/s\_hirono/status/619495612153606144} 

 最初にあったのが2015年7月10日のツイートで,非常上告-最高検察庁御中\_ツイッター(@s\_hirono)のツイートですが,まだ,スクリーンショットや写真の専門化にする前だったのかもしれません。

2016-03-17 01:14:01 ``「35年前の強盗の記事、訂正します」
秋田魁新報の論説委員に経緯を聞く(楊井人文) - Yahoo!ニュース
\url{http://ow.ly/ZxMcY''} 
\url{https://twitter.com/hirono\_hideki/status/710137051069464576} 

 2件目になっています。ちょっと記憶にない内容なのですが,見出しを見ただけで気になる記事です。

\begin{itemize}
\tightlist
\item
  楊井人文の記事一覧 - 個人 - Yahoo!ニュース \url{https://t.co/q1zkeYUFzu} 
\end{itemize}

 記事を読み始める前に,プロフィールのようなリンクを開きました。これは見たことのある顔写真と思いましたが,ここ2,3年ではなく,それより前で,以来,見かけていないように思います。今年だけでもけっこうな数の記事が一覧になっていますが,見かける機会はなかったようです。

\begin{itemize}
\tightlist
\item
  楊井人文 - Wikipedia \url{https://t.co/poFH5iQ5gw}  楊井 人文(やない
  ひとふみ、1980年〈昭和55年〉 -
  )は、日本の弁護士(登録番号:38739、第一東京弁護士会)、ジャーナリスト。元産経新聞記者。かつて存在した一般社団法人日本報道検証機構では代表理事を務めていた。
\end{itemize}

 まず,名前の読み方が気になって調べてみました。Wikipediaが出てきましたが,楊井を「やない」と読むようです。楊貴妃と同じ漢字のようでもありますが,「やない」とは,どう頑張っても読めなかったと思います。

 ジャーナリストの肩書で元新聞記者という経歴のある弁護士で,思い出す弁護士がいますが,名前が思い出せなくなっています。ヤメ蚊こと日隅一雄弁護士(故人)でした。「やめか」の単語登録から変換されました。

\begin{quote}
《引用の始まり》
\end{quote}

\begin{quote}
弁護士になった後、2012年4月、マスコミ誤報検証・報道被害救済サイト「GoHoo」を立ち上げ、同年11月、一般社団法人日本報道検証機構を設立し、同代表理事に就任した。設立の理由として、2011年の東日本大震災発生以降、放射能汚染に対する危険性などについて、二極化する報道に危機感を抱いたことを挙げている[3]。

なお、2019年(令和元年)8月29日、寄付金が充分に集まらず運営継続も困難として、日本報道検証機構を解散した[4]。
\end{quote}

\begin{quote}
《引用の終わり》
\end{quote}

\begin{itemize}
\tightlist
\item
  楊井人文 -
  Wikipedia \url{https://ja.wikipedia.org/wiki/\%E6\%A5\%8A\%E4\%BA\%95\%E4\%BA\%BA\%E6\%96\%87n} 
\end{itemize}

 上記に引用をしましたが,「弁護士になった後、2012年4月、マスコミ誤報検証・報道被害救済サイト「GoHoo」を立ち上げ」とあります。続く,「寄付金が充分に集まらず運営継続も困難として、日本報道検証機構を解散した」という部分は,2,3ヶ月前に一度,読んだ記憶があります。

\begin{quote}
《引用の始まり》
\end{quote}

\begin{quote}
【GoHooトピックス3月16日】「正確でなかったため訂正させていただきます」ー秋田魁(あきたさきがけ)新報が3月4日、35年前に秋田市内で起きた強盗事件の記事を事実上訂正した。1面コラム「北斗星」の筆者が新人時代に取材して以来、被害者宅にあった現金が全て奪われたと思い込んでいたが、最近になってそうではなかったことが判明したという。日本新聞協会の綱領は「新聞は歴史の記録者」とうたっているが、実際に古い記事の誤りを訂正することはめったにない。筆者の論説委員長、相馬高道さん(58)は、日本報道検証機構の取材に応じ「コラムの趣旨は、たとえ遅くなったとしても知らないふりをしてはいけない、という戒めとすることにあります」と話し、事の経緯を明かした(コラム全文は秋田魁新報ウェブサイト参照)。
\end{quote}

\begin{quote}
《引用の終わり》
\end{quote}

\begin{itemize}
\tightlist
\item
  「35年前の強盗の記事、訂正します」
  秋田魁新報の論説委員に経緯を聞く(楊井人文) - 個人 -
  Yahoo!ニュース \url{https://news.yahoo.co.jp/byline/yanaihitofumi/20160316-00055463/n} 
\end{itemize}

 2016/3/16(水)
17:11という日付のある記事ですが,終わりの方になってようやく一度,読んでいるような感覚が出てきました。昭和56年で「当時、夫は84歳、妻は80歳」とあります。

 次のツイートが9番目として記録されていますが,その前が2019年12月18日,その後が2021年5月24日つまり今日ですが,脈絡がここでは確認できず単発の出現となっています。

2021-05-07 19:35:34 ``- 楊井人文 - Wikipedia \url{https://t.co/rZUuwFHlsM''} 
\url{https://twitter.com/hirono\_hideki/status/1390616287844311042} 

 その2019年12月18日のツイートですが,これも気になる内容の記事となっています。たぶんリンクの記事は読んでいると思いますが,よく見ると,野田隼人弁護士のツイートのまとめ記事として記録されていました。

2019-12-18 19:56:18 ``2019年12月18日13時57分の登録:
\弁護士 野田隼人 @nodahayato\FactCheck
「日本の無期懲役は一生刑務所ではなく、出所してくるのが通例」は本当か?(楊井人文)
- Y!ニュース
\url{http://hirono2014sk.blogspot.com/2019/12/nodahayatofactcheck-y.html''} 
\url{https://twitter.com/hirono\_hideki/status/1207253245896167425} 

\begin{itemize}
\tightlist
\item
  FactCheck
  「日本の無期懲役は一生刑務所ではなく、出所してくるのが通例」は本当か?(楊井人文)
  - 個人 - Yahoo!ニュース \url{https://t.co/MmqcFiSgtx}  楊井人文 \textbar{}
  弁護士 ¥\n 2019/12/15(日) 16:52
\end{itemize}

\begin{quote}
《引用の始まり》
\end{quote}

\begin{quote}
「無期懲役でも15年で仮釈放」は昔の話 ときどき「無期懲役でも15年で仮釈放」といった言説が世に出ることがある(例)。たしかに、かつて15年程度で仮釈放される時代もあったが、それは半世紀ほど前の話だ。

 法務省は「犯罪白書」で、無期刑の仮釈放(かつては「仮出獄」と呼んでいた)の運用状況を公表している。確認できた最も古いデータは1967〜69(昭和42~44)年であったが、半世紀前の当時は年平均90人前後の無期刑受刑者が仮釈放され、服役期間は14年以内または16年以内が大半だった。
\end{quote}

\begin{quote}
《引用の終わり》
\end{quote}

\begin{itemize}
\tightlist
\item
  FactCheck
  「日本の無期懲役は一生刑務所ではなく、出所してくるのが通例」は本当か?(楊井人文)
  - 個人 -
  Yahoo!ニュース \url{https://news.yahoo.co.jp/byline/yanaihitofumi/20191215-00155011/n} 
\end{itemize}

\begin{quote}
《引用の始まり》
\end{quote}

\begin{quote}
過去の犯罪白書を見ると、1980年代も年平均50人前後の大半が20年以内に仮釈放されていたが、平成の時代に入ると仮釈放の人数は激減し、服役20年以内の仮釈放は少数に転じたことがわかる(平成11年版犯罪白書)。

 大きな転機になったのが、有期懲役の上限が20年から30年に引き上げられた2005(平成17)年の刑法改正だ。有期刑受刑者より早く無期刑受刑者を仮釈放させるとバランスがおかしくなるため、服役25年以内の仮釈放者はゼロになった。仮釈放が全く行われない年もあった(平成25年版犯罪白書)。
\end{quote}

\begin{quote}
《引用の終わり》
\end{quote}

\begin{itemize}
\tightlist
\item
  FactCheck
  「日本の無期懲役は一生刑務所ではなく、出所してくるのが通例」は本当か?(楊井人文)
  - 個人 -
  Yahoo!ニュース \url{https://news.yahoo.co.jp/byline/yanaihitofumi/20191215-00155011/n} 
\end{itemize}

 「現在、無期懲役刑の受刑者が仮釈放される可能性があるのは刑務所等に収容されてから30年後であり、仮釈放許可率も低い。近年、刑務所内で死亡した人数は仮釈放を許可された受刑者の2倍以上であり、無期懲役刑の受刑者の多くが刑務所内で死亡している。」が結論とあります。

 私の記憶では,私が福井刑務所に服役していた平成6年から平成9年1月頃も,無期懲役は15年から20年で仮釈放になるという情報を見ていたように思います。この記事は前にも読んでいるはずですが,仮釈放の審査自体が収容から30年後というのは初めて目にしたような気がしました。

 この仮釈放制度というのは福井刑務所や金沢刑務所で身近に接した多くの受刑者の大きな関心事でしたが,現在とは事情も異なっているのかもしれません。仮面接,パロルの提出,本面接という経過でしたが,審査の始まりというの仮面接のことかと思われます。

 久しぶりに,福井刑務所で,この仮面接を受けたことを思い出したのですが,審査官のような担当者は,私の話を聞いてあきれ顔の様子でした。大きな広い部屋に大きな机があり,3人が座る裁判官の席にも似ていたと記憶にありますが,その部屋で二人っきりだったように思います。

 ついでに記述しておくと,金沢刑務所では仮面接の話もなかったですが,再犯刑務所でありながら,現役の暴力団関係者以外は仮釈放が多く,それというのも平成13年当時は,刑務所の過剰収容が社会問題としてピークの時期になっている状況でした。

 楊井人文弁護士を大島義則弁護士と同じ憲法学者の仲間なのかという勘違いの確認から始まったのですが,「無期懲役刑の受刑者の多くが刑務所内で死亡している。」に到達するとは思いもよりませんでした。

 twilog-serch-post
で「大島義則」と「憲法ガール」を投稿しましたが,以外に少ない結果でした。さきほど調べたところ憲法ガールは石川県の横断図書検索に蔵書がありました。前回の検索が「能登怪奇譚」だったのですが,意外なことに該当がなかったのです。

 市川寛弁護士のブログの記事を一つ読んだあと,ブログのヘッダのデザインになっているイラストで,雰囲気が似ているように思い出したのが金沢刑務所の官本で読んだ「能登怪奇譚」でした。調べたことはなかったですが,宇出津の図書館にもありそうな蔵書だと思っていました。

\begin{itemize}
\tightlist
\item
  〈〈〈 2021/05/24 11:56:20 Linux Emacs: 〈〈〈
\end{itemize}

\hypertarget{ux691cux5bdfux3092ux3053ux308cux3067ux3082ux304bux3068ux6fc0ux70c8ux306bux6279ux5224ux7f75ux5012ux3059ux308bux518dux5be9ux6cd5ux6539ux6b63ux3092ux3081ux3056ux3059ux5e02ux6c11ux306eux4f1aux306eux5143ux691cux4e8bux3067ux3082ux3042ux308bux5e02ux5dddux5bdbux5f01ux8b77ux58ebux3068ux91d1ux6ca2ux5211ux52d9ux6240ux306eux5b98ux672cux3067ux8aadux3093ux3060ux80fdux767bux602aux5947ux8b5a}{%
\paragraph{検察をこれでもかと激烈に批判,罵倒する「再審法改正をめざす市民の会」の元検事でもある市川寛弁護士と,金沢刑務所の官本で読んだ「能登怪奇譚」}\label{ux691cux5bdfux3092ux3053ux308cux3067ux3082ux304bux3068ux6fc0ux70c8ux306bux6279ux5224ux7f75ux5012ux3059ux308bux518dux5be9ux6cd5ux6539ux6b63ux3092ux3081ux3056ux3059ux5e02ux6c11ux306eux4f1aux306eux5143ux691cux4e8bux3067ux3082ux3042ux308bux5e02ux5dddux5bdbux5f01ux8b77ux58ebux3068ux91d1ux6ca2ux5211ux52d9ux6240ux306eux5b98ux672cux3067ux8aadux3093ux3060ux80fdux767bux602aux5947ux8b5a}}

\begin{itemize}
\tightlist
\item
  〉〉〉 Linux Emacs: 2021/05/24 13:44:42 〉〉〉
\end{itemize}

:CATEGORIES: @kanazawabengosi \#金沢弁護士会 @JFBAsns
\#日本弁護士連合会(日弁連) \#法務省 @MOJ\_HOUMU \#市川寛弁護士 \#再審
\#金沢刑務所

\begin{itemize}
\tightlist
\item
  Rain Saishin \url{https://rain-saishin.org/} 
\end{itemize}

 ホームページのページタイトルというのはプログラムの実行として取得できるのですが,「Rain
Saishin」と見知らぬものとなっていました。今開いたページの最上部に,傘のようなマークのロゴに気がついたのですが,その横には「再審法改正をめざす市民の会」とあります。

 Googleの検索で調べて出てきたページですが,目的は「再審法改正をめざす市民の会」における市川寛弁護士の役職や肩書を確認することでした。結果は,プログラムという画像に運営委員,弁護士,元検察官とありました。

 私と再審請求の関わりというのも長いものがありました。最後の再審請求は平成15年(2003年)ですが,再審請求をするつもりで金沢地方検察庁に電話連絡をし書記官のトップのような人と話をしたと最後に記憶にあるのが,2011年の東日本大震災の当日でした。

 首席書記官であったのか忘れましたが,電話口でそのように呼び出す声が聞こえました。東日本大震災のときは違った人だったのかもしれないですが,金沢地方裁判所から折返しの電話を待っている間に発生したのが,東日本大震災と大津波でした。

 あまり細かく書くと際限がなくなってしまうのですが,「再審法改正をめざす市民の会」の2周年というYouTube動画を視聴し,そのあとに市川寛弁護士のTwitterのタイムラインを読んでいたところ,ブログの記事に行き着きました。

 YouTubeの動画で見ていた一人が市川寛弁護士だったのですが,それまでにネットで見ていた写真とはずいぶんイメージが違っていました。ネットよりさらに強く感じたのは,検察に対する批判精神と攻撃力でした。先程,数日ぶりにku3のまとめ記事を作成していますので,そちらで調べていきます。

\begin{itemize}
\item
  奉納\危険生物・弁護士脳汚染除去装置\金沢地方検察庁御中\_2020:
  @hirono\_hideki(奉納\さらば弁護士鉄道・泥棒神社の物語)のツイート ''.*'' 3228/3228:2021-05-03\_2309〜2021-05-24\_1156 2021年05月24日12時05分の記録
  \url{https://kk2020-09.blogspot.com/2021/05/hironohideki322832282021-05-0323092021.html} 
\item
  (2035/3228)
  @hirono\_hideki(奉納\さらば弁護士鉄道・泥棒神社の物語)のツイート ''.*'' 3228/3228:2021-05-03\_2309〜2021-05-24\_1156
  2021年05月24日12時05分の記録\\
  TW hirono\_hideki(奉納\さらば弁護士鉄道・泥棒神社の物語) 日時:
  2021-05-09 16:36 URL:
  \url{https://twitter.com/hirono\_hideki/status/1391295995934253061} 
  \textgreater{} - 1354:2021-05-09\_16:36:20 \#告発状 \#\#\#\#
  「弁護人には守秘義務があるので、警察のように情報を垂れ流すことができません。マスコミはそれを知りつつ、警察発表に立脚しての犯罪報道」という市川寛弁護士のツイート
  \url{https://t.co/957zdMxRR6} 
\end{itemize}

 上記のツイートの記録は,「市川寛」のページ内検索で6/13となっています。記事の紹介ツイートのようなものですが,その記事の記録が「2021年05月24日12時05分の記録」となっています。これは検索の目的としては行き過ぎなのですが,永久保存版のようなツイートの1つです。

 次の2つのツイートが,探していた目的の位置になると思います。

\begin{itemize}
\item
  (422/3228)
  @hirono\_hideki(奉納\さらば弁護士鉄道・泥棒神社の物語)のツイート ''.*'' 3228/3228:2021-05-03\_2309〜2021-05-24\_1156
  2021年05月24日12時05分の記録\\
  TW hirono\_hideki(奉納\さらば弁護士鉄道・泥棒神社の物語) 日時:
  2021-05-22 15:54 URL:
  \url{https://twitter.com/hirono\_hideki/status/1395996529593253891} 
  \textgreater{} - 検事はなぜ証拠を見せないのか -
  理屈でないところからの試論 \textbar{}
  検事失格  弁護士 市川寛のブログ  \url{https://t.co/i2NIuBnelh} 
\item
  (423/3228)
  @hirono\_hideki(奉納\さらば弁護士鉄道・泥棒神社の物語)のツイート ''.*'' 3228/3228:2021-05-03\_2309〜2021-05-24\_1156
  2021年05月24日12時05分の記録\\
  TW hirono\_hideki(奉納\さらば弁護士鉄道・泥棒神社の物語) 日時:
  2021-05-22 15:51 URL:
  \url{https://twitter.com/hirono\_hideki/status/1395995749813391366} 
  \textgreater{} 2021年05月22日15時51分の実行記録:
  twitterAPI-search-lawList-mydql-add.rb ``市川寛'' ツイート数:6/2424
  リツイート数:0/2424 トータル:32\\
  \textgreater{} ``市川寛''の該当: hirono\_hideki 2/0件 kk\_hirono
  1/0件 s\_hirono 3/0件
\end{itemize}

 テキストのみのツイートだと読みづらく勘違いすることがあるのですが,先程の「「2021年05月24日12時05分の記録」となっています。これは検索の目的としては行き過ぎなのですが,」は訂正しなければならず,5月24日というのは本日の日付でした。

 「日時: 2021-05-22
15:54」というのが探していた目的の時間的位置になると思います。まだまる2日まで1時間以上ありますが,一昨日のことでした。ひとつ上が,能登怪奇譚のツイートとなっています。

\begin{itemize}
\tightlist
\item
  (421/3228)
  @hirono\_hideki(奉納\さらば弁護士鉄道・泥棒神社の物語)のツイート ''.*'' 3228/3228:2021-05-03\_2309〜2021-05-24\_1156
  2021年05月24日12時05分の記録\\
  TW hirono\_hideki(奉納\さらば弁護士鉄道・泥棒神社の物語) 日時:
  2021-05-22 15:54 URL:
  \url{https://twitter.com/hirono\_hideki/status/1395996571095900161} 
  \textgreater{} - 能登怪異譚 (集英社文庫) \textbar{} 半村 良, 村上 豊
  \textbar 本 \textbar{} 通販 \textbar{} Amazon \url{https://t.co/s23cIEc9Ti} 
\end{itemize}

 このまとめ記事だと奉納\さらば弁護士鉄道・泥棒神社の物語(@hirono\_hideki)のツイートだけなので,twilog-serch-post
で日付を指定し,別のまとめ記事を作成します。その方が視野も広がり,記録としての有用性も高まります。

\begin{itemize}
\tightlist
\item
  2021年05月24日14時19分の登録:
  「\^{}2021-05-22.+」を@hirono\_hideki @kk\_hirono @s\_hironoで検索 152件の該当 2021-05-24\_14:18の記録
  \url{https://kk2020-09.blogspot.com/2021/05/2021-05-22hironohidekikkhironoshirono15.html} 
\end{itemize}

 次は,76から91のツイートになっていますが,この番号はHTMLのタグで表示されているだけの番号になります。

2021-05-22 11:05:41 ``Ubuntu 21.04 その20 - Ubuntu
21.04の新機能と変更点・既知の問題 - kledgeb \url{https://t.co/YQGBkdM9sv''} 
\url{https://twitter.com/hirono\_hideki/status/1395923789183995905} 

2021-05-22 12:02:18 ``-
大崎事件・第4次再審請求審 救命救急医の尋問6月9日実施で決定 鹿児島
\url{https://t.co/d0AwobsqkU''} 
\url{https://twitter.com/hirono\_hideki/status/1395938040183357441} 

2021-05-22 14:39:23
``2021-05-22\_142603_作ろう!冤罪をただす 再審の法制度(ルール)を! 再審法改正をめざす市民の会結成2周年記念集会.jpg
\url{https://t.co/IwNE7fLVbS''} 
\url{https://twitter.com/s\_hirono/status/1395977572123496450} 

2021-05-22 14:39:27
``2021-05-22\_142613_作ろう!冤罪をただす 再審の法制度(ルール)を! 再審法改正をめざす市民の会結成2周年記念集会.jpg
\url{https://t.co/UjxhHK6ZLg''} 
\url{https://twitter.com/s\_hirono/status/1395977586723885057} 

2021-05-22 14:39:30
``2021-05-22\_142757_作ろう!冤罪をただす 再審の法制度(ルール)を! 再審法改正をめざす市民の会結成2周年記念集会.jpg
\url{https://t.co/W9I5kUark6''} 
\url{https://twitter.com/s\_hirono/status/1395977601475170310} 

2021-05-22 14:39:34
``2021-05-22\_142804_作ろう!冤罪をただす 再審の法制度(ルール)を! 再審法改正をめざす市民の会結成2周年記念集会.jpg
\url{https://t.co/fPYfsTXXN1''} 
\url{https://twitter.com/s\_hirono/status/1395977616104955905} 

2021-05-22 14:39:37
``2021-05-22\_142829_作ろう!冤罪をただす 再審の法制度(ルール)を! 再審法改正をめざす市民の会結成2周年記念集会.jpg
\url{https://t.co/v5Xt5WGjXg''} 
\url{https://twitter.com/s\_hirono/status/1395977631355473920} 

2021-05-22 14:39:41
``2021-05-22\_142832_作ろう!冤罪をただす 再審の法制度(ルール)を! 再審法改正をめざす市民の会結成2周年記念集会.jpg
\url{https://t.co/1Mli5CPTbv''} 
\url{https://twitter.com/s\_hirono/status/1395977646559830018} 

2021-05-22 14:39:45
``2021-05-22\_142835_作ろう!冤罪をただす 再審の法制度(ルール)を! 再審法改正をめざす市民の会結成2周年記念集会.jpg
\url{https://t.co/i9ydjc4Opa''} 
\url{https://twitter.com/s\_hirono/status/1395977661223034880} 

2021-05-22 15:04:10 ``RT @166mochizuki:
東京オリンピック開会式まで62日。組織委員会で働き始めて最初の1週間が終わった。¥\n¥\n分かったことは、もう延期とか中止なんて言えないほど、準備は最終段階に突入していて、これ全部を無駄にするのは辛すぎる・・・。¥\n¥\nあと、みんな死ぬほどサービス残業してる。23時過ぎても秒でメールの返信が来る。''
\url{https://twitter.com/hirono\_hideki/status/1395983809485578241} 

2021-05-22 15:05:27
``2021-05-22-111924\_サイ太@uwaaaa·5月20日ワシも国の作った法テラスの国選基準とかいうクソシステムの欠陥をブログで指摘したら品位がないとか叩かれたんだけ.jpg
\url{https://t.co/JIkTIYGWGx''} 
\url{https://twitter.com/s\_hirono/status/1395984131268349955} 

2021-05-22 15:05:45
``2021-05-22-150502\_坂本正幸@sakamotomasayuk·2時間現場が頑張ってるんだから、で流される典型的な団体が弁護士会(まあ、会費がふってくると思ってる.jpg
\url{https://t.co/L8NcDamWXF''} 
\url{https://twitter.com/s\_hirono/status/1395984204270211072} 

2021-05-22 15:06:14 ``2021-05-22\_11:13
奉納\\#危険生物・弁護士脳汚染除去装置\\#金沢地方検察庁御中\_2020:
\北白川 @GUv4i6\処罰回避目的で示談を迫る刑事弁護人も、示談金を高くつりあげることを誇る被害者代理人も、なにが正しいのか、俺はよく分からんな。
\url{https://t.co/O8g14kxkvf''} 
\url{https://twitter.com/hirono\_hideki/status/1395984328782278661} 

2021-05-22 15:06:21 ``2021-05-22\_11:19
奉納\\#危険生物・弁護士脳汚染除去装置\\#金沢地方検察庁御中\_2020:
\サイ太 @uwaaaa\ワシも国の作った法テラスの国選基準とかいうクソシステムの欠陥をブログで指摘したら品位がないとか叩かれたんだけど,あの時ワシを批判した人は今回も国
\url{https://t.co/Li493ZQqjg''} 
\url{https://twitter.com/hirono\_hideki/status/1395984355198005252} 

2021-05-22 15:06:27 ``2021-05-22\_11:21
奉納\\#危険生物・弁護士脳汚染除去装置\\#金沢地方検察庁御中\_2020: \都
行志/Miyako
Koji @Miyako\_Koji\弁護士の方にお聞きします。もしあなたが5億円を非弁業者からもらえるとしたら、非弁提携を行いますか?
\url{https://t.co/66tJQQLagR''} 
\url{https://twitter.com/hirono\_hideki/status/1395984381659910145} 

2021-05-22 15:06:33 ``2021-05-22\_14:57
奉納\\#危険生物・弁護士脳汚染除去装置\\#金沢地方検察庁御中\_2020:
\モトケン @motoken\_tw\以前にも言いましたが、私は宇崎ちゃんの献血ポスター騒動から明確にアンチツイフェミに転じました。引用ツイート
\url{https://t.co/DipuclY3i1''} 
\url{https://twitter.com/hirono\_hideki/status/1395984408067280897} 

 「能登怪奇譚」が見当たりませんでした。ページ内検索で調べてみます。

 表示だけの番号で125と126になっていました。時間は近接していて5月22日の15時55分と56分となっています。

2021-05-22 15:55:46
``2021-05-22-154847\_能登怪異譚 (集英社文庫) (日本語) 文庫 -- 1993/7/20半村 良  (著), 村上 豊 (イラスト).jpg
\url{https://t.co/96S6pDehfm''} 
\url{https://twitter.com/s\_hirono/status/1395996791040974852} 

2021-05-22 15:56:03 ``2021-05-22-154940\_能登怪奇譚.jpg
\url{https://t.co/2poMAoImNQ''} 
\url{https://twitter.com/s\_hirono/status/1395996863552102401} 

 2021-05-22-154940\_能登怪奇譚.jpgというスクリーンショットのファイル名ですが,Amazonの商品のページなのかと思ったら石川県図書館横断検索の検索結果のページでした。スクリーンショットとして記録したのは記憶の通りですが,ファイル名の付け方が,記憶と違っていました。

 能登地方の昔話の内容だったので,宇出津の図書館にもありそうと前から頭にはあったのですが,いざ調べてみると,石川県の全図書館に蔵書がないというのは驚きの結果でした。作者の名前も,本を読む前から見かけていました。割と名の知られた作家だったと思います。

 2021-05-22-154847\_能登怪異譚 (集英社文庫) (日本語) 文庫 -- 1993/7/20半村 良  (著), 村上 豊 (イラスト).jpgの方が,Amazonのページでしたが,ページタイトルにAmazonが含まれるものと思い込んでいました。

\begin{itemize}
\tightlist
\item
  まんが 弁護士が教えるウソを見抜く方法 \textbar{} 深澤 諭史 \textbar 本
  \textbar{} 通販 \textbar{} Amazon \url{https://t.co/LBfcXfbrI1} 
\end{itemize}

 試しに1つ深澤諭史弁護士の本をAmazonで調べてみましたが,やはり,「まんが
弁護士が教えるウソを見抜く方法 \textbar{} 深澤 諭史 \textbar 本
\textbar{} 通販 \textbar{}
Amazon」ということで,ページタイトルにAmazonが含まれていました。

\begin{itemize}
\tightlist
\item
  半村良 - Wikipedia \url{https://t.co/Kraw4I7U5u}  半村 良(はんむら
  りょう、1933年10月27日 - 2002年3月4日{[}1{]})は、日本の小説家。
  男性。本名は清野 平太郎(きよの へいたろう)。
\end{itemize}

 今まで調べたことはなかったと思いますが,Wikipediaに1933年生まれと出てきて驚き,とっさに明治33年と勘違いしていたのですが,68歳没ということで納得したら,昭和8年の勘違いだと気が付きました。

 ちょっと納得したのですが,「東京府東京市葛飾区生まれ。小学1年生の時に父を失う{[}1{]}。1942年から1945年まで石川県能登地方に疎開していた{[}1{]}。」とありました。映画「少年時代」をYouTubeで視聴したことを思い出しますが,そちらは富山県の東部が疎開先となっていました。

 ざっとWikipediaのページに目を通したところ,「戦国自衛隊」という映画だけは記憶にありました。昭和54年となっていました。昭和50年代の後半で,「セーラー服と機関銃」という映画と同じ頃というイメージでした。けっこう話題になっていたように思います。

\begin{itemize}
\tightlist
\item
  ホラー短編小説 『箪笥』 半村良 - オカル亭 \url{https://t.co/H0xIMu135V} 
  ¥\n
  能登に伝わる怪談として書かれていて創作なのか実話なのか判然としなくなる不気味な話。
  ¥\n 不思議で奇妙な物語で独特の怖さがある。
\end{itemize}

 「半村良」のGoogle検索の入力候補の2番めに「半村良 箪笥」と出ていたのですが,やはり能登怪奇譚の短編の1つでした。どの作品か記憶にないですが,昔の能登はとてもまずしく,旅人をとってくらうような話があったように記憶しています。箪笥は違っていたとは思います。

\begin{quote}
《引用の始まり》
\end{quote}

\begin{quote}
半村良を紹介している文『ふるさと石川の文学』より「東京下町の深川に生まれた半村は、小学校一年の時、父を敗血症で亡くす。彼と彼の弟の二人の子供をかかえた母親は、戦中、戦後の困難な時期を苦労の末に女手ひとつで乗り切っていくのだが、その母の郷里が旧三波村(現、能都町)であり、戦争末期に一時、半村は母の実家に疎開していたことがある。その祖父の家の白壁の土蔵には本や雑誌がつまっており、そこで半村は自身の生涯を決定した国枝史郎『蔦葛木曽桟(つたかずらきそのかけはし)』に出会う。これをむさぼり読んだ半村は伝奇小説の強烈な魅力にとらわれたのであり、母の実家の蔵の中でのこの幸福な記憶が、後に『石の血脈」、『産霊山秘録』、『妖星伝』等の長編伝奇小説を生み出す原点となった。」(水洞幸夫執筆)。
\end{quote}

\begin{quote}
《引用の終わり》
\end{quote}

\begin{itemize}
\tightlist
\item
  半村良氏の疎開先、ご母堂の実家 -
  能登のうみやまブシ(西山郷史) \url{https://umiyamabusi.hatenadiary.org/entry/20160823/1471961483n} 
\end{itemize}

 これまでに何度か見かけているブログですが,疎開先が能都町の三波とありました。2016年8月23日の記事ですが,能都町とあります。平成17年に能都町は,同じ鳳至郡だった柳田村と珠洲郡内浦町と合併し現在の能登町となっています。

 「三波は藤波、波並、矢波地区の総称である。」という部分を読むまで,矢波だと思い込んでいました。宇出津から近い順に藤浪,波並,矢波となっていますが,その次に七見があって,その次が鵜川になります。宇出津から鵜川は距離にして10キロほどです。

\begin{itemize}
\tightlist
\item
  能登町 - Google マップ \url{https://t.co/NtYK8ethIw} 
\end{itemize}

 Googleマップでみても三波というのは三波簡易郵便局ぐらいですが,これは波並だと聞いていました。もう2,3年見ていないように思いますが,地元ではけっこうレトロな建物で知られていました。宇出津でも昭和40年代に見ていたような建物です。

 旧能都町でも三波は,三波地区で,神野地区や高倉地区と同じ感覚でした。小浦はわからないですが真脇と姫が高倉地区で姫の港は高倉漁港という看板になっています。

\begin{itemize}
\tightlist
\item
  怖いおはなし 後日談: 消えがてのうた part 2 \url{https://t.co/M7pgAzkQVV} 
  ¥\n
  「能登怪異譚」で語られる物語の、言葉では説明のつかない不気味で異様な恐ろしさは、
  ¥\n
  古い京言葉の気配が残る能登方言の、おっとりとしたテンポの語り口で倍加する。
\end{itemize}

 能登の方言が京都の古い言葉とは聞いたことがないように思うのですが,京都の都の流刑地という歴史はあるようです。渤海とか大陸と交易があったとも聞くので,シベリアに近い東アジアの民族の言葉がルーツにあるのかとは想像したことはありました。

\begin{itemize}
\item
  \begin{enumerate}
  \def\labelenumi{(\arabic{enumi})}
  \setcounter{enumi}{10}
  \tightlist
  \item
    ``能登'' (from: OR from:hanmura\_ryo) - Twitter検索 / Twitter
    \url{https://t.co/n78WkDq1en} 
  \end{enumerate}
\end{itemize}

 これまで経験のないことですが,何度やってもOR検索になりました。直接,ORをANDに書き換えてやってみます。

\begin{itemize}
\item
  \begin{enumerate}
  \def\labelenumi{(\arabic{enumi})}
  \setcounter{enumi}{10}
  \tightlist
  \item
    ``能登'' (from: AND from:hanmura\_ryo) - Twitter検索 / Twitter
    \url{https://t.co/q4mEDvcJFr}  「``能登'' (from: AND
    from:hanmura\_ryo)」の検索結果はありません
  \end{enumerate}
\end{itemize}

 強制的なOR検索で,能登有沙という名前が出てきたのですが,タレントのようです。声優の名前はよく見かけてきたのですが,能登という名前は身近には聞いたことのない名前で,全国的には割と多い名前なのかと思いました。

〉〉〉 kk\_hironoのリツイート 〉〉〉

\begin{itemize}
\tightlist
\item
  RT
  kk\_hirono(刑事告発・非常上告_金沢地方検察庁御中)|notoarisa(能登有沙ファンクラブはじめました)
  日時:2021-05-24 15:39/2021/04/01 20:12 URL:
  \url{https://twitter.com/kk\_hirono/status/1396717490076405764} 
  \url{https://twitter.com/notoarisa/status/1377579691167547394} 
  \textgreater{}
  ありありのあり!に投票してくれた皆さんありがとうございます!!
  大丈夫です!能登は地雷系ではなく(地雷系のつもりの写真だったw)今まで通り行きます!!
  が!! fansでは、普段の私とは違う「変わり種の能登」もみれるかも!?
  能登有沙officialファンクラブが出来ました✨😆🎉 \#fans
  \url{https://t.co/wYJ3tcXfcQ}  \url{https://t.co/E2NSzgOdii} 
\end{itemize}

〉〉〉 kk\_hironoのリツイート 〉〉〉

\begin{itemize}
\tightlist
\item
  RT
  kk\_hirono(刑事告発・非常上告_金沢地方検察庁御中)|chabobunko(🐓)
  日時:2021-05-24 15:41/2019/09/14 20:42 URL:
  \url{https://twitter.com/kk\_hirono/status/1396718101505339395} 
  \url{https://twitter.com/chabobunko/status/1172838142799306752} 
  \textgreater{}
  最近でかい本屋行かなくなってたから、半村良の能登怪異譚が復刊してたの9月になってから気付いた
  わたしあれの「箪笥」て話がすごい好きで・・・
\end{itemize}

 復刊というのは初めて見たように思いますが,そういえば,今朝,深澤諭史弁護士のタイムラインが更新され1つリツイートが追加されていました。前の晩に,見ていたツイートでした。

〉〉〉 kk\_hironoのリツイート 〉〉〉

\begin{itemize}
\tightlist
\item
  RT
  kk\_hirono(刑事告発・非常上告_金沢地方検察庁御中)|yoakenoswimmer(夜明けの睡魔)
  日時:2021-05-24 15:44/2021/03/17 23:01 URL:
  \url{https://twitter.com/kk\_hirono/status/1396718757574180864} 
  \url{https://twitter.com/yoakenoswimmer/status/1372186298279096327} 
  \textgreater{}
  半村良「能登怪異譚」。地の文まで能登の方言で書かれている怪談集。
  冒頭の「箪笥」は名作すぎてあちこちの怪奇アンソロジーに収録されているので読んでる人も多いでしょうがこの本は村上豊画伯のイラスト付き。例のシーンも見開きイラストでゾワゾワします。純和風怪談「仁助と甚八」も良い味。
  \url{https://t.co/dBWkxrTw4h} 
\end{itemize}

〉〉〉 kk\_hironoのリツイート 〉〉〉

\begin{itemize}
\tightlist
\item
  RT
  kk\_hirono(刑事告発・非常上告_金沢地方検察庁御中)|utsunomiyaco(株式会社 うつのみや)
  日時:2021-05-24 15:45/2019/10/05 13:58 URL:
  \url{https://twitter.com/kk\_hirono/status/1396718885915676672} 
  \url{https://twitter.com/utsunomiyaco/status/1180346419703451649} 
  \textgreater{} 【うつのみや 金沢百番街店】
  今夏、集英社文庫のホラーフェアで断トツの売行きだった「能登怪異譚」(半村良著 税抜480円)が再入荷いたしました。能登弁で語られるがゆえの怖さをぜひご堪能下さい。#集英社文庫 #集英社
  \url{https://t.co/4zV5RJOyuK} 
\end{itemize}

 能登の方言が怖いという感想が多いのですが,記憶にあるのは「面妖い(もっしょい)」ぐらいです。これは相手を疑い難詰する場面でよく使われていたように記憶にあるのですが,胡散臭い,とか,欺瞞に斬り込みを入れる,あるいは問い質す意味合いがあったように思います。

 端的にいえば,「おかしいだろ,それは!」という突っ込みです。昭和の時代は,そのような攻撃的な話し言葉がいくつかあったように思いますが,相手に喧嘩を売るのに近い,開き直りがあったようにも思い出します。今はよほどの泥酔状態でもない限り,ないように思います。

 だいたい,「もっしょいやっつやな,われ! なんやそれ言うってみい!」という感じであったように思います。昭和50年代は,優等生のような生徒でも,普通に相手のことを「われ」と読んでいたのですが,随分前からほとんど聞かなくなりました。

〉〉〉 kk\_hironoのリツイート 〉〉〉

\begin{itemize}
\tightlist
\item
  RT
  kk\_hirono(刑事告発・非常上告_金沢地方検察庁御中)|nanseido(南西堂)
  日時:2021-05-24 15:57/2020/02/17 01:10 URL:
  \url{https://twitter.com/kk\_hirono/status/1396721983241342978} 
  \url{https://twitter.com/nanseido/status/1229075591787241472} 
  \textgreater{}
  半村良『能登怪異譚』(集英社文庫)読了。半村氏のというより日本の怪談のなかでも指折りの傑作「箪笥」を収録。すべて能登半島を舞台に方言で語られる。民話めいた話もあり『遠野物語』の趣き。全9話だが「箪笥」の不気味さが圧倒的。怪談にありがちな因果の説明がないゆえに恐ろしいのだ。
  \url{https://t.co/KrFSfM4hAE} 
\end{itemize}

 「遠野物語」という題名は聞いたことがありますが,内容は知りません。能登怪奇譚は怖い話として思っていた以上に有名で評価が高いようですが,たしかに不思議な世界観があって,ずっと残っているように思います。それに似た独自の世界観を感じるのが,市川寛弁護士でもあります。

 最近はツイートの数自体が少なくなっている様子の落合洋司弁護士(東京弁護士会)ですが,この元検事の弁護士も,尋常ではない検察,警察批判がありました。批判という言葉でくくれるようなものではなく,人知を超えた凄まじいパワーを感じたものです。

\begin{itemize}
\tightlist
\item
  TW
  yjochi(弁護士落合洋司🌸感染拡大を招く東京(頭狂)オリンピック中止!🌸)
  日時: 2021/05/24 13:56:24 URL:
  \url{https://twitter.com/yjochi/status/1396691528349732870} 
  \textgreater{}
  シーボルトには、手に入れた地図で日本の知人に多大な迷惑をかけた反省が欲しい。笑
  \url{https://t.co/pRrPiuCvTr} 
\end{itemize}

〉〉〉 kk\_hironoのリツイート 〉〉〉

\begin{itemize}
\tightlist
\item
  RT
  kk\_hirono(刑事告発・非常上告_金沢地方検察庁御中)|numrock(スエヒロ)
  日時:2021-05-24 16:06/2021/05/23 12:12 URL:
  \url{https://twitter.com/kk\_hirono/status/1396724227483475968} 
  \url{https://twitter.com/numrock/status/1396302970099101710} 
  \textgreater{} 「来日された先輩偉人の声」を作ってみました。
  \url{https://t.co/34IqNzdXAM} 
\end{itemize}

 教科書や参考書に出てくるようなイラスト付きの説明文となっていました。落合洋司弁護士(東京弁護士会)の引用ツイートに,シーボルトとありますが,たくさんの本を読んでいるらしく,歴史物も多いようです。

 以前,小中学生の少女で場所は札幌市であったような薄い記憶なのですが,本屋で本が崩れ死亡したという事故で,落合洋司弁護士(東京弁護士会)が,本屋で死ねるなら本望などと,書いていたことを思い出しました。ブログでは砕けた話は見かけないので,ツイートだったように思います。

\begin{itemize}
\item
  ``本屋'' (from:yjochi) - Twitter検索 / Twitter
  \url{https://twitter.com/search?lang=ja\&q=\%22\%E6\%9C\%AC\%E5\%B1\%8B\%22\%20(from\%3Ayjochi)\&src=typed\_query} 
\item
  TW
  yjochi(弁護士落合洋司🌸感染拡大を招く東京(頭狂)オリンピック中止!🌸)
  日時: 2018/02/28 09:14:14 URL:
  \url{https://twitter.com/yjochi/status/968640448557232128} 
  \textgreater{}
  惨軽やあべちゃんのFacebook読んで、本屋で嫌韓、嫌中本買って読んでいると、脳がだんだんあんな感じになるんかな。怖。
\item
  ``本'' (from:yjochi) - Twitter検索 / Twitter
  \url{https://twitter.com/search?q=\%22\%E6\%9C\%AC\%22\%20(from\%3Ayjochi)\&src=typed\_query} 
\item
  TW
  yjochi(弁護士落合洋司🌸感染拡大を招く東京(頭狂)オリンピック中止!🌸)
  日時: 2018/11/23 10:16:18 URL:
  \url{https://twitter.com/yjochi/status/1065776017589956608} 
  \textgreater{}
  無知が日本史の本なんか書くとこんなことに。無知が書いて馬鹿が読む。亡国。
  \url{https://t.co/4evHp0uRDv} 
\item
  TW
  yjochi(弁護士落合洋司🌸感染拡大を招く東京(頭狂)オリンピック中止!🌸)
  日時: 2020/12/03 17:59:26 URL:
  \url{https://twitter.com/yjochi/status/1334421978309033986} 
  \textgreater{}
  いつまた凍結されるかわからないので、暇な方は下記のサブアカウントもフォローしてください。本アカウントが永久凍結されたら、そちらに移ります。\\
  \textgreater{} @flytonextjapan
\end{itemize}

 Twitterの検索では見つからなかったのですが,twilog-serch
``本望'' で見つけることが出来ました。

\begin{lstlisting}
py37_env ❯ twilog-serch 本望|grep @yjochi
\end{lstlisting}

\begin{itemize}
\tightlist
\item
  ./hirono\_hideki2021-05-24\_143303.csv:2019-05-31 23:01:58
  ``2019年05月31日19時05分の登録: \?落合洋司?Yoji
  Ochiai? @yjochi\いくら健康でも、暴漢に刺されて死んだらそれまでだし。好きなもの食って飲んで早死にするのは仕方ないし本望。
  \url{http://hirono2014sk.blogspot.com/2019/05/yoji-ochiaiyjochi\_65.html''} 
  \url{https://twitter.com/hirono\_hideki/status/1134460008970706944} 
\item
  ./hirono\_hideki2021-05-24\_143303.csv:2012-03-27 12:24:38
  ``落合洋司弁護士は、本が崩れて死ぬのなら本望というぐらい本が好きで、沢山読んでいらっしゃるそうですね。ブログでみかけたことあります。RT
  @nyanmayu: 読みたい。 RT @yjochi: 裁かれた命 死刑囚から届いた手紙
  \url{http://bit.ly/GU2QQ3''} 
  \url{https://twitter.com/hirono\_hideki/status/184481026859479040} 
\end{itemize}

 さきほどのツイートが出てきて,それを勘違いしたのかと思ったのですが,2012-03-27
12:24:38のツイートでした。落合洋司弁護士(東京弁護士会)本人のツイートはまだ見つかっていません。

 キーワードを「崩」に変更しました。落合洋司弁護士(東京弁護士会)のツイートを探しています。

\begin{itemize}
\tightlist
\item
  ./hirono\_hideki2021-05-24\_143303.csv:2019-07-02 09:18:31
  ``2019年07月02日09時05分の登録: %@yjochi ?落合洋司?Yoji
  Ochiai?%警察への信頼が強かったのがこの事件で一気に崩壊したと言っていたのを思い出した。→志布志事件:鹿児島県警本部長が謝罪
  - 毎日新聞
  \url{http://hirono2014sk.blogspot.com/2019/07/yjochiyoji-ochiai.html''} 
  \url{https://twitter.com/hirono\_hideki/status/1145849194311106560} 
\end{itemize}

 twilog-serch 本屋\textbar grep
崩 の結果がありませんでした。Googleで検索してみますが,2012年より前になりそうです。

\begin{itemize}
\item
  本棚倒壊で古書店元経営者ら書類送検 札幌の女児重傷事故: 日本経済新聞
  \url{https://t.co/Ep9Jk9uRZn} 
  天井や床にしっかり固定せずに倒壊させ、小学5年だった鈴木愛菜さん(12)に重傷を、中学3年だった姉愛梨さん(16)に軽傷を負わせた疑い。愛菜さんは事故から2年3カ月たった現在も意識が戻っていない
\item
  本棚倒壊で古書店元経営者ら書類送検 札幌の女児重傷事故: 日本経済新聞
  \url{https://t.co/Ep9Jk9uRZn} 
  鑑定などの結果、本の陳列方法に問題があり、倒壊防止などの対策を怠ったと判断。事故では高さ約2.1メートル、幅約6.3メートルの木製本棚3列が倒れた。
\item
  本棚倒壊で古書店元経営者ら書類送検 札幌の女児重傷事故: 日本経済新聞
  \url{https://t.co/Ep9Jk9uRZn}  2012年1月11日 22:16
\end{itemize}

 記憶違いで死亡事故ではなかったのかと思ったのですが,少し読み進めると「愛菜さんは事故から2年3カ月たった現在も意識が戻っていない」と出てきたので,衝撃的でした。本屋ではなく中古書店とあります。Amazonプライムビデオの銀河鉄道999,彗星図書館も思い出しています。

 平成31年3月の終わり頃のことになりますが,落合洋司弁護士(東京弁護士会)の参院選出馬公認取り消しのニュースと,同時的にYouTubeで「子供が好きな神様」という昔話を視聴したことが強く印象に残っており,ずっと前から取り上げて記録化しておきたいと思っています。

\begin{lstlisting}
py37_env ❯ twilog-serch 子どもが好きな神様
\end{lstlisting}

\begin{itemize}
\tightlist
\item
  ./hirono\_hideki2021-05-24\_143303.csv:2019-11-17 20:43:10 ``5739:
  能登町波並の集落:深澤諭史弁護士の「組体操・ブラック校則対策弁護団」という妄想と、子どもが好きな神様
  /石川県警察珠洲警察署御中
  \url{http://hirono-hideki.hatenablog.com/entry/2019/10/10/215441''} 
  \url{https://twitter.com/hirono\_hideki/status/1196031018538389504} 
\item
  ./kk\_hirono2021-05-24\_170903.csv:2020-02-13 19:56:19
  ``平成21年より前からブログをみていた落合洋司弁護士(東京弁護士会)のことで最も印象深いのは、昨年2019年3月末から4月初めに掛けた立憲民政党の参院選公認取り消しの話題ですが、ちょうど同じ頃にみたのが「子どもが好きな神様」というYouTube動画でした。''
  \url{https://twitter.com/kk\_hirono/status/1227909358043795456} 
\item
  ./kk\_hirono2021-05-24\_170903.csv:2019-12-31 02:17:18
  ``検索に直江津市が出てきて驚いたのですが、昭和46年に直江津市と高田市が合併して上越市になったとあり、納得しました。「中頸城郡谷浜村」という記載が見えますが、これは名立谷浜インターの近くで、4月の下旬に知った「子どもが好きな神様」の伝承地のようでした。''
  \url{https://twitter.com/kk\_hirono/status/1211697783951900675} 
\item
  ./kk\_hirono2021-05-24\_170903.csv:2019-10-16 22:20:26
  ``この組み体操の問題で、「子どもが好きな神様」を含むエントリーを作成したように思いますが、その点はまだ手付かずだったかと思います。今年の3月の終わり頃のことになりますが、この3月は、性犯罪の無罪判決が集中し、法クラの弁護士らが活況を呈していました。''
  \url{https://twitter.com/kk\_hirono/status/1184459083358732289} 
\item
  ./kk\_hirono2021-05-24\_170903.csv:2019-10-10 21:31:39 ``-*-
  能登町波並の集落:深澤諭史弁護士の「組体操・ブラック校則対策弁護団」という妄想と、子どもが好きな神様''
  \url{https://twitter.com/kk\_hirono/status/1182272482004238337} 
\item
  ./kk\_hirono2021-05-24\_170903.csv:2019-10-10 21:30:37 ``FILE\_NAME:
  2019-10-10-212956\_深澤諭史弁護士の「組体操・ブラック校則対策弁護団」という妄想と、子どもが好きな神様.org''
  \url{https://twitter.com/kk\_hirono/status/1182272218190860288} 
\item
  ./kk\_hirono2021-05-24\_170903.csv:2019-10-10 21:30:31
  ``\#+TITLE:''深澤諭史弁護士の「組体操・ブラック校則対策弁護団」という妄想と、子どもが好きな神様"
  " \url{https://twitter.com/kk\_hirono/status/1182272194665050113} 
\end{itemize}

 上記のtwilog-serch の検索結果で,「-*-
能登町波並の集落:深澤諭史弁護士の「組体操・ブラック校則対策弁護団」という妄想と、子どもが好きな神様」という内容のあるファイルや記事を探したのですが,見つからず,見出しだけを作成し本文を作っていなかったのかと思いました。

 ``5739:
能登町波並の集落:深澤諭史弁護士の「組体操・ブラック校則対策弁護団」という妄想と、子どもが好きな神様
/石川県警察珠洲警察署御中
\url{http://hirono-hideki.hatenablog.com/entry/2019/10/10/215441''} ,という投稿があったことがあとになってわかりました。

 能登町波並の集落が気になったのですが,神野からバイクで波並の集落に行ったことしか書いてありませんでした。昭和40年代の自転車事故のことが書いてあるのかと気になったのです。

\begin{itemize}
\tightlist
\item
  TW fukazawas(深澤諭史) 日時: 2019/10/10 17:42:58 URL:
  \url{https://twitter.com/fukazawas/status/1182214929882861568} 
  \textgreater{}
  「組体操・ブラック校則対策弁護団」とかでもいいかも。妄想ですが。\\
  \textgreater{}
  ブラック校則って,子どもの人権問題もそうですが,むしろ子どもに「圧倒的な力関係を背景にすれば理不尽を押しつけてもよい」と思わせるので教育上最悪ですし,それで社会に出られると社会にも悪影響なんですよね。\\
  \textgreater{} (・∀・;) \url{https://t.co/EMwK1g7OO1} 
\end{itemize}

 埋め込みツイートがブログの記事に表示されていました。これはツイートが削除されていない印になると思いますが,やはりAPIで取得が出来ました。神をも凌駕する凄まじい弁護士パワーを感じる深澤諭史弁護士のツイートです。神様がいたらどう思うのかと深く考えさせられました。

 そういえば,東京都が弁護士をつけていない,というニュースを本日見かけたところですが,旭川の女子中学生いじめ自殺の事件でも,学校側の弁護士に対する反発が,ストレートにぶつけられていたことを思い出しました。

 私は,ちょっと買い物に出かけるときなど,「twitterAPI-search-lawList-mydql-add.rb
`再審' \&\& systemctl
suspend」というコマンドを実行するのですが,データベースへの追加処理が終わると,そのままサスペンドに入り,パソコンが休止状態となります。

 それでなくても毎日欠かさないように「twitterAPI-search-lawList-mydql-add.rb
`再審'」という処理を実行しているのですが,まとめ記事はしばらく作成しておらず,他に旭川の事件について情報を見かけることもなく,検索で調べることもしていませんでした。

 コマンドを貼り付け実行させてから気がついたのですが,キーワードが違ったものとなっていました。「twitterAPI-search-lawList-mydql-add.rb
`旭川
少女'」の間違いです。再審の方は3千台で止まっていますが,旭川の方は8500のリミットに決まって到達していました。

 8500のリミットに達するような検索を短時間に続けていると,リミット制限に引っ掛かり,しばらくTwitterAPIが利用できなくなります。

 今日の午前中になるのかと思うのですが,たぶん初めて市川寛弁護士の年齢が気になり調べてみました。年齢より先に出身地が神奈川県川崎市ということがわかったのですが,不思議と出身地も気にしたことがありませんでした。

 年齢が1965年生まれとなっていました。生まれた月で1つは年齢が違ってきますが,私が1964年生まれなので一つ違いになります。1月から3月の生まれだと同じ学年の同級生にもなるのですが,繰り返すようですが,これまで市川寛弁護士の年齢を気にすることは不思議とありませんでした。

 ただ,なんとなくですが,一回りぐらいは年下の世代と考えていたように思います。検事を辞めないでいれば,今頃はどこかの地検の検事正になっていたかもしれない年齢になりそうで,県警の本部長として見かける年齢というのも54歳前後が多いように思います。

\begin{itemize}
\item
  TW imarockcaster42(弁護士 市川 寛) 日時: 2021/05/24 08:47:47 URL:
  \url{https://twitter.com/imarockcaster42/status/1396613863609491457} 
  \textgreater{} 2018年撮影。江ノ島東浜海岸 \url{https://t.co/AEYMl7kESx} 
\item
  TW imarockcaster42(弁護士 市川 寛) 日時: 2021/05/24 13:12:07 URL:
  \url{https://twitter.com/imarockcaster42/status/1396680385027661832} 
  \textgreater{}
  某裁判所に刑事事件の連絡をしたところ、書記官が被告人を呼び捨てにしていたのにちょっと驚きました。検察庁では平然とやっていましたが、裁判所の一部も同じなんですかね
\end{itemize}

 今日は,市川寛弁護士のタイムラインで,江ノ島東浜海岸という写真を最新ツイートとして見かけていたのですが,1つツイートが更新されていることに気が付きました。

 江ノ島は有名でテレビで見かけることも多いですが,東浜海岸というのは聞いたことがないものの方角で,テレビでよく見かける江ノ島より,東京寄りになるのかと思われます。2018年撮影とあるのですが,昭和50年代後半を思い起こす雰囲気の写真だと感じていました。

 次のツイートにあるのが2日ほど前に読んだ市川寛弁護士のアメブロのブログ記事になると思いますが,5月20日のツイートとあります。今日が24日なので4日前ですが,記事の日付は2019年かあるいは2018年になっていたように思います。

\begin{itemize}
\tightlist
\item
  TW imarockcaster42(弁護士 市川 寛) 日時: 2021/05/20 16:23:24 URL:
  \url{https://twitter.com/imarockcaster42/status/1395278972032741381} 
  \textgreater{} 『検事はなぜ証拠を見せないのか -
  理屈でないところからの試論』\\
  \textgreater{} ⇒ \url{https://t.co/zuldOACOqe}  \#アメブロ
  @ameba\_officialより
\end{itemize}

 2021年05月24日17時59分の実行記録:
twitterAPI-search-lawList-mydql-add.rb ``再審'' ツイート数:195/2426
リツイート数:70/2426 トータル:3626 ¥\n ``再審''の該当: hirono\_hideki
55/3件 kk\_hirono 73/3件 s\_hirono 14/0件

 忘れていたのですが,間違えて実行したtwitterAPI-search-lawList-mydql-add.rb
``再審''の結果も出ていて,トータルで3626件でした。たぶん私のツイートが一番多そうですが,それでも上昇傾向にはある感じです。

\begin{itemize}
\tightlist
\item
  検事はなぜ証拠を見せないのか - 理屈でないところからの試論 \textbar{}
  検事失格  弁護士 市川寛のブログ \url{https://t.co/fZaNB84tkJ}  2019-06-27
  04:16:10
\end{itemize}

 確認したところ,やはり2019年6月27日の記事でした。過去の記事をツイートで紹介するのは深澤諭史弁護士もよくやっていましたが,ツイートをよく削除する目撃体験がある市川寛弁護士が,過去の記事を再掲のように紹介するのは珍しく感じました。

 ツイートの場合は修正が出来ず削除して再投稿するしかないはずですが,ブログだと普通に書き換えができます。市川寛弁護士のブログを読んでいて内容の書き換えというのは全く感じたことがなかったです。なぜツイートを削除するのか,その方が不思議でなりません。

\begin{itemize}
\tightlist
\item
  奉納\危険生物・弁護士脳汚染除去装置\金沢地方検察庁御中\_2020:
  \弁護士 市川
  寛 @imarockcaster42\検事の本分は犯罪捜査の最前線にいることではなく、法律家として、警察の違法捜査をチェックし、戒めることだと思います
  \url{https://t.co/HqdCnX3SMJ} 
\end{itemize}

 「2021-04-30 08:58から100件:最新2021-05-23
13:34という範囲(23日4時間36分)の取得」という記録ですが,やはり次のツイートが削除されていたようです。表示されないことを確認しています。

\begin{itemize}
\tightlist
\item
  (2/100) TW imarockcaster42(弁護士 市川 寛) 日時: 2021-05-23
  13:30 URL(削除されたらしいツイート):
  \url{https://twitter.com/imarockcaster42/status/1396322629401780224\textgreater} {}
  「巨悪」とやらに立ち向かう特捜部だって、冤罪を生んでいますよ。もっとも、特捜部万歳のメディアのおかげで滅多に表に出ませんがね
  \url{https://t.co/8sLekUqApm} 
\end{itemize}

 埋め込みツイートだと削除されたツイートに気が付きやすいのですが,埋め込みツイートの読み込みは100件を限界の目安として使っています。それでも読み込みが出来ずブラウザから処理が完了できないというようなメッセージが出ることが多いです。

 最近はないですが,以前は,埋め込みツイートが表示されていなくても個別に開くと普通に表示されるツイートというのはありました。なので,ツイートが削除されているとは即断せず,確認をするようにしています。

 ちょうど100件目になる最後のツイートですが,引用ツイートを含めてかなり気になる内容となっていました。

\begin{itemize}
\item
  TW imarockcaster42(弁護士 市川 寛) 日時: 2021/04/30 08:58:44 URL:
  \url{https://twitter.com/imarockcaster42/status/1387919311776456705} 
  \textgreater{} \url{https://t.co/RLAb09Wzt9} 
  この本を読むだけでも、冤罪だと思えるのではないでしょうか
\item
  TW imarockcaster42(弁護士 市川 寛) 日時: 2021/04/30 08:57:14 URL:
  \url{https://twitter.com/imarockcaster42/status/1387918932053430275} 
  \textgreater{}
  「合理的な疑い」があるのに有罪にしていれば冤罪ですよね。また、「真相がわからない」から冤罪と言うのに抵抗があるというのは、「それなら真犯人を連れてこい」とばかりに被告人側に立証責任を転換するような考え方だと思います。有罪にできない事件を有罪にしたら、それはすなわち冤罪でしょう
  \url{https://t.co/RAvuTD9ENu} 
\end{itemize}

〉〉〉 kk\_hironoのリツイート 〉〉〉

\begin{itemize}
\item
  RT
  kk\_hirono(刑事告発・非常上告_金沢地方検察庁御中)|yomu\_kokkai(平河エリ
  Eri Hirakawa \textbar{} 読む国会) 日時:2021-05-24 18:24/2021/04/30
  08:53 URL: \url{https://twitter.com/kk\_hirono/status/1396759000041988099} 
  \url{https://twitter.com/yomu\_kokkai/status/1387917975601041408} 
  \textgreater{} 和歌山カレー事件って冤罪なんですか?
  (合理的な疑いが残るという意味では冤罪かな、と思いつつ、素人目には真相がどうなのかはよくわからず・・・・・・)
  \url{https://t.co/xvd7EgPs9z} 
\item
  TW imarockcaster42(弁護士 市川 寛) 日時: 2021/04/29 20:35:38 URL:
  \url{https://twitter.com/imarockcaster42/status/1387732301748989953} 
  \textgreater{}
  「紀州のドン・ファン」元妻が遺産相続に執着 和歌山カレー事件と同じ消去法で立件〈dot.〉\url{https://t.co/8Fe3rDPMOb} 
  和歌山県警では過去にそっくりの事件があった。(略)和歌山カレー事件だ。←これ、冤罪ですよ
\end{itemize}

 前にも少し見かけていたような市川寛弁護士のツイートですが,4月29日となっています。和歌山カレー事件が冤罪という話は,ここ1年ほどでしょうか割とよく見かけるようになりました。

 この和歌山カレー事件は,被告人が一審で黙秘を貫き,死刑判決となった控訴審で積極的に発言するようになったと聞きます。随分前のことですがよく憶えています。最近は見かけない情報となっているので知らない人が多いかもしれません。

 4月29日よりは後と思いますが,今度は一審で無罪判決だった被告人が控訴審で黙秘を貫き逆転有罪判決になるというニュースがありました。平成31年の3月頃,4つの性犯罪無罪事件の一つとしてネットで大きな話題となっていたのですが,その最初だったと思う。福岡県久留米市の事件です。

 どうも毎日新聞の一人の女性記者が取材をし,無罪判決を記事にしたところ大きな反響となり,毎日新聞の上の方も動き出したというような流れでした。ブックマークに入れているTwitterアカウントですが,半月ほど前に見てから開いていません。半月だと短いスパンになります。

\begin{itemize}
\tightlist
\item
  安部志帆子@毎日新聞(@mai\_shihoko)さんの返信があるツイート / Twitter
  \url{https://t.co/mViqUnSfRD} 
\end{itemize}

 Twitterのアイコンが田んぼで飛び跳ねるような写真で,ずいぶん高く飛んでいるように今みて思ったのですが,ヘッダ画像の方が前から印象的で,火まつりの写真となっています。

 プロフィールの方が,更新されたのだと思いのですが,「毎日新聞記者。2019年3月、福岡地裁久留米支部の準強姦事件無罪判決(2020年2月に福岡高裁で逆転有罪判決。上告中)」となっていました。

 よく見ると,昨年の2020年2月に福岡高裁で逆転有罪判決,上告中とあります。控訴審の黙秘が法クラのツイートで話題になっていたのは,今月の5月に入ってからだと思うのですが,その前に,判決文が公開されているという話があったかもしれません。

〉〉〉 kk\_hironoのリツイート 〉〉〉

\begin{itemize}
\tightlist
\item
  RT
  kk\_hirono(刑事告発・非常上告_金沢地方検察庁御中)|mainichi\_kokura(毎日新聞小倉報道部)
  日時:2021-05-24 18:44/2021/05/14 21:55 URL:
  \url{https://twitter.com/kk\_hirono/status/1396763967859564544} 
  \url{https://twitter.com/mainichi\_kokura/status/1393188093063098371} 
  \textgreater{}
  深酔い状態の女性に性的暴行をしたとして準強姦罪(刑法改正で準強制性交等罪に名称変更)に問われた福岡市の会社役員の上告審で、\#最高裁
  は12日付で被告の上告を棄却する決定を出しました。
  フラワーデモ契機になった福岡の性犯罪 上告棄却、逆転有罪確定へ
  \textbar{} 毎日新聞 \url{https://t.co/fNYQbVMRxe} 
\end{itemize}

 リツイートとして上告棄却で逆転有罪判決が確定へ,という毎日新聞小倉報道部のツイートが出てきました。もともとこの刑事裁判は,毎日新聞の久留米支局とかの報道で始まっていました。

\begin{itemize}
\tightlist
\item
  フラワーデモ契機になった福岡の性犯罪 上告棄却、逆転有罪確定へ
  \textbar{} 毎日新聞 \url{https://t.co/dDD59rYWyL}  毎日新聞 2021/5/14
  20:28(最終更新 5/14 20:28) 563文字
\end{itemize}

 どうも被告人の住所が福岡市というだけではなく,犯行現場も福岡市の飲食店となっているのですが,やはり福岡地裁久留米支部とありました。被害者の居住地が久留米支部の管轄地域ぐらいしか思い当たるところがないですが,被告人や家族にすれば,負担もありそうです。

 この事件は話題になっているとき,こだわりと評価実績の大きい和食の飲食店の経営者か料理人が被告人という情報をネット記事で読んでいました。注目されたから出てきた情報だったと思いますが,仮に,誘われて美人局ではめられたのだとしたら・・・と想像はしたことがありました。

 毎日新聞も「フラワーデモ契機になった福岡の性犯罪 上告棄却、逆転有罪確定へ」という見出しにしているぐらいなので,事件の真相などもともと眼中になかったのかもしれません。被害者の女性に関する情報は見たことがなく,嵐のようなものを感じました。

 同じようにフラワーデモという騒ぎに当初なっていたと思う,草津温泉の問題もありました。どうなったのかさっぱり情報を見かけなくなっていますが,性的な被害を主張した女性の方が責任の追求をされていました。

\begin{itemize}
\tightlist
\item
  \#草津町のフラワーデモに連帯します - Twitter検索 / Twitter
  \url{https://t.co/w0YjnzgliP} 
\end{itemize}

 検索するとTwitterのハッシュタグが出てきました。「話題のツイート」というタブになっていますが,昨年2020年12月のツイートばかりになっていました。今年2021年のツイートはさきほどツイートしたばかりの私のアカウントのツイートが1つありました。

 タブを最新のツイートに切り替えると,タイムラインが空白のまま時間が過ぎています。この切り替えは時間がかかると1,2ヶ月前から感じていたのですが,今回はやがて5分近くで,まだ表示がありません。「待つのだぞ」という昔のCMを思い出したのですが,子連れ狼だったと思います。

 ページの再読込をするとすぐに表示されましたが,タブは「最新」とだけなっていました。いつもTwitterの検索で最初に表示されるタブは「話題のツイート」となっています。

 \#草津町のフラワーデモに連帯します というハッシュタグのまま検索していることに気がついたのですが,私の10分前と表示されているツイートの下は2020年12月26日のツイートで,その下も2020年のツイートが並んでいました。

 二艘引きという漁法を思い出したのですが,フラワーデモを推進する弁護士と,批判する弁護士は世論を扇動しながら,依頼者確保を目的に大きな網を仕掛けているように思えてなりません。小倉秀夫弁護士は痛烈に批判する側でした。容赦のない攻撃を被害者という女性にぶつけているとも感じました。

 2021年05月24日19時20分の実行記録:8500で処理を終了
twitterAPI-search-lawList-mydql-add.rb ``草津'' ツイート数:22/2426
リツイート数:0/2426 トータル:8500 ¥\n ``草津''の該当: hirono\_hideki
0/0件 kk\_hirono 1/0件 s\_hirono 0/0件

 草津をtwitterAPI-search-lawList-mydql-add.rb
で検索したのはずいぶん久しぶりですが,組み合わせが多様だったと思うので,「草津」のみで実行しました。これだと結構な割合で,滋賀県草津市が入ることは承知しています。

 弁護士脳の思惑と打算が吹き荒れた町,というのが,銀河鉄道999に出てくる不思議な星のような,最近の草津温泉のイメージとなっています。こちらも網を広げて,収集をしています。

 思ったより少ない結果でしたが,毎日全ツイートを記録している小倉秀夫弁護士以外は,他のキーワードと抱き合わせで掛かったものしか結果を期待できません。長らく検索をすることがなかったからです。完全に忘れることはなかったと思うのですが,思い出すことは少なかったです。

\begin{itemize}
\tightlist
\item
  2021年05月24日17時25分の登録:
  REGEXP:''再審''/データベース登録済みツイートの検索:2021-05-21〜2021-05-24/2021年05月24日17時23分の記録:ユーザ・投稿:34/96件
  \url{https://kk2020-09.blogspot.com/2021/05/regexp2021-05-212021-05\_24.html} 
\item
  2021年05月24日19時26分の登録:
  REGEXP:''草津''/データベース登録済みツイートの検索:2014-01-28〜2021-05-24/2021年05月24日19時21分の記録:ユーザ・投稿:112/397件
  \url{https://kk2020-09.blogspot.com/2021/05/regexp2014-01-282021-05.html} 
\end{itemize}

 2021年05月24日18時08分の実行記録:
twitterAPI-search-lawList-mydql-add.rb ``旭川 少女'' ツイート数:62/2426
リツイート数:1/2426 トータル:4221 ¥\n ``旭川 少女''の該当:
hirono\_hideki 1/0件 kk\_hirono 2/0件 s\_hirono 0/0件

 旭川の少女の検索の方も忘れていたのですが,端末の別に開いているタブで処理が終わっていて,今回はトータルが4221件ということで8500のリミットには到達していませんでした。こちらは久々に一度,期間制限無しでまとめ記事を作成しておきます。

 総数はすぐにポップアップのメッセージで出るようにしているのですが460件のツイートとなっていました。

 草津のまとめ記事は小倉秀夫弁護士のツイートが26件でした。滋賀県|弁護士あけぼの法律事務所のツイートが69件と多いですが,これは純粋に草津市のツイートだと思います。前にも見かけていました。

\begin{itemize}
\tightlist
\item
  (005/397) TW
  hirono\_hideki(奉納\さらば弁護士鉄道・泥棒神社の物語) 日時:
  2019-05-06 18:45:00 +0900 URL:
  \url{https://twitter.com/hirono\_hideki/status/1125335744938045440\textgreater} {}
  『草津へ至る路:名も寂し暮坂峠』中之条(群馬県)の旅行記・ブログ by
  Ot.Kasperさん【フォートラベル】 \url{https://t.co/uYAw5ZOfn4} 
\end{itemize}

 前に「暮坂」という地名を検索したことがあったのですが,意外に都市部にはなかった地名だったと思います。1つが群馬県と長野県の辺りの地名ということはよく憶えていたのですが,草津温泉の道中とは憶えていませんでした。輪島市門前町の暮坂がきっかけです。

※ @kk\_hironoのアカウントがブロックされ,リツイートに失敗したツイート

\begin{itemize}
\tightlist
\item
  TW momo3580(momo3580@意外に多忙) 日時:2020/01/25 19:31:52 URL:
  \url{https://twitter.com/momo3580/status/1221017836287483904} 
  \textgreater{} @nodahayato
  叙勲に浴した元上司のお祝いに・・・草津だとお店がどうもあれなので、大津でやろうかなと思ってます。
\end{itemize}

 滋賀県高島市の野田隼人弁護士への返信ツイートとなっていますが,前からブロックされている弁護士らしいアカウントが,群馬県内と思っていたのが,滋賀県内のことで出てきました。草津と大津の組み合わせです。アイコンが群馬県のゆるキャラだと思います。

 このアカウントを最初に見た頃,野球の清原選手が薬物で逮捕され,群馬県の高崎市の辺りのコンビニで売人と授受をしたようなニュースを見かけたと記憶にあります。この高崎市の辺りというのは,ドリル優子などという不名誉な呼び名を法クラの弁護士に与えられた元総理の娘のことも。

※ @kk\_hironoのアカウントがブロックされ,リツイートに失敗したツイート

\begin{itemize}
\tightlist
\item
  TW nodahayato(野田隼人 Atty. NODA Hayato) 日時:2020/01/25 19:37:30
  URL: \url{https://twitter.com/nodahayato/status/1221019255417688065} 
  \textgreater{} @momo3580
  鳥取におられた方ですね。ご予算と方向性をお知らせいただければご相談に。(当地のGの方が協力してくれたりするんですか?
\end{itemize}

 被告発人小島裕史裁判長も叙勲していたと思います。被告発人ではないですが,山口成良金沢大学教授も叙勲していたと思います。他にもいたかもしれないですが,すぐには思い出せない本件告発事件の関係者です。

\begin{itemize}
\item
  TW nodahayato(野田隼人 Atty. NODA Hayato) 日時: 2020/01/25 19:42:22
  URL: \url{https://twitter.com/nodahayato/status/1221020480301191168} 
  \textgreater{} @momo3580
  (お祝いの心づもりが必要なので叙勲はチェックしています)大津でも草津でも比較的多い苗字で、名家が多いと認識しています。
\item
  TW momo3580(momo3580@意外に多忙) 日時: 2020/01/25 19:45:52 URL:
  \url{https://twitter.com/momo3580/status/1221021360798883842} 
  \textgreater{} @nodahayato
  かなりの名家のようです。本物のかなり昔の楽茶碗をお持ちです。
\end{itemize}

 本日は,市川寛弁護士のツイートのことでだらだらとやっていますが,名家だったのか基準がわからないものの,またしても先祖のことがきっかけで,大きな発見に繋がり,記録の必要性を感じて,準備は済ませています。

 次のツイートが最初にありました,前に見たように思ったのですが,データベースには未登録だったので初見かもしれません。高速深夜バスとあるので,滋賀県在住の弁護士ではなさそうです。

\begin{itemize}
\tightlist
\item
  TW momo3580(momo3580@意外に多忙) 日時: 2020/01/25 19:19:40 URL:
  \url{https://twitter.com/momo3580/status/1221014768737079297} 
  \textgreater{}
  来週、急きょ大津か草津(いずれも琵琶湖を海だと認識してる・・・)へ行くことになりました。日帰りにするか、高速深夜バスで戻るか・・・\\
  \textgreater{} 普通に新幹線往復だと、泊まっても費用変わらないし。
\end{itemize}

\begin{quote}
《引用の始まり》
\end{quote}

\begin{quote}
momo3580@意外に多忙@momo3580【注】TW/RT等は、全て単なる私的なものです。【自己紹介】薄暗い窓際にいるおじさんです。サラリーマンとOLさんの街で個食個飲2010年3月からTwitterを利用しています548
フォロー中1,502 フォロワー
\end{quote}

\begin{quote}
《引用の終わり》
\end{quote}

\begin{itemize}
\tightlist
\item
  momo3580@意外に多忙さん (@momo3580) / Twitter
  \url{https://twitter.com/momo3580} 
\end{itemize}

 久しぶりにTwitterのページをみたと思うのですが,プロフィールに弁護士をうかがわせる記載はありませんでした。改行を入れて,「【注】TW/RT等は、全て単なる私的なものです。」,「【自己紹介】薄暗い窓際にいるおじさんです。」とあります。

 アイコンの写真が新しいものに変わっているような気がするのですが,拡大するとハッピに「ぐんま」とあったので,このゆるキャラは,「ぐんまちゃん」という愛称であったように思います。

 ツイートの数が23万となっています。2010年3月からの利用とありますが,私の最初の奉納\さらば弁護士鉄道・泥棒神社の物語(@hirono\_hideki)のアカウントが同年4月2日の開始です。私はTwitterAPIを多用しているので,いくらでもツイートの数は増やせるのですが,確認すると23.2万のツイートでした。

\begin{itemize}
\item
  TW momo3580(momo3580@意外に多忙) 日時: 2021/05/23 17:28:35 URL:
  \url{https://twitter.com/momo3580/status/1396382539720237057} 
  \textgreater{}
  「道路に紙が散乱している」・・・回収してみたら県警OBが処分した捜査文書
  : 社会 : ニュース : 読売新聞オンライン \url{https://t.co/I9esYbsQVn} 
\item
  「道路に紙が散乱している」・・・回収してみたら県警OBが処分した捜査文書
  : 社会 : ニュース : 読売新聞オンライン \url{https://t.co/eYakYayqM7} 
  同県警が回収すると、名前や住所が記載された起訴状の写しや、徳島県警の職員、徳島県の防犯団体などの名簿類の写しなど約100点があったという。
\item
  「道路に紙が散乱している」・・・回収してみたら県警OBが処分した捜査文書
  : 社会 : ニュース : 読売新聞オンライン \url{https://t.co/eYakYayqM7} 
  男性は5月上旬、自宅で保管していた文書の廃棄を香川県のリサイクル業者に依頼。業者が20日に運んだ際、文書が入った段ボールが国道に落下したらしい。
\end{itemize}

 記事の日付は昨日の23日13時37分ですが,初めて知るニュースです。すごい偶然があるものと感心したのですが,リサイクル業者のトラックから捜査文書のダンボール箱が国道上に落下し,散乱したということです。記述し忘れていた,千葉県の我孫子警察署のことを思い出しました。

 名前も憶えていませんが,2件の強制性交等致傷で逮捕された千葉市の弁護士の,再逮捕も起訴あるいは不起訴のニュースが出ていそうです。すっかり忘れていました。

 台所で昼の洗い物を済ませて戻ったのですが,時刻を見ると20時21分です。今日は月曜日で,イチケイのカラスの放送日だと思い出しました。記録に挑戦しているつもりはないですが,たぶん過去の記録的に家にいながらテレビをつけていません。

 ここで気がついたのも縁なので,録画機器を操作して録画スペースを確保したいと思います。

 時刻は20時34分です。録画機器を見ると,本日5月24日のNEWS7まで録画されていたのですが,イチケイのカラスは4月19日の月曜日を最後に録画されておらず,NEWS9が録画されていました。本日のNEWS9は録画予約を取り消したので録画できると思いますが,万引きと書記官などとなっていました。

 テレビは録画機器の操作が終わるとすぐに消したのですが,つけたときと録画機器のリモコン操作から戻ったときがどちらもCMでした。テレビの音がうるさく感じたのですが,母親も2009年に入院してからテレビをうるさがり,全く見ようとしなくなりました。

 子供の頃からずっと一緒に生活しているときは,銭湯以外に夜に出かけることのない母親で,台所のかたづけのあとは,寝るまでずっと首を左に向け,テレビをみていたという印象しかない母親ですが,いつも満足そうにテレビをみていました。

 2009年3月31日の朝に珠洲市総合病院に救急車で救急搬送されてからずっと半身不随で寝たきり状態の母親ですが,昭和4年1月19日,京都生まれだと聞いています。終の棲家というのも本日取り上げた能登怪奇譚の短編の1つで読んだように思うのですが,情報が見つかりませんでした。

 直接,母親の昔話というのは一度も聞いたことがなかったのですが,その人生を想うとき,市川寛弁護士の物語のような検察批判のツイート,ブログ記事が重なります。では,とりあえず,市川寛弁護士については区切りをつけ,次に移ります。

\begin{itemize}
\tightlist
\item
  〈〈〈 2021/05/24 20:48:43 Linux Emacs: 〈〈〈
\end{itemize}

\hypertarget{ux5e73ux62104ux5e74ux3068ux306fux9006ux306bux306aux308bux5343ux8449ux770cux6211ux5b6bux5b50ux8b66ux5bdfux7f72ux304bux3089ux91d1ux6ca2ux897fux8b66ux5bdfux7f72ux3068ux3044ux3046ux6d41ux308cux4ed9ux53f0ux9ad8ux88c1ux5ca1ux53e3ux57faux4e00ux5224ux4e8bux3078ux306eux6c11ux4e8bux63d0ux8a34ux306eux30cbux30e5ux30fcux30b9ux304cux3072ux304dux304cux306d}{%
\paragraph{平成4年とは逆になる千葉県我孫子警察署から金沢西警察署という流れ、仙台高裁岡口基一判事への民事提訴のニュースがひきがね}\label{ux5e73ux62104ux5e74ux3068ux306fux9006ux306bux306aux308bux5343ux8449ux770cux6211ux5b6bux5b50ux8b66ux5bdfux7f72ux304bux3089ux91d1ux6ca2ux897fux8b66ux5bdfux7f72ux3068ux3044ux3046ux6d41ux308cux4ed9ux53f0ux9ad8ux88c1ux5ca1ux53e3ux57faux4e00ux5224ux4e8bux3078ux306eux6c11ux4e8bux63d0ux8a34ux306eux30cbux30e5ux30fcux30b9ux304cux3072ux304dux304cux306d}}

\begin{itemize}
\tightlist
\item
  〉〉〉 Linux Emacs: 2021/06/04 14:39:40 〉〉〉
\end{itemize}

:CATEGORIES: @kanazawabengosi \#金沢弁護士会 @JFBAsns
日本弁護士連合会(日弁連) \#法務省 @MOJ\_HOUMU \#金沢西警察署
\#岡口基一裁判官

\begin{itemize}
\tightlist
\item
  1393:2021-06-04\_14:36:05 \#告発状 \#\#\#\#
  2019年7月24日にTwitterでブロックされていることに気がついた、令和2年度の金沢弁護士会副会長、樋詰哲朗弁護士(金沢弁護士会)
  \url{https://hirono-hideki.hatenadiary.jp/entry/2021/06/04/143558} 
\end{itemize}

 エントリーナンバーが1393となっている直前のエントリーです。金沢弁護士会の前年度の副会長であった樋詰哲朗弁護士(金沢弁護士会)をメインに設定したのですが、内容があちこちに飛びました。金沢西警察署と能登町死体遺棄事件のことも取り上げています。

 次に我孫子警察署について取り上げることは決めていたのですが、本件告発状のどの主要テーマと関連付けるか分類するのにしばらく悩んでいました。

 しばらく前に我孫子警察署の写真を見つけており、さらに建物の竣工時期も特定できて、平成4年2月当時と変わらない建物であることが確認できました。そこで以前にも調べたことがあったのですが、平成4年当時の金沢西警察署の写真がないかと思ってGoogleで検索を掛けました。

 するとすぐに見つかったのが、金沢西警察署の署長のご挨拶のようなページにある写真でした。警察署長に注目するのは昨日の埼玉県警深谷警察署に続いています。

\begin{quote}
《引用の始まり》
\end{quote}

\begin{quote}
令和2年3月31日付で刑事部捜査第一課長から金沢西警察署長として着任いたしました鈴木でございます。

 金沢西警察署勤務は2度目となり、26年前の巡査部長時代に署の刑事として勤務しておりました。

 当時は、署員数が100人を切る小規模警察署であり、金沢市でも閑静な農漁村地域を受け持ち、歴史のある警察署というイメージでしたが、県庁の移転や西都地区の開発などにより、あっという間に大規模警察署へと変貌を遂げたものであります。
\end{quote}

\begin{quote}
《引用の終わり》
\end{quote}

\begin{itemize}
\tightlist
\item
  金沢西警察署長あいさつ - 石川県警察本部
  \url{https://www2.police.pref.ishikawa.lg.jp/about/about12/about29.html} 
\end{itemize}

 ドラマを見るような発見だったのですが、令和2年3月31日付で金沢西警察署長に着任という始まりで、金沢西警察署の勤務は2度目、前回が26年前で巡査部長時代とありました。「鈴木でございます。」とあるだけで、まだフルネームはわかっていません。

 昨日も考えていたのですが、検察庁の検事正も警察本部長もこれまで着任の挨拶で見てきた年齢は、例外なく50代前半だと思うのです。50代後半だと他にどのようなポストになるのかとも想像したのですが、現場の一線で体力気力の衰えも加味されているのかと想像しました。

 私は今年の11月で57歳になるところですでに50代後半になっているのですが、平成4年4月1日の夜に金沢西警察署に出頭した時点では27歳でした。

 我孫子警察署については全く別のことで取り上げておき必要を考え、予告するような記述もしていたと思うのですが、それも20数年前につながる内容となっていたかと思います。小さいニュースでした。たまたま見つけることができたという感じです。

 短い内容の単発のニュースだったので、まずそちらからご紹介しておきたいと思います。

\begin{itemize}
\tightlist
\item
  2021年06月04日13時49分の登録:
  REGEXP:''岡口''/データベース登録済みツイートの検索:2008-10-08〜2021-06-04/2021年06月04日13時28分の記録:ユーザ・投稿:547/4566件
  \url{https://kk2020-09.blogspot.com/2021/06/regexp2008-10-082021-06.html} 
\item
  2021年06月04日15時04分の登録:
  「我孫子」を@hirono\_hideki @kk\_hirono @s\_hironoで検索 57件の該当 2021-06-04\_15:04の記録
  \url{https://kk2020-09.blogspot.com/2021/06/hironohidekikkhironoshirono572021-06.html} 
\end{itemize}

2020-08-23 16:41:23
``以前,この2月下旬とも3月上旬の運行とも考えていた,我孫子警察署で大型車の通行許可証をもらった運行が,運行日報の記録で,2月13日付の通行許可申請書を確認しています。午前2時から午前10時までともなっています。''
\url{https://twitter.com/kk\_hirono/status/1297438774310260736} 

2021-04-19 14:33:41
``捜査資料6251点、自宅などに隠し持つ 巡査部長を懲戒処分(毎日新聞) -
Yahoo!ニュース \url{https://t.co/lgjvGIg4gM} 
約30年にわたり事件の捜査資料など約6000点を自宅などに隠し持っていたとして、千葉県警は16日、我孫子署の男性巡査部長(49)を停職3カ月の懲戒処分とし、''
\url{https://twitter.com/hirono\_hideki/status/1384017337599160321} 

2021-05-09 16:36:01
``この千葉の弁護士の強制性交等致傷事件については,この事件で思い出した我孫子警察署の警察官の事件と一緒に取り上げておきたいと考えていました。弁護士が警察に掛けた過大な負担や,水面下での攻防のようなものを想像していました。''
\url{https://twitter.com/kk\_hirono/status/1391295882134392834} 

 リンクのあるYahoo!ニュースはリンク切れとなっていました。``捜査資料6251点、自宅などに隠し持つ 巡査部長を懲戒処分(毎日新聞)''が記事の見出しのようですが、これだけで十分に内容がわかりそうです。本文の引用と思われる部分に、約30年にわたり、男性巡査部長49歳とあります。

 もうずっと前から全く見かけないのですが、昭和から平成のはじめ頃は、下っ端の警察官を揶揄するような言葉に「万年巡査」というのがありました。巡査部長というのは下から2番目の階級だったと思います。

 この「我孫子署の男性巡査部長(49)」という部分は、前に間を置いて2回は記事を読んだとき印象には残っていなかったのですが、金沢西警察署の署長が26年前に巡査部長だったのと奇妙な符合を感じました。

 49歳で約30年前ということは19歳には警察官になっていた可能性があり、高卒での採用と思われますが、それから約30年間も続けて、捜査資料6251点を自宅などに隠し持ったというのは、小さいニュースだと思いましたが、内容な歳月の積み重ねもあり、影響も大きかったことでしょう。

 男性巡査部長49歳とあるだけで、処分というのも停職3ヶ月となっています。これまでほぼ例外なく自主退職のなるケースを見ているので、事実上は解雇に等しいのかもしれません。そういえば同じ千葉県警で問題を起こした若い警察官が佐藤大和弁護士に救われたようなニュースがありました。

\begin{itemize}
\tightlist
\item
  2021年06月04日15時25分の登録:
  「佐藤大和」を@hirono\_hideki @kk\_hirono @s\_hironoで検索 382件の該当 2021-06-04\_15:25の記録
  \url{https://kk2020-09.blogspot.com/2021/06/hironohidekikkhironoshirono3822021-06.html} 
\end{itemize}

2018-08-30 21:48:28 ``\textgreater{}
控訴審を担当した佐藤大和弁護士は千葉県警巡査に逆転無罪=公然わいせつ、故意否定-東京高裁:時事ドットコム
\url{https://www.jiji.com/jc/article?k=2018083000864\&g=soc} 
@jijicomさんから''
\url{https://twitter.com/kk\_hirono/status/1035147237892874240} 

2018-08-30 22:04:07
``現役の警察官が若手の佐藤大和弁護士を頼みにして控訴審で逆転の無罪判決を得たというのも、ちょっと想定し難い漫画の世界のような話に思えました。それも千葉県警柏警察署とのことです。私が最も漫画的と思う弁護士の一人、三浦義隆弁護士と同じ千葉県内になります。''
\url{https://twitter.com/kk\_hirono/status/1035151175207636992} 

 個人的に千葉県と聞くだけで、三浦義隆弁護士の強烈なキャラクター性が充満してしまうのですが、問題なく弁護士をつづけ、Twitterでもかなり刺激的な内容のツイートを見かけてきたのですが、それほど大きな問題にはなっていないようですが、危険な曲芸師という印象があります。

 思いつきの方法で検索したのですが、予想より多く数が出てきました。

\begin{lstlisting}
base ❯ d|grep lawkus|grep 素人
\end{lstlisting}

\begin{itemize}
\tightlist
\item
  2017年10月20日22時37分の登録:
  \ystk @lawkus\正面から反論して議論になっちゃったりすると収拾付かなくなるから反論は下策。素人と議論しても無意味だし
  \url{http://hirono2014sk.blogspot.com/2017/10/ystklawkus\_20.html} 
\item
  2017年10月23日06時09分の登録:
  \ystk @lawkus\そりゃ法曹相手に素人が誘導尋問しようとしても成功するわけないわな。
  \url{http://hirono2014sk.blogspot.com/2017/10/ystklawkus\_23.html} 
\item
  2017年10月23日06時13分の登録:
  \弁護士あだちけいた㌠ @keita\_adachi RT: @lawkus\そりゃ法曹相手に素人が誘導尋問しようとしても成功するわけないわな。
  \url{http://hirono2014sk.blogspot.com/2017/10/keitaadachirtlawkus.html} 
\item
  2017年10月26日18時44分の登録:
  REGEXP:''素人''/ystk(@lawkus)の検索(2010-02-02 21:45〜2017-10-23
  00:24/2017年10月26日18時44分の記録170件)
  \url{http://hirono2014sk.blogspot.com/2017/10/regexpystklawkus2010-02-02-21452017-10.html} 
\item
  2017年10月26日18時46分の登録:
  %@lawkus ystk%最近モトケン先生が苛立っているご様子。まあようやく検察にも、裁判所や弁護士や官僚や政治家と同様に「素人の的外れな批判」が盛んに
  \url{http://hirono2014sk.blogspot.com/2017/10/lawkusystk\_26.html} 
\item
  2017年11月21日05時16分の登録:
  \ystk @lawkus\素人が生半可に判決文なんか読むと、「強く抵抗しなかった。よって同意があった。よって無罪」式の乱暴な読み方をしてしまうことがあるが、これはもう仕
  \url{http://hirono2014sk.blogspot.com/2017/11/ystklawkus\_63.html} 
\item
  2017年11月22日21時57分の登録:
  \ystk @lawkus\どうしてツイッタラーは素人のくせにプロに独自見解を披露しようと思ってしまうのか
  \url{http://hirono2014sk.blogspot.com/2017/11/ystklawkus\_13.html} 
\item
  2017年11月22日21時58分の登録:
  \モトケン @motoken\_tw RT: @lawkus\どうしてツイッタラーは素人のくせにプロに独自見解を披露しようと思ってしまうのか
  \url{http://hirono2014sk.blogspot.com/2017/11/motokentwrtlawkus.html} 
\item
  2017年11月24日21時31分の登録:
  \ystk @lawkus\素人丸出しの言い分。元士業といっても弁護士以外だろう。俺が代理人で申立てた労働審判、9割以上は労働審判までで解決してるし、たいてい申立から1〜
  \url{http://hirono2014sk.blogspot.com/2017/11/ystklawkus91.html} 
\item
  2017年11月30日21時30分の登録:
  \ystk @lawkus\素人が適当なこと書くなというのにまったくどいつもこいつも・・・(´・ω・`)
  \url{http://hirono2014sk.blogspot.com/2017/11/ystklawkus\_18.html} 
\item
  2017年12月04日23時39分の登録:
  \ystk @lawkus\同業者なら畏敬するレベルの奥村先生の高い専門性も素人はしばしば理解できないのだから、この世界では玄人と素人の情報の非対称性が本当に大きいなと思
  \url{http://hirono2014sk.blogspot.com/2017/12/ystklawkus\_81.html} 
\item
  2017年12月13日23時17分の登録:
  \ystk @lawkus\「警察官は法律の素人です」 「認めれば不起訴にしてやる
  認めれば執行猶予がある 黙秘すれば認めたのと同じになる
  詳しくは言えないが動かぬ証拠を
  \url{http://hirono2014sk.blogspot.com/2017/12/ystklawkus\_22.html} 
\item
  2017年12月23日09時25分の登録:
  \ystk @lawkus\おっ。twitter名物、プロに講釈してくださる素人さんだ。しかも間違ってるし超カコイイ。
  \url{http://hirono2014sk.blogspot.com/2017/12/ystklawkustwitter\_23.html} 
\item
  2017年12月27日15時43分の登録:
  \ystk @lawkus\「弁護人にもやったと話し」とわざわざ書いてるのに、こういうこと言ってくる素人がいるからなあ。そりゃ、弁護人にやったと言ってる場合に冤罪の危険が
  \url{http://hirono2014sk.blogspot.com/2017/12/ystklawkus\_68.html} 
\item
  2018年01月14日21時00分の登録:
  \ystk @lawkus\トラブルを抱えて不満なのに法律問題ではないということはかなり多い。そこの判断ができるのは法曹だけで、その素人との知識の差こそが我々の飯の種。だ
  \url{http://hirono2014sk.blogspot.com/2018/01/ystklawkus\_14.html} 
\item
  2018年03月28日08時02分の登録:
  \ystk @lawkus\心証が灰色のときは白と扱う決まり(疑わしきは被告人の利益に)もない。この点、半端に黙秘権とか無罪推定とかいうキーワードを知ってる素人ほど誤解し
  \url{http://hirono2014sk.blogspot.com/2018/03/ystklawkus\_28.html} 
\item
  2018年04月19日15時57分の登録:
  \ystk @lawkus\その点、福田事務次官は旧司法試験合格しているらしいが修習行かずに入省しちゃってるのでやっぱり素人。
  \url{http://hirono2014sk.blogspot.com/2018/04/ystklawkus\_35.html} 
\item
  2018年05月31日16時38分の登録:
  \ystk @lawkus\やはり情況証拠からの事実認定は素人には無理だなと思いながら見ていた。
  \url{http://hirono2014sk.blogspot.com/2018/05/ystklawkus\_31.html} 
\item
  2018年06月16日18時38分の登録:
  \ystk @lawkus\反証できるなら社会的地位は低下しないから名誉毀損にならないなどという珍説は今初めて聞いた。俺が繰り返し述べている「素人による法律論は常に間違い
  \url{http://hirono2014sk.blogspot.com/2018/06/ystklawkus\_16.html} 
\item
  2018年06月19日04時20分の登録:
  \ystk @lawkus\¥\n¥\n警察に聞いたというのはつまり素人に聞いたということ。¥\n¥\nystkさんが追加¥\n
  \url{http://hirono2014sk.blogspot.com/2018/06/ystklawkus-ystk.html} 
\item
  2018年11月27日21時02分の登録:
  \ystk @lawkus\代理人弁護士の意図がある程度想像つく感じはある。素人が軽々に代理人が無能なのではとか論評するようなものでもないと思う。¥\n0件の返信
  3件のリツ \url{http://hirono2014sk.blogspot.com/2018/11/ystklawkus-0-3.html} 
\item
  2019年01月12日08時35分の登録:
  REGEXP:''素人''/ystk(@lawkus)の検索(2010-02-02〜2019-01-10/2019年01月12日08時35分の記録226件)
  \url{http://hirono2014sk.blogspot.com/2019/01/regexpystklawkus2010-02-022019-01.html} 
\item
  2019年01月12日08時36分の登録:
  REGEXP:''警察.*素人''/ystk(@lawkus)の検索(2017-06-07〜2018-08-13/2019年01月12日08時36分の記録4件)
  \url{http://hirono2014sk.blogspot.com/2019/01/regexpystklawkus2017-06-072018-08.html} 
\item
  2019年01月13日10時32分の登録:
  \ystk @lawkus\解説しよう。素人がボクサーに喧嘩を売ったがなぜか「ボクシングのルールでやろう」と申し入れてしまったのが話題の大量懲戒請求事件だ。
  \url{http://hirono2014sk.blogspot.com/2019/01/ystklawkus\_13.html} 
\item
  2019年02月06日00時41分の登録:
  \ystk @lawkus\村松謙さんがリツイート> ystk(@lawkus): 素人が捜査段階で被疑者を犯人と決めつけてるの見るとドン引きするけど、冷静に考えてみる
  \url{http://hirono2014sk.blogspot.com/2019/02/ystklawkusystklawkus.html} 
\item
  2019年02月22日22時10分の登録:
  \ystk @lawkus\一般論として、弁護士が素人に内容証明を出すなどして損害賠償請求をする場合、初手は多少吹っかけ気味でも問題ない場合と初手から落とし所に近いライン
  \url{http://hirono2014sk.blogspot.com/2019/02/ystklawkus\_83.html} 
\item
  2019年02月28日11時45分の登録:
  \ystk @lawkus\素人の「弁護士は仕事のため安易に破産を勧めるから破産せず踏ん張った方がいい」とかいう陰謀論レベルの話を安易に信じるくらい安易な人なら、安易に専
  \url{http://hirono2014sk.blogspot.com/2019/02/ystklawkus\_13.html} 
\item
  2019年03月15日06時40分の登録:
  \ystk @lawkus\「性交中だいたい寝てる様子だったけどときどき目を開けたり声も出したりしたことがある」くらいの状態なら素人目に見ても抗拒不能だろうから、話題の判
  \url{http://hirono2014sk.blogspot.com/2019/03/ystklawkus\_15.html} 
\item
  2019年03月15日20時39分の登録:
  \ystk @lawkus\放送中止せざるを得ない)という趣旨のツイートを見た。素人さんの発想は独創的でよろしおすなあという感想。
  \url{http://hirono2014sk.blogspot.com/2019/03/ystklawkus\_59.html} 
\item
  2019年04月08日03時29分の登録:
  \ystk @lawkus\ゴーン氏の妻が現時点で被疑者ですらなく、移動の自由を制限されるべき理由がないことくらい素人でもさすがにわかるはず。たまたま夫に対する捜査の巻添
  \url{http://hirono2014sk.blogspot.com/2019/04/ystklawkus\_8.html} 
\item
  2019年04月13日12時04分の登録:
  \ystk @lawkus\法を学んだことがない素人でも、記事本文に二重の危険云々と書いてあるのを読めるなら、「同じロジック」に基づいて有罪判決に対する控訴禁止論など成り
  \url{http://hirono2014sk.blogspot.com/2019/04/ystklawkus\_13.html} 
\item
  2019年05月27日19時22分の登録:
  \ystk @lawkus\深澤諭史さんがリツイート> ystk(@lawkus): 弁護士つかまえて法の何たるかを解いてくださる素人さんかっこいい。例によって「法治国
  \url{http://hirono2014sk.blogspot.com/2019/05/ystklawkusystklawkus\_27.html} 
\item
  2019年06月05日15時00分の登録:
  \ystk @lawkus\【悲報】元自称憲法学者の素人竹田氏、特殊刑法学にまで手を出してしまう
  \url{http://hirono2014sk.blogspot.com/2019/06/ystklawkus\_58.html} 
\item
  2019年06月07日22時28分の登録:
  \ystk @lawkus\これって大麻を違法どころか犯罪とする理由としては全く足りてないので、EARL先生も医療に関しては立派な専門家でも、まあ素人考えの域を出ないなと
  \url{http://hirono2014sk.blogspot.com/2019/06/ystklawkusearl.html} 
\item
  2019年06月26日17時27分の登録:
  \ystk @lawkus\同サイトによれば、露木幸彦氏は素人レベルの知識に基づきこれまで離婚協議書作成900件を達成しているそうなので、既に多数の顧客に誤った説明をし、
  \url{http://hirono2014sk.blogspot.com/2019/06/ystklawkus900.html} 
\item
  2019年06月26日17時28分の登録:
  \ystk @lawkus\【注意喚起】自称「離婚に特化した」行政書士の露木幸彦氏が、「離婚協議書に包括的清算条項を入れれば後に養育費増額請求ができなくなる」という素人レ
  \url{http://hirono2014sk.blogspot.com/2019/06/ystklawkus\_61.html} 
\item
  2019年07月04日20時31分の登録:
  \ystk @lawkus\「書類送検」に何か特別の(有罪になりそうという方向の)意味を見出したがる素人の誤解は本当によく見るので、少しでも解消したいな。微力だが久々にブ
  \url{http://hirono2014sk.blogspot.com/2019/07/ystklawkus\_4.html} 
\item
  2019年12月06日00時08分の登録:
  \ystk @lawkus\またもへもへがデタラメ言ってるのか。¥\nもへもへに限らず素人にはよく見られる誤解なので、俺のブログでも貼っておきますね。情況(状況)証拠だけで有
  \url{http://hirono2014sk.blogspot.com/2019/12/ystklawkus\_6.html} 
\item
  2019年12月13日11時58分の登録:
  \ystk @lawkus\前にも書いたけど、弁護士と素人の知識の格差こそがまさに独占資格たる弁護士資格の根拠であり、我々の飯のタネです。弁護士から見たら誤っている話を持
  \url{http://hirono2014sk.blogspot.com/2019/12/ystklawkus\_13.html} 
\item
  2020年01月02日02時16分の登録:
  \ystk @lawkus\これ「勝ち取られたのに、」までが完全に誤りだから素人の戯言以上のものではないのだが、玉井克哉先生がこれをRTしているのはどういうつもりなのであ
  \url{http://hirono2014sk.blogspot.com/2020/01/ystklawkusrt.html} 
\item
  2020年01月04日20時51分の登録:
  \ystk @lawkus\「日本の裁判官は買収されないから日本の司法は問題ない」的なことを述べる素人をここ数日で大勢見たが、じゃあ買収されず皆平等に保釈は一切認めず無罪
  \url{http://hirono2014sk.blogspot.com/2020/01/ystklawkus\_19.html} 
\item
  2020年03月11日02時25分の登録:
  \ystk @lawkus\↓適当なことをふかしてイキる素人の例
  \url{http://hirono2014sk.blogspot.com/2020/03/ystklawkus\_46.html} 
\item
  2020年04月04日01時43分の登録:
  \ystk @lawkus\20年以上30年未満の弁護士経験があるが素人並の経験知しか身につかなかった自己紹介はいたたまれないからやめてほしい、
  \url{http://hirono2014sk.blogspot.com/2020/04/ystklawkus2030.html} 
\item
  2020年12月02日23時52分の登録:
  \ystk @lawkus\法クラ弁護士が相手方代理人になったとき修習期を持ち出してマウンティングしようとしたりするのは素人。効果的に困らせたいならTwitterではご活
  \url{http://kk2020-09.blogspot.com/2020/12/ystklawkustwitter.html} 
\item
  2021年05月22日15時26分の登録:
  %@lawkus ystk%教唆になるわけねえだろw
  素人のくせにデタラメ言うなよ馬鹿。
  \url{https://kk2020-09.blogspot.com/2021/05/lawkusystkw.html} 
\end{itemize}

 「看板」が見当たりませんでした。

\begin{itemize}
\item
  奉納\危険生物・弁護士脳汚染除去装置\金沢地方検察庁御中:
  \ystk @lawkus\「警察官は法律の素人です」 「認めれば不起訴にしてやる
  認めれば執行猶予がある 黙秘すれば認めたのと同じになる
  詳しくは言えないが動かぬ証拠を \url{https://t.co/btXaokllnw} 
\item
  TW lawkus(ystk) 日時: 2017/12/13 20:38:50 URL:
  \url{https://twitter.com/lawkus/status/940908871001845761} 
  \textgreater{} 「警察官は法律の素人です」\\
  \textgreater{} 「``認めれば不起訴にしてやる''
  ``認めれば執行猶予がある'' ``黙秘すれば認めたのと同じになる''
  ``詳しくは言えないが動かぬ証拠を押さえてるから自白してしまえ''
  などは全て詐欺です」\\
  \textgreater{}
  などと書いた看板を警察署近辺に設置する活動を、弁護士会がやるべきだと思う
\end{itemize}

 TwitterAPIですぐにツイートの内容が取得できました。今もそのまま残っているツイートのようです。

〉〉〉 kk\_hironoのリツイート 〉〉〉

\begin{itemize}
\tightlist
\item
  RT
  kk\_hirono(刑事告発・非常上告_金沢地方検察庁御中)|s\_hirono(非常上告-最高検察庁御中\_ツイッター)
  日時:2021-06-04 15:54/2021/06/04 15:47 URL:
  \url{https://twitter.com/kk\_hirono/status/1400707502979776512} 
  \url{https://twitter.com/s\_hirono/status/1400705789417848839} 
  \textgreater{}
  2021-06-04-154450\_ystk@lawkus「警察官は法律の素人です」「''認めれば不起訴にしてやる'' ''認めれば執行猶予がある'' ''黙秘すれば認めたのと同じになる.jpg
  \url{https://t.co/Uw8oMlbcr4} 
\end{itemize}

 時計を見ると15時55分でした。さきほどちらりと見たときは17時に見えたので、そういう時間なのかと納得していたのですが、今日は妙に時間の流れが遅く感じます。

 三浦義隆弁護士の強烈な個性で、何を記述するつもりだったのかわからなくなってしまったのですが、しばらく前までは深澤諭史弁護士のTwitterタイムラインでもよくあった現象で、いったいどれだけの時間を無駄にされたのかと、何度も思っています。

 相変わらず本人のツイートがないのですが、だいたい一日1つのペースでリツイートをしているようです。さきほども一つ見かけているのですが、三浦義隆弁護士のことと相まって、かなり考えさせられる内容でした。発言者によっても受け止めが大きく違ってきそうな元のツイートです。

 一部の弁護士の活躍も大きいのか、リツイートすることを本人の発言と同視するような裁判所の判断、判例も出ていたかと思います。リツイートの仕方というのも人それぞれですが、深澤諭史弁護士のリツイートの場合は、特に強い同調性や高い評価を感じさせるものです。

\begin{itemize}
\item
  RT fukazawas(深澤諭史)|O59K2dPQH59QEJx(ピピピーッ)
  日時:2021-06-03 15:42/2021-06-03 15:36 URL:
  \url{https://twitter.com/fukazawas/status/1400342085648060417} 
  \url{https://twitter.com/O59K2dPQH59QEJx/status/1400340576352968707} 
  \textgreater{}
  被告人、弁護人、裁判官の三者が実力を備えていて、さらに運が味方すれば、ようやくたどりつけるのが無罪判決。\\
  \textgreater{}\\
  \textgreater{} 何か一つ欠けていても、たどりつけない、それが無罪判決。
\item
  RT fukazawas(深澤諭史)|shimadayusuke66(島田雄左) 日時:2021-06-04
  12:36/2021-06-02 14:22 URL:
  \url{https://twitter.com/fukazawas/status/1400657630561931266} 
  \url{https://twitter.com/shimadayusuke66/status/1399959710074163203} 
  \textgreater{}
  小さい頃「年収は社会にどれだけ貢献してるかを現してる」と教わりました。実際、社会に価値を提供して、役立てば役立つほど、年収も上がっていく仕組みになってます。社会に役立つというのは、多くの人が抱えてる課題を解決することです。まずは身の回りの人が抱えてる課題は?そこにヒントがあります
\end{itemize}

 昨日の2021-06-03
15:42というリツイートもずいぶんと奥深く考えさせられる内容でした。「被告人、弁護人、裁判官の三者が実力を備えていて、さらに運が味方すれば、ようやくたどりつけるのが無罪判決。」というこれを本気で信じているなら、自分の経験に照らし、狂気を感じてしまいます。

 深澤諭史弁護士が社会貢献をしているつもりなのは間違いなさそうです。納税のツイートもけっこうありましたが、それは事実なのでしょう。なにが社会貢献なのか、弁護士が社会の利益にかなうのか、弁護士という職業について腹の底から考えさせてくれるという意味なら感謝すべき貢献かもしれません。

 岡口基一裁判官の件ですが、ニュースの配信時間が今朝の5時となっているものを1つ見かけたように思うのですが、午前中でも昼に近い時間になって最初に見かけたように思います。それも刑裁サイ太のタイムラインがきっかけだったと思います。

〉〉〉 kk\_hironoのリツイート 〉〉〉

\begin{itemize}
\tightlist
\item
  RT
  kk\_hirono(刑事告発・非常上告_金沢地方検察庁御中)|s\_hirono(非常上告-最高検察庁御中\_ツイッター)
  日時:2021-06-04 16:16/2021/06/04 13:42 URL:
  \url{https://twitter.com/kk\_hirono/status/1400713061028175875} 
  \url{https://twitter.com/s\_hirono/status/1400674222091890689} 
  \textgreater{}
  2021-06-04-133814\_椎名つよし(何でも屋さん)@t\_417\_kawasaki·22分ワタクシ、たまたまその時裁判官訴追委にいた結果、たまたま弾劾裁判を実際にやっ.jpg
  \url{https://t.co/Neal6Ja3zI} 
\end{itemize}

〉〉〉 kk\_hironoのリツイート 〉〉〉

\begin{itemize}
\tightlist
\item
  RT
  kk\_hirono(刑事告発・非常上告_金沢地方検察庁御中)|s\_hirono(非常上告-最高検察庁御中\_ツイッター)
  日時:2021-06-04 16:16/2021/06/04 13:41 URL:
  \url{https://twitter.com/kk\_hirono/status/1400713170839171073} 
  \url{https://twitter.com/s\_hirono/status/1400674077480656906} 
  \textgreater{}
  2021-06-04-132131\_モトケン@motoken\_tw·3時間弁護士は(当事者ではなく)代理人なので冷静になれる、という面がある。でも、ときどき当事者性を持つ弁護士.jpg
  \url{https://t.co/PqtAbh6VWZ} 
\end{itemize}

〉〉〉 kk\_hironoのリツイート 〉〉〉

\begin{itemize}
\tightlist
\item
  RT
  kk\_hirono(刑事告発・非常上告_金沢地方検察庁御中)|s\_hirono(非常上告-最高検察庁御中\_ツイッター)
  日時:2021-06-04 16:17/2021/06/04 13:41 URL:
  \url{https://twitter.com/kk\_hirono/status/1400713194650300420} 
  \url{https://twitter.com/s\_hirono/status/1400673932663951360} 
  \textgreater{}
  2021-06-04-125028\_【写真特集】女子刑務所の高齢受刑者「塀の中のおばあさん」.jpg
  \url{https://t.co/BOypjqHoIY} 
\end{itemize}

〉〉〉 kk\_hironoのリツイート 〉〉〉

\begin{itemize}
\tightlist
\item
  RT
  kk\_hirono(刑事告発・非常上告_金沢地方検察庁御中)|s\_hirono(非常上告-最高検察庁御中\_ツイッター)
  日時:2021-06-04 16:17/2021/06/04 13:40 URL:
  \url{https://twitter.com/kk\_hirono/status/1400713325906849799} 
  \url{https://twitter.com/s\_hirono/status/1400673787536834562} 
  \textgreater{}
  2021-06-04-062101\_弁護士 吉峯耕平(「カンママル」撲滅委員会)さんがリツイート弁護士 市川 寛@imarockcaster42·6月2日捜査機関が証拠を収集す.jpg
  \url{https://t.co/HEsXzZMCK9} 
\end{itemize}

 刑裁サイ太のタイムラインのスクリーンショットがなかったですが、スクリーンショットを記録するだけの必要性は考えていなかったようです。12時50分に女子刑務所の高齢受刑者というスクリーンショットがありますが、これはリンクの記事を読み終えてからの記録でした。

 とても大きな発見に思えたので見つけた経緯の記録を残しておいたのですが、記事というよりほとんどが写真集のようなページで、それもこれまでにない報道のスタイルとして印象的でした。

 女子刑務所のことは昨日も思い出して調べていたのですが、刑事部長のテレビドラマから泉ピン子につながったのがきっかけでした。見つけた女子刑務所のテレビドラマには泉ピン子の出演はなかったかもしれません。

 一昔前に比べると刑務所の話題を見かけることが少ないとも感じていたのですが、笠松刑務所の写真を見て、刑務所の処遇がずいぶん変わり、一般社会の生活と違いが少なくなったように感じました。福祉が充実したとも思えます。

〉〉〉 kk\_hironoのリツイート 〉〉〉

\begin{itemize}
\tightlist
\item
  RT
  kk\_hirono(刑事告発・非常上告_金沢地方検察庁御中)|hirono\_hideki(奉納\さらば弁護士鉄道・泥棒神社の物語)
  日時:2021-06-04 16:31/2021/06/04 06:02 URL:
  \url{https://twitter.com/kk\_hirono/status/1400716858328641538} 
  \url{https://twitter.com/hirono\_hideki/status/1400558461599944705} 
  \textgreater{}
  石川優実さん完全勝訴|弁護士業界にはびこる女性蔑視にモノ申す!|コペルくんwithアヤ先生💕@note大学初代教授|note
  \url{https://t.co/fSKDdfACVF} 
\end{itemize}

〉〉〉 kk\_hironoのリツイート 〉〉〉

\begin{itemize}
\tightlist
\item
  RT
  kk\_hirono(刑事告発・非常上告_金沢地方検察庁御中)|hirono\_hideki(奉納\さらば弁護士鉄道・泥棒神社の物語)
  日時:2021-06-04 16:31/2021/06/04 06:11 URL:
  \url{https://twitter.com/kk\_hirono/status/1400716897126010892} 
  \url{https://twitter.com/hirono\_hideki/status/1400560910398230529} 
  \textgreater{} 私を誹謗中傷した吉峯耕平弁護士へ \textbar{} Yumi
  Ishikawa officialsite \url{https://t.co/yJ7ShRBXvG} 
  ⇛ それでも、私はどうしても反論しておきたい相手がいます。私の本が「著作権法違反だ」と最初に言い出した人、田辺総合法律事務所所属の、吉峯耕平弁護士です。
\end{itemize}

〉〉〉 kk\_hironoのリツイート 〉〉〉

\begin{itemize}
\tightlist
\item
  RT
  kk\_hirono(刑事告発・非常上告_金沢地方検察庁御中)|hirono\_hideki(奉納\さらば弁護士鉄道・泥棒神社の物語)
  日時:2021-06-04 16:32/2021/06/04 12:21 URL:
  \url{https://twitter.com/kk\_hirono/status/1400716957729521664} 
  \url{https://twitter.com/hirono\_hideki/status/1400653975284633600} 
  \textgreater{} 殺人事件遺族が仙台高裁判事を提訴 | 共同通信
  \url{https://t.co/pZxx9iUJ2y} 
\end{itemize}

 上記の12時21分のツイートが最初だったと思います。いくつかニュースになっていましたが、Twitterでの話題性、関心も低そうでした。

 記事によると岡口基一裁判官を提訴したのは当時17歳だった女子高生の両親となっていたように思いますが、これまでは母親の活動した見た覚えがなく、Twitterのアカウントがあったことで調べ始めたのです。女性の名前で実名のアカウントだったと思います。

 実名のアカウントといってもTwitterのプロフィールの名前で、プロフィールの名前は変更されている可能性もあるのですが、遺族が被害者の実名報道にこだわったということからTwitterで、被害者の名前で検索していたところ、画像でしたが、我孫子警察署に出頭したという情報を見たのです。

 事件現場も被疑者、被害者の住所も東京都内となっていましたが、他に調べたところやはり千葉県の我孫子警察署に出頭とありました。

 殺人事件の二日後に電話連絡をした上で出頭となっていたかもしれません。二日後というのは確かだと思います。

 自暴自棄で自殺を考えながら女性に対する殺人事件を起こしたというのは、昨日の立川市の事件とも似ているように思いました。

 強盗致死であったか、もう一つに強姦未遂という罪名、強盗強姦未遂だったかもしれないですが、調べていくうちに性的暴行の痕跡はなかったという情報がありました。自首出頭しているのに無期懲役というのも珍しく感じたのですが、強盗の目的があったとも思えないところでした。

\begin{itemize}
\tightlist
\item
  実名審理を希望した遺族「岩瀬加奈を法廷で呼べてよかった」女子高生殺害に無期懲役
  - 弁護士ドットコム \url{https://t.co/Gv1G1if6xx} 
\end{itemize}

 確認しておきたくて調べたところ、上記の弁護士ドットコムの記事が見つかったのですが、一通り読み終えてから求刑を探したところ、すぐには見つからず、記事の最初の方に「求刑通り無期懲役を言い渡した。」という記載があることに気が付きました。

 個人的に岡口基一裁判官のことが大きく、これまでに何度か調べている事件でしたが、加害者と被害者とは2年間同じコンビニでアルバイトをした関係性があったと知りました。忘れていたか、気に留めていなかった可能性はあるのですが、自分の傷害・準強姦被告事件により近いものを感じました。

 コンビニでアルバイトをしたという経験はないですが、2年間というのは長い関係性に思え、加害者がどこまで事実を述べているのか疑問にも思え、自虐的により重い刑罰を望んで受けたようにも思えてきました。被告人側の弁護士の存在が全く感じられないのも特徴です。

〉〉〉 kk\_hironoのリツイート 〉〉〉

\begin{itemize}
\tightlist
\item
  RT
  kk\_hirono(刑事告発・非常上告_金沢地方検察庁御中)|hirono\_hideki(奉納\さらば弁護士鉄道・泥棒神社の物語)
  日時:2021-06-04 17:03/2021/06/04 12:36 URL:
  \url{https://twitter.com/kk\_hirono/status/1400724850130317313} 
  \url{https://twitter.com/hirono\_hideki/status/1400657623066705923} 
  \textgreater{}
  トイレ紙持ち去りの警察署長、官舎からさらに13個見つかる(読売新聞オンライン)
  - Yahoo!ニュース \url{https://t.co/t9DkZjNYs5} 
\end{itemize}

 岡口基一裁判官が提訴されたというニュースより前に見かけていたとばかり思っていたのですが、12時36分のツイートになっていた「トイレ紙持ち去りの警察署長、官舎からさらに13個見つかる(読売新聞オンライン)
- Yahoo!ニュース」というニュースです。

 この続報を見てずいぶん印象も変わったのですが、無料のトイレットペーパーを使う目的だけで持ち去りをしていたようです。それも警察署長の官舎とあるので、漫画家の発想にも出てこないような未知との遭遇です。

 10年ほど前は公衆のトイレに持ち去らないで、という張り紙を見かけていたような気がしますが、いつの間にか全く見かけることもなくなっていたので、自宅に持ち去るという発想自体がなく、それも警察署長というのですから驚きも大きいニュースです。テレビは相変わらずつけていませんが、気になります。

 思えば警察署長の年収というのは見かけたこともないのですが、一般の警察官でも750万円ほどと以前に情報を見かけたように思います。例えば子供の進学とか住宅ローンとか事情はあったのかもしれないですが、ドラッグストアで298円の値札で買ってきても10個ぐらいは入っていたように思います。

 見つかったときはトイレットペーパーを5個持っていたそうですが、手に持って歩いていれば目立つし通報されるリスクを考えなかったのか不思議です。そういえば初めのニュースに、酒に酔っていたという話がありました。

 記憶にないまま酔っ払って変なものを自宅に持って帰るという話は、ネットで見たような気もしますが、魔が差したとも思えず、不思議な力が働いたようなとても不可解な事件です。

 一方で、最初のきっかけより事後対応が裏目裏目に出て、坂道を転げ落ちているように思えるのが岡口基一裁判官です。弾劾裁判や罷免に近づいたように思えるのですが、罷免になれば退職金もなくなり、出版した本の印税も、本の出版の継続自体が大問題になりうるという指摘を前にみていると思います。

 名誉毀損は民法の不法行為になるはずなので3年の時効にかかるはずですが、あえてこの時期を狙ってきたとしか思えず、遺族の凄まじい執念を感じました。まるで神社のお百度参りです。

〉〉〉 kk\_hironoのリツイート 〉〉〉

\begin{itemize}
\tightlist
\item
  RT
  kk\_hirono(刑事告発・非常上告_金沢地方検察庁御中)|t\_417\_kawasaki(椎名つよし(何でも屋さん))
  日時:2021-06-04 17:34/2021/06/04 13:15 URL:
  \url{https://twitter.com/kk\_hirono/status/1400732591796281344} 
  \url{https://twitter.com/t\_417\_kawasaki/status/1400667522194546690} 
  \textgreater{}
  ワタクシ、たまたまその時裁判官訴追委にいた結果、たまたま弾劾裁判を実際にやったことがあります(証拠等関係カードを読み上げただけですがw)。
  訴追委の実務はそこそこ知ってますが、岡口判事の行為は弾劾法第2条の「著しい」要件で跳ねられるべきと直感的に思います。
  \url{https://t.co/4VlczeHGwQ} 
\end{itemize}

〉〉〉 kk\_hironoのリツイート 〉〉〉

\begin{itemize}
\tightlist
\item
  RT
  kk\_hirono(刑事告発・非常上告_金沢地方検察庁御中)|t\_417\_kawasaki(椎名つよし(何でも屋さん))
  日時:2021-06-04 17:34/2021/06/04 13:35 URL:
  \url{https://twitter.com/kk\_hirono/status/1400732660633280515} 
  \url{https://twitter.com/t\_417\_kawasaki/status/1400672576247005186} 
  \textgreater{}
  裁判官訴追委は真面目にやると意外に大変で、1国会に1回程度、200件程度まとめて、訴追・不訴追を決めるために委員会を開きます。弾劾裁判は開かれなくても訴追委は開催するんですね。私は調査小委員だったから、訴追委開く前の小委員会で事務方交えて、各案件の訴追・不訴追方針を協議していました
\end{itemize}

〉〉〉 kk\_hironoのリツイート 〉〉〉

\begin{itemize}
\tightlist
\item
  RT
  kk\_hirono(刑事告発・非常上告_金沢地方検察庁御中)|t\_417\_kawasaki(椎名つよし(何でも屋さん))
  日時:2021-06-04 17:34/2021/06/04 13:36 URL:
  \url{https://twitter.com/kk\_hirono/status/1400732677779558400} 
  \url{https://twitter.com/t\_417\_kawasaki/status/1400672814529597446} 
  \textgreater{}
  裁判官訴追委に来る案件の大半が、その裁判官の書いた判決の内容に対する不満から裁判官を訴追したいと申し出るものです。判決への不満は、三審制の制度の中で上訴という形で表現すべきもので、裁判官を訴追することは適当ではなく、訴追委は、こういう案件はほぼノータイムで不訴追の決定を出します
\end{itemize}

〉〉〉 kk\_hironoのリツイート 〉〉〉

\begin{itemize}
\tightlist
\item
  RT
  kk\_hirono(刑事告発・非常上告_金沢地方検察庁御中)|t\_417\_kawasaki(椎名つよし(何でも屋さん))
  日時:2021-06-04 17:34/2021/06/04 13:38 URL:
  \url{https://twitter.com/kk\_hirono/status/1400732702685364224} 
  \url{https://twitter.com/t\_417\_kawasaki/status/1400673335206301696} 
  \textgreater{}
  分限裁判を既に経ているもの、刑事事件化しているものなど、弾劾の可能性が一定程度あるものについては、事務方がセレクトして、判決不満系の訴追申出とは別に協議をできるようにしてくれており、訴追委の小委員会を構成する国会議員が小委員会で訴追不訴追の方針を議論します。
\end{itemize}

〉〉〉 kk\_hironoのリツイート 〉〉〉

\begin{itemize}
\tightlist
\item
  RT
  kk\_hirono(刑事告発・非常上告_金沢地方検察庁御中)|t\_417\_kawasaki(椎名つよし(何でも屋さん))
  日時:2021-06-04 17:34/2021/06/04 13:41 URL:
  \url{https://twitter.com/kk\_hirono/status/1400732712772636673} 
  \url{https://twitter.com/t\_417\_kawasaki/status/1400673927953731588} 
  \textgreater{}
  小委員会での協議については、小委員会を構成する国会議員が、裁判官弾劾法第2条の弾劾要件である「職務上の義務に著しく違反」、「職務を甚だしく怠つた」、「威信を著しく失うべき非行」があるかを協議します。
  私がいたときは割とフリーディスカッションでした。
\end{itemize}

〉〉〉 kk\_hironoのリツイート 〉〉〉

\begin{itemize}
\tightlist
\item
  RT
  kk\_hirono(刑事告発・非常上告_金沢地方検察庁御中)|t\_417\_kawasaki(椎名つよし(何でも屋さん))
  日時:2021-06-04 17:34/2021/06/04 13:43 URL:
  \url{https://twitter.com/kk\_hirono/status/1400732721916243969} 
  \url{https://twitter.com/t\_417\_kawasaki/status/1400674513164013571} 
  \textgreater{}
  訴追委の委員長が寛大な故鳩山邦夫先生であり、私が法曹出身だというのもあって、プロなんだから色々意見して欲しい、みたいな促しをいただき、調子に乗った私は、逐条解説と先例を読み込み、事前に会館の部屋で訴追委の事務方と各案件の内容について協議したうえで、小委員会に臨んだりしていました
\end{itemize}

〉〉〉 kk\_hironoのリツイート 〉〉〉

\begin{itemize}
\tightlist
\item
  RT
  kk\_hirono(刑事告発・非常上告_金沢地方検察庁御中)|t\_417\_kawasaki(椎名つよし(何でも屋さん))
  日時:2021-06-04 17:34/2021/06/04 13:48 URL:
  \url{https://twitter.com/kk\_hirono/status/1400732731173007362} 
  \url{https://twitter.com/t\_417\_kawasaki/status/1400675858101116932} 
  \textgreater{}
  そんな感じで訴追委で仕事していたので、感覚的に理解しますが、岡口判事は分限裁判で2度戒告を受けているので、訴追委の中では判決不満系の訴追申出とは別扱いになっているはずだし、事務方が取り上げて、小委員会で協議はしてると思います。場合によっては、継続協議にして次国会持越しもあります
\end{itemize}

〉〉〉 kk\_hironoのリツイート 〉〉〉

\begin{itemize}
\tightlist
\item
  RT
  kk\_hirono(刑事告発・非常上告_金沢地方検察庁御中)|t\_417\_kawasaki(椎名つよし(何でも屋さん))
  日時:2021-06-04 17:34/2021/06/04 13:50 URL:
  \url{https://twitter.com/kk\_hirono/status/1400732740471775232} 
  \url{https://twitter.com/t\_417\_kawasaki/status/1400676356434792449} 
  \textgreater{}
  そのうえで、直感的な結論ですが、「威信を著しく失うべき非行」とは言えないと思うんですね。まぁ分限裁判で戒告を受けているので、非行なのかもしれませんが、「威信を著しく失う」かどうか、という意味でいえば、やっぱ違うんだと思います。
\end{itemize}

〉〉〉 kk\_hironoのリツイート 〉〉〉

\begin{itemize}
\tightlist
\item
  RT
  kk\_hirono(刑事告発・非常上告_金沢地方検察庁御中)|t\_417\_kawasaki(椎名つよし(何でも屋さん))
  日時:2021-06-04 17:34/2021/06/04 13:52 URL:
  \url{https://twitter.com/kk\_hirono/status/1400732750164856845} 
  \url{https://twitter.com/t\_417\_kawasaki/status/1400676752809103364} 
  \textgreater{}
  結局、弾劾裁判は、司法権に対する民主的コントロールの発露であって、公正な裁判をするために一定程度身分保障がなされた裁判官について、民意の代表者である国会による決議を経ないと弾劾できないという制度なので、基本的には抑制的であるべきだと思っています。
\end{itemize}

〉〉〉 kk\_hironoのリツイート 〉〉〉

\begin{itemize}
\tightlist
\item
  RT
  kk\_hirono(刑事告発・非常上告_金沢地方検察庁御中)|t\_417\_kawasaki(椎名つよし(何でも屋さん))
  日時:2021-06-04 17:34/2021/06/04 14:25 URL:
  \url{https://twitter.com/kk\_hirono/status/1400732762378706944} 
  \url{https://twitter.com/t\_417\_kawasaki/status/1400685096496099335} 
  \textgreater{}
  国会は国民の民度を表すので、国民が岡口判事disをして世論形成されれば、訴追委を構成する国会議員が世論に左右され「2度の戒告で合わせ技一本」との判断で訴追決定する余地は無くはないと思いますが、戒告は柔道でいえば、今はなき効果とか有効だと思うので、合わせ技一本はないと思うんですよねぇ
\end{itemize}

 我孫子警察署と金沢西警察署については、稿を改めて続きを記述します。

\hypertarget{ux5f01ux8b77ux58ebux9244ux9053ux306eux6b74ux53f2ux63a2ux8a2aux5965ux80fdux767bux7d00ux884cux7de8ux9023ux7d9aux30c6ux30ecux30d3ux5c0fux8aacux307eux308cux306eux821eux53f0ux3068ux306aux3063ux305fux8f2aux5cf6ux5e02ux5927ux6ca2ux3068ux307bux3046ux305fux308dux3046ux3068ux3044ux3046ux5996ux602aux306eux98a8ux5442ux6577ux30a2ux30a4ux30b3ux30f3ux306etwitterux30a2ux30abux30a6ux30f3ux30c8ux306eux3053ux3068}{%
\paragraph{弁護士鉄道の歴史探訪・奥能登紀行編:「連続テレビ小説まれ」の舞台となった輪島市大沢と、「ほうたろう」という妖怪の風呂敷アイコンのTwitterアカウントのこと}\label{ux5f01ux8b77ux58ebux9244ux9053ux306eux6b74ux53f2ux63a2ux8a2aux5965ux80fdux767bux7d00ux884cux7de8ux9023ux7d9aux30c6ux30ecux30d3ux5c0fux8aacux307eux308cux306eux821eux53f0ux3068ux306aux3063ux305fux8f2aux5cf6ux5e02ux5927ux6ca2ux3068ux307bux3046ux305fux308dux3046ux3068ux3044ux3046ux5996ux602aux306eux98a8ux5442ux6577ux30a2ux30a4ux30b3ux30f3ux306etwitterux30a2ux30abux30a6ux30f3ux30c8ux306eux3053ux3068}}

\begin{itemize}
\tightlist
\item
  〉〉〉 Linux Emacs: 2021/06/06 15:20:12 〉〉〉
\end{itemize}

:CATEGORIES: @kanazawabengosi \#金沢弁護士会 @JFBAsns
日本弁護士連合会(日弁連) \#法務省 @MOJ\_HOUMU \#金沢西警察署
\#谷内孝志警部補

 まず、奥能登ということで珠洲警察署のカテゴリーに入れるつもりだったのですが、ブックマークで金沢西警察署に飛んでいて、そこに見出しを置いてから気が付きました。やり直そうと考えたのですが、珠洲市と輪島市は違いもあって、金沢西警察署の方が谷内孝志警部補で近いという繋がりがあります。

 取調室で私の質問に輪島市の出身と答えていた谷内孝志警部補ですが、どうも初めから輪島市内の中心部というイメージはなく、大沢のある西保地区や輪島市でも離れた場所になりますが、町野の辺りをイメージしていました。

 西保地区というのも近年しったのですが、平成4年当時は、輪島市でも西側の外れで、鳳至郡門前町と隣接していたはずです。違ったルートもあるのですが、だいたいは海岸線で大沢、上大沢を通り、皆月に出て、猿山岬に行っていました。今Googleマップで見ると大沢町とありました。

\begin{itemize}
\tightlist
\item
  皆月海岸 から 大沢町 - Google マップ \url{https://t.co/PwkfMlv0SL} 
\end{itemize}

 Googleマップで調べると、大沢から皆月は13.5kmと思ったより遠くはないのですが、これ以上の距離は感じました。まだ皆月の方が店もあり家の数も多そうですが、皆月町とは出てきませんでした。

 さきほど「ほうたろう」についてまとめ記事を作成してあります。少し目を通しているのですが、前回調べたときと同じく記憶の時期にずれがあって、2015年の秋でも11月になっていたのですが、連続テレビ小説まれ、の放送は10月の中頃に終わっていたと思います。

 ドラマでは逆恨みをされたまれの父親が、地元でいたずらをされるという場面があり、「ぺ」と間垣にあったのが気になったので調べ始めたところ、行き着いたのが、ちゃんぺ饅頭と七尾市の大地主神社でした。ほうたろうと出会ったのも同じ頃と思っていたのですが、だいぶんズレがあったようです。

\begin{itemize}
\tightlist
\item
  2021年06月06日14時09分の登録:
  「@lawyerhotaro」を@hirono\_hideki @kk\_hirono @s\_hironoで検索 358件の該当 2021-06-06\_14:09の記録
  \url{https://kk2020-09.blogspot.com/2021/06/lawyerhotarohironohidekikkhironoshirono.html} 
\item
  2021年06月06日14時10分の登録:
  @lawyerhotaro(ほうたろう)のツイート ''.*'' 3219/3219:2018-11-27\_0548〜2021-06-04\_1143 2021年06月06日14時10分の記録
  \url{https://kk2020-09.blogspot.com/2021/06/lawyerhotaro321932192018-11-2705482021.html} 
\end{itemize}

2015-11-17 21:16:44 ``,@lawyerhotaro
はじめまして、ほうたろうさん。新宿の弁護士とプロフィールにありますが、一つ疑問があります。アイコンの背中の担いだ風呂敷は、「泥棒」をイメージされたものなのでしょうか? 妖怪にも見えますが、私の「泥棒神社の弁護士界」というコンセプトに合致します。''
\url{https://twitter.com/hirono\_hideki/status/666590791473565696} 

 3件目のツイートとして記録されていますが、3分前の21時13分から始まったもので14分、16分という短時間での経過です。

2015-11-17 21:18:15 ``RT @lawyerhotaro:
無料相談なんてやめなされ。占い師なんて30分で2万円ですよ。少なくともみんな占い師以上の情報は提供できるでしょ。''
\url{https://twitter.com/hirono\_hideki/status/666591174270947328} 

\begin{itemize}
\item
  〉〉〉 アカウント(@hirono\_hideki)は,@kk\_hironoをブロックしています。リツイートできませんでした。
  〉〉〉 ¥\n ¥\n \url{https://t.co/Q1rAxi1BT8} 
\item
  TW hirono\_hideki(奉納\さらば弁護士鉄道・泥棒神社の物語)
  日時:2015/11/17 21:18:15 URL:
  \url{https://twitter.com/hirono\_hideki/status/666591174270947328} 
  \textgreater{} RT @lawyerhotaro:
  無料相談なんてやめなされ。占い師なんて30分で2万円ですよ。少なくともみんな占い師以上の情報は提供できるでしょ。
\end{itemize}

※ @kk\_hironoのアカウントがブロックされ,リツイートに失敗したツイート

\begin{itemize}
\tightlist
\item
  TW lawyerhotaro(ほうたろう) 日時:2015/11/17 15:14:55 URL:
  \url{https://twitter.com/lawyerhotaro/status/666499740792545280} 
  \textgreater{}
  無料相談なんてやめなされ。占い師なんて30分で2万円ですよ。少なくともみんな占い師以上の情報は提供できるでしょ。
\end{itemize}

 そういえば、ほうたろういうアカウントには占い師のイメージも感じていたのですが、このツイートのことは忘れていました。

 批判が多く目についた連続テレビ小説まれ、でしたが、母娘が大沢の神社の賽銭箱からお釣りを取り出そうとする場面がありました。そんな風景によく溶け込んでいたのが、妖精のようにも思えたほうたろうという謎のTwitterアカウントでした。

〉〉〉 kk\_hironoのリツイート 〉〉〉

\begin{itemize}
\tightlist
\item
  RT
  kk\_hirono(刑事告発・非常上告_金沢地方検察庁御中)|s\_hirono(非常上告-最高検察庁御中\_ツイッター)
  日時:2021-06-06 15:55/2015/11/17 21:17 URL:
  \url{https://twitter.com/kk\_hirono/status/1401432480347947010} 
  \url{https://twitter.com/s\_hirono/status/666591051579334656} 
  \textgreater{}
  2015-11-17-211741\_廣野秀樹\さらば弁護士鉄道・泥棒神社の夜さんはTwitterを使っています: '',@lawyerhotaro はじめまして、ほうたろうさん。.jpg
  \url{https://t.co/qEzubEZALq} 
\end{itemize}

 大地主神社について調べてみます。

〉〉〉 kk\_hironoのリツイート 〉〉〉

\begin{itemize}
\tightlist
\item
  RT
  kk\_hirono(刑事告発・非常上告_金沢地方検察庁御中)|hirono\_hideki(奉納\さらば弁護士鉄道・泥棒神社の物語)
  日時:2021-06-06 15:56/2015/08/12 12:03 URL:
  \url{https://twitter.com/kk\_hirono/status/1401432889435189250} 
  \url{https://twitter.com/hirono\_hideki/status/631300028368330754} 
  \textgreater{} 「七尾市最大の神社で
  青柏祭や祇園祭などの七尾市中心部の祭の拠点である。」「明治15年(1882年)、山王社と祇園牛頭天王社(八坂神社から勧請)を統合し大地主神社と改称。」 
  \url{http://t.co/DdElijfRQL} 
\end{itemize}

〉〉〉 kk\_hironoのリツイート 〉〉〉

\begin{itemize}
\item
  RT
  kk\_hirono(刑事告発・非常上告_金沢地方検察庁御中)|hirono\_hideki(奉納\さらば弁護士鉄道・泥棒神社の物語)
  日時:2021-06-06 15:57/2016/02/06 13:49 URL:
  \url{https://twitter.com/kk\_hirono/status/1401432988303335427} 
  \url{https://twitter.com/hirono\_hideki/status/695831747036971013} 
  \textgreater{}
  ``石川県七尾市大地主神社(おおとこぬし神社)のちゃんぺ饅頭'' /
  ``眞鍋かをり説「ひな祭りの菱餅は''女性自身の象徴``」大竹まこと採用 -
  Docspot.fr \textbar{} Facilite les recherches de vidéo・・・''
  \url{https://t.co/TT320klpHs} 
\item
  奉納\さらば弁護士鉄道・泥棒神社の物語(@hirono\_hideki)/「大地主神社」の検索結果
  - Twilog \url{https://t.co/6yQJFkjWNb} 
\end{itemize}

 2015年に大地主神社のツイートは8月12日1件だけで、ちゃんペ饅頭が出てきたのも2016年2月6日のツイートです。

\begin{itemize}
\tightlist
\item
  奉納\さらば弁護士鉄道・泥棒神社の物語(@hirono\_hideki)/「ちゃんぺ饅頭」の検索結果
  - Twilog \url{https://t.co/CAFggQDNs1} 
\end{itemize}

 やはり「ちゃんぺ饅頭」は、2016年2月6日が最初ですが、最初に調べたときは、その前に「加藤ちゃんぺ」が出てきたという記憶がはっきりあります。芸名だったらしいのですが、知らなかったもので、少なくとも石川県内では問題がありそうでした。

〉〉〉 kk\_hironoのリツイート 〉〉〉

\begin{itemize}
\item
  RT
  kk\_hirono(刑事告発・非常上告_金沢地方検察庁御中)|hirono\_hideki(奉納\さらば弁護士鉄道・泥棒神社の物語)
  日時:2021-06-06 16:03/2018/08/19 19:54 URL:
  \url{https://twitter.com/kk\_hirono/status/1401434550010482694} 
  \url{https://twitter.com/hirono\_hideki/status/1031132342230765569} 
  \textgreater{}
  19時52分ごろ、テレビに「加藤ちゃんぺ」というお茶のCM。見たのは2回目かもしれない。前回は銭湯のテレビだったかもしれない。
\item
  奉納\さらば弁護士鉄道・泥棒神社の物語(@hirono\_hideki)/「加藤ちゃんぺ」の検索結果
  - Twilog \url{https://t.co/gqaCGx0uey} 
\end{itemize}

 1件だけ見つかったツイートには、「加藤ちゃんぺ」をお茶のCMとしているのですが、これも理解と納得のできないところです。

〉〉〉 kk\_hironoのリツイート 〉〉〉

\begin{itemize}
\tightlist
\item
  RT
  kk\_hirono(刑事告発・非常上告_金沢地方検察庁御中)|musicapiccolino(ひぞっこ)
  日時:2021-06-06 16:06/2015/08/12 08:49 URL:
  \url{https://twitter.com/kk\_hirono/status/1401435267085455360} 
  \url{https://twitter.com/musicapiccolino/status/631251182292353025} 
  \textgreater{}
  \#まれ【イタズラ】間垣に「ぺ」の赤い文字。一瞬、ペンキか?スプレーか?とドキッとしたが、よくよく見ると
  苦竹(ニガタケ)1本1本に、丁寧に赤いリボンが結び付けてある。原状回復までを考慮した、ある意味
  優しいイタズラかもしれないww \url{http://t.co/G8SIEh9AFK} 
\end{itemize}

 2015年8月12日の放送だったようです。

\begin{itemize}
\tightlist
\item
  奉納\さらば弁護士鉄道・泥棒神社の物語(@hirono\_hideki)/2018年08月12日
  - Twilog \url{https://t.co/7afwzke84X} 
\end{itemize}

 2015年8月12日のツイートは全部で6件しかなく拍子抜けしました。

 時刻は16時35分です。2015年8月から11月までのスクリーンショットの記録をみていたのですが、11月17日までほうたろうのツイートを記録したツイートがありませんでした。他にもいろいろあって、特にモトケンこと矢部善朗弁護士(京都弁護士会)と深澤諭史弁護士の社会的攻撃性が顕著に感じられました。

〉〉〉 kk\_hironoのリツイート 〉〉〉

\begin{itemize}
\tightlist
\item
  RT
  kk\_hirono(刑事告発・非常上告_金沢地方検察庁御中)|s\_hirono(非常上告-最高検察庁御中\_ツイッター)
  日時:2021-06-06 16:43/2021/06/06 16:42 URL:
  \url{https://twitter.com/kk\_hirono/status/1401444608102387721} 
  \url{https://twitter.com/s\_hirono/status/1401444457585594376} 
  \textgreater{}
  2015-11-17-212551\_ほうたろうさんはTwitterを使っています: ''弁護士は相手方や依頼者から逆恨みされたり反社会的勢力の標的になったりする危険が常にある仕事.jpg
  \url{https://t.co/VbRc9ZkMY0} 
\end{itemize}

〉〉〉 kk\_hironoのリツイート 〉〉〉

\begin{itemize}
\tightlist
\item
  RT
  kk\_hirono(刑事告発・非常上告_金沢地方検察庁御中)|s\_hirono(非常上告-最高検察庁御中\_ツイッター)
  日時:2021-06-06 16:43/2021/06/06 16:42 URL:
  \url{https://twitter.com/kk\_hirono/status/1401444632291004417} 
  \url{https://twitter.com/s\_hirono/status/1401444443794776072} 
  \textgreater{}
  2015-11-17-211341\_ほうたろう(@lawyerhotaro)さん | Twitter - Mozilla Firefox.jpg
  \url{https://t.co/4VXOlC8zGU} 
\end{itemize}

 「弁護士は相手方や依頼者から逆恨みされたり反社会的勢力の標的になったりする危険が常にある仕事」というほうたろうのツイートのスクリーンショットが出てきましたが、逆恨みといえば深澤諭史弁護士も徹底していました。

 モトケンこと矢部善朗弁護士(京都弁護士会)の刺激的なツイートがとりわけよく目立ったのですが、みているだけで疲れました。最近はツイートの数自体が少なくなっているようですが、かなり批判も受けていたので、モトケンこと矢部善朗弁護士の思惑が通用しなくなった部分もありそうです。

 北周士弁護士のツイートも2015年11月までスクリーンショットで記録したものが多かったですが、当初のアカウントと変わりがないように思えました。ただ、気になるヘッダ画像があって、それを見た後に大きな変化があったような気もするのです。

 時刻は17時11分です。次に北周士弁護士について取り上げたいとも考えているのですが、被告発人岡田進弁護士から軸足を話すと集中力が途切れがちとなり、モトケンこと矢部善朗弁護士(京都弁護士会)の過去のツイートに重苦しい重圧も感じるようになりました。

 時刻は6月7日5時59分です。昨日の夕方、モトケンこと矢部善朗弁護士(京都弁護士会)のツイートを見たことで、そちらに意識が逸れてしまい集中した取り組みが出来なくなってしまいました。

 見出しを確認したところ、連続テレビ小説まれ、とほうたろうについてとりあげていたところで、一通りのことは書き尽くしていたように思います。スクリーンショットを探したことで新たに視野が開ける発見があったのですが、モトケンこと矢部善朗弁護士(京都弁護士会)のものもありました。

 夕方に発見したモトケンこと矢部善朗弁護士(京都弁護士会)のツイートは、過去を整理するという意義もあるので、よいタイミングで出現したと思えるような内容のものでした。モトケンこと矢部善朗弁護士(京都弁護士会)のことは珠洲警察署と関連付けて記録したいと思います。

\begin{itemize}
\tightlist
\item
  〈〈〈 2021/06/07 06:07:54 Linux Emacs: 〈〈〈
\end{itemize}

\hypertarget{section-3}{%
\paragraph{}\label{section-3}}

\hypertarget{ux91d1ux6ca2ux5f01ux8b77ux58ebux4f1a}{%
\subsection{金沢弁護士会}\label{ux91d1ux6ca2ux5f01ux8b77ux58ebux4f1a}}

\hypertarget{ux6cd5ux5f8bux76f8ux8ac7}{%
\subsubsection{法律相談}\label{ux6cd5ux5f8bux76f8ux8ac7}}

\hypertarget{ux305fux3076ux3093ux5e73ux621014ux5e74ux4e03ux5c3eux5e02ux306eux4e03ux5c3eux30b5ux30f3ux30e9ux30a4ux30d5ux30d7ux30e9ux30b6ux3067ux306eux7121ux6599ux6cd5ux5f8bux76f8ux8ac7ux3068ux56f3ux66f8ux9928ux306eux544aux8a34ux72b6ux544aux767aux72b6ux30e2ux30c7ux30ebux6587ux4f8bux96c6}{%
\paragraph{たぶん平成14年,七尾市の七尾サンライフプラザでの無料法律相談と,図書館の「告訴状・告発状モデル文例集」}\label{ux305fux3076ux3093ux5e73ux621014ux5e74ux4e03ux5c3eux5e02ux306eux4e03ux5c3eux30b5ux30f3ux30e9ux30a4ux30d5ux30d7ux30e9ux30b6ux3067ux306eux7121ux6599ux6cd5ux5f8bux76f8ux8ac7ux3068ux56f3ux66f8ux9928ux306eux544aux8a34ux72b6ux544aux767aux72b6ux30e2ux30c7ux30ebux6587ux4f8bux96c6}}

\begin{itemize}
\tightlist
\item
  〉〉〉 Linux VsCode: 2021-05-05 10:21:41 〉〉〉
\end{itemize}

:CATEGORIES: @kanazawabengosi \#金沢弁護士会 @JFBAsns
日本弁護士連合会(日弁連) \#法務省 @MOJ\_HOUMU

\begin{quote}
《引用の始まり》
\end{quote}

\begin{quote}
新型コロナウイルス感染拡大防止のため、今年も山車(だし)「でか山」の巡行が中止となった七尾市の青柏祭。若衆らが三日、でか山の展示や木遣(や)り唄の披露で伝統の祭りの灯を受け継ぐ意気込みを見せ、住民らは「来年こそは」の思いを募らせた。(大野沙羅) 例年でか山を運行する三町のうち、唯一展示を決めた府中町が地元の印鑰(いんにゃく)神社で、でか山をお披露目した。上部に作られた舞台は歌舞伎の名場面で、「橋弁慶」がテーマ。武蔵坊弁慶、源義経、義経の母の常盤御前(ときわごぜん)の人形三体を飾った。 神社周辺には屋台約二十店が出たほか、若衆ら約三十〜四十人による祭りの祝い唄「七尾まだら」や木遣り唄なども披露され、多くの人が見物に集まった。
\end{quote}

\begin{quote}
《引用の終わり》
\end{quote}

\begin{itemize}
\tightlist
\item
  でか山展示
  伝統つなぐ 「コロナ禍の七尾・青柏祭」府中町が披露:北陸中日新聞Web \url{https://www.chunichi.co.jp/article/247856?rct=k\_ishikawan} 
\end{itemize}

 上記のニュースの内容ですが,印鑰神社とあります。昨日の5月4日のことになると思いますが,初めて目にする神社のように思い,場所を調べたところ全く意外な場所で驚きました。七尾市の港の近くで食祭市場の向かいのような場所でした。

 七尾市の青柏祭のデカ山は府中町,鍛冶町,魚町の3つと知っていましたが,府中町は港から離れた国道159号線沿いの場所だと思っていました。妙な記憶違いもあるものだと思ったのですが,調べると記憶にあった場所は本府中町となっていました。もともと数年前,Googleマップで見かけた地名でした。

\begin{itemize}
\item
  のと鉄道 ディーゼル - Google 検索 \url{https://t.co/4GXCAYPER2} 
\item
  のと鉄道車両 \url{https://t.co/mlzEm9SgNu} 
\end{itemize}

 言葉の確認のため調べたのですが,国鉄時代の車両と同じく気動車となっていました。個人的には聞き慣れない見慣れない言葉で違和感があるのですが,地元ではずっと「きしゃ」と呼ばれていました。汽車というのは蒸気機関車のことだという話もあるので,言葉選びに困っていました。

 これまで自分から調べた以外に気動車を見かけることはなく,唯一の例外が今年の1月になると思いますが,宇出津図書館で見た北國新聞縮小版で,蛸島事件の無罪判決の記事でした。七尾から気動車で蛸島に帰るとなっていましたが,蛸島というのは能登線の終着駅の1つでもありました。

 もう一つの終着駅は輪島駅ですが,穴水駅が蛸島線との分岐点となっていて,輪島線の方が1,2年ほど前に廃線となっていました。蛸島線の廃線は平成17年だったと思います。余り記憶がはっきりしないのでこのあと調べて確認をしますが,能都町が能登町となったのも同じ平成17年だったと思います。

\begin{itemize}
\tightlist
\item
  のと鉄道 - Wikipedia \url{https://t.co/bzjRIAr8SX}  ¥\n
  2001年(平成13年)4月1日 - 七尾線 穴水 - 輪島間廃止 ¥\n
  2005年(平成17年) ¥\n 4月1日 - 能登線 穴水 - 蛸島間廃止。 ¥\n 6月 -
  本社を能登町から穴水町に移転。
\end{itemize}

 のと鉄道の本社が能登町から穴水町に移転とありますが,これは宇出津駅以外に考えられないものの宇出津駅にのと鉄道の本社があったとは知りませんでした。駅の隣に事務所のある建物があったことは記憶にあるのですが,バスの会社だと思っていましたし,その前にバスも多数停まっていました。

 金沢から七尾へは長距離トラックの仕事でよく行ったのですが,それも平成4年4月1日の傷害・準強姦被告事件発生の半年ほどの間に集中していました。ほとんどが国道159号線だったと思います。

 金沢から七尾へは別にルートが1つあるのですが,Googleマップで見ると県道2号線となっていて,「七尾羽咋線」とも表示があります。

\begin{itemize}
\item
  能登木材工業(株) - Google マップ \url{https://t.co/BGZM8G0ghB} 
\item
  林ベニヤ産業 七尾工場 - Google マップ \url{https://t.co/kFZCrp96Xb} 
\end{itemize}

 調べると「能登木材工業」となっていましたが,能登木材とだけ聞いていました。国道160号線沿いにあるのも記憶どおりです。林ベニアの場所が七尾市の公設市場に隣接していますが,この市場の方は夜にしか行ったことがないかもしれず,林ベニアに行った時も気が付かなかったように思います。

 七尾市の公設市場は,他にも少し行った記憶があるのですが,これも集中したのは,平成3年の11月から平成4年の2月で和歌山県のかつらぎ町からみかんやいよかんを運んだ仕事でした。能登木材,林ベニアと同じ,七尾市の丸一運輸の仕事でした。被告発人東渡好信が絡んだ仕事です。

\begin{itemize}
\item
  〈〈〈 2021-05-05 11:34:28 Linux VsCode: 〈〈〈
\item
  〉〉〉 Linux Emacs: 2021/05/05 11:36:39 〉〉〉
\item
  七尾サンライフプラザ - Google マップ \url{https://t.co/rA224p8Vjj} 
  〒926-0021 石川県七尾市本府中町 本府中町ヲ−38
\end{itemize}

 未来型の新しい建物に見えましたが,建物より敷地が広く,最新のゴミ処理場のような印象を受けました。七尾サンライフプラザという名称ははっきり記憶にはないですが,場所は記憶にあっており,周辺にそれらしい建物は他にないので間違いないと思います。

 思い出せないのは,なぜその七尾サンライフプラザで行われる無料法律相談を知ったかですが,弁護士の名前というのも記憶には残らず,雰囲気だけが影のように印象に残っています。割とよくいそうなタイプの弁護士で,うまくあしらわれたようにも思います。魔法にかけられたともいいますか。

 hatena-log-search "七尾.*相談" の結果はありませんでした。twilog-serch
には本日の分を含め4件の該当がありますが,次のような内容です。

「無料の法律相談には七尾市まで行ったことがありましたが、往復の電車代を含め、朝から夕方までの時間を潰し、なんの成果も得られず、大損をさせられただけだというのが率直な感想でした。」

 諦めのような気持が強かったとも思いますが,ときどき思い出しながら今日に至っています。金沢市役所での無料法律相談のこともずっと記憶にありますが,内容というものがありませんでした。それだけは七尾市の無料法律相談と共通してよく憶えています。

 奇妙な風景を見たような気もする弁護士の法律相談ですが,銀河鉄道999のようなSFの世界に近いものがあるのかと最近考えるようになりました。銀河鉄道999に出てくる機械人間ならぬ弁護士人間です。処理場で働くロボットのようでもありました。

 血が通ったという表現もありますが,弁護士人間にはその辺りの要素が欠落しているのかもしれません。時間だけが無駄となりました。

 無料の法律相談が終わったあとになると思いますが,その七尾サンライフプラザの建物の中で,図書館を見つけ気まぐれで入ったような記憶があります。そこで何かきっかけがあったと思うのですが,同じ七尾市内の中心部にある別の図書館に向かいました。

 Amazonプライムビデオの銀河鉄道999に彗星であったのか,図書館がタイトルにある回があったのですが,悪人は弁護士ではなく医師でしたが,とんでもない内容が,短い話の中で簡潔に表現されていました。

\begin{quote}
《引用の始まり》
\end{quote}

\begin{quote}
第6話 彗星図書館   1978年(昭和53年)10月19日放送脚本・藤川桂介  美術・勝又 激  作画監督・田中 保  演出・坂田ゆう

~彗星の巣コメット・ステーションに降り立った鉄郎は、1時間30分の停車時間をつぶすために、地球で発行されたあらゆる本が揃っている本屋に立ち寄るが、そこで帽子の男に因縁を付けられ、擦り傷をを負わされてしまった。さらに病院に担ぎ込まれた鉄郎は、医師から無理矢理、法外な値段の機械の身体の手術を施されそうになり大ピンチ!帽子の男と医師が仲間であったことに気づいたメーテルによって、からくも鉄郎は救出された。
\end{quote}

\begin{quote}
《引用の終わり》
\end{quote}

\begin{itemize}
\tightlist
\item
  銀河鉄道999 第6話 \url{http://galaxyrailway.com/ge999/station/999station/tv999/006.htmln} 
\end{itemize}

 間に言葉が入り,少なくとも「彗星の図書館」かと思っていたのですが,そのまま「彗星図書館」となっていました。彗星という星の名前は子供の頃から知っていましたが,なぜ悪徳医師の出てくる物語が彗星と結び付けられたのか,なにか理由がありそうで気になります。

 上記の引用に「第6話 彗星図書館   1978年(昭和53年)10月19日放送」とありますが,子供に向けたテレビのアニメ放送としては,問題とされなかったのかと気になる内容ですが,10歳の男の子が銃を使う場面も多く,テレビでは再放送が難しいのかとも考えます。

 同じページの「彗星の巣、コメットステーションまでの道のりは、太陽系の庭の中のようなもの。」という部分で,コメットを調べたところ彗星のことだと知りました。

 テレビで銀河鉄道999の放送と同じ頃だと思いますが,たぶん日曜日の,夕方に「コメットさん☆」という番組がありました。mozcの変換で星のマークが一緒に出てきました。大場久美子がコメットさん☆だったのですが,ここ数年名前を見かけていません。久しぶりに思い出しました。

 七尾市ということで山口成良金沢大学教授のことも思い出しているのですが,七尾サンライフプラザの近く国道159号線沿いに松原病院がありました。気がついたのは,スマホを買いに歩いていたときが最初かもしれません。七尾青柏祭で,大地主神社から歩いて向かいました。

 山口成良金沢大学教授については別の項目で取り上げておきたいと思いますが,ネットで知った情報では七尾市の出身,生まれた年も私の母親と同じ昭和4年でした。1年ほど調べていないかもしれません。

\begin{itemize}
\tightlist
\item
  連載 \textbar{} 北國新聞 健康・医療情報サイト 丈夫がいいねっと
  \url{https://t.co/lOh9Pybuov}  ¥\n
  「分かっているのに止められない。これがこの症状の恐ろしいところだ」。女性を診断し、医学誌で報告した金大名誉教授で松原病院(金沢市)名誉院長の山口成良医師はこう解説する。
\end{itemize}

 初めて見る山口成良金沢大学教授らしき顔写真ですが,記憶にある人物とまるで雰囲気が違っています。ここまで人が変わるものかと驚きました。記事に日付が見当たらなかったのですが,URLに20181229とありました。2018年12月のようです。

\begin{itemize}
\item
  〈〈〈 2021/05/05 14:13:01 Linux Emacs: 〈〈〈
\item
  〉〉〉 Linux Emacs: 2021/05/05 15:39:10 〉〉〉
\item
  奉納\危険生物・弁護士脳汚染除去装置\金沢地方検察庁御中\_2020:
  「七尾」を過去のはてなダイアリーの記事から検索 \url{https://t.co/N1jQrXaivd} 
\end{itemize}

 「七尾」をキーワードに含むはてなブログは,全部で60件という意外に少ない結果でしたが,「図書館」を含むものはありませんでした。

 記憶を喚起する手掛かりは得られなかったことになりますが,七尾サンライフプラザから向かった七尾市内の図書館は,小丸山城址公園の近くだったと思います。長い間,七尾城址だと思い込んでいた場所ですが,2,3年前に七尾城址は,市内の外れの山城だと知りました。

 近くに小丸山城址公園があると思ったのですが,古い建物の図書館で,室内が昭和50年代前半に見ていた景色に思え,古い時代にタイムスリップしたような感覚にもなっていました。

 たぶんですが,七尾サンライフプラザの図書館で,本の貸出が出来,宇出津の図書館で返却が出来ると聞いたのだと思います。たぶん,その七尾サンライフプラザの方ですが,告訴・告発の書式集のような本を見かけ,他にも役立ちそうな本を探しに,次の図書館に向かったのだと思います。

\begin{itemize}
\tightlist
\item
  七尾市立図書館 - Wikipedia \url{https://t.co/Hsj7mTt6LV}  ¥\n
  1977年(昭和52年)10月15日、馬出町ツ部49の新館に移転した。 ¥\n
  2004年(平成16年)10月、合併により七尾市立中央図書館となった。 ¥\n
  2006年(平成18年)7月1日、JR七尾駅前ビル「ミナ.クル」内に移転した。
  ¥\n 2018年(平成30年)4月、七尾市立図書館
\end{itemize}

 この前の長崎市の魚市場の移転ほど驚きはないですが,七尾駅前のビルに移転していたとのことです。昔から目にしていた古い感じのビルで,前回,青柏祭に行った時,ビルの中に入ってみた記憶があります。よく憶えていないですが,少し不思議な感じがした建物だったかと思います。

 前回というのは2019年5月のことです。5月5日だったかもしれません。食祭市場の前に3つのデカ山が並んでいました。そのあと10年ぶりに羽咋市内まで行ったのですが,結構な数,写真も撮影しています。撮影し忘れていたのは,子グマの剥製でした。

 馬出町ツ部49という七尾市内の住所をGoogleマップで調べると,同じ住所が「花嫁のれん館」となっていました。観光のスポットで何度かテレビで見かけていましたが,記憶にある七尾市の図書館は,小丸山城址公園の山裾にあったような気がします。

\begin{quote}
《引用の始まり》
\end{quote}

\begin{quote}
告訴状・告発状モデル文例集水野
基/著東京:新日本法規出版19874-7882-4287-7×4館所蔵石川県立図書館金沢市能登町金沢大学
\end{quote}

\begin{quote}
《引用の終わり》
\end{quote}

\begin{itemize}
\tightlist
\item
  石川県内図書館横断検索 \url{https://www.library.pref.ishikawa.lg.jp/wo/cross/?q=\%E5\%91\%8A\%E7\%99\%BA\%E7\%8A\%B6\#4788242877n} 
\end{itemize}

 午前中からブラウザで開いていたページかと思いますが,「告発状」で検索をしています。「44件見つかりました。」とありますが,それらしい専門書は「告訴状・告発状モデル文例集」だけで,所蔵が4館所蔵で,石川県立図書館,金沢市,能登町,金沢大学となっています。

 今まで気が付かなかったのか,意識はしていなかったと思うのですが,宇出津図書館ではなく能登町となっています。能登町には宇出津にしか図書館がないのかと思いましたが,柳田村の公民館のような新しい建物に図書館という案内が出ていたように思います。

\begin{itemize}
\tightlist
\item
  柳田教養文化館 - Google マップ \url{https://t.co/WNUvpjTOOj}  〒928-0331
  石川県鳳珠郡能登町柳田礼8−1
\end{itemize}

 外から見ただけで,何の建物かよくわからなかったのですが,調べると「柳田教養文化館」とありました。

\begin{itemize}
\tightlist
\item
  柳田教養文化館|各課・施設|能登町役場 \url{https://t.co/PnSSA9Uzn0} 
  文化振興に係る展示、図書の貸し出しを行っています。 ¥\n
  明るい図書館で、お気に入りの一冊を探してください。
\end{itemize}

 そういえば,宇出津の図書館で,昭和40年代の柳田温泉の入浴施設の場所を調べたのですが,建物の正面と休憩室のような座敷の写真はあったものの,どのあたりなのか皆目わかりませんでした。旧柳田村のコミュニティ,集会所に近いものと考えていたのですが,能登町の施設のようです。

\begin{itemize}
\tightlist
\item
  能登町立図書館 \url{https://t.co/1hHVHp0g5K} 
\end{itemize}

 「告発状」の検索をしたのですが,ページタイトルには反映されず,石川県の横断図書検索と同じようです。今まで宇出津図書館と思っていたのですが,能登町立図書館が正確なのかもしれません。思わぬ発見が柳田教養文化館からありました。

\begin{itemize}
\tightlist
\item
  能登町立図書館 \url{https://t.co/q7lRELWDao}  ¥\n
  タイトル「警察組織」完全読本 ¥\n
  サブタイトル約30万人の巨大組織、その仕組みと力に迫る! ¥\n
  叢書名TJ MOOK ¥\n 出版者東京:宝島社 ¥\n 2020年05月
\end{itemize}

 別に読みたい本ではないのですが,赤字で「利用可能な資料があるため、予約できません」と表示されています。この本が「教養文化館」となっているのですが,貸出中のマークも多い「新着資料」の検索ページでは,中央図書が目立ちました。これが宇出津の図書館のようです。

 宇出津の図書館にない本が,旧柳田村の柳田教養文化館にあるというのも1つの発見です。

\begin{itemize}
\tightlist
\item
  弁護士・検察官・裁判官になるには:中古本・書籍:野村二郎(著者):ブックオフオンライン
  \url{https://t.co/8f84tkJapO} 
\end{itemize}

 宇出津の図書館の蔵書となっているようですが,どこにあるのか見かけたことのない本です。ブックオフで200円となっています。

 「弁護士」の検索結果が106件あり,本の内容で調べているのだとわかりますが,意外な発見がありました。「正木ひろし 事件・信念・自伝」とあります。95件目となっています。

\begin{itemize}
\tightlist
\item
  正木ひろし―事件・信念・自伝 (人間の記録 (119)) \textbar{} 正木 ひろし
  \textbar 本 \textbar{} 通販 \textbar{} Amazon \url{https://t.co/OJiAWfqji7} 
\end{itemize}

 拍子に大きな顔写真があるのですが,これだけでもとても珍しく感じました。昭和の時代にもよく見かけた手塚治虫のような帽子なのですが,グーで頬杖をつき,微笑んでいる顔写真です。このあと確認しますが,あの首切りの弁護士と同一人物の写真であれば,驚きです。

\begin{itemize}
\tightlist
\item
  正木ひろし - Wikipedia \url{https://t.co/McMfLxsLi5} 
  第二次世界大戦前より軍国主義批判を繰り広げ、戦時中には官憲による拷問を告発した首なし事件で有名となった。戦後も多くの反権力裁判、冤罪裁判に関与した。
\end{itemize}

 少し心配だったものの名前の記憶に間違いはなかったですが,正木ひろし弁護士は1975年に亡くなっていたと確認しました。「正木ひろし―事件・信念・自伝
(人間の記録 (119))
」という本は本人の自伝のようですが,1999年の出版となっていたように思います。

 正木ひろし弁護士は,著書が多く,正木ひろし著作集(全6巻)』(三省堂 1987年)(2008年、学術出版会から再版)まであると気がついたのですが,「人間の記録」というのは見当たらず,ページ内検索も該当なしでした。

 ちょうど昨日,佐藤博史弁護士が正木ひろし弁護士に影響を受けたと記事で読んだのですが,名誉毀損で刑事裁判になっていた理由もその記事で理解が出来ました。殺人事件の真犯人と名指しして名誉毀損で起訴されたようです。

 脚注からリンクを開くと,「観音堂事件(1951年)」という気になる事件名が目に入ったのですが,マウスオーバーで存在しないページと表示されました。「有罪確定後に再審請求を行い、その後の再審で被告人の無罪が確定した。」という内容です。

 その脚注の方に,「同名誉毀損事件では被告人正木は死亡により公訴棄却、鈴木忠五は1976年に最高裁で有罪が確定し弁護士資格を6か月剥奪された。」とあります。

\begin{itemize}
\item
  「平成の正木ひろし」を目指したが・・・ \textbar{} 弁護士ブログ
  \textbar{} 名古屋で医療過誤のご相談は 北口雅章法律事務所
  \url{https://t.co/BVifd4LT6y}  ¥\n
  「正木ひろし(弁護士)」は,今や伝説的存在である。 ¥\n
  特に法科大学院世代の若手弁護士は,彼の業績や思想のことなど殆ど知らず(実は,私もあまり知らない。)
\item
  「平成の正木ひろし」を目指したが・・・ \textbar{} 弁護士ブログ
  \textbar{} 名古屋で医療過誤のご相談は 北口雅章法律事務所
  \url{https://t.co/BVifd4LT6y}  ,関心もないのではないかと懸念されるが, ¥\n
  彼の存在は,私の中では「反骨と正義の鏡」である。
\end{itemize}

 さきほどの本の表紙と同じ写真からリンクを開くと,北口雅章弁護士のブログ記事が出てきました。ウィンドウが下にあるのですが,目を移すと,ブログ記事に「(石川県七尾高校)」とあります。

 「昔,亡父の郷里で高校(石川県立七尾高校)時代からの親友で名古屋で弁護士をされている方(亡水口先生)の事務所を訪ねたことがある。」とありますが,よくわからない日本語です。亡くなった父親の郷里が七尾市ということは間違いなさそうですが,亡父の親友が水口弁護士なのか。

 なぜ,唐突に七尾高校が出てきたのかと思ったのですが,正木ひろし弁護士の生年をみると,七尾市で弁護士をしていたという益谷秀次に年代が近いように思えてきました。正木ひろし弁護士は1896年(明治29年)生まれとなっています。

 北口雅章弁護士は,懲戒処分が直近の話題となっていたように思いますが,依頼者に肩入れをし,裁判の相手を批判したことが懲戒処分になっていたかと思います。その相手というのも,最近はめっきり名前を見かけなくなった。伊藤詩織氏であります。懲戒処分が出たのか確認します。

〉〉〉 kk\_hironoのリツイート 〉〉〉

\begin{itemize}
\tightlist
\item
  RT
  kk\_hirono(刑事告発・非常上告_金沢地方検察庁御中)|aritayoshifu(有田芳生)
  日時:2021-05-05 19:14/2019/12/19 19:13 URL:
  \url{https://twitter.com/kk\_hirono/status/1389886255039610886} 
  \url{https://twitter.com/aritayoshifu/status/1207604757042356224} 
  \textgreater{}
  元TBS記者の代理人弁護士 伊藤詩織さんへの侮辱的な書き込みで懲戒の可能性
  - 毎日新聞 \url{https://t.co/HSIkDV53nM} 
  北口雅章弁護士。昨日の記者会見で論点をそらしながら、しどろもどろに多弁だった弁護士です。
\end{itemize}

〉〉〉 kk\_hironoのリツイート 〉〉〉

\begin{itemize}
\item
  RT
  kk\_hirono(刑事告発・非常上告_金沢地方検察庁御中)|aojimami1(Mami
  Isambert) 日時:2021-05-05 19:15/2021/03/31 19:31 URL:
  \url{https://twitter.com/kk\_hirono/status/1389886522669731844} 
  \url{https://twitter.com/aojimami1/status/1377206893064507400} 
  \textgreater{}
  山口敬之氏の弁護士、北口雅章氏懲戒。自身のブログに「妄想」「虚構」「虚偽」などと記載。
  読んで胸が悪くなったし法廷でのセカンドレイプも悪辣でした。こんな弁護士がいるのかと思ったほど。
  \url{https://t.co/BPOaTU34FT} 
\item
  北口 弁護士 懲戒 - Twitter検索 / Twitter \url{https://t.co/WDcYFonygz} 
\item
  伊藤詩織さんをブログで侮辱か 男性弁護士を戒告処分:朝日新聞デジタル
  \url{https://t.co/2dUPi5cKOC}  2021年3月31日 18時01分
\end{itemize}

 弁護士会の懲戒処分と言っても最も軽いと聞いたように思う戒告処分でした。それより,今年の3月31日のニュースになっているのが驚きで,一月と5日ほどしか経っていません。昨年のことかとも考えていたところでした。

 この北口雅章弁護士といえば,名古屋刑務所の受刑者虐待死という事件の弁護人ということも私の中では大きいのですが,昨日は,海渡雄一弁護士の自伝のようなインタビュー記事も読んでいて,まだ読了していなかったかもしれないですが,革手錠を禁止させたような話がありました。

 北口雅章弁護士の記事の最後に,本の表紙の写真がありますが,顔写真の真下に「まさきひろし」とあり,一番下の中央に「日本図書センター」とあります。出版社なら聞いたことがないですが,穏やかに微笑む表情の顔写真は,正木ひろし弁護士に間違いなさそうです。

 正木ひろし弁護士のWikipediaには写真が1枚あって,「万能事件の現場検証時(1952年)」と真下に書いてあるのを読んだのですが,大きな鍬のようなものをくわえタバコで持つある種,異様な写真で,これを首なし事件の墳墓発掘時の写真と思っていました。

 万能事件というのもマウスオーバーで「存在しないページ」となっています。

\begin{itemize}
\tightlist
\item
  益谷秀次 - Wikipedia \url{https://t.co/yYYoqcoofH} 
  益谷秀次先生顕彰碑(石川県珠洲市、道の駅すずなり)
\end{itemize}

 益谷秀次の検索で,これまでに見たことのない石碑の写真が出てきたのですが,Wikipediaのページで写真の下に,石川県珠洲市,道の駅すずなり,とあります。この道の駅は,もともと国鉄の珠洲駅の跡地で,ホームの一部が残されていました。

\begin{itemize}
\tightlist
\item
  益谷秀次 - Wikipedia \url{https://t.co/yYYoqcoofH}  ¥\n 益谷 秀次(ますたに
  しゅうじ、1888年1月17日 - 1973年8月18日)は、日本の政治家。 ¥\n ¥\n
  衆議院議員(14期)、衆議院議長(第46代)、副総理、建設大臣、行政管理庁長官などを歴任した。
\end{itemize}

 明治21年生まれとなっていましたが,没年は昭和48年8月18日となっています。宇出津小学校の低学年の時と思いますが,遠島山公園の銅像を見にいきました。たぶん出来たばかりのしらさぎ橋も,そのとき一緒に見ているのではと思います。

 「京都帝国大学法科大学仏法科を卒業。浦和地方裁判所判事から、弁護士となった。」と益谷秀次の経歴にありますが,京都帝国大学法科大学仏法科とあるので,京都市とも縁があったらしく,意外に感じました。前にも文字は読んでいるはずですが,気に留めずにいたようです。

\begin{itemize}
\tightlist
\item
  珠洲城山公園「益谷秀次銅像」 \textbar{} 山彦耀Ⅱのブログ
  \url{https://t.co/rQO0tDHT2s}  1888年1月17日宇出津に生まれる ¥\n ¥\n
  建設大臣(吉田内閣)、副総理、衆議院議長などを歴任、憲政の父と呼ばれた
\end{itemize}

 道の駅すずなり,にいつ益谷秀次の石碑が出来たのか調べ始めたのですが,珠洲市に城址公園があったのかと意外に思いました。4月の17日だったように思いますが,宇出津の白山神社のご神事に出席し,珠洲市正院の宮司のご挨拶があったのですが,後で調べると正院が城下町だったと知りました。

 城址といえば高台をイメージするのですが,珠洲市正院町の中心部に高台はなかったように思います。昭和56年の秋に仕事で行った蛸島町の火葬場の辺りがいくらか高台になっていたと記憶にあるのですが,その近くに城址があるのかもしれません。

 珠洲市正院町の範囲がずいぶん広いこともGoogleマップで確認しています。蛸島町の人口が500人というのが気になって調べたのですが,蛸島町の方はずいぶん狭い範囲となっていました。

 益谷秀次が「憲政の父と呼ばれた」というのも初めて知ったように思ったのですが,その下の銅像の写真には,「益谷秀次君 吉田茂書」とあることに気が付きました。先程は「益谷秀次君」という部分しか見ておらず,なぜ銅像に君付けがあるのか不思議に思っていました。

 益谷秀次の銅像は舳倉島でも見ています。事前にネットで情報は見ていたのですが,探し回っても見つからず,帰り際に誘われて行った道の途中に銅像がありました。

\begin{itemize}
\tightlist
\item
  のと鉄道と益谷秀次 : のとツーリズム Blog \url{https://t.co/ICMWf0lehl} 
  益谷さんは小学生の頃から『いつの日か能登に鉄道を引いてやる!』という大きな夢をもっていたそうです。政治家になってから能登に汽車が来た日まで、モノ凄い執念で夢を実現しました。
\end{itemize}

 ページのデザインが違うように思うものの「のとツーリズム」というのは見覚えのあるブログ名で,以前の記事が2004年5月から2005年6月という古いものとなっています。その後,メインのブログを移されたのかもしれないですが,「財団法人能都町ふるさと創生公社」とあることに気が付きました。

 よく考えてみると,珠洲駅にはもともと益谷秀次の石碑があったと思われます。1つ手前が飯田駅になるのですが,飯田駅までは行ったことがはっきり記憶に残っており,一度,終着駅の蛸島駅まで行ったようにも考えたのですが,途中にあるはずの正院駅というのが全く記憶にないのです。

 珠洲駅の方も存在はよく知っていましたが,行ったことがあるのか記憶がはっきりせず,道の駅すずなり,に残されたホームがあると知ったのは,2009年に母親が珠洲市総合病院に入院した後になると思います。住所は珠洲市野々江だと思いますが,歩いて行ったところにあったかもしれません。

 母親が珠洲市総合病院に入院した頃は,シメノドラッグの店舗が飯田高校の近くにあって,そこによく介護用品を買いに行っていたのですが,その後,離れた場所に移転となって,それが道の駅すずなり,のすぐ近くだったことは憶えています。

\begin{itemize}
\item
  Google マップ (Google Maps) で緯度経度を指定して場所を見つける方法 --
  ラボラジアン \url{https://t.co/QgYuMjIqAG} 
\item
  正院駅 - Wikipedia \url{https://t.co/Xzt9U5xTyC}  ¥\n
  所在地石川県珠洲市正院町正院 ¥\n 北緯37度26分45.1秒 ¥\n
  東経137度17分18.6秒
\end{itemize}

 37°26'45.1``N 137°17'18.6''E

\begin{itemize}
\tightlist
\item
  37°26'45.1``N 137°17'18.6''E - Google マップ \url{https://t.co/PJyoAqpjLz} 
\end{itemize}

 前から気になっていた正院駅の場所を調べてみました。珠洲市正院町というのは蛸島町や三崎町への通過点という記憶しかないのですが,割と意外な場所に駅がありました。Wikipediaにはのと鉄道への転換で無人駅となり,それまでは簡易委託の乗車券販売とあります。

 蛸島駅も無人駅とはなっていなかった情報で,駅舎に記憶はないものの珠洲駅が無人駅とは考えられないですが,現在のGoogleマップで見た住宅の数からも無人駅だったとは考えにくいと思いながら調べてみました。

\begin{itemize}
\tightlist
\item
  正院川尻城 - Google マップ \url{https://t.co/3X6EAgsgXG} 
\end{itemize}

 そういえば前に珠洲市正院町を調べた時も,正院川尻城を見かけたと思いだしたのですが,Googleマップで調べると,住所が正院町とあるものの蛸島の漁港のすぐ近くだとわかりました。最近知った正院川尻城ですが,九里川尻駅が現在の能登町,以前の珠洲郡内浦町にあります。

\begin{itemize}
\tightlist
\item
  〈〈〈 2021/05/05 21:06:46 Linux Emacs: 〈〈〈
\end{itemize}

\hypertarget{ux9280ux6cb3ux9244ux9053uxff19uxff19uxff19ux3068ux77f3ux5dddux770cux73e0ux6d32ux5e02ux86f8ux5cf6ux753aux306eux86f8ux5cf6ux4e8bux4ef6ux91d1ux6ca2ux5408ux540cux6cd5ux5f8bux4e8bux52d9ux6240ux306eux68a8ux6728ux4f5cux6b21ux90ceux5f01ux8b77ux58ebux3068ux5f01ux8b77ux58ebux9244ux9053ux306eux6b74ux53f2}{%
\paragraph{銀河鉄道999と石川県珠洲市蛸島町の蛸島事件,金沢合同法律事務所の梨木作次郎弁護士と弁護士鉄道の歴史}\label{ux9280ux6cb3ux9244ux9053uxff19uxff19uxff19ux3068ux77f3ux5dddux770cux73e0ux6d32ux5e02ux86f8ux5cf6ux753aux306eux86f8ux5cf6ux4e8bux4ef6ux91d1ux6ca2ux5408ux540cux6cd5ux5f8bux4e8bux52d9ux6240ux306eux68a8ux6728ux4f5cux6b21ux90ceux5f01ux8b77ux58ebux3068ux5f01ux8b77ux58ebux9244ux9053ux306eux6b74ux53f2}}

\begin{itemize}
\tightlist
\item
  〉〉〉 Linux Emacs: 2021/05/05 21:59:49 〉〉〉
\end{itemize}

:CATEGORIES: @kanazawabengosi \#金沢弁護士会 @JFBAsns
日本弁護士連合会(日弁連) \#法務省 @MOJ\_HOUMU \#蛸島事件
\#梨木作次郎弁護士 \#刑裁サイ太 \#深澤諭史弁護士

 まず,さきほど見かけてざっと目を通した刑裁サイ太のブログ記事からご紹介したいと思います。銀河鉄道999と蛸島事件について,金沢弁護士会と絡めて取り上げようとした矢先でしたが,蛸島事件での大きな発見は,同じユーチューバーがきっかけでした。

※ @kk\_hironoのアカウントがブロックされ,リツイートに失敗したツイート

\begin{itemize}
\item
  TW uwaaaa(サイ太) 日時:2021/05/05 20:23:07 URL:
  \url{https://twitter.com/uwaaaa/status/1389903478718337030} 
  \textgreater{}
  ゴ3ネタブログを更新しました。今回は弁護士youtuberをまとめました。\\
  \textgreater{}\\
  \textgreater{}
  弁護士youtuberをまとめてみた結果wwwwwwww\url{https://t.co/OHBDSrDUsN} 
\item
  弁護士youtuberをまとめてみた結果wwwwwwww -
  刑裁サイ太のゴ3ネタブログ \url{https://t.co/KEtLc7HFFV}  2021-05-05
\end{itemize}

 過去にも前例があり,そのうち削除されるかもしれない刑裁サイ太のブログ記事ですが,細かいところを調べると,リンクにあったURLにある202211というのが,2021年5月5日20時22分11秒という投稿時刻を意味するものと思われます。

\begin{itemize}
\tightlist
\item
  弁護士youtuberをまとめてみた結果wwwwwwww -
  刑裁サイ太のゴ3ネタブログ
  \url{https://keisaisaita.hatenablog.jp/entry/2021/05/05/202211} 
\end{itemize}

 次が,さきほど投稿した最新エントリーになります。

\begin{itemize}
\tightlist
\item
  1344:2021-05-05\_21:07:14 \#告発状 \#\#\#\#
  たぶん平成14年,七尾市の七尾サンライフプラザでの無料法律相談と,図書館の「告訴状・告発状モデル文例集」
  \url{https://hirono-hideki.hatenadiary.jp/entry/2021/05/05/210711} 
\end{itemize}

 21時7分14秒として記録されていますが,20時22分11秒に前後したツイートのことが気になってきました。

\begin{itemize}
\tightlist
\item
  奉納\危険生物・弁護士脳汚染除去装置\金沢地方検察庁御中\_2020:
  「\^{}2021-05-05
  20:.*」を@hirono\_hideki @kk\_hirono @s\_hironoで検索 12件の該当 2021-05-05\_22:17の記録
  \url{https://t.co/gaUMyPOtAv} 
\end{itemize}

 まとめ記事を作成して眺めてみると,2021-05-05 20:19:26と2021-05-05
20:41:25との間に,約20分ほどの空白の時間がありました。パソコンの前には座っていた時間帯とは思います。

 「Google マップ (Google Maps)
で緯度経度を指定して場所を見つける方法」が20時41分のツイートの内容となっているので,ちょうど正院駅の場所を調べていた時間帯になりそうです。

 刑裁サイ太がまとめた「弁護士youtuberをまとめてみた結果wwwwwwww」というブログ記事は,知らなかった弁護士も多いのですが,よく名前が知られているはずの弁護士のチャンネル登録数,再生回数が意外に少なく,この違いはどこからくるのかと考える材料にもなりました。

\begin{lstlisting}
py37_env ❯ tun uwaaaa 5
\end{lstlisting}

\begin{itemize}
\item
  TW uwaaaa(サイ太) 日時: 2021-05-02 23:58 URL:
  \url{https://twitter.com/uwaaaa/status/1388870510914592773} 
  \textgreater{}
  ああいう人が弁護士登録して、誰が依頼するんだろうというのはある
\item
  RT uwaaaa(サイ太)|kanjizaik(かんじざい) 日時:2021-05-05
  15:17/2021-05-04 23:07 URL:
  \url{https://twitter.com/uwaaaa/status/1389826662167060480} 
  \url{https://twitter.com/kanjizaik/status/1389582567754534916} 
  \textgreater{} 司法試験の会場、マイドーム大阪が緊急事態宣言期間中の\\
  \textgreater{} 会場使用の中止または延期を要請\\
  \textgreater{}\\
  \textgreater{} 大阪の宣言が延長になれば、司法試験の延期は確定か\\
  \textgreater{}\\
  \textgreater{}
  もし法務省が強行開催しようとしても、クラスター発生時の\\
  \textgreater{}
  風評被害を恐れ、マイドーム大阪側が契約解除をする可能性が高い
  \url{https://t.co/CnTaCP05ZB} 
\item
  TW uwaaaa(サイ太) 日時: 2021-05-05 19:25 URL:
  \url{https://twitter.com/uwaaaa/status/1389888883530158092} 
  \textgreater{} 【誤変換速報】\\
  \textgreater{} 意識高い系 → 石北会計
\item
  TW uwaaaa(サイ太) 日時: 2021-05-05 19:54 URL:
  \url{https://twitter.com/uwaaaa/status/1389896257489965062} 
  \textgreater{}
  ということでyoutuber弁護士をまとめてるんですけど,頭がおかしくなりそうになった
\item
  TW uwaaaa(サイ太) 日時: 2021-05-05 20:23 URL:
  \url{https://twitter.com/uwaaaa/status/1389903478718337030} 
  \textgreater{}
  ゴ3ネタブログを更新しました。今回は弁護士youtuberをまとめました。\\
  \textgreater{}\\
  \textgreater{}
  弁護士youtuberをまとめてみた結果wwwwwwww\url{https://t.co/OHBDSrDUsN} 
\end{itemize}

 上記はTwitterAPIの自作コマンドで,刑裁サイ太のTwitterタイムラインの最新ツイート5件を取得したものです。5月2日の23時58分のツイートから5月5日の5時15分のリツイートまで時間差がありますが,深澤諭史弁護士と同じ状態に突入するのかと,いささか危惧をしていました。

 弁護士業界の実態調査あるいは生態調査として貴重な情報源である刑裁サイ太のTwitterタイムラインですが,はるかに大きな情報源だった深澤諭史弁護士のTwitterタイムラインが謎の停止を続けており,他の弁護士らがまるで無反応という興味深い弁護士の生態現象が続いています。

\begin{lstlisting}
py37_env ❯ tun fukazawas 5
- RT fukazawas(深澤諭史)|shin2_ota(太田 伸二) 日時:2021-04-25 12:57/2021-04-25 12:57 URL: \url{https://twitter.com/fukazawas/status/1386167570831613954}  \url{https://twitter.com/shin2_ota/status/1386167466498281477}   
> 生活保護を利用している方が、弁護士や支援者に相談して福祉事務所との間のトラブルを解決した後に、ケースワーカーから「なんで弁護士なんかに相談したんだ」と言われたことがあったと聞いた。  
> もし、僕の依頼者にそんなことを言うのなら、僕に言ってきなさい。できる限り、やれる限りの抗議をする。  

- RT fukazawas(深澤諭史)|msbt987(驟々みそばた𓃟) 日時:2021-04-25 19:11/2021-04-23 20:47 URL: \url{https://twitter.com/fukazawas/status/1386261587653136389}  \url{https://twitter.com/msbt987/status/1385560935725559813}   
> かぐや姫 \url{https://t.co/mvfJzD79mA}   

- RT fukazawas(深澤諭史)|1924DADA(古本屋 弐拾dB(藤井)) 日時:2021-04-25 19:56/2021-04-25 13:29 URL: \url{https://twitter.com/fukazawas/status/1386272924571627523}  \url{https://twitter.com/1924DADA/status/1386175472959066115}   
> 店の二階で、同居人がギターとシャボン玉を弾いていた。  
> これじゃまるで漫画の世界だ。 \url{https://t.co/knwW5V60LB}   

- RT fukazawas(深澤諭史)|tako_kora_(Jはお前なんだよ) 日時:2021-04-25 19:58/2021-04-25 16:42 URL: \url{https://twitter.com/fukazawas/status/1386273305561231364}  \url{https://twitter.com/tako_kora_/status/1386224170384576513}   
> 74期の皆さんは4/30から実務修習開始ですね😊実務修習では事実認定はもちろんですが手続も意識してみて下さい。裁判所の行っていることは基本的に全て法律や規則に根拠があるので、まずはそれらをしっかり押さえるとよいと思います😌  
> #74期  
> #修習生  

- RT fukazawas(深澤諭史)|shimadayusuke66(島田雄左) 日時:2021-04-26 07:41/2021-04-23 18:57 URL: \url{https://twitter.com/fukazawas/status/1386450393157169156}  \url{https://twitter.com/shimadayusuke66/status/1385533372798173191}   
> 弁護士や司法書士の資格取得後、以前は、丁稚奉公したら独立するのスタンダードでした。でも、今は独立に限らず、専門性を高めたり事務所の総合化などに伴い、色んなキャリアプランが増えたように感じます。そう考えると、資格者の役割も多様化してるので、資格取得後のキャリアプランは無限大です。  

\end{lstlisting}

 あれほど活発だった深澤諭史弁護士のタイムラインが9日以上停止状態が続くとは予想しませんでしたが,いつでもTwitterアカウントの非公開設定やアカウントの削除に至る可能性は十分予想し,そのために令和3年3月31日付告発状でも深澤諭史弁護士を中心とした記録作業を優先しました。

 このまま弁護士鉄道の記録遺産となるかもしれない深澤諭史弁護士のTwitterアカウントですが,いくらか似た傾向もある刑裁サイ太が,予想外の活性でブログ記事のツイートをしていたのは,その内容の記録性という意味でも,人知を超えたような超常現象性を感じたところであります。

 深澤諭史弁護士も憑き物が落ちたように我に返りTwitterの更新を停止させたのかもと想像するのですが,深澤諭史弁護士本人の心境の変化の可能性など,これまでも私の見込が外れたことは度々あり,緊急入院でTwitterが出来なくなった可能性なども想像しているところではあります。

 ただの偶然なのかもしれないですが,Amazonプライムビデオの銀河鉄道999を多数視聴して,機械の体を目的に旅立った星野鉄郎が,人間の体でいることに迷いを募らせていったという物語の過程を知り,あれほど弁護士の利益を追求した深澤諭史弁護士に変化があったのかと重ねて見るようになりました。

\begin{itemize}
\item
  TV版「銀河鉄道999」停車駅案内 \url{https://t.co/vq0cJL4bEO} 
\item
  銀河鉄道999 第53話 \url{https://t.co/pTvc1yG9UD}  ¥\n
  第53話 鏡の星の鉄郎   1979年(昭和54年)11月15日放送
\end{itemize}

 資料性が高く,深澤諭史弁護士批判していたこともあるまとめサイトとは違っているように思いますが,「TV版「銀河鉄道999」停車駅案内」という資料の一覧を見つけたのも,本日2021年5月5日の収穫でした。思えば,こどもの日であり,偶然ではない出会いということもあるのかもしれません。

\begin{itemize}
\tightlist
\item
  2021年05月05日22時53分の登録:
  REGEXP:''@fukazawas''/データベース登録済みツイートの検索:2021-05-03〜2021-05-03/2021年05月05日22時52分の記録:ユーザ・投稿:2/9件
  \url{https://kk2020-09.blogspot.com/2021/05/regexpfukazawas2021-05-032021-05.html} 
\item
  2021年05月05日23時03分の登録:
  REGEXP:''深澤先生''/データベース登録済みツイートの検索:2021-05-04〜2021-05-04/2021年05月05日23時02分の記録:ユーザ・投稿:1/2件
  \url{https://kk2020-09.blogspot.com/2021/05/regexp2021-05-042021-05-0420210505230212.html} 
\item
  2021年05月05日23時08分の登録:
  REGEXP:''深澤.*弁護士''/データベース登録済みツイートの検索:2021-05-03〜2021-05-05/2021年05月05日23時08分の記録:ユーザ・投稿:3/18件
  \url{https://kk2020-09.blogspot.com/2021/05/regexp2021-05-032021-05.html} 
\end{itemize}

 私が独自に開発した弁護士脳の記録装置で,雛の爪のような貧弱さは感じておりますが,弁護士という惑星の観測衛星の打ち上げにいくらか近い働きはしてくれていると思います。3日間に限定した上記のまとめ記事ですが。2407のリストで私以外のアカウントのツイートはゼロとなっていました。

\begin{quote}
《引用の始まり》
\end{quote}

\begin{quote}
鏡の星に降りた鉄郎は自分とうり二つの少年、砂山学と出会った。亡くなった兄と共に必死の思いでパスを手に入れた学は、鉄郎と共に999号に乗車しようとするが、偽造パスだったため乗車を拒否されてしまう。鉄郎は学を助けてパスを売ったブローカーと対決。だが、相手の銃撃で倒れて意識を失ってしまった。このまま鉄郎になりすませば999号に乗れる・・・学の脳裏を一瞬そんな思いがよぎるが・・・
\end{quote}

\begin{quote}
《引用の終わり》
\end{quote}

\begin{itemize}
\tightlist
\item
  銀河鉄道999 第53話 \url{http://galaxyrailway.com/ge999/station/999station/tv999/053.htmln} 
\end{itemize}

 上記の引用にある「第53話 鏡の星の鉄郎   1979年(昭和54年)11月15日放送」は,Amazonプライムビデオの銀河鉄道999でも特に印象に残る回の1つだったのですが,星で出会った主要な登場人物の名前が,砂山学と砂山武という兄弟になっていました。

 その直前だったとも思うのですが,銀河鉄道999の主人公,星野鉄郎の年齢が10歳と設定されていたことを思い出し,これは蛸島事件の被害者と同じ年齢ではと考え始めていました。そのあとに確認をしたのですが,蛸島事件の被害者の名前が砂山となっていました。

 砂山という名前は他に聞いたことがなく,石川県奥能登でおきた蛸島学童殺人事件裁判記録を読み始めた当初は,被疑者の名前と同じで仮名ではないかと考えていたのですが,少し調べるとネットでいずれも実名であることを確認しました。

 原作者の松本零士氏が蛸島事件の被害者のことを知って,登場人物の兄弟に砂山という名前をつけたとは考えにくいのですが,偶然にしろ奇妙なことがあるものだと思いました。

 悪質な詐欺被害にあったという点で,昼に取り上げてご紹介したAmazonプライムビデオの銀河鉄道999,「彗星図書館」とも内容が似ているのですが,詐欺によるより深刻な被害,兄弟の人生に与えた致命的で重大な悪影響を考えさせられたのが,この「第53話 鏡の星の鉄郎」でした。

 この別の星で星野鉄郎が自分と瓜二つの少年に出会うという物語の設定は,過去に見たように薄々記憶の片隅に残っていたのですが,テレビでみたという記憶ははっきりしません。「鉄郎」で過去のツイートも調べてみたいと思います。

 hatena-log-search 鉄郎 では該当なしでしたが,twilog-serch
でも予想よりずいぶん少ない数でした。

\begin{itemize}
\tightlist
\item
  2021年05月05日23時33分の登録:
  「鉄郎」を@hirono\_hideki @kk\_hirono @s\_hironoで検索 36件の該当 2021-05-05\_23:32の記録
  \url{https://kk2020-09.blogspot.com/2021/05/hironohidekikkhironoshirono362021-05.html} 
\end{itemize}

2016-11-14 22:52:38
``2016-11-14-225233\_弁護士あだちけいた㌠シン・ゴジラはいいぞさんがリツイートメーテル @\_maeter 11月12日その言い方だと意味が違ってきこえるわ、鉄郎.jpg
\url{http://pic.twitter.com/cJ6LO74gTN''} 
\url{https://twitter.com/s\_hirono/status/798161723098181632} 

 最初に「鉄郎」をキーワードに記録されたツイートが上記の2016年11月14日のツイートで,2015年ですらなかったのが非常に意外な結果ですが,2015年には銀河鉄道999のことをかなり意識していて,弁護士鉄道というテーマ性も確立したものになっていたように思います。これも確認しておきます。

\begin{itemize}
\tightlist
\item
  2021年05月05日23時47分の登録:
  「泥棒神社」を@hirono\_hideki @kk\_hirono @s\_hironoで検索 5158件の該当 2021-05-05\_23:47の記録
  \url{https://kk2020-09.blogspot.com/2021/05/hironohidekikkhironoshirono51582021-05.html} 
\item
  2021年05月05日23時47分の登録:
  「弁護士鉄道」を@hirono\_hideki @kk\_hirono @s\_hironoで検索 4295件の該当 2021-05-05\_23:46の記録
  \url{https://kk2020-09.blogspot.com/2021/05/hironohidekikkhironoshirono42952021-05.html} 
\end{itemize}

2013-09-08 12:20:07 ``@hideo\_ogura
毎度のことですが小倉秀夫弁護士の理屈をみていると、弁護士というのは因縁を売りにする「泥棒神社」なのかと思えてきます。''
\url{https://twitter.com/hirono\_hideki/status/376545456098390016} 

2015-02-13 14:24:02 ``太田裕美 さらばシベリア鉄道 - YouTube
\url{http://ow.ly/IZ0Nd} 
この巨躯を聴いていると「さらば弁護士鉄道」という言葉が頭に浮かんできました。このテーマで行きたいと思います。よろしくです。''
\url{https://twitter.com/hirono\_hideki/status/566105495648620544} 

 巨躯という誤変換になっていることも思い出しましたが,【弁護士鉄道」の始まりは「さらば弁護士鉄道」と同じで,2015年2月13日のツイートと確認しました。「泥棒神社」の方が早く,2013年9月8日のツイートで,小倉秀夫弁護士が端緒,きっかけのようなツイートとなっています。

 いずれもう少し詳しく記述しておきたいと考えていますが,泥棒神社というのは,宇出津のあばれ祭りの起源である江戸時代,「泥棒風」をモチーフにしており,多くのものを奪い去った疫病あるいは疫病神を意味するものと個人的な経験に照らしながら理解をしていたものです。

 蛸島事件といえば,梨木作次郎弁護士になりますが,梨木作次郎弁護士も明治生まれだったように思います。さきほど取り上げた正木ひろし弁護士とは違い,純粋な共産党員で共産党の国会議員でもあったようです。

 石川県蝶屋村の極貧家庭に生まれ育ちそこから這い上がっていったような自叙伝も石川県の図書館にあるようなのですが,Amazonプライムビデオの銀河鉄道999でもたびたび,地球での差別された貧民の生活が描写され,重ねて思いを巡らせるところがありました。

 最初に「鉄郎」をキーワードに記録されたツイートが上記の2016年11月14日のツイートで,2015年ですらなかったのが非常に意外な結果ですが,2015年には銀河鉄道999のことをかなり意識していて,弁護士鉄道というテーマ性も確立したものになっていたように思います。これも確認しておきます。

\begin{itemize}
\tightlist
\item
  梨木作次郎 - Wikipedia \url{https://t.co/kDpplars85}  ¥\n 梨木
  作次郎(なしき さくじろう、1907年9月24日 -
  1993年4月9日)は、日本の弁護士、社会運動家、政治家。元衆議院議員(日本共産党公認)。金沢弁護士会会長、自由法曹団常任幹事などを歴任する。
\end{itemize}

 Wikipediaの明らかと思われる内容の更新は,さきほど益谷秀次のWikipediaに,珠洲市の道の駅すずなり,の石碑の写真で思い知ったと思ったところですが,たぶん記憶の感覚にずれがあるだけと思うものの,金沢弁護士会会長というのは新たに目にするようなインパクトを感じました。

 間違いない石川県では大物弁護士だったと思う梨木作次郎弁護士です。名前を知った頃は亡くなっていたかもしれないのですが,作次郎という名前から昔の人で高齢者というイメージだけがありました。今年の1月になってからだと思いますが,若い頃の劇団の俳優のような写真も見ています。

 大物弁護士というより,個人的にはサーカス一座の座長というイメージが強い梨木作次郎弁護士ですが,現在も金沢弁護士会に相当の影響力を与えていると思われる金沢合同法律事務所の創始者のようです。名前がはっきり思い出せないですが,前年度の金沢弁護士会の会長も同事務所の女性弁護士でした。

 少し思い出したような気がするのですが,宮西香という名前の弁護士だったかもしれません。4月1日頃に調べた今年度の金沢弁護士会の4人の副会長の1人も,この金沢合同法律事務所の所属弁護士だったように思います。あれ以来調べていませんし,情報も見かけていません。

\begin{itemize}
\tightlist
\item
  〈〈〈 2021/05/06 00:20:23 Linux Emacs: 〈〈〈
\end{itemize}

\hypertarget{ux6a0bux8a70ux54f2ux6717ux5f01ux8b77ux58ebux91d1ux6ca2ux5f01ux8b77ux58ebux4f1a}{%
\subsubsection{樋詰哲朗弁護士(金沢弁護士会)}\label{ux6a0bux8a70ux54f2ux6717ux5f01ux8b77ux58ebux91d1ux6ca2ux5f01ux8b77ux58ebux4f1a}}

\hypertarget{ux5e747ux670824ux65e5ux306btwitterux3067ux30d6ux30edux30c3ux30afux3055ux308cux3066ux3044ux308bux3053ux3068ux306bux6c17ux304cux3064ux3044ux305fux4ee4ux548c2ux5e74ux5ea6ux306eux91d1ux6ca2ux5f01ux8b77ux58ebux4f1aux526fux4f1aux9577ux6a0bux8a70ux54f2ux6717ux5f01ux8b77ux58ebux91d1ux6ca2ux5f01ux8b77ux58ebux4f1a}{%
\paragraph{2019年7月24日にTwitterでブロックされていることに気がついた、令和2年度の金沢弁護士会副会長、樋詰哲朗弁護士(金沢弁護士会)}\label{ux5e747ux670824ux65e5ux306btwitterux3067ux30d6ux30edux30c3ux30afux3055ux308cux3066ux3044ux308bux3053ux3068ux306bux6c17ux304cux3064ux3044ux305fux4ee4ux548c2ux5e74ux5ea6ux306eux91d1ux6ca2ux5f01ux8b77ux58ebux4f1aux526fux4f1aux9577ux6a0bux8a70ux54f2ux6717ux5f01ux8b77ux58ebux91d1ux6ca2ux5f01ux8b77ux58ebux4f1a}}

\begin{itemize}
\tightlist
\item
  〉〉〉 Linux Emacs: 2021/06/03 09:43:15 〉〉〉
\end{itemize}

:CATEGORIES: @kanazawabengosi \#金沢弁護士会 @JFBAsns
日本弁護士連合会(日弁連) \#法務省 @MOJ\_HOUMU
\#樋詰哲朗弁護士(金沢弁護士会)

\begin{itemize}
\tightlist
\item
  2019年07月24日、ブロックされていることに気がついた樋詰哲朗弁護士(金沢弁護士会)のTwitterアカウント
  - 告発\金沢地方検察庁\最高検察庁\法務省\石川県警察御中
  \url{https://hirono-hideki.hatenablog.com/entry/2019/07/26/144331} 
\end{itemize}

 さきほどGoogleの検索で見つけた記事ですが、すっかり忘れていた見出しの記事になります。内容を読み進めていくと終わりの方になって気がついたのですが、その2019年の7月26日に永來宏隆弁護士の記述がありました。

 始めのうちは軽く飛ばし読み気味だったのでよくわかっていないところもあるのですが、全体の流れとして、内田清隆弁護士が出て、その内田清隆弁護士の経歴に野田政仁弁護士の法律事務所での勤務があり、内田清隆弁護士の法律事務所の所属が永來宏隆弁護士でした。

 夕方の早めの時間で陽光が明るかったと記憶にあるのですが、永來という名前を見て思い出したのが島根県安来市で、久しぶりに奇石のことを思い出し、恋路海岸にも少し似た奇石があることから恋路について調べ、翌日に現地に行くことになったのです。

\begin{itemize}
\tightlist
\item
  2021年06月03日09時59分の登録:
  「\^{}2019-07-27.+」を@hirono\_hideki @kk\_hirono @s\_hironoで検索 141件の該当 2021-06-03\_09:59の記録
  \url{https://kk2020-09.blogspot.com/2021/06/2019-07-27hironohidekikkhironoshirono14.html} 
\item
  2021年06月03日09時59分の登録:
  「\^{}2019-07-26.+」を@hirono\_hideki @kk\_hirono @s\_hironoで検索 292件の該当 2021-06-03\_09:58の記録
  \url{https://kk2020-09.blogspot.com/2021/06/2019-07-26hironohidekikkhironoshirono29.html} 
\end{itemize}

2019-07-26 14:38:14
``島根県の安来市は、国道9号線で九州方面に向かう時、夕方の早い時間に通過することが多かったのですが、国道沿いにちょっと変わった岩のような山があるのが印象的でした。大きさは余り記憶にないですが、似ていると思うのが能登町の恋路海岸にある道路沿いの岩でした。''
\url{https://twitter.com/kk\_hirono/status/1154626964243673088} 

 大きさは記憶にないとありますが、岩山のようになっていました。Googleマップやストリートビューで調べたのですが、みつからず、特に名所というわかでもなさそうでした。石切り場の作業場に近い感じでもあったかと思います。

 恋路海岸の奇石も特に名所というわけではなく、名前がついているとも聞かないのですが、いつ頃からかGoogleマップで能登町を検索すると、その奇石の写真が出るようになっていました。

\begin{itemize}
\tightlist
\item
  能登町 - Wikipedia \url{https://t.co/xS3XPw8yNh} 
\end{itemize}

 Googleマップというより、Googleで能登町を検索すると奇石のある恋路海岸の写真が出てきたのですが、上記のWikipediaのページにある写真のようです。

 能登町といっても50メートルほど先は珠洲市になる場所です。少しカメラを右の方向(海側)に向ければ、見附島も映り込むはずかと思います。恋路海岸からは珠洲市の内浦の海岸線も一望されますが、蛸島町と三崎町の境目あたりまで見えているのではないかと個人的に予想しています。

 記事の作成が「2019-07-26(金曜日)11:17」から始まり、「2019-07-26(金曜日)14:42」で終わっていますが、樋詰哲朗弁護士(金沢弁護士会)にブロックされたことから書き始め、島根県安来市が出て「能登町の恋路海岸にある道路沿いの岩でした。」で終わっています。

 この樋詰哲朗弁護士(金沢弁護士会)から始まり恋路海岸につながる関連性というのは今朝まで気が付かなかったのですが、昨夜の日付が変わる10分ほど前には、次に樋詰哲朗弁護士(金沢弁護士会)について取り上げておくことを決めていました。

\begin{itemize}
\tightlist
\item
  1392:2021-06-02\_21:59:56 \#告発状 \#\#\#\#
  道路を挟んで金沢地方検察庁の横にあった被告発人若杉幸平弁護士の法律事務所の建物と、金沢地検の歴史
  \url{https://hirono-hideki.hatenadiary.jp/entry/2021/06/02/215952} 
\end{itemize}

 昨夜の23時30分を過ぎてからの仕上げと投稿とばかり思っていたのですが、21時59分が投稿処理の時刻となっていました。地震があったかもしれないです。それも真下の方から。

 家の真下の方から突き上げるような揺れがあったのですが、テレビをつけても地震の情報は出ていませんでした。突風でもけっこう家が揺れることがあるのですが、下からの衝撃のあとは、それに近い揺れ方になっていました。

 速報が出る前にアナウンサーが読み上げたのですが、やはり地震があったらしく能登町で震度3といっていました。能登町だけの局地的な地震だったようです。数年前にも似たことがあったのですが、そのときは海上の方が震源地の印となっていました。

 私の家のある場所は宇出津でも岩盤が強いと聞いたことがあるのですが、宇出津でも住所が宇出津新となっている辺りや、港の方が埋立地が多く、地震のときもよく揺れるという話を聞いています。

 自分の家の真下が震源地のようにも思えたのですが、数年前の台風のときのことを思い出しました。他に被害は聞かなかったのですが、夕方のまだ暗くなる前の時間に、家の戸が一つ外れて落ちたことがありました。

 これから記述しようとしていたことと関連があるのですが、2019年7月27日に初めて行ったのが珠洲市立図書館で、珠洲総合病院の前に建物があったのですが、病院の近くによくある薬局ができたのだと思い、図書館だと理解するまでにしばらく時間がありました。

 その後、台風の後の温帯低気圧と聞いたようにも思うのですが、他に被害は余り出ていないらしいのに珠洲市立図書館の屋根がめくれ上がり、億円を超える被害が出たというニュースになっていたように思います。他に三崎町の小学校か中学校でも被害が出たニュースを見たような気もします。

\begin{itemize}
\tightlist
\item
  台風17号 暴風、各地で被害 新設図書館、屋根めくれる 珠洲 /石川
  \textbar{} 毎日新聞 \url{https://t.co/sduTyi8PcY} 
  台風17号に伴う暴風が吹き荒れた23日、珠洲市民図書館(同市野々江町)の屋根がめくれる被害があった。図書館は今年3月にオープンしたばかり。市は24日から休館としたが、再開のめど
\end{itemize}

 同じ2019年とは思っていなかったのですが、9月25日の毎日新聞の記事で、9月23日午後1時半ごろに消防に通報があったとあります。

\begin{quote}
《引用の始まり》
\end{quote}

\begin{quote}
昨日(12月10日)の珠洲市議会一般質問では珠洲市民図書館の屋根破損問題に対する質問が相次いだ。しかし、結果的に新しい事実や新たな対応方針はなんら示されず、残念ながら議論は平行線で終わった。若干、感想を述べたい。

9月23日、台風17号は温帯低気圧に変わりながら能登半島沖を通過し、県内では非常に強い風が吹き荒れた。津幡町では男性が転倒し負傷、各地で倒木が相次ぎ、交通機関は乱れ、能登では停電した地域もあった。輪島市では瞬間最大風速31.2mを記録するなどそれなりに強力な台風には違いなかった。が、県内での建物被害はほとんどなかった。そんな中、今年3月に竣工したばかりの珠洲市民図書館のトタン屋根が大きく剥がれ落ちたのだ。
\end{quote}

\begin{quote}
《引用の終わり》
\end{quote}

\begin{itemize}
\tightlist
\item
  「新築の屋根だけ飛ばす風の技」究明を! ~市議会一般質問~ -
  北野進の活動日記
  \url{https://blog.goo.ne.jp/11kitano22/e/5ed4abe8dd5d334053e401040fabefee} 
\end{itemize}

 上記の引用部分に「県内での建物被害はほとんどなかった。そんな中、今年3月に竣工したばかりの珠洲市民図書館のトタン屋根が大きく剥がれ落ちたのだ。」とあります。

\begin{quote}
《引用の始まり》
\end{quote}

\begin{quote}
珠洲市は、剥がれ落ちる際の目撃者がおらず、どこからどんなふうに剥がれたか不明であること、図書館を襲った強風の風速が不明なこと、施工時の工法は建築基準を満たし屋根の留め金の強度も確保されていたことなどから業者の瑕疵を証明することは困難とし、原因究明を断念。1億円近くにのぼるとみられる修繕費用は事業者側が全額を負担し、当初と比べ強度を高めた工法で張り替えを終えたことで、幕引きを図る考えだ。
\end{quote}

\begin{quote}
《引用の終わり》
\end{quote}

\begin{itemize}
\tightlist
\item
  「新築の屋根だけ飛ばす風の技」究明を! ~市議会一般質問~ -
  北野進の活動日記
  \url{https://blog.goo.ne.jp/11kitano22/e/5ed4abe8dd5d334053e401040fabefee} 
\end{itemize}

 「業者の瑕疵を証明することは困難とし、原因究明を断念。1億円近くにのぼるとみられる修繕費用は事業者側が全額を負担し、」とあります。当時、少し聞いた話では能登町の業者が施工をしたという話もあったのですが、屋根の修繕だけで1億円近くというのは、相当大きな被害があったようです。

\begin{quote}
《引用の始まり》
\end{quote}

\begin{quote}
自己紹介珠洲原発反対運動に関わり31歳から石川県議を3期務める。その後、石川県平和運動センター事務局で平和運動に携わり、2011年から2期珠洲市議を務める。現在「志賀原発を廃炉に!訴訟原告団」共同代表。
\end{quote}

\begin{quote}
《引用の終わり》
\end{quote}

\begin{itemize}
\tightlist
\item
  「新築の屋根だけ飛ばす風の技」究明を! ~市議会一般質問~ -
  北野進の活動日記
  \url{https://blog.goo.ne.jp/11kitano22/e/5ed4abe8dd5d334053e401040fabefee} 
\end{itemize}

 プロフィールに名前がないと思ったのですが、ブログ名にある「北野進」がお名前のようです。原発反対運動ということで金沢の弁護士とも関わりのありそうな人ですが、昭和の時代、同姓同名の人がいたような気もしました。宇出津にもある名前だと思いますが、数は少ないようにもおもいます。

 珠洲市といえば原発誘致と原発反対運動の歴史もあったのですが、ずっと前に完全決着しているのではないかと思っています。僅差で反対が決まったともテレビニュースで見たような記憶が残っているので、それなりに禍根を残すことはあったのかもしれないですが、聞いたことはない話です。

 北陸中日新聞の川柳の記事の写真があることに気がついたのですが、「新築の屋根だけ飛ばす風の技」などとあります。そういえば月曜日に蛸島町の八ケ崎海水浴に行ったときも、大きな松の木を見上げながら台風で倒れないのを不思議に思っていました。大きいというよりとても高い松の木です。

 10年ほど前の夏にも一度行った場所だと思ったのですが、海に海水浴客が多かったという記憶しか残っておらず、松の木のことはまったく気にかけていなかったようです。

 Twitterで探している松の木の写真があるのですが、あるツイートにテレビでは能登町だけだった地震情報が、能登町松波だけが震度3、能登町宇出津の他、七尾市能登島向田町、輪島市鳳至町、輪島市河井町、珠洲市三崎町、珠洲市正院町、珠洲市大谷町がいずれも震度2となっています。

 こういう地域の情報は余り細かく書かれないと思いますが、恋路海岸も能登町松波になりそうです。そもそも恋路の住所がよくわからないのですが、能登町恋路町ということはまずないと思います。

\begin{itemize}
\tightlist
\item
  恋路海岸 - Google マップ \url{https://t.co/ZpncaCNCmv}  〒927-0601
  石川県鳳珠郡能登町恋路8
\end{itemize}

 能登町恋路になっていましたが、考えてみれば松波も能登町松波でした。平成17年の合併前は珠洲郡内浦町の役場もあった中心地なので松波町と勘違いをしていたようです。なお、内浦町の役場の庁舎は、現在も能登町役場の一部となっているようです。

〉〉〉 kk\_hironoのリツイート 〉〉〉

\begin{itemize}
\tightlist
\item
  RT
  kk\_hirono(刑事告発・非常上告_金沢地方検察庁御中)|MikiAdjust(三木秀記
  / Adjust Photo Service) 日時:2021-06-03 11:36/2021/05/22 09:22
  URL: \url{https://twitter.com/kk\_hirono/status/1400280152064495617} 
  \url{https://twitter.com/MikiAdjust/status/1395897835636158466} 
  \textgreater{}
  北國新聞2021年5月15日夕刊「カラーでよみがえる、ふるさと」より(カラー化写真)。
  1960(昭和35)年4月17日、石川県の宇出津。
  国鉄能登線が開通したこの日、通りは祝賀行事に押し寄せた4万人の人波であふれた。
  夏のあばれ祭で繰り出す大キリコの他、仮装行列や獅子舞、旗行列が祝賀行事を飾った。
  \url{https://t.co/5vJ3vvdycd} 
\end{itemize}

 探していたツイートの写真です。新町通りにキリコが1つでやたらと人出が多いのが気になっていたのですが、ツイートの本文を読み直して、あばれ祭りではなく、「
1960(昭和35)年4月17日、石川県の宇出津。
国鉄能登線が開通したこの日、通りは祝賀行事」ということに気が付きました。

 昨夜も映画「不敵に笑う男」の撮影と公開日が昭和35年ということを確認していたのですが、国鉄能登線が開通した祝賀行事が昭和35年4月17日とあります。宇出津駅までの開通は昭和37年と見たような記憶もあるのですが、確認をしておきたいところです。

\begin{itemize}
\tightlist
\item
  のと鉄道能登線 - Wikipedia \url{https://t.co/z5Mcq6kKk6}  4月17日 - 鵜川 -
  宇出津間 (9.9km)
  延伸開業{[}16{]}{[}17{]}、同区間に1.6倍の擬制キロを設定{[}12{]}。穴水
  - 宇出津間で貨物営業を開始。矢波駅、波並駅、藤波駅、宇出津駅開業。
\end{itemize}

 宇出津駅の開業も昭和35年4月17日とありました。鵜川ー宇出津間9.9Kmとあります。緩いカーブがあるだけでほとんど一直線の線路だと思っていたのですが、これだと国道249号線より少し距離がありそうに思えます。

\begin{itemize}
\tightlist
\item
  旧のと鉄道 鵜川駅跡 から 旧宇出津駅 - Google マップ
  \url{https://t.co/RLxFu4WWiX} 
\end{itemize}

 前に調べたとき9Kmぐらいだったという記憶だったのですが、駅と駅の間の区間を調べたのは初めてで、10.3Kmとありました。

 さきほどの昭和35年4月17日の宇出津の写真ですが、庭に3階建ての木造建築の屋根を超えるような木があるのが気になっていました。今見ると、けっこう細長い木で、松の木とは少し違っているような気もしたのですが、昭和40年代中頃とは、ずいぶん異なった町並みです。

 地元でないとわからないと思いますが、バスが映っている右側の家は、道路が上り坂になっていることもあり、家の場所が見た目よりかなり低くなっています。町内は仙人町になりますが、ずいぶん大きく見える家は記憶にないもので、3階建て以上はありそうです。

 とても高く見える木の左手にビルのような建物がありますが、これが昔の北國銀行の建物だと思います。昭和40年代にはかすかに記憶に残る建物なのですが、昭和50年代にはなくなっていたように思います。取り壊しの様子なども見た記憶がないのですが、よく前を歩いていたはずです。

 その昔の北國銀行の建物というのもぼやけた記憶しかないのですが、似ているように思ったのが福井地裁の建物の写真です。建物が話題になっている情報は見かけたことがないのですが、相当古そうな建築物でした。そういえば昨日は富山地裁の建て替えのニュースを見たところでした。

〉〉〉 kk\_hironoのリツイート 〉〉〉

\begin{itemize}
\tightlist
\item
  RT
  kk\_hirono(刑事告発・非常上告_金沢地方検察庁御中)|syouwaretoroha(ラジオネーム
  げじまゆ) 日時:2021-06-03 12:13/2020/11/28 18:03 URL:
  \url{https://twitter.com/kk\_hirono/status/1400289400387293187} 
  \url{https://twitter.com/syouwaretoroha/status/1332611085539573760} 
  \textgreater{}
  \#かっこいい庁舎選手権 庁舎といえるか微妙ですが、福井地裁。なんと戦後に建てられたんですよね。モダニズム建築が一世を風靡する中、戦後に建てられた数少ない様式建築のうちの一つです
  \url{https://t.co/aYh1gQkSOK} 
\end{itemize}

 福井地裁の建物は古い様式の建築物というだけでなく、部屋数の多さも気になっていました。高裁は名古屋高裁金沢支部の管轄になるはずなのに、名古屋高裁金沢支部が入る金沢地方裁判所の建物より大きく、部屋数が多そうに思えていました。

\begin{quote}
《引用の始まり》
\end{quote}

\begin{quote}
福井地方・家庭裁判所庁舎は,昭和24年に着工されたものですが,戦後の裁判所庁舎建築の先駆けとなる庁舎で,被災した福井市の復興とまさに期を同じくして建設されました。昭和28年11月の落成式における田中耕太郎最高裁長官の祝辞にも,「福井地方裁判所旧庁舎は,昭和20年7月戦災により焼失し,再建後日なお浅い前庁舎も昭和23年6月不測の大震災により烏有に帰して,相次ぐ惨禍を被りましたが・・・・今まのあたりに見る北陸地方に冠絶する規模と新機軸を織り込んだ偉容が竣工したのであります。」とあるように,福井市復興の大きなシンボルとなったものです。
\end{quote}

\begin{quote}
《引用の終わり》
\end{quote}

\begin{itemize}
\tightlist
\item
  福井地方・家庭裁判所の紹介 \textbar{} 裁判所
  \url{https://www.courts.go.jp/fukui/about/syokai/index.html} 
\end{itemize}

 調べてみると福井地裁のホームページに詳しい情報がありましたが、福井地裁の建物は昭和20年7月に戦災で消失し、再建後の昭和23年6月は福井地震の大震災により烏有に帰した、とあります。福井市復興の大きなシンボルになったともあります。

 前に、テレビでアパホテルの名物女社長をみたことがきっかけで、福井地震について調べたことがあったのですが、東京大空襲より人口密度の死亡率が高かったという話でした。その福井地震で、仏壇の観音扉が開いたまま倒れ、赤子だった女社長が助かったという話もありました。

 アパホテルはもともと金沢市の会社だったと思いますが、女社長の夫の元谷外志雄氏が、本を出していたはずで、北國新聞の広告で見たという記憶なのですが、本の題名なのか、「快刀乱麻」などとあったことをよく憶えています。見た場所も金沢刑務所の拘置所の独居房でした。

\begin{itemize}
\tightlist
\item
  「異端児の哲学 快刀乱麻 自らの世界観で時代を駆け抜ける男」 \textbar{}
  元谷 外志雄 \textbar 本 \textbar{} 通販 \textbar{} Amazon
  \url{https://t.co/Mmo5mCL1eC}  ¥\n 1993/5/30
\end{itemize}

 スーパーカーの写真が表紙にあったことはよく憶えていたのですが、ずいぶん若く見える写真で、その後の顔写真とは雰囲気も含め別人に思えます。女社長がテレビにもよく出て全国的な知名度となっていったのですが、それより何年も前のことで、本も平成5年5月30日とあります。

 今はアパホテルとなっていますが、始めの頃はマンションの賃貸物件で不動産業というイメージが強く、金沢というか石川県の不動産業で3本の指に入るとも聞いたのが本陣不動産株式会社や本陣グループでした。

 被告発人大網健二がその本陣不動産株式会社に入社したのは平成元年頃だったと思いますが、平成9年には営業課長となっていました。営業課長という名刺ももらっていたと思いますが、平成11年の1月になると、会社をやめたのかはっきりしない言動をするようになり、それでも会社には出入りしていました。

 はっきりとは思い出せないですが、契約社員になったという話だったかもしれません。そういえば、最初に私の面接をすると言い出した場所も、駅西本町の本陣不動産株式会社のなかでした。

 本の表紙にある赤いスポーツカーですが、ランボルギーニ・カウンタックかミウラのいずれかと思われます。私が宇出津小学校の5年生か6先生の頃に、スーパーカーブームというのがありました。プラモデルも買って作ったような記憶です。

 やはり出版社が、北國新聞社出版局とあります。アパなんとかという大きな看板の分譲マンションのようなものを金沢市内で見かけることが多くなりましたが、身近で話題を聞いたことはなかったように思います。

 樋詰哲朗弁護士(金沢弁護士会)のことですが、法律事務所の住所が金沢市橋場ということでも随分前から気になっていました。あるいは橋場町になるのかもしれないですが、これから確認します。

\begin{itemize}
\tightlist
\item
  橋場町 - Google マップ \url{https://t.co/7gUj3KpTeX} 
\end{itemize}

 金沢市橋場でもGoogleマップでそのまま検索できたのですが、金沢市橋場町が正しかったようです。金沢市内の知られた住所では割合珍しいようにも思ったのですが、橋場町だけで丁目という区分はないようです。

 橋場という大きな交差点がありますが、この交差点の近くに古いパチンコ店があって、他には見かけなくなった古い時代のパチンコ店だったのでとても印象的に憶えています。金沢市場輸送の運転手とも関わりがあったことは、昨夜、ブログの記事にしていたことも確認しています。

 実際にチンドン屋というのを見たという記憶ははっきりしないのですが、テレビで見たのかパチンコ店の新装開店にはチンドン屋がつきものというイメージがありました。これも実際に見た記憶はないですが大道芸人にも通じる古い時代の文化ではないかと思います。

 私は金沢に親戚がなかったのですが、子供の頃に何度か遊びに行った記憶があり、その宿泊先として1つだけ記憶にあるのが卯辰山の金沢ヘルスセンターでした。正確な情報は忘れましたが、昭和60年代に入った頃、サニーランドに名称変更となっていましたが、どちらも動物園がメインの宿泊施設でした。

 金沢ヘルスセンターでは、これも実際に見た記憶がないので、館内にあるポスターを見ていたのかもしれないですが、よく旅芸人の一座の公演が行われていたようです。

 2,3日前、Googleマップで見つけた大手町の交差点ですが、金沢駅の方から来て交差点を左折すると、Googleマップに卯辰山公園線とある道路です。浅野川に架かる天神橋を渡ると曲がりくねった上り坂になり、登りきった先に金沢ヘルスセンターがありました。

 金沢駅からどうやっていったのか記憶はないのですが、金沢駅から橋場町を通ったその道路は、幼い頃の記憶として強く残っています。金沢駅から送迎バスのようなものがあったのかもしれないですが、バスに乗ったという記憶もはっきりしません。

 平成元年の5月か6月から始め11月いっぱいだったと思いますが、金沢市場輸送の古い4トン保冷車で、北都運輸の市内配達をしたことがありました。メインの荷物は業務用のマヨネーズやドレッシング、ジャムがありました。

\begin{itemize}
\tightlist
\item
  材木町 - Google マップ \url{https://t.co/O2em98rjoA} 
\end{itemize}

 卯辰山公園線の卯辰山に向かって右側の住所を調べてみると、金沢市材木町で、見聞きした憶えのある住所だと思ったのですが、Googleマップの検索で表示すると、ずいぶん広い範囲なのだとわかりました。なお、左側の道路のほとんどが橋場町となっていますが、割と最近まで知りませんでした。

 これも樋詰哲朗弁護士(金沢弁護士会)の法律事務所を調べたことがきっかけで、橋場町の範囲がわかったのですが、橋場の交差点の付近だけが橋場町というイメージでした。浅野川沿いは並木町となっているようですが、天神橋のすぐ近くまで橋場町となっています。

 Googleマップに辻文食料品店というのが見えますが、ここもたまに行くことのある市内配達の配達先になっていたように思います。

 材木町の方になりますが、右手の道路に入った右側に、時代劇に出てくる大店のような店があり、醤油屋だったような気もするのですが、北都運輸の市内配達でもそこもよく行っていました。そして毎回、天神橋を渡った先で、無理に4トン保冷車をUターンで方向転換させていたこともよく憶えています。

 仕事以外では平成3年の春になると思いますが、2人の子供を連れて家族4人で、サニーランドに動物を見に行ったことがありました。動物園から離れた場所に水族館もありました。YouTubeの動画でも視聴しています。

\begin{itemize}
\tightlist
\item
  金沢(その3) - YouTube \url{https://t.co/wDdt39Agjk} 
\end{itemize}

 かなざわ水族館、サニーランドの映像が出てきた後、続けて視聴していると、最後に松原病院と思われる建物が出てきました。小立野の近くですが住所は金沢市石引になっていたかと思います。ずいぶん大きな建物だと思いましたが、今も現存するのか確認しておきたいと考えていました。

 今は他にも大きなビルがありそうですが、昭和60年代は近くに大きなビルがなく、いっそう目立つ存在感があった松原病院になります。少なくとも昭和の時代は、事実上、精神病院や精神異常者の代名詞でもあった松原病院です。

\begin{itemize}
\tightlist
\item
  松原病院 - Google マップ \url{https://t.co/KzXiykAmHu}  ¥\n 〒920-8654
  石川県金沢市石引4丁目3−5
\end{itemize}

 ざっと数えて地上9階建ての建物ですが、記憶よりだいぶん大きく見えたことと、同じ場所での建て替えがあったとも考えにくいのですが、昭和57年1月当時と同じ建物とは思えない金沢市の松原病院です。

 ここは兼六園の裏側から一直線の道路になり、さきほどGoogleマップで見たところ小立野通りとなっていました。金沢市に精神病患者の比率が高いという話は聞いたことがないのですが、相当な需要もありそうだと思えてきました。

 昭和57年1月であるいは2月、3月だったのかもしれないですが、3人で兼六園にいたところ、一人が松原病院に被告発人安田敏が入院しているはずなので見舞いに行こうと言い出したのです。その言い出しっぺが、とっさの機転だったのか被告発人安田敏の弟だと受付で言ったのですが、それが通用しました。

 昭和57年1月であるいは2月、3月だったのかもしれないですが、3人で兼六園にいたところ、一人が松原病院に被告発人安田敏が入院しているはずなので見舞いに行こうと言い出したのです。その言い出しっぺが、とっさの機転だったのか被告発人安田敏の弟だと受付で言ったのですが、それが通用しました。

 表の一階に受付があり、そこで当時の看護婦の女の人と話をしたのですが、裏口の方からエレベーターで上の階まで行きました。とんでもない大きさの鉄扉を見たのも衝撃的によく憶えています。閉鎖病棟でした。

 そういえば昨日、Googleマップで金沢市花里の平成3年当時の被告発人安田敏が住んでいたアパートから市場急配センターへの経路を調べるつもりでいたのですが、実行しないままになっていました。

\begin{itemize}
\tightlist
\item
  金沢市中央卸売市場 から 〒920-0951 石川県金沢市花里町15−5 - Google
  マップ \url{https://t.co/2wsuRzMy2k} 
\end{itemize}

 市場急配センターの事務所の場所が変わっているはずなので、金沢中央卸売市場を指定しました。7.7Kmで24分とかあります。玉川図書館の前を通る道が選択されていますが、これは私がしらない新しい道路のようです。

 市場急配センターで仕事が終わってから被告発人安田敏の花里町のアパートまで遊びに行くことが数回あったのですが、被告発人安田敏は松原病院の前を通って、小立野から三口新町に出ていたような記憶があります。小立野のゲームセンターに入ったようなこともありました。

 小立野といえば、平成2年の春になるのかと思いますが、酒を飲んだ店の前で被告発人安田敏がへたり込み、確かタクシー代をもらって、タクシーに乗り込んで別れたということがありました。矢沢永吉に似たマスターの店でアメリカのような雰囲気の店でしたが、飲んだのもバーボンだったかもしれません。

 矢沢永吉に似ていたというか似せていたというスタイルでしたが、花里町のアパートではけっこうたくさんの写真が入った箱があって、きやすく被告発人安田敏は私にそれを見せていたのですが、東京の池袋のサンシャイン60とかいうビルを背景に被告発人安田敏と2人で写る写真もあった人物です。

 店は金沢大学病院の前にあったと思うのですが、若者向けの飲み屋が集まる一角で、市内配達の仕事ではすぐ近くまで行っていたものの、気が付かなかったような場所でした。

 被告発人安田敏があのような写真を私に見せていたのも、気を許し油断させるのが目的だったと思うのですが、なにか写真の種類にもずいぶん選別されたような偏りを感じました。

 被告発人安田敏の妻となった女性については、その不審な言動を含め一通りのことを令和3年3月31日付告発状に記載したと思います。

 その妻となった女性の写真も同じ写真の箱に入っていました。数は少なかったようにも思いますが、有線と聞いた会社での写真で男女6人ぐらいで写った写真だけが印象に残っています。仲睦まじく楽しそうな雰囲気の集合写真でした。

 このあと連続で写真をアップロードしますが、またパソコンならではの意外な発見がありました。2018年9月の写真です。

〉〉〉 kk\_hironoのリツイート 〉〉〉

\begin{itemize}
\tightlist
\item
  RT
  kk\_hirono(刑事告発・非常上告_金沢地方検察庁御中)|s\_hirono(非常上告-最高検察庁御中\_ツイッター)
  日時:2021-06-03 15:53/2021/06/03 15:50 URL:
  \url{https://twitter.com/kk\_hirono/status/1400344826474680320} 
  \url{https://twitter.com/s\_hirono/status/1400344201619906561} 
  \textgreater{}
  2018-09-05\_063845_北陸中日新聞 また雨 能登疲弊 台風21号 県内防風 不安な一夜.jpg
  \url{https://t.co/nC38LFBtrr} 
\end{itemize}

〉〉〉 kk\_hironoのリツイート 〉〉〉

\begin{itemize}
\tightlist
\item
  RT
  kk\_hirono(刑事告発・非常上告_金沢地方検察庁御中)|s\_hirono(非常上告-最高検察庁御中\_ツイッター)
  日時:2021-06-03 15:53/2021/06/03 15:50 URL:
  \url{https://twitter.com/kk\_hirono/status/1400344850428428292} 
  \url{https://twitter.com/s\_hirono/status/1400344187376062467} 
  \textgreater{}
  2018-09-05\_063818_北陸中日新聞 県内 3人けが1万4480戸停電 台風21号.jpg
  \url{https://t.co/ayk9PP79sG} 
\end{itemize}

〉〉〉 kk\_hironoのリツイート 〉〉〉

\begin{itemize}
\tightlist
\item
  RT
  kk\_hirono(刑事告発・非常上告_金沢地方検察庁御中)|s\_hirono(非常上告-最高検察庁御中\_ツイッター)
  日時:2021-06-03 15:53/2021/06/03 15:50 URL:
  \url{https://twitter.com/kk\_hirono/status/1400344866219913218} 
  \url{https://twitter.com/s\_hirono/status/1400344172977025026} 
  \textgreater{}
  2018-09-05\_063740_北陸中日新聞 関空連絡橋タンカー衝突 台風21号全国で7人死亡.jpg
  \url{https://t.co/Q5wzCy2E4k} 
\end{itemize}

〉〉〉 kk\_hironoのリツイート 〉〉〉

\begin{itemize}
\tightlist
\item
  RT
  kk\_hirono(刑事告発・非常上告_金沢地方検察庁御中)|s\_hirono(非常上告-最高検察庁御中\_ツイッター)
  日時:2021-06-03 15:53/2021/06/03 15:50 URL:
  \url{https://twitter.com/kk\_hirono/status/1400344893310992387} 
  \url{https://twitter.com/s\_hirono/status/1400344159303598081} 
  \textgreater{}
  2018-09-04\_183045_台風21号の強風で落ち、ガラスが割れた古い窓枠.jpg
  \url{https://t.co/ReXfygyq2Z} 
\end{itemize}

〉〉〉 kk\_hironoのリツイート 〉〉〉

\begin{itemize}
\tightlist
\item
  RT
  kk\_hirono(刑事告発・非常上告_金沢地方検察庁御中)|s\_hirono(非常上告-最高検察庁御中\_ツイッター)
  日時:2021-06-03 15:53/2021/06/03 15:50 URL:
  \url{https://twitter.com/kk\_hirono/status/1400344908351762434} 
  \url{https://twitter.com/s\_hirono/status/1400344146015965186} 
  \textgreater{}
  2018-09-04\_183028_台風21号の強風で外れた古い窓枠.jpg
  \url{https://t.co/WLsJ2XIxTk} 
\end{itemize}

〉〉〉 kk\_hironoのリツイート 〉〉〉

\begin{itemize}
\tightlist
\item
  RT
  kk\_hirono(刑事告発・非常上告_金沢地方検察庁御中)|s\_hirono(非常上告-最高検察庁御中\_ツイッター)
  日時:2021-06-03 15:53/2021/06/03 15:50 URL:
  \url{https://twitter.com/kk\_hirono/status/1400344939985137666} 
  \url{https://twitter.com/s\_hirono/status/1400344132573298688} 
  \textgreater{}
  2018-09-04\_151726_宇出津新港 時の広場 海蔵院末観音堂方面.jpg
  \url{https://t.co/rBnmnOtfm6} 
\end{itemize}

〉〉〉 kk\_hironoのリツイート 〉〉〉

\begin{itemize}
\tightlist
\item
  RT
  kk\_hirono(刑事告発・非常上告_金沢地方検察庁御中)|s\_hirono(非常上告-最高検察庁御中\_ツイッター)
  日時:2021-06-03 15:53/2021/06/03 15:50 URL:
  \url{https://twitter.com/kk\_hirono/status/1400344963553009665} 
  \url{https://twitter.com/s\_hirono/status/1400344118283292672} 
  \textgreater{}
  2018-09-04\_131648_宇出津新港 釣り公園 エギングをする人.jpg
  \url{https://t.co/r0HqjTTQLm} 
\end{itemize}

〉〉〉 kk\_hironoのリツイート 〉〉〉

\begin{itemize}
\tightlist
\item
  RT
  kk\_hirono(刑事告発・非常上告_金沢地方検察庁御中)|s\_hirono(非常上告-最高検察庁御中\_ツイッター)
  日時:2021-06-03 15:53/2021/06/03 15:50 URL:
  \url{https://twitter.com/kk\_hirono/status/1400344980984504320} 
  \url{https://twitter.com/s\_hirono/status/1400344104664322050} 
  \textgreater{} 2018-09-04\_123702_宇出津新港 能登高校南のバス停.jpg
  \url{https://t.co/pbxH2HYE1l} 
\end{itemize}

〉〉〉 kk\_hironoのリツイート 〉〉〉

\begin{itemize}
\tightlist
\item
  RT
  kk\_hirono(刑事告発・非常上告_金沢地方検察庁御中)|s\_hirono(非常上告-最高検察庁御中\_ツイッター)
  日時:2021-06-03 15:54/2021/06/03 15:50 URL:
  \url{https://twitter.com/kk\_hirono/status/1400345001674952710} 
  \url{https://twitter.com/s\_hirono/status/1400344090751889408} 
  \textgreater{}
  2018-09-04\_123402_かね八前から宇出津港、能登町役場方面.jpg
  \url{https://t.co/42ojgplTAm} 
\end{itemize}

〉〉〉 kk\_hironoのリツイート 〉〉〉

\begin{itemize}
\tightlist
\item
  RT
  kk\_hirono(刑事告発・非常上告_金沢地方検察庁御中)|s\_hirono(非常上告-最高検察庁御中\_ツイッター)
  日時:2021-06-03 15:54/2021/06/03 15:50 URL:
  \url{https://twitter.com/kk\_hirono/status/1400345026073223169} 
  \url{https://twitter.com/s\_hirono/status/1400344076986118146} 
  \textgreater{}
  2018-09-03\_184635_上町の病院から柳田温泉 笹川のバス停.jpg
  \url{https://t.co/CST3JoH3ty} 
\end{itemize}

〉〉〉 kk\_hironoのリツイート 〉〉〉

\begin{itemize}
\tightlist
\item
  RT
  kk\_hirono(刑事告発・非常上告_金沢地方検察庁御中)|s\_hirono(非常上告-最高検察庁御中\_ツイッター)
  日時:2021-06-03 15:54/2021/06/03 15:50 URL:
  \url{https://twitter.com/kk\_hirono/status/1400345042984660996} 
  \url{https://twitter.com/s\_hirono/status/1400344063203708928} 
  \textgreater{}
  2018-09-03\_165622_上町の病院から柳田温泉 柳田温泉.jpg
  \url{https://t.co/BUS3pJ1Gu8} 
\end{itemize}

〉〉〉 kk\_hironoのリツイート 〉〉〉

\begin{itemize}
\tightlist
\item
  RT
  kk\_hirono(刑事告発・非常上告_金沢地方検察庁御中)|s\_hirono(非常上告-最高検察庁御中\_ツイッター)
  日時:2021-06-03 15:54/2021/06/03 15:50 URL:
  \url{https://twitter.com/kk\_hirono/status/1400345057421447170} 
  \url{https://twitter.com/s\_hirono/status/1400344048875937792} 
  \textgreater{}
  2018-09-03\_164409_上町の病院から柳田温泉 柳田白山神社の横の雑木林前 見たことのない鮮やかな色の竹.jpg
  \url{https://t.co/r6xGN7idaO} 
\end{itemize}

〉〉〉 kk\_hironoのリツイート 〉〉〉

\begin{itemize}
\tightlist
\item
  RT
  kk\_hirono(刑事告発・非常上告_金沢地方検察庁御中)|s\_hirono(非常上告-最高検察庁御中\_ツイッター)
  日時:2021-06-03 15:54/2021/06/03 15:50 URL:
  \url{https://twitter.com/kk\_hirono/status/1400345071346540547} 
  \url{https://twitter.com/s\_hirono/status/1400344033755500549} 
  \textgreater{}
  2018-09-03\_164341_上町の病院から柳田温泉 柳田白山神社 白雉山 安養寺址.jpg
  \url{https://t.co/TEMP5WtzCe} 
\end{itemize}

〉〉〉 kk\_hironoのリツイート 〉〉〉

\begin{itemize}
\tightlist
\item
  RT
  kk\_hirono(刑事告発・非常上告_金沢地方検察庁御中)|s\_hirono(非常上告-最高検察庁御中\_ツイッター)
  日時:2021-06-03 15:54/2021/06/03 15:50 URL:
  \url{https://twitter.com/kk\_hirono/status/1400345085431074816} 
  \url{https://twitter.com/s\_hirono/status/1400344019402514432} 
  \textgreater{}
  2018-09-03\_164016_上町の病院から柳田温泉 柳田白山神社.jpg
  \url{https://t.co/S5GjRRkL6V} 
\end{itemize}

〉〉〉 kk\_hironoのリツイート 〉〉〉

\begin{itemize}
\tightlist
\item
  RT
  kk\_hirono(刑事告発・非常上告_金沢地方検察庁御中)|s\_hirono(非常上告-最高検察庁御中\_ツイッター)
  日時:2021-06-03 15:54/2021/06/03 15:50 URL:
  \url{https://twitter.com/kk\_hirono/status/1400345111297359875} 
  \url{https://twitter.com/s\_hirono/status/1400344005401997316} 
  \textgreater{}
  2018-09-03\_163122_上町の病院から柳田温泉 旧柳田村 四谷というバス停の小屋.jpg
  \url{https://t.co/yLiZgxGkvV} 
\end{itemize}

〉〉〉 kk\_hironoのリツイート 〉〉〉

\begin{itemize}
\tightlist
\item
  RT
  kk\_hirono(刑事告発・非常上告_金沢地方検察庁御中)|s\_hirono(非常上告-最高検察庁御中\_ツイッター)
  日時:2021-06-03 15:54/2021/06/03 15:49 URL:
  \url{https://twitter.com/kk\_hirono/status/1400345126740787200} 
  \url{https://twitter.com/s\_hirono/status/1400343991174918144} 
  \textgreater{}
  2018-09-03\_162653_上町の病院から柳田温泉 旧柳田村笹川 川沿い バス停方面.jpg
  \url{https://t.co/XedNhdwhKh} 
\end{itemize}

〉〉〉 kk\_hironoのリツイート 〉〉〉

\begin{itemize}
\tightlist
\item
  RT
  kk\_hirono(刑事告発・非常上告_金沢地方検察庁御中)|s\_hirono(非常上告-最高検察庁御中\_ツイッター)
  日時:2021-06-03 15:54/2021/06/03 15:49 URL:
  \url{https://twitter.com/kk\_hirono/status/1400345144331673605} 
  \url{https://twitter.com/s\_hirono/status/1400343976788467714} 
  \textgreater{}
  2018-09-03\_162516_上町の病院から柳田温泉 旧柳田村 笹川のバス停前の道路.jpg
  \url{https://t.co/s4O95Yz2xj} 
\end{itemize}

〉〉〉 kk\_hironoのリツイート 〉〉〉

\begin{itemize}
\tightlist
\item
  RT
  kk\_hirono(刑事告発・非常上告_金沢地方検察庁御中)|s\_hirono(非常上告-最高検察庁御中\_ツイッター)
  日時:2021-06-03 16:09/2021/06/03 16:06 URL:
  \url{https://twitter.com/kk\_hirono/status/1400348959328018436} 
  \url{https://twitter.com/s\_hirono/status/1400348271382462464} 
  \textgreater{}
  2018-09-17\_005451_柳田大祭 宇出津 梶川橋の手前 珠洲まで23キロ、松波まで12キロの道路案内標識.jpg
  \url{https://t.co/0ujp3FiGfq} 
\end{itemize}

〉〉〉 kk\_hironoのリツイート 〉〉〉

\begin{itemize}
\tightlist
\item
  RT
  kk\_hirono(刑事告発・非常上告_金沢地方検察庁御中)|s\_hirono(非常上告-最高検察庁御中\_ツイッター)
  日時:2021-06-03 16:09/2021/06/03 16:06 URL:
  \url{https://twitter.com/kk\_hirono/status/1400348975861944328} 
  \url{https://twitter.com/s\_hirono/status/1400348247772733441} 
  \textgreater{}
  2018-09-17\_005338_柳田大祭 宇出津新町 茂平食堂前の道路.jpg
  \url{https://t.co/Jjyk89wkCT} 
\end{itemize}

〉〉〉 kk\_hironoのリツイート 〉〉〉

\begin{itemize}
\tightlist
\item
  RT
  kk\_hirono(刑事告発・非常上告_金沢地方検察庁御中)|s\_hirono(非常上告-最高検察庁御中\_ツイッター)
  日時:2021-06-03 16:09/2021/06/03 16:06 URL:
  \url{https://twitter.com/kk\_hirono/status/1400348986523799560} 
  \url{https://twitter.com/s\_hirono/status/1400348223076671489} 
  \textgreater{}
  2018-09-17\_003157_柳田大祭 旧柳田村 笹川のバス停.jpg
  \url{https://t.co/WP4oGJaby6} 
\end{itemize}

〉〉〉 kk\_hironoのリツイート 〉〉〉

\begin{itemize}
\tightlist
\item
  RT
  kk\_hirono(刑事告発・非常上告_金沢地方検察庁御中)|s\_hirono(非常上告-最高検察庁御中\_ツイッター)
  日時:2021-06-03 16:09/2021/06/03 16:06 URL:
  \url{https://twitter.com/kk\_hirono/status/1400349002772606976} 
  \url{https://twitter.com/s\_hirono/status/1400348198254743554} 
  \textgreater{} 2018-09-17\_002947_柳田大祭.jpg
  \url{https://t.co/MDribtQs4j} 
\end{itemize}

〉〉〉 kk\_hironoのリツイート 〉〉〉

\begin{itemize}
\tightlist
\item
  RT
  kk\_hirono(刑事告発・非常上告_金沢地方検察庁御中)|s\_hirono(非常上告-最高検察庁御中\_ツイッター)
  日時:2021-06-03 16:09/2021/06/03 16:06 URL:
  \url{https://twitter.com/kk\_hirono/status/1400349017314193414} 
  \url{https://twitter.com/s\_hirono/status/1400348173365743619} 
  \textgreater{} 2018-09-17\_002928_柳田大祭.jpg
  \url{https://t.co/YMbXi5Hsxq} 
\end{itemize}

〉〉〉 kk\_hironoのリツイート 〉〉〉

\begin{itemize}
\tightlist
\item
  RT
  kk\_hirono(刑事告発・非常上告_金沢地方検察庁御中)|s\_hirono(非常上告-最高検察庁御中\_ツイッター)
  日時:2021-06-03 16:10/2021/06/03 16:06 URL:
  \url{https://twitter.com/kk\_hirono/status/1400349039695065095} 
  \url{https://twitter.com/s\_hirono/status/1400348148892008452} 
  \textgreater{}
  2018-09-17\_002806_柳田大祭 旧柳田村中心部の坂 町野方面.jpg
  \url{https://t.co/bk73cdkmDL} 
\end{itemize}

〉〉〉 kk\_hironoのリツイート 〉〉〉

\begin{itemize}
\tightlist
\item
  RT
  kk\_hirono(刑事告発・非常上告_金沢地方検察庁御中)|s\_hirono(非常上告-最高検察庁御中\_ツイッター)
  日時:2021-06-03 16:10/2021/06/03 16:06 URL:
  \url{https://twitter.com/kk\_hirono/status/1400349053833994243} 
  \url{https://twitter.com/s\_hirono/status/1400348122853769219} 
  \textgreater{} 2018-09-17\_002122_柳田大祭.jpg
  \url{https://t.co/oQi55yfcs0} 
\end{itemize}

〉〉〉 kk\_hironoのリツイート 〉〉〉

\begin{itemize}
\tightlist
\item
  RT
  kk\_hirono(刑事告発・非常上告_金沢地方検察庁御中)|s\_hirono(非常上告-最高検察庁御中\_ツイッター)
  日時:2021-06-03 16:10/2021/06/03 16:06 URL:
  \url{https://twitter.com/kk\_hirono/status/1400349066282749955} 
  \url{https://twitter.com/s\_hirono/status/1400348100435222529} 
  \textgreater{} 2018-09-17\_001939_柳田大祭 後にするお旅所.jpg
  \url{https://t.co/1B3Bk6wi8O} 
\end{itemize}

〉〉〉 kk\_hironoのリツイート 〉〉〉

\begin{itemize}
\tightlist
\item
  RT
  kk\_hirono(刑事告発・非常上告_金沢地方検察庁御中)|s\_hirono(非常上告-最高検察庁御中\_ツイッター)
  日時:2021-06-03 16:10/2021/06/03 16:06 URL:
  \url{https://twitter.com/kk\_hirono/status/1400349099799445509} 
  \url{https://twitter.com/s\_hirono/status/1400348076515094528} 
  \textgreater{} 2018-09-17\_001740_燃えるお旅所の松明.jpg
  \url{https://t.co/Ews42clOaG} 
\end{itemize}

〉〉〉 kk\_hironoのリツイート 〉〉〉

\begin{itemize}
\tightlist
\item
  RT
  kk\_hirono(刑事告発・非常上告_金沢地方検察庁御中)|s\_hirono(非常上告-最高検察庁御中\_ツイッター)
  日時:2021-06-03 16:10/2021/06/03 16:06 URL:
  \url{https://twitter.com/kk\_hirono/status/1400349125997076483} 
  \url{https://twitter.com/s\_hirono/status/1400348051324112896} 
  \textgreater{}
  2018-09-17\_001659_燃えるお旅所の松明 2時間近く火にさらされ燃えなかった青竹と御幣.jpg
  \url{https://t.co/APomA0ipOv} 
\end{itemize}

〉〉〉 kk\_hironoのリツイート 〉〉〉

\begin{itemize}
\tightlist
\item
  RT
  kk\_hirono(刑事告発・非常上告_金沢地方検察庁御中)|s\_hirono(非常上告-最高検察庁御中\_ツイッター)
  日時:2021-06-03 16:10/2021/06/03 16:06 URL:
  \url{https://twitter.com/kk\_hirono/status/1400349142270894081} 
  \url{https://twitter.com/s\_hirono/status/1400348026577723400} 
  \textgreater{}
  2018-09-17\_001617_燃えるお旅所の松明 燃えなかった御幣.jpg
  \url{https://t.co/1rSjGOYRI2} 
\end{itemize}

〉〉〉 kk\_hironoのリツイート 〉〉〉

\begin{itemize}
\tightlist
\item
  RT
  kk\_hirono(刑事告発・非常上告_金沢地方検察庁御中)|s\_hirono(非常上告-最高検察庁御中\_ツイッター)
  日時:2021-06-03 16:10/2021/06/03 16:05 URL:
  \url{https://twitter.com/kk\_hirono/status/1400349173161938945} 
  \url{https://twitter.com/s\_hirono/status/1400348002351341569} 
  \textgreater{} 2018-09-16\_235244_燃えるお旅所の松明.jpg
  \url{https://t.co/dlX3Egcwj7} 
\end{itemize}

〉〉〉 kk\_hironoのリツイート 〉〉〉

\begin{itemize}
\tightlist
\item
  RT
  kk\_hirono(刑事告発・非常上告_金沢地方検察庁御中)|s\_hirono(非常上告-最高検察庁御中\_ツイッター)
  日時:2021-06-03 16:10/2021/06/03 16:05 URL:
  \url{https://twitter.com/kk\_hirono/status/1400349192833277954} 
  \url{https://twitter.com/s\_hirono/status/1400347978498338823} 
  \textgreater{} 2018-09-16\_232103_燃えるお旅所の松明.jpg
  \url{https://t.co/RfOzKyiQWS} 
\end{itemize}

〉〉〉 kk\_hironoのリツイート 〉〉〉

\begin{itemize}
\tightlist
\item
  RT
  kk\_hirono(刑事告発・非常上告_金沢地方検察庁御中)|s\_hirono(非常上告-最高検察庁御中\_ツイッター)
  日時:2021-06-03 16:10/2021/06/03 16:05 URL:
  \url{https://twitter.com/kk\_hirono/status/1400349208062746626} 
  \url{https://twitter.com/s\_hirono/status/1400347956541214720} 
  \textgreater{} 2018-09-16\_231716_燃えるお旅所の松明.jpg
  \url{https://t.co/bpbzyX2fVT} 
\end{itemize}

〉〉〉 kk\_hironoのリツイート 〉〉〉

\begin{itemize}
\tightlist
\item
  RT
  kk\_hirono(刑事告発・非常上告_金沢地方検察庁御中)|s\_hirono(非常上告-最高検察庁御中\_ツイッター)
  日時:2021-06-03 16:10/2021/06/03 16:05 URL:
  \url{https://twitter.com/kk\_hirono/status/1400349233719304192} 
  \url{https://twitter.com/s\_hirono/status/1400347934990897156} 
  \textgreater{}
  2018-09-16\_231506_柳田大祭 22時30分に火をつけられたというお旅所の松明.jpg
  \url{https://t.co/n0oLRTVJ3i} 
\end{itemize}

〉〉〉 kk\_hironoのリツイート 〉〉〉

\begin{itemize}
\tightlist
\item
  RT
  kk\_hirono(刑事告発・非常上告_金沢地方検察庁御中)|s\_hirono(非常上告-最高検察庁御中\_ツイッター)
  日時:2021-06-03 16:10/2021/06/03 16:05 URL:
  \url{https://twitter.com/kk\_hirono/status/1400349262521671689} 
  \url{https://twitter.com/s\_hirono/status/1400347915516739587} 
  \textgreater{} 2018-09-16\_221401_柳田大祭 柳田白山神社 竹.jpg
  \url{https://t.co/lqRbFG4lGr} 
\end{itemize}

〉〉〉 kk\_hironoのリツイート 〉〉〉

\begin{itemize}
\tightlist
\item
  RT
  kk\_hirono(刑事告発・非常上告_金沢地方検察庁御中)|s\_hirono(非常上告-最高検察庁御中\_ツイッター)
  日時:2021-06-03 16:10/2021/06/03 16:05 URL:
  \url{https://twitter.com/kk\_hirono/status/1400349277642104832} 
  \url{https://twitter.com/s\_hirono/status/1400347887087669248} 
  \textgreater{} 2018-09-16\_215323_柳田大祭 柳田白山神社 キリコ.jpg
  \url{https://t.co/fap8sDdMNr} 
\end{itemize}

〉〉〉 kk\_hironoのリツイート 〉〉〉

\begin{itemize}
\tightlist
\item
  RT
  kk\_hirono(刑事告発・非常上告_金沢地方検察庁御中)|s\_hirono(非常上告-最高検察庁御中\_ツイッター)
  日時:2021-06-03 16:11/2021/06/03 16:05 URL:
  \url{https://twitter.com/kk\_hirono/status/1400349290757722116} 
  \url{https://twitter.com/s\_hirono/status/1400347865499639809} 
  \textgreater{}
  2018-09-16\_210434_柳田大祭 国民宿舎能登やなぎだ荘 浴場.jpg
  \url{https://t.co/3nv5zqOUZM} 
\end{itemize}

〉〉〉 kk\_hironoのリツイート 〉〉〉

\begin{itemize}
\tightlist
\item
  RT
  kk\_hirono(刑事告発・非常上告_金沢地方検察庁御中)|s\_hirono(非常上告-最高検察庁御中\_ツイッター)
  日時:2021-06-03 16:11/2021/06/03 16:05 URL:
  \url{https://twitter.com/kk\_hirono/status/1400349305819467780} 
  \url{https://twitter.com/s\_hirono/status/1400347851830349827} 
  \textgreater{} 2018-09-16\_204425_柳田大祭 柳田白山神社 到着.jpg
  \url{https://t.co/hrbuW3Ple8} 
\end{itemize}

〉〉〉 kk\_hironoのリツイート 〉〉〉

\begin{itemize}
\tightlist
\item
  RT
  kk\_hirono(刑事告発・非常上告_金沢地方検察庁御中)|s\_hirono(非常上告-最高検察庁御中\_ツイッター)
  日時:2021-06-03 16:11/2021/06/03 16:05 URL:
  \url{https://twitter.com/kk\_hirono/status/1400349318872244226} 
  \url{https://twitter.com/s\_hirono/status/1400347838794539010} 
  \textgreater{}
  2018-09-16\_203012_柳田大祭 能登町上町中又 汗かき地蔵.jpg
  \url{https://t.co/CiVk1MjPlS} 
\end{itemize}

〉〉〉 kk\_hironoのリツイート 〉〉〉

\begin{itemize}
\tightlist
\item
  RT
  kk\_hirono(刑事告発・非常上告_金沢地方検察庁御中)|s\_hirono(非常上告-最高検察庁御中\_ツイッター)
  日時:2021-06-03 16:11/2021/06/03 16:05 URL:
  \url{https://twitter.com/kk\_hirono/status/1400349344843194371} 
  \url{https://twitter.com/s\_hirono/status/1400347825376960512} 
  \textgreater{}
  2018-09-13\_062122_北陸中日新聞 野球部員転落死で告訴 父親 金沢西高指導者3人を.jpg
  \url{https://t.co/AtswInJovN} 
\end{itemize}

〉〉〉 kk\_hironoのリツイート 〉〉〉

\begin{itemize}
\tightlist
\item
  RT
  kk\_hirono(刑事告発・非常上告_金沢地方検察庁御中)|s\_hirono(非常上告-最高検察庁御中\_ツイッター)
  日時:2021-06-03 16:11/2021/06/03 16:05 URL:
  \url{https://twitter.com/kk\_hirono/status/1400349361578545155} 
  \url{https://twitter.com/s\_hirono/status/1400347811183349760} 
  \textgreater{}
  2018-09-05\_195033_辺田の浜バス停 能登町立能都中学校300m看板.jpg
  \url{https://t.co/NHRsycDmaX} 
\end{itemize}

〉〉〉 kk\_hironoのリツイート 〉〉〉

\begin{itemize}
\tightlist
\item
  RT
  kk\_hirono(刑事告発・非常上告_金沢地方検察庁御中)|s\_hirono(非常上告-最高検察庁御中\_ツイッター)
  日時:2021-06-03 16:11/2021/06/03 16:05 URL:
  \url{https://twitter.com/kk\_hirono/status/1400349376703188995} 
  \url{https://twitter.com/s\_hirono/status/1400347796184592390} 
  \textgreater{}
  2018-09-03\_163130_上町の病院から柳田温泉 旧柳田村 四谷というバス停の小屋 柳田大祭のポスター.jpg
  \url{https://t.co/0jaWFijFlH} 
\end{itemize}

〉〉〉 kk\_hironoのリツイート 〉〉〉

\begin{itemize}
\tightlist
\item
  RT
  kk\_hirono(刑事告発・非常上告_金沢地方検察庁御中)|s\_hirono(非常上告-最高検察庁御中\_ツイッター)
  日時:2021-06-03 16:18/2021/06/03 16:15 URL:
  \url{https://twitter.com/kk\_hirono/status/1400351251326324738} 
  \url{https://twitter.com/s\_hirono/status/1400350499736485892} 
  \textgreater{} 2018-07-31\_181410_柳田温泉 到着.jpg
  \url{https://t.co/JB5PZ7LXj0} 
\end{itemize}

〉〉〉 kk\_hironoのリツイート 〉〉〉

\begin{itemize}
\tightlist
\item
  RT
  kk\_hirono(刑事告発・非常上告_金沢地方検察庁御中)|s\_hirono(非常上告-最高検察庁御中\_ツイッター)
  日時:2021-06-03 16:18/2021/06/03 16:15 URL:
  \url{https://twitter.com/kk\_hirono/status/1400351272843182080} 
  \url{https://twitter.com/s\_hirono/status/1400350485572317187} 
  \textgreater{} 2018-07-31\_174731_能登町黒川 集落の川.jpg
  \url{https://t.co/c289lxtaF3} 
\end{itemize}

〉〉〉 kk\_hironoのリツイート 〉〉〉

\begin{itemize}
\tightlist
\item
  RT
  kk\_hirono(刑事告発・非常上告_金沢地方検察庁御中)|s\_hirono(非常上告-最高検察庁御中\_ツイッター)
  日時:2021-06-03 16:18/2021/06/03 16:15 URL:
  \url{https://twitter.com/kk\_hirono/status/1400351287028244482} 
  \url{https://twitter.com/s\_hirono/status/1400350471420727296} 
  \textgreater{} 2018-07-31\_172739_能登町当目 岩井戸神社.jpg
  \url{https://t.co/V37OrJAoy5} 
\end{itemize}

 43枚の写真をアップロードし、そのツイートをリツイートしました。2回の追加をしているので日付の順序が違っているものもあると思います。

 台風で家の窓枠が外れたときの写真を探したのですが、それが2018年9月4日で、意外な発見だったのはその前日の9月3日に柳田の白山神社に行っていたことです。追加した7月31日の岩井戸神社の写真で確認したのですが、初めて柳田白山神社に入ったときの写真でした。

 7月31日の岩井戸神社の帰りに、大きな神社があることに気がついたのですが、小さい子供連れの親子がキャッチボールをしていたので、そのまま立ち寄らずに通過したのです。この柳田白山神社の前は平成14年の秋に、一月か一月半ほど通勤で前を通っていたのですが、気にすることはなかったようです。

 同年(2018年)9月16日には、柳田大祭で柳田白山神社に行ったときの写真がありますが、初めに柳田白山神社に行ってから翌年に行ったものと思っていました。また、初めに柳田白山神社に行ったときは、笹川という集落を経由して初めて通る道路から向かいました。

 柳田大祭のポスターがはられたバス停がありますが、交差点を右折して柳田白山神社に向かっています。この広い道路も新しい発見に思えていたのですが、それからしばらくして、能登町女性遺体遺棄事件の現場が近くだと知って驚きました。テレビニュースで1つだけ周辺の地図が出ていたのです。

\begin{quote}
《引用の始まり》
\end{quote}

\begin{quote}
起訴されるまでは、ほぼ連日のように報道が続いていただけに、ぴたりと報道がなくなってまったく進展がわかりませんでした。昨日は、ミヤネ屋が終わった後の石川県内ニュースだったように思いますが、初公判も求刑も、判決もしらないまま、控訴したということを知りました。

 能登町の旧柳田村に遺体遺棄があり、同じ能登町出身者も一人関与があって逮捕されていたのですが、それ以外にもいろいろと気になり、参考になる事件でした。具体的には共犯関係と幇助になります。

 首謀者や主犯とされたのが29歳の被告で、判決が懲役17年、求刑が懲役20年とのことです。殺害の実行犯は45歳の一人だけのようですが、懲役15年、求刑が懲役18年となっています。共謀共同正犯という言葉はこのニュースで一度も見ていないのですが、そのようです。

 他2人は幇助犯とされていますが、一人は殺害方法を提案、もう一人は被害者を誘い出し指示で死体遺棄に使う車を用意したとあります。前者は懲役10年(求刑も同じ)、後者は懲役6年で求刑は懲役7年とあります。

 石川県では殺人事件自体が少なく年間数件かと思いますが、複数による殺害事件というのは10年ほど前に1件あったぐらいかと思います。白山市の自動車工場か運送会社の車のなかで遺体が発見された事件だったと思います。その事件も大阪の方で共犯者がいたように記憶にあります。
\end{quote}

\begin{quote}
《引用の終わり》
\end{quote}

\begin{itemize}
\tightlist
\item
  8人の弁護士が金沢地方裁判所に集結? 9月3日に裁判員裁判の初公判が始まり、同月18日判決が出ていた金沢片町ガールズバーの女性経営者殺害、能登町死体遺棄事件
  - 告発\金沢地方検察庁\最高検察庁\法務省\石川県警察御中
  \url{https://hirono-hideki.hatenablog.com/entry/2019/10/04/032430} 
\end{itemize}

\begin{quote}
《引用の始まり》
\end{quote}

\begin{quote}
データベースに記録済みのもので笹川をファイル名に含むのは、昨年5月11日で高洲山に行ったときだとわかりました。輪島市の高洲山は奥能登で最も標高の高い山と聞きますが、それでも五百メートル台とのことです。道を間違えて輪島市に入って驚きました。

 笹川のバス停は割と立派で新しく見えるバス停となっていますが、小学生の低学年の頃に、時代劇に出てくるような古い掘っ立て小屋のようなバス停で、いらいらしながらバスが来るのを待っていたという記憶が断片的に残っています。

 結局なぜ小さいときに笹川のバス停にいたのかは謎のままなのですが、一つ考えられたのは、母親に連れられ、丘の上のような場所のいくらか西洋風にもみえた夫婦の家に遊びに行ったことです。テレビで見た満州の風景にも似ていました。

 その夫婦のこともぼんやりとしか記憶にないのですが、私の母親のことをずいぶんと熱烈に歓迎していました。それも戦争時代のドラマで見たような場面だったのですが、引き上げ船で命からがら日本に戻ってきた喜びを噛み締めているようにも見えました。

 笹川には、それと思われる丘のような場所はありませんでした。集落の間際に未開の割と大きな山がありました。あとは川沿いです。

 小高い丘の上で、小さい頃の記憶にある風景に似ていると思ったのは、消防署と公民館がある辺りの上町です。これはだいぶん前から感じていたのですが、バス停の後ろに田んぼが広がる上町のバス停で時間をつぶしたような記憶はまったく残っていません。

 笹川については、もう一つ記憶があって、昭和58年の9月だと思うのですが、川の橋を渡って笹川の集落に入り、同行者の知り合いと聞いた人の家にあがったことです。それも祭りのときだったと思います。被告発人大網健二が在所の祭りを巡って飲み食いし、神主だと言っていた頃のことです。

 柳田村の中心部では、まったく見ず知らずの家にあがりこんで飲み食いをしていました。「あんさまおるかね」の一言で飲み食いが出来るとも彼らは笑っていましたが、私は落ち着きもなく一緒にいるのがたまりませんでした。

 近年になって、祭りに知らない人を招いて飲み食いさせるのも、昔は地域の習慣だったと知ったのですが、当時はそういう知識もなく、非常識なふるまいにしか思えませんでした。祭りに友人知人を招いてごちそうを振る舞うのはいまでも普通のことで、よばれ、と呼ばれています。
\end{quote}

\begin{quote}
《引用の終わり》
\end{quote}

\begin{itemize}
\tightlist
\item
  8人の弁護士が金沢地方裁判所に集結? 9月3日に裁判員裁判の初公判が始まり、同月18日判決が出ていた金沢片町ガールズバーの女性経営者殺害、能登町死体遺棄事件
  - 告発\金沢地方検察庁\最高検察庁\法務省\石川県警察御中
  \url{https://hirono-hideki.hatenablog.com/entry/2019/10/04/032430} 
\end{itemize}

 上記に2箇所の引用しましたが、2つ目に笹川の説明がありました。これは2019年10月4日の記事となっていますが、「データベースに記録済みのもので笹川をファイル名に含むのは、昨年5月11日で高洲山に行ったときだとわかりました。」ありました。

 高洲山に行ったのはまだ1回だけですが、道を間違えて輪島市三井に出て、そのまま輪島市内に向かったのです。

\begin{quote}
《引用の始まり》
\end{quote}

\begin{quote}
私が記憶にある範囲で初めて明るい時間に橋を渡って笹川の集落に入ったのは昨年2018年10月3日だったと確認しました。柳田の白山神社を初めて知ったのも、その年の8月ではなかったかと思います。道を間違えずに当目から来たときですが、キャッチボールの親子がいたので立ち寄りませんでした。

 少なくとも一月以上は間があいていたと思っていたのですが、笹川から通り抜けた広い道路に出て遺体遺棄現場の近くを通ったのは、遺体遺棄事件が起こったのと同じ10月だったようです。

» 山林に女性の遺体遺棄容疑 44歳男逮捕、石川県警 - 産経ニュース
https://t.co/23YxRaGt9x

 上記の記事は配信が2018年10月14日22時46分のようですが、14日に死体遺棄容疑で再逮捕、13日に1人、11日に2人を逮捕していたとあります。能登町の山林に遺体を埋めたのは9日ごろ、とされています。

 繰り返しますが、一月は間があるような感覚でした。同じ10月でも月末に近い頃と考えたのですが、11日に最初の逮捕があったとされています。このニュースを初めて知ったのもテレビだったと思いますが、前年3月の能登高校の女子高生殺害事件の記憶も蘇りました。
\end{quote}

\begin{quote}
《引用の終わり》
\end{quote}

\begin{itemize}
\tightlist
\item
  8人の弁護士が金沢地方裁判所に集結? 9月3日に裁判員裁判の初公判が始まり、同月18日判決が出ていた金沢片町ガールズバーの女性経営者殺害、能登町死体遺棄事件
  - 告発\金沢地方検察庁\最高検察庁\法務省\石川県警察御中
  \url{https://hirono-hideki.hatenablog.com/entry/2019/10/04/032430} 
\end{itemize}

 「私が記憶にある範囲で初めて明るい時間に橋を渡って笹川の集落に入ったのは昨年2018年10月3日だったと確認しました。」とありますが、これは記憶違いで9月3日が最初でした。遺体を埋めたのが10月9日、10月11日に最初の逮捕がテレビのニュースになったようです。

 柳田の能登高校の男子生徒が1月に宇出津新港で行方不明になったのも2018年だとわかったのですが、ナマコひきの船の網に遺体が揚がったと聞き、宇出津新港の公園の前の道路で歩いていた人とその話をしたのですが、それから一月ほどすると家で倒れて死んでいたという話を聞きました。

 台風21号となっていたように思います。夕方のテレビで家の窓が外れると、瞬間的に強い風が家の中に入り込み上に持ち上げられて吹き飛ぶ危険性があるという話を聞いていた矢先に、その家の窓枠が落ちたので慌てたのですが、こういう出来事が続いて、神様の意向のようなものを意識するようになりました。

 「11日に最初の逮捕があったとされています。このニュースを初めて知ったのもテレビだったと思いますが、前年3月の能登高校の女子高生殺害事件の記憶も蘇りました。」と記載もあります。確かに2017年3月10日が殺害事件の発生で、テレビの報道が翌11日の昼前、11時台のニュースになっていたと思います。

 記事の見出しになっている「8人の弁護士が金沢地方裁判所に集結?」という部分は、ページ内検索を行っても該当する箇所を見つけることができませんでした。後回しにすると、そのまま忘れてしまうことがあるようです。

 「首謀被告に懲役17年 金沢地裁判決
他3人は15~6年:北陸発:北陸中日新聞から:中日新聞(CHUNICHI
Web)」という記事がリンク切れになっていました。「車内に死体遺棄容疑、5人逮捕 39歳男を指名手配:朝日新聞デジタル」という記事も同じくリンク切れです。

 8人の弁護士というのがすごく気になるのですが、起訴された後、判決が出るまでほとんど報道を見かけなかったのも弁護士らの活躍がありそうです。

 首謀者で主犯格が懲役17年、殺害の実行犯が懲役15年となっていますが、共謀での計画的な犯行であり、刑が軽いように思われます。主犯格の求刑が懲役20年となっているので、余り変わらない気がしますが、実行犯も求刑が懲役18年で、どちらも3年の違いとなっています。

 被害者のガールズバー経営という30歳の女性が、きつい性格で理不尽な振る舞いをしていたように当初から報道がありましたが、これも金沢西警察署の事件でした。たしか被害者の家族の声というのも報道がなく、後で見かけたような気もするのですが、事実は藪の中なのかと思えるところもありました。

\begin{itemize}
\tightlist
\item
  喜多あき乃さんのガールズバーの店舗を特定か?セレクトショップも経営し、事件現場も判明
  \url{https://t.co/VlMmR2uU14} 
\end{itemize}

 上記のページにJNNというテレビの画面のニュース報道のような画像もありますが、金沢市糸田一丁目と住所を特定しながら被害者の自宅と思われる家の写真や映像がよく出ていたような事件でもありました。連れ去りの現場だったようですが、他の殺人事件には見ないような報道の特異性があった気がします。

 Googleマップで金沢市糸田一丁目が狭いエリアであることを確認しましたが、北陸本線を挟んで向かい合う糸田新町も同じぐらいの広さのようです。そういえば被告発人松平日出男の供述調書で住所が金沢市糸田新となっていたようなことを思い出しました。

 平成3年の7月か8月、市場急配センターの事務所の前で、被告発人松平日出男の息子という若者に会ったのですが、その時、一緒にいたのか、被告訴人多田敏明が同級生とか話し、西南部中学校とも聞いたように思います。

 もっと前に聞いていたようにも思うのですが、被告発人松平日出男の自宅というのは当時の石川郡野々市町に近い、八日市と聞いていたように思います。あるいは押野ですが、そこなら西南部中学校に近そうだと考えていました。

\begin{itemize}
\tightlist
\item
  糸田新町 から 金沢市立西南部中学校 - Google マップ
  \url{https://t.co/HWzpNUrX7C} 
\end{itemize}

 2.5km、2.7km、3.1kmとあるので、それほど離れてはいないような気がしますが、他に近い中学校がありそうには思います。

 「金沢市糸田新町一七番八号」が平成4年4月13日付の被告発人松平日出男の供述調書の住所となっていました。ずっと前に作成したテキストファイルのものですが、やはり4月2日付とかになっていたもう1通の被告発人松平日出男の供述調書が存在しません。

 Googleマップとストリートビューで見ると、普通の住宅が出てきました。玄関あたりは比較的新しく見えるのですが、二階の部分は築30年は経っていそうな気がします。転売されている可能性はありますが、住所に間違いないならば平成4年当時の被告発人松平日出男の自宅だった可能性が高そうです。

\begin{itemize}
\tightlist
\item
  金沢市南斎場 - Google マップ \url{https://t.co/2pBKv2X9nB}  ¥\n 〒921-8043
  石川県金沢市西泉6丁目64
\end{itemize}

 近くに金沢市南斎場という大きな敷地が見えたのですが、住所も金沢市西泉となっています。昨日も思い出していたのですが、北都運輸の市内配達をしているときは、その辺りのカナカンの大きな配送センターのような場所によく行っていて、すぐ近くにセレモニーホールのような建物が見えていました。

 葬式の形式が変わり始めた頃で、セレモニーホールらしい名前になっていたようになっていたように思うのですが、最新型の火葬場と斎場なのだろうと思っていました。平成元年のことです。

 ほかに金沢市内で火葬場や斎場らしい建物をみた憶えはなかったのですが、前に小立野に特別な墓地があるような情報を見たことがあり、最初、処刑場だったのかと思いながら調べ始めたのですが、しばらくして金沢監獄があったという情報を見つけたのです。なお、医学解剖の墓地とは別です。

 医学解剖の墓地の話は、テレビで金沢市天神町の取材をみているときに知りました。天神町は金沢マラソンのコースにもなっていて、そのマラソンの中継というのも同じ頃にテレビでみたような記憶です。

 時刻は6月4日6時50分です。

\begin{itemize}
\tightlist
\item
  TW kk\_hirono(刑事告発・非常上告_金沢地方検察庁御中) 日時:
  2021-06-03 18:18 URL:
  \url{https://twitter.com/kk\_hirono/status/1400381442715123717} 
  \textgreater{}
  医学解剖の墓地の話は、テレビで金沢市天神町の取材をみているときに知りました。天神町は金沢マラソンのコースにもなっていて、そのマラソンの中継というのも同じ頃にテレビでみたような記憶です。
\end{itemize}

 どうも昨日を6月4日と勘違いしていたらしいことに気がついたのですが、夕方にカレーを作るつもりで冷蔵庫の豚肉を見たところ、消費期限が6月3日となっていたので一日過ぎたものと思い込み、ゴミ袋に捨てていました。

 昨日の夕方の18時18分から中断していたことを確認しましたが、その直後の辺りに、自分が大きいと感じるニュースが2つあり、そちらに集中力が移ったまま告発状の作成作業に戻ることがありませんでした。23時過ぎまで起きていたことは憶えています。

 起きたのが5時30分でしたが、またLEDの明るい照明を頭上につけたまま寝ていました。モーターボートの事故で人が死んだと聞いた夢をみていたのですが、その前にもなにか印象的な夢をみていました。事故の大きな音を聞いたというリアルな夢でした。

〉〉〉 kk\_hironoのリツイート 〉〉〉

\begin{itemize}
\tightlist
\item
  RT
  kk\_hirono(刑事告発・非常上告_金沢地方検察庁御中)|hirono\_hideki(奉納\さらば弁護士鉄道・泥棒神社の物語)
  日時:2021-06-04 07:02/2021/06/03 20:02 URL:
  \url{https://twitter.com/kk\_hirono/status/1400573739117334528} 
  \url{https://twitter.com/hirono\_hideki/status/1400407637653475329} 
  \textgreater{}
  女逮捕・・・鋭い洞察力のドコモショップ深谷店員、女の不正を見抜く 警官到着まで女引き留め、県警が感謝状
  \url{https://t.co/QwkBt8kq7q} 
  (右から)大久保昇署長、ドコモショップ深谷店の橋本祐樹さん、上遠野伸也さん=15日午後、深谷署(同署提供)
\end{itemize}

〉〉〉 kk\_hironoのリツイート 〉〉〉

\begin{itemize}
\tightlist
\item
  RT
  kk\_hirono(刑事告発・非常上告_金沢地方検察庁御中)|hirono\_hideki(奉納\さらば弁護士鉄道・泥棒神社の物語)
  日時:2021-06-04 07:03/2021/06/03 20:08 URL:
  \url{https://twitter.com/kk\_hirono/status/1400573794939326469} 
  \url{https://twitter.com/hirono\_hideki/status/1400409113188724737} 
  \textgreater{} 田中敬 顔画像 埼玉県警深谷署署長が窃盗で逮捕
  依願退職ってマジ \url{https://t.co/Co9lQwSDs1} 
  埼玉県警深谷署の所長が商業施設でトイレットペーパーを盗み書類送検されることが分かりました
  書類送検されるのは田中敬署長です
\end{itemize}

〉〉〉 kk\_hironoのリツイート 〉〉〉

\begin{itemize}
\tightlist
\item
  RT
  kk\_hirono(刑事告発・非常上告_金沢地方検察庁御中)|hirono\_hideki(奉納\さらば弁護士鉄道・泥棒神社の物語)
  日時:2021-06-04 07:03/2021/06/03 20:17 URL:
  \url{https://twitter.com/kk\_hirono/status/1400573929001943040} 
  \url{https://twitter.com/hirono\_hideki/status/1400411242208972805} 
  \textgreater{}
  『桶川ストーカー殺人事件』不祥事を起こした警察と上尾署のその後
  \textbar{} 女性のライフスタイルに関する情報メディア
  \url{https://t.co/TpJWsz7zOf}  2021年03月30日公開 2021年03月30日更新
  雑学(547)
\end{itemize}

〉〉〉 kk\_hironoのリツイート 〉〉〉

\begin{itemize}
\tightlist
\item
  RT
  kk\_hirono(刑事告発・非常上告_金沢地方検察庁御中)|hirono\_hideki(奉納\さらば弁護士鉄道・泥棒神社の物語)
  日時:2021-06-04 07:04/2021/06/03 20:27 URL:
  \url{https://twitter.com/kk\_hirono/status/1400574011298312195} 
  \url{https://twitter.com/hirono\_hideki/status/1400413750071496705} 
  \textgreater{} 小学校のいじめで千葉市と同級生の親に賠償命令 東京高裁
  \textbar{} 教育 \textbar{} NHKニュース \url{https://t.co/UyFuyEl6ZK} 
  2021年6月3日 16時48分
\end{itemize}

〉〉〉 kk\_hironoのリツイート 〉〉〉

\begin{itemize}
\tightlist
\item
  RT
  kk\_hirono(刑事告発・非常上告_金沢地方検察庁御中)|hirono\_hideki(奉納\さらば弁護士鉄道・泥棒神社の物語)
  日時:2021-06-04 07:04/2021/06/03 20:48 URL:
  \url{https://twitter.com/kk\_hirono/status/1400574134711570433} 
  \url{https://twitter.com/hirono\_hideki/status/1400419071934406658} 
  \textgreater{} 笹川刑事部長 - Twitter検索 / Twitter
  \url{https://t.co/Nqo9Baw36G} 
   俳優の名前が全くわからないが、数年前、娘の結婚相手と一緒に風呂に入るCMに出ていた人物と思い出した。たぶんそれ依頼。結婚相手は当時の大河ドラマの主役だったかも。黒田官兵衛とか。かなり前。
\end{itemize}

〉〉〉 kk\_hironoのリツイート 〉〉〉

\begin{itemize}
\tightlist
\item
  RT
  kk\_hirono(刑事告発・非常上告_金沢地方検察庁御中)|hirono\_hideki(奉納\さらば弁護士鉄道・泥棒神社の物語)
  日時:2021-06-04 07:04/2021/06/03 20:52 URL:
  \url{https://twitter.com/kk\_hirono/status/1400574153573294080} 
  \url{https://twitter.com/hirono\_hideki/status/1400420167264346114} 
  \textgreater{} 本田博太郎 - Wikiwand \url{https://t.co/ohFrcIIAW8} 
  1981年には工藤栄一に認められ、『必殺仕舞人』『新・必殺仕舞人』に主演メンバーとして参加し、代表作となる。
\end{itemize}

〉〉〉 kk\_hironoのリツイート 〉〉〉

\begin{itemize}
\tightlist
\item
  RT
  kk\_hirono(刑事告発・非常上告_金沢地方検察庁御中)|hirono\_hideki(奉納\さらば弁護士鉄道・泥棒神社の物語)
  日時:2021-06-04 07:04/2021/06/03 20:53 URL:
  \url{https://twitter.com/kk\_hirono/status/1400574206970978314} 
  \url{https://twitter.com/hirono\_hideki/status/1400420496877842438} 
  \textgreater{} 能登の花ヨメ - Wikiwand \url{https://t.co/1tqeWFKLQe} 
  映画の企画は2004年にスタートしたが、2007年3月25日の能登半島地震で脚本を根本的に見直し、震災復興に立ち向かう能登の人々の姿を大幅にとり入れた。
\end{itemize}

〉〉〉 kk\_hironoのリツイート 〉〉〉

\begin{itemize}
\tightlist
\item
  RT
  kk\_hirono(刑事告発・非常上告_金沢地方検察庁御中)|hirono\_hideki(奉納\さらば弁護士鉄道・泥棒神社の物語)
  日時:2021-06-04 07:05/2021/06/03 21:23 URL:
  \url{https://twitter.com/kk\_hirono/status/1400574345869631488} 
  \url{https://twitter.com/hirono\_hideki/status/1400427937124405251} 
  \textgreater{} ★☆菊川怜(28)女囚役に初挑戦♪ \textbar{}
  ★独身よろず屋ドットコム★(●^o^●) - 楽天ブログ
  \url{https://t.co/EjYm7NYwpe} 
\end{itemize}

〉〉〉 kk\_hironoのリツイート 〉〉〉

\begin{itemize}
\tightlist
\item
  RT
  kk\_hirono(刑事告発・非常上告_金沢地方検察庁御中)|bengo4topics(弁護士ドットコムニュース)
  日時:2021-06-04 07:05/2021/06/03 19:02 URL:
  \url{https://twitter.com/kk\_hirono/status/1400574355277418497} 
  \url{https://twitter.com/bengo4topics/status/1400392559445221385} 
  \textgreater{}
  立川市のホテルで起きた死傷事件で、亡くなった被害者を実名で報じた新聞社とテレビ局があります。女性は風俗店で働いていたとされ、「被害者の人権は無視か」と疑問を持つ読み手も少なくありません。各紙・各局はどのように報じたのでしょう。
  \url{https://t.co/byae9PYkcN} 
\end{itemize}

〉〉〉 kk\_hironoのリツイート 〉〉〉

\begin{itemize}
\tightlist
\item
  RT
  kk\_hirono(刑事告発・非常上告_金沢地方検察庁御中)|hirono\_hideki(奉納\さらば弁護士鉄道・泥棒神社の物語)
  日時:2021-06-04 07:05/2021/06/03 21:51 URL:
  \url{https://twitter.com/kk\_hirono/status/1400574383375060997} 
  \url{https://twitter.com/hirono\_hideki/status/1400434915276050435} 
  \textgreater{}
  刺された女性、ネット速報で「実名」、紙で「匿名」にする新聞も 立川死傷事件、報道をチェック
  - 弁護士ドットコム \url{https://t.co/GF15TqNkBZ}  2021年06月03日 19時00分
\end{itemize}

 変な夢を見ていたようなニュースだったのですが、記憶の通りの経過でツイートがありました。警察幹部ということで名前を探したのですが、見つけたのが前任の署長だったらしく、別に実名報道されたニュースを見つけたことで、別人であったことがわかりました。

 名前を知らなかった俳優のことですが、刑事部長が主役らしいテレビドラマで、それもずいぶんと奇抜な現実離れした役柄と演技という様子でした。次に見つけたのが風俗の仕事をしていた被害者が実名で被疑者が19歳の少年で匿名報道という論議ですが、きっかけは刑裁サイ太のタイムラインでした。

 警察の刑事部長というのは余り聞いた覚えがないのですが、よくみる刑事課長の上司にあたることは間違いなさそうです。確か福井刑務所では、刑事部長と呼ばれた人が、事実上現場を取り仕切るトップで、署長というのは外来の接待役や広報が事実上の仕事内容と聞いたことがありました。

 その刑事部長も所長もよく見かける機会があったのが福井刑務所での生活でした。金沢刑務所でも所長を見ることはありましたが機会は極端に少なくずいぶんとよそよそしく距離を置いていると感じられるところがありました。

 受刑者に向き合う話などなかったと思います。かたちだけいることがあったという感じです。福井刑務所の所長は温厚そうな紳士にみえましたが、金沢刑務所の所長は違っていました。

 ずっと前にも書いていると思いますが、金沢刑務所の場合、組織図が不明で、所長に次ぐナンバー2と目されたのが、第三統括と呼ばれた人でした。まるで昔の戦争映画に出てくる軍人のような雰囲気でしたが、もともと軍隊のような雰囲気が全体にあって、駐屯地のように思えることもありました。

 映画で見る刑務所の雰囲気とはずいぶん違っていましたが、金沢刑務所では侵入教育のとき、暴動が起きれば躊躇わず銃を使うと言われていました。首席矯正処遇官だったと思います。ちょうど平成5年の頃に、願箋に書く名称が変更になったのですが、その前は保安課長でした。

 願箋には「処遇首席殿」と書くように指示されていたように思います。ほとんどの願箋が「処遇首席殿」になっていたと思いますが、大型封筒の購入などが「会計課長殿」になっていました。願箋の標題が「大型封筒使用願い」だったかもしれません。

 トイレットペーパーを5個持ち出して、自主退職となり窃盗容疑で書類送検されるという埼玉県警深谷警察署長ですが、本人にしても悪夢のような出来事だったに違いなく、家族や親族もしばらくは放心状態で現実を受け入れがたいのではと想像します。

 埼玉県の深谷市というのも国道17号線沿いだったように思います。京都の九条葱と並んで全国的に有名なネギの産地として知名度がありそうです。地図で確認はしていませんが、桶川市にも近く、桶川ストーカー殺人事件では、桶川市以外の警察署名となっていたことが記憶にありました。

 もしやと思ったのですが、調べて確認すると上尾警察署でした。前に上尾警察署をテレビニュースで見ることがあったのですが、新潟市内での女性殺害事件で、直前に自殺か殺害を仄めかす話があって母親が上尾警察署に相談していたというような内容のニュースとして、私の記憶には残っています。

 若い被疑者で逮捕後にはネットでイケメンとも言われていましたが、逮捕時の様子もかなり異様なものでした。一週間ほど前になるのか、拘置所か刑務所で死亡したというネット記事のニュースを見かけていました。裁判は確定していないような話もあったので服役ではなさそうです。

\begin{itemize}
\item
  斎藤涼介容疑者がイケメンすぎるとネット上で話題に・・・新潟女性殺害で全国指名手配中!
  - 韓information. \url{https://t.co/UOfOhsyFH9} 
\item
  【斎藤涼介!】イケメン殺人犯(26)、刑務所で病死する
  \url{https://t.co/cVUMlSaNrp} 
\end{itemize}

 事件が2019年11月15日夜で、まだ公判が始まっておらず、公訴棄却となる見通しとあります。正気を疑う連行の写真がありますが、明確に精神異常を指摘したニュースは見ていなかったかもしれません。検察の方で精神鑑定をやりそうな事件と思いますが、それも情報を見ていないように思います。

\begin{itemize}
\item
  斎藤涼介 精神鑑定 - Google 検索 \url{https://t.co/2k1OYuQUwC} 
\item
  斎藤涼介容疑者を逮捕。精神鑑定の可能性やネットでの出会いの危険性を考察。
  \textbar{} ビズキャリonline \url{https://t.co/SXSaCTdsJj} 
\item
  新潟女性刺殺、勾留中の被告病死 \textbar{} Reuters
  \url{https://t.co/IOt3nO01ib} 
  刑務所によると、斎藤被告の死因は急性腎不全。今月3日午後0時半ごろ、単独室の布団であおむけに寝たまま、呼吸が停止していた。
\end{itemize}

 金沢地検でも起訴前の精神鑑定というニュースを見たように思います。

\begin{itemize}
\item
  起訴前本鑑定とは?流れや鑑定留置について弁護士が解説 \textbar{}
  逮捕・示談に強い東京の刑事事件弁護士 \url{https://t.co/sZn9Z9S1al}  ¥\n
  簡易鑑定は、通常、検察の庁舎内で1回30分~1時間程度しか行われません。これに対して、起訴前本鑑定は、被疑者を2,3か月にわたって病院や拘置所に留置し、医師が継続
\item
  急性腎障害(急性腎不全)とは(症状・原因・治療など)|ドクターズ・ファイル
  \url{https://t.co/AWGK1wfr8k}  ¥\n
  尿量が1日当たり約500ml以下まで減少したり(ほとんどの健康な成人の尿量は1日当たり約750ml)、完全に尿が出なくなったりする場合がある。
\end{itemize}

 金沢大学病院の精神鑑定で精神科閉鎖病棟にいるとき、右翼団体が原因で精神がおかしくなり、本人はうつ病と診断されたと話していたように思いますが、自分で排尿ができず、看護婦の若い女性に毎回、管で出してもらっているという人がいました。金沢の運送会社でしたが、七尾の人でした。

 もう一人、自衛隊のいじめが原因で入院したという若者がいましたが、顔の表情が乏しくいつもうつろな目をしていて、そちらの方が症状は重そうに見えていました。七尾の人は気さくで明るい感じでしたが、重い症状が出ることがあるとも話していたように思います。

 完全に顔から表情が消えた若者が特に印象的でしたが、人と会話をするような素振りは全く感じられず、食事のときだけ姿を見ていたように思います。

 もう一人は躁鬱病というのか、精神分裂病と診断されたと聞いたかもしれないですが、陽気なときと落ち込むときが極端で、恐怖で逃げ回る発作のような症状を起こしていました。ときどき母親と父親が面会に来ていましたが、母親はその病気を受け入れていないように感じられました。いじめが原因とも。

 私が受けた精神鑑定は一月で、平成5年3月1日から3月31日でした。金沢大学前だったのかバスターミナルのような場所に私服の刑務官としばらく一緒にいたことを不意に思い出しました。古い写真を一枚見つけたような感覚ですが、ずっと眠っていた記憶のようです。

\begin{itemize}
\tightlist
\item
  逮捕の少年「人殺す動画見て刺激」 立川の男女殺傷:朝日新聞デジタル
  \url{https://t.co/q4XLLsFZrA}  母親によると、少年は小学校・・・ ¥\n
  この記事は会員記事です。無料会員になると月5本までお読みいただけます。
  ¥\n ¥\n 残り:257文字/全文:911文字
\end{itemize}

 被疑者の19歳少年の母親のインタビューがニュース記事になっていましたが、今調べたところ、仕事にでかけたものと思っていたという、ものしか見当たらず、採用を断る連絡があったというものがあったと思うのです。その前の日とかに、仕事が決まったと話していたとも。

\begin{quote}
《引用の始まり》
\end{quote}

\begin{quote}
少年の母親:「きのう(事件当日)は仕事に行くと言って朝8時すぎに出掛けて行きました。夜になって帰宅しないため、心配して電話やLINEをしてもつながらないし、既読にもなりませんでした。主人の電話には出ましたが、『何もしゃべらずに切られた』と話していました。心配していた矢先に刑事さんが家に訪ねてきました。暴力的というのは全くありませんでした。事件のことは本当にショックで何とも言えません。[テレ朝news]
https://news.tv-asahi.co.jp
\end{quote}

\begin{quote}
《引用の終わり》
\end{quote}

\begin{itemize}
\item
  \begin{enumerate}
  \def\labelenumi{(\arabic{enumi})}
  \tightlist
  \item
    ホテルで男女殺傷 逮捕された19歳少年の母親は・・・(2021年6月2日) -
    YouTube \url{https://www.youtube.com/watch?v=yf4qQc-KcNs} 
  \end{enumerate}
\item
  逮捕の少年、事件当日「仕事に行ってくる」と噓 感情の起伏激しく暴言も 立川ホテル殺傷
  - 産経ニュース \url{https://t.co/q7VzBQVEQk} 
\end{itemize}

 しばらく探したのですが、昨夜見たはずの、仕事が決まり、次に不採用の連絡があった、という話が見当たりません。

 一度就職が決まってから不採用の連絡があるというのも珍しいと思うのですが、自暴自棄になる原因になったような可能性はありそうです。私も平成9年に、金沢市内の運送会社の面接に行き、指定された時間に採否の連絡がなかったのですが、こちらから電話をすると、ずいぶん迷惑そうな対応を受けました。

 運送会社の名前が思い出せないですが、佐川急便の仕事をする会社でした。埼玉県との定期便と聞いていたように思います。

 平成10年11月20日頃に私は市場急配センターの事務所で、被告発人松平日出男と現在の市場急配センターの社長の可能性の高い堂野さんとの二人と会って話をしているのですが、私の質問に被告発人松平日出男が、慣れた手付きで、覚醒剤の注射器を腕に打つ仕草を見せました。

 細かく正確なことは今の記憶で思い出せないですが、他の運送会社から面接採用のことを問い合わせを受けたことはなかったのかという趣旨の私の質問で、被告発人松平日出男は自信あり気な態度になって、一度だけ、伝えたことがあったといい、上記の動作で覚醒剤の乱用者を示唆したのです。

\begin{itemize}
\tightlist
\item
  本社・営業所案内 \url{https://t.co/YbMbQwHouh} 
\end{itemize}

 ページタイトルに会社名がなく、「本社・営業所案内」となっていますが、会社案内のページには「大東実業株式会社 代表取締役 石上恒明」とあります。「大東ブループ」という名称も見えます。

\begin{itemize}
\tightlist
\item
  沿革 \url{https://t.co/hFvEylOzCJ}  ¥\n
  平成7年大阪府東大阪市に「大阪営業所」を新設。 ¥\n
  石川県金沢市に「金沢営業所」を新設。
\end{itemize}

 平成7年に金沢営業所を新設、とありますが、昭和60年当時にも金沢の佐川急便でトラックを見ていたように思います。佐川急便の専属という印象でしたが、他の多くの運送会社とは違い、佐川カラーにはなっていなかったと思います。

 昭和60年当時の話ですが、佐川急便では長距離輸送の全てを他の運送会社に任せていると聞き、少なくともその半分以上が佐川カラーの大型車になっていたと思います。小さな文字で運送会社名が分かりづらいこともあったのですが、わかりやすく箱に大きな文字があったのが九州運送でした。

 今も名前はそのまま続く佐川急便ですが、昭和の時代の佐川急便というのは特別な存在感がつとに知られていました。

 昭和59年12月から昭和61年2月一杯、昭和62年1月から3月ぐらいに、私が中西運輸商にいたときの仕事のメインが、その佐川急便の広島・九州便でしたが、親会社は東広島市にある西日本運輸興業でした。上の名前は今もなんとなく憶えていますが、金沢のホームには駐在員のような人がいつもいました。

 話し言葉から地元石川県の人だったと思います。控えめで真面目に仕事に取り組む姿が印象に残っていますが、他の運転手とは一線を引いた感じで余り打ち解けた話、個人的な話はしていなかったように思います。直接指示を受けることもなかったと思います。

 正確に言えば、運転手は中西水産輸送の社員で、配車をする中西運輸商の下請けという体裁になっていたようです。税金対策とも聞きました。どちらも有限会社になっていたと思います。

 数年前までは大洋マルハ、の冷凍魚を中心に運んでいたと聞きましたが、白ナンバーで摘発を受け、そのあと窮地を救ったのが佐川急便のしごとをする西日本運輸興業と聞いていました。なぜ東広島市と金沢なのかという接点に疑問はありましたが、あまり気にかける運転手はいなかったように思います。

 ちょっと大洋マルハの正式な書き方がわからないですが、戦後のどさくさでマグロ船に身寄りのない人を乗せて遠洋で海に突き落とし、保険金でもうけて大きくなった会社だと中西運輸商では聞いていました。こういう話は他にもありそうですが、もちろん真偽は不明です。

\begin{itemize}
\tightlist
\item
  マルハ - Wikipedia \url{https://t.co/1uGCOHqMrH} 
  かつて日本に存在した水産加工食品を製造販売する会社である。旧社名は大洋漁業株式会社(たいようぎょぎょう)。法人格としては持株会社化やニチロとの経営統合、グループ企業との吸収合併を経て現在の「マルハニチロ」として継続している。
\end{itemize}

 マルハニチロになったという話は知っていましたが、日魯という水産会社があったと思います。記憶にあるのは守田水産輸送が青森から九州に運んでいるというホタテで、その発泡スチロールのタンクの魚箱に、日魯というテープのようなシールが貼られていたように思います。

 昭和60年当時は、金沢の運送会社でも運転手に生命保険を掛け、事故死で大金が入るのを期待するという話は実際に聞いていました。

 中西運輸商の場合、事故が多すぎて、のちに佐川急便に切られることにもなるのですが、実際に運転手に保険を掛けていたとか、保険金を受け取ったという話は聞いたことがありませんでした。

 中西運輸商のの社長の夫人が、4t車を運転しながらだったと思いますが、2つの運送会社の名前をあげ、そこの社長の夫人が運転手が死ねば儲かって嬉しいと話していたが、自分はそうは思わないと、真面目な話をしていました。金石街道の走行中だったと思います。

 令和3年3月31日付告発状には書いたように思いますが、中西運輸商は金石街道沿いで昭和60年当時は路上駐車する他なく、事務所にいることが少なかったのですが、社長の夫人の姿を見ることも稀だったと思います。

 一度だけなにかの機会に、社長の夫人と二人で用事にでかけ、帰りに二人で4t車に乗っていたように思います。けっこう大きな箱の4t車でファミーだったと思います。字が薄くなっていたと思いますが、箱に潰れたスーパーのような名前がありました。

 他によく憶えているのは、昭和59年の12月で年末の忘年会でしたが、中西運輸商の社長の自宅でした。会社からは離れていましたが歩いてでもいける距離で畝田だったと思います。金石街道から少し入ったところで、ずいぶん古い、昔のドラマに出てくるような部屋が印象に残っています。

 隣辺りに、当時親しくしていた友人の彼女の家があったらしいのですが、その彼女というのも気性が激しく、姉妹と弟がいて、姉妹でおかずの奪い合いをして、弟の手にフォークを突き刺したという話を聞きました。私は被告発人大網健二と彼女が実際に包丁を持ち出した現場にもいたことがあります。

 名前も顔も余り憶えていませんが、その友人の彼女のことが、被害者安藤文さんに対する認識を歪めた影響はけっこう大きかったと思います。また、包丁のことがあったしばらく後に、その友人がガス自殺の未遂をやって、入院しているのを見舞いに行ったのが金沢大学病院でした。

 金沢大学病院にガス自殺未遂をした友人を見舞いに行ったのは、昭和60年2月だったと思います。この月をよく憶えているのは、北海道の札幌市にアルミサッシを運ぶ仕事から戻った直後で、海が時化て避難していたということもあり、日本海は2月が一番荒れるとも遠洋漁師の友人に話を聞いていました。

 平成4年4月1日の傷害・準強姦被告事件当日、最初に走行中の車内で、助手席に座る被害者安藤文さんの顔面を左手の甲で殴ったのも、今思えば、中西運輸商の社長の家の近くの金石街道で、もう少し手前で、松村の交差点を過ぎた辺りかもしれません。その交差点の角には加州銀行がありました。

 初めて銀行のカードを作ったのもその加州銀行なのですが、昭和60年になると思います。当時も同じ言葉があったのか憶えていませんが、今で言う銀行のATMで、その出始めだったように思います。加州銀行はその後、石川銀行となり、刑事事件にもなって消えていきました。

\begin{itemize}
\item
  石川銀行 敦賀 弁護士 - Google 検索 \url{https://t.co/icoS2PXAKb} 
\item
  【石川】依頼者の意向を無視して裁判続け報酬1億3000万円 業務停止に|士業広告のパイオニア、株式会社L-net
  \url{https://t.co/l1iMPSJVJe}  ¥\n
  金沢弁護士会は22日、訴えを取り下げようとした依頼者の意向に反して裁判を続け、高額な報酬を受け取ったなどとして、同会所属の敦賀彰一弁護士(63)を
\end{itemize}

 新聞で最後に見たと記憶にある石川銀行関連のニュースですが、石川銀行が経営破綻するまでは他にもいろいろとニュースがあり、それが珠洲原発の反対運動のニュースと、協奏曲のように響き合っていたような印象が残っています。最後の幕となった敦賀彰一弁護士もその後、亡くなったと聞きます。

 新聞で見たという印象が強く残っている「意向無視し裁判続け報酬1億3000万円 2弁護士に業務停止2カ月」というニュースですが、2013年3月22日となっています。

\begin{itemize}
\item ~
  \hypertarget{ux5927ux5d0eux4e8bux4ef6ux3068ux4fddux967aux91d1ux76eeux7684ux6bbaux4ebaux662dux548c54ux5e74ux5f53ux6642ux306eux6642ux4ee3ux80ccux666f---ux544aux767aux91d1ux6ca2ux5730ux65b9ux691cux5bdfux5e81ux6700ux9ad8ux691cux5bdfux5e81ux6cd5ux52d9ux7701ux77f3ux5dddux770cux8b66ux5bdfux5fa1ux4e2d2020-httpst.cou9e6203fy5-ux6566ux8cc0ux5f70ux4e00ux5f01ux8b77ux58ebuxff12uxff10uxff11uxff15ux5e74uxff12ux6708ux306buxff16uxff14ux6b73ux3067ux6b7bux53bbux3068ux3068ux3082ux306bux539fux544aux4ee3ux7406ux4ebaux3092ux52d9ux3081ux305fux6d45ux91ceux96c5ux5e78ux5f01ux8b77ux58ebux306bux3088ux308bux3068ux3068ux3042ux308aux307eux3059ux304cux91d1ux6ca2ux5927ux5b66ux9644ux5c5eux75c5ux9662ux306eux5199ux771fux304c}{%
  \paragraph{大崎事件と保険金目的殺人,昭和54年当時の時代背景 -
  告発\金沢地方検察庁\最高検察庁\法務省\石川県警察御中2020
  \url{https://t.co/u9e6203Fy5} 
  「敦賀彰一弁護士(2015年2月に64歳で死去)とともに原告代理人を務めた浅野雅幸弁護士によると、」とありますが,金沢大学附属病院の写真が}\label{ux5927ux5d0eux4e8bux4ef6ux3068ux4fddux967aux91d1ux76eeux7684ux6bbaux4ebaux662dux548c54ux5e74ux5f53ux6642ux306eux6642ux4ee3ux80ccux666f---ux544aux767aux91d1ux6ca2ux5730ux65b9ux691cux5bdfux5e81ux6700ux9ad8ux691cux5bdfux5e81ux6cd5ux52d9ux7701ux77f3ux5dddux770cux8b66ux5bdfux5fa1ux4e2d2020-httpst.cou9e6203fy5-ux6566ux8cc0ux5f70ux4e00ux5f01ux8b77ux58ebuxff12uxff10uxff11uxff15ux5e74uxff12ux6708ux306buxff16uxff14ux6b73ux3067ux6b7bux53bbux3068ux3068ux3082ux306bux539fux544aux4ee3ux7406ux4ebaux3092ux52d9ux3081ux305fux6d45ux91ceux96c5ux5e78ux5f01ux8b77ux58ebux306bux3088ux308bux3068ux3068ux3042ux308aux307eux3059ux304cux91d1ux6ca2ux5927ux5b66ux9644ux5c5eux75c5ux9662ux306eux5199ux771fux304c}}
\end{itemize}

 軽く目を通しましたが、他にもいろいろ記述してありました。最初に敦賀彰一弁護士の死去を何で知ったのか思い出さないのですが、2015年2月に64歳で死去ということはこれで確認できました。この弁護士は川県貨物自動車運送とも関連があったようですが、記憶にはありませんでした。

\begin{itemize}
\tightlist
\item
  石川銀行 - Wikipedia \url{https://t.co/gQvKXNBOre}  2007年7月24日 -
  高木元頭取に対し、検察側は「銀行を犠牲にして自己保身を図り、多大な損失を負わせた」として懲役五年を求刑したと報じられる{[}7{]}。
\end{itemize}

 刑事事件になっていたのかも心もとない記憶だったのですが、特別背任罪というのは刑の上限も高くはなかったと思うのですが、懲役5年の求刑というのは、よほど被害額が大きく悪質とみられた事件だったようです。なんだかまったく理解のできない事件だと思っていました。

\begin{itemize}
\item
  北國銀行背任事件 - Wikipedia \url{https://t.co/zqVVJl5gqG}  ¥\n
  1997年10月に名古屋地検特捜部は、これをめぐり当時の北國銀行頭取と協会役員3人を背任行為をしたとして逮捕した。
\item
  北國銀行背任事件 - Wikipedia \url{https://t.co/zqVVJl5gqG} 
  2005年10月28日、差し戻し控訴審となる名古屋高等裁判所は無罪判決を言い渡した。裁判長は「経済取引上許される行為」とした。
\end{itemize}

 割と最近に、この北國銀行背任事件で無罪判決というのを、他の検索の途中で見かけていたように思うのですが、この刑事裁判で、石川銀行の事件も刑事裁判になったものと混同しているのかとも先程まで考えていました。関心のないニュースでしたが、新聞を見るたびに見かけたような気はします。

 考えてみると平成4年の秋の時点で、被告発人岡田進弁護士は、この北國銀行の顧問弁護士だったのですが、それというのも若い行員の横領事件で500万円程度だったとも思いますが、新聞の記事となっていたのを、私は金沢刑務所の拘置所で読んでいました。

 新聞の購読は2週間単位ぐらいで延長をしていたと思いますが、北國新聞と読売新聞の二紙で、1,2回だけ読売新聞を購読したことがあったと思います。

\begin{quote}
《引用の始まり》
\end{quote}

\begin{quote}
今日は挨拶周りからスタート。朝9時30分に出発して、戻ってきたのが5時15分でした。どこでも話題はコロナの話しかでません。特に金沢弁護士会で3月30日、3月31日とコロナ関係の法律相談を行ったので、関心はどのような相談が多かったのかということでした。守秘義務があるので詳細は話せませんが、資金繰りに困った事業者からの相談や、雇止めされそうな労働者からの相談が多かったとのことです。
\end{quote}

\begin{quote}
《引用の終わり》
\end{quote}

\begin{itemize}
\tightlist
\item
  事業主のためのコロナウィルス関係の法律相談 \textbar{}
  弁護士樋詰哲朗のブログ \url{https://hizumelaw.com/?p=375} 
\end{itemize}

 「2020年度金沢弁護士会の副会長に就任しました」という見出しのリンクを開いた記事ですが、ページタイトルが「事業主のためのコロナウィルス関係の法律相談」となっていました。

 2020年4月2日付けの記事で、「本日から金沢弁護士会の副会長に就任しました。」という書き出しです。cal
4
2020というコマンドを実行したところ、2020年4月1日は水曜日となっていました。傷害・準強姦被告事件の平成4年4月1日も水曜日だったのです。これは忘れることがなかったはずなのです。

 2020年4月2日付けの記事で、「本日から金沢弁護士会の副会長に就任しました。」という書き出しです。cal
4
2020というコマンドを実行したところ、2020年4月1日は水曜日となっていました。傷害・準強姦被告事件の平成4年4月1日も水曜日だったのです。これは忘れることがなかったはずなのです。

 東日本大震災ならば3月11日ですが、私の場合はなにより4月1日が特別な日になり、思い出すことも思い浮かべることも多かったと思います。

 2019年4月1日は金沢にいたのでよく憶えていますが、令和という新年号が発表された日でもありました。金沢市福久のイオンモールで、芝寿しのとり弁当を買って帰ったのですが、平成3年当時に、市場急配センターの近くのジャスコ若宮店のものをよく買って食べたという思い出の弁当でした。

〉〉〉 kk\_hironoのリツイート 〉〉〉

\begin{itemize}
\tightlist
\item
  RT
  kk\_hirono(刑事告発・非常上告_金沢地方検察庁御中)|s\_hirono(非常上告-最高検察庁御中\_ツイッター)
  日時:2021-06-04 10:55/2021/06/04 10:54 URL:
  \url{https://twitter.com/kk\_hirono/status/1400632265265143810} 
  \url{https://twitter.com/s\_hirono/status/1400632114148642817} 
  \textgreater{} 2019-04-01\_194556_芝寿し とり弁当.jpg
  \url{https://t.co/Gf8vaNkO21} 
\end{itemize}

 撮影した写真でわかったのですが、四角い形の弁当で、写り具合のためかずいぶん真四角に近い縦横比に見えます。平成3年当時の芝寿しの弁当は、他にシャケ弁当があったのですが、どちらも角が丸くなった細長いかたちの弁当だったと思います。

〉〉〉 kk\_hironoのリツイート 〉〉〉

\begin{itemize}
\tightlist
\item
  RT
  kk\_hirono(刑事告発・非常上告_金沢地方検察庁御中)|s\_hirono(非常上告-最高検察庁御中\_ツイッター)
  日時:2021-06-04 11:02/2021/06/04 11:02 URL:
  \url{https://twitter.com/kk\_hirono/status/1400634126475030534} 
  \url{https://twitter.com/s\_hirono/status/1400634015548248069} 
  \textgreater{} 2020-04-01\_124852_.jpg \url{https://t.co/sM3wQD42Sr} 
\end{itemize}

〉〉〉 kk\_hironoのリツイート 〉〉〉

\begin{itemize}
\tightlist
\item
  RT
  kk\_hirono(刑事告発・非常上告_金沢地方検察庁御中)|s\_hirono(非常上告-最高検察庁御中\_ツイッター)
  日時:2021-06-04 11:02/2021/06/04 11:02 URL:
  \url{https://twitter.com/kk\_hirono/status/1400634147496873987} 
  \url{https://twitter.com/s\_hirono/status/1400633949601230848} 
  \textgreater{} 2020-04-01\_124850_.jpg \url{https://t.co/i5xNuYEk85} 
\end{itemize}

 2020年4月1日は、同じようなアジの刺し身の写真が2枚あるだけでした。自分で海から釣ってきた以外のアジを刺し身にしたことはなく、ここ数年は年に2,3回ぐらいしか刺し身を自分で作ることはないはずですが、一日にその2枚だけというのも珍しく思いました。何をしていたのかもわかりません。

 他の写真を確認すると前日の2020年3月31日は、宇出津新港の職業安定所の横で撮影した写真が1枚あるだけで、小木港東一文字堤防でアジ釣りをした写真は3月30日でした。近年は釣った翌日でも刺し身にするようになったのですが、明後日に刺し身にして食べたという記憶はありませんでした。

 その3月30日は12時22分に新聞を撮影した写真があって、「呼吸器事件あす再審判決 大津地裁」とありました。これはガソリンスタンドであった写真をテーブルの上で撮影したものと思います。この日は、大崎事件の再審請求もあったはずです。

 時刻は14時31分です。今日は朝から雨が降り続いている様子で、雨が多いと聞いていた昔の金沢のことなども思い出すのですが、また予想外の方向に発見があったので、樋詰哲朗弁護士(金沢弁護士会)のことはひとまず切り上げ、金沢西警察署にフォーカスを移します。

\begin{itemize}
\tightlist
\item
  〈〈〈 2021/06/04 14:34:42 Linux Emacs: 〈〈〈
\end{itemize}

\hypertarget{ux5f01ux8b77ux58ebux9244ux9053ux3068ux91d1ux6ca2ux5f01ux8b77ux58ebux4f1a}{%
\subsubsection{弁護士鉄道と金沢弁護士会}\label{ux5f01ux8b77ux58ebux9244ux9053ux3068ux91d1ux6ca2ux5f01ux8b77ux58ebux4f1a}}

\hypertarget{ux9ce5ux53d6ux304cux821eux53f0ux306eux5f01ux8b77ux58ebux5287ux5834ux7121ux7f6aux3092ux52ddux3061ux53d6ux3063ux305fux56fdux9078ux5f01ux8b77ux4eba3ux4ebaux304cux6cd5ux30c6ux30e9ux30b9ux306bux5bfeux3059ux308bux5831ux916cux8acbux6c42ux306eux5f15ux7528ux30c4ux30a4ux30fcux30c8ux306bux601dux3046ux601dux60d1ux3068ux6253ux7b97ux306eux5f01ux8b77ux58ebux672cux7dda}{%
\paragraph{鳥取が舞台の弁護士劇場「「無罪を勝ち取った国選弁護人3人が法テラスに対する報酬請求」の引用ツイート」に思う思惑と打算の弁護士本線}\label{ux9ce5ux53d6ux304cux821eux53f0ux306eux5f01ux8b77ux58ebux5287ux5834ux7121ux7f6aux3092ux52ddux3061ux53d6ux3063ux305fux56fdux9078ux5f01ux8b77ux4eba3ux4ebaux304cux6cd5ux30c6ux30e9ux30b9ux306bux5bfeux3059ux308bux5831ux916cux8acbux6c42ux306eux5f15ux7528ux30c4ux30a4ux30fcux30c8ux306bux601dux3046ux601dux60d1ux3068ux6253ux7b97ux306eux5f01ux8b77ux58ebux672cux7dda}}

\begin{itemize}
\tightlist
\item
  〉〉〉 Linux Emacs: 2021/06/12 14:03:53 〉〉〉
\end{itemize}

:CATEGORIES: @kanazawabengosi \#金沢弁護士会 @JFBAsns
日本弁護士連合会(日弁連) \#法務省 @MOJ\_HOUMU \#国選弁護 \#弁護士報酬
\#無罪判決

\begin{itemize}
\tightlist
\item
  1409:2021-06-09\_14:44:00 \#告発状 \#\#\#\#
  「日本の検察のモノの考え方は・・・無意味な情熱、有害な拘りとも見えるあのモチベーション」という深澤諭史弁護士の共著者という匿名弁護士のツイート
  \url{https://hirono-hideki.hatenadiary.jp/entry/2021/06/09/144356} 
\item
  1410:2021-06-09\_15:56:45 \#告発状 \#\#\#\#
  「多発する性犯罪・・・・・・``示談の相場''はいくら? 弁護士が解説する「ドキュメント 示談の現場」」という田畑淳弁護士の文春オンラインの記事
  \url{https://hirono-hideki.hatenadiary.jp/entry/2021/06/09/155642} 
\item
  1411:2021-06-09\_18:04:01 \#告発状 \#\#\#\#
  強制わいせつで450万円、盗撮で300万円の示談金をもらっています、というキャバ嬢のツイートと、それに対する弁護士Twitterアカウントの反応
  \url{https://hirono-hideki.hatenadiary.jp/entry/2021/06/09/180358} 
\item
  1412:2021-06-09\_19:01:52 \#告発状 \#\#\#\#
  強制わいせつで450万円、盗撮で300万円の示談金をもらっています、というキャバ嬢のツイートと、それに対する弁護士Twitterアカウントの反応 %gimu13 @gimu13%
  \url{https://hirono-hideki.hatenadiary.jp/entry/2021/06/09/190149} 
\item
  1413:2021-06-09\_22:10:24 \#告発状 \#\#\#\#
  「刑事民事の各種法規のルールを無視する姿勢のアカウントはブロックします。」という弁護士しのだ奈保子(立憲・道7区総支部長)@yorisoibengoshiにブロックされていた件
  \url{https://hirono-hideki.hatenadiary.jp/entry/2021/06/09/221019} 
\item
  1414:2021-06-10\_18:04:15 \#告発状 \#\#\#\#
  2021年6月10日0時過ぎに高橋雄一郎弁護士のタイムラインで発見した国選弁護人3人で無罪判決900万円というツイート〜「刑弁」の神様に至る発見
  \url{https://hirono-hideki.hatenadiary.jp/entry/2021/06/10/180412} 
\item
  1415:2021-06-12\_13:21:42 \#告発状 \#\#\#\#
  2021年6月10日0時過ぎに高橋雄一郎弁護士のタイムラインで発見した国選弁護人3人で無罪判決900万円というツイート〜「刑弁」の神様に至る発見:引用ツイートのまとめ
  \url{https://hirono-hideki.hatenadiary.jp/entry/2021/06/12/132138} 
\item
  1416:2021-06-12\_13:30:54 \#告発状 \#\#\#\#
  2019年7月24日にTwitterでブロックされていることに気がついた、令和2年度の金沢弁護士会副会長、樋詰哲朗弁護士(金沢弁護士会)
  \url{https://hirono-hideki.hatenadiary.jp/entry/2021/06/12/133047} 
\end{itemize}

 最近のエントリーの一覧です。最新の投稿となっている「2019年7月24日にTwitterでブロックされていることに気がついた、令和2年度の金沢弁護士会副会長、樋詰哲朗弁護士(金沢弁護士会)」は、投稿し忘れていたものです。

 「国選弁護人3人で無罪判決900万円というツイート」は金沢地方検察庁の分類で記述していたのですが、金沢弁護士会に変更することにしました。その際に新規に作成したのがレベル3の見出しになる「弁護士鉄道と金沢弁護士会」です。

 鳥取といえば山陰本線になるはずですが、金沢や氷見強姦冤罪事件を含む北陸本線とは共通点が多々あるので、そういうテーマ性をもたせて弁護士鉄道の記録にしたいと考えました。

 「「刑弁」の神様に至る発見」という部分は最初に見出しに入れながら、取り上げることはなかったと思います。ちょっとした発見だったのですが、テーマ性に強い結びつきを感じました。

\begin{itemize}
\tightlist
\item
  \#告発状 \#\#\#\#
  2021年6月10日0時過ぎに高橋雄一郎弁護士のタイムラインで発見した国選弁護人3人で無罪判決900万円というツイート〜「刑弁」の神様に至る発見:引用ツイートのまとめ
  - 告発\金沢地方検察庁\最高検察庁\法務省\石川県警察御中2020
  \url{https://t.co/TnR5PI6vYA} 
\end{itemize}

 もう少ししっかり記録しておけばよかったと悔やまれる点があるのですが、いくつか引用ツイートが追加されていました。悔やまれるというのは追加前の引用ツイートの数のことです。50台であったとは思います。

〉〉〉 kk\_hironoのリツイート 〉〉〉

\begin{itemize}
\tightlist
\item
  RT
  kk\_hirono(刑事告発・非常上告_金沢地方検察庁御中)|kutusita\_wanko(靴下わんこ)
  日時:2021-06-12 14:21/2021/06/11 09:29 URL:
  \url{https://twitter.com/kk\_hirono/status/1403583246537924614} 
  \url{https://twitter.com/kutusita\_wanko/status/1403147244048568321} 
  \textgreater{}
  裁判所が報酬を決めていた時代を知らない世代なのだけど、同時と比べて、よくなった部分、悪くなった部分の両方あるのだと思うが、全体的にこれでもよくなってるの?
  明確な基準で法テラスの裁量がないようになっている、と言いつつ、恣意的な解釈することもあるよ。
  \url{https://t.co/gVcyIzP3jr} 
\end{itemize}

〉〉〉 kk\_hironoのリツイート 〉〉〉

\begin{itemize}
\tightlist
\item
  RT
  kk\_hirono(刑事告発・非常上告_金沢地方検察庁御中)|funkadelawyer(🀍)
  日時:2021-06-12 14:21/2021/06/10 19:44 URL:
  \url{https://twitter.com/kk\_hirono/status/1403583283087044608} 
  \url{https://twitter.com/funkadelawyer/status/1402939754455334912} 
  \textgreater{} つらい \url{https://t.co/vKVd3PpK1c} 
\end{itemize}

〉〉〉 kk\_hironoのリツイート 〉〉〉

\begin{itemize}
\tightlist
\item
  RT
  kk\_hirono(刑事告発・非常上告_金沢地方検察庁御中)|samoyedkamen(静先輩)
  日時:2021-06-12 14:21/2021/06/10 18:44 URL:
  \url{https://twitter.com/kk\_hirono/status/1403583325416030212} 
  \url{https://twitter.com/samoyedkamen/status/1402924508424511490} 
  \textgreater{}
  そもそも2週間で切りたい理由は何かしらあるんですかねそういえば
  うっかり請求忘れてくれるのを期待してるみたいな制度ですけど、どういうルールなのか気になります
  \url{https://t.co/UWlzAbdirZ} 
\end{itemize}

 現時点で「117 件のリツイート60 件の引用ツイート252
件のいいね」となっています。最初に高橋雄一郎弁護士のタイムラインで見つけた次のツイートです。

\begin{itemize}
\tightlist
\item
  TW sk123454321(木下宗一郎【弁護士/福岡県久留米市】) 日時:
  2021/06/09 18:03:17 URL:
  \url{https://twitter.com/sk123454321/status/1402551865779232772} 
  \textgreater{} 鳥取地判平成30年4月27日\\
  \textgreater{}
  無罪を勝ち取った国選弁護人3人が法テラスに対する報酬請求の期限14日を数日徒過した。報酬約900万円が払われなくなり,弁護人3人は法テラスを相手に請求。\\
  \textgreater{} 鳥取地裁は完全に請求棄却。無慈悲。\\
  \textgreater{}
  弁護人選任され6か月経過後は報酬等の中間払請求ができること等が理由。
\end{itemize}

 返信ツイートの数の表示がないのですが、「返信をさらに表示」としてリンクを開く形ですが、全部で5つのツイートが並んでいます。このうち最初の3件は滝本太郎弁護士のツイートで、取り上げ済みです。

 そのあとに「まゆろん」という女性弁護士のツイートがあるのですが、このツイートで米子市のラブホテル支配人殺害事件のことだと確信しました。二審で無罪、最高裁で破棄差し戻し、差し戻された高裁での一審への差し戻しとあるので、鳥取地裁として他にはありえないでしょう。

\begin{itemize}
\item
  TW mayukotaniguchi(まゆろん🏋️‍♂️💪筋トレの効果を実感中) 日時:
  2021/06/11 00:37:37 URL:
  \url{https://twitter.com/mayukotaniguchi/status/1403013492387373058} 
  \textgreater{} @takitaro2 @sk123454321
  報酬請求事件は最高裁まで争って上告棄却です。\\
  \textgreater{}
  なおもともとの刑事事件は、報酬払われなかった一審が一部無罪、同じ弁護人らが担当した二審は全部無罪、同じ弁護人らが担当した最高裁で破棄差し戻し、差し戻された高裁は一審に差し戻し、やり直し裁判員裁判もやはり同じ弁護人らが担当しましたよ。
\item
  TW sk123454321(木下宗一郎【弁護士/福岡県久留米市】) 日時:
  2021/06/11 07:55:45 URL:
  \url{https://twitter.com/sk123454321/status/1403123752259395585} 
  \textgreater{} @mayukotaniguchi @takitaro2
  まゆろん先生情報提供ありがとうございました。
\end{itemize}

 これまでにも注目してきたまゆろんという女性弁護士ですが、アイコンは顔写真で実名もすぐにわかっていました。夫の弁護士を東京に残し鳥取に単身赴任となっていたように思うのですが、京都地検の女とも似ていて、そのままドラマのような設定の弁護士でした。

 平成4年に金沢刑務所の拘置所で生活を始めた早い段階であったような気もするのですが、なんとか山陰本線という本を購入して読んだことがありました。数年前に家の中で見つけているのですが、まだ手にとって開くことはしていないように思います。

 まれに糊が剥がれて落ちたようなものも見ているのですが、拘置所や刑務所で読んだ私本には必ず「私本閲読許可証」という張り紙があります。月刊誌や単行本は閲読期間が一月だったと思います。月刊誌は原則廃棄でした。

 余りはっきり憶えていないところもあるのですが、私本の所持数は3冊だったと思います。免業日以外は私本交換ができました。配湯と一緒になっていたのですが、午前と午後の2回あったのか、それとも一日に1回だったのか記憶がはっきりしません。

 他に冊数外というのがあって、願箋で教育課長の許可が必要でしたが、それを全部ひっくるめると所持できる本は10冊だったと思います。

 配湯というのは、紙コップのコーヒーとカップ麺でした。カップ麺は3品あったようにも思うのですが、思い出すのは「もちもちラーメン」と日清製粉のシーフードヌードルでした。コーヒーはUCCで、今も同じに見える商品がスーパーで売っています。

 紙コップのコーヒーは2回分だったのですが、一日に2回飲んだという記憶はないような気がします。週刊誌の配布は金曜日でしたが、午前と午後に週刊誌の私本交換をしたような記憶はあるので、それだと配湯私本交換は2回あったことになりそうです。

 毎回、通路に「はいゆ、しほんこうかん」という受刑者の大きな声が鳴り響いていました。掃夫(そうふ)と呼ばれていたようにも思うのですが、雑役とも呼ばれていました。拘置所で世話役のような作業をする受刑者です。

 私本の購入には難儀したのですが、新聞の広告と本の終わりにある他の本の紹介欄が唯一の情報源でした。他にすることのない退屈しのぎという意味が大きい読書でした。なお、平成4年から6年当時は、官本の借りれる数も少なく、次までの期間も長かったと思います。

 山陰本線の本はトラベルミステリーというジャンルだったと思いますが、まるで興味のないジャンルなのに、その場の出来心のようにたまたま買った本でした。長距離トラックの仕事で国道9号線をよく走っていたことが購入動機になった可能性はあります。

 昨日の6月11日は、朝にGoogleマップで鳥取市周辺の道路状況を調べたのですが、想像以上に変わっていて、記憶にある風景とはまるで違ったものになっていました。当時の国道9号線は旧道になっている可能性がありますが、手がかりもつかめませんでした。

\begin{itemize}
\tightlist
\item
  〈〈〈 2021/06/12 15:17:12 Linux Emacs: 〈〈〈
\end{itemize}

\hypertarget{ux91d1ux6ca2ux5f01ux8b77ux58ebux4f1akanazawabengosiux3078ux91cdux8981ux306aux304aux77e5ux3089ux305bux3054ux6848ux5185}{%
\subsubsection{金沢弁護士会@kanazawabengosiへ重要なお知らせご案内}\label{ux91d1ux6ca2ux5f01ux8b77ux58ebux4f1akanazawabengosiux3078ux91cdux8981ux306aux304aux77e5ux3089ux305bux3054ux6848ux5185}}

\hypertarget{ux72ecux81eaux30c9ux30e1ux30a4ux30f3ux3092ux53d6ux5f97ux3057ux305fux91d1ux6ca2ux5f01ux8b77ux58ebux4f1aux6240ux5c5eux5f01ux8b77ux58ebux306bux5bfeux3059ux308bux544aux767aux30b5ux30a4ux30c8ux306eux958bux8a2d}{%
\paragraph{独自ドメインを取得した、金沢弁護士会所属弁護士に対する告発サイトの開設}\label{ux72ecux81eaux30c9ux30e1ux30a4ux30f3ux3092ux53d6ux5f97ux3057ux305fux91d1ux6ca2ux5f01ux8b77ux58ebux4f1aux6240ux5c5eux5f01ux8b77ux58ebux306bux5bfeux3059ux308bux544aux767aux30b5ux30a4ux30c8ux306eux958bux8a2d}}

金沢弁護士会@kanazawabengosiへ重要なお知らせご案内/独自ドメインを取得した、金沢弁護士会所属弁護士に対する告発サイトの開設

告発に至る経緯/
金沢弁護士会/金沢弁護士会@kanazawabengosiへ重要なお知らせご案内/独自ドメインを取得した、金沢弁護士会所属弁護士に対する告発サイトの開設

:CATEGORIES: @kanazawabengosi \#金沢弁護士会 @JFBAsns
日本弁護士連合会(日弁連) \#法務省 @MOJ\_HOUMU

\begin{itemize}
\item
  〉〉〉 Linux Emacs: 2021/07/14 09:12:48 〉〉〉
\item
  告発・非常上告\_2021\金沢地方検察庁御中 \textbar{} Just another
  WordPress site \url{http://hirono-hideki.info/wp/} 
\end{itemize}

 現在のところ上記がサイトのトップページになりますが、表示されるのは「金沢地方検察庁」という固定ページになります。当初、金沢地方検察庁を中心に据える方向性でいたのですが、当面は御会すなわち金沢弁護士会を対象にご案内を進めます。

 石川県民、金沢市民、能登町民に対して、金沢地方検察庁に対して手続きを進める告発事件の問題や事実関係の正しいご理解を求めるという意義が大きく、そのための情報公開でもあることもなにとぞご理解願います。

 今回の刑事告発は、殺人未遂になりますが、御会すなわち金沢弁護士会所属の被告発人は、被告発人岡田進弁護士、被告発人木梨松嗣弁護士、被告発人長谷川紘之弁護士、被告発人若杉幸平弁護士の4名になり、求める刑事処分は無期懲役刑になります。

 他にも関連した御会すなわち金沢弁護士会所属の弁護士が数名おり、重要な参考人という立場でもあり、共同不法行為のいったんを担ったとも推定されるところで、民事的、行政的な責任はもちろんのこと、今後の対応次第では刑事責任の追求を前提とします。

 本件告発事件との関わりに差異はありますが、平成4年の傷害・準強姦被告事件の一審、国選弁護人だったのが被告発人岡田進弁護士、同じく控訴審の私選弁護士人だったのが被告発人木梨松嗣弁護士になります。

 傷害・準強姦被告事件は平成4年4月1日です。懲役4年の判決が最高裁で確定したのが平成6年2月20日頃、同年3月17日に移送された福井刑務所での受刑生活が始まり、その間の平成6年11月頃、被害者安藤文さんの訴訟代理人となって損害賠償の訴訟提起をしてきたのが被告発人長谷川紘之弁護士になります。

 正確にいえば、7月か8月の訴訟提起となっていたと思いますが、金沢地方裁判所民事部A係経由で甲号証などの書面が届いたのが11月の初めになります。

\begin{itemize}
\tightlist
\item
  PICSA 一件記録・写真/2014年作成/訴状 原告訴訟代理人 弁護士 長谷川紘之 平成6年7月5日付 金沢地方裁判所御中
  - 廣野秀樹 - Picasa ウェブ アルバム : 金沢地方検察庁御中
  \url{https://t.co/p4f1FJxggR} 
\end{itemize}

 現在のパソコン内で写真ファイルが見つからず、ネットのGoogle検索で探し出したものになりますが、現在はGoogleフォトとなっているPicasaウェブアルバムの公開アルバムが記事のリンクにあります。リンクのページはページタイトルが「アルバム
アーカイブ」となっていました。

 この写真ファイルを探したのは福井刑務所の受付印の日付を確認したかったのですが、本来はあるはずの受付印がなく、これは甲号証の書類に同梱されていたものと考えられます。なお、そのときの担当裁判官が、同じく被告発人の被告発人古川龍一裁判官になります。

 福井刑務所を満期出所したのが平成9年1月18日になるかと思います。確か1月17日が満期日でその翌日の釈放になっていたと思います。その平成9年の秋になりますが、法律相談に行ったのが金沢合同法律事務所の西村依子弁護士になります。

 相談の内容など記憶は曖昧になっていますが、2010年4月2日から始めたTwitterでは、すべてのツイートをパソコン内で記録し、検索結果をブログ記事として投稿する仕組みを構築してあり、キーワードを引数に実行するコマンドがtwilog-serch-postになります。

\begin{itemize}
\tightlist
\item
  2021年07月14日10時15分の登録:
  「長谷川紘之弁護士」を@hirono\_hideki @kk\_hirono @s\_hironoで検索 425件の該当 2021-07-14\_10:15の記録
  \url{https://kk2020-09.blogspot.com/2021/07/hironohidekikkhironoshirono4252021-07.html} 
\item
  2021年07月14日10時15分の登録:
  「西村依子」を@hirono\_hideki @kk\_hirono @s\_hironoで検索 121件の該当 2021-07-14\_10:15の記録
  \url{https://kk2020-09.blogspot.com/2021/07/hironohidekikkhironoshirono1212021-07.html} 
\end{itemize}

 キーワードの検索結果だけでは不十分な情報ということもあると思います。Twilogでツイートの日付を指定すれば、当日の前後のツイートを全て閲覧することができるかと思います。なお、Twitterの検索は精度が悪いですが、Twilogの検索はほぼ間違いをみたことがありません。

\begin{quote}
《引用の始まり》
\end{quote}

\begin{quote}
2011-01-20 14:54:51
``やはりこちらに所属の弁護士でした。一度相談に行った西村依子弁護士が独立したことは知っていました。共産系の弁護士事務所だという噂も聞いています。''``金沢合同法律事務所 弁護士の紹介「飯森和彦」''
- \url{http://j.mp/gCFVu1}  "
\url{https://twitter.com/hirono\_hideki/status/27967270533726208} 
\end{quote}

\begin{quote}
《引用の終わり》
\end{quote}

\begin{itemize}
\tightlist
\item
  奉納\危険生物・弁護士脳汚染除去装置\金沢地方検察庁御中\_2020:
  「西村依子」を@hirono\_hideki @kk\_hirono @s\_hironoで検索 121件の該当 2021-07-14\_10:15の記録
  \url{https://kk2020-09.blogspot.com/2021/07/hironohidekikkhironoshirono1212021-07.html} 
\end{itemize}

 上記のツイートはアカウントがhirono\_hidekiで、ツイートの日付が2011年1月20日となっています。これだと、\url{http://twilog.org/hirono\_hideki/date-110120} 
で該当ページが開けます。よくあるリクエストパラメータの指定になっていないのもTwilogの特徴です。

 ツイートにはブログ記事のリンクが多数ありますが、そちらのブログでも調べることは出来ます。Bloggerとはてなブログをメインにしていますが、Bloggerの方は資料性が高く1つの記事で1万件以上のツイートを掲載してものもあります。

\begin{itemize}
\tightlist
\item
  奉納\危険生物・弁護士脳汚染除去装置\金沢地方検察庁御中\_2020:
  西村依子の検索結果
  \url{https://kk2020-09.blogspot.com/search?q=\%E8\%A5\%BF\%E6\%9D\%91\%E4\%BE\%9D\%E5\%AD\%90} 
\end{itemize}

 Bloggerのブログですが、前にメインのブログを変更しています。次が前のブログになります。

\begin{itemize}
\tightlist
\item
  奉納\危険生物・弁護士脳汚染除去装置\金沢地方検察庁御中:
  西村依子の検索結果
  \url{http://hirono2014sk.blogspot.com/search?q=\%E8\%A5\%BF\%E6\%9D\%91\%E4\%BE\%9D\%E5\%AD\%90} 
\end{itemize}

 ドメインに違いがありますが、前のものがhirono2014skで、現在のものがkk2020-09となっています。この2つのBloggerの記事はデータベースに記録してあり、dというコマンドで全件の取得、ddというコマンドで最新の10件を取得しています。次が実例です。

\begin{lstlisting}
base ❯ d|grep 西村依子
\end{lstlisting}

\begin{itemize}
\tightlist
\item
  2021年01月12日08時38分の登録:
  REGEXP:''西村依子''/データベース登録済みツイート:2021年01月12日08時38分の記録:ユーザ・投稿:2/2件
  \url{http://kk2020-09.blogspot.com/2021/01/regexp20210112083822.html} 
\item
  2021年02月19日23時37分の登録:
  「(梨木作次郎\textbar 金沢合同法律事務所\textbar 西村依子)」を過去のはてなダイアリーの記事から検索
  \url{https://kk2020-09.blogspot.com/2021/02/blog-post\_95.html} 
\item
  2021年02月19日23時38分の登録:
  「(梨木作次郎\textbar 金沢合同法律事務所\textbar 西村依子)」を@hirono\_hideki @kk\_hirono @s\_hironoで検索 271件の該当 2021-02-19\_23:38の記録
  \url{https://kk2020-09.blogspot.com/2021/02/hironohidekikkhironoshirono2712021-02.html} 
\item
  2021年07月14日10時15分の登録:
  「西村依子」を@hirono\_hideki @kk\_hirono @s\_hironoで検索 121件の該当 2021-07-14\_10:15の記録
  \url{https://kk2020-09.blogspot.com/2021/07/hironohidekikkhironoshirono1212021-07.html} 
\end{itemize}

\begin{lstlisting}
base ❯ d|grep 長谷川紘之弁護士
\end{lstlisting}

\begin{itemize}
\tightlist
\item
  2021年02月19日14時48分の登録:
  「長谷川紘之弁護士」を@hirono\_hideki @kk\_hirono @s\_hironoで検索
  \url{https://kk2020-09.blogspot.com/2021/02/hironohidekikkhironoshirono\_86.html} 
\item
  2021年07月14日10時15分の登録:
  「長谷川紘之弁護士」を@hirono\_hideki @kk\_hirono @s\_hironoで検索 425件の該当 2021-07-14\_10:15の記録
  \url{https://kk2020-09.blogspot.com/2021/07/hironohidekikkhironoshirono4252021-07.html} 
\end{itemize}

 パイプとgrepというコマンドを使っています。UNIX環境では定番のテキスト処理になります。オプションの指定が必要になりますが、正規表現を使うことで高度な検索が出来、パイプをつなぐことで絞り込みや結果の除外もできます。

 次に、はてなブログになりますが、これもメインで使ってきたものが2つあります。1つ目は、はてなダイアリーからのエクスポートを含んでいるのですが、タイトル名に問題のあるものが過去に含まれています。はてなダイアリーは2006年12月からの投稿になります。

 ファイル名の問題というのは本来のタイトル名が失われた記事になります。はてなダイアリーでは記事というよりエントリーと呼ばれていたとも思いますが、文字コードがEUC-JPだったということがあり、プログラムによる処理の過程でそのような不具合が発生しました。

\begin{itemize}
\tightlist
\item
  羽咋, 決定 の検索結果 -
  告発\金沢地方検察庁\最高検察庁\法務省\石川県警察御中
  \url{https://hirono-hideki.hatenablog.com/search?q=\%E7\%BE\%BD\%E5\%92\%8B\%E3\%80\%80\%E6\%B1\%BA\%E5\%AE\%9A} 
\end{itemize}

 上記が前のブログになります。はてなブログでは検索結果が要約で表示されるのでいくらか実用に耐えるものとなっていますが、Twitterのようなピンポイントの検索は難しいことがあるかもしれません。

\begin{itemize}
\tightlist
\item
  西村依子, 金沢合同法律事務所 の検索結果 -
  告発\金沢地方検察庁\最高検察庁\法務省\石川県警察御中2020
  \url{https://hirono-hideki.hatenadiary.jp/search?q=\%E8\%A5\%BF\%E6\%9D\%91\%E4\%BE\%9D\%E5\%AD\%90+\%E9\%87\%91\%E6\%B2\%A2\%E5\%90\%88\%E5\%90\%8C\%E6\%B3\%95\%E5\%BE\%8B\%E4\%BA\%8B\%E5\%8B\%99\%E6\%89\%80} 
\end{itemize}

 上記の「告発\金沢地方検察庁\最高検察庁\法務省\石川県警察御中2020」が現在メインのはてなのブログで、これはgというコマンドで、データベースに登録済みの記事のタイトルとURLを取得しています。Bloggerはど数は多くないので、最新10件のみのコマンドは未作成です。

\begin{lstlisting}
base ❯ g|grep 金沢弁護士会
\end{lstlisting}

\begin{itemize}
\tightlist
\item
  94:2020-01-15\_17:20:55 *
  樋詰哲朗弁護士(金沢弁護士会)のツイートから知ることになった、新潟市の白山神社、新潟総鎮守、Googleマップで確認した場所は新潟地裁の向かいだった
  \url{https://hirono-hideki.hatenadiary.jp/entry/2020/01/15/172050} 
\item
  237:2020-04-04\_00:42:40 \#\#
  弁護士樋詰哲朗さんはTwitterを使っています
  「金沢弁護士会の副会長に就任。初日は怒涛の挨拶周り。なんと33か所。」 /
  Twitter \url{https://hirono-hideki.hatenadiary.jp/entry/2020/04/04/004237} 
\item
  238:2020-04-04\_01:29:15 \#\#
  「昨日、あいさつ回りでこの建物に入ったのだよねえ。金沢税務署に新型コロナ感染者 申告相談など中断
  - 毎日新聞」という樋詰哲朗弁護士(金沢弁護士会)のツイート
  \url{https://hirono-hideki.hatenadiary.jp/entry/2020/04/04/012912} 
\item
  244:2020-04-04\_20:03:54 \#\#
  野田隼人弁護士(滋賀弁護士会)への返信ツイートとなっている2020年4月3日の樋詰哲朗弁護士(金沢弁護士会)の最新ツイート(4月4日19時40分現在)
  \url{https://hirono-hideki.hatenadiary.jp/entry/2020/04/04/200351} 
\item
  429:2020-04-28\_10:39:24 \#\#
  樋詰哲朗弁護士(金沢弁護士会)のTwitterタイムラインでの倉重公太朗弁護士のTwitterアカウントの発見
  \url{https://hirono-hideki.hatenadiary.jp/entry/2020/04/28/103922} 
\item
  448:2020-05-03\_14:05:45 \#
  請求に至る経緯/金沢弁護士会の対応及び問題意識/樋詰哲朗弁護士(金沢弁護士会所属・令和2年度金沢弁護士会副会長)/樋詰哲朗弁護士(金沢弁護士会)のTwitterタイムラインで知ったAWS(Amazon Web Services)
  \url{https://hirono-hideki.hatenadiary.jp/entry/2020/05/03/140542} 
\item
  461:2020-05-19\_20:59:32 \#\#
  金沢西警察署及び石川県警察と対比した,再捜査,刑事告発,非常上告事件に対する金沢弁護士会の反応と対応
  \url{https://hirono-hideki.hatenadiary.jp/entry/2020/05/19/205930} 
\item
  462:2020-05-19\_21:00:03 \#\#\#
  令和2年5月18日,「長崎ぶらぶら節」,金沢市の思案橋からの発見となった令和2年度金沢弁護士会会長の宮西香弁護士
  \url{https://hirono-hideki.hatenadiary.jp/entry/2020/05/19/210001} 
\item
  464:2020-05-20\_09:44:31 \#\#\#\#
  金沢弁護士会のTwitterアカウントを探すため,Twitterで検索をしたところでの宮西香金沢弁護士会会長の会長声明の発見(1)
  \url{https://hirono-hideki.hatenadiary.jp/entry/2020/05/20/094428} 
\item
  465:2020-05-20\_09:48:36 \#\#\#\#
  金沢弁護士会のTwitterアカウントを探すため,Twitterで検索をしたところでの宮西香金沢弁護士会会長の会長声明の発見(2)
  \url{https://hirono-hideki.hatenadiary.jp/entry/2020/05/20/094830} 
\item
  466:2020-05-20\_10:00:34 \#\#\#\#
  金沢弁護士会のTwitterアカウントを探すため,Twitterで検索をしたところでの宮西香金沢弁護士会会長の会長声明の発見(3)
  \url{https://hirono-hideki.hatenadiary.jp/entry/2020/05/20/100031} 
\item
  467:2020-05-20\_10:01:41 \#\#\#\#
  金沢弁護士会のTwitterアカウントを探すため,Twitterで検索をしたところでの宮西香金沢弁護士会会長の会長声明の発見(4)
  \url{https://hirono-hideki.hatenadiary.jp/entry/2020/05/20/100134} 
\item
  468:2020-05-20\_10:04:41 \#\#\#\#
  金沢弁護士会のTwitterアカウントを探すため,Twitterで検索をしたところでの宮西香金沢弁護士会会長の会長声明の発見(5)
  \url{https://hirono-hideki.hatenadiary.jp/entry/2020/05/20/100436} 
\item
  470:2020-05-20\_14:24:32 \#\#\#\#
  埋め込みツイートのoembedメソッドの変換に失敗して,5分割にした「金沢弁護士会のTwitterアカウントを探すため,Twitterで検索をしたところでの宮西香金沢弁護士会会長の会長声明の発見」という記事
  \url{https://hirono-hideki.hatenadiary.jp/entry/2020/05/20/142429\textbackslash}  end\{lstlisting\}
\end{itemize}

 この現在メインのはてなブログは、告発状本体の記述を見出し付きで投稿しています。その見出しの書式を変更したのですが、この記事がその最初の投稿になるかと思います。書式は、「(見出しレベル3)/(見出しレベル4)」になると思います。

 告発状は、今年、令和3年3月31日付で能都郵便局から郵送し同月13日付になっていたと思いますが、金沢地方検察庁から郵送で返戻されて現在手元にあるものがあり、PDFファイルとしてネットで公開していますが、779ページになっていたかと思います。

 文字サイズを小さくしてあるのですが、通常の裁判所の文書の書式と文字サイズだと2千ページ相当になるかもしれません。ワープロソフトう使わず、プログラムを利用してMarkdownのテキストファイルからPDFファイルを作成しています。

\begin{itemize}
\tightlist
\item
  TW hirono\_hideki(奉納\さらば弁護士鉄道・泥棒神社の物語) 日時:
  2021/04/01 19:25:36 URL:
  \url{https://twitter.com/hirono\_hideki/status/1377567816644141057} 
  \textgreater{} 2021-03-31\_令和3年3月31日付け 告発状.pdf - Google
  ドライブ \url{https://t.co/2bukX1dO6a} 
  \textgreater{}
   昨日の3月31日(1ページと506ページから779ページ)と本日4月1日(2ページから505ページ,能都郵便局から郵送しました。金沢市大手町615号,金沢地方検察庁です。
\end{itemize}

 昨日の7月13日になりますが、現在のパソコン環境でPDFファイルを作成しました。LaTeXの環境が必要になるのですが、これをインストールするところから始めました。自作したプログラム(スクリプト)の使い方も忘れていたのですが、エラーが多発して修正に時間が掛かりました。

 最後まで時間の掛かったのは、1つの絵文字が致命的なエラーの原因だったのですが、PDFファイルでは表示されない絵文字もあるようです。見た目が同じコピペした絵文字でもエラーが発生しなかったのですが、PDFファイルにはその絵文字が表示されていませんでした。

 ワープロソフトを使わないのはいくつか理由があるのですが、LaTeXから作成したPDFファイルでは、目次が階層に応じた連番付きで自動生成され、その目次の見出しが、本文のリンクになっているという利便性も、ワープロソフトからのPDFファイルでは難しいのかもしれません。

 2021-04-17-211205\_告発状.mdが現在編集中のMarkdown形式のテキストファイルになります。

 現時点でwcコマンドの結果が「21961 55416 2930782
2021-04-17-211205\_告発状.md」となっていますが、行数が21,961、ワード数が55,416、文字数が2,930,782ということになります。

 「27733 70291 3611321
2021-02-27-083344\_告発状2021-03.md」が郵送から2週間ほどで返戻された令和3年3月31日付告発状になると思いますが、これで779ページでした。なお、引用部分などフォントサイズを変えた部分もあったかと思います。

 Linuxではbcというコマンドで計算ができるのですが、「3611321/400」の結果が9028となっていました。昔、400文字詰め原稿用紙というのを聞いた憶えがあって計算してみたのですが、これだと9千ページを超えることになりそうです。

 これまでは私が「弁護士鉄道」と呼ぶ、弁護士業界の歴史や現状を中心にした、いわば序章のようなものでした。歴史的な記録としてまとめた意義は大きいと思うのですが、ページ数はなるべく増やさないように心掛け、それに伴った中断も多くありました。

 全てではないですが、告発状の記述は、一行ごとにTwitterにツイートをしています。そのアカウントが3つの1つで告発\市場急配センター殺人未遂事件\金沢地方検察庁・石川県警察御中(@kk\_hirono)になります。これもテキストファイルとしてツイートを保存しています。

 更新したてですが、「146628 1048077 49064498
kk\_hirono2021-07-14\_124845.csv」となっています。空行はなく1行が1件のツイートに対応しているはずなので、146,628件のツイートになりますが、中断していた時期も多いアカウントで、特別多いツイート数ではないと思います。

 「242504 2104725 82330798
hirono\_hideki2021-07-14\_112503.csv」が奉納\さらば弁護士鉄道・泥棒神社の物語(@hirono\_hideki)のアカウントになります。2,3日前に、Bloggerのブログ記事の紹介ツイートを非常上告-最高検察庁御中\_ツイッター(@s\_hirono)に変更しました。

 これまで数年間、非常上告-最高検察庁御中\_ツイッター(@s\_hirono)のアカウントはスクリーンショットの画像や写真付きツイートに限定していたのですが、奉納\さらば弁護士鉄道・泥棒神社の物語(@hirono\_hideki)のフォロワーの負担を減らすことにしました。

 自分からフォローすることはほとんどなかったのですが、半年ほど前になるのかフォロー返しもしないことにしました。ピークは5千近くのフォロワー数だったのですが、逓減しながら現時点で4,257となっています。

 「83171 603506 26349270
s\_hirono2021-07-14\_102504.csv」が非常上告-最高検察庁御中\_ツイッター(@s\_hirono)のアカウントになります。ツイートの数は比較的少ないですが、まったく投稿しない時期が多かったとも思います。

 次に、被告発人若杉幸平弁護士ですが、すでにまとまった記述を今回作成の告発状でやっていると思います。

 「472,313件のツイートを検索し,147件の該当がありました。」という結果ですが、コマンドを実行し、ブログに投稿します。意外に少なく感じました。

\begin{itemize}
\tightlist
\item
  2021年07月14日13時29分の登録:
  「若杉幸平弁護士」を@hirono\_hideki @kk\_hirono @s\_hironoで検索 147件の該当 2021-07-14\_13:29の記録
  \url{https://kk2020-09.blogspot.com/2021/07/hironohidekikkhironoshirono1472021-07.html} 
\end{itemize}

\begin{quote}
《引用の始まり》
\end{quote}

\begin{quote}
2021-06-02 19:14:04
``本件告発事件の一つの鍵を握るのが、この被告発人若杉幸平弁護士で、まるで古本屋にあるミステリー小説の世界のようですが、雑然と置かれた書類のようなものも、古本屋に置かれた資料のように見えたのだと思います。部屋の中がかなり薄暗く感じられたのも印象的でした。''
https://twitter.com/kk\_hirono/status/1400032962859524100

2021-06-02 19:25:43
``今改めて考えると、女性事務員は被告発人若杉幸平弁護士の妻だった可能性も一応ありそうですが、法律事務所内に人は被告発人若杉幸平弁護士とその女性の2人だけではなかったと思います。''
https://twitter.com/kk\_hirono/status/1400035894740938766

2021-06-02 19:26:59
``被告発人若杉幸平弁護士の法律事務所に男性がいたという記憶はないのですが、他に女性事務員がいたとしても他に印象に強すぎることがあり、記憶に残らなかったのかもしれません。他に弁護士らしい人の姿を見たという記憶もありません。けっこうワンマンにも思える雰囲気が感じられました。''
https://twitter.com/kk\_hirono/status/1400036215416442881

2021-06-02 21:59:58 ``- 1392:2021-06-02\_21:59:56 \#告発状 \#\#\#\#
道路を挟んで金沢地方検察庁の横にあった被告発人若杉幸平弁護士の法律事務所の建物と、金沢地検の歴史
https://t.co/bab4VCnbH3''
https://twitter.com/hirono\_hideki/status/1400074711925596166

2021-06-03 10:31:07 ``- 1392:2021-06-02\_21:59:56 \#告発状 \#\#\#\#
道路を挟んで金沢地方検察庁の横にあった被告発人若杉幸平弁護士の法律事務所の建物と、金沢地検の歴史
https://t.co/vkZVSgDzlA''
https://twitter.com/kk\_hirono/status/1400263745604030479

2021-06-04 23:20:29 ``@shukan\_shincho - 1392:2021-06-02\_21:59:56
\#告発状 \#\#\#\#
道路を挟んで金沢地方検察庁の横にあった被告発人若杉幸平弁護士の法律事務所の建物と、金沢地検の歴史
https://t.co/bab4VCnbH3''
https://twitter.com/hirono\_hideki/status/1400819752495292422

2021-06-04 23:20:57 ``@shujoprime - 1392:2021-06-02\_21:59:56
\#告発状 \#\#\#\#
道路を挟んで金沢地方検察庁の横にあった被告発人若杉幸平弁護士の法律事務所の建物と、金沢地検の歴史
https://t.co/bab4VCEN5D''
https://twitter.com/hirono\_hideki/status/1400819868509913091

2021-07-02 17:18:33
``図書館の金沢市内の住宅地図で、被告発人若杉幸平弁護士の金沢市田井町の住所を調べたものですが、表記は「若杉」だけになっていることを確認しました。金沢弁護士会のホームページでも若杉法律事務所の住所となっていた番地です。''
https://twitter.com/kk\_hirono/status/1410875530748956680

2021-07-14 09:32:16
``今回の刑事告発は、殺人未遂になりますが、御会すなわち金沢弁護士会所属の被告発人は、被告発人岡田進弁護士、被告発人木梨松嗣弁護士、被告発人長谷川紘之弁護士、被告発人若杉幸平弁護士の4名になり、求める刑事処分は無期懲役刑になります。''
https://twitter.com/kk\_hirono/status/1415106841046962177

2021-07-14 13:27:16
``次に、被告発人若杉幸平弁護士ですが、すでにまとまった記述を今回作成の告発状でやっていると思います。''
https://twitter.com/kk\_hirono/status/1415165977185058821
\end{quote}

\begin{quote}
《引用の終わり》
\end{quote}

\begin{itemize}
\tightlist
\item
  奉納\危険生物・弁護士脳汚染除去装置\金沢地方検察庁御中\_2020:
  「若杉幸平弁護士」を@hirono\_hideki @kk\_hirono @s\_hironoで検索 147件の該当 2021-07-14\_13:29の記録
  \url{https://kk2020-09.blogspot.com/2021/07/hironohidekikkhironoshirono1472021-07.html} 
\end{itemize}

 上記の引用部分は、記事の139件目から147件目の最後のツイートになるのですが、この番号はHTMLのタグで表示しているだけの番号なので、コピペが出来ません。

 ざっとみて、本日の7月14日の前が7月2日、その前が6月4日、6月3日、6月2日という流れがつかめます。

 告発事件の被害者は、被害者安藤文さんになります。その父親が安藤健次郎さんになるのですが、その安藤健次郎さんに対する傷害事件を起こしたのが平成11年8月8日で、同月12日に金沢中警察署に逮捕されました。

 安藤健次郎さんとは最初に二人で会ったのが平成9年の9月の20日頃になりますが、月に1,2回のペースで、安藤健次郎さんが部長をする西鐵工所という会社に電話をしていました。最終的には傷害事件になったのですが、この間に、御会すなわち金沢弁護士会の弁護士と法律相談をしています。

 今の記憶では時期がはっきりしないのですが、平成9年の秋が先程の金沢合同法律事務所の西村依子弁護士で、金沢弁護士会からの紹介で向かったのが畠山美智子弁護士、兼六法律事務所の小堀秀行弁護士との相談だけが、お金の支払いがあったもので1時間で1万円だったように思います。

 安藤健次郎さんに対する傷害事件では、一審の国選弁護人が野田政仁弁護士、控訴審が前に法律相談に行った小堀秀行弁護士でした。どちらも1回だけの接見だったと思いますが、安藤健次郎さんとの事情など一切話を聞いてもらえず、再犯なので実刑間違いなしとだけ繰り返し強調されました。

 この辺りの事情は、上告審の国選弁護人となった東京都渋谷区桜丘の山口治夫弁護士がわかりやすい記録を手紙として残してくれたのですが、先日、家の中で母親と父親の結婚式の写真を見つける少し前に、思わぬ場所で、上告審判決後の手紙を見つけていました。

 一審の野田政仁弁護士、控訴審の小堀秀行弁護士との総意のような内容の手紙でした。弁護士鉄道の次の乗車券のようなもので、そこに被害者安藤文さんや父親の安藤健次郎さんの姿はなく、殺人未遂の告発事件という事実の存在はどこにもありません。

 被害者安藤文さんに対する殺人未遂という事実をこの世の中から消し去り、覆い隠したかたちになりますが、それを弁護士や裁判官の立場でやったわけです。最後まで警察や検察が無視と放置をしてくれるという期待があるのかもしれないですが、こちらは手続きを進めるだけです。

 ここらで一区切りをつけますが、このエントリーのタイトルは前段に「金沢弁護士会@kanazawabengosiへ重要なお知らせご案内」を含むシリーズ物になります。

\begin{itemize}
\tightlist
\item
  〈〈〈 2021/07/14 14:11:51 Linux Emacs: 〈〈〈
\end{itemize}

\hypertarget{section-4}{%
\paragraph{}\label{section-4}}

\hypertarget{ux544aux767aux306eux4e8bux5b9f}{%
\section{告発の事実}\label{ux544aux767aux306eux4e8bux5b9f}}

\hypertarget{ux88abux544aux767aux4ebaux3089ux306eux95a2ux4e0eux3068ux5f79ux5272ux53caux3073ux5177ux4f53ux7684ux72afux7f6aux4e8bux5b9f}{%
\subsection{被告発人らの関与と役割及び具体的犯罪事実}\label{ux88abux544aux767aux4ebaux3089ux306eux95a2ux4e0eux3068ux5f79ux5272ux53caux3073ux5177ux4f53ux7684ux72afux7f6aux4e8bux5b9f}}

\hypertarget{ux88abux544aux767aux4ebaux5ca1ux7530ux9032ux5f01ux8b77ux58eb}{%
\subsubsection{被告発人岡田進弁護士}\label{ux88abux544aux767aux4ebaux5ca1ux7530ux9032ux5f01ux8b77ux58eb}}

\hypertarget{ux88abux544aux767aux4ebaux5ca1ux7530ux9032ux5f01ux8b77ux58ebux304cux4ee3ux7406ux4ebaux3068ux306aux308aux904eux5931ux5272ux5408ux304c45ux306bux306aux3063ux305fux91d1ux6ca2ux5e02ux5185ux30671ux7d1a1ux53f7ux5f8cux907aux969cux5bb3ux3092ux6b8bux3057ux305f13ux6b73ux5973ux5b50ux306eux81eaux8ee2ux8ecaux4e8bux6545}{%
\paragraph{被告発人岡田進弁護士が代理人となり過失割合が45%になった,金沢市内で1級1号後遺障害を残した13歳女子の自転車事故}\label{ux88abux544aux767aux4ebaux5ca1ux7530ux9032ux5f01ux8b77ux58ebux304cux4ee3ux7406ux4ebaux3068ux306aux308aux904eux5931ux5272ux5408ux304c45ux306bux306aux3063ux305fux91d1ux6ca2ux5e02ux5185ux30671ux7d1a1ux53f7ux5f8cux907aux969cux5bb3ux3092ux6b8bux3057ux305f13ux6b73ux5973ux5b50ux306eux81eaux8ee2ux8ecaux4e8bux6545}}

\begin{itemize}
\tightlist
\item
  〉〉〉 Linux Emacs: 2021/04/26 06:14:57 〉〉〉
\end{itemize}

:CATEGORIES: @kanazawabengosi \#金沢弁護士会 @JFBAsns
日本弁護士連合会(日弁連) \#法務省 @MOJ\_HOUMU \#被告発人岡田進弁護士

\begin{itemize}
\tightlist
\item
  1332:2021-04-26\_06:03:53 \#告発状 \#\#\#\#
  「一番すごいもので慰謝料20万円に対して、弁護士費用95万円を認容したものがある。」という裁判官と弁護士の異常性を印象づける深澤諭史弁護士の過去のツイート
  \url{https://hirono-hideki.hatenadiary.jp/entry/2021/04/26/060351} 
\end{itemize}

 きっかけは上記のエントリーでした。〈〈〈 2021/04/24 16:10:52 Linux
Emacs:
〈〈〈で中断をしていますが,そのあと宇出津の図書館に向かいました。そのまま再開するつもりでいたのですが,新たな発見が多すぎて対処が困難と判断し,強制的に切り上げたかたちになります。

\begin{itemize}
\tightlist
\item
  TW fukazawas(深澤諭史) 日時: 2020/07/07 14:15:53 URL:
  \url{https://twitter.com/fukazawas/status/1280369926843203584} 
  \textgreater{}
  余り扇情的なことをいうつもりはないけど,ネットの誹謗中傷投稿について,150万円の慰謝料と,弁護士費用等400万円が認められた判決も最近出ていますね(欠席判決ではない)。\\
  \textgreater{}
  被害者・投稿者・プロバイダを弁護してる経験上いえることですが,自己判断でネット情報と心中前に悩まず弁護士へ。
\end{itemize}

 ネットの誹謗中傷投稿で150万円の慰謝料というのもにわかに信じがたく,思い出したのが「金沢市内で1級1号後遺障害を残した13歳女子の自転車事故」の両親に対する慰謝料でした。生涯介護を必要とするともなっていたと思います。

 その両親に対する慰謝料は各々275万円でしたが,あとで500万円の算定が45%の過失相殺でその金額になったと理解しました。過去の記事を読むと過去にもその旨理解をしていたようですが,記憶にはなく中途半端な金額に思えたことが印象的でした。

 ココログフリーの自分のブログと被告発人岡田進弁護士が代理人だったことで記憶にあったのですが,改めてGoogleで検索すると情報の手がかりすらつかめず,見つけることが出来たのは次のこれも自分のブログ記事で,内容は読み込みに異常に時間の掛かったココログの転載になるはずです。

\begin{itemize}
\tightlist
\item
  生涯介護にあたることになった両親に慰謝料各々275万円の民事裁判 -
  告発\金沢地方検察庁\最高検察庁\法務省\石川県警察御中
  \url{https://t.co/usc4dAGT8H} 
\end{itemize}

 確認するとココログフリーからの転載ではなく,2009年4月18日の記事となっていました。引用の元記事はリンク切れです。記事の削除ではなくドメイン自体が消滅している様子です。

\begin{itemize}
\item
  BAOBABNET(バオバブネット) \textbar{} は \textbar{}
  インターネットサービスプロバイダー全国一覧表 \textbar{} JAIPA -
  一般社団法人日本インターネットプロバイダー協会 \url{https://t.co/aKHafaHmxx} 
\item
  \url{https://t.co/EGLxmZtALg}  - Google 検索 \url{https://t.co/F4dqCcMDJZ} 
\end{itemize}

 高知県高知市に存在したプロバイダのようです。社名変更になった可能性もあるかと思いますが,URLに半角の〜があるので,ユーザに割り当てられたホームページになるかと思います。注意深く観察しているわけではないですが,最近はほとんど利用を見かけないサービスの形態です。

 余談になりますが,バオバブというのは強く印象に残る言葉で,2009年,珠洲市飯田町に職業訓練に通っている時,2級のワープロ検定の練習問題に「バオバブの木」というのがあって,繰り返しタイピングをしていました。

\begin{itemize}
\item
  石川)原発差し止め訴訟7年半 住民側弁護団長に聞く:朝日新聞デジタル
  \url{https://t.co/HYRrow9rsI}  聞き手・波多野陽 2019年12月30日 3時00分
\item
  石川)原発差し止め訴訟7年半 住民側弁護団長に聞く:朝日新聞デジタル
  \url{https://t.co/HYRrow9rsI} 
  金沢地裁で続く北陸電力志賀原発の差し止め訴訟は2012年6月の提訴から7年半が過ぎた。何が争われ、そこから何が読み解けるのか、住民側の弁護団長の岩淵正明弁護士(69)に聞いた。
\end{itemize}

 名前は見たことのある弁護士だと思ったのですが,検索をすると意外な情報が出てきました。2019年12月30日という比較的新しい記事で,北陸電力志賀原発の差し止め訴訟の原告住民の弁護団長とあります。

\begin{itemize}
\item
  岩淵 正明 - 弁護士法人 北尾法律事務所 \url{https://t.co/pfprk6lW5N}  ¥\n
  金沢家庭裁判所 家事調停委員 ¥\n
  日本弁護士連合会公害対策・環境保全委員会委員 ¥\n ¥\n
  連合石川法曹団代表 ¥\n ¥\n 石川県社会法律センター理事長
\item
  奥村 回 - 弁護士法人 北尾法律事務所 \url{https://t.co/E3BEcAwelA}  ¥\n
  2016(平成28)~2017(平成29)年度 ¥\n ¥\n
  日本弁護士連合会刑事弁護センター委員長 ¥\n
  2018(平成30)~2019(令和元年) ¥\n ¥\n
  日本弁護士連合会国選弁護本部本部長代行
\end{itemize}

 見覚えのある弁護士の名前だと思ったら奥村回弁護士と同じ北尾法律事務所の弁護士でした。奥村回弁護士が金沢弁護士会の会長だったのは2012年度と再確認しました。ちょうど,金沢弁護士会に電話を掛けていた頃になるのかもしれません。ちょっと確認をしておきます。

\begin{itemize}
\item
  2021年04月26日07時13分の登録:
  「金沢弁護士会.*電話」を@hirono\_hideki @kk\_hirono @s\_hironoで検索 26件の該当 2021-04-26\_07:13の記録
  \url{https://kk2020-09.blogspot.com/2021/04/hironohidekikkhironoshirono262021-04.html} 
\item
  2010-08-18 23:16:33
  ``「金沢法テラス、金沢弁護士会、金沢地方検察庁への電話連絡(2010年8月18日)のまとめ」をトゥギャりました。
  \url{https://t.co/1dTqVFrHyo''}  \url{https://t.co/odSNcjj2pb} 
\item
  2011-08-08 12:54:24
  ``2011-08-02-岡田進弁護士(金沢弁護士会)との電話.flv -- YouTube
  \textbar{} 再審請求\_金沢地方裁判所御中 \url{https://t.co/SW66xMvaFj} 
  タイトルを一部変更しました。(金沢弁護士会)を追加です。''
\item
  2012-08-09 16:44:31 ``金沢弁護士会に電話をかけて相談しました。''
  \url{https://t.co/K3PFma9ust} 
\end{itemize}

 被告発人岡田進弁護士の法律事務所電話を掛け,会話を録音しYouTubeで公開したのが2011年8月でした。私の記憶では金沢弁護士会に電話を掛けて懲戒請求の相談をしたのも同じ頃になるのですが,時間も掛かりそうなのでまたの機会に確認をしたいと思います。

 「原告ら訴訟代理人弁護士 岩淵正明」となっているのが先程調べた北尾法律事務所の岩淵正明弁護士,並んでひとつ下に「被告訴訟代理人弁護士 岡田 進」とあります。金沢弁護士会に同姓同名の弁護士はいないはずなので岡田進弁護士に間違いはないでしょう。

 さきほど「(口頭弁論終結日 平成18年10月6日)」となっていることに気がついたのですが,判決期日のような日付は見当たりません。早とちりもあって平成18年10月に判決があったものと思い込み,そのまま図書館に行って調べました。

\begin{itemize}
\tightlist
\item
  〈〈〈 2021/04/26 07:27:13 Linux Emacs: 〈〈〈
\end{itemize}

\hypertarget{ux670824ux65e5ux306825ux65e5ux306eux4e21ux65e5ux5b87ux51faux6d25ux56f3ux66f8ux9928ux3067ux8abfux3079ux305fux5e73ux621015ux5e74ux306eux91d1ux6ca2ux5e02ux5185ux30671ux7d1a1ux53f7ux5f8cux907aux969cux5bb3ux3092ux6b8bux3057ux305f13ux6b73ux5973ux5b50ux306eux81eaux8ee2ux8ecaux4e8bux6545ux5ca1ux7530ux9032ux5f01ux8b77ux58ebux304cux4ee3ux7406ux4ebaux3067ux904eux5931ux76f8ux6bba45}{%
\paragraph{4月24日と25日の両日,宇出津図書館で調べた平成15年の「金沢市内で1級1号後遺障害を残した13歳女子の自転車事故」,岡田進弁護士が代理人で過失相殺45%}\label{ux670824ux65e5ux306825ux65e5ux306eux4e21ux65e5ux5b87ux51faux6d25ux56f3ux66f8ux9928ux3067ux8abfux3079ux305fux5e73ux621015ux5e74ux306eux91d1ux6ca2ux5e02ux5185ux30671ux7d1a1ux53f7ux5f8cux907aux969cux5bb3ux3092ux6b8bux3057ux305f13ux6b73ux5973ux5b50ux306eux81eaux8ee2ux8ecaux4e8bux6545ux5ca1ux7530ux9032ux5f01ux8b77ux58ebux304cux4ee3ux7406ux4ebaux3067ux904eux5931ux76f8ux6bba45}}

\begin{itemize}
\tightlist
\item
  〉〉〉 Linux Emacs: 2021/04/26 07:37:37 〉〉〉
\end{itemize}

:CATEGORIES: @kanazawabengosi \#金沢弁護士会 @JFBAsns
日本弁護士連合会(日弁連) \#法務省 @MOJ\_HOUMU \#被告発人岡田進弁護士

\begin{itemize}
\tightlist
\item
  1333:2021-04-26\_07:29:01 \#告発状 \#\#\#\#
  被告発人岡田進弁護士が代理人となり過失割合が45%になった,金沢市内で1級1号後遺障害を残した13歳女子の自転車事故
  \url{https://hirono-hideki.hatenadiary.jp/entry/2021/04/26/072859} 
\end{itemize}

 上記エントリーの締めくくりに「早とちりもあって平成18年10月に判決があったものと思い込み,そのまま図書館に行って調べました。」と記述しましたが,北國新聞縮小版平成18年10月号に記事は見当たりませんでした。

 北國新聞縮小版平成18年10月号の索引,「県内(裁判)」にあるのは,「小松基地騒音訴訟」,「県立中央病院・医療過誤 県に55万円の支払い命令,説明義務違反のみ認定,名高裁金沢」の2つだけでした。

 はてなブログの記事を読み直すと,最初の方に「金沢地裁 平成18年10月11日判決(碓定)」とありました。口頭弁論終結日というのは刑事裁判の結審に相当しそうですが,2,3ヶ月後の判決期日も普通だと考えていました。同じ10月の8日の終結で11日の判決というのも驚きです。

 他にも大きな見落としがあったことに気が付きました。「平成15年
8月7日午前11時08分ころ、」という部分です。私は平成15年という情報しかないものと思い込み,24日の土曜日は,北國新聞縮小版の3月,4月,5月,次に6月,7月を閲覧しました。

 ちょうど平成18年以降は図書館内の棚にあるとも係の女の人に聞いたのですが,他は倉庫から持ってきてもらうことになり,最初3冊ぐらいはと言われたのですが,ずいぶん重そうに見えたので2回目は2冊でお願いしたのです。

 昨日の25日になりますが,平成16年6月の記事を調べるのに図書館に行きました。目的の記事は見つからず,土日17時の閉館時間までは時間も少なかったので,前日の続きで平成15年の8月,9月を二冊お願いし索引に目を通しました。

 平成15年としか情報がないという思い込みで,1月2月に外に自転車に乗ることは少ないという見込で範囲を絞り3月から初めて12月まで探し,それで記事がなければ1月2月を調べるつもりでいました。

 平成15年8月の\textless 事件・事故\textgreater の索引に,それらしい自転車の交通事故は見当たりません。少し記憶にあった事件で,金沢市内のファミレスでの暴力団の抗争事件があり,2人刺殺とあります。見出しに「笠舞で2人刺殺,けんか,数人逃走」とあります。

 再び図書館に探しに行く,必要性の手間が省けたのですが,平成15年としか情報がないものと勘違いしたために,今後にも繋がる意外な発見がいくつかあり,それが被告発人木梨松嗣弁護士に結びついたものもあります。

\begin{itemize}
\tightlist
\item
  〈〈〈 2021/04/26 08:43:15 Linux Emacs: 〈〈〈
\end{itemize}

\hypertarget{ux5f01ux8b77ux58ebux4efbux5b98ux3069ux3069ux3044ux3064ux96c6-ux5f01ux8b77ux58ebux304bux3089ux88c1ux5224ux5b98ux306bux306aux3063ux305fux7af9ux5185ux6d69ux53f2ux306eux3069ux3069ux3044ux3064ux96c6ux3068ux3044ux3046ux30d6ux30edux30b0ux3067ux77e5ux3063ux305fux9ad8ux5ca1ux306eux7d44ux9577ux592bux5a66ux5c04ux6bbaux4e8bux4ef6ux3067ux6b7bux5211ux6c42ux5211ux306eux5171ux72afux8005ux306eux7121ux7f6aux5224ux6c7a}{%
\paragraph{「弁護士任官どどいつ集 弁護士から裁判官になった竹内浩史のどどいつ集」というブログで知った,高岡の組長夫婦射殺事件で死刑求刑の共犯者の無罪判決}\label{ux5f01ux8b77ux58ebux4efbux5b98ux3069ux3069ux3044ux3064ux96c6-ux5f01ux8b77ux58ebux304bux3089ux88c1ux5224ux5b98ux306bux306aux3063ux305fux7af9ux5185ux6d69ux53f2ux306eux3069ux3069ux3044ux3064ux96c6ux3068ux3044ux3046ux30d6ux30edux30b0ux3067ux77e5ux3063ux305fux9ad8ux5ca1ux306eux7d44ux9577ux592bux5a66ux5c04ux6bbaux4e8bux4ef6ux3067ux6b7bux5211ux6c42ux5211ux306eux5171ux72afux8005ux306eux7121ux7f6aux5224ux6c7a}}

\begin{itemize}
\tightlist
\item
  〉〉〉 Linux Emacs: 2021/04/26 10:27:05 〉〉〉
\end{itemize}

:CATEGORIES: @kanazawabengosi \#金沢弁護士会 @JFBAsns
日本弁護士連合会(日弁連) \#法務省 @MOJ\_HOUMU \#被告発人岡田進弁護士

 この事件も24日に宇出津図書館でみた北國新聞縮小版で少し思い出していたのですが,指示役と実行犯の二人に死刑判決が出ていたことも知らなかったように思います。

 富山県高岡市といえば,金沢市場輸送と市場急配センターの竹沢俊寿会長が,高岡市の大日本平和会という右翼団体のような暴力団の関係者だと福井刑務所で聞いたことがありました。

 その富山県高岡市の右翼団体,具体的な名称は聞いていなかったようにも思いますが,平成11年の7月ぐらいに関係者と何度か一緒に仕事をしたことがありました。人夫出しとも呼ばれる人材派遣の仕事でしたが,早朝に高岡市内に立ち寄って,数人を車に乗せたか,別の車と待ち合わせで現場に向かいました。

 富山市内のたぶん富山インターの近くの大きな現場のことはいくらか記憶に残っているのですが,魚津市内の松下電器の現場に一緒に行ったのか,現在記憶ははっきりしません。御庁つまり金沢地方検察庁には,当時の克明な記述を書面で提出してあるはずです。

 その建設,建築現場への人材派遣の仕事では,現場で事故に見せかけられて殺害される危険性を常に意識していましたが,それは関係者KYNの配管工事の仕事をしていた頃も同じで,元暴力団員のMT君の不審な言動や小競り合いなど,これも克明に記録を書面で提出してあります。

 派遣先には,被告発人大網健二が営業部長を務めた本陣不動産株式会社の本陣グループで,本陣建設もありました。石川県内の暴力団との関係は被告発人大網健二から聞いており,当時の昭成会のナンバー2の人物が会社に出入りしているとも彼は話していたことがありました。

 (※)ちょうど被告発人大網健二が本陣不動産株式会社で働き出した頃になると思いますが,金沢市では金融業の夫婦が殺される事件があってプロの殺し屋の犯行という見立てがありましたが,未解決で終わっているはずです。

 高岡市の暴力団組長夫婦殺害事件は,2000年7月とあります。平成12年7月といえば,安藤健次郎さんの傷害事件で金沢刑務所の拘置所にいた頃になりますが,ニュースで知ることはなく,その後に裁判などの報道で事件があったことを知ったように思います。

 ただでさえ,ニュース放送の少ない金沢刑務所でしたが,過去の記録には記述してある通り,私が平成12年11月に第5工場に出役してからは,新聞には一度の墨塗りがなく,被告発人古川龍一裁判官の妻のストーカー事件や,東京地検検事の痴漢事件も読売新聞で読んでいました。

 金沢刑務所の刑務官の万引き事件というのも新聞にそのまま閲覧できていました。ただ,入浴日だとじっくり新聞を読む時間のないことがありました。読売新聞は1紙だけだったように思いますが,それを多いときで65人ぐらいの受刑者で休憩時間に回し読みをしていました。

 福井刑務所の場合は,昼休みに新聞が紙面ごとに通路に吊るされ,夜間にも舎房に新聞が回され,じっくり読むことが出来ました。レクと呼ばれたレクレーションの時間は,就業日の入浴日以外でしたが,その時はどちらもゆっくり新聞が読め,スポーツ新聞もありました。

\begin{quote}
《引用の始まり》
\end{quote}

\begin{quote}
21日の富山地裁の無罪判決から。殺人事件で死刑を求刑されていた被告人に無罪判決というのは、それだけで大ニュースのはず。しかし、各新聞とも記事が小さいか、そもそも取り上げてもいません。

(東京新聞22日朝刊から抜粋)「組長夫婦射殺事件 元副組長に無罪」
\end{quote}

\begin{quote}
《引用の終わり》
\end{quote}

\begin{itemize}
\tightlist
\item
  指示を否認し 無罪となって 組長射殺も 薮の中 - 弁護士任官どどいつ集
  \url{https://blog.goo.ne.jp/gootest32/e/2c07960ab6c21cbf3323da3f02d0b1ed} 
\end{itemize}

 twilog-serch
で「竹内浩史」の検索結果もなかったのですが,意外なことに「どどいつ」でも結果がありませんでした。過去に見覚えのあるブログで,見つけた記事も2006年11月22日という15年近く前の日付でした。メニューのバックナンバーを見ると同年3月から始まっています。

\begin{lstlisting}
py37_env ❯ hatena-log-search どどいつ
\end{lstlisting}

\begin{itemize}
\tightlist
\item
  ./20090807: どどいつで有名という裁判官のブログを拝見していたら,次のようなことが書かれていた。
\end{itemize}

 はてなダイアリーの記録の検索,hatena-log-search
では1件だけ上記の該当がありました。

\begin{itemize}
\item
  2021年04月26日11時11分の登録:
  「どどいつ」を過去のはてなダイアリーの記事から検索
  \url{https://kk2020-09.blogspot.com/2021/04/blog-post\_26.html} 
\item
  hatena-diary\_20090807 -
  告発\金沢地方検察庁\最高検察庁\法務省\石川県警察御中
  \url{https://t.co/gQP6w1QIby} 
\item
  Entries are not found -
  銀座のマチ弁から勝どきの宅弁へ(弁護士遠藤きみのブログ) (新)
  \url{https://t.co/gyQPN9ELxQ} 
\end{itemize}

 「どどいつで有名という裁判官のブログを拝見していたら,次のようなことが書かれていた。」という部分は,記事は見つからないとなっていますが,上記の遠藤きみ弁護士の記事からの引用部分だったようです。

\begin{itemize}
\tightlist
\item
  裁判員制 あくまで阻止は 悪魔であろうと 手を組むの? -
  弁護士任官どどいつ集 \url{https://t.co/2JljX6eImz} 
\end{itemize}

 遠藤きみ弁護士に引用された記事はそのまま残っているようです。2009年7月12日の記事ですが,リンクが張ってあるので開いていたものと思います。しかし,2010年4月2日から始めたTwitterでは,「どどいつ」さえ該当がありませんでした。

 どどいつといえば,お座敷小唄に出てくる歌詞ですが,都々逸とパソコンだと簡単に変換ができてしまいます。前に意味を調べたこともあったと思いますが,幕末の流行歌のような話であったように記憶にあります。

 2021年04月26日11時29分の実行記録 ¥\n
twitterAPI-search-lawList-mydql-add.rb ``どどいつ'' ¥\n
ツイート数:12/2403 リツイート数:0/2403 トータル:142 ¥\n
hirono\_hideki 2/0件 ¥\n kk\_hirono 10/0件 ¥\n s\_hirono 0/0件

 「弁護士任官どどいつ集」というブログのタイトル名は,2006年3月から本日2021年4月26日の間に変更されている可能性というのは考えてみたのですが,まとめ記事を作成してみると,私の2つのアカウント以外に該当はなしでした。

\begin{itemize}
\tightlist
\item
  2021年04月26日11時30分の登録:
  REGEXP:''どどいつ''/データベース登録済みツイートの検索:2016-01-09〜2021-04-26/2021年04月26日11時29分の記録:ユーザ・投稿:3/13件
  \url{https://kk2020-09.blogspot.com/2021/04/regexp2016-01-092021-04.html} 
\end{itemize}

 竹内浩史裁判官について調べると,日本裁判官ネット-ワークが出てくるのですが,これも2006年には見ていたと思いますし,最近でも中村元弥弁護士のツイートで見かけていました。

\begin{quote}
《引用の始まり》
\end{quote}

\begin{quote}
さいたま地家裁川越支部判事 竹内浩史

 私のこの都々逸(どどいつ)は、東京高裁の裁判官室で,5年前のある夜に生まれた。 2003年4月に弁護士任官して民事部の陪席になったばかりの新米裁判官の私が、一緒に居残って分厚い記録を読んでいる先輩裁判官に話しかけた時である。 「やっぱりこの仕事量は大変ですね。もう少し裁判官を増やしてもらえると楽になりますのにね。」 「増やしても変わらないよ。」 「どうしてですか?」 「今、3回読んでいる記録を、念のためもう1回読み直すだけだから。」 想定外のお答に絶句し,思わず目が潤んでしまった。 確かにそう言われてみると,日本の裁判官の生真面目な気質からは、結局そうなってしまいそうな気もしてきた。 話は40年も昔に飛ぶが、私の幼い頃の体験で鮮明に記憶している一件がある。 保育園の庭で停んでいると突然2人の女の子が半ベソをかきながら,手を引っ張りあって私の前に来た。「ねえ、ひろしくん。どっちが悪いと思う?」 喧嘩の原因がどっちにあるのか,私に判定してほしいというのである。ガキ大将でもなく,ただ少し頭が良さそうで,おとなしいだけの私に。幼心になぜ私のところに来るのか訳が分からず,立ち往生するばかりだった。 任官して初めて気がついた。これが日本の裁判官の姿なのだと。 「裁判員制度」導入以前の日本の司法は、学校でガリ勉と思われているような優等生がクラスメートを裁く「優等生司法」だったと思う。このモデルで、日本の裁判官の良い点も悪い点もうまく説明がつく。 優等生の模範のような裁判官が,時間制限なしで完璧な答案(判決)を提出しようと全力を尽くせば、休暇や睡眠も削ってぎりぎりまでオーバーワークをしてしまう可能性が高い。 それが日本の裁判官の美風なのだが、度を越すのもやはり良くない。制限時間(審理期間や労働時間)を設定して厳守していく方が良いのかも知れない。そうすれば、裁判官が仕事ばかりしていて「世間知らず」になる危険も小さくなるだろう。(本稿は,日本裁判官ネットワーク通信NO1「ちょっといい話」から転載しました。)
\end{quote}

\begin{quote}
《引用の終わり》
\end{quote}

\begin{itemize}
\tightlist
\item
  日本裁判官ネットワーク
  \url{http://www.j-j-n.com/coffee/090401/090401\_hanzaifuetemo.html} 
\end{itemize}

 大事なことが書いてあるので正確に全文を引用しましたが,ページタイトルにはなっていない見出しに「●
「判事増えても楽にはならぬ 記録もう一度開くから」」とあります。

 数日前から,裁判官は記録を,少なくとも私の上申書はほとんど読んでいなかったものと考えていたのですが,本来一番の発見となったのは,このあと集中的に取り上げる,伊東一廣裁判長のことです。これが思わぬ発見となりました。それも用意されていたような今頃の時期です。

 「富山地裁の手崎政人裁判長は21日、殺人罪などに問われ、死刑を求刑された渡一家の元副組長栗原組夫被告(55)に対し、「共犯者の供述に信用性が認められない」として無罪判決を言い渡した。」と竹内浩史裁判官の記事にはあるのですが,これも見覚えのある裁判長の名前です。

 どこうがどうにているのかは書けないのですが,似ている名前の同級生がいて,高校卒業した年,珠洲の祭りに行って,その帰りに同乗していた車が崖に転落,首から下が動かない付随の状態になったと聞きます。介護をしながら親が死んでくれれば良かったと話しいているとも聞きました。

 (※)この続きは,被告発人安田敏の項目として記述をしておきたいと思うのですが,被告発人安田敏の後輩になる私の同級生なので事故のことはしっていたはず,それを意識したような発言が,金沢西警察署の面会の時にありました。

 このあと別のエントリーでの引用を予定して「(※)」という記号を2箇所に入れました。

\begin{itemize}
\tightlist
\item
  〈〈〈 2021/04/26 12:01:54 Linux Emacs: 〈〈〈
\end{itemize}

\hypertarget{ux5f01ux8b77ux58ebux60d1ux661f2021ux5e745ux67083ux65e5ux9ad8ux6a4bux96c4ux4e00ux90ceux5f01ux8b77ux58ebux306eux30bfux30a4ux30e0ux30e9ux30a4ux30f3ux30c4ux30a4ux30fcux30c8ux304bux3089ux5de1ux308aux5de1ux308bux88abux544aux767aux4ebaux5ca1ux7530ux9032ux5f01ux8b77ux58ebux306eux56fdux9078ux5f01ux8b77ux306eux72afux7f6aux6027}{%
\paragraph{弁護士惑星,2021年5月3日高橋雄一郎弁護士のタイムライン,ツイートから巡り巡る被告発人岡田進弁護士の国選弁護の犯罪性}\label{ux5f01ux8b77ux58ebux60d1ux661f2021ux5e745ux67083ux65e5ux9ad8ux6a4bux96c4ux4e00ux90ceux5f01ux8b77ux58ebux306eux30bfux30a4ux30e0ux30e9ux30a4ux30f3ux30c4ux30a4ux30fcux30c8ux304bux3089ux5de1ux308aux5de1ux308bux88abux544aux767aux4ebaux5ca1ux7530ux9032ux5f01ux8b77ux58ebux306eux56fdux9078ux5f01ux8b77ux306eux72afux7f6aux6027}}

\begin{itemize}
\tightlist
\item
  〉〉〉 Linux Emacs: 2021/05/03 14:58:30 〉〉〉
\end{itemize}

:CATEGORIES: @kanazawabengosi \#金沢弁護士会 @JFBAsns
日本弁護士連合会(日弁連) \#法務省 @MOJ\_HOUMU \#高橋雄一郎弁護士
\#被告発人岡田進弁護士

 記憶のままに「星めぐりの歌」というのを思い出したのですが,銀河鉄道の夜だったと思います。

\begin{itemize}
\tightlist
\item
  1340:2021-05-03\_12:14:02 \#告発状 \#\#\#\#
  2021年4月27日から5月1日にかけ視聴したAmazonプライムビデオの「銀河鉄道999」シーズン1,エピソード(113)
  \url{https://hirono-hideki.hatenadiary.jp/entry/2021/05/03/121400} 
\end{itemize}

 エントリーナンバー1340となっていますが,買い物に出掛ける前に投稿したものになります。ある店の前に,いつもより待っている人の数が多いと思ったのですが,ゴールデンウィーク中であることに気がつくまで1分程時間が掛かりました。

 店の辺りは昭和50年代の中頃,被告発人大網健二の家の自動車工場の駐車場と出入り口のようになっていたのですが,ずいぶん変わったものだと改めて思いました。エントリーナンバー1340では,その時代,テレビで銀河鉄道999が放送していた頃の述懐も記述をしています。

 振り返ると,その当時の被告発人大網健二も銀河鉄道999の主人公,星野鉄郎のように目的意識が強かったのですが,私立の金沢高校に進学し,金沢市内のアパートで一人暮らしをしていました。大和町のアパートには泊りがけで遊びに行っているのですが,玉鉾のアパートは話だけ聞いていました。

 Aコープ能都店で弁当などを買った後,連休中なので定休日の月曜日でもやっているかもしれないと思い,コンセールのとの宇出津図書館に立ち寄ったのですが,やはり休みで,明日と明後日が臨時の開館日とカレンダーに書いてありました。

\begin{itemize}
\item
  2015-12-22\_11.05.06.jpg \url{https://t.co/xKS0YHdkKt} 
\item
  2021-05-03\_130008_コンセールのと.jpg \url{https://t.co/arbcYplLhb} 
\item
  2021-05-03\_130015_コンセールのと.jpg \url{https://t.co/7RfNvY9RQ2} 
\item
  2021-05-03\_130022_コンセールのと.jpg \url{https://t.co/ofwLRVsNje} 
\item
  2021-05-03\_130744_Aコープ能都店で買ってきた弁当.jpg
  \url{https://t.co/zmavKd8Fcg} 
\end{itemize}

 for i in \texttt{ls\ *.jpg}; do ks ``\$i''; sleep 2;
done というコマンドの実行でTwitterに写真ファイルをアップロードしたのですが,最初に実行したフォルダが違っていて,そこに1つだけ写真ファイルがあったことと,内容をビューアで開いてから知りました。

\begin{lstlisting}
py37_env ❯ ls
\end{lstlisting}

2015-12-22\_11.05.06.jpg 2020-11-15.md Googleフォト・アップロード待ち
iphone mov p q work x2020-03-16\_法務大臣 非常上告職権発動要請書.docx
2021-02-01\_蛸島学童殺人事件裁判記録 dropbox-photo iphone7 o png tv
work2

 フォルダの内容です。1つだけjpgの写真ファイルが置いてありますが,いつ置いたのか記憶になく,内容も知らずにいました。新聞の紙面の撮影ですが,床の上に置いてあるように見え,そういう撮影をやった憶えはなかったので,同じ2015年12月22日の写真を調べました。次が前後のファイルです。

\begin{itemize}
\item
  2015-12-22\_10.10.15.jpg \url{https://t.co/bUMkg5QdvU} 
\item
  2015-12-22\_11.05.13.jpg \url{https://t.co/flZMKyFToe} 
\item
  2015-12-22\_11.05.20.jpg \url{https://t.co/kJJzLu12BM} 
\item
  2015-12-22\_13.15.19.jpg \url{https://t.co/QKjpeezzRz} 
\item
  2015-12-22\_16.11.26.jpg \url{https://t.co/D4OHU6t2f3} 
\end{itemize}

 午前10時10分に国重と源平の間の陸橋の上,13時15分に小木港での東一文字堤防に向かう渡船の上での写真がありました。能登町上町の母親の入院している病院での新聞紙面の撮影だったようです。木目の机があったとも記憶にないですが,床に置いて撮影するようなことはなかったはずです。

 新聞紙面の内容はよく憶えていた内容ですが,38年前の事故の教訓とあり,忘れていたのですが「交通安全を願う銅像を眺める児童」という写真が掲載されています。最近似たような銅像の写真を見たと思ったのですが,鉄郎とメーテルの写真でした。

 内灘町に自転車競技場があることは,今初めて知ったと思ったのですが,それらしい場所というのは見当もつきません。海と河北潟に繋がる川に挟まれた丘陵地というのが内灘のイメージです。

 昭和52年8月31日の夏休み最後の日に,小学3年生の男児が自転車事故で亡くなったとあります。私が能都中学校の1年生の夏休みになりますが,2015年12月21日の北國新聞に,「38年前の事故教訓に」というのは,時の流れをあらためて感じる機会となりました。

 自転車事故といえば,次のエントリーでも被告発人岡田進弁護士を中心に取り上げていました。

\begin{itemize}
\tightlist
\item
  1333:2021-04-26\_07:29:01 \#告発状 \#\#\#\#
  被告発人岡田進弁護士が代理人となり過失割合が45%になった,金沢市内で1級1号後遺障害を残した13歳女子の自転車事故
  \url{https://hirono-hideki.hatenadiary.jp/entry/2021/04/26/072859} 
\item
  1334:2021-04-26\_08:43:34 \#告発状 \#\#\#\#
  4月24日と25日の両日,宇出津図書館で調べた平成15年の「金沢市内で1級1号後遺障害を残した13歳女子の自転車事故」,岡田進弁護士が代理人で過失相殺45%
  \url{https://hirono-hideki.hatenadiary.jp/entry/2021/04/26/084332} 
\end{itemize}

 Bloggerとは違い控えめに作成してきたつもりのはてなのブログもエントリーが現時点で1340となっています。Bloggerのエントリーはデータベースの記録が現時点で70738となっています。

 Amazonプライムビデオの銀河鉄道999はシーズン1で113話あり,1話の視聴時間が約25分間となっていました。銀河鉄道999の視聴の途中で気がついたのですが,YouTubeとは違い,再生の残り時間がカウントダウンのように表示されていました。

 買い物から戻り,買ってきたAコープ能都店の弁当を食べ始めたのが,13時07分頃となっていました。その少しあとになるのではと思うのですが,高橋雄一郎弁護士のタイムラインをスクリーンショットで記録していると思います。

\begin{itemize}
\item
  2021-05-03-122609\_都 行志/Miyako Koji@Miyako\_Koji·21時間同期の橘真理夫先生から著書「弁護士の現場力」を頂きました。実践的な情報満載.jpg
  \url{https://t.co/PuNnGP4dxt} 
\item
  2021-05-03-133853\_高橋雄一郎さんがリツイートピピピーッ@O59K2dPQH59QEJx·1時間それなのにツイッターランド入り。ようこそ、お待ちしておりました.jpg
  \url{https://t.co/CDGTSaMsBH} 
\item
  2021-05-03-134341\_高橋雄一郎@kamatatylaw·1時間「痴漢」でgoogle検索したら上位に防犯グッズとか警察の被害対策ページが出てきて、アダルトサイト.jpg
  \url{https://t.co/WNIJKdgICm} 
\item
  2021-05-03-134639\_高橋雄一郎@kamatatylaw笑->w->草生える みたいな言い換えのネットスラングと同じような 接見->石鹸->ソープ という言い換え.jpg
  \url{https://t.co/wncc5qe6t7} 
\item
  2021-05-03-134742\_高橋雄一郎さんがリツイートおかまつさん@relaxpop·15時間休日に接見することを、休日ソープと呼んでいるツイートに接したが、あまりに.jpg
  \url{https://t.co/14GJwTeTRS} 
\item
  2021-05-03-135052\_高橋雄一郎さんがリツイート井垣孝之@igaki·15時間弁護士の課税所得3000万という数字を見かけますが、個人で課税所得が3000万もあ.jpg
  \url{https://t.co/VhZQkYMC7S} 
\item
  2021-05-03-140745\_高橋雄一郎@kamatatylaw·4月30日直接弁護士に依頼するのではなく,事業会社を経由して提携弁護士に依頼する形になるので,どうしても.jpg
  \url{https://t.co/jT0ekfGo1V} 
\end{itemize}

 リツイートとして井垣孝之‏弁護士のツイートも高橋雄一郎弁護士のタイムラインに出てきたのですが,けっこう久しぶりに見たのと,このタイミングで,思い出した弁護士の志のような記事のことを思い出しました。弁護士鉄道の志のようなエピローグでした。

 次が,その高橋雄一郎弁護士のタイムラインでみかけたツイートになります。リツイートは失敗すると思いますが,繰り返しもしやすいので好都合でもあります。

〉〉〉 kk\_hironoのリツイート 〉〉〉

\begin{itemize}
\item
  RT
  kk\_hirono(刑事告発・非常上告_金沢地方検察庁御中)|relaxpop(おかまつさん)
  日時:2021-05-03 16:27/2021/05/02 22:43 URL:
  \url{https://twitter.com/kk\_hirono/status/1389119405846503429} 
  \url{https://twitter.com/relaxpop/status/1388851659393015815} 
  \textgreater{}
  休日に接見することを、休日ソープと呼んでいるツイートに接したが、あまりに下品だし、被疑者を馬鹿にしてない?信じがたいのだが。
\item
  〉〉〉 アカウント(@kamatatylaw)は,@kk\_hironoをブロックしています。リツイートできませんでした。
  〉〉〉 ¥\n ¥\n \url{https://t.co/xBVp9uD1Ue} 
\item
  〉〉〉 アカウント(@kamatatylaw)は,@kk\_hironoをブロックしています。リツイートできませんでした。
  〉〉〉 ¥\n ¥\n \url{https://t.co/kZ2oOuY0Me} 
\item
  〉〉〉 アカウント(@kamatatylaw)は,@kk\_hironoをブロックしています。リツイートできませんでした。
  〉〉〉 ¥\n ¥\n \url{https://t.co/lyXhIryhay} 
\item
  〉〉〉 アカウント(@kamatatylaw)は,@kk\_hironoをブロックしています。リツイートできませんでした。
  〉〉〉 ¥\n ¥\n \url{https://t.co/NEGvChBfrR} 
\item
  〉〉〉 アカウント(@igaki)は,@kk\_hironoをブロックしています。リツイートできませんでした。
  〉〉〉 ¥\n ¥\n \url{https://t.co/yOAMaIYBND} 
\item
  〉〉〉 アカウント(@kamatatylaw)は,@kk\_hironoをブロックしています。リツイートできませんでした。
  〉〉〉 ¥\n ¥\n \url{https://t.co/KNx7TCafW0} 
\item
  〉〉〉 アカウント(@kamatatylaw)は,@kk\_hironoをブロックしています。リツイートできませんでした。
  〉〉〉 ¥\n ¥\n \url{https://t.co/Kqm5kgg2Dz} 
\item
  〉〉〉 アカウント(@kame\_ishi)は,@kk\_hironoをブロックしています。リツイートできませんでした。
  〉〉〉 ¥\n ¥\n \url{https://t.co/MDufIlY6PL} 
\item
  〉〉〉 アカウント(@kamatatylaw)は,@kk\_hironoをブロックしています。リツイートできませんでした。
  〉〉〉 ¥\n ¥\n \url{https://t.co/3iEmKipokq} 
\item
  〉〉〉 アカウント(@kamatatylaw)は,@kk\_hironoをブロックしています。リツイートできませんでした。
  〉〉〉 ¥\n ¥\n \url{https://t.co/IlHJZ2cp1q} 
\item
  〉〉〉 アカウント(@kamatatylaw)は,@kk\_hironoをブロックしています。リツイートできませんでした。
  〉〉〉 ¥\n ¥\n \url{https://t.co/rJfX0kPNRb} 
\end{itemize}

※ @kk\_hironoのアカウントがブロックされ,リツイートに失敗したツイート

\begin{itemize}
\tightlist
\item
  TW kamatatylaw(高橋雄一郎) 日時:2021/05/03 12:08:20 URL:
  \url{https://twitter.com/kamatatylaw/status/1389054186159951872} 
  \textgreater{}
  「痴漢」でgoogle検索したら上位に防犯グッズとか警察の被害対策ページが出てきて、アダルトサイトなんか全然表示されなかったが、俺の検索エンジンの問題だろうか。
  \url{https://t.co/hsxBJgOvLZ} 
\end{itemize}

※ @kk\_hironoのアカウントがブロックされ,リツイートに失敗したツイート

\begin{itemize}
\tightlist
\item
  TW kamatatylaw(高橋雄一郎) 日時:2021/05/03 10:10:17 URL:
  \url{https://twitter.com/kamatatylaw/status/1389024480664510465} 
  \textgreater{} 笑-\textgreater w-\textgreater 草生える
  みたいな言い換えのネットスラングと同じような
  接見-\textgreater 石鹸-\textgreater ソープ
  という言い換えでしょう。これを下品に感じる人は風俗を連想しちゃったからだろうけど,弁護士で風俗に行く人はほとんどいないから変な連想をするほうがやばいと思うよ。
\end{itemize}

※ @kk\_hironoのアカウントがブロックされ,リツイートに失敗したツイート

\begin{itemize}
\tightlist
\item
  TW kamatatylaw(高橋雄一郎) 日時:2021/05/03 06:50:27 URL:
  \url{https://twitter.com/kamatatylaw/status/1388974188963143683} 
  \textgreater{}
  弁護士になったばかりの頃は,コンビニのレジの横にレシートがたくさん捨ててあるのをみて,アレを鷲掴みにして持って帰りたくなっただろう?あれから幾歳がすぎ,無数の事件をこなし,判例時報にも何度か掲載された。もはやコンビニのレシートなんかに胸がときめかなくなったじゃないか。
\end{itemize}

※ @kk\_hironoのアカウントがブロックされ,リツイートに失敗したツイート

\begin{itemize}
\tightlist
\item
  TW kamatatylaw(高橋雄一郎) 日時:2021/05/03 06:42:12 URL:
  \url{https://twitter.com/kamatatylaw/status/1388972112333791233} 
  \textgreater{}
  弁護士の課税所得が3000万円ぐらいだと法人に売上を移して節税とか魅力的にみえちゃうけど,個人の課税所得が億超えだと法人に移すのにも限界があるし使うのにも制約があるかさらに課税されるし,どっちも大差ないなら納税したらいいじゃんっていう気持ちになるって,マックで女子高生が話してました。
\end{itemize}

※ @kk\_hironoのアカウントがブロックされ,リツイートに失敗したツイート

\begin{itemize}
\tightlist
\item
  TW igaki(井垣孝之) 日時:2021/05/02 22:21:22 URL:
  \url{https://twitter.com/igaki/status/1388846076724080641} 
  \textgreater{}
  弁護士の課税所得3000万という数字を見かけますが、個人で課税所得が3000万もあると、所得税+住民税+個人事業税で約1300万なのに対し、個人の課税所得が1500万、法人の課税所得が1500万だと、それぞれ530万+460万になり、約300万の税負担の差があります。たくさん納めてくれて感謝、という感じですね。
\end{itemize}

※ @kk\_hironoのアカウントがブロックされ,リツイートに失敗したツイート

\begin{itemize}
\tightlist
\item
  TW kamatatylaw(高橋雄一郎) 日時:2021/05/02 08:09:22 URL:
  \url{https://twitter.com/kamatatylaw/status/1388631661705777155} 
  \textgreater{}
  オリンピックのためにボランティアの弁護士が必要だって無償で集めようとしてますよ。そして,語学力のある志のある弁護士が相当数応募し,時間をさいて言われるままに「研修」を受けてるんですよ。オリンピックなら責任感と義務感と奉仕感を無償で動員できるというのはどこも同じことね。
  \url{https://t.co/6kkkM9mlAU} 
\end{itemize}

※ @kk\_hironoのアカウントがブロックされ,リツイートに失敗したツイート

\begin{itemize}
\tightlist
\item
  TW kamatatylaw(高橋雄一郎) 日時:2021/04/30 22:49:47 URL:
  \url{https://twitter.com/kamatatylaw/status/1388128448695001088} 
  \textgreater{}
  直接弁護士に依頼するのではなく,事業会社を経由して提携弁護士に依頼する形になるので,どうしてもえげつない中抜きが発生しちゃうよね。養育費から30\%も抜くのって,関係者は胸が傷まないんだろうかな。ビジネスだから傷まないんだろうね。
\end{itemize}

※ @kk\_hironoのアカウントがブロックされ,リツイートに失敗したツイート

\begin{itemize}
\tightlist
\item
  TW kame\_ishi(かめもち) 日時:2021/04/30 22:41:10 URL:
  \url{https://twitter.com/kame\_ishi/status/1388126282919993345} 
  \textgreater{} くそ暴利やんけ!\\
  \textgreater{}
  生活費の一部である養育費から30%も抜かれたら生活できんやん!
  \url{https://t.co/GuM0Dg0W91} 
\end{itemize}

※ @kk\_hironoのアカウントがブロックされ,リツイートに失敗したツイート

\begin{itemize}
\tightlist
\item
  TW kamatatylaw(高橋雄一郎) 日時:2021/04/30 22:44:02 URL:
  \url{https://twitter.com/kamatatylaw/status/1388127003979620352} 
  \textgreater{}
  木村草太「平等なき平等条項論」(東京大学出版会)はなかなかいい論文だよ。説得力を高めるための表現上の工夫があざとくて,13年前に購入して,この論文のスタイルを真似して準備書面を書いているんだ,俺。
\end{itemize}

※ @kk\_hironoのアカウントがブロックされ,リツイートに失敗したツイート

\begin{itemize}
\tightlist
\item
  TW kamatatylaw(高橋雄一郎) 日時:2021/04/30 22:40:38 URL:
  \url{https://twitter.com/kamatatylaw/status/1388126149910286337} 
  \textgreater{}
  木村先生のご意見にすべて賛成するわけではないが,これはよくないね
  -\textgreater{}
  憲法記念日の講演に憲法学者・木村草太さんの起用NG 鎌倉市が「9条に言及する懸念」で拒否
  \url{https://t.co/pEmSUsljLJ} 
\end{itemize}

※ @kk\_hironoのアカウントがブロックされ,リツイートに失敗したツイート

\begin{itemize}
\tightlist
\item
  TW kamatatylaw(高橋雄一郎) 日時:2021/04/30 17:17:14 URL:
  \url{https://twitter.com/kamatatylaw/status/1388044763362783232} 
  \textgreater{}
  小さな一歩のいちばんよくないところは,困っているひとり親に妙な期待を持たせて,さんざん待たせたあげく,とても同意できないような条件を突如として提示するところだよね。困っている人の貴重な時間と労力を奪うところが本当によくない。元パートナーはほくそ笑んでいる。
\end{itemize}

 記録作成の肝,メインとなるのは,「接見ー>石鹸ー>ソープ
という言い換えでしょう。これを下品に感じる人は風俗を連想しちゃったからだろうけど,弁護士で風俗に行く人はほとんどいないから変な連想をするほうがやばいと思うよ。」という内容のツイートになります。

\begin{itemize}
\tightlist
\item
  〈〈〈 2021/05/03 16:41:05 Linux Emacs: 〈〈〈
\end{itemize}

\hypertarget{ux5f01ux8b77ux58ebux60d1ux661f2021ux5e745ux67083ux65e5ux306eux9ad8ux6a4bux96c4ux4e00ux90ceux5f01ux8b77ux58ebux306eux30bfux30a4ux30e0ux30e9ux30a4ux30f3ux30c4ux30a4ux30fcux30c8ux304bux3089ux5de1ux308aux5de1ux308bux88abux544aux767aux4ebaux5ca1ux7530ux9032ux5f01ux8b77ux58ebux306eux56fdux9078ux5f01ux8b77ux306eux72afux7f6aux6027ux5ca1ux6751ux9686ux53f2ux6c0fux306eux98a8ux4fd7ux708eux4e0aux767aux8a00}{%
\paragraph{弁護士惑星,2021年5月3日の高橋雄一郎弁護士のタイムライン,ツイートから巡り巡る被告発人岡田進弁護士の国選弁護の犯罪性:岡村隆史氏の風俗炎上発言}\label{ux5f01ux8b77ux58ebux60d1ux661f2021ux5e745ux67083ux65e5ux306eux9ad8ux6a4bux96c4ux4e00ux90ceux5f01ux8b77ux58ebux306eux30bfux30a4ux30e0ux30e9ux30a4ux30f3ux30c4ux30a4ux30fcux30c8ux304bux3089ux5de1ux308aux5de1ux308bux88abux544aux767aux4ebaux5ca1ux7530ux9032ux5f01ux8b77ux58ebux306eux56fdux9078ux5f01ux8b77ux306eux72afux7f6aux6027ux5ca1ux6751ux9686ux53f2ux6c0fux306eux98a8ux4fd7ux708eux4e0aux767aux8a00}}

\begin{itemize}
\tightlist
\item
  〉〉〉 Linux Emacs: 2021/05/03 16:45:32 〉〉〉
\end{itemize}

:CATEGORIES: @kanazawabengosi \#金沢弁護士会 @JFBAsns
日本弁護士連合会(日弁連) \#法務省 @MOJ\_HOUMU \#高橋雄一郎弁護士

\begin{itemize}
\tightlist
\item
  1341:2021-05-03\_16:41:53 \#告発状 \#\#\#\#
  弁護士惑星,2021年5月3日高橋雄一郎弁護士のタイムライン,ツイートから巡り巡る被告発人岡田進弁護士の国選弁護の犯罪性
  \url{https://hirono-hideki.hatenadiary.jp/entry/2021/05/03/164151} 
\end{itemize}

 エントリー1341になりますが,プロローグからのエピソードに近い内容になるかもしれません。次のように検索をしたところ該当は1件だけでしたが,他の弁護士のツイートや発言より,なぜか高橋雄一郎弁護士のことが印象に強く関連付けられた記憶となっています。その辺りを探っていきます。

\begin{lstlisting}
py37_env ❯ d|grep 高橋雄一郎|grep 岡村
\end{lstlisting}

\begin{itemize}
\tightlist
\item
  2020年04月29日19時24分の登録:
  \高橋雄一郎 @kamatatylaw\岡村氏の「女子が困窮して性風俗で稼働せざるを得なくなるのが楽しみ」発言だけど,就職難の修習生を買い叩くのが楽しみなボス弁とどこが違
  \url{http://hirono2014sk.blogspot.com/2020/04/kamatatylaw\_29.html} 
\end{itemize}

4件目 戻る ツイート: kamatatylaw(高橋雄一郎) 日時: 2020-04-29 12:03
URL:
\url{https://twitter.com/kamatatylaw/status/1255331805386981376\textgreater} {}
岡村氏の「女子が困窮して性風俗で稼働せざるを得なくなるのが楽しみ」発言だけど,就職難の修習生を買い叩くのが楽しみなボス弁とどこが違うか。搾取するために経済的弱者が生まれるのを期待するという資本主義社会の鉄則に沿っているのは同じだが,前者は「性」が絡むから批判されるのだろう。

 そういえば,Amazonプライムビデオの銀河鉄道999では,いくつかの場面でメーテルが鉄郎に宇宙の厳しさを説く話がありました。高橋雄一郎弁護士はTwitterのプロフィールに「ほぼ特許事件だけで生活しています。」というだけ,技術系で専門性が高く,成功しているように見えます。

 弁護士として成功しているだけにゆとりもあるのか,Twitterでは他分野に関連したようなツイートが多く,同業者である弁護士が成功をして業界を盛り上げていくことに関心があるようです。「勝訴」という大吟醸酒を郵送で振る舞うことが象徴的と思えるところです。

 つい最近もいくつか見かけているのですが,「勝訴」をキーワードにした全体のまとめ記事はまだ作成したことがなかったかもしれません。確認をしてから実行します。

\begin{lstlisting}
py37_env ❯ d|grep 勝訴
\end{lstlisting}

\begin{itemize}
\tightlist
\item
  2014年01月10日16時08分の登録:
  勝訴したら希望の裁判所、敗訴したら絶望の裁判所、という本じゃないだろう。¥\n笑/落合洋司弁護士
  \url{http://hirono2014sk.blogspot.com/2014/01/blog-post\_430.html} 
\item
  2017年10月05日23時20分の登録: %@Seiryu\_Miwako せいりゅう
  みわこ%8月8日に得た勝訴判決。声明を事務所のブログに掲載しました。
  \url{http://hirono2014sk.blogspot.com/2017/10/seiryumiwako-88.html} 
\item
  2017年10月10日21時53分の登録:
  \中村元弥 @1961kumachin RT: @Motomitsu\_N\生業訴訟、国と東電に勝訴!
  \url{http://hirono2014sk.blogspot.com/2017/10/1961kumachinrtmotomitsun.html} 
\item
  2017年10月24日11時19分の登録:
  \浜木綿弁窯衛門 @leplusallez\未払残業代金訴訟で、残業代1250万円、付加金1250万円の勝訴で、法テラスから報酬金84000円(税込み)っ
  \url{http://hirono2014sk.blogspot.com/2017/10/leplusallez\_24.html} 
\item
  2017年10月26日04時11分の登録:
  \ツンデレブログ @tsundereblog\未払残業代金訴訟。残業代1250万円、付加金1250万円の勝訴で、法テラスから報酬金84000円(税込み)を獲得しました
  \url{http://hirono2014sk.blogspot.com/2017/10/tsundereblog\_26.html} 
\item
  2017年11月22日03時53分の登録:
  \高島章(弁護士) @BarlKarth\マスメディアの皆さんが注目している「びろーん」ですが、あらゆる事態に備えて、「逆転全面勝訴」「逆転不当判決」「9名中○名認定(1
  \url{http://hirono2014sk.blogspot.com/2017/11/barlkarth\_22.html} 
\item
  2017年12月03日02時47分の登録:
  \高島章(弁護士) @BarlKarth\NHKローカルニュース 新潟水俣病行政認定訴訟〔原告9名全員勝訴)、被告新潟市は上告しない方針。
  \url{http://hirono2014sk.blogspot.com/2017/12/barlkarthnhk9.html} 
\item
  2017年12月06日19時38分の登録:
  \北白川 @GUv4i6\戒告と業務停止の間に「各地の刑務所で無料相談を受けて、国賠で10万円以上の勝訴判決をもらえるまでかえれまテン」の設置を認める嘆願書
  \url{http://hirono2014sk.blogspot.com/2017/12/guv4i610.html} 
\item
  2017年12月06日21時35分の登録:
  \高橋雄一郎 @kamatatylaw\今夜あたり,超高額事件で大勝利して巨額の成功報酬を獲得された某弁護士が「純米大吟醸・勝訴」を飲まれることだろう。http://
  \url{http://hirono2014sk.blogspot.com/2017/12/kamatatylawhttp.html} 
\item
  2017年12月14日05時28分の登録: \村松
  謙 @kmuramatsu\「有罪率99.9\%」の刑事裁判で無罪連発 ``勝訴請負人``弁護士の信念とは(AbemaTIMES)-
  Yahoo!ニュース /
  \url{http://hirono2014sk.blogspot.com/2017/12/kmuramatsu999abematimes-yahoo.html} 
\item
  2017年12月22日09時38分の登録:
  \HRK @hKodama\うまく書けた自分の書面を後で見返すときの快感もさることながら、勝訴判決の理由中(当裁判所の判断)に自分の書面で特にイケてた部分がまるっきり採用
  \url{http://hirono2014sk.blogspot.com/2017/12/hrkhkodama\_22.html} 
\item
  2018年01月18日14時45分の登録:
  \ワーキングプア弁護士 @sokudoku\_65\みんなが思い描くいい弁護士って、金にならない事件を嫌がりもせず受けてくれて、それでも諦めず闘って勝訴してくれて、しまいに
  \url{http://hirono2014sk.blogspot.com/2018/01/sokudoku65.html} 
\item
  2018年03月15日10時20分の登録:
  \とりろーる @triroll33\もう8年ぐらい前の話。法テラスで受任した言いがかり訴訟で原審勝訴して控訴され(高裁までは特急で往復12,000円)た件で、法テラスから
  \url{http://hirono2014sk.blogspot.com/2018/03/triroll33812000.html} 
\item
  2018年03月30日22時17分の登録:
  \高島章(弁護士) @BarlKarth\最高新潟水俣病訴訟、上告するかどうかは、近日中に公表するが、勝訴判決が出たとしても破棄自判はなく高裁差戻だろう。弁護団長が生きて
  \url{http://hirono2014sk.blogspot.com/2018/03/barlkarth\_30.html} 
\item
  2018年09月09日01時48分の登録:
  \弁護士鎌田幸夫 @yukiokamada\建設アスベスト訴訟大阪高裁判決京都ルート、国の責任は9度目の責任を認め、一人親方に対する責任、企業10社の責任を認める全面勝訴
  \url{http://hirono2014sk.blogspot.com/2018/09/yukiokamada.html} 
\item
  2018年10月20日19時56分の登録:
  \高島章(弁護士) @BarlKarth\私は、新潟水俣病行政事件で岡口裁判官が主任となり(秘密でも何でもありません)、ひょっとすると逆転全面勝訴もあるなと思いました。岡
  \url{http://hirono2014sk.blogspot.com/2018/10/barlkarth.html} 
\item
  2018年10月30日07時18分の登録: \落合洋司??Yoji Ochiai
  承詔必謹 @yjochi\一種のカントリーリスクだが、日本企業の韓国からの撤退もやむをえんかも。→韓国最高裁で「元徴用工」勝訴濃厚 
  \url{http://hirono2014sk.blogspot.com/2018/10/yoji-ochiai-yjochi\_30.html} 
\item
  2018年11月01日19時04分の登録:
  \渡辺輝人 @nabeteru1Q78\国民からの国賠請求はほとんど棄却してますね。神戸地裁で勝訴し、そこから政治解決に向かった中国残留孤児の例などはありますが。
  \url{http://hirono2014sk.blogspot.com/2018/11/nabeteru1q78.html} 
\item
  2018年11月05日21時54分の登録:
  \弁護士 戸舘圭之 @todateyoshiyuki\キッザニアの裁判所のメニューに弁護士の仕事として勝訴のときの「旗出し」が含まれていて自分も「やってみたい!!」と思っ
  \url{http://hirono2014sk.blogspot.com/2018/11/todateyoshiyuki.html} 
\item
  2018年12月18日23時11分の登録:
  \ささきりょう @ssk\_ryo\年末にとてもいいものをもらった。来年は勝訴ラッシュになるだろう。
  \url{http://hirono2014sk.blogspot.com/2018/12/sskryo\_18.html} 
\item
  2018年12月19日22時33分の登録:
  \た @bato\_saretai\今年は簡裁で敗訴判決食らって地裁で逆転勝訴したのが何件かあるから、今後の簡裁判事の事実認定教育のために地裁の判決文を送りつけようかと思っ
  \url{http://hirono2014sk.blogspot.com/2018/12/batosaretai.html} 
\item
  2019年01月11日23時14分の登録:
  \ぽぽひと@常時発動型煽りスキル持ち @popohito\裁判やったらどう考えても相談者の勝訴になるのに相談者がやたら不安がってる場合でも「絶対勝てる」と弁護士は相談に言
  \url{http://hirono2014sk.blogspot.com/2019/01/popohito.html} 
\item
  2019年01月30日16時43分の登録:
  \そらまめ @sollamame\勝訴したのは我々が裁判所の和解案に応じず最後までファイティングポーズを崩さなかったからだとか何とか言って1円も成功報酬を払ってくれなかっ
  \url{http://hirono2014sk.blogspot.com/2019/01/sollamame.html} 
\item
  2019年02月20日22時37分の登録: \小笠原
  淳 @ogasawarajun\秋田の弁護士刺殺事件で、遺族が逆転勝訴の報。¥\n県警は上告するのだろうか。
  \url{http://hirono2014sk.blogspot.com/2019/02/ogasawarajun.html} 
\item
  2019年02月20日22時38分の登録: \紀藤正樹
  MasakiKito @masaki\_kito\速報=津谷弁護士刺殺事件で秋田県警を訴えていた国賠事件で、本日午後2時、高裁で逆転勝訴です。津谷さんは、殺害当
  \url{http://hirono2014sk.blogspot.com/2019/02/masakikitomasakikito2.html} 
\item
  2019年02月27日18時16分の登録:
  \弁護士 野田隼人 @nodahayato\大津で2年くらいやっていた訴訟で全部勝訴。一安心。
  \url{http://hirono2014sk.blogspot.com/2019/02/nodahayato2.html} 
\item
  2019年03月04日23時42分の登録: \高橋雄一郎 @kamatatylaw\古酒・勝訴
  \url{http://hirono2014sk.blogspot.com/2019/03/kamatatylaw.html} 
\item
  2019年03月06日17時38分の登録:
  \日本弁護士連合会(日弁連)広報キャラクター「ジャフバ」 @JFBAkouhou\勝訴判決をもらっても、相手が判決内容にすんなり従うとは限らないフバ。従わない場合には、強
  \url{http://hirono2014sk.blogspot.com/2019/03/jfbakouhou\_6.html} 
\item
  2019年03月18日18時31分の登録:
  \おっぴ44 @oppi4444\お金を請求する裁判で勝訴しても、裁判所が相手から回収してくれるわけではない。
  \url{http://hirono2014sk.blogspot.com/2019/03/44oppi4444.html} 
\item
  2019年03月24日22時42分の登録: \弁護士山下敏雅 :
  子どもの法律ブログ @children\_ymlaw\判決前の被告のギブアップは,勝訴判決以上の,原告側の圧倒的な勝利を意味します。
  もし判決であれば
  \url{http://hirono2014sk.blogspot.com/2019/03/childrenymlaw.html} 
\item
  2019年03月28日20時26分の登録:
  \町村泰貴 @matimura\たくさん訴訟を提起し、勝訴判決をブログに載せている某氏、敗訴した判決は載せないのか。連戦連勝に見えることが大事なのか。
  それはそれで合理的
  \url{http://hirono2014sk.blogspot.com/2019/03/matimura\_28.html} 
\item
  2019年03月29日09時37分の登録:
  \津村啓介 @Tsumura\_Keisuke\ある高裁で勝訴の当事者を、同じ高裁の別の判事が横からツイッターでディスっちゃった(当事者がそう受け止めた)わけですよね?
  \url{http://hirono2014sk.blogspot.com/2019/03/tsumurakeisuke.html} 
\item
  2019年03月29日12時22分の登録: \S\_INUKAI @shunya1011\これきた!
  接見国賠、高裁で逆転勝訴! 内容もかなり濃いぜ!
  \url{http://hirono2014sk.blogspot.com/2019/03/sinukaishunya1011.html} 
\item
  2019年04月08日21時56分の登録:
  \深澤諭史 @fukazawas\勝訴判決の直後にマスコミから控訴するのかと聞かれる者。¥\nあっちに聞けよ。¥\n(・∀・)¥\n¥\n
  \url{http://hirono2014sk.blogspot.com/2019/04/fukazawas\_84.html} 
\item
  2019年04月13日11時43分の登録:
  \泥濘大魔王サイケ @k\_sawmen\不当懲戒請求受けた人はどんどん訴訟したらいいと思う。先駆者たちに勝訴判決が出たんだし乗っかったらいいと思う。「勝ち馬に乗るのはださ
  \url{http://hirono2014sk.blogspot.com/2019/04/ksawmen\_46.html} 
\item
  2019年04月29日04時57分の登録:
  \えきなんローヤー? @ekinan\_lawyer\法廷とか接客は、スーツだわー。¥\n¥\n裁判所や弁護士会に書類提出とかは、勝訴Tシャツとかだけどさ・・・
  \url{http://hirono2014sk.blogspot.com/2019/04/ekinanlawyer-t.html} 
\item
  2019年05月03日23時04分の登録:
  \嶋﨑量(弁護士) @shima\_chikara\勝手に「つまづき」にされても全額認容勝訴判決です・・・とりあえず意味不明過ぎ。誰が何を偽造したのか、解説してほしい。¥\n→ 「
  \url{http://hirono2014sk.blogspot.com/2019/05/shimachikara.html} 
\item
  2019年05月10日14時36分の登録:
  \嶋﨑量(弁護士) @shima\_chikara\本日、原告私の \#不当懲戒請求
  訴訟の判決が2つ(被告1名+5名の合計6名)ありました。@横浜地裁¥\nまた勝訴!これで勝訴
  \url{http://hirono2014sk.blogspot.com/2019/05/shimachikara-2156.html} 
\item
  2019年05月10日22時24分の登録:
  \ヨコチン刑事㊙実況 @yokotindeka\_DJ\いくら勝訴や和解が多いB型肝炎訴訟でも、「確実」や「100\%」という表現は、裁判という性質上、使ってはならん。¥\n¥\nあ
  \url{http://hirono2014sk.blogspot.com/2019/05/yokotindekadjb100.html} 
\item
  2019年05月28日22時51分の登録: \YOKO IDE
  井手洋子 @ideyou\布川国賠訴訟地裁判決勝訴。記者会見2。弁護団長。県と国の責任を認めた。誘導尋問の違法性も認めた。健全な判断。起訴の違法は認め
  \url{http://hirono2014sk.blogspot.com/2019/05/yoko-ide-ideyou2.html} 
\item
  2019年06月29日16時34分の登録:
  \毎日新聞熊本支局 @mai\_kumamoto\【速報】原告勝訴。熊本地裁が国の責任を認めました。¥\n\#毎日新聞
  \#ニュース \#ハンセン病家族訴訟 \#ハンセン病 \#熊本
  \url{http://hirono2014sk.blogspot.com/2019/06/maikumamoto.html} 
\item
  2019年07月01日20時27分の登録:
  \高橋雄一郎 @kamatatylaw\純米大吟醸勝訴は山田錦35\%で山形酵母というカプロン酸エチル高生産酵母を使ってるので、○祭には圧勝だと思うよ。削るんだったらカプロ
  \url{http://hirono2014sk.blogspot.com/2019/07/kamatatylaw35.html} 
\item
  2019年08月05日17時41分の登録:
  REGEXP:''純米大吟醸勝訴''/データベース登録済みツイート:2019年08月05日17時41分の記録:ユーザ・投稿:3/7件
  \url{http://hirono2014sk.blogspot.com/2019/08/regexp20190805174137.html} 
\item
  2019年10月04日18時47分の登録:
  \なかがわ もとみつ (中川素充) @Motomitsu\_N\【速報】DHCスラップ「反撃」訴訟、勝訴!DHCと吉田嘉明会長による澤藤統一郎弁護士への訴訟を不当訴訟と認め
  \url{http://hirono2014sk.blogspot.com/2019/10/motomitsundhcdhc.html} 
\item
  2019年10月07日08時47分の登録:
  \弁護士神原元 @kambara7\担当事件、勝訴確定です!¥\n¥\n有田議員「とてもうれしい」
  橋下氏からの名誉毀損訴訟、最高裁で勝訴確定\textbar 弁護士ドットコムニュース
  \url{http://hirono2014sk.blogspot.com/2019/10/kambara7.html} 
\item
  2019年10月15日12時56分の登録:
  \ツンデレブログ @tsundereblog\実際経験したから、その話はもうやめて。¥\n境界で勝訴。取得時効で敗訴。
  \url{http://hirono2014sk.blogspot.com/2019/10/tsundereblog.html} 
\item
  2019年10月25日20時01分の登録:
  \ささきりょう @ssk\_ryo\<ご報告>¥\n先ほど、余命ブログの発信者情報開示請求で依頼していた代理人の先生から、高裁では逆転勝訴となった旨、ご連絡いただきました。また
  \url{http://hirono2014sk.blogspot.com/2019/10/sskryo\_25.html} 
\item
  2019年10月27日12時13分の登録:
  \高橋雄一郎 @kamatatylaw\勝訴パーティーの写真ばっかりアップしている頭のおかしい弁護士がいて,まあ「稼げてる偽装」なんだろうね。
  \url{http://hirono2014sk.blogspot.com/2019/10/kamatatylaw\_27.html} 
\item
  2019年10月30日17時27分の登録:
  \高橋雄一郎 @kamatatylaw\流石にTwitterではあまりみかけないが,FBだと海外出張時の写真や航空会社のダイヤモンド会員とか勝訴パーティーの写真ばっかりア
  \url{http://hirono2014sk.blogspot.com/2019/10/kamatatylawtwitterfb.html} 
\item
  2019年11月01日15時35分の登録:
  \深澤諭史 @fukazawas\お,6年越しの脳腫瘍見落としの医療訴訟,原告勝訴か。認容額は1億6000万円とのこと。
  \url{http://hirono2014sk.blogspot.com/2019/11/fukazawas.html} 
\item
  2019年11月27日23時06分の登録: \中野
  俊徳 @kanonjilawfirm\訴訟費用を原告の負担とする内容の請求の趣旨を書き、勝訴判決でその旨の判決をいただいたことは何度かあります。
  \url{http://hirono2014sk.blogspot.com/2019/11/kanonjilawfirm\_27.html} 
\item
  2019年12月12日02時24分の登録:
  \モトケン @motoken\_tw\弁護士懲戒訴訟、逆転勝訴 請求者に賠償命令、名古屋高裁 | 2019/12/11 - 共同通信
  \url{http://hirono2014sk.blogspot.com/2019/12/motokentw20191211.html} 
\item
  2019年12月13日23時31分の登録:
  \てらやさん☆ @terayasan\何度かツイートしてるけど、接見国賠の勝訴判決で法務大臣の椅子を動産執行しようとした某先生とかね。
  \url{http://hirono2014sk.blogspot.com/2019/12/terayasan\_13.html} 
\item
  2019年12月17日17時32分の登録:
  \福島みずほ @mizuhofukushima\私が屋山太郎氏の私にたいする名誉毀損に対して提訴した裁判は私の全面勝訴の判決が東京地方裁判所で出されました。この判決につい
  \url{http://hirono2014sk.blogspot.com/2019/12/mizuhofukushima.html} 
\item
  2019年12月17日18時25分の登録:
  \弁護士神原元 @kambara7\「森友学園」情報開示訴訟 市議側が全面勝訴 値引き根拠示さぬ国「違法」 大阪高裁(毎日新聞)
  - Yahoo!ニュース
  \url{http://hirono2014sk.blogspot.com/2019/12/kambara7-yahoo.html} 
\item
  2019年12月18日11時40分の登録: \Shoko
  Egawa @amneris84\報道陣に促されて、勝訴の垂れ幕を持つ伊藤詩織さん。東京地裁前で
  \url{http://hirono2014sk.blogspot.com/2019/12/shoko-egawaamneris84\_18.html} 
\item
  2019年12月18日11時49分の登録:
  \ystk @lawkus\伊藤詩織さん勝訴か。詳細はわからないが330万円という金額は慰謝料300万+弁護士費用30万と思われる。本邦では性被害に限らず慰謝料相場が低廉
  \url{http://hirono2014sk.blogspot.com/2019/12/ystklawkus33030030.html} 
\item
  2019年12月18日19時59分の登録:
  \渡辺輝人 @nabeteru1Q78\``伊藤詩織さん涙「長かった」性暴力民事裁判で勝訴
  - 社会 : 日刊スポーツ''
  \url{http://hirono2014sk.blogspot.com/2019/12/nabeteru1q78\_18.html} 
\item
  2019年12月18日21時14分の登録:
  \ささきりょう @ssk\_ryo\【判決詳報】伊藤詩織さん勝訴、「合意ないまま性行為」と判断された理由・・・元TBS記者の証言に疑念\textbar 弁護士ドットコムニュース
  \url{http://hirono2014sk.blogspot.com/2019/12/sskryotbs.html} 
\item
  2019年12月20日19時15分の登録:
  \郷原信郎 @nobuogohara\今日の詩織さん勝訴判決を受けて、このような海外メディアの論調に改めて注目する必要があると思う。⇒【英BBCが「日本の恥」と特集! 山
  \url{http://hirono2014sk.blogspot.com/2019/12/nobuogoharabbc.html} 
\item
  2019年12月27日00時00分の登録: \望月宣武 Hiromu
  MOCHIZUKI @166mochizuki\引き出し業者に対する勝訴判決。他の事例でも参考となる裁判所の判断のポイントを書いておきます。¥\n¥\n
  \url{http://hirono2014sk.blogspot.com/2019/12/hiromu-mochizuki166mochizuki.html} 
\item
  2019年12月29日20時13分の登録:
  \高橋雄一郎 @kamatatylaw\おせち料理も注文したし「大吟醸勝訴」4合瓶も確保したので、自分も年末年始は趣味の準備書面起案に明け暮れることにしよう。
  \url{http://hirono2014sk.blogspot.com/2019/12/kamatatylaw\_29.html} 
\item
  2020年01月11日13時22分の登録:
  \ささきりょう @ssk\_ryo\私の勝訴が確定しました!
  私の存在自体が被害をもたらしているとの主張もされていただけに、嬉しい勝訴確定です。
  \url{http://hirono2014sk.blogspot.com/2020/01/sskryo\_11.html} 
\item
  2020年01月31日23時42分の登録: \K 9 9 9
  9 @k999941457035\どのような訴訟に関しても私は完全に勝訴可能な準備書面の記述を見つけたがこのTwitterの字数制限はそれを書くには狭す
  \url{http://hirono2014sk.blogspot.com/2020/01/k-9-9-9-9k999941457035twitter.html} 
\item
  2020年02月26日23時31分の登録:
  \亀石 @kame\_ishi\いきなり電話で「この件、勝訴可能性はありますか、取りっぱぐれはないですか、足は出ませんか、足が出るなら頼みません」とか言われても、俺は予言者
  \url{http://hirono2014sk.blogspot.com/2020/02/kameishi.html} 
\item
  2020年02月27日00時44分の登録:
  \安藤俊文 @toshifumi\_Ando\絶望の裁判所。¥\n¥\n「心が折れました」9割が勝訴してきた原爆症の裁判、最高裁で敗訴に
  \url{http://hirono2014sk.blogspot.com/2020/02/toshifumiando-9.html} 
\item
  2020年03月21日15時11分の登録:
  \モトケン @motoken\_tw\私は、たぶんあなたより、司法救済の限界を知っています。¥\n原告勝訴判決を、どれだけ社会変革(国民の意識変革)に繋げていくかが社会運動論。
  \url{http://hirono2014sk.blogspot.com/2020/03/motokentw\_64.html} 
\item
  2020年03月27日21時14分の登録:
  \高橋雄一郎 @kamatatylaw\おいおい-\textgreater「勝訴した弁護士は、親権を奪われた親から奪い取った子どもの養育費の一部を「成功報酬」として懐に入れる。取材を進めると、
  \url{http://hirono2014sk.blogspot.com/2020/03/kamatatylaw\_90.html} 
\item
  2020年03月30日23時51分の登録:
  \指宿昭一 @ibu61\最高裁で勝訴しました!!¥\n運転手敗訴の高裁判決を取り消し|NHK
  首都圏のニュース
  \url{http://hirono2014sk.blogspot.com/2020/03/ibu61-nhk.html} 
\item
  2020年04月02日13時01分の登録:
  \嶋﨑量(弁護士) @shima\_chikara\「あの嶋崎量弁護士」(笑)が、国際自動車最高裁勝訴判決で波に乗る、指宿昭一弁護士とご一緒!¥\n4月1日22時~ これから
  \url{http://hirono2014sk.blogspot.com/2020/04/shimachikara-4122.html} 
\item
  2020年06月15日11時55分の登録:
  REGEXP:''時効を教えず町が勝訴''/データベース登録済みツイート:2020年06月15日11時54分の記録:ユーザ・投稿:3/4件
  \url{http://hirono2014sk.blogspot.com/2020/06/regexp20200615115434.html} 
\item
  2020年08月13日11時23分の登録:
  \深澤諭史 @fukazawas\発信者情報開示請求事件において,スゴイ先例的価値のある勝訴判決をゲット(ただ,賛否両論あると思う
  。)。 特に企業に対するクレーム(苦情
  \url{http://hirono2014sk.blogspot.com/2020/08/fukazawas\_66.html} 
\item
  2020年09月18日11時07分の登録:
  \深澤諭史 @fukazawas\(・∀・)なぜか,こういう悪い口コミって,全く身に覚えがないどころか,(仮)差し押さえが華麗に決まったり,劇的勝訴したり,そういうタイミ
  \url{http://hirono2014sk.blogspot.com/2020/09/fukazawas\_44.html} 
\item
  2020年10月04日01時20分の登録:
  \深澤諭史 @fukazawas\Q68.投稿者です。相手方が「訴える」と連呼しているのですが・・・Q69.SNSとか掲示板で「劇的勝訴しました!」っていう話をよく聞きますが
  \url{http://kk2020-09.blogspot.com/2020/10/fukazawas\_4.html} 
\item
  2020年11月13日20時28分の登録:
  \高橋雄一郎 @kamatatylaw\弊職が、(1)不当懲戒請求者への訴訟のカンパをするとともに(2)勝訴したあかつきには「大吟醸勝訴」を配布すると約束したことが「スラ
  \url{http://kk2020-09.blogspot.com/2020/11/kamatatylaw12.html} 
\item
  2020年11月27日23時19分の登録:
  \ふたつのいす @eru19721103\高裁までの距離が300キロあるので「勝訴しても相手が控訴したらお金かかるよ」と依頼者に言って和解することが多い。こんなんでいいの
  \url{http://kk2020-09.blogspot.com/2020/11/eru19721103\_27.html} 
\item
  2020年12月04日16時29分の登録:
  \海渡雄一 @kidkaido\大阪地裁が、大飯原発の設置許可取消訴訟で住民側が勝訴し、原発設置許可取消の判決!
  \url{http://kk2020-09.blogspot.com/2020/12/kidkaido.html} 
\item
  2020年12月05日03時11分の登録:
  \鳩屋 @haya\_rt\大ニュースだ!大飯原発の設置許可取り消し 住民ら原告側勝訴 大阪地裁判決(毎日新聞)
  \url{http://kk2020-09.blogspot.com/2020/12/hayart.html} 
\item
  2020年12月12日20時46分の登録:
  \ないないな @nainaina1220\尋問後に裁判官の心証が覆り、勝訴的和解!よくがんばった。完全に負け筋やったけど、がんばって良かった。
  \url{http://kk2020-09.blogspot.com/2020/12/nainaina1220.html} 
\item
  2020年12月31日21時04分の登録:
  \shoya @sho\_ya\司法試験に合格したときに行きつけの立ち飲み屋さんから頂いた熱燗器で、高橋雄一郎先生から頂いた大吟醸『勝訴』の燗をつけ、大学に行かせてくれた母
  \url{http://kk2020-09.blogspot.com/2020/12/shoyashoya\_11.html} 
\item
  2021年01月27日09時59分の登録:
  \高橋雄一郎 @kamatatylaw\74期司法修習予定者か元修習生で、蒲田周辺にいる又は蒲田を訪れる方には、タイミングが合えば「大吟醸勝訴」をふるまいますので、遠慮な
  \url{http://kk2020-09.blogspot.com/2021/01/kamatatylaw\_27.html} 
\item
  2021年01月29日21時49分の登録:
  \ピピピーッ @O59K2dPQH59QEJx\碧海信用金庫が払戻を拒絶して被告にまでなった事情が気になる。碧海信用金庫は、勝訴しても相続人に払い戻さなきゃならないから、
  \url{http://kk2020-09.blogspot.com/2021/01/o59k2dpqh59qejx\_1.html} 
\item
  2021年02月22日19時18分の登録:
  \ピピピーッ @O59K2dPQH59QEJx\生活保護裁判。勝訴した途端に、「最初から勝つと思ってた」「実はワイもイッチョカミしとる」みたいなツイがワラワラ沸いてて笑っ
  \url{https://kk2020-09.blogspot.com/2021/02/o59k2dpqh59qejx\_22.html} 
\item
  2021年02月22日21時23分の登録:
  \くまちん(弁護士中村元弥) @1961kumachin\生活保護引き下げ取り消し 受給者側初の勝訴 大阪地裁判決
  \textbar{} 毎日新聞
  \url{https://kk2020-09.blogspot.com/2021/02/1961kumachin\_22.html} 
\item
  2021年03月19日09時26分の登録:
  \モトケン @motoken\_tw\判決の主文(結論部分)においては原告敗訴だけど、判決全体の内容は現時点で考えうる最も原告完全勝訴に近い判決だと思う。原告弁護団の法制史
  \url{https://kk2020-09.blogspot.com/2021/03/motokentw\_47.html} 
\item
  2021年03月23日00時17分の登録:
  \高橋雄一郎 @kamatatylaw\大吟醸勝訴という日本酒が売られているが,訴訟で敗訴した人の気持ちをちゃんと理解しているのだろうか。販売をしている会社の代表者は謝罪
  \url{https://kk2020-09.blogspot.com/2021/03/kamatatylaw\_16.html} 
\item
  2021年04月25日22時33分の登録: \弁護士
  石田龍@Commons @Ishida\_law\高橋雄一郎先生(@kamatatylaw)より、純米大吟醸「勝訴」が届きました!最高に嬉しい!!勝訴した弁理士・
  \url{https://kk2020-09.blogspot.com/2021/04/commonsishidalawkamatatylaw.html} 
\item
  2021年05月01日11時04分の登録: \都 行志/Miyako
  Koji @Miyako\_Koji\高橋雄一郎先生から純米大吟醸「勝訴」を頂きました!事務所を訪れる方と一緒に頂きたいと思います。この名にあやか
  \url{https://kk2020-09.blogspot.com/2021/05/miyako-kojimiyakokoji.html} 
\end{itemize}

 時刻は5月4日00時13分で日付が変わっています。Twilogで5月3日のツイートが791件と一日としては記録的数と思ったのですが,その後すぐ,リツイートをすると「上限に達しました。明日またツイートしてください。」などというエラーが出るようになりました。

 一日あたり2千件から2千5百件のツイートは出来たと思いますが,短時間に集中させたのが悪かったのかと思います。その辺りは試行錯誤ですが,どれぐらい集中すればエラーになるのかTwitter社も公表はしていないようです。

〉〉〉 kk\_hironoのリツイート 〉〉〉

\begin{itemize}
\tightlist
\item
  RT
  kk\_hirono(刑事告発・非常上告_金沢地方検察庁御中)|sato\_\_michiko(§
  佐藤倫子) 日時:2021-05-04 00:19/2021/05/03 16:41 URL:
  \url{https://twitter.com/kk\_hirono/status/1389238310573199360} 
  \url{https://twitter.com/sato\_\_michiko/status/1389122975513726979} 
  \textgreater{}
  憲法記念日。基本書にもほとんど記載がないくらい軽視されてきた24条がこんなにも注目される日がくるなんて感慨深い。火事場泥棒のような改憲論議を許さず、私たちの不断の努力(12条)で個人の尊厳(13条)と法の下の平等(14条)を実現していかなければならないと、改めて思う
\end{itemize}

〉〉〉 kk\_hironoのリツイート 〉〉〉

\begin{itemize}
\tightlist
\item
  RT
  kk\_hirono(刑事告発・非常上告_金沢地方検察庁御中)|sho\_ya(shoya)
  日時:2021-05-04 00:20/2021/05/03 22:18 URL:
  \url{https://twitter.com/kk\_hirono/status/1389238438096900099} 
  \url{https://twitter.com/sho\_ya/status/1389207632116781056} 
  \textgreater{}
  法律学を納めたことがないにもかかわらず、結論の妥当性やその理由の根幹たる考え方が分かる方っていらっしゃいませんか? 一言で言えば、センスが良い方と言うことになろうかと存じます。政治家の先生方に求められるのは基本的にはこれだと私は考えます。研究者である必要はありませんし、なれません
\end{itemize}

※ @kk\_hironoのアカウントがブロックされ,リツイートに失敗したツイート

\begin{itemize}
\tightlist
\item
  TW akagilaw(赤木真也(弁護士・LEC専任講師)) 日時:2021/05/03
  22:17:47 URL:
  \url{https://twitter.com/akagilaw/status/1389207562235510793} 
  \textgreater{}
  【独自】「7月末までに高齢者ワクチン接種完了は無理」全国の地方自治体の6割回答 菅首相の指示で混乱〈dot.〉(AERA
  dot.) 無能すぎる菅と厚労省。優先順位の付け方も知らんのか。\\
  \textgreater{} \#Yahooニュース\\
  \textgreater{} \url{https://t.co/kvHedTCUTI} 
\end{itemize}

 記録として残しておきたかったツイートです。赤木真也(弁護士・LEC専任講師)というアカウントは,奉納\さらば弁護士鉄道・泥棒神社の物語(@hirono\_hideki)のアカウントでもブロックされていました。一時期,ブロックが解除されたような気もするのですが,はっきり記憶にありません。

\begin{itemize}
\tightlist
\item
  2020年04月29日19時24分の登録:
  \高橋雄一郎 @kamatatylaw\岡村氏の「女子が困窮して性風俗で稼働せざるを得なくなるのが楽しみ」発言だけど,就職難の修習生を買い叩くのが楽しみなボス弁とどこが違
  \url{http://hirono2014sk.blogspot.com/2020/04/kamatatylaw\_29.html} 
\end{itemize}

〉〉〉 kk\_hironoのリツイート 〉〉〉

\begin{itemize}
\tightlist
\item
  RT
  kk\_hirono(刑事告発・非常上告_金沢地方検察庁御中)|tamai1961(玉井克哉(Katsuya
  TAMAI)) 日時:2021-05-04 00:26/2020/04/29 12:46 URL:
  \url{https://twitter.com/kk\_hirono/status/1389240061372166146} 
  \url{https://twitter.com/tamai1961/status/1255342736062009346} 
  \textgreater{}
  という面もありますが、「就きたくない職業」にいやいや就くのを期待しているところがクズだと思います。
  なので、「職に貴賎はないはず」というのも再批判になってない。「不況になって弁護士の口が減れば低賃金で法学教師になろうとする者が増えるだろう。しめしめ」というのも、同じくらいクズ。
  \url{https://t.co/sxjY6ijqsV} 
\end{itemize}

 上記のツイートは,2件目の高橋雄一郎弁護士のリツイートとして記録されていました。書式が修正前となっているので,そのままコピペはしませんでした。

\begin{itemize}
\item
  TW kamatatylaw(高橋雄一郎) 日時: 2020/04/25 19:41:07 URL:
  \url{https://twitter.com/kamatatylaw/status/1253997462349426688} 
  \textgreater{}
  「一部のキ印が自粛警察と化して」紅衛兵のように大活躍し実力行使に及べば多くの人は身を守るためにじっと自宅に籠もらざるをえなくなる。その結果,夢にまで見た「接触8割減」が実現できるね。
\item
  TW kamatatylaw(高橋雄一郎) 日時: 2020/04/25 07:09:07 URL:
  \url{https://twitter.com/kamatatylaw/status/1253808213251637248} 
  \textgreater{}
  「弁護士は嘘をつけないとなれない」というのは,弁護士は平気で虚偽事実を述べるという意味ではなく,事実は異なるのではないかという一抹の不安というか疑いがあっても被告人を有罪にするための証拠が揃っていない以上は無罪の弁論をしなければならないという,刑事弁護人の職責からきているよね。
\item
  TW yiwapon(いわぽん) 日時: 2020/04/25 02:50:17 URL:
  \url{https://twitter.com/yiwapon/status/1253743075001679873} 
  \textgreater{}
  政治家と弁護士は嘘をつけないとなれないといっている人がいるみたいなんだが、政治家は知らんけど弁護士が嘘つくと職業倫理に反する度合いは非常に強いと見られるはずなんだがなあ。つまりそういうことを平気で言っちゃうようなアレな人に弁護士をやらせたらいけなかったのである。
\item
  TW kamatatylaw(高橋雄一郎) 日時: 2020/04/24 10:28:46 URL:
  \url{https://twitter.com/kamatatylaw/status/1253496067997315072} 
  \textgreater{}
  やばそうな弁護士が激増しているの?原因はコロナだろうね。
\item
  TW kamatatylaw(高橋雄一郎) 日時: 2020/04/18 06:53:06 URL:
  \url{https://twitter.com/kamatatylaw/status/1251267469164994561} 
  \textgreater{}
  10万円の一律給付,もちろんこれで一息つく人はいるのかもしれないがじきになくなるだろうしさらに10万円とか毎月10万円といった要求もでてくるだろう。そしてその財源は将来の世代が負担するか,物価上昇という形で投資ができないインフレ耐性の低い貧困層が結果的に負担する。
\end{itemize}

 上記の高橋雄一郎弁護士のツイートは,101件目として記録されています。埋め込みツイートの表示に時間がかかり,何度かやり直しをやっていましたが,100件内に限定する前のまとめ記事で埋め込みツイートが表示されたのは久しぶりかと思います。

\begin{itemize}
\tightlist
\item
  TW tamai1961(玉井克哉(Katsuya TAMAI)) 日時: 2020/04/14 15:44:18
  URL: \url{https://twitter.com/tamai1961/status/1249951599197290496} 
  \textgreater{} まさか。\\
  \textgreater{}
  蒲田には、識見の高い女子高生の集うマックがあるのだと信じておりまする。
  \url{https://t.co/jTZD7xw4dZ} 
\end{itemize}

 先ほどと同じく,高橋雄一郎弁護士のツイートを引用した玉井克哉氏のツイートですが,このアカウントも最近,すっかり見かけなくなっていました。

\begin{itemize}
\tightlist
\item
  TW kamatatylaw(高橋雄一郎) 日時: 2020/04/14 14:46:46 URL:
  \url{https://twitter.com/kamatatylaw/status/1249937117188423684} 
  \textgreater{}
  弁護士も業務が関係する限り守秘義務はしっかり守っているので、弁護士がいう「知人の・・・が」とか、「マックで女子高生が」とか、みんなネタだと疑っているよ。
\end{itemize}

 弁護士の守秘義務について深く考えさせられる高橋雄一郎弁護士のツイートですが,自分意外も断定的に述べているのが弁護士らしさの欺瞞性,偽善性を窺わせる資料となっています。問題外と言い換えてもいいでしょう。

\begin{itemize}
\tightlist
\item
  TW stdaux(スドー🌸) 日時: 2020/04/13 23:22:47 URL:
  \url{https://twitter.com/stdaux/status/1249704590640996353} 
  \textgreater{}
  私の知人の医師はみんな守秘義務にうるさいし口が硬いので、「友達の医者に聞いたんだけど」系の煽りツイートはその時点で信頼度が少し落ちる。そんなネットで拡散しそうな人に秘密を話すわけないだろという
\end{itemize}

 今度は高橋雄一郎弁護士のリツイートとしてスドーこと・・・弁護士のツイートが出てきました。名前が思い出せません。

\begin{itemize}
\item
  TW kamatatylaw(高橋雄一郎) 日時: 2020/04/14 08:48:41 URL:
  \url{https://twitter.com/kamatatylaw/status/1249847002931724288} 
  \textgreater{}
  ちなみに日本の自殺者数は年間2万数千人で,今年はコロナに伴う景気低迷でリーマンショックの3万人を超えるのではないか。コロナよりも経済苦による自殺で亡くなる人のほうが圧倒的に多いだろうと予想する。
\item
  TW kamatatylaw(高橋雄一郎) 日時: 2020/04/14 08:37:00 URL:
  \url{https://twitter.com/kamatatylaw/status/1249844063076925440} 
  \textgreater{}
  日本では年間135万人が死亡し一日平均3700人が死亡しているわけだが,コロナでの死亡者は毎日数人なわけで,基礎疾患のある人が大半だということならば,これは平時とそれほど変わらないのではないか,がん,心疾患,自殺よりはマイナーな死因が一つ増えただけではないか,という意見もあるだろうね。
\item
  TW kamatatylaw(高橋雄一郎) 日時: 2020/04/12 13:45:29 URL:
  \url{https://twitter.com/kamatatylaw/status/1249196919383117826} 
  \textgreater{} 安部総理が自宅でくつろぐ動画、なかなか好評だね。
\item
  TW kamatatylaw(高橋雄一郎) 日時: 2020/04/08 17:55:41 URL:
  \url{https://twitter.com/kamatatylaw/status/1247810332753444865} 
  \textgreater{}
  不況下に本性を顕にしたブラック企業らは,まず高給バブル世代をバンバンリストラし,新規に正社員雇用などはせず,業務委託契約を積極活用し,若手から凄まじい搾取をして,うつ病になったら業務委託契約を解除するのだろう。
\end{itemize}

 上記が最終の199件目の記録された高橋雄一郎弁護士ツイートですが,他に岡村隆史氏に言及したツイートは見当たりませんでした。見落としがあったのかもしれないえすが,ちょうど記録した節目が冒頭に偏りすぎたタイミングだったのかもしれません。

\begin{itemize}
\tightlist
\item
  〈〈〈 2021/05/04 00:55:39 Linux Emacs: 〈〈〈
\end{itemize}

\hypertarget{ux5e73ux62104ux5e74ux306eux50b7ux5bb3ux6e96ux5f37ux59e6ux88abux544aux4e8bux4ef6ux3067ux56fdux9078ux5f01ux8b77ux4ebaux3060ux3063ux305fux88abux544aux767aux4ebaux5ca1ux7530ux9032ux5f01ux8b77ux58ebux306eux63a5ux898b}{%
\paragraph{平成4年の傷害・準強姦被告事件で国選弁護人だった被告発人岡田進弁護士の接見}\label{ux5e73ux62104ux5e74ux306eux50b7ux5bb3ux6e96ux5f37ux59e6ux88abux544aux4e8bux4ef6ux3067ux56fdux9078ux5f01ux8b77ux4ebaux3060ux3063ux305fux88abux544aux767aux4ebaux5ca1ux7530ux9032ux5f01ux8b77ux58ebux306eux63a5ux898b}}

\begin{itemize}
\tightlist
\item
  〉〉〉 Linux Emacs: 2021/05/04 09:58:51 〉〉〉
\end{itemize}

:CATEGORIES: @kanazawabengosi \#金沢弁護士会 @JFBAsns
日本弁護士連合会(日弁連) \#法務省 @MOJ\_HOUMU \#被告発人岡田進弁護士

 3日ほど前になりますが,ずっと前に綴りから外れていた1枚の書面に,第2回公判が6月29日とありました。私はずっと6月30日だと思い込みそのように記述してきたように思います。第2回公判とありましたが,これが結審で,審理の終わりに判決公判が8月3日と指定されました。

 初公判を6月18日と記憶しているのですが,本来の初公判の期日は5月28日で,後になって理解をしたのですが,準強姦で追起訴され併合審理となったため,また,川口泰司裁判官の単独審理だったのが,三宅俊一郎裁判長,山田徹裁判官の合議制となったための期日の取り消しでした。

 よく憶えているのは,本来予定されていた初公判の5月28日に,金沢西警察署から金沢刑務所の拘置所に移送されたことです。私の記憶というか残った感覚では移送された翌々日になるのですが,被告発人岡田進弁護士の接見がありました。事前の連絡もなくいきなりで,もちろん初対面です。

 Linuxの環境ではcalというコマンドがあるのですが,「cal 5
1992」と実行するといくぶんアスキーアートに似た整形されたテキストで平成4年5月のカレンダーが表示されます。それによると5月28日は木曜日となっています。

 大都市では刑務所と拘置所が別になっているようですが,地方都市では刑務所の敷地の中に拘置舎があって,巡回する刑務官なども同じです。食事も同じですが,違うのは起床時間ぐらいで2,30分,受刑者の起床時間が早くなっていました。

 土日祭日が免業日と呼ばれ,休日になりますが,刑務官の数も少なく,面会や接見もなく信書と呼ばれた手紙の受発信もありませんでした。5月28日が木曜日ということは,翌日が29日の金曜日ですが,その次の月曜日が6月1日になります。

 被告発人岡田進弁護士の接見があったのは,この5月29日か6月1日のいずれかになるはずですが,拘置所に来た翌日に接見をしたという記憶はなく,また,月が変わって6月に入ってから接見があったという記憶もありません。

 また,よく憶えていることですが,拘置所に来た当日は,私服ではなく映画やドラマでみたような灰色の受刑者の服を与えられていました。検査のためという話であったように思います。翌日の午後には私服が入っていたかもしれないですが,受刑者の服で接見をしたことはないはずです。

 被告発人岡田進弁護士の接見の内容ですが,短い時間の簡単なものでした。今も一通りは記憶しているつもりですが,過去の記述の方が詳しく正確だと思います。これからまとめたものをブログ記事として投稿します。令和3年3月31日付告発状でも使った2つのコマンドになります。

\begin{itemize}
\tightlist
\item
  2021年05月04日10時37分の登録:
  「岡田進」を@hirono\_hideki @kk\_hirono @s\_hironoで検索 1446件の該当 2021-05-04\_10:37の記録
  \url{https://kk2020-09.blogspot.com/2021/05/hironohidekikkhironoshirono14462021-05.html} 
\item
  2021年05月04日10時37分の登録:
  「岡田進」を過去のはてなダイアリーの記事から検索
  \url{https://kk2020-09.blogspot.com/2021/05/blog-post\_4.html} 
\end{itemize}

 同じ10時37分ですが,実行したコマンドはhatena-log-search-post
が先になります。タイトルに該当件数がないですが,これも早めにスクリプトの修正をしておきたいところです。

20070403:{[}link{]}
 印象に残ったのは、わずかな接見しかしていない、岡田進弁護士から、16歳の頃に一月ほど仕事をした防水工事屋のことでした。被告訴人のO兄弟のおじさんの会社でしたが、その時は、身元引受人のことで、出た話しかと考えました。接見の時は、そんな話しなど全くなかったので、質問の意図も測りかねたのです。

20061030:{[}link{]}
 拘置所に移ってから3,4日後ぐらいに国選弁護人になった岡田進弁護士が接見に来たのですが、事前の連絡などなく、まったく初めての出会いでした。また、私の方からも連絡などしておらず、このあたりも普通の被告人とは異なるという不可解な印象を与えていたのかもしれません。

20060924:{[}link{]}
 以前にも書いたことがあると思いますが、とにかくこの岡田進弁護士は、あきれ果てるほど全くの役立たずでした。「後悔、反省するぐらいなら、初めからするな」と突き放し、それっきり連絡も何もないまま、公判に望んだのです。罪状の認否などという刑事裁判の手続きに関する説明も一切ありませんでした。とにかく、ただでさえ、打ち拉がれているところに追い打ちをかけられたようなもので、一審は、気がついた頃には終わっていたような感じでした。

20060924:{[}link{]}
 尋常ではなかったのは、むしろ国選弁護人になった岡田進弁護士でした。一度だけ、金沢刑務所の拘置所に面会に来たのですが、理知的に見えながら、異常に興奮しているというか、全身で緊迫感を発奮させているようにも見えました。

20051218:{[}link{]}
 これは夜に掛けた電話でしたが、同じ頃、岡田進、長谷川紘之、木梨松嗣の弁護士事務所にも電話を掛けましたが、いずれも弁護士本人は不在だといわれました。長谷川弁護士だったと思いますが、会話のなかでご本人かもしれないという印象を受けました。私のことを知っているといい、詰るようなことを言っていたからです。

 はてなブログの検索では岡田進の該当が全部で24件となっています。数が少ないですが,内容も思った以上に乏しいものでした。被告発人大網健二のおじさんのことが記述されていましたが,これはすっかり忘れていました。平成19年4月3日の記事のようです。

 令和3年3月31日付告発状では,蛸島町の火葬場のことで少し触れていると思いますが,コーキングの防水工事で大網武久商会でした。昭和56年の10月頃,長くて一月ほど仕事をしたと思うのですが,給料は払えないと言われ,その場でやめました。

\begin{lstlisting}
py37_env ❯ twilog-serch 火葬場|grep 蛸島
\end{lstlisting}

\begin{itemize}
\tightlist
\item
  ./kk\_hirono2021-05-04\_103621.csv:2021-01-11 22:50:16
  ``蛸島の火葬場と思っていたのですが,これは珠洲市全体での火葬場になるのかもしれません。なお,宇出津の場合は,平体にあった火葬場が,現在の能登三郷に移転しています。''
  \url{https://twitter.com/kk\_hirono/status/1348628293717614592} 
\item
  ./kk\_hirono2021-05-04\_103621.csv:2021-01-11 22:46:08
  ``記憶にあるのと同じ辺りなのですが,昭和56年の秋,給料はもらえなかった被告発人大網健二のおじさんのコーキング屋の仕事で,蛸島の火葬場と聞くところに行き,屋上でコーキングの仕事をしたことをよく憶えています。''
  \url{https://twitter.com/kk\_hirono/status/1348627256042684417} 
\end{itemize}

 被告発人大網健二兄弟のおじさんですが,3週間ほど前,年配の人との会話に名前が出て,金沢市内のある場所に住んでいると聞きました。数年前に聞いたときは内灘か粟崎だったようにも思うのですが,金沢市内のやや中心部でよく知る地名でした。

 被告発人大網健二兄弟の祖父の弟と聞いていたように思いますが,その祖父は昭和60年代に亡くなったと思いますし,75歳ぐらいと聞いていたような記憶もかすかですが残っています。

 宇出津の下岩屋に3階建てのビルがあって,そこが被告発人大網健二兄弟のおじさんの会社と家だったのですが,建物は現在もそのまま残っており,人が住んでいる様子です。ちょうど3日ほど前の夕方遅い時間になりますが,前を歩いて通った時,2階から薄灯りが漏れているように見えました。

\begin{itemize}
\tightlist
\item
  2021-05-01\_191305_宇出津下岩屋.jpg \url{https://t.co/lbuhGhwexm} 
\end{itemize}

 ちょうどその付近の写真です。まだ外が薄明るい時間に傘を持って歩いて出掛けていました。Aコープ能都店で買い物をした帰りですが,いやさか道路と聞いたような新しい道路が昨年あたりに開通し,それからは余り通ることがなくなっていた下岩屋の町並みで,気まぐれでスマホで撮影しました。

 この下岩屋には4人ほど男子の同級生がいて,被告発人大網健二の家で一緒に遊ぶこともありました。被告発人大網健二の家には2階に当時珍しかったユニットバスがあり,その中で撮影をした同級生の写真も残っていたと思います。他に写真もないのに珍しい写真でした。

 被告発人大網健二の家のユニットバスは,たぶんですが,ガス器具販売をしていた展示品に近かったのかもしれません。プロパンガスの配達もしていましたが,大通りに面した方がガス器具の店で,家の裏側が自動車工場になっていました。

\begin{itemize}
\tightlist
\item
  2021-05-01\_185101_Aコープ能都店の駐車場から能登町庁舎方面,高速バス?.jpg
  \url{https://t.co/2ynYomOk13} 
\end{itemize}

 そういえばと写真を見て思い出したのですが,コンセールのとはバスの待合所にもなっているのですが,高速バスのようなものは初めて見たように思いました。

 たまたまよく高速バスを見かけたのは,前の能登町庁舎の横にある浜小路のバス停で,16時台であったように思います。

 よくよく考えてみると,浜小路にあるバス停は金沢方面だけのバス停で,道路の向かいが駐車場のある公衆トイレになるのですが,そちらにバスが停車しているのは見たことがなく,停留所にもなっていなかったと思います。

 コンセールのとを出たバスを撮影したのは18時51分で,19時30分に閉店となるAコープ能都店にはよく買い物に行く時間帯なのですが,高速バスを見かけたような記憶はなく,とても珍しいと思いながら,銀河鉄道999に出ているターミナル駅のようにも思えました。

 昭和50年代は国鉄の宇出津駅があった場所になります。宇出津駅のことは記憶にしかないのですが,ネットでは当時の写真を見かけることはあります。それでも数は少なく,昭和50年代のものはみていないかもしれません。

 取り壊し前にデジカメで撮影した写真もあったと思うのですが,廃線となり数年の間に駅のホームはボロボロとなっていました。そこでAコープ能都店で買ってきた弁当を食べたような記憶もあるのですが,ずいぶん風の通る場所だと思ったことも憶えています。

 写真を見返すと,やはり高速バスに違いはないと思いますが,見かけたときは,今から金沢に向かうものと思っていました。よく考えると金沢から来て珠洲に向かう高速バスであった可能性が高そうです。

 この高速バスは2回ほど利用した憶えがあるのですが,2009年3月に宇出津に戻ってからは乗っていないと思います。高速バスは金沢に向かっただけで,高速バスで宇出津に戻ったということはなかったと思います。

 1つだけはっきりしているのは,平成14年9月のことです。これも一週間ほど前に理解に苦しむほど意外な発見があって取り上げる予定にしていたのですが,被告発人大網健二にあずかってもらい,車検を受けて取りに向かったホンダトゥデイの軽四のことです。

 先に少しだけご紹介をしておくと,伊東一廣裁判長の再審請求棄却の決定書が見つかったのですが,驚いたことに平成14年(た)1号となっていたのです。記憶にあるのは平成15年(た)1号で,現物が見つからないのですが,撮影された写真の一部で確認はしています。

\begin{itemize}
\tightlist
\item
  〈〈〈 2021/05/04 12:36:27 Linux Emacs: 〈〈〈
\end{itemize}

\hypertarget{ux5e73ux62104ux5e74ux306eux50b7ux5bb3ux6e96ux5f37ux59e6ux88abux544aux4e8bux4ef6ux3067ux56fdux9078ux5f01ux8b77ux4ebaux3060ux3063ux305fux88abux544aux767aux4ebaux5ca1ux7530ux9032ux5f01ux8b77ux58ebux306eux63a5ux898btwitterux306eux8a18ux9332}{%
\paragraph{平成4年の傷害・準強姦被告事件で国選弁護人だった被告発人岡田進弁護士の接見:Twitterの記録}\label{ux5e73ux62104ux5e74ux306eux50b7ux5bb3ux6e96ux5f37ux59e6ux88abux544aux4e8bux4ef6ux3067ux56fdux9078ux5f01ux8b77ux4ebaux3060ux3063ux305fux88abux544aux767aux4ebaux5ca1ux7530ux9032ux5f01ux8b77ux58ebux306eux63a5ux898btwitterux306eux8a18ux9332}}

\hypertarget{ux5e73ux62109ux5e742ux6708ux9803ux306bux639bux3051ux305fux88abux544aux767aux4ebaux5ca1ux7530ux9032ux5f01ux8b77ux58ebux306eux6cd5ux5f8bux4e8bux52d9ux6240ux3078ux306eux96fbux8a71}{%
\paragraph{平成9年2月頃に掛けた被告発人岡田進弁護士の法律事務所への電話}\label{ux5e73ux62109ux5e742ux6708ux9803ux306bux639bux3051ux305fux88abux544aux767aux4ebaux5ca1ux7530ux9032ux5f01ux8b77ux58ebux306eux6cd5ux5f8bux4e8bux52d9ux6240ux3078ux306eux96fbux8a71}}

\begin{itemize}
\tightlist
\item
  〉〉〉 Linux Emacs: 2021/05/04 12:49:20 〉〉〉
\end{itemize}

:CATEGORIES: @kanazawabengosi \#金沢弁護士会 @JFBAsns
日本弁護士連合会(日弁連) \#法務省 @MOJ\_HOUMU \#被告発人岡田進弁護士
\#被告発人長谷川紘之弁護士

20051218:{[}link{]}
 これは夜に掛けた電話でしたが、同じ頃、岡田進、長谷川紘之、木梨松嗣の弁護士事務所にも電話を掛けましたが、いずれも弁護士本人は不在だといわれました。長谷川弁護士だったと思いますが、会話のなかでご本人かもしれないという印象を受けました。私のことを知っているといい、詰るようなことを言っていたからです。

 上記の引用は前回の次のエントリーで取り上げたはてなブログの記録の一部ですが,前回のエントリーで取り上げるのを忘れていたことと,時期も平成4年と平成9年とまるで違うので,個別に取り上げておくことにしました。

\begin{itemize}
\item
  1343:2021-05-04\_12:36:56 \#告発状 \#\#\#\#
  平成4年の傷害・準強姦被告事件で国選弁護人だった被告発人岡田進弁護士の接見
  \url{https://hirono-hideki.hatenadiary.jp/entry/2021/05/04/123654} 
\item
  〈〈〈 2021/05/04 12:53:19 Linux Emacs: 〈〈〈
\item
  〉〉〉 Linux Emacs: 2021/05/04 13:04:19 〉〉〉
\item
  〈〈〈 2021/05/04 14:47:06 Linux Emacs: 〈〈〈
\end{itemize}

\hypertarget{ux88abux544aux767aux4ebaux5ca1ux7530ux9032ux5f01ux8b77ux58ebux306eux7b2c2ux56deux516cux5224ux8abfux66f8ux30e2ux30c8ux30b1ux30f3ux3053ux3068ux77e2ux90e8ux5584ux6717ux5f01ux8b77ux58ebux4eacux90fdux5f01ux8b77ux58ebux4f1aux306eux30c4ux30a4ux30fcux30c8ux3067ux77e5ux3063ux305fux4e2dux5b66ux751fux306bux30d6ux30edux30b0ux3067ux30d8ux30a4ux30c8ux30b9ux30d4ux30fcux30c160ux4ee3ux306eux7537ux6027ux306bux640dux5bb3ux8ce0ux511f130ux4e07ux5186ux306eux5224ux6c7a}{%
\paragraph{被告発人岡田進弁護士の第2回公判調書:モトケンこと矢部善朗弁護士(京都弁護士会)のツイートで知った「中学生にブログでヘイトスピーチ、60代の男性に損害賠償130万円の判決」}\label{ux88abux544aux767aux4ebaux5ca1ux7530ux9032ux5f01ux8b77ux58ebux306eux7b2c2ux56deux516cux5224ux8abfux66f8ux30e2ux30c8ux30b1ux30f3ux3053ux3068ux77e2ux90e8ux5584ux6717ux5f01ux8b77ux58ebux4eacux90fdux5f01ux8b77ux58ebux4f1aux306eux30c4ux30a4ux30fcux30c8ux3067ux77e5ux3063ux305fux4e2dux5b66ux751fux306bux30d6ux30edux30b0ux3067ux30d8ux30a4ux30c8ux30b9ux30d4ux30fcux30c160ux4ee3ux306eux7537ux6027ux306bux640dux5bb3ux8ce0ux511f130ux4e07ux5186ux306eux5224ux6c7a}}

\begin{itemize}
\tightlist
\item
  〉〉〉 Linux Emacs: 2021/05/13 11:00:43 〉〉〉
\end{itemize}

:CATEGORIES: @kanazawabengosi \#金沢弁護士会 @JFBAsns
日本弁護士連合会(日弁連) \#法務省 @MOJ\_HOUMU \#被告発人岡田進弁護士
\#モトケンこと矢部善朗弁護士(京都弁護士会)

 「被告発人岡田進弁護士の第2回公判調書:」という接頭辞のエントリーをシリーズ化する作業に入る直前のタイミングだったのですが,モトケンこと矢部善朗弁護士(京都弁護士会)のタイムラインで気になるツイートがいくつかあり,タイムラインがいつもと違った様相となっていました。

 モトケンこと矢部善朗弁護士(京都弁護士会)が記事タイトルとURLでツイートするのも珍しいと最初に感じたのですが,まるで直前に短く切り上げた出口絢記者のツイートにあった記者会見の記事を裏付け,具体化するような内容となっていました。

\begin{itemize}
\item
  1363:2021-05-13\_10:36:48 \#告発状 \#\#\#\#
  三宅俊一郎裁判長の審理,数年ぶりの公判調書の読み込みと,弁護士ドットコム出口絢記者のTwitterタイムラインでの発見
  \url{https://hirono-hideki.hatenadiary.jp/entry/2021/05/13/103643} 
\item
  中学生にブログでヘイトスピーチ、60代の男性に損害賠償130万円の判決(BuzzFeed
  Japan) - Yahoo!ニュース
  \url{https://news.yahoo.co.jp/articles/f9f90f8ba44c8cdc45475a39d67c1029418b610f} 
\end{itemize}

 モトケンこと矢部善朗弁護士(京都弁護士会)のツイートは上記のYahoo!ニュースです。BuzzFeed
Japanという部分がリンクになっていたので開くと,トップページと思われる次のページが表示されました。

\begin{itemize}
\tightlist
\item
  BuzzFeed - バズフィードジャパン \url{https://t.co/Vc7zLK90yD} 
\end{itemize}

 BuzzFeedというニュースサイトを見たのも久しぶりだったのですが,このトップページのようなものは余り見覚えがなく,ここにも右手にランキングがあって,その1位がモトケンこと矢部善朗弁護士(京都弁護士会)のツイートで紹介されていた次の記事でした。

\begin{itemize}
\tightlist
\item
  中学生にブログでヘイトスピーチ、60代の男性に損害賠償130万円の判決
  \url{https://t.co/p6Hafuo6hV} 
\end{itemize}

 記事の配信時刻が見当たらないと思ったのですが,見出しのすぐ上に「掲載日18時間前」とあるのがそうなのかと思います。Yahoo!ニュースでは5/12(水)
17:33配信となっていました。

 また横道にそれるようですが,そういえば,Yahoo!ニュースのトップページというのも余り見た覚えがないとリンクを開いたところ,右手のアクセスランキングが5位までと見覚えのあるものでしたが,「あなたにおすすめ」とメイン部分にずらりと並んだ一覧に,次の記事を見つけました。

\begin{itemize}
\tightlist
\item
  袴田事件、弁護団長「西嶋勝彦さん」の素顔 真犯人は「警察と利害が一致していた人物」(デイリー新潮)
  - Yahoo!ニュース \url{https://t.co/vqZTYcnW7U}  5/13(木) 6:01配信
\end{itemize}

 数時間前に注目していたデイリー新潮の記事だったのも意外なことですが,よく見ると今朝の6時01分に配信されたばかりの新着記事であったようです。今のところYahoo!ニュースには入っていないようですが,5位より下のランキングが見れないものかと先程考えていました。

 時刻は11時44分です。4ページに分かれていましたが記事を読み終えました。ずいぶん踏み込んだ内容になっているという感想です。Twitterのトレンドにあがることがあるのか,その辺りも注目したいと思います。

\begin{itemize}
\tightlist
\item
  アクセスランキング(ニュース - 国内) - Yahoo!ニュース
  \url{https://t.co/W6l2GD1u4r} 
\end{itemize}

 40位までランキングがあったのですが,袴田事件の記事は見当たらず,袴田と再審でページ内検索をしてみましたが,やはり該当はありませんでした。

 この袴田事件については,ビートルズとレコードという接点で個人的に蛸島事件と強い結びつきがあり,いずれ取り上げる予定にしていました。なお,「袴田事件、弁護団長「西嶋勝彦さん」の素顔」という記事には,八海事件や正木ひろし弁護士の名前も出ていました。

 正木ひろし弁護士の本は,図書館から借りてきてからまだページも開いていないかもしれません。一度だけ開いてパラパラみましたが,割と大きな文字で文字数が少なめ読みやすそうでした。文字数が多かったのが石川県奥能登でおきた蛸島学童殺人事件裁判記録です。

 モトケンこと矢部善朗弁護士(京都弁護士会)のタイムラインですが,他にも気になるツイートがあったので,まとめてご紹介しておきます。しばらく見ていないので更新のツイートがあるかもしれません。

〉〉〉 kk\_hironoのリツイート 〉〉〉

\begin{itemize}
\tightlist
\item
  RT
  kk\_hirono(刑事告発・非常上告_金沢地方検察庁御中)|sasakitoshinao(佐々木俊尚)
  日時:2021-05-13 11:58/2021/05/13 08:21 URL:
  \url{https://twitter.com/kk\_hirono/status/1392675630521614339} 
  \url{https://twitter.com/sasakitoshinao/status/1392620859211255812} 
  \textgreater{}
  「1980年代まで日本のワクチン技術は高く、米国などに技術供与していた」。ところが予防接種の副作用訴訟や薬害エイズ、さらに反ワクチンなどで後退が続き今の状況に。過去の経緯がよくわかる日経記事。/必然だったワクチン敗戦 不作為30年、民のはしご外す
  \url{https://t.co/lsOaCtaX3n} 
\end{itemize}

〉〉〉 kk\_hironoのリツイート 〉〉〉

\begin{itemize}
\tightlist
\item
  RT
  kk\_hirono(刑事告発・非常上告_金沢地方検察庁御中)|nabeteru1Q78(渡辺輝人)
  日時:2021-05-13 11:59/2021/05/12 13:11 URL:
  \url{https://twitter.com/kk\_hirono/status/1392675802148315137} 
  \url{https://twitter.com/nabeteru1Q78/status/1392331630778347520} 
  \textgreater{}
  「セクシャルハラスメント」は、受け手の主観を基準にした迷惑行為から、民法上の不法行為に該当するもの、刑法上の犯罪までかなり幅広い概念。民事で違法の疑いが出ない程度に身を処すのはそんなに難しくないはず。その先は「セクハラです」と言われたら「すいません」で行動修正すれば済むのでは。
\end{itemize}

〉〉〉 kk\_hironoのリツイート 〉〉〉

\begin{itemize}
\item
  RT
  kk\_hirono(刑事告発・非常上告_金沢地方検察庁御中)|okumuraosaka(okumuraosaka)
  日時:2021-05-13 11:59/2021/05/12 15:42 URL:
  \url{https://twitter.com/kk\_hirono/status/1392675923833487361} 
  \url{https://twitter.com/okumuraosaka/status/1392369652085321728} 
  \textgreater{}
  わいせつ行為に及んだ教員を一律に排除するわけではなく、憲法が保障する職業選択の自由と整合性がとれると判断した。わいせつ教員、免許再交付拒否可能に 自民部会が了承:日本経済新聞
  \url{https://t.co/j6CauPHU6s} 
\item
  〉〉〉 アカウント(@motoken\_tw)は,@kk\_hironoをブロックしています。リツイートできませんでした。
  〉〉〉 ¥\n ¥\n \url{https://t.co/ZaQfUP973j} 
\item
  〉〉〉 アカウント(@km0bake)は,@kk\_hironoをブロックしています。リツイートできませんでした。
  〉〉〉 ¥\n ¥\n \url{https://t.co/JeOiJwPCrF} 
\item
  〉〉〉 アカウント(@motoken\_tw)は,@kk\_hironoをブロックしています。リツイートできませんでした。
  〉〉〉 ¥\n ¥\n \url{https://t.co/bL4DRy4Sde} 
\item
  〉〉〉 アカウント(@motoken\_tw)は,@kk\_hironoをブロックしています。リツイートできませんでした。
  〉〉〉 ¥\n ¥\n \url{https://t.co/knkCk1tJTS} 
\item
  〉〉〉 アカウント(@motoken\_tw)は,@kk\_hironoをブロックしています。リツイートできませんでした。
  〉〉〉 ¥\n ¥\n \url{https://t.co/rfGLPCuDyc} 
\item
  〉〉〉 アカウント(@motoken\_tw)は,@kk\_hironoをブロックしています。リツイートできませんでした。
  〉〉〉 ¥\n ¥\n \url{https://t.co/1PjY6YiE7i} 
\end{itemize}

※ @kk\_hironoのアカウントがブロックされ,リツイートに失敗したツイート

\begin{itemize}
\tightlist
\item
  TW motoken\_tw(モトケン) 日時:2021/05/13 09:52:25 URL:
  \url{https://twitter.com/motoken\_tw/status/1392643861344780288} 
  \textgreater{}
  茨城一家殺傷事件「被疑者は社長の息子」デマ拡散 無関係の会社にいたずら電話300件\textbar 弁護士ドットコムニュース
  \url{https://t.co/3vW58yEe4j}  @bengo4topicsより
\end{itemize}

※ @kk\_hironoのアカウントがブロックされ,リツイートに失敗したツイート

\begin{itemize}
\tightlist
\item
  TW km0bake(👁️つまらむ👁️) 日時:2021/05/12 13:33:33 URL:
  \url{https://twitter.com/km0bake/status/1392337122556747778} 
  \textgreater{} めずらしくナベテル先生に同意w
\end{itemize}

※ @kk\_hironoのアカウントがブロックされ,リツイートに失敗したツイート

\begin{itemize}
\tightlist
\item
  TW motoken\_tw(モトケン) 日時:2021/05/13 09:50:51 URL:
  \url{https://twitter.com/motoken\_tw/status/1392643467981979650} 
  \textgreater{} @km0bake w\\
  \textgreater{} めずらしくすごく穏当w
\end{itemize}

※ @kk\_hironoのアカウントがブロックされ,リツイートに失敗したツイート

\begin{itemize}
\tightlist
\item
  TW motoken\_tw(モトケン) 日時:2021/05/13 08:56:21 URL:
  \url{https://twitter.com/motoken\_tw/status/1392629751517302788} 
  \textgreater{}
  このまとめは石川さんの擁護者が作成したようだけど、擁護になっているとは思えない。\\
  \textgreater{} 石川さんの問題点がよく表れている。\\
  \textgreater{}\\
  \textgreater{}
  フェミニストの石川優実さんへつきまといのようなアカウントの一部
  5/9-5/12 \#石川優実さんへの誹謗中傷をやめろ - Togetter
  \url{https://t.co/VVsCf5fF66}  @togetter\_jpより
\end{itemize}

※ @kk\_hironoのアカウントがブロックされ,リツイートに失敗したツイート

\begin{itemize}
\tightlist
\item
  TW motoken\_tw(モトケン) 日時:2021/05/13 08:35:53 URL:
  \url{https://twitter.com/motoken\_tw/status/1392624603869052931} 
  \textgreater{} 青識さん @BlauerSeelowe
  界隈では少子化と女子教育が話題みたいだけど\\
  \textgreater{}\\
  \textgreater{}
  女性には出産育児より優越する又は優先すべきことがあるというツイフェミ的価値観の女性が増えれば少子化は進むと思う。\\
  \textgreater{}\\
  \textgreater{}
  「教育」に社会に出てからの価値観の取り込みを含めると、教育の内容次第では少子化と論理的に関連する。
\end{itemize}

※ @kk\_hironoのアカウントがブロックされ,リツイートに失敗したツイート

\begin{itemize}
\tightlist
\item
  TW motoken\_tw(モトケン) 日時:2021/05/13 07:39:58 URL:
  \url{https://twitter.com/motoken\_tw/status/1392610531245125638} 
  \textgreater{}
  中学生にブログでヘイトスピーチ、60代の男性に損害賠償130万円の判決(BuzzFeed
  Japan)\\
  \textgreater{} \#Yahooニュース\\
  \textgreater{} \url{https://t.co/JROfSCOeFD} 
\end{itemize}

 モトケンこと矢部善朗弁護士(京都弁護士会)のタイムラインに更新はなかったですが,奥村徹弁護士のツイートのリツイートがあって,これは先程気が付かずにいたようです。

 しばらく前はあったと思うモトケンこと矢部善朗弁護士(京都弁護士会)の固定されたツイートですが,タイムラインの一番上にあるのが佐々木俊尚というアカウントのツイートのリツイートです。

 漫画の似顔絵のようなアイコンでずいぶん前から見かけていたように思うのですが,1年か2年ほど前になって人物をネットで知るようになり,前後ははっきり覚えていないですが,AbemaPrimeの番組で見かけるようになりました。一昨日の夜も深澤諭史弁護士と一緒に見ています。

 ツイートにあるプロフィールの名前の部分をポップアップさせると,フォロワー数が77.9万とありました。これまで意識してこなかったのですが,ジャーナリストの江川紹子氏のアカウントを確認すると29.6万で記憶にあった29.5万より0.1万増えているかもしれません。

\begin{itemize}
\tightlist
\item
  TW amneris84(Shoko Egawa) 日時: 2021/05/13 08:58:32 URL:
  \url{https://twitter.com/amneris84/status/1392630301466107904} 
  \textgreater{}
  「人格権の侵害」と認め、一審より賠償額を増額。刑事では侮辱罪で有罪確定。木村花さんのケースも含め、侮辱罪の量刑は議論の必要ありでは? →中学生にブログでヘイトスピーチ、損害賠償130万円 東京高裁「人種差別」:東京新聞
  TOKYO Web \url{https://t.co/JvpO7Sycmk} 
\end{itemize}

 ジャーナリストの江川紹子氏のタイムラインにも本人のツイートで同じ内容の記事がありましたが,こちらは東京新聞
TOKYO Webとなっています。

 やたらと明るい天井の蛍光灯のような灯りがツイートの写真にありますが,たぶんこれだとわからなかった神原元弁護士の姿があり,隣の体格の良い白い半袖シャツの人物にカメラが向いているようです。胸に馬のようなイラストのあるシャツですが,これも記者会見では珍しい服装です。

\begin{itemize}
\tightlist
\item
  中学生にブログでヘイトスピーチ、損害賠償130万円 東京高裁「人種差別」:東京新聞
  TOKYO Web \url{https://www.tokyo-np.co.jp/amp/article/103821} 
\end{itemize}

 東京新聞の記事ということは親会社が中日新聞で,北陸中日新聞の記事にもなりそうです。このような主要なメディアで,個性の強い神原元弁護士の記事を見るのは,とても珍しいことかもしれません。最近見かけるツイートは割とマイルドという印象は受けていました。

 最近見かけていた神原元弁護士のツイートは,ほとんどが小倉秀夫弁護士のタイムラインで,それも小倉秀夫弁護士との返信や引用ツイートのやりとりとなっていました。同業者としてか神原元弁護士の方でかなり気をつかったような言葉遣いも感じていました。 

 東京新聞の記事に弁護士の名前はなく代理人弁護士となっていました。写真の下にある説明文を読んでわかったのですが,原告の男性とあります。ジムで鍛えているような体格でとても18歳には見えず,弁護士なのかと思っていました。

 このモトケンこと矢部善朗弁護士(京都弁護士会)のツイートにある記事を取り上げておこうと思ったのは,中学生の時の被害が大学生になってから,というのが大きく,モトケンこと矢部善朗弁護士(京都弁護士会)と私のネット上の関係性に類似しているからです。

 東京新聞の記事を読むと,名誉毀損の認定も求めて控訴した,とあり,「東京高裁は「一般の読み方からすれば、事実と解されない」としてこの訴えは退けたが、中学生という多感な時期に受けた精神的苦痛や「読者の差別的言動をあおり、」と続きます。

 私の場合もモトケンこと矢部善朗弁護士(京都弁護士会)のブログで,モトケンこと矢部善朗弁護士(京都弁護士会)のあおりはありましたが,それほど同調し過激化する人物が出てこなかったので,比較的影響は少なかったともいえますが,精神的なダメージは,警察検察との信頼関係を破壊。

\begin{itemize}
\tightlist
\item
  〈〈〈 2021/05/13 12:36:43 Linux Emacs: 〈〈〈
\end{itemize}

\hypertarget{ux88abux544aux767aux4ebaux5ca1ux7530ux9032ux5f01ux8b77ux58ebux306eux7b2c2ux56deux516cux5224ux8abfux66f8ux6b63ux6728ux3072ux308dux3057ux5f01ux8b77ux58ebux3068ux88b4ux7530ux4e8bux4ef6ux305dux3053ux304bux3089ux6ff1ux7530ux6b66ux5f8bux88c1ux5224ux9577ux306eux9006ux8ee2ux7121ux7f6aux5224ux6c7aux3060ux3063ux305fux68a8ux6728ux4f5cux6b21ux90ceux5f01ux8b77ux58ebux306eux5f37ux5236ux308fux3044ux305bux3064ux4e8bux4ef6ux5224ux6c7aux6587ux3092ux767aux898b}{%
\paragraph{被告発人岡田進弁護士の第2回公判調書:正木ひろし弁護士と袴田事件,そこから濱田武律裁判長の逆転無罪判決だった梨木作次郎弁護士の強制わいせつ事件判決文を発見}\label{ux88abux544aux767aux4ebaux5ca1ux7530ux9032ux5f01ux8b77ux58ebux306eux7b2c2ux56deux516cux5224ux8abfux66f8ux6b63ux6728ux3072ux308dux3057ux5f01ux8b77ux58ebux3068ux88b4ux7530ux4e8bux4ef6ux305dux3053ux304bux3089ux6ff1ux7530ux6b66ux5f8bux88c1ux5224ux9577ux306eux9006ux8ee2ux7121ux7f6aux5224ux6c7aux3060ux3063ux305fux68a8ux6728ux4f5cux6b21ux90ceux5f01ux8b77ux58ebux306eux5f37ux5236ux308fux3044ux305bux3064ux4e8bux4ef6ux5224ux6c7aux6587ux3092ux767aux898b}}

\begin{itemize}
\tightlist
\item
  〉〉〉 Linux Emacs: 2021/05/13 16:02:37 〉〉〉
\end{itemize}

:CATEGORIES: @kanazawabengosi \#金沢弁護士会 @JFBAsns
日本弁護士連合会(日弁連) \#法務省 @MOJ\_HOUMU \#袴田事件

2021-05-13 13:08:11 ``RT @itmedia\_news:
PC版Chrome、Twitterの検索窓を「ID入力欄」と勘違い? Facebookも¥\n\url{https://t.co/VBQN9mgQi7} 
\url{https://t.co/LFdFiWDFXs''} 
\url{https://twitter.com/hirono\_hideki/status/1392693127085920257} 

 上記のツイートが買い物に出掛ける前だったと思いますが,その出掛ける少し前に,正木ひろし弁護士の本の最初の見出しの文章を一つだけ読んでいました。「首なし事件の法廷」で7ページから始まり次の見出しが11ページと目次にあります。

 本当はなんと呼ぶのかよくわからないのですが,大きな見出しが「事件」「信念」「自伝」「年譜」とあり,それぞれその中に小さな見出しがあります。短い見出しが多く,中には「''問答''」というものも。「年譜」は小見出しが1つで異なる扱いになるのかと思います。

 「この事件をやりとげたということは私個人の力ではなく,過去・現在・未来にわたる無限の人霊(ヒューマニティー)の業であったと信ずるものである。」と「首なし事件の法廷」は結ばれていました。

 午前中は,Yahooニュースで袴田事件の最新記事を読んでいて,次のエントリーとしても取り上げていますが,先程確認したところランキング40には入っていませんでした。

\begin{itemize}
\tightlist
\item
  1364:2021-05-13\_12:37:13 \#告発状 \#\#\#\#
  被告発人岡田進弁護士の第2回公判調書:モトケンこと矢部善朗弁護士(京都弁護士会)のツイートで知った「中学生にブログでヘイトスピーチ、60代の男性に損害賠償130万円の判決」
  \url{https://hirono-hideki.hatenadiary.jp/entry/2021/05/13/123710} 
\end{itemize}

\begin{quote}
《引用の始まり》
\end{quote}

\begin{quote}
西嶋勝彦氏は福岡県出身。長兄はシベリア抑留され、旧制中学生の次兄は通学途中に米軍の機銃掃射で亡くなった。1963年(昭和38年)に中央大学を卒業、現在よりはるかに難しい司法試験に合格し、東京五輪の翌年の1965年に弁護士登録した。翌年に袴田事件が起きたが当時は関わっていない。しかし、山口県で起きた八海(やかい)事件、仁保事件などの殺人冤罪事件を担当した。そして有名な「四大死刑囚冤罪」のひとつ島田事件に関わる。静岡県の事件だ。

「島田事件の赤堀政夫さんが無罪になった後、袴田弁護団から声がかかった。だいぶ後で巌さんに面会したがすでに言動はおかしかった。でもそうなる前の袴田さんが家族に出した書簡の文章は非常にレベルが高い。上告趣意書も素晴らしい文章。八海事件の死刑囚も獄中で高いレベルの文章を書けるようになった。獄中で勉強したのですね」。
\end{quote}

\begin{quote}
《引用の終わり》
\end{quote}

\begin{itemize}
\tightlist
\item
  袴田事件、弁護団長「西嶋勝彦さん」の素顔 真犯人は「警察と利害が一致していた人物」(デイリー新潮)
  -
  Yahoo!ニュース \url{https://news.yahoo.co.jp/articles/b5e35686b1b76110bc7dcca71b378f41759459f2?page=3n} 
\end{itemize}

\begin{quote}
《引用の始まり》
\end{quote}

\begin{quote}
日弁連に再審法改正のための独立の対策本部設置を要求している西嶋氏は「そもそも検察官が三者協議の一員として加わることがおかしい」と強調する。「再審手続きは請求人と裁判所の関係で検察は当事者ではない。検事が不服申し立てすることもけしからん。規定を改正しないと検察はどこまでも即時抗告や特別抗告できてしまう。東京高裁が再審開始決定しても検察は特別抗告できる。不服申し立てを許している国は珍しい、ドイツは禁止している。日本は世界の趨勢から遅れています」。

 八海事件などで若い頃から正木ひろし、上田誠吉、後藤昌次郎、青木英五郎など錚々たる「冤罪弁護士」に接した。「主に自由法曹団の本部が同居していた事務所にいたけど、検察に敗北した地方の弁護士が支援を求めてきた。再審事件は日弁連の人権擁護委員会でかかわった。丸正事件、徳島ラジオ商事件などです。ラジオ商事件の主任裁判官が秋山(賢三)さん。袴田事件では僕が秋山さん(現弁護士)を誘ったら頑張ってくれました」。
\end{quote}

\begin{quote}
《引用の終わり》
\end{quote}

\begin{itemize}
\tightlist
\item
  袴田事件、弁護団長「西嶋勝彦さん」の素顔 真犯人は「警察と利害が一致していた人物」(デイリー新潮)
  -
  Yahoo!ニュース \url{https://news.yahoo.co.jp/articles/b5e35686b1b76110bc7dcca71b378f41759459f2?page=3n} 
\end{itemize}

 上記に2箇所の引用をしましたが,八海事件がでてきたときもしやと思っていたところ,ほどなく「八海事件などで若い頃から正木ひろし、上田誠吉、後藤昌次郎、青木英五郎など錚々たる「冤罪弁護士」に接した。「主に自由法曹団の本部が同居していた事務所にいたけど、」と出てきました。

 この自由法曹団というのは,現在,本日,モトケンこと矢部善朗弁護士(京都弁護士会)と一緒に取り上げた神原元弁護士が所属しているように思います。神原元弁護士のTwitterのプロフィールを見て確認しておきます。

\begin{quote}
《引用の始まり》
\end{quote}

\begin{quote}
弁護士神原元@kambara72000年から弁護士。武蔵小杉合同法律事務所を主宰。自由法曹団常任幹事。植村隆東京弁護団事務局長。著作「ヘイトスピーチに抗する人々」(新日本出版社2014年)。「9条の挑戦
~非軍事中立戦略のリアリズム」(大月書店2018年)。布施辰治弁護士を敬愛。愛読書「レ・ミゼラブル」。非武装中立論者。死刑廃止論者。野党共闘支持。mklo.org2012年8月からTwitterを利用しています1,081
フォロー中2.2万 フォロワー
\end{quote}

\begin{quote}
《引用の終わり》
\end{quote}

\begin{itemize}
\tightlist
\item
  弁護士神原元(@kambara7)さんの返信があるツイート / Twitter
  \url{https://twitter.com/kambara7/with\_replies} 
\end{itemize}

 やはり,「自由法曹団常任幹事。植村隆東京弁護団事務局長」とありました。自由法曹団のホームページは前に見たことがあったと思いますが,最近はほとんど名前を見かけていなかったように思います。その辺りもtwilog-serch-post
でまとめ記事を作成し確認をしておきます。

\begin{itemize}
\tightlist
\item
  2021年05月13日16時38分の登録:
  「自由法曹団」を@hirono\_hideki @kk\_hirono @s\_hironoで検索 120件の該当 2021-05-13\_16:38の記録
  \url{https://kk2020-09.blogspot.com/2021/05/hironohidekikkhironoshirono1202021-05.html} 
\end{itemize}

 データベースに登録する方に少々時間がかかるのですが,テキストのみのまとめ記事は仕上がりが早いです。「袴田事件」のキーワードもまとめ記事を作成しておこうと思いますが,今度は処理時間を計測してみます。

 今度は思いの外時間が掛かったように見えましたが,「arg-bpost.py 0.21s
user 0.03s system 2\% cpu 10.217
total」という処理事件の計測結果です。約11秒なのかと思います。これでブログへの投稿が完了しています。

 記事の内容は次のコマンドで一時ファイルに書き出しをしていますが,0.24秒かと思います。CPUとある項目の数値の意味がわからず,一瞬だったので10秒はありえないです。

\begin{lstlisting}
arg-bpost.py   0.21s user 0.03s system 2% cpu 10.217 total453,614件のツイートを検索し,1022件の該当がありました。
\end{lstlisting}

2011-02-01 16:01:32
``自由法曹団愛知支部紹介(ホームページ)97.4-Mozilla Firefox
\url{http://ow.ly/3NRTe} 
名張毒ブドウ酒事件、星野事件(金沢市の新聞配達員が通りすがりの女性にイタズラしたとの虚偽の事実により起訴されたもの)等の冤罪事件に取り組んでいます。(引用)''
\url{https://twitter.com/hirono\_hideki/status/32332705689505792} 

 自由法曹団をキーワードに含むツイートとして最初に記録されていたのが上記の2011年2月1日のツイートでした。金沢の新聞記者のわいせつ事件としてずっと記憶にあったのですが,新聞配達員となっています。星野事件という事件名も記憶になかったですが,金沢で聞かない地名です。

 西村依子弁護士が,私に苦い経験として受任を断る理由にもしていた冤罪とされた事件ですが,私は自分で購入した無罪事例集のような本で読んでいました。しかし,前に探しても該当部分は見つかりませんでした。梨木作次郎弁護士の名前があったこともよく憶えています。

\begin{itemize}
\tightlist
\item
  しがらみ太郎の事件簿2 金沢・能登殺人事件 - ドラマ詳細データ -
  ◇テレビドラマデータベース◇ \url{https://t.co/h9WCGD4DMl} 
\end{itemize}

 「保険金の不正請求の調査を開始した白髪太郎(小林稔侍)」などと内容紹介にありますが,能登のどこがロケ地になっていたのかわからず,新開ホテルや金沢ファーストホテルというのがあります。Googleの「星野事件 金沢」の検索結果に出てきました。

 金沢と能登を舞台にしたサスペンスドラマで,「今回は故郷の金沢で保険金殺人事件を解決する。」とあるので内容が気になるところです。平成8年10月21日の放送で,私はまだ福井刑務所にいましたが,こういうサスペンスドラマも録画で結構よく放送がありました。

\begin{itemize}
\tightlist
\item
  金沢能登殺人事件 (徳間文庫) \textbar{} 斎藤 栄 \textbar 本 \textbar{}
  通販 \textbar{} Amazon \url{https://t.co/RYYpy8DNzr} 
\end{itemize}

 価格が40円になっていました。まとめ買いをすれば暇つぶしもできそうですが,1冊40円というのはちょっと驚きました。

\begin{itemize}
\tightlist
\item
  social\_security\_system.pdf \url{https://t.co/badowQ3cH6} 
\end{itemize}

 PDFファイルですが,Googleの検索結果には「<訴訟記録の部>訴訟01-40 -
明治学院大学」となっていました。表形式の資料のようでしたが,分類のような項目に「弾圧・冤罪」とありました。星野雅美さんの最終弁論にあたっての上申書,というのも見えます。

\begin{itemize}
\tightlist
\item
  星野雅美 強制わいせつ - Google 検索 \url{https://t.co/e9BYVItjna}  ¥\n
  検索条件と十分に一致する結果が見つかりません。
\end{itemize}

 なにか見たことのないGoogleの検索結果のメッセージが出てきました。強制わいせつというのは先程のPDFファイルにあったものです。

\begin{itemize}
\tightlist
\item
  名古屋高等裁判所金沢支部 平成2年(う)27号 判決 - 大判例
  \url{https://t.co/N3UeFagQse} 
  本件控訴の趣意は、被告人名義、弁護人鳥毛美範名義及び弁護人梨木作次郎・同加藤喜一・同鳥毛美範・同西村依子・同飯森和彦共同名義の各控訴趣意書に、これに対する答弁は、検察官川又敬治名義の答弁書に、
\end{itemize}

 Googleで検索方法を変え,「梨木作次郎 強制わいせつ」とやると出てきたページですが,驚いたことに名古屋高裁金沢支部での逆転無罪判決となっていました。控訴審の事件番号が平成2年ですが,この判決の日付が見当たりません。懲役と検索しても該当がなく一審の量刑も不明です。

\begin{quote}
《引用の始まり》
\end{quote}

\begin{quote}
四  本件公訴事実の要旨

本件公訴事実の要旨は、「被告人は、昭和六三年一二月八日午前二時四五分ころ、石川県金沢市〈番地略〉先路上において、同所を通行中のH(当二四年)を認め、強いて同女にわいせつの行為をしようと企て、いきなり同女の後方から抱きついた上、同所先の陸橋下に引きずり込み、無理やり同女の着衣内に右手を差し入れて、同女の左乳房を弄び、もって、強いてわいせつの行為をなしたものである。」というものであり、検察官は更に、本件犯行の開始から終了までの時間は大体数分から一〇分間であり、被告人がHに最初に抱きついた地点から左乳房を弄んだ地点までの距離は約三五メートルである旨釈明した。
\end{quote}

\begin{quote}
《引用の終わり》
\end{quote}

\begin{itemize}
\tightlist
\item
  名古屋高等裁判所金沢支部 平成2年(う)27号 判決 -
  大判例 \url{https://daihanrei.com/l/\%E5\%90\%8D\%E5\%8F\%A4\%E5\%B1\%8B\%E9\%AB\%98\%E7\%AD\%89\%E8\%A3\%81\%E5\%88\%A4\%E6\%89\%80\%E9\%87\%91\%E6\%B2\%A2\%E6\%94\%AF\%E9\%83\%A8\%20\%E5\%B9\%B3\%E6\%88\%90\%EF\%BC\%92\%E5\%B9\%B4\%EF\%BC\%88\%E3\%81\%86\%EF\%BC\%89\%EF\%BC\%92\%EF\%BC\%97\%E5\%8F\%B7\%20\%E5\%88\%A4\%E6\%B1\%BAn} 
\end{itemize}

 公訴事実では,「昭和六三年一二月八日午前二時四五分ころ、石川県金沢市〈番地略〉先路上」「強いて同女にわいせつの行為をしようと企て、いきなり同女の後方から抱きついた上、同所先の陸橋下に引きずり込み、無理やり同女の着衣内に右手を差し入れて、同女の左乳房を弄び」とあります。

 PDFファイルではないのでページ数も不明ですが,かなりの長文の判決で,これを全て読む気にはなりません。やってみないとわからないですが,数時間は掛かりそうです。文字数を確認してみます。

 115535という結果でした。改行自体が少ないですが333行となっています。115,535文字の判決文です。

\begin{itemize}
\tightlist
\item
  裁判文書(裁判所提出書類)の標準的な書式,表記法
  \url{https://t.co/EFRAuloQ4b} 
  文字サイズは12ポイント,1行文字数は37字,1ページの行数は26行。
\end{itemize}

 115535 / (37 *
26)という計算をpythonで実行すると,120.0987525987526という結果でした。およそ120ページですが,これは折り返しの改行や空行をまったく考慮しない計算になります。普通の書面の形式だと130ページはありそうです。これは今日中に読みきれないかもしれません。

 平成4年の秋の控訴審初公判で対面した濱田武律裁判長の判決文というのも感慨深いです。控訴審の無罪判決ですが,冤罪を叫んだまま有罪になったものと思い込んでいました。最高裁での逆転有罪というのはほとんど聞かないですが,破棄差し戻しで有罪というのも聞かないと思います。

 完全に一度も記憶になかったのか断言はできないですが,新聞配達員でも赤旗の新聞配達員と判決文にありました。ネット上の情報が乏しいですが,確定判決の確認ぐらいはしておきたいところです。

\begin{itemize}
\tightlist
\item
  梨木作次郎 - Wikipedia \url{https://t.co/kDpplars85}  ¥\n 梨木
  作次郎(なしき さくじろう、1907年9月24日 -
  1993年4月9日)は、日本の弁護士、社会運動家、政治家。元衆議院議員(日本共産党公認)。金沢弁護士会会長、自由法曹団常任幹事などを歴任する。
\end{itemize}

 忘れていたのかと思いますが,梨木作次郎弁護士も自由法曹団常任幹事となっていました。Wikipediaのページですが,前に見た時あったと思う舞台俳優のような若い頃の梨木作次郎弁護士の顔写真が見当たりません。

\begin{itemize}
\tightlist
\item
  梨木 作次郎とは - コトバンク \url{https://t.co/25vW5dwWX8}  ¥\n
  珠洲市蛸島町の小学生殺人事件、山中事件など数々の冤罪事件を手がけた他、イタイイタイ病訴訟弁護団の副団長もつとめた。また、23年小松製作所労働争議に関連、建造物侵入罪で起訴されたこともある。(名古屋高裁差し戻しで無罪確定)。
\end{itemize}

 「蒔絵職人・霜上則男の冤罪―山中温泉殺人事件」に梨木作次郎弁護士の名前は出てこなかったと思いますが,同じ金沢合同法律事務所ということで,関与していたのも自然な流れかと思います。ただ,今日まで山中事件と梨木作次郎弁護士を結びつけた情報は見なかったような気がします。

\begin{itemize}
\tightlist
\item
  新聞配達員 冤罪 金沢 - Google 検索 \url{https://t.co/sEltNcx1fY}  ¥\n
  最も的確な検索結果を表示するために、上の 9
  件と似たページは除外されています。
\end{itemize}

 どうも上記の検索結果をみるといくら頑張ってもネットで情報を見つけ出すのは難しそうです。平成2年の名古屋高裁金沢支部の控訴審で判決が何年何月にあったのか確認したかったのですが,宇出津図書館で北國新聞縮小版から探す時の対象範囲が絞り込めませんでした。

 同所先の右地方道の通称神田陸橋の北西側の橋梁下である、新神田二丁目線六号道路と新神田三丁目線七号道路とを結ぶ道路が右地方道と立体交差してガード状態になっている部分に、同女を背後から抱えるようにして連れて行って引きずり込み」というのも目に入りました。

 昭和六三年一二月八日午前二時四五分ころとあるので,私が近くの東力2丁目に住んでいた頃のことです。神田陸橋のガードしたというのは昼でも人通りは少なく死角の多い場所になるかと思います。深夜2時45分頃に24歳の女性が一人歩きをしていたというのも珍しく思いました。

\begin{quote}
《引用の始まり》
\end{quote}

\begin{quote}
関係各証拠のうち、深夜現場道路を一人通行中被告人に襲われたとして本件公訴事実に沿う被害状況等を供述するH証言及び当時たまたま現場付近に通り合わせて悲鳴を聞くとともに本件犯行の一部と思われる状況を目撃し、そこから逃走する被告人をわいせつ行為に及んだ犯人として追跡して捕らえたというY証言は、いずれも信用するに足るものであり、また、逮捕直後、被告人が司法警察員に本件犯行を自白した弁解録取書は、その作成経過や供述内容に照らして任意性に問題はなく、信用性にも欠けるところがないのに対し、これを否定する被告人の原審公判供述、つまり、被告人は、日本共産党の赤旗新聞の配達業務に従事していたものであるところ、当日午前二時過ぎころ自動車で新聞配達に出向く途中、本件現場付近の前記地方道の上り線側道の路上に駐車して仮眠を取ったが、眠気が覚めなかったので近くのスーパー店(サークルK)に缶コーヒーを買いに行くことにし、小走りで同側道を本件ガード入口付近まで来て、たまたま被告人の先を同方向に歩いていたHに追いつき、そのまま追い抜こうとしてその右斜め約1.5メートル付近に迫ったとき、突然、同女が悲鳴を上げたため、逆に被告人の方が驚き、深夜のことでトラブルに巻き込まれることを恐れて、思わずその場から走り去ったものであり、弁解録取書は、やったと言えば身柄拘束を解くかのような捜査官の利益誘導により事実に反する供述をしたものであるなどという弁解(以下、「被告人弁解」という。)は、その内容自体が不自然、不合理なものであり、前記H、Y証言と対比してみても信用することはできない。
\end{quote}

\begin{quote}
《引用の終わり》
\end{quote}

\begin{itemize}
\tightlist
\item
  名古屋高等裁判所金沢支部 平成2年(う)27号 判決 -
  大判例 \url{https://daihanrei.com/l/\%E5\%90\%8D\%E5\%8F\%A4\%E5\%B1\%8B\%E9\%AB\%98\%E7\%AD\%89\%E8\%A3\%81\%E5\%88\%A4\%E6\%89\%80\%E9\%87\%91\%E6\%B2\%A2\%E6\%94\%AF\%E9\%83\%A8\%20\%E5\%B9\%B3\%E6\%88\%90\%EF\%BC\%92\%E5\%B9\%B4\%EF\%BC\%88\%E3\%81\%86\%EF\%BC\%89\%EF\%BC\%92\%EF\%BC\%97\%E5\%8F\%B7\%20\%E5\%88\%A4\%E6\%B1\%BAn} 
\end{itemize}

 昭和63年というのは今からおよそ33年前なので全体的な記憶が不鮮明にはなっているものの,神田陸橋の近くに,仮眠していた車から小走りに眠気覚ましの缶コーヒーを買いに向かうのような「スーパー店(サークルK)」があったのか甚だ疑問です。コンビニともなっていません。

\begin{itemize}
\tightlist
\item
  セブン-イレブン 金沢米泉10丁目店 - Google マップ
  \url{https://t.co/b4Iwlo8xvQ}  〒921-8044 石川県金沢市米泉町10丁目39−12
\end{itemize}

 昭和63年12月当時,神田陸橋の近くにサークルKのコンビニがあったとして考えられる最寄りが,大体ですが現在,上記のセブン-イレブン
金沢米泉10丁目店となっている場所です。保古町かと思っていましたが,米泉というのは意外でした。

 神田陸橋から増泉方面のことはよくわかりませんが,大豆田大橋方面というのは金沢中央卸売市場から深夜に仕事が終わって東力2丁目のアパートに帰る通り道でもあったので,途中にコンビニがあったとは記憶にないです。なんとかディリ−ストアのようなものはあったかも。

 「たまたま現場付近に通り合わせて悲鳴を聞くとともに本件犯行の一部と思われる状況を目撃し、そこから逃走する被告人をわいせつ行為に及んだ犯人として追跡して捕らえた」という経過がありながら,控訴審で逆転無罪になったというのもすごい判決だと思います。

 「被告人はHの悲鳴に逆に驚いてトラブルを避けるため逃げたというが、畑やたんぼにも入り、他人の店の看板などを伝って建物の上まで至るという逃走経路や形態は、被告人がいうように何にもしていない者の行動としては余りに不自然であり」ともあるのですが,これは一審の事実認定なのでしょう。

\begin{quote}
《引用の始まり》
\end{quote}

\begin{quote}
六  当裁判所の判断

はじめに

本件においては、右のように、H証言やY証言と被告人弁解とが極端に対立して犯罪の成否が争われている場合であるところ、原判決は、前述のとおり、被告人供述の弁解内容は不自然、不合理で措信するに足りないものであり、H、Y両証言は、細部において不明瞭、不正確な点があり、前後の供述が一致しない部分もないわけではないが、全体的には十分信用することができ、また、被告人の弁解録取書も措信するに足るものとして、有罪の認定をするのであるが、相対立するこれら各供述の信用性を検討するに当たっては、単に全体的観察から窺われるその供述自体の一般的な特徴や傾向に頼るだけではなく、他の客観的な証拠や状況との整合性を十分に吟味し、経験則に照らしての合理性の有無を考究し、また供述変遷の状況とその理由を確かめ、時には供述の真実を歪める特別の状況の存否にも配慮しながら、相互の対立点の疑問を慎重に質していかなければならないのはいうまでもない。
\end{quote}

\begin{quote}
《引用の終わり》
\end{quote}

\begin{itemize}
\tightlist
\item
  名古屋高等裁判所金沢支部 平成2年(う)27号 判決 -
  大判例 \url{https://daihanrei.com/l/\%E5\%90\%8D\%E5\%8F\%A4\%E5\%B1\%8B\%E9\%AB\%98\%E7\%AD\%89\%E8\%A3\%81\%E5\%88\%A4\%E6\%89\%80\%E9\%87\%91\%E6\%B2\%A2\%E6\%94\%AF\%E9\%83\%A8\%20\%E5\%B9\%B3\%E6\%88\%90\%EF\%BC\%92\%E5\%B9\%B4\%EF\%BC\%88\%E3\%81\%86\%EF\%BC\%89\%EF\%BC\%92\%EF\%BC\%97\%E5\%8F\%B7\%20\%E5\%88\%A4\%E6\%B1\%BAn} 
\end{itemize}

\begin{quote}
《引用の始まり》
\end{quote}

\begin{quote}
(エ) Hの交際関係と証言態度について

その他にも、H証言の信用性に影響する事情として、同女の交際関係とこれに関わる証言態度にも触れないわけにはいかない。

すなわち、Hは、原審証言中に前記Yらとともに被告人を追い詰めて逮捕した際、被告人に対し「私の彼氏は警察だから」と言った旨認める証言をしたものの、それは単なる脅しであるとして交際事実は頭から否定していたところ、当審において弁護人側から同女がかって現職警察官と共同生活までも含んだ親密な男女の交際があり、その関係は現在も継続しているらしいことの証拠書類が請求されて取り調べられ、Hも改めての当審証言ではあえてこれを否定しなかったことからすると、本件事件当時においてもその警察官との交際があったことが推測され、原審証言に際しては、ことはプライバシーに関わる問題として事実を秘匿して置きたかったという立場はあったかもしれないが、本件犯行の成否を決める大事な証言を行う者としてはやはり不誠実な供述態度というほかなく、これがいろいろと不審点が多い本件に関する同女の証言の背景に疑念を生じさせることになるのもやむを得ない。
\end{quote}

\begin{quote}
《引用の終わり》
\end{quote}

\begin{itemize}
\tightlist
\item
  名古屋高等裁判所金沢支部 平成2年(う)27号 判決 -
  大判例 \url{https://daihanrei.com/l/\%E5\%90\%8D\%E5\%8F\%A4\%E5\%B1\%8B\%E9\%AB\%98\%E7\%AD\%89\%E8\%A3\%81\%E5\%88\%A4\%E6\%89\%80\%E9\%87\%91\%E6\%B2\%A2\%E6\%94\%AF\%E9\%83\%A8\%20\%E5\%B9\%B3\%E6\%88\%90\%EF\%BC\%92\%E5\%B9\%B4\%EF\%BC\%88\%E3\%81\%86\%EF\%BC\%89\%EF\%BC\%92\%EF\%BC\%97\%E5\%8F\%B7\%20\%E5\%88\%A4\%E6\%B1\%BAn} 
\end{itemize}

\begin{quote}
《引用の始まり》
\end{quote}

\begin{quote}
結局のところ、その際被告人とHとの間にどのような出来事があったかを明らかに知ることまではできないのであるが、以上検討の結果としては、被告人が本件公訴事実の範囲内で強制わいせつの行為を行ったことを証拠上は認めることはできないと結論すべきものであって、それ以上の真相を探索することは、H証言についての疑問の解明も含めて当裁判所の任務とするところではない。

七  結語

以上の次第であって、本件公訴事実について、これを被告人の行為として認定するには証拠上合理的疑いを拭い切れず、その犯罪の証明は不十分なものといわざるを得ないのであって、これを有罪と認定した原判決は、証拠の評価を誤って事実を誤認したものというほかなく、これが判決に影響することは明らかであり、破棄を免れない。論旨は理由がある。

よって、刑事訴訟法三九七条一項、三八二条により原判決を破棄し、同法四〇〇条但書により更に判決することとするが、本件公訴事実については犯罪の証明がないので、同法三三六条により無罪の言渡しをすることとし、主文のとおり判決する。

(裁判長裁判官濱田武律 裁判官井垣敏生 裁判官田中敦)
\end{quote}

\begin{quote}
《引用の終わり》
\end{quote}

\begin{itemize}
\tightlist
\item
  名古屋高等裁判所金沢支部 平成2年(う)27号 判決 -
  大判例 \url{https://daihanrei.com/l/\%E5\%90\%8D\%E5\%8F\%A4\%E5\%B1\%8B\%E9\%AB\%98\%E7\%AD\%89\%E8\%A3\%81\%E5\%88\%A4\%E6\%89\%80\%E9\%87\%91\%E6\%B2\%A2\%E6\%94\%AF\%E9\%83\%A8\%20\%E5\%B9\%B3\%E6\%88\%90\%EF\%BC\%92\%E5\%B9\%B4\%EF\%BC\%88\%E3\%81\%86\%EF\%BC\%89\%EF\%BC\%92\%EF\%BC\%97\%E5\%8F\%B7\%20\%E5\%88\%A4\%E6\%B1\%BAn} 
\end{itemize}

 ナカヤビル前というのが何度も出てきて被害者が最初に襲われ,ガード入り口まで連行されたとあります。最初の方に神田陸橋の北西とあったと思うのですが,これは犀川方面の海側の方向になると思います。この道路は本来,野田専光寺線になるはずと思うのですが,見当たりませんでした。

 Googleマップで調べても金沢市内にナカヤビルというのは見当たりませんでした。丁度,神田陸橋の先のその辺りには家電店の店舗があって前がバス停になっていました。被告発人安田敏と片町で会うのに乗ったバス停です。

 結局,強制わいせつの女性被害者の証言を信用できないと結論づけ無罪としているようです。ずいぶんと強引な論理ですが,こんな判決も実際にあるのかと勉強になりました。被害女性が現職の警察官と交際している可能性も信用できない理由としていました。

 「当夜はアルバイト先である片町のスナックの仕事を終えたのち、酔いざましを兼ね、夜間には初めて通る道を一人で自宅まで歩いて帰ることとしたこと、店から四〇分くらいも歩いて本件現場付近の神田陸橋の上り線側道に差しかかったとき」とあります。

 神田陸橋を歩いて渡ったことはないのですが,平成15年の1月に行った関係者KYNの配管設備の会社の事務所は,ちょうど神田陸橋のました辺りで,線路の向こう側に,話に聞いていて前にも見たことのある暴力団滝本組の事務所という一軒家のような建物がありました。

 その関係者KYNの配管設備工事の会社の事務所というのは,かなり古びた建物で,昭和30年代に貨物の取り扱い所として使われていたような雰囲気があって,古い駅舎の一部のようにも思えました。

 見出しの変更というか追記を行いました。

\begin{itemize}
\tightlist
\item
  2021年05月13日16時00分の登録:
  REGEXP:''袴田事件''/データベース登録済みツイートの検索:2021-05-13〜2021-05-13/2021年05月13日16時00分の記録:ユーザ・投稿:4/6件
  \url{https://kk2020-09.blogspot.com/2021/05/regexp2021-05-132021-05-1320210513160046.html} 
\end{itemize}

 時刻は19時20分です。いつのまにか19時を過ぎていましたが,窓の外を見ると,まだ完全には夜になっていない一歩手前で,わずかに薄明るさが残っているように見えます。

2021年05月13日19時20分の実行記録: twitterAPI-search-lawList-mydql-add.rb
``袴田事件'' ツイート数:14/2412 リツイート数:3/2412
トータル:84``袴田事件''の該当: hirono\_hideki 4/0件 kk\_hirono 8/0件
s\_hirono 1/0件

 先程もまとめ記事を作成し,ユーザ・投稿:4/6件という結果でしたが,TwitterAPIの検索でトータル84件のツイートというのは信じられないような反応の乏しさです。「袴田事件」の検索結果で,朝早くにデイリー新潮の新着記事がありました。

 追加されたツイートのアカウントというのは,「ユーザ名称:宮台真司
{[}screen\_name{]}ユーザ名:miyadai フォロー数:1042
フォロワー数:155394
ツイート数:132786」だけのようです。他に私のツイートの分だけ増えたのか6件だったのが16件になっています。

\begin{itemize}
\tightlist
\item
  2021年05月13日19時25分の登録:
  REGEXP:''袴田事件''/データベース登録済みツイートの検索:2021-05-13〜2021-05-13/2021年05月13日19時24分の記録:ユーザ・投稿:7/16件
  \url{https://kk2020-09.blogspot.com/2021/05/regexp2021-05-132021-05.html} 
\end{itemize}

\begin{quote}
《引用の始まり》
\end{quote}

\begin{quote}
アカウント名 ツイート数 リツイート数デイリー新潮(dailyshincho) 1
0Fukuda Hayato(hayato821) 0
1刑事告発・非常上告_金沢地方検察庁御中(kk\_hirono) 8
0ふなざわひろゆき(FLetlRmdM7gs5vS) 0
1奉納\さらば弁護士鉄道・泥棒神社の物語(hirono\_hideki) 3
0非常上告-最高検察庁御中\_ツイッター(s\_hirono) 1 0宮台真司(miyadai)
0 1
\end{quote}

\begin{quote}
《引用の終わり》
\end{quote}

\begin{itemize}
\item
  奉納\危険生物・弁護士脳汚染除去装置\金沢地方検察庁御中\_2020:
  REGEXP:''袴田事件''/データベース登録済みツイートの検索:2021-05-13〜2021-05-13/2021年05月13日19時24分の記録:ユーザ・投稿:7/16件 \url{https://kk2020-09.blogspot.com/2021/05/regexp2021-05-132021-05.htmln} 
\item
  (14/16) RT
  miyadai(宮台真司)|videonewscom(ビデオニュース・ドットコム)
  日時:2021-05-13 16:43:44 +0900/2021-01-09 20:42:00 +0900 URL:
  \url{https://twitter.com/miyadai/status/1392747375022665734} 
  \url{https://twitter.com/videonewscom/status/1347871351843954691\textgreater} {}
  マル激トーク・オン・ディマンド
  第1031回(2021年1月9日)『録音テープが物語る袴田事件の真実』ゲスト:\#浜田寿美男
  氏(心理学者・袴田事件取り調べ録音テープ鑑定人)司会:\#神保哲生、\#宮台真司・・・
\end{itemize}

 ここ数日やたらとデイリー新潮の記事を見かけているとも思うのですが,間が長く見かけなかった時期もあるように思います。法クラの誰かが記事を見つけてツイートとなりリツイートをすれば,いっきに数が増える可能性というのもあるのかもしれません。

\begin{itemize}
\tightlist
\item
  2021年05月13日19時35分の登録:
  「デイリー新潮」を@hirono\_hideki @kk\_hirono @s\_hironoで検索 337件の該当 2021-05-13\_19:35の記録
  \url{https://kk2020-09.blogspot.com/2021/05/hironohidekikkhironoshirono3372021-05.html} 
\end{itemize}

2021-05-02 08:42:45 ``-
「メーテル」のイラストをヤフオクに無断出品 「銀河鉄道999」アニメーターの強欲ぶり
\textbar{} デイリー新潮
\url{https://www.dailyshincho.jp/article/2019/07300557/?all=1''} 
\url{https://twitter.com/hirono\_hideki/status/1388640062233972736} 

2021-05-12 19:40:09 ``-
野田聖子の夫は「元暴力団員」と裁判所が認定 約10年間組員として活動
\textbar{} デイリー新潮 \url{https://t.co/ix4NG16i3x''} 
\url{https://twitter.com/hirono\_hideki/status/1392429380895342594} 

 5月2日の次が5月12日となっていました。

 時刻は19時42分です。ふと置いてある正木ひろし弁護士の本に目をやると,ページが束になった真下の部分に「能都町中央図書館」とハンコがあり,裏面を見ると下の方にバーコード付きのシールがあって,そこには「能登町立中央図書館」とありました。

 町民にはおなじみの旧鳳至郡能都町と現在の鳳珠郡能登町の違いがあるのですが,柳田村,珠洲郡内浦町との合併で能登町になったのは平成17年のことです。「正木ひろし 事件・信念・自伝」という日本図書センターの本で,人間の記録というシリーズの119という番号もあります。

 再捜査要請書_警察庁・石川県警察御中(@kk\_hirono)のアカウントなのでブロックされたアカウントのツイートは表示されないですが,Twitterで検索し,弁護士やジャーナリストのツイートがあればリツイートをしたいと思います。APIより確実かもしれません。

\begin{itemize}
\item
  \begin{enumerate}
  \def\labelenumi{(\arabic{enumi})}
  \setcounter{enumi}{9}
  \tightlist
  \item
    袴田事件 - Twitter検索 / Twitter \url{https://t.co/ke75DswgUt} 
  \end{enumerate}
\end{itemize}

〉〉〉 kk\_hironoのリツイート 〉〉〉

\begin{itemize}
\tightlist
\item
  RT
  kk\_hirono(刑事告発・非常上告_金沢地方検察庁御中)|todateyoshiyuki(弁護士戸舘圭之オフィシャル/とってぃ/袴田事件弁護団)
  日時:2021-05-13 19:53/2021/05/12 19:16 URL:
  \url{https://twitter.com/kk\_hirono/status/1392795028230004739} 
  \url{https://twitter.com/todateyoshiyuki/status/1392423466431696901} 
  \textgreater{} @Y\_Tochikawa
  ありがとうございます。ちゃんとした研究の成果発表なんですね。
\end{itemize}

〉〉〉 kk\_hironoのリツイート 〉〉〉

\begin{itemize}
\tightlist
\item
  RT
  kk\_hirono(刑事告発・非常上告_金沢地方検察庁御中)|todateyoshiyuki(弁護士戸舘圭之オフィシャル/とってぃ/袴田事件弁護団)
  日時:2021-05-13 19:53/2021/05/12 18:50 URL:
  \url{https://twitter.com/kk\_hirono/status/1392795039613345792} 
  \url{https://twitter.com/todateyoshiyuki/status/1392416881072504833} 
  \textgreater{}
  大友良英さんのジャズトゥナイト大好きなんですよね。私は、ジャズに詳しくないんですが、ほんとにジャズが好きで好きでたまらない感じが言葉の端々からあふれでていて、聴いているこっちまで楽しくなってきちゃいます。NHKラジオFMで毎週放送していますがらじるらじるでいつも聴いています。
\end{itemize}

 ちょっとおかしいと思ったのですが,ブラウザでのツイートの検索は,アカウントのプロフィールの名前もヒットするのでした。TwitterAPIの検索では確認していない現象かもしれません。そういえば不思議と戸舘圭之弁護士のタイムラインというのも長く閲覧がご無沙汰でした。

\begin{itemize}
\item
  \begin{enumerate}
  \def\labelenumi{(\arabic{enumi})}
  \setcounter{enumi}{9}
  \tightlist
  \item
    弁護士戸舘圭之オフィシャル/とってぃ/袴田事件弁護団(@todateyoshiyuki)さんの返信があるツイート
    / Twitter \url{https://t.co/JEwihDXKFj} 
  \end{enumerate}
\end{itemize}

〉〉〉 kk\_hironoのリツイート 〉〉〉

\begin{itemize}
\tightlist
\item
  RT
  kk\_hirono(刑事告発・非常上告_金沢地方検察庁御中)|todateyoshiyuki(弁護士戸舘圭之オフィシャル/とってぃ/袴田事件弁護団)
  日時:2021-05-13 19:57/2021/05/13 13:04 URL:
  \url{https://twitter.com/kk\_hirono/status/1392796167122026496} 
  \url{https://twitter.com/todateyoshiyuki/status/1392692220826771461} 
  \textgreater{}
  ひさしぶりに中央線乗って大久保付近を通りかかったら、見慣れていた某司法試験予備校の看板がなくなってる!!!
  ちょっとしょっく。
\end{itemize}

〉〉〉 kk\_hironoのリツイート 〉〉〉

\begin{itemize}
\tightlist
\item
  RT
  kk\_hirono(刑事告発・非常上告_金沢地方検察庁御中)|i\_yoshitatsu(今泉義竜)
  日時:2021-05-13 19:58/2021/05/13 09:20 URL:
  \url{https://twitter.com/kk\_hirono/status/1392796318230212612} 
  \url{https://twitter.com/i\_yoshitatsu/status/1392635739184058368} 
  \textgreater{} 素晴らしい勝訴。 \url{https://t.co/Z0t5rp0YF1} 
\end{itemize}

 引用のツイートが「このツイートはありません。」で表示されないと思ったら,再捜査要請書_警察庁・石川県警察御中(@kk\_hirono)のアカウントでもブロックされている神原元弁護士のツイートが引用ツイートとなっているようです。

※ @kk\_hironoのアカウントがブロックされ,リツイートに失敗したツイート

\begin{itemize}
\tightlist
\item
  TW kambara7(弁護士神原元) 日時:2021/05/12 20:05:05 URL:
  \url{https://twitter.com/kambara7/status/1392435656563908613} 
  \textgreater{} 当職の担当事件です。\\
  \textgreater{}\\
  \textgreater{}
  中学生にブログでヘイトスピーチ、60代の男性に損害賠償130万円の判決(BuzzFeed
  Japan)\\
  \textgreater{} \#Yahooニュース\\
  \textgreater{} \url{https://t.co/SVcLRm0SSA} 
\end{itemize}

 リツイートに失敗している上記の神原元弁護士のtwitterですが,現時点で,1,424のリツイート,59件の引用ツイート,2,225件のいいねとなっています。回収の見込があるのかも不明な130万円の賠償認容判決ですが,もちろん弁護士報酬,その他の費用も不明です。

\begin{itemize}
\item
  TW kambara7(弁護士神原元) 日時: 2021/05/12 21:02:25 URL:
  \url{https://twitter.com/kambara7/status/1392450084088864769} 
  \textgreater{}
  本件判決の意義は二つあります。第一に、人種差別を「賠償額の算定要素」ではなく、違法性の根拠そのものとしたこと。これは京都朝鮮学校襲撃事件判決の論理を乗り越えるものです。\\
  \textgreater{}\\
  \textgreater{}
  第二に、ひとつの書き込みで130万遠藤という高額な賠償を認めたこと。名誉毀損認定なしでこの額は高額です。
\item
  TW kambara7(弁護士神原元) 日時: 2021/05/12 21:29:43 URL:
  \url{https://twitter.com/kambara7/status/1392456954509873154} 
  \textgreater{}
  裁判を起こした時の幼い少年は立派な青年に成長しました。そして実名を堂々と公表し、差別を無くすよう、社会に強く訴えたのです。\\
  \textgreater{}\\
  \textgreater{}
  正義感と勇気を持った依頼者に出会えることこそ、弁護士の最大の喜びです。今日は弁護士冥利に尽きる一日でした。
\item
  TW kambara7(弁護士神原元) 日時: 2021/05/12 23:06:39 URL:
  \url{https://twitter.com/kambara7/status/1392481351337529344} 
  \textgreater{} 俺は全国の弁護士に呼びかけるのだが、\\
  \textgreater{}\\
  \textgreater{} 「ヘイトスピーチ=高額賠償」\\
  \textgreater{}\\
  \textgreater{} という流れを確立しよう。\\
  \textgreater{}\\
  \textgreater{} 一つの書き込みで100万円を相場にするのは悪くない。\\
  \textgreater{}\\
  \textgreater{} 誰もヘイトスピーチをしなくなるだろう。
\end{itemize}

〉〉〉 kk\_hironoのリツイート 〉〉〉

\begin{itemize}
\tightlist
\item
  RT
  kk\_hirono(刑事告発・非常上告_金沢地方検察庁御中)|kojin\_syugi(橋本太地(弁護士・あなたのみかた法律事務所))
  日時:2021-05-13 20:09/2021/05/12 23:28 URL:
  \url{https://twitter.com/kk\_hirono/status/1392799072872910848} 
  \url{https://twitter.com/kojin\_syugi/status/1392486801231679490} 
  \textgreater{} @kambara7 おめでとうございます!
\end{itemize}

 奉納\さらば弁護士鉄道・泥棒神社の物語(@hirono\_hideki)のアカウントではブロックされていたように思う橋本太地弁護士のツイートですが,再捜査要請書_警察庁・石川県警察御中(@kk\_hirono)ではリツイートが出来ました。前にも出来ていたようには思っていました。

〉〉〉 kk\_hironoのリツイート 〉〉〉

\begin{itemize}
\tightlist
\item
  RT
  kk\_hirono(刑事告発・非常上告_金沢地方検察庁御中)|ngc2497(ngc2497)
  日時:2021-05-13 20:11/2021/05/12 21:14 URL:
  \url{https://twitter.com/kk\_hirono/status/1392799532157521921} 
  \url{https://twitter.com/ngc2497/status/1392453155456045062} 
  \textgreater{} @kambara7 勝訴おめでとうございます👏👏👏
  以下の続報ですね。神原弁護士は情けをお持ちです(英字紙は容赦なく本名晒しているのですから)
  \url{https://t.co/88VkYdsAEm} 
\end{itemize}

〉〉〉 kk\_hironoのリツイート 〉〉〉

\begin{itemize}
\tightlist
\item
  RT
  kk\_hirono(刑事告発・非常上告_金沢地方検察庁御中)|ngc2497(ngc2497)
  日時:2021-05-13 20:11/2021/05/12 21:16 URL:
  \url{https://twitter.com/kk\_hirono/status/1392799664395603973} 
  \url{https://twitter.com/ngc2497/status/1392453750288109573} 
  \textgreater{} @kambara7 もう一つぶら下げて置きますです。
  \url{https://t.co/iyF8muy5Xk} 
\end{itemize}

〉〉〉 kk\_hironoのリツイート 〉〉〉

\begin{itemize}
\tightlist
\item
  RT
  kk\_hirono(刑事告発・非常上告_金沢地方検察庁御中)|aritayoshifu(有田芳生)
  日時:2021-05-13 20:11/2019/01/17 01:58 URL:
  \url{https://twitter.com/kk\_hirono/status/1392799726492258308} 
  \url{https://twitter.com/aritayoshifu/status/1085581975757484034} 
  \textgreater{}
  大分の66歳男性は英字紙では名前が報道されました。もともと中学生(当時)の実名をネットに書いて差別の煽動を行なったのですから、刑事罰を受けたこの男性も「タケシタユウジ」という名前は明らかにされて当然でしょう。
  \url{https://t.co/vGmpnWSaGK} 
\end{itemize}

〉〉〉 kk\_hironoのリツイート 〉〉〉

\begin{itemize}
\tightlist
\item
  RT
  kk\_hirono(刑事告発・非常上告_金沢地方検察庁御中)|ishibs\_kanagawa(石橋学)
  日時:2021-05-13 20:12/2019/01/17 01:39 URL:
  \url{https://twitter.com/kk\_hirono/status/1392799864736477188} 
  \url{https://twitter.com/ishibs\_kanagawa/status/1085577349817487360} 
  \textgreater{}
  「人間として認めないという侮辱ではすまない傷を負わせた。匿名でも処罰されることが示された」。ヘイトクライムに対応していない量刑の軽さ、法の不備についても書きました。在日コリアンの中学生を侮辱 匿名ブログの66歳男性に略式命令 川崎簡裁
  \url{https://t.co/532ucrzOSe}  \#神奈川新聞
\end{itemize}

 神原元弁護士のツイートの返信欄では,橋本太地弁護士の他に弁護士,法クラと確認できる返信ツイートが見当たりませんでした。130万円で高額の賠償とありますが,控訴審の判決までどれだけの手間と費用,時間が掛かったのかと考えると敗訴のリスクを含め,採算が弁護士的にどうなのかとも思えます。

〉〉〉 kk\_hironoのリツイート 〉〉〉

\begin{itemize}
\tightlist
\item
  RT
  kk\_hirono(刑事告発・非常上告_金沢地方検察庁御中)|todateyoshiyuki(弁護士戸舘圭之オフィシャル/とってぃ/袴田事件弁護団)
  日時:2021-05-13 20:21/2021/05/12 23:54 URL:
  \url{https://twitter.com/kk\_hirono/status/1392802273370152967} 
  \url{https://twitter.com/todateyoshiyuki/status/1392493356744286212} 
  \textgreater{}
  羞恥心よりみんなに歌声を聴かせて知らしめたいといういわば周知心ですね!!
  \url{https://t.co/6huY6k7fus} 
\end{itemize}

 Twitterのプロフィールに袴田事件弁護団とある戸舘圭之弁護士が,デイリー新潮の記事をツイートしていないのも不思議ですが,なにか幻を見ていたような気分にもなってきました。スクリーンショットの記録はいくつか残してあると思います。

 大崎事件ほど真犯人の特定が具体的ではないですが,大崎事件の再審請求の流れと袴田事件も似たような展開となってきました。それも飛躍的です。ずいぶん前から真犯人と指摘され,袴田巌さんの釈放当日か翌日に自殺したという長女の嫌疑は明確に否定され,これも大きな動きに思えました。

 「袴田事件」のブラウザでのTwitter検索を「話題のツイート」から「最新」に切り替えました。タイムラインの上の方からいくつか再捜査要請書_警察庁・石川県警察御中(@kk\_hirono)でリツイートをしたいと思います。

〉〉〉 kk\_hironoのリツイート 〉〉〉

\begin{itemize}
\tightlist
\item
  RT
  kk\_hirono(刑事告発・非常上告_金沢地方検察庁御中)|Zombies99614927(雷の鱗模様)
  日時:2021-05-13 20:33/2021/05/12 15:41 URL:
  \url{https://twitter.com/kk\_hirono/status/1392805120522428417} 
  \url{https://twitter.com/Zombies99614927/status/1392369391698800643} 
  \textgreater{} \#ゴゴスマ
  今イチケイのカラスでもやってるよね。日本の裁判の形式優先の問題を。袴田事件でも明らかになった裁判官の年功序列のお約束優先判決〜良い加減司法を根底から変えて行かないと日本は駄目になる。
  東京オリ・パラやって〜1回日本民族が死滅すると良いかもなw
\end{itemize}

〉〉〉 kk\_hironoのリツイート 〉〉〉

\begin{itemize}
\tightlist
\item
  RT
  kk\_hirono(刑事告発・非常上告_金沢地方検察庁御中)|todateyoshiyuki(弁護士戸舘圭之オフィシャル/とってぃ/袴田事件弁護団)
  日時:2021-05-13 20:33/2021/05/12 12:51 URL:
  \url{https://twitter.com/kk\_hirono/status/1392805183885692928} 
  \url{https://twitter.com/todateyoshiyuki/status/1392326429837193217} 
  \textgreater{} ぼくはな、おにぎりがすきなんだな。
  \url{https://t.co/O05r2kqFHa} 
\end{itemize}

〉〉〉 kk\_hironoのリツイート 〉〉〉

\begin{itemize}
\tightlist
\item
  RT
  kk\_hirono(刑事告発・非常上告_金沢地方検察庁御中)|UrokoDrop(目から鱗の面白ネタ)
  日時:2021-05-13 20:34/2021/05/12 09:09 URL:
  \url{https://twitter.com/kk\_hirono/status/1392805329071599616} 
  \url{https://twitter.com/UrokoDrop/status/1392270690510770181} 
  \textgreater{}
  袴田事件「出火直後、寮に」元同僚証言,冤罪にまた一歩近づく!\url{https://t.co/ThWdFkDGjm} 
\end{itemize}

〉〉〉 kk\_hironoのリツイート 〉〉〉

\begin{itemize}
\tightlist
\item
  RT
  kk\_hirono(刑事告発・非常上告_金沢地方検察庁御中)|WellbredWellfed(wellfed-wellbred)
  日時:2021-05-13 20:34/2021/05/11 18:07 URL:
  \url{https://twitter.com/kk\_hirono/status/1392805511515361284} 
  \url{https://twitter.com/WellbredWellfed/status/1392043746041241602} 
  \textgreater{}
  録音テープが物語る袴田事件の真実/浜田寿美男氏(心理学者・袴田事件取り調べ録音テープ鑑定人)(ビデオニュース・ドットコム)
  \url{https://t.co/gLxAw4oxaH} 
\end{itemize}

 デイリー新潮の記事が1件も見当たりませんでしたが,考えてみると私のアカウントのツイートというのも存在するはずです。TwitterのバグというのはときおりTwitterのトレンドでもみかけますが,不思議です。

 さきほどのツイートの直後に,「袴田事件」の検索結果のタイムラインに一度に沢山の更新があり,私のツイートが多かったですが,リツイートをしていくと38件のリツイートとなっていました。ツイートは読み込んでいますが,埋め込みツイートの都合で,別のエントリーとします。

\begin{itemize}
\tightlist
\item
  〈〈〈 2021/05/13 20:45:28 Linux Emacs: 〈〈〈
\end{itemize}

\hypertarget{ux88abux544aux767aux4ebaux5ca1ux7530ux9032ux5f01ux8b77ux58ebux306eux7b2c2ux56deux516cux5224ux8abfux66f8ux30c7ux30a4ux30eaux30fcux65b0ux6f6eux306eux88b4ux7530ux4e8bux4ef6ux5f01ux8b77ux56e3ux9577ux897fux5d8bux52ddux5f66ux3055ux3093ux306eux7d20ux9854-ux771fux72afux4ebaux306fux8b66ux5bdfux3068ux5229ux5bb3ux304cux4e00ux81f4ux3057ux3066ux3044ux305fux4ebaux7269ux3068ux3044ux3046ux8a18ux4e8bux3078ux306eux53cdux5fdc}{%
\paragraph{被告発人岡田進弁護士の第2回公判調書:デイリー新潮の「袴田事件、弁護団長「西嶋勝彦さん」の素顔 真犯人は「警察と利害が一致していた人物」」という記事への反応}\label{ux88abux544aux767aux4ebaux5ca1ux7530ux9032ux5f01ux8b77ux58ebux306eux7b2c2ux56deux516cux5224ux8abfux66f8ux30c7ux30a4ux30eaux30fcux65b0ux6f6eux306eux88b4ux7530ux4e8bux4ef6ux5f01ux8b77ux56e3ux9577ux897fux5d8bux52ddux5f66ux3055ux3093ux306eux7d20ux9854-ux771fux72afux4ebaux306fux8b66ux5bdfux3068ux5229ux5bb3ux304cux4e00ux81f4ux3057ux3066ux3044ux305fux4ebaux7269ux3068ux3044ux3046ux8a18ux4e8bux3078ux306eux53cdux5fdc}}

\begin{itemize}
\tightlist
\item
  〉〉〉 Linux Emacs: 2021/05/13 20:49:17 〉〉〉
\end{itemize}

:CATEGORIES: @kanazawabengosi \#金沢弁護士会 @JFBAsns
日本弁護士連合会(日弁連) \#法務省 @MOJ\_HOUMU \#袴田事件

〉〉〉 kk\_hironoのリツイート 〉〉〉

\begin{itemize}
\tightlist
\item
  RT
  kk\_hirono(刑事告発・非常上告_金沢地方検察庁御中)|kk\_hirono(刑事告発・非常上告_金沢地方検察庁御中)
  日時:2021-05-13 20:38/2021/05/13 20:32 URL:
  \url{https://twitter.com/kk\_hirono/status/1392806388796063745} 
  \url{https://twitter.com/kk\_hirono/status/1392805063978934275} 
  \textgreater{}
  「袴田事件」のブラウザでのTwitter検索を「話題のツイート」から「最新」に切り替えました。タイムラインの上の方からいくつか再捜査要請書_警察庁・石川県警察御中(@kk\_hirono)でリツイートをしたいと思います。
\end{itemize}

〉〉〉 kk\_hironoのリツイート 〉〉〉

\begin{itemize}
\tightlist
\item
  RT
  kk\_hirono(刑事告発・非常上告_金沢地方検察庁御中)|kk\_hirono(刑事告発・非常上告_金沢地方検察庁御中)
  日時:2021-05-13 20:38/2021/05/13 20:28 URL:
  \url{https://twitter.com/kk\_hirono/status/1392806398572986374} 
  \url{https://twitter.com/kk\_hirono/status/1392803878794534917} 
  \textgreater{}
  大崎事件ほど真犯人の特定が具体的ではないですが,大崎事件の再審請求の流れと袴田事件も似たような展開となってきました。それも飛躍的です。ずいぶん前から真犯人と指摘され,袴田巌さんの釈放当日か翌日に自殺したという長女の嫌疑は明確に否定され,これも大きな動きに思えました。
\end{itemize}

〉〉〉 kk\_hironoのリツイート 〉〉〉

\begin{itemize}
\tightlist
\item
  RT kk\_hirono(刑事告発・非常上告_金沢地方検察庁御中)|y\_iOwO(🧃)
  日時:2021-05-13 20:38/2021/05/13 20:24 URL:
  \url{https://twitter.com/kk\_hirono/status/1392806409834618884} 
  \url{https://twitter.com/y\_iOwO/status/1392802938720899073} 
  \textgreater{}
  袴田さんのことを考えたら私はまだまだ子供で、何も力は無いけど。どう見たって袴田事件が
  ``冤罪``
  なことは分かります。司法側は裏社会の人に殺されたくないから認めていない様に感じる。冤罪を認め、厳重な処罰を受けろ。警察側のした事は許されない。なにが「信用できる日本の警察」だよ馬鹿か😔
\end{itemize}

〉〉〉 kk\_hironoのリツイート 〉〉〉

\begin{itemize}
\tightlist
\item
  RT
  kk\_hirono(刑事告発・非常上告_金沢地方検察庁御中)|kk\_hirono(刑事告発・非常上告_金沢地方検察庁御中)
  日時:2021-05-13 20:38/2021/05/13 20:23 URL:
  \url{https://twitter.com/kk\_hirono/status/1392806442684456964} 
  \url{https://twitter.com/kk\_hirono/status/1392802713797238784} 
  \textgreater{}
  Twitterのプロフィールに袴田事件弁護団とある戸舘圭之弁護士が,デイリー新潮の記事をツイートしていないのも不思議ですが,なにか幻を見ていたような気分にもなってきました。スクリーンショットの記録はいくつか残してあると思います。
\end{itemize}

〉〉〉 kk\_hironoのリツイート 〉〉〉

\begin{itemize}
\tightlist
\item
  RT
  kk\_hirono(刑事告発・非常上告_金沢地方検察庁御中)|kk\_hirono(刑事告発・非常上告_金沢地方検察庁御中)
  日時:2021-05-13 20:38/2021/05/13 19:57 URL:
  \url{https://twitter.com/kk\_hirono/status/1392806456924143624} 
  \url{https://twitter.com/kk\_hirono/status/1392796024607969284} 
  \textgreater{} (10)
  弁護士戸舘圭之オフィシャル/とってぃ/袴田事件弁護団(@todateyoshiyuki)さんの返信があるツイート
  / Twitter \url{https://t.co/JEwihDXKFj} 
\end{itemize}

〉〉〉 kk\_hironoのリツイート 〉〉〉

\begin{itemize}
\tightlist
\item
  RT
  kk\_hirono(刑事告発・非常上告_金沢地方検察庁御中)|kk\_hirono(刑事告発・非常上告_金沢地方検察庁御中)
  日時:2021-05-13 20:38/2021/05/13 19:51 URL:
  \url{https://twitter.com/kk\_hirono/status/1392806470027075587} 
  \url{https://twitter.com/kk\_hirono/status/1392794680203497476} 
  \textgreater{} (10) 袴田事件 - Twitter検索 / Twitter
  \url{https://t.co/ke75DswgUt} 
\end{itemize}

〉〉〉 kk\_hironoのリツイート 〉〉〉

\begin{itemize}
\tightlist
\item
  RT
  kk\_hirono(刑事告発・非常上告_金沢地方検察庁御中)|phosph\_Leonida(宮澤波瑠/通知不具合発生中🔔)
  日時:2021-05-13 20:38/2021/05/13 19:49 URL:
  \url{https://twitter.com/kk\_hirono/status/1392806496082141190} 
  \url{https://twitter.com/phosph\_Leonida/status/1392794035236990984} 
  \textgreater{}
  袴田事件、弁護団長「西嶋勝彦さん」の素顔 真犯人は「警察と利害が一致していた人物」
  \url{https://t.co/uzFvpeOsLp} 
\end{itemize}

〉〉〉 kk\_hironoのリツイート 〉〉〉

\begin{itemize}
\tightlist
\item
  RT
  kk\_hirono(刑事告発・非常上告_金沢地方検察庁御中)|CrimeInfo(CrimeInfo)
  日時:2021-05-13 20:38/2021/05/13 19:36 URL:
  \url{https://twitter.com/kk\_hirono/status/1392806569587351556} 
  \url{https://twitter.com/CrimeInfo/status/1392790823524179979} 
  \textgreater{}
  CrimeInfo論文&エッセイ集には、西嶋弁護士に「冤罪」をテーマに寄稿していただきました。冤罪の実態や袴田事件の弁護でのご経験のみならず、冤罪根絶のための方策について書かれています。こちらもぜひご一読を。
  CrimeInfo論文&エッセイ集6 「冤罪」(西嶋勝彦)
  \url{https://t.co/eXSps7i3xH}  \url{https://t.co/8IMxjg543t} 
\end{itemize}

〉〉〉 kk\_hironoのリツイート 〉〉〉

\begin{itemize}
\tightlist
\item
  RT
  kk\_hirono(刑事告発・非常上告_金沢地方検察庁御中)|kk\_hirono(刑事告発・非常上告_金沢地方検察庁御中)
  日時:2021-05-13 20:39/2021/05/13 19:32 URL:
  \url{https://twitter.com/kk\_hirono/status/1392806593419354114} 
  \url{https://twitter.com/kk\_hirono/status/1392789939901763585} 
  \textgreater{} \textgreater{} マル激トーク・オン・ディマンド
  第1031回(2021年1月9日)『録音テープが物語る袴田事件の真実』ゲスト:\#浜田寿美男
  氏(心理学者・袴田事件取り調べ録音テープ鑑定人)司会:\#神保哲生、\#宮台真司・・・
\end{itemize}

〉〉〉 kk\_hironoのリツイート 〉〉〉

\begin{itemize}
\tightlist
\item
  RT
  kk\_hirono(刑事告発・非常上告_金沢地方検察庁御中)|kk\_hirono(刑事告発・非常上告_金沢地方検察庁御中)
  日時:2021-05-13 20:39/2021/05/13 19:30 URL:
  \url{https://twitter.com/kk\_hirono/status/1392806615674363908} 
  \url{https://twitter.com/kk\_hirono/status/1392789431170473986} 
  \textgreater{} - 2021年05月13日19時25分の登録:
  REGEXP:''袴田事件''/データベース登録済みツイートの検索:2021-05-13〜2021-05-13/2021年05月13日19時24分の記録:ユーザ・投稿:7/16件
  \url{https://t.co/xc1pviy1Sv} 
\end{itemize}

〉〉〉 kk\_hironoのリツイート 〉〉〉

\begin{itemize}
\tightlist
\item
  RT
  kk\_hirono(刑事告発・非常上告_金沢地方検察庁御中)|hirono\_hideki(奉納\さらば弁護士鉄道・泥棒神社の物語)
  日時:2021-05-13 20:39/2021/05/13 19:27 URL:
  \url{https://twitter.com/kk\_hirono/status/1392806624423682057} 
  \url{https://twitter.com/hirono\_hideki/status/1392788653307428875} 
  \textgreater{} 2021-05-13\_19:25
  奉納\\#危険生物・弁護士脳汚染除去装置\\#金沢地方検察庁御中\_2020:
  REGEXP:''袴田事件''/データベース登録済みツイートの検索:2021-05-13〜2021-05-13/2021年05月13日19時24分の記録:ユーザ・投稿:7/16件
  \url{https://t.co/XEGB98aoFI} 
\end{itemize}

〉〉〉 kk\_hironoのリツイート 〉〉〉

\begin{itemize}
\tightlist
\item
  RT
  kk\_hirono(刑事告発・非常上告_金沢地方検察庁御中)|kk\_hirono(刑事告発・非常上告_金沢地方検察庁御中)
  日時:2021-05-13 20:39/2021/05/13 19:24 URL:
  \url{https://twitter.com/kk\_hirono/status/1392806644677939201} 
  \url{https://twitter.com/kk\_hirono/status/1392787865948491777} 
  \textgreater{}
  先程もまとめ記事を作成し,ユーザ・投稿:4/6件という結果でしたが,TwitterAPIの検索でトータル84件のツイートというのは信じられないような反応の乏しさです。「袴田事件」の検索結果で,朝早くにデイリー新潮の新着記事がありました。
\end{itemize}

〉〉〉 kk\_hironoのリツイート 〉〉〉

\begin{itemize}
\tightlist
\item
  RT
  kk\_hirono(刑事告発・非常上告_金沢地方検察庁御中)|kk\_hirono(刑事告発・非常上告_金沢地方検察庁御中)
  日時:2021-05-13 20:39/2021/05/13 19:22 URL:
  \url{https://twitter.com/kk\_hirono/status/1392806658082885633} 
  \url{https://twitter.com/kk\_hirono/status/1392787307934085122} 
  \textgreater{} ``袴田事件''の該当: hirono\_hideki 4/0件 kk\_hirono
  8/0件 s\_hirono 1/0件
\end{itemize}

〉〉〉 kk\_hironoのリツイート 〉〉〉

\begin{itemize}
\tightlist
\item
  RT
  kk\_hirono(刑事告発・非常上告_金沢地方検察庁御中)|kk\_hirono(刑事告発・非常上告_金沢地方検察庁御中)
  日時:2021-05-13 20:39/2021/05/13 19:22 URL:
  \url{https://twitter.com/kk\_hirono/status/1392806668711301121} 
  \url{https://twitter.com/kk\_hirono/status/1392787283573567493} 
  \textgreater{} 2021年05月13日19時20分の実行記録:
  twitterAPI-search-lawList-mydql-add.rb ``袴田事件''
  ツイート数:14/2412 リツイート数:3/2412 トータル:84
\end{itemize}

〉〉〉 kk\_hironoのリツイート 〉〉〉

\begin{itemize}
\tightlist
\item
  RT
  kk\_hirono(刑事告発・非常上告_金沢地方検察庁御中)|kk\_hirono(刑事告発・非常上告_金沢地方検察庁御中)
  日時:2021-05-13 20:39/2021/05/13 19:19 URL:
  \url{https://twitter.com/kk\_hirono/status/1392806679171932166} 
  \url{https://twitter.com/kk\_hirono/status/1392786687885938695} 
  \textgreater{} - 2021年05月13日16時00分の登録:
  REGEXP:''袴田事件''/データベース登録済みツイートの検索:2021-05-13〜2021-05-13/2021年05月13日16時00分の記録:ユーザ・投稿:4/6件
  \url{https://t.co/5rm8jbGbhP} 
\end{itemize}

〉〉〉 kk\_hironoのリツイート 〉〉〉

\begin{itemize}
\tightlist
\item
  RT
  kk\_hirono(刑事告発・非常上告_金沢地方検察庁御中)|CrimeInfo(CrimeInfo)
  日時:2021-05-13 20:39/2021/05/13 19:19 URL:
  \url{https://twitter.com/kk\_hirono/status/1392806702198657027} 
  \url{https://twitter.com/CrimeInfo/status/1392786582361415684} 
  \textgreater{}
  袴田事件の弁護団長・西嶋勝彦弁護士へのインタビューが中心の読み応えのある記事です。間質性肺炎のため昨年から酸素吸入ボンベに車椅子で弁護活動されているそうですが、袴田事件差戻審では「最高裁の鼻を明かしてやります」と意気込んでおられるそう。是非ご一読を。
  \url{https://t.co/NJ2OiIP4wP} 
\end{itemize}

〉〉〉 kk\_hironoのリツイート 〉〉〉

\begin{itemize}
\tightlist
\item
  RT
  kk\_hirono(刑事告発・非常上告_金沢地方検察庁御中)|kk\_hirono(刑事告発・非常上告_金沢地方検察庁御中)
  日時:2021-05-13 20:39/2021/05/13 19:17 URL:
  \url{https://twitter.com/kk\_hirono/status/1392806720783613954} 
  \url{https://twitter.com/kk\_hirono/status/1392786078701031424} 
  \textgreater{} \#\#\#\#
  被告発人岡田進弁護士の第2回公判調書:正木ひろし弁護士と袴田事件,そこから濱田武律裁判長の逆転無罪判決だった梨木作次郎弁護士の強制わいせつ事件判決文を発見
\end{itemize}

〉〉〉 kk\_hironoのリツイート 〉〉〉

\begin{itemize}
\tightlist
\item
  RT
  kk\_hirono(刑事告発・非常上告_金沢地方検察庁御中)|Itikawa\_Ziroo(いちかわ じろー)
  日時:2021-05-13 20:39/2021/05/13 17:51 URL:
  \url{https://twitter.com/kk\_hirono/status/1392806749158051840} 
  \url{https://twitter.com/Itikawa\_Ziroo/status/1392764450277519360} 
  \textgreater{}
  >新証拠だけの評価ではなく、白鳥、財田川決定を頭に置いての新旧証拠の総合判断です。DNA鑑定などするまでもなく旧証拠をいくら集めても有罪になどならない。
  袴田事件、弁護団長「西嶋勝彦さん」の素顔 真犯人は「警察と利害が一致していた人物」(デイリー新潮)
  \url{https://t.co/hk9NqPvMEC} 
\end{itemize}

〉〉〉 kk\_hironoのリツイート 〉〉〉

\begin{itemize}
\tightlist
\item
  RT
  kk\_hirono(刑事告発・非常上告_金沢地方検察庁御中)|hirono\_hideki(奉納\さらば弁護士鉄道・泥棒神社の物語)
  日時:2021-05-13 20:39/2021/05/13 16:48 URL:
  \url{https://twitter.com/kk\_hirono/status/1392806767537442822} 
  \url{https://twitter.com/hirono\_hideki/status/1392748515156107265} 
  \textgreater{} 2021-05-13\_16:42
  奉納\\#危険生物・弁護士脳汚染除去装置\\#金沢地方検察庁御中\_2020:
  「袴田事件」を@hirono\_hideki @kk\_hirono @s\_hironoで検索 1022件の該当 2021-05-13\_16:42の記録
  \url{https://t.co/enxXteXo5R} 
\end{itemize}

〉〉〉 kk\_hironoのリツイート 〉〉〉

\begin{itemize}
\tightlist
\item
  RT
  kk\_hirono(刑事告発・非常上告_金沢地方検察庁御中)|kk\_hirono(刑事告発・非常上告_金沢地方検察庁御中)
  日時:2021-05-13 20:39/2021/05/13 16:41 URL:
  \url{https://twitter.com/kk\_hirono/status/1392806779004751873} 
  \url{https://twitter.com/kk\_hirono/status/1392746804312113154} 
  \textgreater{}
  データベースに登録する方に少々時間がかかるのですが,テキストのみのまとめ記事は仕上がりが早いです。「袴田事件」のキーワードもまとめ記事を作成しておこうと思いますが,今度は処理時間を計測してみます。
\end{itemize}

〉〉〉 kk\_hironoのリツイート 〉〉〉

\begin{itemize}
\tightlist
\item
  RT
  kk\_hirono(刑事告発・非常上告_金沢地方検察庁御中)|kk\_hirono(刑事告発・非常上告_金沢地方検察庁御中)
  日時:2021-05-13 20:39/2021/05/13 16:32 URL:
  \url{https://twitter.com/kk\_hirono/status/1392806799850360834} 
  \url{https://twitter.com/kk\_hirono/status/1392744529984323584} 
  \textgreater{} -
  袴田事件、弁護団長「西嶋勝彦さん」の素顔 真犯人は「警察と利害が一致していた人物」(デイリー新潮)
  - Yahoo!ニュース \url{https://t.co/8Djy4t089w} 
\end{itemize}

〉〉〉 kk\_hironoのリツイート 〉〉〉

\begin{itemize}
\tightlist
\item
  RT
  kk\_hirono(刑事告発・非常上告_金沢地方検察庁御中)|kk\_hirono(刑事告発・非常上告_金沢地方検察庁御中)
  日時:2021-05-13 20:39/2021/05/13 16:28 URL:
  \url{https://twitter.com/kk\_hirono/status/1392806810826928132} 
  \url{https://twitter.com/kk\_hirono/status/1392743550371078147} 
  \textgreater{}
  午前中は,Yahooニュースで袴田事件の最新記事を読んでいて,次のエントリーとしても取り上げていますが,先程確認したところランキング40には入っていませんでした。
\end{itemize}

〉〉〉 kk\_hironoのリツイート 〉〉〉

\begin{itemize}
\tightlist
\item
  RT
  kk\_hirono(刑事告発・非常上告_金沢地方検察庁御中)|kk\_hirono(刑事告発・非常上告_金沢地方検察庁御中)
  日時:2021-05-13 20:40/2021/05/13 16:02 URL:
  \url{https://twitter.com/kk\_hirono/status/1392806855974395904} 
  \url{https://twitter.com/kk\_hirono/status/1392736946678472707} 
  \textgreater{} \#\#\#\#
  被告発人岡田進弁護士の第2回公判調書:正木ひろし弁護士と袴田事件
\end{itemize}

〉〉〉 kk\_hironoのリツイート 〉〉〉

\begin{itemize}
\tightlist
\item
  RT
  kk\_hirono(刑事告発・非常上告_金沢地方検察庁御中)|hirono\_hideki(奉納\さらば弁護士鉄道・泥棒神社の物語)
  日時:2021-05-13 20:40/2021/05/13 16:01 URL:
  \url{https://twitter.com/kk\_hirono/status/1392806864291733509} 
  \url{https://twitter.com/hirono\_hideki/status/1392736831096066049} 
  \textgreater{} 2021-05-13\_16:00
  奉納\\#危険生物・弁護士脳汚染除去装置\\#金沢地方検察庁御中\_2020:
  REGEXP:''袴田事件''/データベース登録済みツイートの検索:2021-05-13〜2021-05-13/2021年05月13日16時00分の記録:ユーザ・投稿:4/6件
  \url{https://t.co/8pw7kXmkrv} 
\end{itemize}

〉〉〉 kk\_hironoのリツイート 〉〉〉

\begin{itemize}
\tightlist
\item
  RT
  kk\_hirono(刑事告発・非常上告_金沢地方検察庁御中)|s\_hirono(非常上告-最高検察庁御中\_ツイッター)
  日時:2021-05-13 20:40/2021/05/13 16:00 URL:
  \url{https://twitter.com/kk\_hirono/status/1392806879860985862} 
  \url{https://twitter.com/s\_hirono/status/1392736572148109323} 
  \textgreater{}
  2021-05-13-114346\_袴田事件の真犯人は誰なのか。「父(専務)と不仲と言われ一人生き残った長女(故人)に疑いをかける向きもあるがそれは違う。再審請求になってから警.jpg
  \url{https://t.co/ohpbmbIW7G} 
\end{itemize}

〉〉〉 kk\_hironoのリツイート 〉〉〉

\begin{itemize}
\tightlist
\item
  RT
  kk\_hirono(刑事告発・非常上告_金沢地方検察庁御中)|hirono\_hideki(奉納\さらば弁護士鉄道・泥棒神社の物語)
  日時:2021-05-13 20:40/2021/05/13 15:59 URL:
  \url{https://twitter.com/kk\_hirono/status/1392806892649336834} 
  \url{https://twitter.com/hirono\_hideki/status/1392736169599721477} 
  \textgreater{} 2021年05月13日15時58分の実行記録:
  twitterAPI-search-lawList-mydql-add.rb ``袴田事件'' ツイート数:5/2412
  リツイート数:2/2412 トータル:66 ``袴田事件''の該当: hirono\_hideki
  1/0件 kk\_hirono 3/0件 s\_hirono 0/0件
\end{itemize}

〉〉〉 kk\_hironoのリツイート 〉〉〉

\begin{itemize}
\tightlist
\item
  RT
  kk\_hirono(刑事告発・非常上告_金沢地方検察庁御中)|41Fukui(福井英史
  フクイのカレー) 日時:2021-05-13 20:40/2021/05/13 14:58 URL:
  \url{https://twitter.com/kk\_hirono/status/1392806916082987009} 
  \url{https://twitter.com/41Fukui/status/1392720845806604288} 
  \textgreater{}
  袴田事件の真犯人は誰なのか。「父(専務)と不仲と言われ一人生き残った長女(故人)に疑いをかける向きもあるがそれは違う。・・・」
  袴田事件、弁護団長「西嶋勝彦さん」の素顔 真犯人は「警察と利害が一致していた人物」
  \url{https://t.co/eQVT8mUGbl}  \#デイリー新潮
\end{itemize}

〉〉〉 kk\_hironoのリツイート 〉〉〉

\begin{itemize}
\tightlist
\item
  RT
  kk\_hirono(刑事告発・非常上告_金沢地方検察庁御中)|shizuokakonsama(静岡のヨン様ならぬコン様)
  日時:2021-05-13 20:40/2021/05/13 12:53 URL:
  \url{https://twitter.com/kk\_hirono/status/1392806964053217282} 
  \url{https://twitter.com/shizuokakonsama/status/1392689356847017984} 
  \textgreater{}
  袴田事件、弁護団長「西嶋勝彦さん」の素顔 真犯人は「警察と利害が一致していた人物」(デイリー新潮)
  \url{https://t.co/AOr9qbdMqq} 
\end{itemize}

〉〉〉 kk\_hironoのリツイート 〉〉〉

\begin{itemize}
\tightlist
\item
  RT
  kk\_hirono(刑事告発・非常上告_金沢地方検察庁御中)|kk\_hirono(刑事告発・非常上告_金沢地方検察庁御中)
  日時:2021-05-13 20:40/2021/05/13 11:53 URL:
  \url{https://twitter.com/kk\_hirono/status/1392806979572146180} 
  \url{https://twitter.com/kk\_hirono/status/1392674348293447683} 
  \textgreater{}
  この袴田事件については,ビートルズとレコードという接点で個人的に蛸島事件と強い結びつきがあり,いずれ取り上げる予定にしていました。なお,「袴田事件、弁護団長「西嶋勝彦さん」の素顔」という記事には,八海事件や正木ひろし弁護士の名前も出ていました。
\end{itemize}

〉〉〉 kk\_hironoのリツイート 〉〉〉

\begin{itemize}
\tightlist
\item
  RT
  kk\_hirono(刑事告発・非常上告_金沢地方検察庁御中)|kk\_hirono(刑事告発・非常上告_金沢地方検察庁御中)
  日時:2021-05-13 20:40/2021/05/13 11:50 URL:
  \url{https://twitter.com/kk\_hirono/status/1392806993425932289} 
  \url{https://twitter.com/kk\_hirono/status/1392673476134072321} 
  \textgreater{}
  40位までランキングがあったのですが,袴田事件の記事は見当たらず,袴田と再審でページ内検索をしてみましたが,やはり該当はありませんでした。
\end{itemize}

〉〉〉 kk\_hironoのリツイート 〉〉〉

\begin{itemize}
\tightlist
\item
  RT
  kk\_hirono(刑事告発・非常上告_金沢地方検察庁御中)|kk\_hirono(刑事告発・非常上告_金沢地方検察庁御中)
  日時:2021-05-13 20:40/2021/05/13 11:25 URL:
  \url{https://twitter.com/kk\_hirono/status/1392807008445689859} 
  \url{https://twitter.com/kk\_hirono/status/1392667231868252164} 
  \textgreater{}
  袴田事件、弁護団長「西嶋勝彦さん」の素顔 真犯人は「警察と利害が一致していた人物」(デイリー新潮)
  - Yahoo!ニュース \url{https://t.co/vqZTYcnW7U}  5/13(木) 6:01配信
\end{itemize}

〉〉〉 kk\_hironoのリツイート 〉〉〉

\begin{itemize}
\tightlist
\item
  RT
  kk\_hirono(刑事告発・非常上告_金沢地方検察庁御中)|todateyoshiyuki(弁護士戸舘圭之オフィシャル/とってぃ/袴田事件弁護団)
  日時:2021-05-13 20:40/2021/05/13 10:35 URL:
  \url{https://twitter.com/kk\_hirono/status/1392807039861018624} 
  \url{https://twitter.com/todateyoshiyuki/status/1392654709840744449} 
  \textgreater{}
  声のデカさだけは定評のある当職(法廷で裁判官から声がでかいと注意された経験あり・・・)
  \url{https://t.co/smuMIRlod5} 
\end{itemize}

〉〉〉 kk\_hironoのリツイート 〉〉〉

\begin{itemize}
\tightlist
\item
  RT
  kk\_hirono(刑事告発・非常上告_金沢地方検察庁御中)|dailyshincho(デイリー新潮)
  日時:2021-05-13 20:40/2021/05/13 09:08 URL:
  \url{https://twitter.com/kk\_hirono/status/1392807070613704712} 
  \url{https://twitter.com/dailyshincho/status/1392632760276299779} 
  \textgreater{}
  袴田事件、弁護団長「西嶋勝彦さん」の素顔 真犯人は「警察と利害が一致していた人物」
  「この事件は証拠の捏造ということで静岡地裁が画期的な判断を示している。事件から一年以上経って味噌漬けの5点の衣類が見つかった。捜査官の捏造が疑われる」(弁護団長・西嶋氏)
  \url{https://t.co/pGqL2U77RO} 
\end{itemize}

〉〉〉 kk\_hironoのリツイート 〉〉〉

\begin{itemize}
\tightlist
\item
  RT
  kk\_hirono(刑事告発・非常上告_金沢地方検察庁御中)|forum\_90(死刑廃止フォーラム90)
  日時:2021-05-13 20:41/2021/05/13 08:23 URL:
  \url{https://twitter.com/kk\_hirono/status/1392807096765214726} 
  \url{https://twitter.com/forum\_90/status/1392621380772892674} 
  \textgreater{}
  袴田事件、弁護団長「西嶋勝彦さん」の素顔 真犯人は「警察と利害が一致していた人物」(デイリー新潮)
  \#Yahooニュース \url{https://t.co/YvbZDtd4Uc} 
\end{itemize}

〉〉〉 kk\_hironoのリツイート 〉〉〉

\begin{itemize}
\tightlist
\item
  RT
  kk\_hirono(刑事告発・非常上告_金沢地方検察庁御中)|yutaro\_tamura(田村雄太郎)
  日時:2021-05-13 20:41/2021/05/13 07:03 URL:
  \url{https://twitter.com/kk\_hirono/status/1392807114347716615} 
  \url{https://twitter.com/yutaro\_tamura/status/1392601345010503682} 
  \textgreater{}
  袴田事件、弁護団長「西嶋勝彦さん」の素顔 真犯人は「警察と利害が一致していた人物」
  \url{https://t.co/Aqish59ELE}  \#デイリー新潮 \#袴田事件
\end{itemize}

〉〉〉 kk\_hironoのリツイート 〉〉〉

\begin{itemize}
\tightlist
\item
  RT
  kk\_hirono(刑事告発・非常上告_金沢地方検察庁御中)|Junko\_Hoshino(星野 順子)
  日時:2021-05-13 20:41/2021/05/13 06:54 URL:
  \url{https://twitter.com/kk\_hirono/status/1392807172271067137} 
  \url{https://twitter.com/Junko\_Hoshino/status/1392599057260896256} 
  \textgreater{}
  横須賀で講演なさったお姉さんの秀子さんのお話も聴いてみて!⇒ストップ冤罪!袴田事件について語る【姉・袴田ひで子さん】Vol.1
  \url{https://t.co/kI8UCaPg46}  \#yokosuka \#横須賀
\end{itemize}

〉〉〉 kk\_hironoのリツイート 〉〉〉

\begin{itemize}
\item
  RT
  kk\_hirono(刑事告発・非常上告_金沢地方検察庁御中)|Junko\_Hoshino(星野 順子)
  日時:2021-05-13 20:41/2021/05/13 00:25 URL:
  \url{https://twitter.com/kk\_hirono/status/1392807184342286337} 
  \url{https://twitter.com/Junko\_Hoshino/status/1392501098070478849} 
  \textgreater{}
  【拡散希望】⇒【冤罪】袴田巖さんの第一声★袴田事件・再審開始決定報告集会
  \url{https://t.co/v0ynMEyu5h} 
\item
  1365:2021-05-13\_20:46:54 \#告発状 \#\#\#\#
  被告発人岡田進弁護士の第2回公判調書:正木ひろし弁護士と袴田事件,そこから濱田武律裁判長の逆転無罪判決だった梨木作次郎弁護士の強制わいせつ事件判決文を発見
  \url{https://hirono-hideki.hatenadiary.jp/entry/2021/05/13/204650} 
\end{itemize}

 内容は上記のエントリーの続きですが,埋め込みツイートの関係で切り分けました。38件のリツイートと思っていたのですが,リツイートではないツイートも1件取得されていて,リツイートの数は37件だったようです。

 「twitterAPI-search-lawList-mydql-add.rb
袴田事件」の実行をしているのですが,今度は時間がかかっていると記述しようと思ったタイミングで処理が終わりました。私のアカウントのツイートが多く時間が掛かったようですが,ちょうど処理が終わりトータルで148件でした。

2021年05月13日21時01分の実行記録: twitterAPI-search-lawList-mydql-add.rb
``袴田事件'' ツイート数:30/2412 リツイート数:44/2412
トータル:148``袴田事件''の該当: hirono\_hideki 6/0件 kk\_hirono
22/39件 s\_hirono 1/0件

 再捜査要請書_警察庁・石川県警察御中(@kk\_hirono)のアカウントだけで「袴田事件」の検索に該当したツイートは22件で同じくリツイートが39件となっています。トータルで148件の中の数です。

 時刻は21時08分です。そういえば今朝辺りからTwitterのタイムラインの右側に,トレンドの下に「おすすめトピック」というのが表示されていることに気がついたのですが,その分なのか,2つほどトレンドの数が減ったようにも思えています。

\begin{itemize}
\tightlist
\item
  袴田事件、弁護団長「西嶋勝彦さん」の素顔 真犯人は「警察と利害が一致していた人物」
  \textbar{} デイリー新潮 \url{https://t.co/gu3JPgejem} 
\end{itemize}

 これまでYahooニュースの記事を見ていて,デイリー新潮の記事を開いたのは,あるいは初めてになるかもしれません。日本外国特派員協会の記者会見やジャーナリストの神保哲生氏が記事に出ていたことを思い出したのですが,そちらに関するツイートはまだ見ていないように思います。

 文字列の範囲選択が出来ないですが,下に「著名な冤罪事件を多く手がけた西嶋勝彦弁護士(撮影・粟野仁雄)」という写真があって,顔にむくみがあり,微笑んで見えますが,痴呆の症状が出ているようにも思えてきました。記事に年齢は見当たりません。袴田さん姉弟は年齢があります。

 小川秀世事務局長というのは大崎事件の鴨志田裕美弁護士が事務局長というのとは違い,余り名前を見かけなかったと思いますし,顔写真を見たという記憶もありません。西嶋勝彦団長と似た年齢とは考えにくいので,痴呆症の妄言という可能性は低く思われます。年齢を調べて確認します。

\begin{itemize}
\tightlist
\item
  西嶋勝彦 - Wikipedia \url{https://t.co/roWsRtAVU2}  西嶋 勝彦(にしじま
  かつひこ、1941年 - )は、日本の弁護士。
\end{itemize}

 Wikipediaで生年のみの表示というのも珍しく感じましたが,1941年生まれとあります。昭和16年生まれなので,思ったほど高齢ではなさそうですが,70代で認知症や寝たきりになる人もいるとは聞きます。

\begin{itemize}
\tightlist
\item
  「私は悪魔ではない」 小川秀世弁護士「『悪魔の判決教本』による事実認定」に反論する
  - 現代人文社 \url{https://t.co/P5Lv0CNWtg} 
\end{itemize}

 検索で初めて見る変わった本のタイトルが出てきましたが,著者が木谷明となっているので,元裁判官の木谷明弁護士の可能性が確定的に高いと思います。

\begin{itemize}
\tightlist
\item
  袴田巖さん 「夢の中へ」続編・小川秀世法律事務所を訪問:袴田家物語
  \url{https://t.co/QPOkPk9nBa} 
\end{itemize}

 袴田家物語というサイトは初めてみたように思いますが,記事は昨年の2020年4月8日付けで,複数の写真がありますが,ベテランの弁護士としては若そうで,年齢に関する情報がみつかりませんが,まだ50代にも見えます。弁護士としては経験もあり,一番脂の乗った時期にも思えるところです。

 デイリー新潮の取材で,西嶋勝彦弁護士が単独で暴走したという可能性も否定はできませんが,それを疑いたくなるような同業弁護士の「袴田事件、弁護団長「西嶋勝彦さん」の素顔 真犯人は「警察と利害が一致していた人物」
\textbar{}
デイリー新潮」という記事への反応の乏しさです。皆無に近いかも。

 ふと思い出したのですが,思い出したのも本日2回目ぐらいで,昨日辺りに,志布志事件の映画があったというような情報をネットで見かけていました。最近の映画とは思えないですが,なぜ今頃見つけたのか不思議に思っていました。気に留めず忘れていたという可能性も否定は出来ません。

\begin{itemize}
\item
  志布志事件 映画 - Google 検索 \url{https://t.co/alQp1nVBDg} 
\item
  つくられる自白 -志布志の悲劇- \textbar{} 映画の動画・DVD -
  TSUTAYA/ツタヤ \url{https://t.co/2PW0wuM0EA}  ¥\n ¥\n 製作年:2008年 ¥\n
  メーカー:エースデュース ¥\n 出演者:鳥越俊太郎 、 原田宏二 、
  方岸みつこ 、 秋山賢三 ¥\n 監督:池田博穂 ¥\n 脚本:池田博穂 、
  毛利甚八
\end{itemize}

 昨日か見かけたときとは,映画のタイトルの印象がまるで違うのですが,「志布志の悲劇」という大仰なタイトルは,今初めて目にしたように思いました。2008年制作とあるから平成20年になります。これはtwilog-serch
でも結果がなさそうです。

 ちょっと驚いて自分の頭を疑ったのですが,3件の該当がありました。

\begin{lstlisting}
py37_env ❯ twilog-serch 志布志の悲劇
\end{lstlisting}

\begin{itemize}
\tightlist
\item
  ./hirono\_hideki2021-05-13\_210059.csv:2021-01-31 19:41:37 ``RT
  @yuuheipapa:
  えー、業務連絡の続報。「志布志の悲劇。作られる自白」は、鹿児島県議会選挙で13人もの逮捕者を出しながら全員無罪になった志布志事件を題材にしたドキュメンタリー映画。脚本は「家栽の人」の原作者毛利甚派八、出演は鳥越俊太郎氏ほか。予告編は\url{http://bit.ly/7LwE1f''} 
  \url{https://twitter.com/hirono\_hideki/status/1355828576659021830} 
\item
  ./hirono\_hideki2021-05-13\_210059.csv:2019-07-02 12:20:55
  ``2019年07月02日09時19分の登録:
  %@yuuheipapa ユウヘイ・パパ%えー、業務連絡の続報。「志布志の悲劇。作られる自白」は、鹿児島県議会選挙で13人もの逮捕者を出しながら全員無罪になった志布志事件を題材にしたドキュメンタリー映画。
  \url{http://hirono2014sk.blogspot.com/2019/07/yuuheipapa.html''} 
  \url{https://twitter.com/hirono\_hideki/status/1145895095759065088} 
\item
  ./kk\_hirono2021-05-13\_213303.csv:2020-03-05 18:24:20 ``{[}link:{]}
  2019年07月02日09時19分の登録:
  %@yuuheipapa ユウヘイ・パパ%えー、業務連絡の続報。「志布志の悲劇。作られる自白」は、鹿児島県議会選挙で13人もの逮捕者を出しながら全員無罪になった志布志事件を題材にしたドキュメンタリー映画。
  \url{http://hirono2014sk.blogspot.com/2019/07/yuuheipapa.html''} 
  \url{https://twitter.com/kk\_hirono/status/1235496355331624960} 
\end{itemize}

 3件とも同じアカウントのツイートのリツイートのようですが,アイコンが漫画のヤドカリのようになっていたアカウントだと思います。そういえば最近は見かけていないかもしれません。1つはリツイートではなかったですが,よく読まずに志布志事件だけで反応した可能性もあります。

〉〉〉 kk\_hironoのリツイート 〉〉〉

\begin{itemize}
\tightlist
\item
  RT
  kk\_hirono(刑事告発・非常上告_金沢地方検察庁御中)|bengoshi\_info(弁護士界談)
  日時:2021-05-13 21:59/2015/05/30 19:37 URL:
  \url{https://twitter.com/kk\_hirono/status/1392826712925249541} 
  \url{https://twitter.com/bengoshi\_info/status/604597384639684608} 
  \textgreater{}
  日弁連が2008年に志布志事件の悲劇をテーマに制作した短編ドキュメンタリー「Presumed
  Guilty」 英語版トレイラー(11分): \url{http://t.co/LPuB3SNsjQ}  ・・・
  LiveLeak
\end{itemize}

 再捜査要請書_警察庁・石川県警察御中(@kk\_hirono)のアカウントでログインしたままTwitter検索を始めたのですが,日弁連制作というツイートが出てきてさらに驚きました。今年に入ってからこういう驚きの発見がとても多くなっていると感じます。

\begin{itemize}
\tightlist
\item
  〈〈〈 2021/05/13 22:03:14 Linux Emacs: 〈〈〈
\end{itemize}

\hypertarget{ux88abux544aux767aux4ebaux5ca1ux7530ux9032ux5f01ux8b77ux58ebux306eux7b2c2ux56deux516cux5224ux8abfux66f8ux5fd7ux5e03ux5fd7ux306eux60b2ux5287ux4f5cux3089ux308cux308bux81eaux767dux3068ux3044ux30462008ux5e74ux306eux65e5ux5f01ux9023ux5236ux4f5cux6620ux753bux306eux767aux898b}{%
\paragraph{被告発人岡田進弁護士の第2回公判調書:「志布志の悲劇。作られる自白」という2008年の日弁連制作映画の発見}\label{ux88abux544aux767aux4ebaux5ca1ux7530ux9032ux5f01ux8b77ux58ebux306eux7b2c2ux56deux516cux5224ux8abfux66f8ux5fd7ux5e03ux5fd7ux306eux60b2ux5287ux4f5cux3089ux308cux308bux81eaux767dux3068ux3044ux30462008ux5e74ux306eux65e5ux5f01ux9023ux5236ux4f5cux6620ux753bux306eux767aux898b}}

\begin{itemize}
\tightlist
\item
  〉〉〉 Linux Emacs: 2021/05/13 22:07:33 〉〉〉
\end{itemize}

:CATEGORIES: @kanazawabengosi \#金沢弁護士会 @JFBAsns
日本弁護士連合会(日弁連) \#法務省 @MOJ\_HOUMU \#志布志事件

\begin{itemize}
\tightlist
\item
  1366:2021-05-13\_22:03:51 \#告発状 \#\#\#\#
  被告発人岡田進弁護士の第2回公判調書:デイリー新潮の「袴田事件、弁護団長「西嶋勝彦さん」の素顔 真犯人は「警察と利害が一致していた人物」」という記事への反応
  \url{https://hirono-hideki.hatenadiary.jp/entry/2021/05/13/220347} 
\end{itemize}

 内容は上記エントリーの続きになります。

\begin{itemize}
\tightlist
\item
  2021年05月13日22時02分の登録:
  REGEXP:''志布志の悲劇''/データベース登録済みツイートの検索:2009-11-21〜2021-05-13/2021年05月13日22時02分の記録:ユーザ・投稿:3/8件
  \url{https://kk2020-09.blogspot.com/2021/05/regexp2009-11-212021-05-1320210513220238.html} 
\end{itemize}

\begin{quote}
《引用の始まり》
\end{quote}

\begin{quote}
アカウント名 ツイート数 リツイート数ユウヘイ・パパ(yuuheipapa) 1
0奉納\さらば弁護士鉄道・泥棒神社の物語(hirono\_hideki) 1
0刑事告発・非常上告_金沢地方検察庁御中(kk\_hirono) 6 0
\end{quote}

\begin{quote}
《引用の終わり》
\end{quote}

\begin{itemize}
\tightlist
\item
  奉納\危険生物・弁護士脳汚染除去装置\金沢地方検察庁御中\_2020:
  REGEXP:''志布志の悲劇''/データベース登録済みツイートの検索:2009-11-21〜2021-05-13/2021年05月13日22時02分の記録:ユーザ・投稿:3/8件 \url{https://kk2020-09.blogspot.com/2021/05/regexp2009-11-212021-05-1320210513220238.htmln} 
\end{itemize}

 ないと思っていたのですが,hatena-log-search
の検索で1つ結果が出ました。それも2008年6月19日の記事となっています。

\begin{lstlisting}
py37_env ❯ hatena-log-search 志布志の悲劇
\end{lstlisting}

\begin{itemize}
\tightlist
\item
  ./20080619: 鹿児島県議選で選挙違反に問われた元被告12人全員の無罪が確定した冤罪事件をめぐり、日弁連は18日、事件を題材に制作してきた初のドキュメンタリー映画「つくられる自白 志布志の悲劇」が完成したと発表した。23日、東京・霞が関の弁護士会館で試写会を開く。
\end{itemize}

 1件だけですが,twilog-serch-post
でブログに投稿しないとリンクの記事が開けないのでした。

\begin{itemize}
\tightlist
\item
  2021年05月13日22時17分の登録:
  「志布志の悲劇」を過去のはてなダイアリーの記事から検索
  \url{https://kk2020-09.blogspot.com/2021/05/blog-post\_13.html} 
\end{itemize}

 hatena-log-search
は,はてなダイアリーからはてなブログへの転載がほとんどだと思いますが,はてなダイアリーの記事は,プログラムを利用した関係もあり,日付単位の記事として記録されているかと思います。ここに実感の湧かない意外な発見がありました。次に引用します。

\begin{quote}
《引用の始まり》
\end{quote}

\begin{quote}
*[司法]司法修習生の体験ブログ、検事が言う社会のゴミ掃除で視野狭窄

取り調べ体験ブログに 司法修習生守秘義務違反か 長崎市 ばあちゃんに説教しまくり 社会のゴミ掃除で視野狭くhttp://www.nishinippon.co.jp/nnp/item/29540

2008年6月19日 14:29 カテゴリー:社会 九州・山口 \textgreater{} 長崎

 長崎市内の法律事務所で実務修習中の20代後半の男性司法修習生がインターネット上に自身が開設したブログに、検察の実務修習中に体験した容疑者の取り調べ状況などを書き込んでいたことが19日、分かった。長崎地裁は「内容に修習生としてふさわしくないとみられる言葉があった」として、裁判所法に基づく守秘義務違反に当たる可能性もあるとみて事実関係を調査している。

 同地裁などによると、ブログは「司法修習生のなんとなく日記」と題し、2月15日付に「今日、はじめて取り調べやりました。相手は0歳のばあちゃん。最初はいろいろ話を聞いてたけど、途中から説教しまくり。おばあちゃん泣きまくり」(ブログ上は実年齢を記載)などと記述。同29日付には刑務所見学について「自分が取り調べ中の被疑者は、刑務所出所後5日目に、また犯罪行為に出た」などと書き込んだ。

 3月20日付では司法解剖に立ち会った様子を「死体を切り刻んで内臓とかを全部出して全部調べると臭(にお)いはきつい」、同16日付には「検事が言っていたけれど、社会のゴミ掃除ばっかりやってると視野が狭くなると」などと記載していた。
\end{quote}

\begin{quote}
《引用の終わり》
\end{quote}

\begin{itemize}
\tightlist
\item
  hatena-diary\_20080619 -
  告発\金沢地方検察庁\最高検察庁\法務省\石川県警察御中 \url{https://hirono-hideki.hatenablog.com/entry/2008/06/19/000000n} 
\end{itemize}

 長崎といえば,「長崎ぶらぶら節」や郷原信郎弁護士が次席検事をしていたことでも注目をしてきた土地になりますが,上記の引用部分の記載を思い出すことがなく,潜在意識では残るものがあったような不思議な感覚を覚えています。すごい司法修習生が2008年にもいたものです。

 「長崎地裁は今月12日、最高裁から連絡を受けて調査を開始した。同地裁総務課は「事実関係を調査し(処分については)厳正に対応したい」としている。=2008/06/19付
西日本新聞夕刊=」ともあるのですが,調査を開始というのは私には光明と映る明るいニュースです。

 「私自身、司法修習生に取調べを受けた経験があります。教官は下平豪検察官でした。これまでにも書いたことがあると思いますが、また触れることもあるだろうと思います。いろいろと考えさせられるところのある報道です。」とありますが,過去の自分の発言として実感が乏しく,不思議にも思えます。

 司法修習生に取調べを受けたこと,ただ一人の指導教官と思われた下平豪検事のことは,脳裏に焼き付いた記憶で,余り日をおかず度々思い出していましたが,最近は余り記述していなかったかもしれません。ベトナム人の取調べのことで2月ほど前に記述したかもしれません。

 これを機会に下平豪というtwilog-serch-post
のまとめ記事も作っておきます。

\begin{itemize}
\tightlist
\item
  2021年05月13日22時37分の登録:
  「下平豪」を@hirono\_hideki @kk\_hirono @s\_hironoで検索 16件の該当 2021-05-13\_22:37の記録
  \url{https://kk2020-09.blogspot.com/2021/05/hironohidekikkhironoshirono162021-05.html} 
\item
  2021年05月13日22時38分の登録:
  「下平豪」を過去のはてなダイアリーの記事から検索
  \url{https://kk2020-09.blogspot.com/2021/05/blog-post\_55.html} 
\end{itemize}

 検索結果が少ないので,そちらの方で驚き,目を疑いました。

〉〉〉 kk\_hironoのリツイート 〉〉〉

\begin{itemize}
\tightlist
\item
  RT
  kk\_hirono(刑事告発・非常上告_金沢地方検察庁御中)|hirono\_hideki(奉納\さらば弁護士鉄道・泥棒神社の物語)
  日時:2021-05-13 22:42/2014/04/19 13:16 URL:
  \url{https://twitter.com/kk\_hirono/status/1392837597139922954} 
  \url{https://twitter.com/hirono\_hideki/status/457372171166175232} 
  \textgreater{} H11-09-02\_起訴状\_金沢地方検察庁検察官検事下平豪.jpg
  \url{http://t.co/8j2hYAu3h2} 
\end{itemize}

 ずいぶんきれいな写真だと思ったのですが,下平豪検事の起訴状の写真が出てきて,これも驚きました。次から次に新たな意外な発見が続きます。探せば家のどこかにありそうですが,横に金沢地方裁判所の封筒があって,宛先が金沢中警察署となっているのも初めの発見に思えています。

\begin{itemize}
\tightlist
\item
  平成11年9月2日付起訴状\_金沢地方検察庁検察官検事下平豪 \textbar{}
  再審請求\_金沢地方裁判所御中 \url{https://t.co/krQpEPbWK6} 
\end{itemize}

 久しぶりに見かけたBloggerのブログの一つにも思いましたが,下平豪検事の起訴状の公訴事実の写真もありました。すぐに忘れてしまうので昨日辺りも気になっていたのですが,安藤健次郎さんの怪我はやはり全治10日間だったと確認しました。

 16件という数は少なかったですが,けっこう濃い内容のツイートでした。それほど記憶の喚起に繋がらなかったのが残念ですが,憶えているうちにもっと具体的な記述をしておけばよかったとも悔やまれます。

20051218:{[}link{]}
 印象的だったことは、担当検事の下平豪検察官が、法廷で私の横に歩み寄り、父親の怪我の程度を写した写真を見せたことでした。思ったよりひどい怪我で、とても10日で直るとは思えませんでしたが、誇らしげで、強い意志を感じる姿でした。

 Twitterの記録としては見当たらず多少気になっていたのですが,「担当検事の下平豪検察官が、法廷で私の横に歩み寄り、父親の怪我の程度を写した写真を見せたことでした。」という事実は,ずっと憶えていたことで,特に最近は思い出す頻度が多くなっていました。

20061212:{[}link{]}
 司法修習生の教官をされていたとかで、まっさきに頭に浮かんだのは、下平豪検事のことでした。まさかのまさかですが、ご本人だとすれば仰天です。およそ2千分の一の確率になりそうですが、44歳とのことなので、それに近い年齢だったとは思います。

 下平豪検事の近況で仰天するほど驚いたことがあったのかと思ったのですが,リンクの記事をみると,大阪地検の検事の自殺未遂,割腹自殺のニュースでした。これもずっと憶えていましたが,はっきり思い出せないものの庁舎内での割腹自殺だったと思います。

\begin{itemize}
\tightlist
\item
  <自殺未遂>検事が割腹 大阪地検の取調室で 仕事に悩み? - misola's
  diary \url{https://t.co/ZbAQeQ4Ia6} 
\end{itemize}

 ニュース記事の引用があり,やはり大阪地検の庁舎内での割腹自殺だと確認しましたが,記憶になかったことは,事務官が発見したということと,腹部を3箇所刺して倒れていたとあることです。弁護士法違反の取調べたということは記事を読んで思い出しました。

\begin{itemize}
\tightlist
\item
  裁判官と検察官の自殺について \textbar{} 弁護士ブログ
  \url{https://t.co/svO4egVl3P} 
\end{itemize}

 大阪地検の検事の割腹自殺未遂と同じ頃に,裁判官の自殺というのも話題として見かけることが多かったと記憶に残るのですが,そちらも大阪だったと記憶にあり,上記の記事に大阪高等裁判所の部総括判事で,来年には定年を控えていたとあります。

 記事のタイトルには弁護士ブログとだけありますが,表示したページには「ヒューマンネットワーク三森法律事務所 弁護士 三森敏明」とあります。今風のデザインのページですが,記事の日付が2006年12月13日更新となっていて,多少違和感のようなものを感じています。

 アーカイブが2017年11月で途切れているのも気になるところです。みやすいよく出来たデザインのページだと思います。独自ドメインのようですが,契約は続いているようで,記事の更新がないのが気になります。

\begin{itemize}
\item
  \begin{enumerate}
  \def\labelenumi{(\arabic{enumi})}
  \setcounter{enumi}{11}
  \tightlist
  \item
    弁護士 三森敏明さん (@lawyerTM053) / Twitter
    \url{https://t.co/OB05uzqbo2} 
  \end{enumerate}
\end{itemize}

 名前で検索すると,見覚えのある印象的な幼い子供の笑顔のアイコンの実名Twitterアカウントが出てきました。リツイートが多いようですが,5月1日のツイートのリツイートが最新の更新となっています。

 検索結果も同じだったのですが,Twitterのプロフィールにあるホームページのリンクを開くと,「この接続ではプライバシーが保護されません」という警告が出て先に進めませんでした。たまに見かける警告ですが,この先に進んだことはありません。

〉〉〉 kk\_hironoのリツイート 〉〉〉

\begin{itemize}
\tightlist
\item
  RT
  kk\_hirono(刑事告発・非常上告_金沢地方検察庁御中)|masahirosogabe(曽我部真裕)
  日時:2021-05-13 23:23/2021/05/01 12:41 URL:
  \url{https://twitter.com/kk\_hirono/status/1392847903907934218} 
  \url{https://twitter.com/masahirosogabe/status/1388337756548386818} 
  \textgreater{}
  子どもの権利条約根拠に画期的判決 日照権訴訟で名古屋地裁:中日新聞Web
  \url{https://t.co/0Bok6sB7lx} 
\end{itemize}

〉〉〉 kk\_hironoのリツイート 〉〉〉

\begin{itemize}
\tightlist
\item
  RT
  kk\_hirono(刑事告発・非常上告_金沢地方検察庁御中)|lawyerTM053(弁護士 三森敏明)
  日時:2021-05-13 23:25/2021/01/23 11:52 URL:
  \url{https://twitter.com/kk\_hirono/status/1392848408562466818} 
  \url{https://twitter.com/lawyerTM053/status/1352811349210882049} 
  \textgreater{}
  東京五輪中止の決断ができない理由が賠償問題であるというのは、正しい指摘なのだと私も思う。もうすぐ2月。7月まで半年を切った現時点での東京の感染状況をみて「東京五輪開催は可能」と本気で思っている政治家は、本当にいるのだろうか?あと、海外の選手、来ないと思うし。
  \url{https://t.co/kyIu8yo3Ct} 
\end{itemize}

〉〉〉 kk\_hironoのリツイート 〉〉〉

\begin{itemize}
\tightlist
\item
  RT
  kk\_hirono(刑事告発・非常上告_金沢地方検察庁御中)|lawyerTM053(弁護士 三森敏明)
  日時:2021-05-13 23:25/2021/01/12 18:33 URL:
  \url{https://twitter.com/kk\_hirono/status/1392848454267801608} 
  \url{https://twitter.com/lawyerTM053/status/1348926168834035713} 
  \textgreater{}
  ある卓球ストーカーの全日本卓球 弁護士への道も公務員の職も捨てて・・・(伊藤条太)
  - Y!ニュース \url{https://t.co/wedOkZwNT4} 
\end{itemize}

〉〉〉 kk\_hironoのリツイート 〉〉〉

\begin{itemize}
\item
  RT
  kk\_hirono(刑事告発・非常上告_金沢地方検察庁御中)|lawyerTM053(弁護士 三森敏明)
  日時:2021-05-13 23:25/2021/01/05 11:18 URL:
  \url{https://twitter.com/kk\_hirono/status/1392848494856081424} 
  \url{https://twitter.com/lawyerTM053/status/1346279842384252928} 
  \textgreater{}
  「スーツは経費になる」と、関弁連理事のときに当時の日弁連の偉い人が言っていましたけど。
  \url{https://t.co/NmhcZHJK9d} 
\item
  ある卓球ストーカーの全日本卓球 弁護士への道も公務員の職も捨てて・・・(伊藤条太)
  - 個人 - Yahoo!ニュース \url{https://t.co/Lf9jI7vaPs}  ¥\n 伊藤条太
  \textbar{} 卓球コラムニスト ¥\n 1/11(月) 15:26
\end{itemize}

 ストーカーが男女関係以外で使われているのも珍しく初めて見るように思いましたが,これまで一月も経たないうちにリンク切れになると思い込んでいたYahooニュースの記事が表示されたのも珍しいと思ったのですが,コラムニストの個人の記事であることに気が付きました。

 そういえば,最近は以前よく見かけたジャーナリストの江川紹子氏のYahooニュース個人の記事を見かけていないかもしれません。ちょっと気になるので調べてみます。

\begin{itemize}
\item
  「江川紹子」の検索結果 - Yahoo!ニュース \url{https://t.co/wKA6eBKdwg} 
\item
  江川紹子氏が証拠改ざん隠蔽事件有罪の元特捜部長の弁護士登録に「驚き!」(東スポWeb)
  - Yahoo!ニュース \url{https://t.co/P31fbsHl3b} 
\end{itemize}

 5月2日の記事が出てきましたが,下に江川紹子氏とある丁度履歴書サイズのような顔写真が,これまで見たことのないもので,無気力な脱力感のようなものを感じるのですが,これはどうみてもずいぶん若い時の写真ではと思います。雰囲気が全く違って別人のようです。

 ジャーナリストの江川紹子氏本人のYahooニュース個人の記事にしてはおかしいと思ったのですが,東スポWebが元記事らしいと気が付きました。

 「この報道を受け江川氏は「驚き!日弁連はこの判断の理由を説明する必要があると思う」と驚きとともに日弁連に対し詳細な説明の必要性を訴えた。」とありますが短い内容の記事で,率直に腑抜けのように見える顔写真の掲載と,ジャーナリストの江川紹子氏本来の攻撃性のギャップが大きいです。

 時刻は0時4分です。5月14日日付が変わっていて驚いたのですが,まだ夕食も食べていません。ビジネスジャーナルのジャーナリストの江川紹子氏の記事を読み終えたところですが,ジャーナリストの江川紹子氏の発する大きなパワーと,影響力の低下を感じました。

\begin{itemize}
\tightlist
\item
  〈〈〈 2021/05/14 00:08:01 Linux Emacs: 〈〈〈
\end{itemize}

\hypertarget{ux88abux5bb3ux8005ux5973ux6027ux306eux4f9bux8ff0ux3092ux5426ux5b9aux3057ux5f37ux5236ux308fux3044ux305bux3064ux4e8bux4ef6ux3067ux9006ux8ee2ux7121ux7f6aux5224ux6c7aux3068ux3057ux305fux540dux53e4ux5c4bux9ad8ux88c1ux91d1ux6ca2ux652fux90e8ux6ff1ux7530ux6b66ux5f8bux88c1ux5224ux9577ux306eux5224ux6c7aux6587ux306bux5bfeux3059ux308bux5927ux304dux306aux6b74ux53f2ux7684ux7591ux554f}{%
\paragraph{被害者女性の供述を否定し強制わいせつ事件で逆転無罪判決とした名古屋高裁金沢支部,濱田武律裁判長の判決文に対する大きな歴史的疑問}\label{ux88abux5bb3ux8005ux5973ux6027ux306eux4f9bux8ff0ux3092ux5426ux5b9aux3057ux5f37ux5236ux308fux3044ux305bux3064ux4e8bux4ef6ux3067ux9006ux8ee2ux7121ux7f6aux5224ux6c7aux3068ux3057ux305fux540dux53e4ux5c4bux9ad8ux88c1ux91d1ux6ca2ux652fux90e8ux6ff1ux7530ux6b66ux5f8bux88c1ux5224ux9577ux306eux5224ux6c7aux6587ux306bux5bfeux3059ux308bux5927ux304dux306aux6b74ux53f2ux7684ux7591ux554f}}

\begin{itemize}
\tightlist
\item
  〉〉〉 Linux Emacs: 2021/05/14 08:40:27 〉〉〉
\end{itemize}

:CATEGORIES: @kanazawabengosi \#金沢弁護士会 @JFBAsns
日本弁護士連合会(日弁連) \#法務省 @MOJ\_HOUMU \#梨木作次郎弁護士

\begin{quote}
《引用の始まり》
\end{quote}

\begin{quote}
(1) H証言自体に内在する問題点について

そこでまずは、H証言自体に内在する問題点を取り上げてみるのに、以下に示すような各点を挙げることができる。

(ア) Hが最初に抱きつかれたとする地点について

H証言では、被告人から最初に襲われた地点の特定について様々な供述をし、結局は、H実況見分(昭和六三年一二月一四日実施)の際に指示した「ナカヤビル」前の点(前記歩道橋の下り階段下から約四〇メートル)を正確な箇所といっているものと理解されるのであるが、その証言中では、他に、「(階段を)下りてしばらく歩いたら」「一〇〇メートルも歩いたか、歩かないか」「一〇〇メートルというのは歩き過ぎですけど、歩数で言えば一五歩くらいと思う」(第五回公判)とか、「(階段を)下りてすぐだった、まだそんなに歩いていない記憶」「(実況見分で立ち会って指示した地点は)この辺りというのが周りから何となく分かる」「階段からガード下入口辺りまでの中間くらいだった」「(ガード寄りでなく)まだ階段寄りだったと思う」(第六回公判)とか供述していて、その地点の特定は甚だ心許なく、特に距離の表現においては一〇〇メートルと一五歩という常識はずれの食い違い供述をしている点など誠に不自然で、これを単なる記憶の混乱や距離感覚のズレということで済ますことができるものか首を傾げざるを得ないのである。

この点原判決は、当時の被告人の犯行態様に対するH証言の不確かさも含め、深夜突然見ず知らずの男から襲われたのであるから、その際の正確な状況の再現を求めること自体に無理があるとの理解を示すのであるが、深夜一人歩きの途中に痴漢に襲われるという特異な体験をした女性が、最初に抱きつかれた地点という程度の比較的単純な事柄について、それほどに記憶が曖昧であるということは容易に肯きがたく、それでもそれが事実というのなら、証言は、漠然としたままの答えにとどめておけばいいものを、相互に矛盾するようなあれこれの供述をして帰一しないというのはやはり不自然というほかなく、かりに、Hが本件被害時の記憶に自信がなかったとしても、証言は、同女が実況見分に立ち会って現場においてその地点を既に指示して確定したのちのことであってみると、当該地点がそこであることをはっきり供述することができないはずはないのであって、にもかかわらず前記のように供述が混乱する理由を原判決のような理解で追認するわけにはいかない。

これをHの捜査段階での供述によって確かめてみると、「歩道橋を渡り終えて一〇歩と歩かない内に」(一二月八日付員面)と述べてわざわざ階段を下りてすぐの地点を図示し(同調書添付)、同じく「一〇歩も歩くか歩かない内に」(一二月九日付員面)と述べる一方で同調書の中でそれよりはるか前方の「ナカヤビル」前をその地点と指示し(添付住宅地図)、検察官の取調に至っては「さきに一〇歩も歩かない内にと供述したのは、私の感じで話したのであり、(実際は)歩道橋を渡り終えて少し歩いた時で、現場で警察の人に説明した場所が正確である」旨(一二月一四日付検面)供述が変化しているのであるが、これら供述内容を前記H証言と対比するときは、供述全体の矛盾は更に広がるのであり、とくに、まさに被害直後の捜査官に対する取調では、歩道橋を下りて一〇歩も行かないすぐにと説明していた最初の被害地点を、のちに三〇メートル以上も前方の「ナカヤビル」前と言い直したその供述の変遷は、理由が薄弱で直ちには納得できない。

結局、このように最初に被告人から抱きつかれたという地点の特定に関するHの供述が混乱し、かつ変遷するというのは、同女が正確にその地点を記憶していないのではないかという疑いを抱かせるだけで済まず、歩道橋下り口から本件ガード入口付近までの間で同女が被告人に後ろから抱きつかれるという実体験をしたこと自体にも疑念を投げかける状況と捉えることも可能である。

なお、原審及び当審における各現場の検証結果によれば、Hが最初に被告人から襲われたと供述する前記「ナカヤビル」前の点付近は、街灯やビルの玄関灯などで相当に明るく、しかも同ビルは住宅兼用建物でもあったということで、そもそも通行中の女性にわいせつ行為に及ぼうとして襲いかかる場所として相応しいところとは思えず、現に被告人は、その地点からは約三五メートルも先の暗いガード下にまでHを連れ込んでから初めてわいせつ行為に及んだというものであって、そこに至るまでの間は同女の身体に対していかがわしい所業に出ることはなかったというのであるから、それなら何のためにそのような地点で早々の犯行に着手する必要があったのか、犯人心理をいろいろ忖度して状況を推理してみてもその合理的理由を見出すのは困難であり疑惑は一層高まるのである。

(イ) 本件被害をHが予見できなかったという点について

H証言によれば、同女は、当時痴漢が出るという噂があるのを知りながら本件現場付近道路を深夜初めて一人で通行中いきなり背後から被告人に襲われたと供述する一方で、その際抱きつかれるまで人が近づいて来ることには全く気が付かなかったと述べ、そのことは同女の捜査段階からの言い分としても一貫しているのであるが、同女が供述する被害状況に即して容易に信じがたいことである。

すなわち、Hは、前記歩道橋の手前の路傍に停車中の乗用車内に男が座っているのを見たのち、付近に痴漢が出没する噂を聞いていたこともあって気になり、歩道橋の階段を上がる前には後ろを振り返って用心もしたというのであるから、当然神経は後方にも向けられて歩行していたと考えられるところ、もし、Hが証言するように、歩道橋を下りて四〇メートル先の前記「ナカヤビル」前で被告人に襲われたのが事実とすれば、乗用車内にいたという被告人はHに倍する早さの小走り歩調で追いかけて来なければ追いつけない計算になり、そのときには、当時サンダル履きであったという被告人が接近してくる足音がHの耳に入らないはずはないのであって(当審検証によって確認済み)、果たしてHに気付かれないように密かに近づいてきた被告人が歩道橋下り口から約四〇メートルも先の点でいきなりHに襲いかかるといった事態が実際にあったのか疑問を抱かされるのもやむを得まい。

(ウ) 被告人の抱きつき方等について

Hが被告人から抱きつかれた態様等について、H証言は、最初は、被告人の右手は同女の右肩からみぞおち辺りに、左手は同女の左肘を上から押さえ込むようにして抱きかかえ、やがては左手が同女の左脇に移って胴を抱えるような形にはなったものの、その両手が同女の体の前で組まれることはなかったこと、被告人はそのようにHを抱いたまま後ろから押すようにして黙って前方に進んだが、その間同女の乳房を掴むなどわいせつ行為には及んでおらず、その不安も感じなかったことなどと供述しているが、これは強制わいせつを目的とした犯人の行為としては極めて不自然なものといわなければならない。

すなわち、結局は、そこから約三五メートルも先の暗がりのガード下に至って初めてわいせつ行為をするつもりであったとみられる男が、相手女性に騒がれたり、暴れて逃げられたり、あるいは通行人や近所の住民に咎められて捕まったりするおそれもある明るい道路上でいち早く犯行に着手することの不可解さもさることながら、それでもあえて行為に出るというのなら、相手を取り逃がさないように後ろからその両手をしっかり羽交い締めにするとか、自分の手を前で組み合わせるとかして確実にその自由を制圧しようとするのが当然と思われるのに、H証言がいう程度の抱きつき方では、相手が本気になって抵抗した場合、果たしてその逃走を防ぐに足るものか甚だ疑わしいからである。

(この点、原判決は、弁護人側の行った犯行再実験による被害者の脱出可能とする結論に対し、当時の条件を同じように設定することは困難であるとしてその証拠価値を否定するのであるが、確かにそのときの被害者の驚愕や畏怖心など心理的要因まで加えた厳格な意味での状況再現はむずかしいとはいえようが、犯人側の行動を予測するに当たりその将来の可能性を知る限りの資料としてなら、被害者が全力で犯人の手を逃れようとしたときという前提での実験も決して無意味なものとはいえない。)

ところで、以上の本件犯行態様等につき、捜査段階におけるHの供述をみてみると、前記証言内容とは実質的に大きな違いがあることが分かる。

つまり、被害当日である昭和六三年一二月八日付員面では、「(犯人は)両手で私の体を羽交い締めにした」「羽交い締めの状態のまま暗い本件ガード下まで引きずって行った」「陸橋の下で羽交い締めにしていた右手を私の右肩に持ってきて左手は羽交い締めの状態」と、翌九日付員面(告訴調書)では、「後ろから抱きつき、服の上から私の両方のおっぱいをギュッと掴んできた、胸が痛かった」「いつの間にか、犯人の手は、右手は私の右肩から胸、左手は左脇腹から胸という恰好になった」とそれぞれ述べたのち、同月一四日実施されたH実況見分に立ち会って指示説明したのち、当日の検察官の取調(同日付検面)において、「男は私の後ろから抱きついて手を私の胸のあたりに回していた。警察で私の体を羽交い締めにしたと言ったのは、今日説明したように後ろから両手で抱きついてきたという状況を説明したもの」と供述するところ、そこでいう今日の説明という内容をその実況見分調書(同月二六日作成)添付の写真五葉目によって確かめてみると、犯人がHの両上腕部あたりを後ろから両手で抱え込むような情景が写されているのであって、してみると、検面供述は、羽交い締めというのでなければ、H証言がいう抱きつき方でもない態様に訂正されていると見ざるを得ないことになるが、これら捜査段階でのHの供述は、相互にも大きく変遷し、そのいずれもが原審証言と実質的に軽視できない違いがあることは明らかである。

H証言では、H自身「羽交い締め」というのがどんなものか知らず、警察での取調のときには自分からはそのような表現はしていない旨説明し、前記検面調書でもその趣旨の釈明をしていることになるが、各員面調書の取調官がHが供述もしないことを勝手に調書化したり、「羽交い締め」の意味を誤解して記述したりしたとは、それら供述内容がその後の抱き方の変化にも触れていることからしても信じがたいところであって、検面調書において、起訴(昭和六三年一二月一八日)後一週間以上も経てから作成された前記実況見分調書の写真描写を遡って引用した恰好の供述が、真相を語ったものとも考えられない。

要するに、Hは、本件事件直後及びその翌日には、捜査官に対し、被告人から最初は羽交い締めの状態で体の自由を制約され、しかもすぐその場で両方の乳房をきつく掴まれたというわいせつ被害までも申告しておきながら、のちにこれを撤回し、原審証言では、前述のとおり、およそ強制わいせつ犯人の犯行としては状況的に不自然不合理といわざるを得ないような態様の説明に後退しているとみられるのであって、もし後者の証言が事実とするなら、事件直後における前者の供述は、当然虚偽もしくは過剰な被害申告といわざるを得ないことになるが、Hの口からそのような供述変遷についての納得がいく理由をついに聞くことはできない。

H証言中、被告人から最初に抱きつかれた際の態様については、先に指摘したとおり捜査段階供述とも大きな相違があり、事実被告人からいうような襲われ方をしたのか多分に疑問があるものといわざるを得ない。

(原判決は、H証言とその捜査段階供述とは、大綱において供述内容が一致していて信用するに足るものとするようだが、以上に検討したような重要な事実関係の矛盾に言及しないで、他の大筋が合致することのみで信用性の保証があるとすることはできない。)

(エ) 連行方法やこれに対するHの対応等について

H証言による被告人の犯行態様が、最初の襲撃地点の地理的条件や抱きつき方において強制わいせつ犯人の行為として不自然と思える点があることは前述のとおりであるが、更に、Hを本件ガード入口方向へ連行する方法につき、羽交い締めもせず、かつ両手をHの体の前で組むということもしないまま、ただ押すように前方に進み、Hも前に逃げようとしていたため二人はもつれる様なこともなく、小走りといった感じで同ガード入口あたりまで移動した旨述べている部分も、強制わいせつを目的とする犯人の行為として何ら切実感を伴わない事実経緯であって、その際の犯人が真剣に相手女性を逃がさないようしっかり身柄を確保する制圧手段に出ていたものか疑問である。

そして、このような被害に遇ったというHは、その難を免れるため本能的にでも相当の抵抗を行うはずであるのに、H証言が述べるその際の同女の対応というのは、およそ抵抗と呼ぶには程遠い行為にとどまっているばかりでなく、同女自身が必死の抵抗をしたこと自体を言いはばかっているようにさえ見えるのが奇妙である。

すなわち、前記「ナカヤビル」前付近で突然後ろから被告人に抱きかかえられたというHは、とっさに痴漢に襲われたものと直感し、「キャー」とか「助けて」とか大声で叫びながら、掴んでいる犯人の手を解こうとして自分の手を伸ばしたり、右手に持っていたカバンを一応振り回したりしたが駄目で、とにかく明るいところへ行こうと思って必死に前に進んだが、犯人の方もちょうど押してくる感じで二人はそのまま前方に進み犯人の手は解けなかった、というのであるが、そこで述べられているHの対応振りが、深夜一人通行中の女性が痴漢に襲われて必死に難を免れようと抵抗している場面のものと眺めて極めて現実味に乏しいことは説明の要を見ないほどである。

普通女性が当夜のHと同じ被害に出くわしたとしたならば、まずは我が身を守るため、本能的、反射的にも犯人の手を振り払って逃げようとする行為に出るはずであり、そのときにはその手を掴んで引き外そうとし、あるいは身を捩って暴れ、場合によっては犯人の手を引っ掻いたり噛んだりもするなど必死の防御ないし抵抗を行うことが予想され、しかも、犯人の最初の抱きつき方というのが、前記のような甘い態様のものであったとしたら、被害者が女性とはいえ瞬発的に両手を拡げて前に駆け出すことで制縛を脱することが決して不可能とは思えず、少なくともその方法を試して当然と思えるのに、そのような行為には出ず、ただ手を前に伸ばしたり、持っていたカバンを手放しもしないで振り回すだけで、二人三脚よろしく犯人が女性の後ろから抱きついた状態のまま小走りで進んで行ったという犯行形態では、誠に間延びした悠長さが目立ち現実的迫真性がまるで感じられないのである。

この点をH証言で具体的にみてみると、例えば第六回公判で、当時どういう抵抗をしたかとの質問に対し、「一応、手をとにかく取ろうと思うんだけども、取れないというのかな、後はもう、押されるから、そんな抵抗どころじゃなくて、私も行かなきゃいけないと思うから、ある程度、鞄は振り回すけど、そっちのほうが、なんか二つ一緒にできないのかな、そっちのほう向かって歩いてた、必死で、もう歩いてたという感じです」と答えているが、これなどHがまともな抵抗をしていないことを自認したうえでの弁明とも受け取れそうな供述内容であって、中に必死とか暴れたとかの証言部分はあってもそれは内容空疎であり、宙に浮いたものといわざるを得ない。

当審におけるH証言によっても、以上の疑問点はより深まりこそしても、これを解消することはできないものである。

もっとも、一般的にいえば、後ろから突然男に抱きつかれるという被害に遇った女性が恐怖心のあまり萎縮して十分な抵抗ができなかったという場合がないわけではないが、本件当時のHの場合、痴漢が出るという噂を知りながら、当夜とくにその必要があったとも思えないのに初めてというその現場付近の夜道を一人で歩いていたものであるうえ、証言によれば、同女は、身長163.5センチと大柄で、剣道二段の腕前があるほか、学生時代から各種スポーツに親しんでいて体力に自信もあり、性格は自ら気丈であることを認めており、更に本件にあっては、被告人が逃げ出したのを「この野郎、なにするげ」「逃げるな、くそ」など女性にしては口汚いと思える言葉を吐きながら直ちに追跡し(告訴調書)、被告人を捕らえる際にはその頭を叩いたりもしたというのであって、そのようなHが畏怖心によって抵抗できなかったということは到底考えられないのである。

なお、Hの捜査段階供述では、「必死にもがき暴れた」(告訴調書)とか、「逃げようとしたが、男の力が強く、振り払うこともできず、ガード下の方へ引っ張り込まれた」(検面調書)とか述べている部分もあるが、いずれも抽象的な表現で具体的内容に欠け、これによってH証言の信用性を補完するに足るようなものではない。

(オ) とくに悲鳴の点について

Hが本件被害に遇ったときの対応として、犯人に抱きつかれた最初から犯人が逃走するまでの間、ずっと大声で悲鳴を上げ続けていたというH証言は、本件事実関係を究明するうえでとくに重要である。

証言では、Hは、前記「ナカヤビル」前の点で突然被告人に後ろから抱きつかれ、痴漢に襲われたと直感し、その場で「キャー」とか「助けて」とか大声で悲鳴を上げ、更にそこから抱きつかれたままの状態で押されるように前に進んで本件ガード入口付近に至ったのち暗いガード下に引きずり込まれてわいせつ行為に及ばれるまでの約三五メートルの間、ずっと悲鳴を上げ続けていたというのであるが、この事実は、捜査段階からも一貫してHが供述するところでもあり、また、当審における同女の改めての証言でもこれは変わらず、「人に気付いてもらおうと思ってかなりの大声で何回も悲鳴を上げ続け、一〇回前後は叫んだと思う」旨明言しているのである。

そしてもし、本件当時、H証言どおりに何回も悲鳴を上げ続けていたという事実が立証されるのであれば、同女が供述するような態様の本件犯行が実際に行われたことの有力な証左になるだろうし、それが逆に否定されるならばH証言の信用性が大きく減殺されることになるのはいうまでもない。

そこで検討するに、本件犯行があったとされる当夜「ナカヤビル」前付近から遠くない自宅で寝ていたというMの証人尋問調書によれば、同女はその時刻ころ寝つかれないまま布団に横たわっていたところ、「キャー」という女性特有の甲高い大きな悲鳴を一回聞き、何だろうなと思って耳を澄ましていたが再びは聞こえなかったというのであり、一方で、不審者として被告人を追尾していたというYの原審証言(第七回公判)でも同旨の供述がなされていたことからすると、Hが何回も悲鳴を上げ続けていたという証言の真実性は極めて怪しいものであり、当時Hが発した悲鳴は一回限りと認定するのが相当であろう。

原判決はこの点、Hはとっさの出来事で記憶が混乱して、抵抗した以上何回も大きな悲鳴を上げたはずと思い込んで証言したものと考えられ、事実はYが第八回公判で証言し直したとおり、悲鳴の大きなものは二回であったと認められると判断するのであるが、当審検証結果によって悲鳴が上げられたと推定される付近よりはせいぜい約三〇メートルと目測される近距離の自宅二階の寝室にいて悲鳴を聞き、その後も意識的に気をつけて耳を澄ましていたというM証言の証拠価値を否定する状況は何もなく、これに対し二回の悲鳴を聞いたというYの訂正証言は、本人自身の捜査段階から通じての過去の供述内容と対比しても措信することは困難である。

原判決の認定では、Hが悲鳴を上げ続けたというのは錯覚かも知れないとし、また、大きな悲鳴は二回だけとする反面で小さな悲鳴なら上げていたかも知れないと考える余地を残すことでH証言の信用性をカバーしているように見えるのであるが、助けを求めて人に気付いてもらおうと大声で悲鳴を上げ続けたという同証言に曖昧さはなく、これが誤信に基づいての供述とは考えられない。

となると、実際には一回に過ぎなかった悲鳴を、何回も叫び続けたと述べるH証言は、事実に反していることを知りながらことさらの主張をしている疑いが濃厚であり、これは、原判決がいうように、抵抗した以上は何回も悲鳴を上げたはずと思い込んだとみるよりか、同女が捜査段階から維持する本件被害状況に基づけば、犯人に抱きかかえられて相当距離を移動する間に一切抵抗をしなかったというわけにはいかず、そのときには当然悲鳴を上げ続けていたと説明せざるを得なかったという事情によるのではなかろうかと疑う方がより理に適ったものというべきである。

そして、当時の悲鳴が一回に限るものとすれば、それはHが最初に被告人から襲われたという「ナカヤビル」前付近とみるほかないのであるが、その場合には、それに引き続いて約三五メートルも被告人に体を抱えられてガード下まで連行されわいせつ行為に及ばれるまでの間、Hは終始無言のままでいたという常識上はあり得ない被害状況を想定しなければならないことになるのである。

そればかりでなく、更に当時被告人を追尾していて悲鳴を聞いたという前記Yの供述証拠及び当審の検証結果によって明らかにされた現場の地理的条件等を基にして、その一声の悲鳴が発せられた地点の位置を推量してみるのに、それはH証言がいう「ナカヤビル」前の点付近ではなく、本件ガード入口の点付近と認めるのが合理的と考えられ、この点に限っていえば、同点付近で初めてHとの関わり合いがあったという被告人弁解と符合するところである。

以上の検討からしても、本件犯行が「ナカヤビル」前の点で始まったとするH証言の信用性は大きく揺らぐものといわなければならない。

(カ) わいせつ行為及び転倒について

H証言では、要するに、被告人は同女を無理やり本件ガード下に引きずり込み、点付近で右手を同女のワンピースの首もとから入れてその左乳房を二、三回弄んだが、同女がよろめいて転んだとたん急に体を離してその場から逃走した旨を供述するのであるが、その供述内容は全体的に曖昧、不分明なきらいがあって安定せず、具体的とはいってもそれはかえって供述の混乱を示すものと見ることもできるなど、それ自体の信用性は決して高いものとはいいがたいところ、他の関係証拠や状況をみても、これを補強するどころか、かえって減殺する消極的事情さえ浮かび上がってくるのである。

まず、わいせつ行為についてのH証言をみれば、それまでHの体にいたずらをする気配など全くなかったという被告人が、本件ガード下に来て、急に右手をHの着衣の中に突っ込んできたというのであるが、その態様は、ガード下に引っ張り込まれて点に向かって移動している最中に手を入れられたもので、一応抵抗もしていたので簡単に入れられたわけではないという一方で、その手がいつ入ってきたか分からないくらいに入ってきたとも説明し、その後被告人に乳房を素手で二、三回揉まれるという女性にとって恥ずかしい行為に及ばれた間にHがそれを嫌がって必死に抵抗した様子を窺わせる供述は欠落しているのであって、これら証言が、強制わいせつ事犯の具体的状況の描写として甚だしく現実感に乏しいものであることはいうまでもなく、更にその時点での犯行を目撃したかのようにいうY証言が原、当審検証結果によって客観的に否定されることは後述のとおりであるし、右わいせつ行為を含めて被告人の犯行を窺わせるHの着衣や身体の損傷、痕跡もないことも証拠上明らかであって、これら事情もH証言の信用性についてマイナス評価を与えるものであることはいうまでもない。

ついで、転倒についての供述内容を探ってみると、H証言は結論的には転倒したと供述しているものとみざるを得ないのではあるが、個々の供述では、「その後私がこけそうになったんです。足がもつれたのか、とにかく転んだ。私が左手で地面を支え、仰向けにちょっと左傾く程度で転んだ。」(第五回公判)、「私はこけそうになった。倒れてしまいはしなかったけど、お尻はつかなかったけど、一応左手ついて倒れた。砂利の道に足がもつれたのか、倒されそうになったか分からない。何かに、先、足もつれて、倒れそうになったときに、ぱっと手が離れた。」(第六回公判)などと状況を説明するものであって、それらが本当に転倒事実をいうものかどうか甚だ不得要領のものであり、これまた心証に結び付かない。

これらの点についてのHの捜査段階での供述の経緯を調べてみると、当審で初めて取り調べた現行犯人逮捕手続書では、「この男がいきなり私に抱きついて来て、陸橋下に引き倒され、私の首から手を入れ胸を触った。」旨警察官に説明したとされ、事件当日のH員面では、「ワンピースの首の所から右手を入れて来て、左のオッパイを直接触わられた。」旨供述するものの、転倒の事実には触れておらず、翌日付の告訴調書では、「胸を触られたとき、倒れそうになったこともあった」旨述べているのに、一二月一四日実施のH実況見分での指示説明では、被告人に乳房を触られたと言いながら(着衣の中に手を入れられたとは述べてない)、転倒状態になった旨を供述した形跡はなく、当日付の検面調書でも、突然男が離れて行ったというのみで転倒事実の供述はみられないのであって、そのような供述経過に照らすと、果たして転倒という事実があったのか極めて疑わしく、それよりも、現行犯人逮捕時にHが「引き倒されて首から手を入れられて胸に触られた。」と述べたというのがもし事実なら、それは同女の他の供述とは完全に遊離した過剰な被害申告とみるほかないのであって、事件直後におけるそのようなHの供述姿勢は、本件におけるその供述全般の信用性にも大きく関わるものといわなければならない。
\end{quote}

\begin{quote}
《引用の終わり》
\end{quote}

\begin{itemize}
\tightlist
\item
  名古屋高等裁判所金沢支部 平成2年(う)27号 判決 -
  大判例 \url{https://daihanrei.com/l/\%E5\%90\%8D\%E5\%8F\%A4\%E5\%B1\%8B\%E9\%AB\%98\%E7\%AD\%89\%E8\%A3\%81\%E5\%88\%A4\%E6\%89\%80\%E9\%87\%91\%E6\%B2\%A2\%E6\%94\%AF\%E9\%83\%A8\%20\%E5\%B9\%B3\%E6\%88\%90\%EF\%BC\%92\%E5\%B9\%B4\%EF\%BC\%88\%E3\%81\%86\%EF\%BC\%89\%EF\%BC\%92\%EF\%BC\%97\%E5\%8F\%B7\%20\%E5\%88\%A4\%E6\%B1\%BAn} 
\end{itemize}

 上記に判決文の「(1) H証言自体に内在する問題点について」という部分を引用しました。けっこうな分量の記述ですが,項目が(ア)から(カ)まで「(カ) わいせつ行為及び転倒について」のような見出し付きで細分化されています。

 被害者をなめくさり愚弄しているとも思える無罪判決を導く理屈が並んでいます。個々に見れば一応の理屈があるようにも見えなくはないですが,他の客観的に明らかな事実関係の骨格あるいは骨組みを軽視するもので,にわかづくりのダンボール小屋のような判決というのが率直な感想です。

 裁判長によるこのようなでたらめな判決がまかり通り無罪判決が出たのも,事件番号が平成2年という当時の社会状況に照らし,無理からぬものがあるかとも思うのですが,この濱田武律裁判長の後任となった被告発人小島裕史裁判長にも似たような傾向,奢りがあると強く感じました。

 この「名古屋高等裁判所金沢支部 平成2年(う)27号 判決 -
大判例」というネット上の情報を昨日,見つけたたのは,神原元弁護士,自由法曹団常任幹事,梨木作次郎弁護士というリレーのような流れがありました。

 上記の引用部分は「(1) H証言自体に内在する問題点について」です。おそらく5分か長くて10分程度の時間の出来事を吟味,評価するのにこれだけの記述を判決文に割いているのだと思いますが,前に読んだ,珠洲市の保険金目的母親殺害事件で無期懲役の判決文より記載が多そうです。

 この逆転無罪判決を,勝ち取ったというのか大手柄をたてたというか,その立役者が梨木作次郎弁護士で,同じ珠洲市の蛸島事件でも金沢地方裁判所七尾支部で殺人事件の無罪判決を出させています。石川県奥能登でおきた蛸島学童殺人事件裁判記録という本の記録が残されています。

 この濱田武律裁判長の控訴審逆転無罪判決ですが,金沢市内の神田陸橋付近を,「Hが最初に被告人から襲われたと供述する前記「ナカヤビル」前の点付近は、街灯やビルの玄関灯などで相当に明るく、」とし,

 さらに「しかも同ビルは住宅兼用建物でもあったということで、そもそも通行中の女性にわいせつ行為に及ぼうとして襲いかかる場所として相応しいところとは思えず、」と続けています。

 車で通行することしかなかった場所ですが,昭和63年12月当時に,そのような「街灯やビルの玄関灯などで相当に明るく」という場所であったとは思えず,疑問です。

 この判決文では「金沢市〈番地略〉先の主要地方道金沢・美川・小松線の上り線側道の路上において」となっていますが,神田陸橋とあるので,野田専光寺線に間違いはないと思います。現在はどうかわからいませんが,それが通常の道路名で,ラジオの交通情報にもありました。

 昭和58年当時ですが,増泉方面から来て神田陸橋の手前に右に入る道路がありました。信号機はあったように思います。右に入ると道路が右斜めに折れるのですが,ちょうどその角の辺りに小さな消防署がありました。

 右斜めに折れたまま直進すると,間もなく北陸本線だと思いますが,その下をくぐるガードがありました。かなり低いガードであったように思います。対向車のすれ違いができないほど道幅が狭かったような気もするのですが,裏道でありながら交通量の少ない道路でした。

 30年以上前の記憶のままなので多少違っている点もあるかもしれないですが,ガード下をくぐると一直線の道路で,右側は住宅が散在し,左側はほとんど田んぼだったように思います。見通しが良かったですが,次の大通りに出る手前,右手に極真空手の道場がありました。

 被告発人大網健二が金沢高校の高校生の時,通っていたとも聞いた極真空手の道場です。その先の大通りの交差点は,向かい側の左手にカーチェイスという車屋があり,昭和63年だと北安江の新店舗に移転していたかもしれません。

 同じく右手には「なんだかんだ」というお好み焼き店がありました。昭和63年の1月に,金沢市場輸送の輪島の社員運転手MYと一緒に食事をし,それから福島方面に向けて一緒に出発したと記憶にあります。

 濱田武律裁判長の判決文には,犯行現場の地理的説明が細かく出てくるのですが,ガード下というのもあります。ページ内検索で47件となっています。

 神田陸橋に歩道はなかったと思うのですが,車道より下に歩道があったのかと当初は想像していました。北陸本線の線路は近くで見ていますが,近くに踏切はなかったと思います。また,昭和63年当時であれば,付近の住宅はまばらだったと思います。

 少し思い出したのですが,増泉方面からから来て神田陸橋を渡ると,左手にマクドナルドのハンバーガー店があったように思います。商業ビルのようなものは記憶にないのですが,向かい側に昨夜記述した家電店がありました。

\begin{itemize}
\tightlist
\item
  神田 - Google マップ \url{https://t.co/DUSrxwOWtu} 
\end{itemize}

 Googleマップでみると,北陸本線を境に神田と新神田がはっきり線引きされています。航空写真に切り替えると,画質や道路がくっきりきれいすぎて,模型の町並みを見ているようです。

 県道25号線ともなっていますが,野田専光寺線は,Googleマップで西インター大通りとなっているようです。

\begin{itemize}
\tightlist
\item
  クリアコーポレーション - Google マップ \url{https://t.co/KG7GGWdvrX} 
\end{itemize}

 上記のクリアコーポレーションとある場所が,昭和63年1月当時,「なんだかんだ」というお好み焼き店があった場所になると思います。前の道路が「まめだ大通り」となっていますが,東力2丁目のアパートから金沢市場輸送への通勤にほとんど使う道路でした。

 神田の消防署のすぐ近くには2つ年上の先輩が,私より3つ年下の少女と同棲するアパートがあって,昭和58年当時にはちょくちょく遊びに行っていました。その近くに,昭和59年の秋,被告発人安田敏と遊びに行ったアパートがあり,被告発人安田敏の珠洲市の友人2組が住んでいました。

 また,濱田武律裁判長の判決文に,被害者の24歳の女性は,アルバイトをする片町のスナックから歩いて神田陸橋に来たとありましたが,徒歩で40分とあり,そんなに時間がかかるものかと思いました。

\begin{itemize}
\tightlist
\item
  片町 から 神田 - Google マップ \url{https://t.co/AdWiesx9Oe} 
\end{itemize}

 経路を調べると,徒歩で21分,1.7kmとあります。裁判所が検証を行ったとは考えにくいので,梨木作次郎弁護士らの主張なのではと思われます。あるいは警察の供述調書にそうあるのかもしれません。

\begin{itemize}
\tightlist
\item
  〈〈〈 2021/05/14 09:59:20 Linux Emacs: 〈〈〈
\end{itemize}

\hypertarget{ux88abux544aux767aux4ebaux5ca1ux7530ux9032ux5f01ux8b77ux58ebux304cux56fdux9078ux5f01ux8b77ux4ebaux3067ux95a2ux4e0eux3057ux305fux5e73ux62104ux5e74ux50b7ux5bb3ux6e96ux5f37ux59e6ux88abux544aux4e8bux4ef6-ux91d1ux6ca2ux5730ux65b9ux88c1ux5224ux6240-ux516cux5224ux306eux5199ux771fux8a18ux9332ux8cc7ux6599ux4e00ux89a7ux4e09ux5b85ux4fcaux4e00ux90ceux88c1ux5224ux9577-ux5dddux53e3ux6cf0ux53f8ux88c1ux5224ux5b98-ux5c71ux7530ux5fb9ux88c1ux5224ux5b98}{%
\paragraph{被告発人岡田進弁護士が国選弁護人で関与した平成4年傷害・準強姦被告事件 金沢地方裁判所 公判の写真記録資料一覧(三宅俊一郎裁判長 川口泰司裁判官 山田徹裁判官)}\label{ux88abux544aux767aux4ebaux5ca1ux7530ux9032ux5f01ux8b77ux58ebux304cux56fdux9078ux5f01ux8b77ux4ebaux3067ux95a2ux4e0eux3057ux305fux5e73ux62104ux5e74ux50b7ux5bb3ux6e96ux5f37ux59e6ux88abux544aux4e8bux4ef6-ux91d1ux6ca2ux5730ux65b9ux88c1ux5224ux6240-ux516cux5224ux306eux5199ux771fux8a18ux9332ux8cc7ux6599ux4e00ux89a7ux4e09ux5b85ux4fcaux4e00ux90ceux88c1ux5224ux9577-ux5dddux53e3ux6cf0ux53f8ux88c1ux5224ux5b98-ux5c71ux7530ux5fb9ux88c1ux5224ux5b98}}

\begin{itemize}
\tightlist
\item
  〉〉〉 Linux Emacs: 2021/05/14 12:14:02 〉〉〉
\end{itemize}

:CATEGORIES: @kanazawabengosi \#金沢弁護士会 @JFBAsns
日本弁護士連合会(日弁連) \#法務省 @MOJ\_HOUMU \#三宅俊一郎裁判長
\#川口泰司裁判官 \#山田徹裁判官 \#被告発人岡田進弁護士

\begin{itemize}
\item
  TW s\_hirono(非常上告-最高検察庁御中\_ツイッター) 日時: 2021-05-14
  11:48 URL: \url{https://twitter.com/s\_hirono/status/1393035359026499587} 
  \textgreater{} H04-06-18\_第一回公判調書\_金沢地方裁判所\_01.jpg
  \url{https://t.co/whLZwmqDMH} 
\item
  TW s\_hirono(非常上告-最高検察庁御中\_ツイッター) 日時: 2021-05-14
  11:48 URL: \url{https://twitter.com/s\_hirono/status/1393035374084063237} 
  \textgreater{} H04-06-18\_第一回公判調書\_金沢地方裁判所\_02.jpg
  \url{https://t.co/aabuEkaXNQ} 
\item
  TW s\_hirono(非常上告-最高検察庁御中\_ツイッター) 日時: 2021-05-14
  11:48 URL: \url{https://twitter.com/s\_hirono/status/1393035390081069057} 
  \textgreater{} H04-06-18\_第一回公判調書\_金沢地方裁判所\_03.jpg
  \url{https://t.co/fb86Ff9bze} 
\item
  TW s\_hirono(非常上告-最高検察庁御中\_ツイッター) 日時: 2021-05-14
  11:51 URL: \url{https://twitter.com/s\_hirono/status/1393036130497359874} 
  \textgreater{}
  H04-06-18\_冒頭陳述要旨\_金沢地方検察庁江村正之検察官\_01.jpg
  \url{https://t.co/YI4zN8lPlq} 
\item
  TW s\_hirono(非常上告-最高検察庁御中\_ツイッター) 日時: 2021-05-14
  11:51 URL: \url{https://twitter.com/s\_hirono/status/1393036145760432129} 
  \textgreater{}
  H04-06-18\_冒頭陳述要旨\_金沢地方検察庁江村正之検察官\_02.jpg
  \url{https://t.co/KbSwyc2d4m} 
\item
  TW s\_hirono(非常上告-最高検察庁御中\_ツイッター) 日時: 2021-05-14
  11:51 URL: \url{https://twitter.com/s\_hirono/status/1393036161396875272} 
  \textgreater{}
  H04-06-18\_冒頭陳述要旨\_金沢地方検察庁江村正之検察官\_03.jpg
  \url{https://t.co/DU8QGytynO} 
\item
  TW s\_hirono(非常上告-最高検察庁御中\_ツイッター) 日時: 2021-05-14
  11:51 URL: \url{https://twitter.com/s\_hirono/status/1393036176806670341} 
  \textgreater{}
  H04-06-18\_冒頭陳述要旨\_金沢地方検察庁江村正之検察官\_04.jpg
  \url{https://t.co/3J6Z4qrSHV} 
\item
  TW s\_hirono(非常上告-最高検察庁御中\_ツイッター) 日時: 2021-05-14
  11:52 URL: \url{https://twitter.com/s\_hirono/status/1393036362291417089} 
  \textgreater{} H04-06-29\_第二回公判調書\_金沢地方裁判所\_01.jpg
  \url{https://t.co/3oZxLbYfdn} 
\item
  TW s\_hirono(非常上告-最高検察庁御中\_ツイッター) 日時: 2021-05-14
  11:52 URL: \url{https://twitter.com/s\_hirono/status/1393036377881681924} 
  \textgreater{} H04-06-29\_第二回公判調書\_金沢地方裁判所\_02.jpg
  \url{https://t.co/tjjkRCldP0} 
\item
  TW s\_hirono(非常上告-最高検察庁御中\_ツイッター) 日時: 2021-05-14
  11:52 URL: \url{https://twitter.com/s\_hirono/status/1393036529895759878} 
  \textgreater{}
  H04-06-29\_論告要旨\_金沢地方検察庁\_検察官検事江村正之\_01.jpg
  \url{https://t.co/C1zLEBIoAY} 
\item
  TW s\_hirono(非常上告-最高検察庁御中\_ツイッター) 日時: 2021-05-14
  11:52 URL: \url{https://twitter.com/s\_hirono/status/1393036543296634886} 
  \textgreater{}
  H04-06-29\_論告要旨\_金沢地方検察庁\_検察官検事江村正之\_02.jpg
  \url{https://t.co/c6I3jp956p} 
\item
  TW s\_hirono(非常上告-最高検察庁御中\_ツイッター) 日時: 2021-05-14
  11:52 URL: \url{https://twitter.com/s\_hirono/status/1393036556428976129} 
  \textgreater{}
  H04-06-29\_論告要旨\_金沢地方検察庁\_検察官検事江村正之\_03.jpg
  \url{https://t.co/lLe50XaloX} 
\item
  TW s\_hirono(非常上告-最高検察庁御中\_ツイッター) 日時: 2021-05-14
  11:53 URL: \url{https://twitter.com/s\_hirono/status/1393036708455731209} 
  \textgreater{} H04-06-30\_公判調書\_金沢地方裁判所\_01.jpg
  \url{https://t.co/WiEGdqHv2P} 
\item
  TW s\_hirono(非常上告-最高検察庁御中\_ツイッター) 日時: 2021-05-14
  11:53 URL: \url{https://twitter.com/s\_hirono/status/1393036721688748037} 
  \textgreater{} H04-06-30\_公判調書\_金沢地方裁判所\_02.jpg
  \url{https://t.co/bRQasvg5wg} 
\item
  TW s\_hirono(非常上告-最高検察庁御中\_ツイッター) 日時: 2021-05-14
  11:53 URL: \url{https://twitter.com/s\_hirono/status/1393036735106273280} 
  \textgreater{} H04-06-30\_公判調書\_金沢地方裁判所\_03.jpg
  \url{https://t.co/YWMGylBD1a} 
\item
  TW s\_hirono(非常上告-最高検察庁御中\_ツイッター) 日時: 2021-05-14
  11:53 URL: \url{https://twitter.com/s\_hirono/status/1393036748469374976} 
  \textgreater{} H04-06-30\_公判調書\_金沢地方裁判所\_04.jpg
  \url{https://t.co/0ad4gsSgmY} 
\item
  TW s\_hirono(非常上告-最高検察庁御中\_ツイッター) 日時: 2021-05-14
  11:53 URL: \url{https://twitter.com/s\_hirono/status/1393036761987633152} 
  \textgreater{} H04-06-30\_公判調書\_金沢地方裁判所\_05.jpg
  \url{https://t.co/OJvojjSTm9} 
\item
  TW s\_hirono(非常上告-最高検察庁御中\_ツイッター) 日時: 2021-05-14
  11:53 URL: \url{https://twitter.com/s\_hirono/status/1393036775405150214} 
  \textgreater{} H04-06-30\_公判調書\_金沢地方裁判所\_06.jpg
  \url{https://t.co/UnR4W54fiT} 
\item
  TW s\_hirono(非常上告-最高検察庁御中\_ツイッター) 日時: 2021-05-14
  11:53 URL: \url{https://twitter.com/s\_hirono/status/1393036788852150274} 
  \textgreater{} H04-06-30\_公判調書\_金沢地方裁判所\_07.jpg
  \url{https://t.co/0MiMC0eupT} 
\item
  TW s\_hirono(非常上告-最高検察庁御中\_ツイッター) 日時: 2021-05-14
  11:53 URL: \url{https://twitter.com/s\_hirono/status/1393036802110279680} 
  \textgreater{} H04-06-30\_公判調書\_金沢地方裁判所\_08.jpg
  \url{https://t.co/H6lLV6bMdz} 
\item
  TW s\_hirono(非常上告-最高検察庁御中\_ツイッター) 日時: 2021-05-14
  11:53 URL: \url{https://twitter.com/s\_hirono/status/1393036815448252417} 
  \textgreater{} H04-06-30\_公判調書\_金沢地方裁判所\_09.jpg
  \url{https://t.co/fMPhGqEyn1} 
\item
  TW s\_hirono(非常上告-最高検察庁御中\_ツイッター) 日時: 2021-05-14
  11:55 URL: \url{https://twitter.com/s\_hirono/status/1393037132957065220} 
  \textgreater{} H04-08-03\_第三回公判調書\_判決\_金沢地方裁判所\_01.jpg
  \url{https://t.co/3wxqrxsNKu} 
\item
  TW s\_hirono(非常上告-最高検察庁御中\_ツイッター) 日時: 2021-05-14
  11:55 URL: \url{https://twitter.com/s\_hirono/status/1393037147830046721} 
  \textgreater{} H04-08-03\_第三回公判調書\_判決\_金沢地方裁判所\_02.jpg
  \url{https://t.co/LHSlI0OvK3} 
\item
  TW s\_hirono(非常上告-最高検察庁御中\_ツイッター) 日時: 2021-05-14
  11:55 URL: \url{https://twitter.com/s\_hirono/status/1393037291598204934} 
  \textgreater{} H04-08-03\_判決\_金沢地方裁判所\_01.jpg
  \url{https://t.co/M5mnbUUUSI} 
\item
  TW s\_hirono(非常上告-最高検察庁御中\_ツイッター) 日時: 2021-05-14
  11:55 URL: \url{https://twitter.com/s\_hirono/status/1393037305561063427} 
  \textgreater{} H04-08-03\_判決\_金沢地方裁判所\_02.jpg
  \url{https://t.co/l8dT9hg0Kq} 
\item
  TW s\_hirono(非常上告-最高検察庁御中\_ツイッター) 日時: 2021-05-14
  11:55 URL: \url{https://twitter.com/s\_hirono/status/1393037319213514754} 
  \textgreater{} H04-08-03\_判決\_金沢地方裁判所\_03.jpg
  \url{https://t.co/ic4r6tu5vH} 
\item
  TW s\_hirono(非常上告-最高検察庁御中\_ツイッター) 日時: 2021-05-14
  11:55 URL: \url{https://twitter.com/s\_hirono/status/1393037332610117633} 
  \textgreater{} H04-08-03\_判決\_金沢地方裁判所\_04.jpg
  \url{https://t.co/Z5lUwojddo} 
\item
  TW s\_hirono(非常上告-最高検察庁御中\_ツイッター) 日時: 2021-05-14
  11:55 URL: \url{https://twitter.com/s\_hirono/status/1393037347281793027} 
  \textgreater{} H04-08-03\_判決\_金沢地方裁判所\_05.jpg
  \url{https://t.co/2o991LJ9qP} 
\item
  TW s\_hirono(非常上告-最高検察庁御中\_ツイッター) 日時: 2021-05-14
  12:04 URL: \url{https://twitter.com/s\_hirono/status/1393039613179011074} 
  \textgreater{} H04-04-30\_公判期日指定書.jpg \url{https://t.co/EcKAJ1dFAf} 
\item
  TW s\_hirono(非常上告-最高検察庁御中\_ツイッター) 日時: 2021-05-14
  12:05 URL: \url{https://twitter.com/s\_hirono/status/1393039628337180672} 
  \textgreater{} H04-05-01\_国選弁護人選任書\_岡田進弁護士.jpg
  \url{https://t.co/RKgFVNqLN1} 
\item
  TW s\_hirono(非常上告-最高検察庁御中\_ツイッター) 日時: 2021-05-14
  12:05 URL: \url{https://twitter.com/s\_hirono/status/1393039643579359232} 
  \textgreater{} H04-05-01\_請書\_弁護士岡田進.jpg
  \url{https://t.co/JrXNy4OhAB} 
\item
  TW s\_hirono(非常上告-最高検察庁御中\_ツイッター) 日時: 2021-05-14
  12:05 URL: \url{https://twitter.com/s\_hirono/status/1393039658670456833} 
  \textgreater{} H04-05-26\_電話聴取書\_弁護人岡田進.jpg
  \url{https://t.co/S9zRNfQ4J6} 
\item
  TW s\_hirono(非常上告-最高検察庁御中\_ツイッター) 日時: 2021-05-14
  12:05 URL: \url{https://twitter.com/s\_hirono/status/1393039673920950274} 
  \textgreater{} H04-05-27\_公判期日指定書\_裁判長三宅俊一郎.jpg
  \url{https://t.co/pNm6wqoHd3} 
\item
  TW s\_hirono(非常上告-最高検察庁御中\_ツイッター) 日時: 2021-05-14
  12:05 URL: \url{https://twitter.com/s\_hirono/status/1393039688923979777} 
  \textgreater{} H04-05-27\_公判期日取消決定\_裁判官川口泰司.jpg
  \url{https://t.co/ELi8jZimUP} 
\item
  TW s\_hirono(非常上告-最高検察庁御中\_ツイッター) 日時: 2021-05-14
  12:05 URL: \url{https://twitter.com/s\_hirono/status/1393039704933638148} 
  \textgreater{} H04-05-27\_公判期日取消決定謄本.jpg
  \url{https://t.co/6kBR6AI3vl} 
\item
  TW s\_hirono(非常上告-最高検察庁御中\_ツイッター) 日時: 2021-05-14
  12:05 URL: \url{https://twitter.com/s\_hirono/status/1393039719848505344} 
  \textgreater{} H04-05-27\_電話聴取書\_金沢地方検察庁事務官天山.jpg
  \url{https://t.co/26HRiHfUdv} 
\item
  TW s\_hirono(非常上告-最高検察庁御中\_ツイッター) 日時: 2021-05-14
  12:05 URL: \url{https://twitter.com/s\_hirono/status/1393039735057047554} 
  \textgreater{} H04-05-27\_弁論併合決定.jpg \url{https://t.co/8PVJuDVRfq} 
\item
  TW s\_hirono(非常上告-最高検察庁御中\_ツイッター) 日時: 2021-05-14
  12:05 URL: \url{https://twitter.com/s\_hirono/status/1393039749988814850} 
  \textgreater{} H04-05-28\_移監通知書.jpg \url{https://t.co/EOVVyIn7Dv} 
\item
  TW s\_hirono(非常上告-最高検察庁御中\_ツイッター) 日時: 2021-05-14
  12:05 URL: \url{https://twitter.com/s\_hirono/status/1393039765704900614} 
  \textgreater{} H04-05-29\_請書\_岡田進弁護士.jpg
  \url{https://t.co/4Q2Z81qxce} 
\item
  TW s\_hirono(非常上告-最高検察庁御中\_ツイッター) 日時: 2021-05-14
  12:05 URL: \url{https://twitter.com/s\_hirono/status/1393039780703719429} 
  \textgreater{} H04-06-18\_令状関係送付書.jpg \url{https://t.co/xBwXkKWVTB} 
\item
  〈〈〈 2021/05/14 12:15:12 Linux Emacs: 〈〈〈
\end{itemize}

\hypertarget{ux30a6ux30b7ux30b8ux30deux304fux3093ux4f5cux8005ux306eux65b0ux4f5cux5f01ux8b77ux58ebux6f2bux753bux4e5dux6761ux306eux5927ux7f6aux306bux6cd5ux66f9ux30afux30e9ux30b9ux30bfux304cux30deux30b8ux30ecux30b9ux30c4ux30c3ux30b3ux30dfux3068ux88abux544aux767aux4ebaux5ca1ux7530ux9032ux5f01ux8b77ux58ebux306eux56fdux9078ux5211ux4e8bux5f01ux8b77ux306eux5b9fux614b}{%
\paragraph{ウシジマくん作者の新作弁護士漫画「九条の大罪」に法曹クラスタがマジレスツッコミ、と被告発人岡田進弁護士の国選刑事弁護の実態}\label{ux30a6ux30b7ux30b8ux30deux304fux3093ux4f5cux8005ux306eux65b0ux4f5cux5f01ux8b77ux58ebux6f2bux753bux4e5dux6761ux306eux5927ux7f6aux306bux6cd5ux66f9ux30afux30e9ux30b9ux30bfux304cux30deux30b8ux30ecux30b9ux30c4ux30c3ux30b3ux30dfux3068ux88abux544aux767aux4ebaux5ca1ux7530ux9032ux5f01ux8b77ux58ebux306eux56fdux9078ux5211ux4e8bux5f01ux8b77ux306eux5b9fux614b}}

\begin{itemize}
\tightlist
\item
  〉〉〉 Linux Emacs: 2021/06/05 09:11:12 〉〉〉
\end{itemize}

:CATEGORIES: @kanazawabengosi \#金沢弁護士会 @JFBAsns
日本弁護士連合会(日弁連) \#法務省 @MOJ\_HOUMU \#被告発人岡田進弁護士
\#国選弁護

\begin{itemize}
\tightlist
\item
  2021年06月04日23時51分の登録:
  \みどやー @X04Sgd56es5g6wJ\受任していた刑事事件の被疑者を「接見も公判も適当に済まして,サックと刑務所に行ってもらえ!」というオーダーを無視したら,事務
  \url{https://kk2020-09.blogspot.com/2021/06/x04sgd56es5g6wj.html} 
\item
  2021年06月05日00時03分の登録:
  ツイートの記録資料:\法務検察・石川県警察宛\/深澤諭史(@fukazawas)/''2021年06月04日'':2件
  \url{https://kk2020-09.blogspot.com/2021/06/fukazawas202106042.html} 
\item
  2021年06月05日00時03分の登録:
  ツイートの記録資料:\法務検察・石川県警察宛\/小倉秀夫(@chosakukenho)/''2021年06月04日'':41件
  \url{https://kk2020-09.blogspot.com/2021/06/chosakukenho2021060441.html} 
\item
  2021年06月05日00時03分の登録:
  ツイートの記録資料:\法務検察・石川県警察宛\/モトケン(@motoken\_tw)/''2021年06月04日'':4件
  \url{https://kk2020-09.blogspot.com/2021/06/motokentw202106044.html} 
\item
  2021年06月05日00時04分の登録:
  2021-06-04の投稿一覧\検察・石川県警察宛記録資料\奉納\危険生物・弁護士脳汚染除去装置\金沢地方検察庁御中:56件
  \url{https://kk2020-09.blogspot.com/2021/06/2021-06-0456.html} 
\item
  2021年06月05日05時55分の登録:
  REGEXP:''再審''/データベース登録済みツイートの検索:2021-06-03〜2021-06-04/2021年06月05日05時54分の記録:ユーザ・投稿:14/16件
  \url{https://kk2020-09.blogspot.com/2021/06/regexp2021-06-032021-06\_5.html} 
\item
  2021年06月05日06時03分の登録:
  REGEXP:''国選''/データベース登録済みツイートの検索:2021-05-31〜2021-06-05/2021年06月05日06時00分の記録:ユーザ・投稿:75/135件
  \url{https://kk2020-09.blogspot.com/2021/06/regexp2021-05-312021-06\_5.html} 
\item
  2021年06月05日06時10分の登録:
  REGEXP:''岡口''/データベース登録済みツイートの検索:2021-06-04〜2021-06-05/2021年06月05日06時08分の記録:ユーザ・投稿:36/223件
  \url{https://kk2020-09.blogspot.com/2021/06/regexp2021-06-042021-06\_5.html} 
\item
  2021年06月05日07時27分の登録:
  REGEXP:''岡口''/データベース登録済みツイートの検索:2021-06-04〜2021-06-05/2021年06月05日07時25分の記録:ユーザ・投稿:34/47件
  \url{https://kk2020-09.blogspot.com/2021/06/regexp2021-06-042021-06\_77.html} 
\item
  2021年06月05日07時34分の登録:
  REGEXP:''再審''/データベース登録済みツイートの検索:2021-06-03〜2021-06-04/2021年06月05日07時33分の記録:ユーザ・投稿:12/13件
  \url{https://kk2020-09.blogspot.com/2021/06/regexp2021-06-032021-06\_97.html} 
\item
  2021年06月05日07時50分の登録:
  Ex-REGEXP:''九条の大罪''/データベース登録済みツイートの検索:2021-06-02〜2021-06-05/2021年06月05日07時50分の記録:ユーザ・投稿:9/17件
  \url{https://kk2020-09.blogspot.com/2021/06/ex-regexp2021-06-022021-06.html} 
\item
  2021年06月05日08時45分の登録:
  \小倉秀夫 @chosakukenho\高裁で逆転しても不思議ではないです。被告のツイートに対するリプライとしてなされたわけでもないツイートをそのようなリプライとしてなさ
  \url{https://kk2020-09.blogspot.com/2021/06/chosakukenho\_5.html} 
\item
  2021年06月05日08時45分の登録:
  \高橋雄一郎 @kamatatylaw\控訴審の代理人をやりましょう。
  \url{https://kk2020-09.blogspot.com/2021/06/kamatatylaw\_5.html} 
\item
  2021年06月05日08時53分の登録:
  Ex-REGEXP:''国選''/データベース登録済みツイートの検索:2009-12-10〜2021-06-05/2021年06月05日08時30分の記録:ユーザ・投稿:606/5456件
  \url{https://kk2020-09.blogspot.com/2021/06/ex-regexp2009-12-102021-06.html} 
\end{itemize}

 0時過ぎまで起きていたことは憶えているのですが、いつの間にか眠っていて目が覚めたのが4時半頃、そのまま寝付けず起きてパソコンで作業を再開したのがだいたい5時頃ではなかったかと思います。

 昨日の6月4日は徹底して岡口基一裁判官に関連した弁護士らのツイートをデータベースに記録していたのですが、私のアカウントのツイートやリツイートが多すぎると実態が見えづらいと思い、私のアカウントを除外するコマンドをコピーで作成しました。

 上書き保存を忘れていたのか2回目か3回目の新しいコマンドの実行でうまく行ったのですが、それが「Ex-REGEXP:''九条の大罪''/データベース登録済みツイートの検索:2021-06-02〜2021-06-05/2021年06月05日07時50分の記録:ユーザ・投稿:9/17件」という記事になります。

 新規に作成したコマンド名はExclusion-kk2021-06\_ajx-all-user-mysql-REGEXP\_blogger.rbなのですが、除外の英語を調べて、タイトルの戦闘にEx-をつけ、「Ex-REGEXP:」とすることで見分けが付きやすいように工夫しました。

 記念すべき最初の新タイトルが、「Ex-REGEXP:''九条の大罪''/データベース登録済みツイートの検索:2021-06-02〜2021-06-05/2021年06月05日07時50分の記録:ユーザ・投稿:9/17件」なのですが、この記事に思わぬ発見がありました。見通しが良くなったおかげかもです。

\begin{itemize}
\item
  奉納\危険生物・弁護士脳汚染除去装置\金沢地方検察庁御中\_2020:
  Ex-REGEXP:''九条の大罪''/データベース登録済みツイートの検索:2021-06-02〜2021-06-05/2021年06月05日07時50分の記録:ユーザ・投稿:9/17件
  \url{https://kk2020-09.blogspot.com/2021/06/ex-regexp2021-06-022021-06.html} 
\item
  (08/17) RT @lawer\_hamachan(浜 ち ゃ ん)|asty\_md(asty\_md)
  日時:2021-06-02 16:20:22 +0900/2021-06-02 15:32:00 +0900 URL:
  \url{https://twitter.com/lawer\_hamachan/status/1399989251710717953} 
  \url{https://twitter.com/asty\_md/status/1399977308069384192\textgreater} {}
  ウシジマくん作者の新作弁護士漫画「九条の大罪」に法曹クラスタがマジレスツッコミ
  - Togetter \url{https://t.co/gwBOKgTbWa}  @togetter\_jpより
\end{itemize}

 これまでは単純に「九条の大罪」という検索しかできなかったのですが、これまで目に触れなかったツイートが結構な数まとまっていました。おなじみの刑裁サイ太やうの字のツイートも揃っていました。弁護士生態調査研究資料の豪華パックのようなものです。

〉〉〉 kk\_hironoのリツイート 〉〉〉

\begin{itemize}
\tightlist
\item
  RT
  kk\_hirono(刑事告発・非常上告_金沢地方検察庁御中)|mamoruhatago(まもるくん)
  日時:2021-06-05 09:32/2021/02/26 21:25 URL:
  \url{https://twitter.com/kk\_hirono/status/1400973731099410437} 
  \url{https://twitter.com/mamoruhatago/status/1365276695075332097} 
  \textgreater{} @luckymangan
  スマホの履歴はPならある程度調べられる。今どき携帯持ってないとなったら周りから固められて携帯番号からLINEの履歴とか結局バレる。
\end{itemize}

〉〉〉 kk\_hironoのリツイート 〉〉〉

\begin{itemize}
\tightlist
\item
  RT
  kk\_hirono(刑事告発・非常上告_金沢地方検察庁御中)|o2441(スラ弁(弁護士大西洋一))
  日時:2021-06-05 09:32/2021/02/26 18:09 URL:
  \url{https://twitter.com/kk\_hirono/status/1400973833822085122} 
  \url{https://twitter.com/o2441/status/1365227526662942721} 
  \textgreater{}
  過小示談だと保険会社は自賠責から回収させてもらえないから、これはあり得ない。つまらぬ揚げ足取りに感じるかもしれないけど、せっかく面白い作品なのだから、こういうところはきちんと法律監修入れてほしい。
  \url{https://t.co/vSxUEWMYE7} 
\end{itemize}

〉〉〉 kk\_hironoのリツイート 〉〉〉

\begin{itemize}
\tightlist
\item
  RT
  kk\_hirono(刑事告発・非常上告_金沢地方検察庁御中)|nabeteru1Q78(渡辺輝人)
  日時:2021-06-05 09:34/2021/02/26 18:33 URL:
  \url{https://twitter.com/kk\_hirono/status/1400974243785969670} 
  \url{https://twitter.com/nabeteru1Q78/status/1365233474609709056} 
  \textgreater{}
  自賠責保険で後遺症だけで1889万円取れる事案で任意保険で1000万円示談は、確かになさそう。保険会社が示談に応じるということは、事前認定で4級が出たということだろう。絵から見ても分かる程度の4級事案だし。
\end{itemize}

〉〉〉 kk\_hironoのリツイート 〉〉〉

\begin{itemize}
\tightlist
\item
  RT
  kk\_hirono(刑事告発・非常上告_金沢地方検察庁御中)|takei\_ben(竹井)
  日時:2021-06-05 09:35/2021/02/26 22:51 URL:
  \url{https://twitter.com/kk\_hirono/status/1400974433368428545} 
  \url{https://twitter.com/takei\_ben/status/1365298465308221442} 
  \textgreater{} 法クラ界隈のマンガ評を見ると、
  九条の大罪は「過小示談だからリアルと違う」と言われ、
  リーガルエッグは「リアルっぽすぎてつまんないから打ち切りになるんだ」と言われ、
  ワイは「どっちやねん!」と思った。
\end{itemize}

〉〉〉 kk\_hironoのリツイート 〉〉〉

\begin{itemize}
\tightlist
\item
  RT
  kk\_hirono(刑事告発・非常上告_金沢地方検察庁御中)|jikapan(自家製パンチェッタ)
  日時:2021-06-05 09:35/2021/02/27 08:47 URL:
  \url{https://twitter.com/kk\_hirono/status/1400974470899146752} 
  \url{https://twitter.com/jikapan/status/1365448487194689536} 
  \textgreater{} \textgreater RT 半端にリアルだから突っ込まれるんだな。
  たしかに、逆転裁判に「無罪推定というか検察官(国)の立証の責任どうなってるんか」「木槌はない」「裁判官ではなく裁判所」とか突っ込むやつは・・・・・・いっぱいおったわ・・・・・・。
\end{itemize}

〉〉〉 kk\_hironoのリツイート 〉〉〉

\begin{itemize}
\tightlist
\item
  RT
  kk\_hirono(刑事告発・非常上告_金沢地方検察庁御中)|ShinHori1(Shin
  Hori) 日時:2021-06-05 09:35/2021/02/27 15:19 URL:
  \url{https://twitter.com/kk\_hirono/status/1400974550825771008} 
  \url{https://twitter.com/ShinHori1/status/1365547069637726211} 
  \textgreater{}
  自賠責保険だけでも後遺障害4等級に対して1889万円が出ます。ここでいう「1000万円」の前提や根拠はよくわかりませんが・・・
  \url{https://t.co/uVuCwia6ku} 
\end{itemize}

〉〉〉 kk\_hironoのリツイート 〉〉〉

\begin{itemize}
\tightlist
\item
  RT
  kk\_hirono(刑事告発・非常上告_金沢地方検察庁御中)|ShinHori1(Shin
  Hori) 日時:2021-06-05 09:35/2021/02/27 15:22 URL:
  \url{https://twitter.com/kk\_hirono/status/1400974584719953921} 
  \url{https://twitter.com/ShinHori1/status/1365547933454594055} 
  \textgreater{} 証拠隠滅で懲戒されるのでは \url{https://t.co/Z6QbIma0m6} 
\end{itemize}

〉〉〉 kk\_hironoのリツイート 〉〉〉

\begin{itemize}
\tightlist
\item
  RT
  kk\_hirono(刑事告発・非常上告_金沢地方検察庁御中)|ShinHori1(Shin
  Hori) 日時:2021-06-05 09:35/2021/02/27 15:57 URL:
  \url{https://twitter.com/kk\_hirono/status/1400974629213130752} 
  \url{https://twitter.com/ShinHori1/status/1365556608764235776} 
  \textgreater{} →
  被疑者が飲み屋で飲んでから運転したのなら、警察は防犯カメラなどを調べて運転開始地点を把握し、その付近の飲み屋を特定する可能性が高い。
  さらに被疑者が取り調べで「弁護人の先生に『スマホを預けろ、領収書は捨てろ、酒は抜け』と指示されました」なんて言ってしまったら、九条先生は終わり
\end{itemize}

〉〉〉 kk\_hironoのリツイート 〉〉〉

\begin{itemize}
\tightlist
\item
  RT
  kk\_hirono(刑事告発・非常上告_金沢地方検察庁御中)|ShinHori1(Shin
  Hori) 日時:2021-06-05 09:36/2021/02/27 16:14 URL:
  \url{https://twitter.com/kk\_hirono/status/1400974667746201608} 
  \url{https://twitter.com/ShinHori1/status/1365560885964496899} 
  \textgreater{} →
  それ以前に警察が携帯電話会社を通じてGPSでスマホの位置情報を調べて、九条先生の事務所のあるビルの中にあることが判明したら・・・
  \url{https://t.co/DoMbJUx77d} 
\end{itemize}

〉〉〉 kk\_hironoのリツイート 〉〉〉

\begin{itemize}
\tightlist
\item
  RT
  kk\_hirono(刑事告発・非常上告_金沢地方検察庁御中)|ShinHori1(Shin
  Hori) 日時:2021-06-05 09:36/2021/02/27 16:33 URL:
  \url{https://twitter.com/kk\_hirono/status/1400974727301136385} 
  \url{https://twitter.com/ShinHori1/status/1365565599405936643} 
  \textgreater{} →
  あと、警察に出頭する前に酒が抜けるのをずっと待っていたら、その間に警察が運転者を把握してしまって、もはや(「発覚後の出頭」になるので)「自首」が成立しなくなってしまう可能性もあるのではないか。\url{https://t.co/AicS2w5njL} 
\end{itemize}

〉〉〉 kk\_hironoのリツイート 〉〉〉

\begin{itemize}
\item
  RT
  kk\_hirono(刑事告発・非常上告_金沢地方検察庁御中)|flannellechat(ネル)
  日時:2021-06-05 09:36/2021/02/27 15:27 URL:
  \url{https://twitter.com/kk\_hirono/status/1400974774780645380} 
  \url{https://twitter.com/flannellechat/status/1365549031653449734} 
  \textgreater{} RT
  これこれ、私は絶対にやらん。懲戒されたら割に合わんもん。他人のためにそんなリスク取りませんよ。
\item
  〉〉〉 アカウント(@luckymangan)は,@kk\_hironoをブロックしています。リツイートできませんでした。
  〉〉〉 ¥\n ¥\n \url{https://t.co/iAGAks5WIX} 
\item
  〉〉〉 アカウント(@luckymangan)は,@kk\_hironoをブロックしています。リツイートできませんでした。
  〉〉〉 ¥\n ¥\n \url{https://t.co/1GZgfMY1l2} 
\item
  〉〉〉 アカウント(@luckymangan)は,@kk\_hironoをブロックしています。リツイートできませんでした。
  〉〉〉 ¥\n ¥\n \url{https://t.co/FrLq9RdVnO} 
\item
  〉〉〉 アカウント(@luckymangan)は,@kk\_hironoをブロックしています。リツイートできませんでした。
  〉〉〉 ¥\n ¥\n \url{https://t.co/9IpbXJWBc3} 
\item
  〉〉〉 アカウント(@un\_co\_the2nd)は,@kk\_hironoをブロックしています。リツイートできませんでした。
  〉〉〉 ¥\n ¥\n \url{https://t.co/lsTEdRzuc1} 
\item
  〉〉〉 アカウント(@uwaaaa)は,@kk\_hironoをブロックしています。リツイートできませんでした。
  〉〉〉 ¥\n ¥\n \url{https://t.co/6Hx6mv9GyB} 
\item
  〉〉〉 アカウント(@uwaaaa)は,@kk\_hironoをブロックしています。リツイートできませんでした。
  〉〉〉 ¥\n ¥\n \url{https://t.co/gPHLMZF5kr} 
\item
  〉〉〉 アカウント(@uwaaaa)は,@kk\_hironoをブロックしています。リツイートできませんでした。
  〉〉〉 ¥\n ¥\n \url{https://t.co/oTlvqTpNkf} 
\item
  〉〉〉 アカウント(@stdaux)は,@kk\_hironoをブロックしています。リツイートできませんでした。
  〉〉〉 ¥\n ¥\n \url{https://t.co/fMf94ZioPS} 
\item
  〉〉〉 アカウント(@stdaux)は,@kk\_hironoをブロックしています。リツイートできませんでした。
  〉〉〉 ¥\n ¥\n \url{https://t.co/LNNNuV1Smb} 
\item
  〉〉〉 アカウント(@stdaux)は,@kk\_hironoをブロックしています。リツイートできませんでした。
  〉〉〉 ¥\n ¥\n \url{https://t.co/eQM0lSFHvX} 
\item
  〉〉〉 アカウント(@uwaaaa)は,@kk\_hironoをブロックしています。リツイートできませんでした。
  〉〉〉 ¥\n ¥\n \url{https://t.co/fpQuIw1Rm5} 
\item
  〉〉〉 アカウント(@uwaaaa)は,@kk\_hironoをブロックしています。リツイートできませんでした。
  〉〉〉 ¥\n ¥\n \url{https://t.co/Qo4jwegRtr} 
\item
  〉〉〉 アカウント(@uwaaaa)は,@kk\_hironoをブロックしています。リツイートできませんでした。
  〉〉〉 ¥\n ¥\n \url{https://t.co/mSJQGZWQzS} 
\item
  〉〉〉 アカウント(@un\_co\_the2nd)は,@kk\_hironoをブロックしています。リツイートできませんでした。
  〉〉〉 ¥\n ¥\n \url{https://t.co/XCVQOmggUe} 
\end{itemize}

 26件のツイートのリツイートでしたが、ブロックされてリツイートできなかったアカウントのツイートは次になります。

※ @kk\_hironoのアカウントがブロックされ,リツイートに失敗したツイート

\begin{itemize}
\tightlist
\item
  TW luckymangan(リーチ一発ツモ裏1) 日時:2020/10/13 22:07:30 URL:
  \url{https://twitter.com/luckymangan/status/1316002623820972034} 
  \textgreater{} \#九条の大罪\\
  \textgreater{} 法的なツッコミ所。法クラの皆さん、修正、追加宜しく\\
  \textgreater{}
  ・轢逃げ考慮されてない。だから実刑事案なのに執行猶予になってる\\
  \textgreater{}
  ・九条の証拠隠滅示唆は警察が裏付け取れば判明するので通常やらない\\
  \textgreater{} ・脅迫の件は被害者証言という証拠で起訴され得る
\end{itemize}

※ @kk\_hironoのアカウントがブロックされ,リツイートに失敗したツイート

\begin{itemize}
\tightlist
\item
  TW luckymangan(リーチ一発ツモ裏1) 日時:2020/10/13 22:08:51 URL:
  \url{https://twitter.com/luckymangan/status/1316002963630874624} 
  \textgreater{} \#九条の大罪\\
  \textgreater{} ・日本一の件は景表法あるので烏丸の方が法的には正しい\\
  \textgreater{} ・出頭翌日=起訴前で弁護人が被害者側の資料集められない\\
  \textgreater{} ・保険金は死亡した父親の方はどうなった?\\
  \textgreater{}
  ・子供の方の保険金少ないと、被告人負担になって加害者側に民事も刑事も不利では?\\
  \textgreater{} ・10\%は報酬でなく不法行為の弁護士費用
\end{itemize}

※ @kk\_hironoのアカウントがブロックされ,リツイートに失敗したツイート

\begin{itemize}
\tightlist
\item
  TW luckymangan(リーチ一発ツモ裏1) 日時:2021/02/26 21:31:28 URL:
  \url{https://twitter.com/luckymangan/status/1365278304987541507} 
  \textgreater{} @kumanomi\_hatago
  携帯電話の隠滅の件はフィクションらしい「嘘」を敢えて言ったかなと思ってます。悪用されたら困りますしね。
\end{itemize}

※ @kk\_hironoのアカウントがブロックされ,リツイートに失敗したツイート

\begin{itemize}
\tightlist
\item
  TW luckymangan(リーチ一発ツモ裏1) 日時:2021/02/26 18:14:43 URL:
  \url{https://twitter.com/luckymangan/status/1365228791686070272} 
  \textgreater{} @o2441
  民事上は仰るとおりで、刑事上も過小示談だと、被害回復と同等といえず、執行猶予つかない可能性もあがりませんかね
\end{itemize}

※ @kk\_hironoのアカウントがブロックされ,リツイートに失敗したツイート

\begin{itemize}
\tightlist
\item
  TW un\_co\_the2nd(うの字を名乗る💩物) 日時:2021/02/26 18:21:44 URL:
  \url{https://twitter.com/un\_co\_the2nd/status/1365230559987818497} 
  \textgreater{}
  過小示談だと,刑事でも被害弁償なしの評価になって実刑食らいかねないので,これでムショ行きにならないとか言っちゃうのはファンタジーですね
  \url{https://t.co/kjzPZpleOK} 
\end{itemize}

※ @kk\_hironoのアカウントがブロックされ,リツイートに失敗したツイート

\begin{itemize}
\tightlist
\item
  TW uwaaaa(サイ太) 日時:2021/02/26 18:21:57 URL:
  \url{https://twitter.com/uwaaaa/status/1365230612076892162} 
  \textgreater{}
  まずツッコミたいのは,「クズが交通事故を起こしても無罪になるのはこんな理由」というTwitter受け狙いのためのタイトル?をつけているけど,別に無罪になっていないというところですね
  \url{https://t.co/Sw6fjAHHm5} 
\end{itemize}

※ @kk\_hironoのアカウントがブロックされ,リツイートに失敗したツイート

\begin{itemize}
\tightlist
\item
  TW uwaaaa(サイ太) 日時:2021/02/26 18:27:37 URL:
  \url{https://twitter.com/uwaaaa/status/1365232040778457090} 
  \textgreater{}
  判決後にイソ弁に「ネット民が荒れてますね」みたいな台詞を言わせてるのは真鍋先生流の自虐なのか何なのか
\end{itemize}

※ @kk\_hironoのアカウントがブロックされ,リツイートに失敗したツイート

\begin{itemize}
\tightlist
\item
  TW uwaaaa(サイ太) 日時:2021/02/26 18:32:35 URL:
  \url{https://twitter.com/uwaaaa/status/1365233289473105923} 
  \textgreater{}
  色々と23条照会で証拠を入手しているっぽいけど,遺族の姿からは事故直後のような描写。こんなに詳細な情報はこんなに早く入手できないのでは。回想シーンならギリギリ成立か。
  \url{https://t.co/MdRK3NtiET} 
\end{itemize}

※ @kk\_hironoのアカウントがブロックされ,リツイートに失敗したツイート

\begin{itemize}
\tightlist
\item
  TW stdaux(スドー🌸) 日時:2021/02/26 18:38:11 URL:
  \url{https://twitter.com/stdaux/status/1365234699090624517} 
  \textgreater{}
  悪徳弁護士もの、なぜその人は悪徳になったのかという描写にあまり説得力ないことが多い気がする(純粋に金儲けが目当てならもっとコスパ良い職業があるので)
\end{itemize}

※ @kk\_hironoのアカウントがブロックされ,リツイートに失敗したツイート

\begin{itemize}
\tightlist
\item
  TW stdaux(スドー🌸) 日時:2021/02/26 18:40:43 URL:
  \url{https://twitter.com/stdaux/status/1365235333483298818} 
  \textgreater{}
  なんだろうな。修習後にブラック事務所に入ってしまいろくにスキルを得られないまま即独、営業に焦って非弁業者とかに引っ掛かって懲戒を食らい、あとは筋悪の事件ばっかり受けざるを得ないようになってモラルが擦り切れたとか?
\end{itemize}

※ @kk\_hironoのアカウントがブロックされ,リツイートに失敗したツイート

\begin{itemize}
\tightlist
\item
  TW stdaux(スドー🌸) 日時:2021/02/26 19:09:48 URL:
  \url{https://twitter.com/stdaux/status/1365242654112419846} 
  \textgreater{}
  現実に寄せず、誰がどう見てもフィクションな世界なら弁護士を出しても法クラは突っ込まないので、九条先生も目からビームとか出すと良い
\end{itemize}

※ @kk\_hironoのアカウントがブロックされ,リツイートに失敗したツイート

\begin{itemize}
\tightlist
\item
  TW uwaaaa(サイ太) 日時:2021/02/26 18:38:24 URL:
  \url{https://twitter.com/uwaaaa/status/1365234750852501523} 
  \textgreater{}
  検察官は危険運転致死傷を争う事件(裁判員裁判なのでおそらく)なのに被害者参加させてないんですかね。おそらく国選付けられるでしょ。その後の示談の下りも含めてちょっと疑問。
  \url{https://t.co/LOuvXHvMaJ} 
\end{itemize}

※ @kk\_hironoのアカウントがブロックされ,リツイートに失敗したツイート

\begin{itemize}
\tightlist
\item
  TW uwaaaa(サイ太) 日時:2021/02/26 18:41:00 URL:
  \url{https://twitter.com/uwaaaa/status/1365235406929727489} 
  \textgreater{}
  被害者側の損害賠償の報酬と比べるべきは加害者側の民事の報酬では。保険会社側の代理人なら300万円もらえれば破格も破格(ですよね各位)
  \url{https://t.co/J2dLv6WFY2} 
\end{itemize}

※ @kk\_hironoのアカウントがブロックされ,リツイートに失敗したツイート

\begin{itemize}
\tightlist
\item
  TW uwaaaa(サイ太) 日時:2021/02/26 18:46:42 URL:
  \url{https://twitter.com/uwaaaa/status/1365236841725583360} 
  \textgreater{} 良かった点\\
  \textgreater{} ・裏側の金具をつけた弁護士バッジの描写がよかった\\
  \textgreater{} ・事務所の描写がリアルだった\\
  \textgreater{} ・裁判所前の看板・・・
\end{itemize}

※ @kk\_hironoのアカウントがブロックされ,リツイートに失敗したツイート

\begin{itemize}
\tightlist
\item
  TW un\_co\_the2nd(うの字を名乗る💩物) 日時:2021/02/27 09:31:17 URL:
  \url{https://twitter.com/un\_co\_the2nd/status/1365459454569766912} 
  \textgreater{}
  九条は、Twitterでわざわざ「無罪になる」と、作中展開にもないのに過激な文言つけたのが一番アカンかったと思う。
  \url{https://t.co/9BLph1AT0b} 
\end{itemize}

 以上が「法曹クラスタのツッコミ」という項目で紹介されているツイートになります。

\begin{itemize}
\tightlist
\item
  〈〈〈 2021/06/05 09:43:07 Linux Emacs: 〈〈〈
\end{itemize}

\hypertarget{ux30a6ux30b7ux30b8ux30deux304fux3093ux4f5cux8005ux306eux65b0ux4f5cux5f01ux8b77ux58ebux6f2bux753bux4e5dux6761ux306eux5927ux7f6aux306bux6cd5ux66f9ux30afux30e9ux30b9ux30bfux304cux30deux30b8ux30ecux30b9ux30c4ux30c3ux30b3ux30dfux3068ux88abux544aux767aux4ebaux5ca1ux7530ux9032ux5f01ux8b77ux58ebux306eux56fdux9078ux5211ux4e8bux5f01ux8b77ux306eux5b9fux614b-2}{%
\paragraph{ウシジマくん作者の新作弁護士漫画「九条の大罪」に法曹クラスタがマジレスツッコミ、と被告発人岡田進弁護士の国選刑事弁護の実態
(2)}\label{ux30a6ux30b7ux30b8ux30deux304fux3093ux4f5cux8005ux306eux65b0ux4f5cux5f01ux8b77ux58ebux6f2bux753bux4e5dux6761ux306eux5927ux7f6aux306bux6cd5ux66f9ux30afux30e9ux30b9ux30bfux304cux30deux30b8ux30ecux30b9ux30c4ux30c3ux30b3ux30dfux3068ux88abux544aux767aux4ebaux5ca1ux7530ux9032ux5f01ux8b77ux58ebux306eux56fdux9078ux5211ux4e8bux5f01ux8b77ux306eux5b9fux614b-2}}

\begin{itemize}
\tightlist
\item
  〉〉〉 Linux Emacs: 2021/06/05 09:45:01 〉〉〉
\end{itemize}

:CATEGORIES: @kanazawabengosi \#金沢弁護士会 @JFBAsns
日本弁護士連合会(日弁連) \#法務省 @MOJ\_HOUMU \#被告発人岡田進弁護士
\#国選弁護

\begin{itemize}
\item
  1395:2021-06-05\_09:43:38 \#告発状 \#\#\#\#
  ウシジマくん作者の新作弁護士漫画「九条の大罪」に法曹クラスタがマジレスツッコミ、と被告発人岡田進弁護士の国選刑事弁護の実態
  \url{https://hirono-hideki.hatenadiary.jp/entry/2021/06/05/094334} 
\item
  ウシジマくん作者の新作弁護士漫画「九条の大罪」に法曹クラスタがマジレスツッコミ
  - Togetter \url{https://togetter.com/li/1674668} 
\end{itemize}

 エントリー1395では、「法曹クラスタのツッコミ」とあるツイートをリツイートでご紹介しましたが、その次にある「みんなの感想」というコーナーのツイートについては、その内容のみを引用でご紹介したいと思います。箇条書きにし''\,''で括ります。

\begin{itemize}
\item
  ''法律家にボコボコにされててわろた。ウシジマくん世界観で法律が機能するわけがない'' 2021-02-26
  23:27:43
\item
  ''これは私も読んでて「真鍋ヤベェな」ってなった。被疑者がぽろっと言っちゃったらどうすんの?
  '' 2021-02-27 16:05:17
\item
  ''フィクションだから成立するけど、現実世界でやると墓穴を掘る手法をあえて作中で紹介して、漫画の真似をして罪を逃れようとした間抜けを痛い目に合わせる手法なのかもしれない。
  twitter.com/ShinHori1/stat・・・'' 2021-02-27 16:08:06
\item
  ''一連のツイートはなるほどなあと思った。弁護士が手ぬかりなかったとしても、依頼人の行動は割と早々に警察に割れる可能性があるから、こうは上手くいかないんじゃないかって話で、まあたしかにそうかもなと。'' 2021-02-27
  16:18:16
\end{itemize}

〉〉〉 kk\_hironoのリツイート 〉〉〉

\begin{itemize}
\tightlist
\item
  RT
  kk\_hirono(刑事告発・非常上告_金沢地方検察庁御中)|Dynamite\_Tommy(ねこぱんち)
  日時:2021-06-05 10:01/2021/02/27 16:22 URL:
  \url{https://twitter.com/kk\_hirono/status/1400981026952732674} 
  \url{https://twitter.com/Dynamite\_Tommy/status/1365562919505694723} 
  \textgreater{}
  法律家の監修をつけずに法律関係の作品を世に出すのは、法律に強いからではなくバカだからだと思うぞ(・ω・)。
\end{itemize}

 例外としてリツイートをした上記のツイートのアカウントは、モトケンこと矢部善朗弁護士(京都弁護士会)のタイムラインでもモトケンこと矢部善朗弁護士(京都弁護士会)とのやりとりとしてよく見かけてきたアカウントです。最近はちょっとみていないかもしれません。

\begin{itemize}
\item
  ''ウシジマくん作者の真鍋さんの漫画での不備を、法律の専門家がツッコんでる。こういうとこのチェックは大手出版社ならやるんだろうけど、作り手としてはないようにケチつけられてる気分なので、ノーチェックのヤツを世間に見せたくもなる、みたいな話なんだろうなぁ。'' 2021-02-27
  16:56:44
\item
  ''確かに、ここに一連の指摘が適当だとは思うが IT犯罪に対する明後日の警察対応を見ていると 漫画の信憑性は余りゆるがない感が有る。 多分、東京だと指摘のとおりになるが、 地方の県警だとどうかわからない感じはする。'' 2021-02-27
  17:04:39
\item
  ''バレたら問題になるけどバレないから問題にならないっていう胸糞漫画なんじゃ・・・・・・と思ったけど、作中世界でもかなり社会的な注目度の高い事件のようだし、警察が本気出してバレる程度の隠蔽だと辻褄が合わなくなるのか。
  '' 2021-02-27 18:49:59
\item
  ''フィクションだし・・・ってなるけど最近はマンガとかの知識でドヤる人が多いからツッコミはしないとね'' 2021-02-27
  20:26:59
\end{itemize}

 一通り読み終えてからTogetterのまとめを作成したアカウントを確認したのですが、プロフィールが次のようになっていました。そのまま昭和14年に66歳で亡くなった小説家・劇作家の名前とプロフィールのようです。

\begin{quote}
《引用の始まり》
\end{quote}

\begin{quote}
劇作家、小説家。本名は敬二、別号に狂綺堂。イギリス公使館に勤めていた元徳川家御家人、敬之助の長男として、東京高輪に生まれる。コナン・ドイルのシャーロック・ホームズ物を原著でまとめて読んだのをきっかけに、江戸を舞台とした探偵小説の構想を得、1916(大正5)年からは「半七捕物帳」を書き始めた。
\end{quote}

\begin{quote}
《引用の終わり》
\end{quote}

\begin{itemize}
\tightlist
\item
  岡本綺堂(@KidouOkamoto)のまとめ(268) - Togetter
  \url{https://togetter.com/id/KidouOkamoto} 
\end{itemize}

 ビューというのか閲覧数になると思いますが、25812件となっています。17とあるのはコメント数ではないかと思います。

 Togetterを見るのも久しぶりですが、見かける機会も少なくなりました。一時期よく使っていたのですが、それも8年ぐらいは前になりそうです。まだモトケンこと矢部善朗弁護士(京都弁護士会)にブロックされていない時期もあったかもしれません。

 最近は「まとめサイト」という言葉自体みかけていないのですが、一時期、弁護士らが批判の矛先を向け、深澤諭史弁護士も批判的というか攻撃対象にしているという印象を受けました。そのまとめサイトでも最初に見たと記憶にあるのがこのTogetterになります。

\begin{lstlisting}
base ❯ d|grep まとめサイト
\end{lstlisting}

\begin{itemize}
\tightlist
\item
  2017年11月17日12時05分の登録:
  \深澤諭史 @fukazawas\まとめサイトの法的責任について話題ですが、かなり前に、そういう行為の責任を認める前提となる判決が出ています。
  \url{http://hirono2014sk.blogspot.com/2017/11/fukazawas\_70.html} 
\item
  2018年01月21日06時00分の登録:
  \ystk @lawkus\職場に乗り込むなどの行為は、2ちゃんまとめサイトの創作実話のせいか、マジでやろうとする人が散見されるのは確か。相談されれば「あべこべに損害賠償
  \url{http://hirono2014sk.blogspot.com/2018/01/ystklawkus2\_21.html} 
\item
  2020年01月04日18時45分の登録:
  \ふて寝べん @hirune\_b\まとめサイトの記事で、ゴーン氏が海外で告訴されていることについて、日本にいたほうが良かったのにwwざまあwwというのが流れて来たけど、正
  \url{http://hirono2014sk.blogspot.com/2020/01/hirunebwwww.html} 
\item
  2020年03月13日08時11分の登録:
  \深澤諭史 @fukazawas\たしかにトレンドブログ・まとめサイトで違法な投稿は散見されますがそれは,これらメディアに限らないわけで。¥\n中には,表現の自由の関係で問題
  \url{http://hirono2014sk.blogspot.com/2020/03/fukazawas\_13.html} 
\item
  2021年04月21日20時10分の登録:
  \サイ太 @uwaaaa\いわゆる京アニ放火事件に関連して,いわゆるまとめサイトの運営者に対して,NHKやその職員が事件に関与したことを窺わせるツイート・ブログ記事を投稿
  \url{https://kk2020-09.blogspot.com/2021/04/uwaaaanhk.html} 
\end{itemize}

 思ったより少ない数でした。「まとめサイト」という検索結果のまとめ記事も作成していなかったようです。

 112件と全体のツイートの数が少ないのにユーザの数が多く、結果を見ると55件となっていました。リツイートが多い傾向がありそうです。

\begin{itemize}
\item
  2021年06月05日10時33分の登録:
  Ex-REGEXP:''まとめサイト''/データベース登録済みツイートの検索:2011-03-16〜2021-04-21/2021年06月05日10時31分の記録:ユーザ・投稿:55/112件
  \url{https://kk2020-09.blogspot.com/2021/06/ex-regexp2011-03-162021-04.html} 
\item
  (003/112) TW @a\_anzai(安西敦 Atsushi Anzai) 日時: 2014-07-08
  19:15:00 +0900 URL:
  \url{https://twitter.com/a\_anzai/status/486453476600184833\textgreater} {}
  実名報道の弊害って極めて大きいと思う。いったん逮捕の報道が出てしまったらネットに永久に残るし,まとめサイトとかでさらされるし。不起訴がちっちゃく報道されたって,逮捕時のネット上の情報は消してもらえないし。
\item
  (011/112) TW @uwaaaa(サイ太) 日時: 2016-12-05 12:23:00 +0900
  URL:
  \url{https://twitter.com/uwaaaa/status/805613422582714368\textgreater} {}
  まとめサイト、記事削除広がる リクルートなど  :日本経済新聞
  \url{https://t.co/EUVus3GYy2} 
\item
  (014/112) TW @motoken\_tw(モトケン) 日時: 2016-12-06 09:08:00
  +0900 URL:
  \url{https://twitter.com/motoken\_tw/status/805926838929235968\textgreater} {}
  まとめサイト閉鎖、大手に飛び火 背景に収益優先の構図:朝日新聞デジタル
  \url{https://t.co/H6PvM09y26} 
\item
  (019/112) TW @mackckckck(弁護士ワック) 日時: 2017-02-26 15:11:00
  +0900 URL:
  \url{https://twitter.com/mackckckck/status/835733901293170689\textgreater} {}
  誰か安倍晋三記念小学校の事実関係・時系列について、まとめサイト作ってくれないかなあ(他力本願)。
\item
  (076/112) RT @uwaaaa(サイ太)|Abema\_Prime(AbemaPrime【公式】)
  日時:2018-12-14 12:44:00 +0900/2018-12-13 21:34:00 +0900 URL:
  \url{https://twitter.com/uwaaaa/status/1073423434212753408} 
  \url{https://twitter.com/Abema\_Prime/status/1073194560229072896\textgreater} {}
  深澤弁護士\textgreater{}
  「他のまとめサイトにも損害賠償の可能性がある」\textgreater\textgreater{}
  ''転載にも表現責任''\textgreater{}
  まとめサイトに「冬の時代」到来‥??\textgreater{} #アベプラ
  で詳しく解説中!
\item
  (081/112) TW @lawkus(ystk) 日時: 2019-02-21 06:54:00 +0900 URL:
  \url{https://twitter.com/lawkus/status/1098340239779622912\textgreater} {}
  医師の無罪の件。せん妄だとすれば患者の主観においては被害は現実のもので、被害申告自体は責められないところがあるよね。捜査機関が逮捕勾留せず、起訴しなければ済んだ話。「女さん
  無罪」とかでツイート検索すると女叩きのまとめサイトを引用したツイートが沢山ヒットするけど、恥を知れとしか。
\end{itemize}

 上記の三浦義隆弁護士のツイートはリツイートが多いようです。次の自分自身のツイートへのリプライとなっていて、そこに見覚えのあるジャーナリストの江川紹子氏の記事が紹介されています。

\begin{itemize}
\tightlist
\item
  TW lawkus(ystk) 日時: 2019/02/20 17:18:55 URL:
  \url{https://twitter.com/lawkus/status/1098134893295370240} 
  \textgreater{} 【速報】わいせつ罪に問われた外科医に無罪判決(江川紹子)
  - Y!ニュース \url{https://t.co/z1T5lAoCME} 
  \textgreater{} おおっ。
\end{itemize}

 日にちに少しずれがある可能性はありますが、2019年2月20日頃にでた一審の無罪判決だったようです。控訴審では逆転の有罪判決で、それも実刑だったように思いますが、上告の話題も見かけず、判決が確定したという話も見かけてはいません。

\begin{itemize}
\item
  (094/112) TW @motoken\_tw(モトケン) 日時: 2019-06-10 10:39:00
  +0900 URL:
  \url{https://twitter.com/motoken\_tw/status/1137897015189237760\textgreater} {}
  この弁護団に対して、「安全ピンで刺すことを推奨している。」という批判がありますが、すでにまとめサイトのコメント欄で私の見解を述べておりますのでそれを再掲します。なお、これは私の個人的見解ですが、パチパチ先生の「今は、そんなトラブルなど起こらないように祈っとる」と同じ気持ちです(続
\item
  (095/112) TW @motoken\_tw(モトケン) 日時: 2019-06-11 08:09:00
  +0900 URL:
  \url{https://twitter.com/motoken\_tw/status/1138221648476221440\textgreater} {}
  @rareboiled
  まとめサイトだけじゃなくて、私の安全ピン関係のツイートを読んでから批判してほしいな。
\end{itemize}

 Togetterを見かけず、ページ内検索すると該当が4つだったのですが、2つしか移動せず、togetterという小文字でヒットしているので二重に評価されているのかもしれません。削除されたというかアカウントが消滅した「こたんせ」のツイートでした。今は名前も見かけず。

 再掲になると思いますが、今回このエントリーでメインに取り上げたのは次のTogetterの記事になります。

\begin{itemize}
\tightlist
\item
  ウシジマくん作者の新作弁護士漫画「九条の大罪」に法曹クラスタがマジレスツッコミ
  (2ページ目) - Togetter \url{https://togetter.com/li/1674668?page=2} 
\end{itemize}

 いろいろと考えさせられる刑裁サイ太、うの字、スドーこと平野敬弁護士のツイートがあったのですが、思い出したのは傷害・準強姦被告事件の一審での公判になります。公判調書の新たな資料の発見で公開を行っていますが、予め取り上げておく必要を感じる機会となりました。

 予定では我孫子警察署の続きだったのですが、後回しにします。

\begin{itemize}
\tightlist
\item
  〈〈〈 2021/06/05 11:01:53 Linux Emacs: 〈〈〈
\end{itemize}

\hypertarget{ux88abux544aux767aux4ebaux5ca1ux7530ux9032ux5f01ux8b77ux58ebux3068ux7570ux8ceaux306aux5171ux901aux9805ux3092ux611fux3058ux308bux5f01ux8b77ux58ebux9244ux9053ux306eux6b74ux53f2ux3092ux5f69ux308bux6df1ux6fa4ux8aedux53f2ux5f01ux8b77ux58ebux306bux99b3ux305bux305fux60f3ux3044ux304bux3089ux5800ux5185ux5b5dux96c4ux306eux611bux3057ux304dux65e5ux3005ux3068ux604bux6587ux3068ux3044ux3046ux66f2ux306eux767aux898b}{%
\paragraph{被告発人岡田進弁護士と異質な共通項を感じる、弁護士鉄道の歴史を彩る深澤諭史弁護士に馳せた想いから、堀内孝雄の「愛しき日々」と、「恋文」という曲の発見}\label{ux88abux544aux767aux4ebaux5ca1ux7530ux9032ux5f01ux8b77ux58ebux3068ux7570ux8ceaux306aux5171ux901aux9805ux3092ux611fux3058ux308bux5f01ux8b77ux58ebux9244ux9053ux306eux6b74ux53f2ux3092ux5f69ux308bux6df1ux6fa4ux8aedux53f2ux5f01ux8b77ux58ebux306bux99b3ux305bux305fux60f3ux3044ux304bux3089ux5800ux5185ux5b5dux96c4ux306eux611bux3057ux304dux65e5ux3005ux3068ux604bux6587ux3068ux3044ux3046ux66f2ux306eux767aux898b}}

\begin{itemize}
\tightlist
\item
  〉〉〉 Linux Emacs: 2021/06/05 22:12:09 〉〉〉
\end{itemize}

:CATEGORIES: @kanazawabengosi \#金沢弁護士会 @JFBAsns
日本弁護士連合会(日弁連) \#法務省 @MOJ\_HOUMU \#深澤諭史弁護士
\#被告発人岡田進弁護士

 ここ3時間ほどの間のことだと思うのですが、最初のきっかけがよく思い出せないのですが、深澤諭史弁護士のツイートを新たな視点で記録し、その先に意外な発見がありました。最近は名前も見かけていなかった堀内孝雄という歌手ですが、宇出津の能登町役場で公演を視聴したことがありました。

\begin{itemize}
\tightlist
\item
  堀内孝雄 宇出津 - Google 検索 \url{https://t.co/WWYymSU1dJ} 
\end{itemize}

\begin{quote}
《引用の始まり》
\end{quote}

\begin{quote}
公立宇出津総合病院だより - 能登町http://www.town.noto.lg.jp ›
prnotoPDF2020/12/12 --- 石川県鳳珠郡能登町字宇出津新. 1字19. 7. 番地1.
☎. 0. 7. 6 \ldots{} 図1 公立宇出津総合病院の医業収益・費用、収支比率.
医業収益 医業費用 医業収支比率 \ldots{} 会場 役場能都庁舎4階大集会場.
出演 小林幸子さん、堀内孝雄さん.
\end{quote}

\begin{quote}
《引用の終わり》
\end{quote}

\begin{itemize}
\item
  堀内孝雄 宇出津 - Google 検索
  \url{https://www.google.com/search?q=\%E5\%A0\%80\%E5\%86\%85\%E5\%AD\%9D\%E9\%9B\%84+\%E5\%AE\%87\%E5\%87\%BA\%E6\%B4\%A5\&oq=\%E5\%A0\%80\%E5\%86\%85\%E5\%AD\%9D\%E9\%9B\%84\%E3\%80\%80\%E5\%AE\%87\%E5\%87\%BA\%E6\%B4\%A5\&aqs=chrome..69i57.1420j0j15\&sourceid=chrome\&ie=UTF-8} 
\item
  0812.indd \url{https://t.co/LL1cWBeDzz} 
\end{itemize}

 まったくダメ元つもりの検索だったのですが、のと広報の2008年12月号だと思います。PDFファイルが出てきました。文字が細かいこともあり、該当箇所を探すのに手間取ったのですが、見つけることができました。

 なにより意外な発見だったのが、ずっと前から気になっていた前の宇出津病院の建物の写真がそこにあり、2階建てと思っていたのが3階建てだったのも意外なのですが、記憶の風景以上に、前の金沢地方裁判所の建物とよく似ていると思いました。

〉〉〉 kk\_hironoのリツイート 〉〉〉

\begin{itemize}
\tightlist
\item
  RT
  kk\_hirono(刑事告発・非常上告_金沢地方検察庁御中)|s\_hirono(非常上告-最高検察庁御中\_ツイッター)
  日時:2021-06-05 22:27/2021/06/05 22:15 URL:
  \url{https://twitter.com/kk\_hirono/status/1401168889262723085} 
  \url{https://twitter.com/s\_hirono/status/1401165875256860673} 
  \textgreater{}
  2021-06-05-215935\_日時平成21年1月15日(木) 会場役場能都庁舎4階大集会場出演小林幸子さん、堀内孝雄さん.jpg
  \url{https://t.co/nifQk3QgAz} 
\end{itemize}

〉〉〉 kk\_hironoのリツイート 〉〉〉

\begin{itemize}
\tightlist
\item
  RT
  kk\_hirono(刑事告発・非常上告_金沢地方検察庁御中)|s\_hirono(非常上告-最高検察庁御中\_ツイッター)
  日時:2021-06-05 22:28/2021/06/05 22:15 URL:
  \url{https://twitter.com/kk\_hirono/status/1401168942245158915} 
  \url{https://twitter.com/s\_hirono/status/1401165803068661761} 
  \textgreater{}
  2021-06-05-215133\_察審査会をごぞんじですか?  交通事故、詐欺、おどしなど犯罪の被害に遭ったが検察官がその事件を起訴してくれない。このような不満をお持ちの人は.jpg
  \url{https://t.co/EyrZVSpxM8} 
\end{itemize}

〉〉〉 kk\_hironoのリツイート 〉〉〉

\begin{itemize}
\tightlist
\item
  RT
  kk\_hirono(刑事告発・非常上告_金沢地方検察庁御中)|s\_hirono(非常上告-最高検察庁御中\_ツイッター)
  日時:2021-06-05 22:28/2021/06/05 22:15 URL:
  \url{https://twitter.com/kk\_hirono/status/1401168968690266113} 
  \url{https://twitter.com/s\_hirono/status/1401165730557558788} 
  \textgreater{}
  2021-06-05-214940\_昭和43年 5月に改築が完成した宇出津病院 当時奥能登唯一の神経科・精神科を新設。昭和53年7月から総合病院.jpg
  \url{https://t.co/ZgiyWElOoc} 
\end{itemize}

 ちょっと勘違いしていたのですが、「昭和43年 5月に改築が完成した宇出津病院 当時奥能登唯一の神経科・精神科を新設。」とあって、能登ではなく奥能登でした。奥能登は現在の市町村で鳳珠郡の穴水町、能登町、輪島市、珠洲市になりますが、勘違いしたのが七尾市になります。

 宇出津病院に精神科があったらしいことは昭和40年代の記憶としてよく憶えているのですが、ずばりキチガイ病院と呼ばれ、「宇出津病院の裏」とも呼ばれていました。小学校低学年の子供には恐れられたお化け屋敷のような存在でした。

 宇出津病院の裏と呼ばれていましたが、正確には正面から左手で奥の方に庭のような出入り口がありました。開放されていましたが、車が出入りするような場所ではなく、文字通り庭のような印象で、真ん中あたりに桜の木があったと記憶にあります。

 たぶん昭和45年辺りになると思うのですが、2週間ぐらいだったと思うのですが、その宇出津病院の裏に入院したことがあり、赤痢という病気であったように思います。どちらが先になるのかわからないですが、大阪万国博覧会の帰りに麻疹になって同じぐらい京都の親戚の家で寝込んでいたことがありました。

 麻疹の方は1,2年ほど前にもTwitterのトレンドで話題になっていた記憶がありますが、赤痢という病気のことはずっと見聞きしたことがなく、どういう病気なのか調べてもおらずわかっていません。

\begin{itemize}
\tightlist
\item
  細菌性赤痢 \url{https://t.co/GKJhDca7El} 
\end{itemize}

 細菌というのは意外でしたが、細菌で思い出すのが「悪魔の飽食」という本の731部隊のことですが、昭和57年の秋に名古屋でその本を読んでいたときも、思い出にある宇出津病院の裏と情景が重なるところがありました。なお、宇出津病院の裏で明らかに精神異常という人を見かけた記憶はありません。

 能登町役場4階での堀内孝雄の公演ですが、のと広報では来月の予定とされ、それが平成21年1月15日となっていました。たぶん予定通り行われたのだと思いますが、私の記憶ではその平成21年3月15日に羽咋市から宇出津に戻った後のこととなっていました。

 調べる前に、もう一人は名のしれた女性歌手がいたように思っていたのですが、小林幸子というのは全く意外で、当時はNHK紅白歌合戦の目玉で出場歌手の常連になっていたと思います。更に開演が14時30分、終演が16時の予定となっているのですが、私の記憶では夜に見たことになっていました。

 似たような経験は過去に一度だけあったのですが、昭和50年の前半のことで、フォークソングの古時計という、歌手の名前か曲の名前なのかも記憶でははっきりしないのですが、同じ場所になりますが、当時は能都町役場だった4階で視聴したという記憶があります。

\begin{itemize}
\tightlist
\item
  古時計 『ロードショー』 1976年 - YouTube \url{https://t.co/Rp8XjlWfvr} 
\end{itemize}

 今で言う一発屋のようなヒット曲だったと記憶にあるのですが、歌手の名前が古時計だったようです。当時はよく知られた曲だったと思います。

 深澤諭史弁護士のツイートを見たり思い出すと、古い曲のメロディが替え歌の歌詞となって思い浮かぶことが多いのですが、「夢のパラダイスさ花の東京」という曲などもまさに「弁護士パラダイス」として脳内変換されます。

 他に「ああ、上野駅」が定番で、「弁護士列車に揺られて揺れて」となります。今回はこれまでと違って、深刻悲壮な悔恨という情感をともなって、「愚か者」が出てきたように思います。同時に思い浮かべたのが堀内孝雄の「愛しき日々」という曲になります。

 逆説的なのですが、噛みしめるほど愚か者の迷惑野郎だと思ったのが深澤諭史弁護士@fukazawasになります。

 金沢刑務所での生活は昔の卯辰山の思い出と一緒になることが多かったのですが、平成12年11月から平成13年12月31日の受刑生活では、ずっと同じ3階の独居房にいたのですが、毎週土曜日の10時頃にテレビで「いい旅夢気分」の放送がありました。これも深澤諭史弁護士で「弁護士いい気分」になります。

 金沢刑務所での生活は昔の卯辰山の思い出と一緒になることが多かったのですが、平成12年11月から平成13年12月31日の受刑生活では、ずっと同じ3階の独居房にいたのですが、毎週土曜日の10時頃にテレビで「いい旅夢気分」の放送がありました。これも深澤諭史弁護士で「弁護士いい気分」になります。

 奉納\さらば弁護士鉄道・泥棒神社の物語(@hirono\_hideki)のツイートで、これまでの流れを確認しておきたいと思います。リツイートします。

〉〉〉 kk\_hironoのリツイート 〉〉〉

\begin{itemize}
\tightlist
\item
  RT
  kk\_hirono(刑事告発・非常上告_金沢地方検察庁御中)|hirono\_hideki(奉納\さらば弁護士鉄道・泥棒神社の物語)
  日時:2021-06-05 23:14/2021/06/05 22:18 URL:
  \url{https://twitter.com/kk\_hirono/status/1401180524492648461} 
  \url{https://twitter.com/hirono\_hideki/status/1401166638280445952} 
  \textgreater{} 堀内孝雄 宇出津 - Google 検索 \url{https://t.co/nkgaCuS0Uf} 
\end{itemize}

〉〉〉 kk\_hironoのリツイート 〉〉〉

\begin{itemize}
\tightlist
\item
  RT
  kk\_hirono(刑事告発・非常上告_金沢地方検察庁御中)|hirono\_hideki(奉納\さらば弁護士鉄道・泥棒神社の物語)
  日時:2021-06-05 23:14/2021/06/05 22:04 URL:
  \url{https://twitter.com/kk\_hirono/status/1401180534412218370} 
  \url{https://twitter.com/hirono\_hideki/status/1401163136003022851} 
  \textgreater{} 時代劇【白虎隊】主題歌 ~『愛しき日々(歌詞付き)』 -
  YouTube \url{https://t.co/YnIGqmL2zl} 
\end{itemize}

〉〉〉 kk\_hironoのリツイート 〉〉〉

\begin{itemize}
\tightlist
\item
  RT
  kk\_hirono(刑事告発・非常上告_金沢地方検察庁御中)|hirono\_hideki(奉納\さらば弁護士鉄道・泥棒神社の物語)
  日時:2021-06-05 23:14/2021/06/05 22:01 URL:
  \url{https://twitter.com/kk\_hirono/status/1401180554632912905} 
  \url{https://twitter.com/hirono\_hideki/status/1401162293090549762} 
  \textgreater{} 都会の天使たち 桂銀淑・堀内孝雄 - YouTube
  \url{https://t.co/iSGJ2oZC1J} 
\end{itemize}

〉〉〉 kk\_hironoのリツイート 〉〉〉

\begin{itemize}
\tightlist
\item
  RT
  kk\_hirono(刑事告発・非常上告_金沢地方検察庁御中)|hirono\_hideki(奉納\さらば弁護士鉄道・泥棒神社の物語)
  日時:2021-06-05 23:14/2021/06/05 21:45 URL:
  \url{https://twitter.com/kk\_hirono/status/1401180576267132928} 
  \url{https://twitter.com/hirono\_hideki/status/1401158158014287877} 
  \textgreater{}
  かくれんぼ【本物】/堀内孝雄/はぐれ刑事純情派2002年エンディングテーマ曲
  - YouTube \url{https://t.co/7nLIV1iQmk} 
\end{itemize}

〉〉〉 kk\_hironoのリツイート 〉〉〉

\begin{itemize}
\tightlist
\item
  RT
  kk\_hirono(刑事告発・非常上告_金沢地方検察庁御中)|hirono\_hideki(奉納\さらば弁護士鉄道・泥棒神社の物語)
  日時:2021-06-05 23:14/2021/06/05 21:43 URL:
  \url{https://twitter.com/kk\_hirono/status/1401180619749552128} 
  \url{https://twitter.com/hirono\_hideki/status/1401157786390519816} 
  \textgreater{} 2021-06-05\_21:41
  奉納\\#危険生物・弁護士脳汚染除去装置\\#金沢地方検察庁御中\_2020:
  hirono-REGEXP:''@fukazawas''/データベース登録済みツイートの検索:2013-03-04〜2021-06-05/2021年06月05日21時40分の記録:ユーザ・投稿:3/3533件
  \url{https://t.co/3m6yHGp1dt} 
\end{itemize}

〉〉〉 kk\_hironoのリツイート 〉〉〉

\begin{itemize}
\tightlist
\item
  RT
  kk\_hirono(刑事告発・非常上告_金沢地方検察庁御中)|hirono\_hideki(奉納\さらば弁護士鉄道・泥棒神社の物語)
  日時:2021-06-05 23:14/2021/06/05 21:43 URL:
  \url{https://twitter.com/kk\_hirono/status/1401180627949395974} 
  \url{https://twitter.com/hirono\_hideki/status/1401157759974854658} 
  \textgreater{} 2021-06-05\_21:40
  奉納\\#危険生物・弁護士脳汚染除去装置\\#金沢地方検察庁御中\_2020:
  Ex-REGEXP:''@fukazawas''/データベース登録済みツイートの検索:2013-03-04〜2021-06-05/2021年06月05日21時39分の記録:ユーザ・投稿:3/3533件
  \url{https://t.co/hrQsoqYiO5} 
\end{itemize}

〉〉〉 kk\_hironoのリツイート 〉〉〉

\begin{itemize}
\tightlist
\item
  RT
  kk\_hirono(刑事告発・非常上告_金沢地方検察庁御中)|hirono\_hideki(奉納\さらば弁護士鉄道・泥棒神社の物語)
  日時:2021-06-05 23:14/2021/06/05 21:43 URL:
  \url{https://twitter.com/kk\_hirono/status/1401180643506081793} 
  \url{https://twitter.com/hirono\_hideki/status/1401157733471055875} 
  \textgreater{} 2021-06-05\_21:37
  奉納\\#危険生物・弁護士脳汚染除去装置\\#金沢地方検察庁御中\_2020:
  Ex-REGEXP:''@fukazawas''/データベース登録済みツイートの検索:2012-10-23〜2021-06-04/2021年06月05日21時14分の記録:ユーザ・投稿:567/8755件
  \url{https://t.co/Kk13RFFBOD} 
\end{itemize}

〉〉〉 kk\_hironoのリツイート 〉〉〉

\begin{itemize}
\tightlist
\item
  RT
  kk\_hirono(刑事告発・非常上告_金沢地方検察庁御中)|hirono\_hideki(奉納\さらば弁護士鉄道・泥棒神社の物語)
  日時:2021-06-05 23:14/2021/06/05 21:11 URL:
  \url{https://twitter.com/kk\_hirono/status/1401180656147726337} 
  \url{https://twitter.com/hirono\_hideki/status/1401149628578373633} 
  \textgreater{} 堀内孝雄 恋文 歌詞 - 歌ネット \url{https://t.co/IB1lF5Cnxl} 
\end{itemize}

〉〉〉 kk\_hironoのリツイート 〉〉〉

\begin{itemize}
\tightlist
\item
  RT
  kk\_hirono(刑事告発・非常上告_金沢地方検察庁御中)|hirono\_hideki(奉納\さらば弁護士鉄道・泥棒神社の物語)
  日時:2021-06-05 23:14/2021/06/05 21:10 URL:
  \url{https://twitter.com/kk\_hirono/status/1401180678528528389} 
  \url{https://twitter.com/hirono\_hideki/status/1401149385723944965} 
  \textgreater{} 恋文 歌詞 - Google 検索 \url{https://t.co/EqvVv1OHAR} 
\end{itemize}

〉〉〉 kk\_hironoのリツイート 〉〉〉

\begin{itemize}
\tightlist
\item
  RT
  kk\_hirono(刑事告発・非常上告_金沢地方検察庁御中)|hirono\_hideki(奉納\さらば弁護士鉄道・泥棒神社の物語)
  日時:2021-06-05 23:14/2021/06/05 21:07 URL:
  \url{https://twitter.com/kk\_hirono/status/1401180694391320577} 
  \url{https://twitter.com/hirono\_hideki/status/1401148688789950464} 
  \textgreater{} 恋文【本物】/堀内孝雄 - YouTube
  \url{https://t.co/y7XUYMVQfF}  105,948 回視聴 •2019/04/14
\end{itemize}

〉〉〉 kk\_hironoのリツイート 〉〉〉

\begin{itemize}
\tightlist
\item
  RT
  kk\_hirono(刑事告発・非常上告_金沢地方検察庁御中)|hirono\_hideki(奉納\さらば弁護士鉄道・泥棒神社の物語)
  日時:2021-06-05 23:14/2021/06/05 20:47 URL:
  \url{https://twitter.com/kk\_hirono/status/1401180716486909956} 
  \url{https://twitter.com/hirono\_hideki/status/1401143677989228547} 
  \textgreater{} 【高音量】懐かしい堀内孝雄の歌15曲‼️ - YouTube
  \url{https://t.co/51Rvo2YW1g} 
\end{itemize}

〉〉〉 kk\_hironoのリツイート 〉〉〉

\begin{itemize}
\tightlist
\item
  RT
  kk\_hirono(刑事告発・非常上告_金沢地方検察庁御中)|hirono\_hideki(奉納\さらば弁護士鉄道・泥棒神社の物語)
  日時:2021-06-05 23:14/2021/06/05 20:47 URL:
  \url{https://twitter.com/kk\_hirono/status/1401180734698577921} 
  \url{https://twitter.com/hirono\_hideki/status/1401143615569600512} 
  \textgreater{} 堀内孝雄 / 恋唄綴り - YouTube \url{https://t.co/jSgQR30P3u} 
\end{itemize}

〉〉〉 kk\_hironoのリツイート 〉〉〉

\begin{itemize}
\tightlist
\item
  RT
  kk\_hirono(刑事告発・非常上告_金沢地方検察庁御中)|hirono\_hideki(奉納\さらば弁護士鉄道・泥棒神社の物語)
  日時:2021-06-05 23:15/2021/06/05 20:34 URL:
  \url{https://twitter.com/kk\_hirono/status/1401180777400860674} 
  \url{https://twitter.com/hirono\_hideki/status/1401140473700388865} 
  \textgreater{} 「愛しき日々」堀内孝雄 - YouTube
  \url{https://t.co/AnuSJhgZ4o}  13,580,578 回視聴 •2008/10/17
\end{itemize}

〉〉〉 kk\_hironoのリツイート 〉〉〉

\begin{itemize}
\tightlist
\item
  RT
  kk\_hirono(刑事告発・非常上告_金沢地方検察庁御中)|hirono\_hideki(奉納\さらば弁護士鉄道・泥棒神社の物語)
  日時:2021-06-05 23:15/2021/06/05 20:30 URL:
  \url{https://twitter.com/kk\_hirono/status/1401180809751502854} 
  \url{https://twitter.com/hirono\_hideki/status/1401139461899972610} 
  \textgreater{} 堀内 孝雄 / 愛しき日々 - YouTube
  \url{https://t.co/804GnSKiYW} 
\end{itemize}

〉〉〉 kk\_hironoのリツイート 〉〉〉

\begin{itemize}
\tightlist
\item
  RT
  kk\_hirono(刑事告発・非常上告_金沢地方検察庁御中)|masaki\_kito(紀藤正樹
  MasakiKito) 日時:2021-06-05 23:15/2021/06/04 21:32 URL:
  \url{https://twitter.com/kk\_hirono/status/1401180876818444291} 
  \url{https://twitter.com/masaki\_kito/status/1400792681408724995} 
  \textgreater{}
  これも出会い系の被害というんですよね>出会い系アプリで知り合い合意の上で性行為をした男性に対し「どう責任取るんだ。警察に被害届を出す」などと現金を要求し100万円以上を脅し取ったとして小高優希容疑者(24)逮捕>性行為後に豹変、「女の体なめんなよ」と脅迫か
  \url{https://t.co/JQWeGxCONU} 
\end{itemize}

〉〉〉 kk\_hironoのリツイート 〉〉〉

\begin{itemize}
\tightlist
\item
  RT
  kk\_hirono(刑事告発・非常上告_金沢地方検察庁御中)|hirono\_hideki(奉納\さらば弁護士鉄道・泥棒神社の物語)
  日時:2021-06-05 23:15/2021/06/05 19:55 URL:
  \url{https://twitter.com/kk\_hirono/status/1401180895537553408} 
  \url{https://twitter.com/hirono\_hideki/status/1401130570852098055} 
  \textgreater{} ▶
  ブロックされたツイート%rippy08(りっぴぃ)%2021/06/04 18:44:30%
  \url{https://t.co/mj1ycqYDrC}  \textgreater{}
  かつてB肝弁護団で署名集めしたときとか,和歌山の先生方にどれだけ世話になったよ・・・
  (署名集めのためにイベントに乗り込んだのですが,イベント会場まで来るまで送迎してくれたのも
\end{itemize}

〉〉〉 kk\_hironoのリツイート 〉〉〉

\begin{itemize}
\tightlist
\item
  RT
  kk\_hirono(刑事告発・非常上告_金沢地方検察庁御中)|hirono\_hideki(奉納\さらば弁護士鉄道・泥棒神社の物語)
  日時:2021-06-05 23:15/2021/06/05 19:52 URL:
  \url{https://twitter.com/kk\_hirono/status/1401180908690886657} 
  \url{https://twitter.com/hirono\_hideki/status/1401129731785773061} 
  \textgreater{} ▶
  ブロックされたツイート%k\_sawmen(泥濘大魔王サイケ)%2021/06/04
  23:24:57% \url{https://t.co/tleRhSLNxd}  \textgreater{}
  弁護士経験積んで離婚案件複数見てからの婚活って精神的にハードモードそうだな・・・・・・
\end{itemize}

〉〉〉 kk\_hironoのリツイート 〉〉〉

\begin{itemize}
\tightlist
\item
  RT
  kk\_hirono(刑事告発・非常上告_金沢地方検察庁御中)|hirono\_hideki(奉納\さらば弁護士鉄道・泥棒神社の物語)
  日時:2021-06-05 23:15/2021/06/05 19:47 URL:
  \url{https://twitter.com/kk\_hirono/status/1401180923752697860} 
  \url{https://twitter.com/hirono\_hideki/status/1401128628608327684} 
  \textgreater{} ▶
  ブロックされたツイート%akagilaw(赤木真也(弁護士・LEC専任講師))%2021/06/05
  13:37:13% \url{https://t.co/8sWoS5vR6L}  \textgreater{}
  マトモさのカケラもないな・・・よくこんな連中を支持できる人間がいるものだ
\end{itemize}

 細菌のYouTubeは強制的に次の動画へと移っていくのですが、15曲という中の1曲で、一覧にある再生時間から曲名を調べました。それが「恋文」というタイトルの曲だったのですが、少し聞いた覚えはあるものの、歌詞の方は不思議なほど記憶にない曲でした。

 初めに見つけたのが曲名が同じ別の曲の歌詞で、次に見つけた歌詞のページに1992年5月25日発売とありました。

 平成4年ですが、その5月28日から始まったのが金沢刑務所の拘置所の生活で、繰り返し記述していると思いますが、ニュース番組が少ない分、ラジオで歌の番組や、まるで有線放送のような感じでも多くの曲が多く流れていました。

 トラック運転手の仕事では交通情報を聞く目的もあってラジオ放送を聴いていることが多かったのですが、それ以上に曲を聴くことが多く、ジャンルも多彩で、「浪曲十八番」というラジオ放送もありました。

 金儲けの目的で弁護士になり、その出世街道ならぬ弁護士街道をひとすじに突き進むのが深澤諭史弁護士@fukazawasという印象で、他者を顧みないところが被告発人岡田進弁護士に似ていると思いました。違いはあって、深澤諭史弁護士の場合、同調者や疑いをもたない人にはとても親切で友好的にみえます。

 ここ1,2年は敵対的他者に対する攻撃性を余り見せなくなり、牙を隠しているようにも見えた深澤諭史弁護士ですが、前には非弁ハンターやバスターなどを自称し、本人は言っていなかったかもしれないですが、○○絶対殺すマン、などと評価され、そのツイートを自らリツイートしていたと思います。

 1つ、まとめ記事作成・投稿の新たにスクリプトを作成したのですが、私の3つのTwitterアカウントのツイート・リツイートのみに絞ったまとめになります。昨日になるのか今朝になるのか思い出せないですが、先に私の3つのアカウントを除外するスクリプト(コマンド)を作成していました。

 一通り全記録を網羅できたかと思うのですが、次のまとめ記事を連続で作成しました。

\begin{itemize}
\tightlist
\item
  2021年06月05日20時18分の登録:
  REGEXP:''(深澤諭史弁護士|深澤先生|深澤諭史)''/データベース登録済みツイートの検索:2021-05-15〜2021-06-05/2021年06月05日20時18分の記録:ユーザ・投稿:17/115件
  \url{https://kk2020-09.blogspot.com/2021/06/regexp2021-05-152021-06.html} 
\item
  2021年06月05日20時30分の登録:
  REGEXP:''(深澤諭史弁護士|深澤先生|深澤弁護士)''/データベース登録済みツイートの検索:2010-12-08〜2021-06-05/2021年06月05日20時19分の記録:ユーザ・投稿:266/4360件
  \url{https://kk2020-09.blogspot.com/2021/06/regexp2010-12-082021-06\_5.html} 
\item
  2021年06月05日20時49分の登録:
  Ex-REGEXP:''(深澤諭史弁護士|深澤先生|深澤弁護士)''/データベース登録済みツイートの検索:2010-12-08〜2021-06-05/2021年06月05日20時39分の記録:ユーザ・投稿:263/1086件
  \url{https://kk2020-09.blogspot.com/2021/06/ex-regexp2010-12-082021-06.html} 
\item
  2021年06月05日21時37分の登録:
  Ex-REGEXP:''@fukazawas''/データベース登録済みツイートの検索:2012-10-23〜2021-06-04/2021年06月05日21時14分の記録:ユーザ・投稿:567/8755件
  \url{https://kk2020-09.blogspot.com/2021/06/ex-regexpfukazawas2012-10-232021-06.html} 
\item
  2021年06月05日21時40分の登録:
  Ex-REGEXP:''@fukazawas''/データベース登録済みツイートの検索:2013-03-04〜2021-06-05/2021年06月05日21時39分の記録:ユーザ・投稿:3/3533件
  \url{https://kk2020-09.blogspot.com/2021/06/ex-regexpfukazawas2013-03-042021-06.html} 
\item
  2021年06月05日21時41分の登録:
  hirono-REGEXP:''@fukazawas''/データベース登録済みツイートの検索:2013-03-04〜2021-06-05/2021年06月05日21時40分の記録:ユーザ・投稿:3/3533件
  \url{https://kk2020-09.blogspot.com/2021/06/hirono-regexpfukazawas2013-03-042021-06.html} 
\end{itemize}

 ここで少し思い出しかけたのですが、インパール作戦にも代表される深澤諭史弁護士の戦争犠牲者に対する評価、これが他の曲の歌詞からつながって、堀内孝雄の「愛しき日々」へバトンをつなぐように行き着いたように思います。

 司法制度改革でインパール作戦の餓死者と同列に、夜を日についで不満と不平をツイートしながら、Twitterにごちそうの写真を並べ、宴などとツイートしていた深澤諭史弁護士@fukazawasのことです。

 投稿済みのまとめ記事でページ内検索を行っても簡単に見つからず、スクリーンショットの記録で深澤諭史弁護士に返信したような新谷泰真弁護士のツイートを見たぐらいだったのですが、キーワードからまとめ記事を作成すると、内容はまだ確認していないですが深澤諭史弁護士のアカウントがありました。

\begin{itemize}
\tightlist
\item
  2021年06月05日23時52分の登録:
  「絶対.*マン」を@hirono\_hideki @kk\_hirono @s\_hironoで検索 45件の該当 2021-06-05\_23:51の記録
  \url{https://kk2020-09.blogspot.com/2021/06/hironohidekikkhironoshirono452021-06.html} 
\item
  2021年06月05日23時55分の登録:
  REGEXP:''非弁絶対殺すマン''/データベース登録済みツイートの検索:2016-07-12〜2017-01-31/2021年06月05日23時54分の記録:ユーザ・投稿:4/9件
  \url{https://kk2020-09.blogspot.com/2021/06/regexp2016-07-122017-01-3120210605235449.html} 
\end{itemize}

 調べれば拡張機能でできるのかもしれないですが、ブラウザのページ内検索では正規表現が使えず、不便を感じます。この不便の解消のためにまとめ記事を作成することもあります。

\begin{itemize}
\tightlist
\item
  (1/9) RT @fukazawas(深澤諭史)|Jakotsunya(魔王じゃこにゃー)
  日時:2016-07-12 17:05:00 +0900/2016-07-12 17:03:00 +0900 URL:
  \url{https://twitter.com/fukazawas/status/752775941949042689} 
  \url{https://twitter.com/Jakotsunya/status/752775270105354240\textgreater} {}
  ・ネット犯罪、SNSから手広く時代に合わせたオールマイティな弁護士さんのイメージがあります。\textgreater{}
  ・どこでも暮らしていけそうです\textgreater{} ・優しい\textgreater{}
  ・人望があつそうです\textgreater{} ・非弁キラー\textgreater{}
  ・非弁絶対殺すマン\textgreater{}
  ・ペットボト(文はここで途切れている)\textgreater{}
  \url{https://t.co/7jWwXrvA8p} 
\end{itemize}

 「(3/9) RT
@fukazawas(深澤諭史)|Noooooooorth(キュアノースライム)
日時:2016-08-31 19:45:00 +0900/2016-08-31 19:45:00 +0900
URL:」がありますが、削除された前の北周士弁護士のTwitterアカウントになると思います。

 たぶん強く印象に残っていたのが上記の北周士弁護士のツイートで、それを深澤諭史弁護士がリツイートしていました。

\begin{itemize}
\item
  (4/9) RT @fukazawas(深澤諭史)|IcyFumin(首席監察官フミン)
  日時:2016-08-31 21:46:00 +0900/2016-08-31 21:38:00 +0900 URL:
  \url{https://twitter.com/fukazawas/status/770965994613334021} 
  \url{https://twitter.com/IcyFumin/status/770964095658930176\textgreater} {}
  司法制度改革絶対許さないマン\textgreater{}
  非弁絶対殺すマン\textgreater{} 我、成仏理論を相手とせず
  \url{https://t.co/EW4FWEcPMD} 
\item
  (7/9) TW @yasumasa218(新谷泰真) 日時: 2017-01-31 08:06:00 +0900
  URL:
  \url{https://twitter.com/yasumasa218/status/826204875629359105\textgreater} {}
  非弁絶対殺すマン \url{https://t.co/QT6uRa68tC} 
\item
  (8/9) RT @fukazawas(深澤諭史)|yasumasa218(新谷泰真)
  日時:2017-01-31 08:16:00 +0900/2017-01-31 08:06:00 +0900 URL:
  \url{https://twitter.com/fukazawas/status/826207557240778752} 
  \url{https://twitter.com/yasumasa218/status/826204875629359105\textgreater} {}
  非弁絶対殺すマン \url{https://t.co/QT6uRa68tC} 
\end{itemize}

 そういえば、と思ったら日付が変わっていました。夕方になると思いますが、非弁についてこれまでとは違った発見があり、それというのも最近知ったばかりの最首克也弁護士のYouTube動画でした。

 非弁を消費者問題として強調してきたのも深澤諭史弁護士ですが、同業の弁護士による消費者被害についてはほとんど触れないか、脳機能障害の少女の問題では、多額の横領をした弁護士が司法制度改革の被害者のような話にしていました。盲腸に並びトップクラスになる深澤諭史弁護士の超絶ツイートです。

\begin{lstlisting}
base ❯ d|grep 脳機能障害
\end{lstlisting}

\begin{itemize}
\tightlist
\item
  2018年04月10日02時39分の登録:
  REGEXP:''脳機能障害.*少女''/データベース登録済みツイート:2018年04月10日02時38分の記録:ユーザ・投稿:5/23件
  \url{http://hirono2014sk.blogspot.com/2018/04/regexp201804100238523.html} 
\item
  2018年09月08日15時54分の登録:
  REGEXP:''脳機能障害.*少女''/データベース登録済みツイート:2018年09月08日15時54分の記録:ユーザ・投稿:5/24件
  \url{http://hirono2014sk.blogspot.com/2018/09/regexp201809081554524.html} 
\item
  2018年09月08日16時08分の登録:
  REGEXP:''脳機能障害.*少女''/データベース登録済みツイート:2018年09月08日16時08分の記録:ユーザ・投稿:6/35件
  \url{http://hirono2014sk.blogspot.com/2018/09/regexp201809081608635.html} 
\item
  2019年06月21日14時39分の登録:
  %@yukihirosasamo ささもひょんⁿ@赤腹魔王%さっきのと同一人物ですか・・・・・・。開いた口が塞がらないよ・・・・・・。¥\n脳機能障害を負った少女の一家から着服
  「示談不成立」とウソ
  \url{http://hirono2014sk.blogspot.com/2019/06/yukihirosasamon.html} 
\item
  2021年02月18日22時20分の登録:
  %@uwaaaa サイ太%高次脳機能障害,体幹機能障害という重篤な障害を負い,今も原告が介護しているという話で,ようやく認容額が175万円ですからね。「不貞慰謝料の相場は
  \url{https://kk2020-09.blogspot.com/2021/02/uwaaaa\_18.html} 
\end{itemize}

 直接、深澤諭史弁護士と結びつけた記事の作成はしていなかったようですが、はてなブログのエントリーになっているかもしれません。

 ありませんでした。しっかり記録を残したつもりだったのですが、すぐに見つからないようでは話になりません。

\begin{itemize}
\item
  2021年06月06日00時19分の登録:
  「脳機能障害.+少女」を@hirono\_hideki @kk\_hirono @s\_hironoで検索 94件の該当 2021-06-06\_00:19の記録
  \url{https://kk2020-09.blogspot.com/2021/06/hironohidekikkhironoshirono942021-06.html} 
\item
  2016-05-17 09:18:46 ``****
  脳機能障害を負った少女の一家から示談不成立と5千万円以上着服した弁護士のニュースに対する、深澤諭史弁護士の司法制度改革批判''
  \url{https://twitter.com/kk\_hirono/status/732364704811651072} 
\item
  2016-05-18 09:28:00
  ``脳機能障害を負った少女の一家から示談不成立と5千万円以上着服した弁護士のニュースに対する、深澤諭史弁護士の司法制度改革批判\ldots{}
  \url{http://fb.me/14BTjRIel''} 
  \url{https://twitter.com/hirono\_hideki/status/732729413842964480} 
\end{itemize}

 最近になって表示が早くなったココログフリーのブログの記事でしたが、埋め込みツイートも使われておらず見づらくなっています。Bloggerのブログの方にも同時に投稿しているはずなのですが見つかりません。もう一度、記録としてしっかり残しておく必要がありそうです。

 非弁と司法制度改革改革批判だけでも山のようにある深澤諭史弁護士のツイートです。その中でも選りすぐりの一品の一つが、盲腸とこの脳機能障害に関連した深澤諭史弁護士のツイートでした。

 この「脳機能障害を負った少女の一家から示談不成立と5千万円以上着服した弁護士のニュース」とよく似たもので、ココログフリーに残されていた記事から再発見できたのが、被告発人岡田進弁護士が被告側代理人として関与したという過失相殺の民事裁判でした。

 過失相殺を見出しに使った記事は少ないと思うのですぐにみつかりそうです。

 過失相殺で見つからず脳機能障害と同じ不思議な現象なのかと疑い始めたのですが、端末の表示を遡って見つけたところ過失割合になっていました。

\begin{itemize}
\tightlist
\item
  1333:2021-04-26\_07:29:01 \#告発状 \#\#\#\#
  被告発人岡田進弁護士が代理人となり過失割合が45%になった,金沢市内で1級1号後遺障害を残した13歳女子の自転車事故
  \url{https://hirono-hideki.hatenadiary.jp/entry/2021/04/26/072859} 
\end{itemize}

 これも深澤諭史弁護士の弁護士費用の割合がきっかけで辿る経過となったのですが、まるでお花畑のような欲望三昧の深澤諭史弁護士の弁護士費用の割合に関するツイートでした。幻想的で危険な蝶に思えてきたのですが、本当に弁護士畑の開墾事業だと思います。

 荘子の胡蝶の夢も、獲物をより大きな獲物が狙うという話でした。荘子はそれに気がついて立ち止まるという話であったように思います。

\begin{itemize}
\tightlist
\item
  胡蝶の夢 - Wikipedia \url{https://t.co/E5xQTMCJUU} 
\end{itemize}

 私の記憶では前段はともかく、後段では李下に冠を正さずに似たような話になっていたように思うのですが、上記のWikipediaでは後段のような部分がありませんでした。

\begin{itemize}
\tightlist
\item
  No.1293 【蟷螂窺蟬】 とうろうきせん|今日の四字熟語・故事成語|福島みんなのNEWS
  - 福島ニュース 福島情報 イベント情報 企業・店舗情報 インタビュー記事
  \url{https://t.co/ca8uoF8b1B} 
\end{itemize}

 四文字熟語としては初めて見たように思うのですが、この「蟷螂窺蟬」の内容を「胡蝶の夢」と取り違え、あるいは混同して勘違いしていたようです。

\begin{itemize}
\tightlist
\item
  1400:2021-06-06\_01:28:23 \#告発状 \#\#\#\#
  脳機能障害を負った少女の一家から示談不成立と5千万円以上着服した弁護士のニュースに対する、2015年7月22日の深澤諭史弁護士の反応の記録(2021年6月6日)
  \url{https://hirono-hideki.hatenadiary.jp/entry/2021/06/06/012821} 
\end{itemize}

 3つほど投稿に失敗したのですが、orgのファイルを無理にMarkdownにしました。

 肝心の深澤諭史弁護士のツイートが掲載されていないようだったので、探し出しました。

\begin{itemize}
\tightlist
\item
  TW fukazawas(深澤諭史) 日時: 2015/07/22 18:49:27 URL:
  \url{https://twitter.com/fukazawas/status/623791961539919873} 
  \textgreater{} >RT\\
  \textgreater{}
  自由競争とか,市場原理とか,無邪気にいっている先生,特に佐藤幸治先生とかには,百回くらい読んで頂きたい記事ですね。\\
  \textgreater{} (#・∀・)
\end{itemize}

 どうも令和3年3月31日付告発状ではなく、参考資料扱いになっていたようです。(2021年6月6日)という部分の日付を変えたのですが、もしかすると令和3年3月31日付告発状の作成に取り掛かる前かもしれません。確認しておきます。

 (2020年10月17日)となっていました転載の日付です。10月17日であれば令和3年3月31日付告発状の作成には取り掛かっていたように思います。

 正式に告発状には取り込んでいないようですが、暫定的な措置で、いずれしっかりと取り上げ、記録をしておきたいと思います。

\begin{itemize}
\tightlist
\item
  〈〈〈 2021/06/06 01:47:01 Linux Emacs: 〈〈〈
\end{itemize}

\hypertarget{ux5b66ux5f92ux51faux9663ux306eux884cux9032ux66f2ux3068ux5f01ux8b77ux58ebux9244ux9053ux306eux6b74ux53f2ux3092ux5f69ux308bux6df1ux6fa4ux8aedux53f2ux5f01ux8b77ux58ebux4ecaux671dux767aux6398ux3057ux305f2012ux5e74ux306eux30e2ux30c8ux30b1ux30f3ux3053ux3068ux77e2ux90e8ux5584ux6717ux5f01ux8b77ux58ebux4eacux90fdux5f01ux8b77ux58ebux4f1aux3068ux306eux95a2ux4fc2ux60271}{%
\paragraph{学徒出陣の行進曲と、弁護士鉄道の歴史を彩る深澤諭史弁護士、今朝、発掘した2012年のモトケンこと矢部善朗弁護士(京都弁護士会)との関係性(1)}\label{ux5b66ux5f92ux51faux9663ux306eux884cux9032ux66f2ux3068ux5f01ux8b77ux58ebux9244ux9053ux306eux6b74ux53f2ux3092ux5f69ux308bux6df1ux6fa4ux8aedux53f2ux5f01ux8b77ux58ebux4ecaux671dux767aux6398ux3057ux305f2012ux5e74ux306eux30e2ux30c8ux30b1ux30f3ux3053ux3068ux77e2ux90e8ux5584ux6717ux5f01ux8b77ux58ebux4eacux90fdux5f01ux8b77ux58ebux4f1aux3068ux306eux95a2ux4fc2ux60271}}

\begin{itemize}
\tightlist
\item
  〉〉〉 Linux Emacs: 2021/06/06 11:13:56 〉〉〉
\end{itemize}

:CATEGORIES: @kanazawabengosi \#金沢弁護士会 @JFBAsns
日本弁護士連合会(日弁連) \#法務省 @MOJ\_HOUMU \#深澤諭史弁護士
\#モトケンこと矢部善朗弁護士(京都弁護士会) \#被告発人岡田進弁護士

 これも被告発人岡田進弁護士をテーマの中心に据えて記述を進めていきます。根っこにあるのが平成4年の被告発人岡田進弁護士の国選刑事弁護にあるからですが、そういう事情も一切顧みることなく、弁護士以外を批判し続ける妄執性あるいは欺瞞性というのが追求すべき論点です。

 昨夜は思い出せなかったのですが、今朝すぐに思い出したのが学徒出陣の行進曲「抜刀隊」です。曲名はずいぶん前から知っていたのですが、今まで気にならなかった作詞作曲を調べ始めたところ、明治時代のフランス人の作曲だったとのことで、陸軍分列行進曲という曲名も初めて知ったように思います。

 昨日の夕方もやたらと頭に浮かんできたのが、この陸軍分列行進曲の旋律にのせた歌詞で、「われはふかざわ、べんごしだ、こきんむそうの」といったものです。余り真剣に歌詞の内容を考えたことはなかったのですが、かなり前からときどき頭に浮かび上がっていました。

〉〉〉 kk\_hironoのリツイート 〉〉〉

\begin{itemize}
\tightlist
\item
  RT
  kk\_hirono(刑事告発・非常上告_金沢地方検察庁御中)|hirono\_hideki(奉納\さらば弁護士鉄道・泥棒神社の物語)
  日時:2021-06-06 11:29/2021/06/06 10:58 URL:
  \url{https://twitter.com/kk\_hirono/status/1401365486013587461} 
  \url{https://twitter.com/hirono\_hideki/status/1401357687753871361} 
  \textgreater{} 陸軍分列行進曲 - Wikiwand \url{https://t.co/78cC8ixu0X} 
  『陸軍分列行進曲』(りくぐんぶんれつこうしんきょく)は、大日本帝国陸軍の行進曲として作曲・制定され、陸上自衛隊、警察の行進曲として使用されている日本の儀礼曲。
\end{itemize}

〉〉〉 kk\_hironoのリツイート 〉〉〉

\begin{itemize}
\tightlist
\item
  RT
  kk\_hirono(刑事告発・非常上告_金沢地方検察庁御中)|hirono\_hideki(奉納\さらば弁護士鉄道・泥棒神社の物語)
  日時:2021-06-06 11:29/2021/06/06 10:42 URL:
  \url{https://twitter.com/kk\_hirono/status/1401365498957164545} 
  \url{https://twitter.com/hirono\_hideki/status/1401353712811921408} 
  \textgreater{} <軍歌>抜刀隊\_陸軍分列行進曲(全番歌詞付き) - YouTube
  \url{https://t.co/vaIpJE8dF6}  3,748,633 回視聴 •2008/11/28
\end{itemize}

〉〉〉 kk\_hironoのリツイート 〉〉〉

\begin{itemize}
\tightlist
\item
  RT
  kk\_hirono(刑事告発・非常上告_金沢地方検察庁御中)|hirono\_hideki(奉納\さらば弁護士鉄道・泥棒神社の物語)
  日時:2021-06-06 11:29/2021/06/06 10:23 URL:
  \url{https://twitter.com/kk\_hirono/status/1401365540715646979} 
  \url{https://twitter.com/hirono\_hideki/status/1401348994970767361} 
  \textgreater{} 2021-06-06\_10:20
  奉納\\#危険生物・弁護士脳汚染除去装置\\#金沢地方検察庁御中\_2020:
  %@motoken\_tw モトケン%死刑事件の冤罪が目立ちますが、それは徹底的に抵抗するからで、罰金や執行猶予で済む比較的軽い犯罪、少年事件については、表面化していない冤
  \url{https://t.co/wbqVPwdfvO} 
\end{itemize}

〉〉〉 kk\_hironoのリツイート 〉〉〉

\begin{itemize}
\tightlist
\item
  RT
  kk\_hirono(刑事告発・非常上告_金沢地方検察庁御中)|hirono\_hideki(奉納\さらば弁護士鉄道・泥棒神社の物語)
  日時:2021-06-06 11:29/2021/06/06 10:23 URL:
  \url{https://twitter.com/kk\_hirono/status/1401365553269284864} 
  \url{https://twitter.com/hirono\_hideki/status/1401348968504696833} 
  \textgreater{} 2021-06-06\_10:19
  奉納\\#危険生物・弁護士脳汚染除去装置\\#金沢地方検察庁御中\_2020:
  %@fukazawas 深澤諭史%@motoken\_tw¥\n
  人質司法とは→文明国「日本みたいな野蛮国と違って,うちには黙秘や否認の自由があります。」 日本「うちだって,
  \url{https://t.co/fJfVcT6RX6} 
\end{itemize}

〉〉〉 kk\_hironoのリツイート 〉〉〉

\begin{itemize}
\tightlist
\item
  RT
  kk\_hirono(刑事告発・非常上告_金沢地方検察庁御中)|hirono\_hideki(奉納\さらば弁護士鉄道・泥棒神社の物語)
  日時:2021-06-06 11:29/2021/06/06 10:23 URL:
  \url{https://twitter.com/kk\_hirono/status/1401365585930252291} 
  \url{https://twitter.com/hirono\_hideki/status/1401348942126718978} 
  \textgreater{} 2021-06-06\_10:18
  奉納\\#危険生物・弁護士脳汚染除去装置\\#金沢地方検察庁御中\_2020:
  %@motoken\_tw モトケン%裁判官に、裏切られた、という思いを生じさせることが問題なんです。RT
  \url{https://t.co/sj8wWOM2CY} 
\end{itemize}

〉〉〉 kk\_hironoのリツイート 〉〉〉

\begin{itemize}
\tightlist
\item
  RT
  kk\_hirono(刑事告発・非常上告_金沢地方検察庁御中)|hirono\_hideki(奉納\さらば弁護士鉄道・泥棒神社の物語)
  日時:2021-06-06 11:29/2021/06/06 10:23 URL:
  \url{https://twitter.com/kk\_hirono/status/1401365599192637446} 
  \url{https://twitter.com/hirono\_hideki/status/1401348915643904001} 
  \textgreater{} 2021-06-06\_10:17
  奉納\\#危険生物・弁護士脳汚染除去装置\\#金沢地方検察庁御中\_2020:
  %@fukazawas 深澤諭史%\url{https://t.co/z4fufRrQwQ} 
  \url{https://t.co/SFMcsKm1F8} 
\end{itemize}

〉〉〉 kk\_hironoのリツイート 〉〉〉

\begin{itemize}
\tightlist
\item
  RT
  kk\_hirono(刑事告発・非常上告_金沢地方検察庁御中)|hirono\_hideki(奉納\さらば弁護士鉄道・泥棒神社の物語)
  日時:2021-06-06 11:29/2021/06/06 10:23 URL:
  \url{https://twitter.com/kk\_hirono/status/1401365611666501633} 
  \url{https://twitter.com/hirono\_hideki/status/1401348889106534401} 
  \textgreater{} 2021-06-06\_10:14
  奉納\\#危険生物・弁護士脳汚染除去装置\\#金沢地方検察庁御中\_2020:
  %@BarlKarth 高島章%\url{https://t.co/thr8V72TCc}  \url{https://t.co/dTb9zVY5zl} 
\end{itemize}

〉〉〉 kk\_hironoのリツイート 〉〉〉

\begin{itemize}
\tightlist
\item
  RT
  kk\_hirono(刑事告発・非常上告_金沢地方検察庁御中)|hirono\_hideki(奉納\さらば弁護士鉄道・泥棒神社の物語)
  日時:2021-06-06 11:29/2021/06/06 10:21 URL:
  \url{https://twitter.com/kk\_hirono/status/1401365620688506882} 
  \url{https://twitter.com/hirono\_hideki/status/1401348528266305547} 
  \textgreater{} (from:motoken\_tw) until:2012-10-24 since:2012-10-21 -
  Twitter検索 / Twitter \url{https://t.co/mACkB80c9j} 
\end{itemize}

〉〉〉 kk\_hironoのリツイート 〉〉〉

\begin{itemize}
\item
  RT
  kk\_hirono(刑事告発・非常上告_金沢地方検察庁御中)|hirono\_hideki(奉納\さらば弁護士鉄道・泥棒神社の物語)
  日時:2021-06-06 11:29/2021/06/06 10:10 URL:
  \url{https://twitter.com/kk\_hirono/status/1401365631211950085} 
  \url{https://twitter.com/hirono\_hideki/status/1401345722855792642} 
  \textgreater{} (from:fukazawas) until:2012-10-24 since:2012-10-21 -
  Twitter検索 / Twitter \url{https://t.co/yxuobJFQbD} 
\item
  \#告発状 -
  告発\金沢地方検察庁\最高検察庁\法務省\石川県警察御中2020
  \url{https://t.co/EUpMJ5ziuM} 
\end{itemize}

 昨夜投稿に失敗したブログ記事の1つですが、内容を開いてみると、埋め込みツイートに脳機能障害となった少女の事故が自転車事故だとありました。忘れていたのだと思うのですが、被告発人岡田進弁護士との結びつきもいっそう強く感じました。

 「(1)」という番号を見出しにつけました。同じテーマ性を維持した連載化にしたいと思います。

\begin{itemize}
\tightlist
\item
  〈〈〈 2021/06/06 11:38:22 Linux Emacs: 〈〈〈
\end{itemize}

\hypertarget{ux5b66ux5f92ux51faux9663ux306eux884cux9032ux66f2ux3068ux5f01ux8b77ux58ebux9244ux9053ux306eux6b74ux53f2ux3092ux5f69ux308bux6df1ux6fa4ux8aedux53f2ux5f01ux8b77ux58ebux4ecaux671dux767aux6398ux3057ux305f2012ux5e74ux306eux30e2ux30c8ux30b1ux30f3ux3053ux3068ux77e2ux90e8ux5584ux6717ux5f01ux8b77ux58ebux4eacux90fdux5f01ux8b77ux58ebux4f1aux3068ux306eux95a2ux4fc2ux60272}{%
\paragraph{学徒出陣の行進曲と、弁護士鉄道の歴史を彩る深澤諭史弁護士、今朝、発掘した2012年のモトケンこと矢部善朗弁護士(京都弁護士会)との関係性(2)}\label{ux5b66ux5f92ux51faux9663ux306eux884cux9032ux66f2ux3068ux5f01ux8b77ux58ebux9244ux9053ux306eux6b74ux53f2ux3092ux5f69ux308bux6df1ux6fa4ux8aedux53f2ux5f01ux8b77ux58ebux4ecaux671dux767aux6398ux3057ux305f2012ux5e74ux306eux30e2ux30c8ux30b1ux30f3ux3053ux3068ux77e2ux90e8ux5584ux6717ux5f01ux8b77ux58ebux4eacux90fdux5f01ux8b77ux58ebux4f1aux3068ux306eux95a2ux4fc2ux60272}}

\begin{itemize}
\tightlist
\item
  〉〉〉 Linux Emacs: 2021/06/06 11:54:26 〉〉〉
\end{itemize}

:CATEGORIES: @kanazawabengosi \#金沢弁護士会 @JFBAsns
日本弁護士連合会(日弁連) \#法務省 @MOJ\_HOUMU \#深澤諭史弁護士
\#モトケンこと矢部善朗弁護士(京都弁護士会) \#被告発人岡田進弁護士

\begin{itemize}
\tightlist
\item
  1402:2021-06-06\_11:39:28 \#告発状 \#\#\#\#
  学徒出陣の行進曲と、弁護士鉄道の歴史を彩る深澤諭史弁護士、今朝、発掘した2012年のモトケンこと矢部善朗弁護士(京都弁護士会)との関係性(1)
  \url{https://hirono-hideki.hatenadiary.jp/entry/2021/06/06/113925} 
\end{itemize}

 いくつか深澤諭史弁護士関連のまとめ記事を作成しました。こういう単独アカウントのまとめは出来上がりも早いです。

\begin{itemize}
\tightlist
\item
  2021年06月06日11時41分の登録:
  REGEXP:''学徒''/深澤諭史(@fukazawas)の検索(2016-04-11〜2020-10-25/2021年06月06日11時41分の記録10件)
  \url{https://kk2020-09.blogspot.com/2021/06/regexpfukazawas2016-04-112020-10.html} 
\item
  2021年06月06日11時42分の登録:
  REGEXP:''インパール''/深澤諭史(@fukazawas)の検索(2013-06-27〜2021-02-02/2021年06月06日11時42分の記録77件)
  \url{https://kk2020-09.blogspot.com/2021/06/regexpfukazawas2013-06-272021-02.html} 
\item
  2021年06月06日11時43分の登録:
  REGEXP:''バスター''/深澤諭史(@fukazawas)の検索(2014-06-06〜2020-12-12/2021年06月06日11時43分の記録12件)
  \url{https://kk2020-09.blogspot.com/2021/06/regexpfukazawas2014-06-062020-12.html} 
\item
  2021年06月06日11時44分の登録:
  REGEXP:''利権''/深澤諭史(@fukazawas)の検索(2012-12-26〜2021-02-07/2021年06月06日11時44分の記録460件)
  \url{https://kk2020-09.blogspot.com/2021/06/regexpfukazawas2012-12-262021-02.html} 
\item
  2021年06月06日11時44分の登録:
  REGEXP:''ロー戦争''/深澤諭史(@fukazawas)の検索(2015-03-12〜2021-03-27/2021年06月06日11時44分の記録33件)
  \url{https://kk2020-09.blogspot.com/2021/06/regexpfukazawas2015-03-122021-03.html} 
\item
  2021年06月06日11時44分の登録:
  REGEXP:''消費者被害''/深澤諭史(@fukazawas)の検索(2013-07-17〜2021-03-13/2021年06月06日11時43分の記録66件)
  \url{https://kk2020-09.blogspot.com/2021/06/regexpfukazawas2013-07-172021-03.html} 
\item
  2021年06月06日11時45分の登録:
  REGEXP:''非弁''/深澤諭史(@fukazawas)の検索(2012-09-08〜2021-04-16/2021年06月06日11時45分の記録2137件)
  \url{https://kk2020-09.blogspot.com/2021/06/regexpfukazawas2012-09-082021-04.html} 
\item
  2021年06月06日11時46分の登録:
  REGEXP:''ハンター''/深澤諭史(@fukazawas)の検索(2015-10-05〜2021-03-30/2021年06月06日11時45分の記録13件)
  \url{https://kk2020-09.blogspot.com/2021/06/regexpfukazawas2015-10-052021-03.html} 
\end{itemize}

\begin{quote}
《引用の始まり》
\end{quote}

\begin{quote}
本日、京都大学に入学された皆さんのすべてが、この6月から選挙に参加できるようになりました。これまで20歳以上の国民に与えられていた選挙権が、公職選挙法の改正により18歳まで引き下げられたのです。皆さんは自分の置かれている環境に対し、その是非について、その政治的判断について、自ら票を投じて参加できるようになったのです。それはとても大きな変化だと思います。京都大学の時計台にある迎賓室には、「学徒出陣図」と題する須田国太郎画伯の絵が掛けられています。須田は京都大学文学部を卒業後、絵画の道を志してヨーロッパに学び、京都大学の学生が召集されて出陣する場面を描きました。それは1943年11月20日のことで、快晴の比叡山を遠望して学生たちが行進するさまを、須田は実に暗い色調で描きました。この戦争で京都大学から4,500名に上る学生が入隊し、文系の学生はそのうち9割近くを占めました。264名の学生が戦没者として確認されています。当時、選挙権は25歳以上の男子と定められており、多くの大学生には政治に参加する資格が与えられていませんでした。20歳以上の男女に選挙権が与えられたのは戦後1946年であり、日本国憲法が公布されたのはその後のことです。学徒出陣に参加した学生たちは自分たちの意思ではなく、上の世代の決定によって戦争に駆り出されていたのです。このことはしっかりと心に留めておかねばなりません。皆さんはこの6月から選挙に参加するとともに、日本の政治の方向性について大きな責任も生じるということを忘れないでください。皆さんの意思によって、揺るぎなき未来を築くために確かな一票を投じてください。
\end{quote}

\begin{quote}
《引用の終わり》
\end{quote}

\begin{itemize}
\tightlist
\item
  平成28年度学部入学式 式辞 (2016年4月7日) \textbar{} 京都大学
  \url{https://www.kyoto-u.ac.jp/ja/about/history/successive/president26/speech/2016/160407-1} 
\end{itemize}

 私もはてなダイアリーをメインに使った平成31年3月より前の頃は、なかなか改行を入れない文章を作成し、拘置所や刑務所で作成した手書きの書面では、文字を罫線に埋め尽くし加筆の余地を入れないようにもしていました。

\begin{itemize}
\tightlist
\item
  京大入学式 学徒出陣にふれた総長/``未来のため確かな一票を''
  \url{https://t.co/F9nuXjmSWI}  しんぶん赤旗 日本共産党
\end{itemize}

 ページタイトルになかった「しんぶん赤旗 日本共産党」を私が追記しましたが、未来へとつながる歴史的な発見にも思えました。

\begin{quote}
《引用の始まり》
\end{quote}

\begin{quote}
雨の神宮外苑で、ペンを銃に持ち替えゲートルを巻き、行進しているまだあどけない面影が残る学生たちが、不条理な死へ追い立てられていく当時の報道を思い出します。多くの母たちが涙を流したことでしょう。いまでもこの映像には涙がでます。

「平和と安全」の言葉にくるまれた戦争への道が動き始めています。しっかり見抜いて人を殺し、殺される道に前途ある若者を送り出さないよう、みなさん手をつないぎましょう。
\end{quote}

\begin{quote}
《引用の終わり》
\end{quote}

\begin{itemize}
\tightlist
\item
  学徒出陣にふれた京大入学式~総長の言葉 !(\^{}\^{})!:
  上野とき子の街かどスケッチ
  \url{https://ueno-tokico.at.webry.info/201604/article\_8.html} 
\end{itemize}

〉〉〉 kk\_hironoのリツイート 〉〉〉

\begin{itemize}
\item
  RT
  kk\_hirono(刑事告発・非常上告_金沢地方検察庁御中)|hiranok(平野啓一郎)
  日時:2021-06-06 12:09/2016/04/08 17:14 URL:
  \url{https://twitter.com/kk\_hirono/status/1401375731993567233} 
  \url{https://twitter.com/hiranok/status/718351347914027008} 
  \textgreater{}
  京大はそうじゃないとね。須田国太郎「学徒出陣壮行の図」(京大所蔵)\url{https://t.co/VwARiAHOSo} 
  【京大:18歳選挙権、大きな責任 入学式で学長】 - 毎日新聞
  \url{https://t.co/fbgrKfNfcO} 
\item
  須田国太郎「京都大学学生出陣壮行式」 - 北仁人のたわ言
  \url{https://t.co/sPIwavvGbe} 
  。その中で彼は京都大学が所蔵するこの絵を取り上げ、学生に政治的覚醒を訴えている(式辞全文は京都大学のホームページを参照)。
\item
  (01/10) RT fukazawas(深澤諭史)|1961kumachin(中村元弥)
  日時:2016-04-11 02:16:00 +0900/2016-04-11 02:15:00 +0900 URL:
  \url{https://twitter.com/fukazawas/status/719348428921131008} 
  \url{https://twitter.com/1961kumachin/status/719348070987669504\textgreater} {}
  京都大学法学部入学式の式辞。18歳選挙権と学徒出陣のくだり(最後の方)必読 \url{https://t.co/ttcSJgkfQk} 
\item
  (02/10) TW fukazawas(深澤諭史) 日時:2016-04-11 02:17:00 +0900
  URL:
  \url{https://twitter.com/fukazawas/status/719348639282298880\textgreater} {}
  京都大学 平成28年度学部入学式 式辞 (2016年4月7日)\textgreater{}
  \url{https://t.co/hS275J0ya9\textgreater} {}
  「学徒出陣に参加した学生たちは自分たちの意思ではなく、上の世代の決定によって戦争に駆り出されていたのです。このことはしっかりと心に留めておかねばなりません。」
\item
  (04/10) TW fukazawas(深澤諭史) 日時:2016-12-30 10:44:00 +0900
  URL:
  \url{https://twitter.com/fukazawas/status/814648274405638144\textgreater} {}
  一応,非法曹・非法学徒の方々もいらっしゃると思うので,簡単に解説。\textgreater{}
  記憶だけで書いているので,問題あったら容赦なく訂正して下さい(笑)\textgreater{}
  刑事訴訟法317条は,「事実の認定は、証拠による。」によると定めている,これは,大きく分けて2つの意味があります。
\item
  (10/10) TW fukazawas(深澤諭史) 日時:2020-10-25 22:45:00 +0900
  URL:
  \url{https://twitter.com/fukazawas/status/1320360730864939008\textgreater} {}
  芦部信喜 平和への憲法学 - 岩波書店
  \url{https://t.co/SCC0uZVVHf\textgreater} {}
  『「総理、芦部信喜さんという憲法学者、ご存じですか」「私は存じ上げておりません」(国会答弁より)。戦後憲法学のスタンダードをつくったのは、どんな人物だったのか。学徒出陣の戦争体験。実際の裁判にも関与』
\end{itemize}

 そういえば、深澤諭史弁護士もですが、憲法について独自のこだわりを格調高く謳い上げていました。

\begin{itemize}
\tightlist
\item
  (02/77) TW fukazawas(深澤諭史) 日時:2013-07-30 22:45:00 +0900
  URL:
  \url{https://twitter.com/fukazawas/status/362207296195018753\textgreater} {}
  旧日本軍軍人と、これからの法曹志望者に共通して求められる素質。\textgreater{}
  「食えるかどうかを心配しないこと」\textgreater{}
  私としては、依頼者に「食えるかどうか心配するな!」なんていわないか、心配です。インパール作戦みたいな事件処理方針立てられたら、依頼者は、いい迷惑ですから\textgreater{}
  (・∀・;)
\end{itemize}

 これ1つでも紹介するには十分かもしれません。まとめ記事にはほかにもごろごろと転がっているはずです。

 「インパール作戦みたいな事件処理方針立てられたら、依頼者は、いい迷惑ですから」という珍しい組み合わせでの発見となりましたが、この依頼者というのも深澤諭史弁護士が独自のこだわり、世界観をのぞかせていました。文化的芸術性を感じる弁護士鉄道の記録資料です。

\begin{itemize}
\item
  (01/12) TW fukazawas(深澤諭史) 日時:2014-06-06 16:56:00 +0900
  URL:
  \url{https://twitter.com/fukazawas/status/474822199044345856\textgreater} {}
  誰か,日弁連非弁バスターズ作りませんか(・∀・*)\textgreater{}
  提案しましょう・・・(・∀・)
\item
  (05/12) TW fukazawas(深澤諭史) 日時:2019-06-25 14:09:00 +0900
  URL:
  \url{https://twitter.com/fukazawas/status/1143385768963932160\textgreater} {}
  (・∀・)そういや、来月の全国非弁バスターの宴、法クラで来る人いるのかな。\textgreater{}
  (^ω^)私は出ますお。ある件の打ち合わせもあるし。
\end{itemize}

 時刻は12時34分です。さきほど能登町の告知放送があって、ツキノワグマの目撃情報でした。少し聞き取りづらかったのですが、真脇トンネル付近と言っていたように思います。

 この深澤諭史弁護士の宴については令和3年3月31日付告発状でも取り上げたと思うのですが、輪島市の皆月に小さい頃に泊まったような気がする気になる旅館の建物があったのですが、あとになって近くに皆月望楼という明治時代の軍事施設があったことを知りました。

\begin{itemize}
\tightlist
\item
  〈〈〈 2021/06/06 12:40:05 Linux Emacs: 〈〈〈
\end{itemize}

\hypertarget{ux5b66ux5f92ux51faux9663ux306eux884cux9032ux66f2ux3068ux5f01ux8b77ux58ebux9244ux9053ux306eux6b74ux53f2ux3092ux5f69ux308bux6df1ux6fa4ux8aedux53f2ux5f01ux8b77ux58ebux4ecaux671dux767aux6398ux3057ux305f2012ux5e74ux306eux30e2ux30c8ux30b1ux30f3ux3053ux3068ux77e2ux90e8ux5584ux6717ux5f01ux8b77ux58ebux4eacux90fdux5f01ux8b77ux58ebux4f1aux3068ux306eux95a2ux4fc2ux60273}{%
\paragraph{学徒出陣の行進曲と、弁護士鉄道の歴史を彩る深澤諭史弁護士、今朝、発掘した2012年のモトケンこと矢部善朗弁護士(京都弁護士会)との関係性(3)}\label{ux5b66ux5f92ux51faux9663ux306eux884cux9032ux66f2ux3068ux5f01ux8b77ux58ebux9244ux9053ux306eux6b74ux53f2ux3092ux5f69ux308bux6df1ux6fa4ux8aedux53f2ux5f01ux8b77ux58ebux4ecaux671dux767aux6398ux3057ux305f2012ux5e74ux306eux30e2ux30c8ux30b1ux30f3ux3053ux3068ux77e2ux90e8ux5584ux6717ux5f01ux8b77ux58ebux4eacux90fdux5f01ux8b77ux58ebux4f1aux3068ux306eux95a2ux4fc2ux60273}}

\begin{itemize}
\tightlist
\item
  〉〉〉 Linux Emacs: 2021/06/06 12:41:19 〉〉〉
\end{itemize}

:CATEGORIES: @kanazawabengosi \#金沢弁護士会 @JFBAsns
日本弁護士連合会(日弁連) \#法務省 @MOJ\_HOUMU \#深澤諭史弁護士
\#モトケンこと矢部善朗弁護士(京都弁護士会) \#被告発人岡田進弁護士

\begin{itemize}
\tightlist
\item
  1402:2021-06-06\_11:39:28 \#告発状 \#\#\#\#
  学徒出陣の行進曲と、弁護士鉄道の歴史を彩る深澤諭史弁護士、今朝、発掘した2012年のモトケンこと矢部善朗弁護士(京都弁護士会)との関係性(1)
  \url{https://hirono-hideki.hatenadiary.jp/entry/2021/06/06/113925} 
\item
  1403:2021-06-06\_12:40:42 \#告発状 \#\#\#\#
  学徒出陣の行進曲と、弁護士鉄道の歴史を彩る深澤諭史弁護士、今朝、発掘した2012年のモトケンこと矢部善朗弁護士(京都弁護士会)との関係性(2)
  \url{https://hirono-hideki.hatenadiary.jp/entry/2021/06/06/124040} 
\end{itemize}

 まず、Googleで皆月の旅館から調べてみたいと思います。前を通ったときは営業をやっているように見えたのですが、ネットで調べるのも今回が初めてになりそうです。同じ場所に旅館があったとも限らないですが近くには皆月湾が見える旅館があって、明治時代、軍人が酒宴を開いていたようにも想像します。

\begin{itemize}
\tightlist
\item
  川島旅館 - Google マップ \url{https://t.co/LYwXzFurQz} 
\end{itemize}

 隣に見える建物に名前があり、海に近い探していた旅館のような建物は名前がなく廃業しているのかと思っていました。名前のある建物は旅館とは思えなかったのですが、よく見ると「川島旅館」とありました。

 Googleで検索しても情報はなさそうです。2015年にはまだ営業していたような情報があり、素泊まりで7500円からとなっていました。ストリートビューで建物を見ると、印象も違い外壁が新しい建物にもみえたのですが、古い歴史がありそうな旅館に思えていました。

\begin{itemize}
\tightlist
\item
  川島旅館 - 能登空港から車で35分の旅館 7500円より 素泊まり可
  皆月バス停前(石川県) - 【デゴイチよく走る!】SL沿線宿泊情報
  \url{https://t.co/ZZMkuASvDb} 
\end{itemize}

 リンクのページを開くと、参考料金が7,500円で、「素泊まり可」となっていました。意外に思ったのは客室の数で3室とあります。1室が広そうにも思えるのですが、もともと旅館というより宴会場というイメージがあり、夜に行ったことはないですが、皆月湾に月見ができそうだと思っていました。

 隣の集落からずいぶんと離れているのも、この皆月の大きな特徴です。猿山岬に行くときにとおる道でした。猿山岬灯台は、明治時代に200メートル崖下の海から資材を運び上げて建設されたという話です。

 深澤諭史弁護士もまた、悪気なく、社会に灯台のような光を与えている救世主と思い込んでいるふしがあります。万能感に溢れたツイートも数多くあるのですが、まるで資料館の展示場のようなTwitterのタイムラインでした。最近は更新が激減し、しばらく前から一日1件のリツイートというペースです。

 私の個人的な体験に照らせば、深澤諭史弁護士灯台と命名したいぐらいです。猿山岬は宇出津小学校のバス遠足で行った記憶がかすかに残っているのですが、そのたぶん昭和40年代の後半辺りから重なって見えるのが、私の母親の人生であり、決定的な打撃がまず被告発人岡田進弁護士にありました。

 昨日は、アカウントの削除で消滅した北周士弁護士のツイートを紹介していましたが、まとめ記事でも数が少なかったのですぐに見つかると思います。

\begin{itemize}
\item
  奉納\危険生物・弁護士脳汚染除去装置\金沢地方検察庁御中\_2020:
  REGEXP:''非弁絶対殺すマン''/データベース登録済みツイートの検索:2016-07-12〜2017-01-31/2021年06月05日23時54分の記録:ユーザ・投稿:4/9件
  \url{https://t.co/tlnfCYACNg} 
\item
  (2/9) TW @Noooooooorth(キュアノースライム) 日時: 2016-08-31
  19:45:00 +0900 URL(削除されたツイート):
  \url{https://twitter.com/Noooooooorth/status/770935523326119936\textgreater} {}
  ウィスキー\textgreater{} 変人(婉曲表現)\textgreater{} 非弁絶対殺すマン
  \url{https://t.co/SJBmq6SQHS} 
\end{itemize}

 書式を変えておかないとあとでエラーが発生するので、昨夜は面倒だと思い、そのまま転載していませんでした。

 アカウント名が@Noooooooorthとなっています。これをツイートすれば、リンクができるはずなので、そのまま確認できそうですが、今もあるはずの復活転生した北周士弁護士のアカウント名も見た目に区別のつかないものでした。

〉〉〉 kk\_hironoのリツイート 〉〉〉

\begin{itemize}
\tightlist
\item
  RT
  kk\_hirono(刑事告発・非常上告_金沢地方検察庁御中)|noooooooorth(教皇ノースライム)
  日時:2021-06-06 13:18/2021/06/06 13:05 URL:
  \url{https://twitter.com/kk\_hirono/status/1401393032730083331} 
  \url{https://twitter.com/noooooooorth/status/1401389838683701256} 
  \textgreater{}
  世界が広がると金がかかるというか、世界を広げようとするときに金がかかる。
\end{itemize}

 前から気になっていたのですが、リンクが開けました。他にもいくつか確認しているのですが、他ではブロックされて告発\市場急配センター殺人未遂事件\金沢地方検察庁・石川県警察御中(@kk\_hirono)ではブロックされていないアカウントにもなります。

\begin{itemize}
\tightlist
\item
  2021年06月06日13時21分の登録:
  「@noooooooorth」を@hirono\_hideki @kk\_hirono @s\_hironoで検索 1069件の該当 2021-06-06\_13:21の記録
  \url{https://kk2020-09.blogspot.com/2021/06/noooooooorthhironohidekikkhironoshirono.html} 
\end{itemize}

2017-06-24 20:34:41
``2017-06-24-203436\_ノースライム(@noooooooorth)さん | Twitterからの返信付きツイート.jpg
\url{http://pic.twitter.com/hIgnapGBBG''} 
\url{https://twitter.com/s\_hirono/status/878577112361205760} 

 最初の記録となっていますが、2017年6月24日です。そもそも北周士弁護士のTwitterアカウントを最初に知った時期が思い出せないのですが、2015年には見ていたように思います。その2015年の秋というのは、弁護士と妖怪の関係性について最も強く意識した時期でした。

 妖怪ウォッチの「コマさん」について知ったのがその時期でした。最初は「法たろう」とかいうプロフィールの名前のTwitterアカウントで、アイコンにイラストをみていたのですが、しばらくして妖怪ウォッチのキャラクターだと知ったのです。

※ @kk\_hironoのアカウントがブロックされ,リツイートに失敗したツイート

\begin{itemize}
\tightlist
\item
  TW lawyerhotaro(ほうたろう) 日時:2021/06/04 09:04:42 URL:
  \url{https://twitter.com/lawyerhotaro/status/1400604385479073798} 
  \textgreater{}
  民主主義社会では無能は悪徳ではない。同様に、生まれつき能力が高いことも何の美徳でもない。誇るようなことでもない。
\end{itemize}

 何度もプロフィールの名前やアイコンも変わっているのですが、基本、このアカウントだと思います。「法たろう」ではなく、現時点で「ほうたろう」となっていましたが、違ったものになっていた時期が長かったようにも思えます。これはまとめ記事を作成することで確認できそうです。

\begin{itemize}
\tightlist
\item
  〈〈〈 2021/06/06 13:32:45 Linux Emacs: 〈〈〈
\end{itemize}

\hypertarget{ux5b66ux5f92ux51faux9663ux306eux884cux9032ux66f2ux3068ux5f01ux8b77ux58ebux9244ux9053ux306eux6b74ux53f2ux3092ux5f69ux308bux6df1ux6fa4ux8aedux53f2ux5f01ux8b77ux58ebux4ecaux671dux767aux6398ux3057ux305f2012ux5e74ux306eux30e2ux30c8ux30b1ux30f3ux3053ux3068ux77e2ux90e8ux5584ux6717ux5f01ux8b77ux58ebux4eacux90fdux5f01ux8b77ux58ebux4f1aux3068ux306eux95a2ux4fc2ux60274}{%
\paragraph{学徒出陣の行進曲と、弁護士鉄道の歴史を彩る深澤諭史弁護士、今朝、発掘した2012年のモトケンこと矢部善朗弁護士(京都弁護士会)との関係性(4)}\label{ux5b66ux5f92ux51faux9663ux306eux884cux9032ux66f2ux3068ux5f01ux8b77ux58ebux9244ux9053ux306eux6b74ux53f2ux3092ux5f69ux308bux6df1ux6fa4ux8aedux53f2ux5f01ux8b77ux58ebux4ecaux671dux767aux6398ux3057ux305f2012ux5e74ux306eux30e2ux30c8ux30b1ux30f3ux3053ux3068ux77e2ux90e8ux5584ux6717ux5f01ux8b77ux58ebux4eacux90fdux5f01ux8b77ux58ebux4f1aux3068ux306eux95a2ux4fc2ux60274}}

\begin{itemize}
\tightlist
\item
  〉〉〉 Linux Emacs: 2021/06/06 14:15:59 〉〉〉
\end{itemize}

:CATEGORIES: @kanazawabengosi \#金沢弁護士会 @JFBAsns
日本弁護士連合会(日弁連) \#法務省 @MOJ\_HOUMU \#深澤諭史弁護士
\#モトケンこと矢部善朗弁護士(京都弁護士会) \#被告発人岡田進弁護士

\begin{itemize}
\tightlist
\item
  1404:2021-06-06\_13:33:16 \#告発状 \#\#\#\#
  学徒出陣の行進曲と、弁護士鉄道の歴史を彩る深澤諭史弁護士、今朝、発掘した2012年のモトケンこと矢部善朗弁護士(京都弁護士会)との関係性(3)
  \url{https://hirono-hideki.hatenadiary.jp/entry/2021/06/06/133314} 
\end{itemize}

 予定を変更し「学徒出陣の行進曲と、弁護士鉄道の歴史を彩る深澤諭史弁護士、今朝、発掘した2012年のモトケンこと矢部善朗弁護士(京都弁護士会)との関係性」という連載は、この4で切り上げたいと思います。紀行という言葉を思い出し調べたのですが、新しいスタイルを採りたいと思います。

\begin{itemize}
\tightlist
\item
  奉納\危険生物・弁護士脳汚染除去装置\金沢地方検察庁御中\_2020:
  Ex-REGEXP:''@fukazawas''/データベース登録済みツイートの検索:2012-10-23〜2021-06-04/2021年06月05日21時14分の記録:ユーザ・投稿:567/8755件
  \url{https://t.co/MDaaTCzGvX} 
\end{itemize}

 記録の作成時刻が昨夜の21時14分となっています。私の3つのTwitterアカウントを除外したまとめなのですが、ユーザ数が567件で、ツイートの数が8755件となっています。ツイートの数はもう少しありそうなぐらいでしたが、ユーザというかアカウントの数が予想より多く567件でした。

 その最初に記録されていたツイートを朝に確認しました。意外に遅く2012年となっていたのですが、これが思わぬ発見につながりました。

\begin{itemize}
\item
  (0001/8755) TW @motoken\_tw(モトケン) 日時: 2012-10-23 22:05:00
  +0900 URL:
  \url{https://twitter.com/motoken\_tw/status/260728657084022785\textgreater} {}
  裁判官に、裏切られた、という思いを生じさせることが問題なんです。RT
  @fukazawas: @motoken\_tw @taniyama @47news
  逃亡も問題ですし,刑罰法規の迅速な適用というのも重大な利益です。ただ,逃げても捕まえれば,ある程度の取り返しはつくでしょうが
\item
  TW hirono\_hideki(奉納\さらば弁護士鉄道・泥棒神社の物語) 日時:
  2021/06/06 10:10:30 URL:
  \url{https://twitter.com/hirono\_hideki/status/1401345722855792642} 
  \textgreater{} (from:fukazawas) until:2012-10-24 since:2012-10-21 -
  Twitter検索 / Twitter \url{https://t.co/yxuobJFQbD} 
\end{itemize}

 期間を指定したTwitterの検索になりますが、範囲を広めにしているのは、日本時間以外で検索されたような経験が過去にあり、ツイートを見つけるのに手間取ったことがあったからです。この検索でけっこうな発見がありました。弁護士鉄道の一時期を彩る特色があります。

〉〉〉 kk\_hironoのリツイート 〉〉〉

\begin{itemize}
\tightlist
\item
  RT
  kk\_hirono(刑事告発・非常上告_金沢地方検察庁御中)|BarlKarth(高島章)
  日時:2021-06-06 14:59/2012/10/22 22:22 URL:
  \url{https://twitter.com/kk\_hirono/status/1401418360705802245} 
  \url{https://twitter.com/BarlKarth/status/260370532413890563} 
  \textgreater{} @fukazawas
  これ以上の話題はフェイスブック,刑事訴訟愛好会でお願いしますです。
\end{itemize}

〉〉〉 kk\_hironoのリツイート 〉〉〉

\begin{itemize}
\item
  RT
  kk\_hirono(刑事告発・非常上告_金沢地方検察庁御中)|BarlKarth(高島章)
  日時:2021-06-06 15:00/2012/10/22 21:55 URL:
  \url{https://twitter.com/kk\_hirono/status/1401418650741968906} 
  \url{https://twitter.com/BarlKarth/status/260363802338545664} 
  \textgreater{}
  PがP面を弾劾証拠で証拠請求して来た(法廷供述を弾劾する趣旨)。当該証拠は,実質証拠として被告人に有利に使える記載があったので「同意するから,同意書面として証拠請求してちょ」とPに述べたら,Pはずいぶん困っていた。これ以上の話はFBの刑事訴訟愛好会で続く。
\item
  〉〉〉 アカウント(@motoken\_tw)は,@kk\_hironoをブロックしています。リツイートできませんでした。
  〉〉〉 ¥\n ¥\n \url{https://t.co/MFQDAH9EXR} 
\item
  〉〉〉 アカウント(@fukazawas)は,@kk\_hironoをブロックしています。リツイートできませんでした。
  〉〉〉 ¥\n ¥\n \url{https://t.co/dDX4tjwsot} 
\item
  〉〉〉 アカウント(@motoken\_tw)は,@kk\_hironoをブロックしています。リツイートできませんでした。
  〉〉〉 ¥\n ¥\n \url{https://t.co/wwonx9vttD} 
\item
  〉〉〉 アカウント(@fukazawas)は,@kk\_hironoをブロックしています。リツイートできませんでした。
  〉〉〉 ¥\n ¥\n \url{https://t.co/MBPKBFMSzy} 
\item
  〉〉〉 アカウント(@fukazawas)は,@kk\_hironoをブロックしています。リツイートできませんでした。
  〉〉〉 ¥\n ¥\n \url{https://t.co/SbZpgN8gft} 
\item
  〉〉〉 アカウント(@fukazawas)は,@kk\_hironoをブロックしています。リツイートできませんでした。
  〉〉〉 ¥\n ¥\n \url{https://t.co/go29qS0RY6} 
\item
  〉〉〉 アカウント(@fukazawas)は,@kk\_hironoをブロックしています。リツイートできませんでした。
  〉〉〉 ¥\n ¥\n \url{https://t.co/ykLRx2ybCU} 
\item
  〉〉〉 アカウント(@yjochi)は,@kk\_hironoをブロックしています。リツイートできませんでした。
  〉〉〉 ¥\n ¥\n \url{https://t.co/i5aCOhXfDT} 
\item
  〉〉〉 アカウント(@fukazawas)は,@kk\_hironoをブロックしています。リツイートできませんでした。
  〉〉〉 ¥\n ¥\n \url{https://t.co/MzPoJYeBeo} 
\item
  〉〉〉 アカウント(@fukazawas)は,@kk\_hironoをブロックしています。リツイートできませんでした。
  〉〉〉 ¥\n ¥\n \url{https://t.co/nw8YTaI3w2} 
\item
  〉〉〉 アカウント(@motoken\_tw)は,@kk\_hironoをブロックしています。リツイートできませんでした。
  〉〉〉 ¥\n ¥\n \url{https://t.co/MJHrClLAtw} 
\item
  〉〉〉 アカウント(@fukazawas)は,@kk\_hironoをブロックしています。リツイートできませんでした。
  〉〉〉 ¥\n ¥\n \url{https://t.co/NPqrvFoNmj} 
\end{itemize}

※ @kk\_hironoのアカウントがブロックされ,リツイートに失敗したツイート

\begin{itemize}
\tightlist
\item
  TW motoken\_tw(モトケン) 日時:2012/10/22 08:55:21 URL:
  \url{https://twitter.com/motoken\_tw/status/260167419128074240} 
  \textgreater{}
  死刑事件の冤罪が目立ちますが、それは徹底的に抵抗するからで、罰金や執行猶予で済む比較的軽い犯罪、少年事件については、表面化していない冤罪がとてもたくさんあると思っています。その最大の理由が人質司法であり、そのA級戦犯が裁判官なんです。
\end{itemize}

※ @kk\_hironoのアカウントがブロックされ,リツイートに失敗したツイート

\begin{itemize}
\tightlist
\item
  TW fukazawas(深澤諭史) 日時:2012/10/22 11:08:37 URL:
  \url{https://twitter.com/fukazawas/status/260200960490602496} 
  \textgreater{} @motoken\_tw
  人質司法とは→文明国「日本みたいな野蛮国と違って,うちには黙秘や否認の自由があります。」 日本「うちだって,黙秘や否認は自由ですよ。」 文明国「そうですか。我が国には,黙秘や否認をした後にも自由があるのですが。」
\end{itemize}

※ @kk\_hironoのアカウントがブロックされ,リツイートに失敗したツイート

\begin{itemize}
\tightlist
\item
  TW motoken\_tw(モトケン) 日時:2012/10/23 21:58:19 URL:
  \url{https://twitter.com/motoken\_tw/status/260726846797275136} 
  \textgreater{}
  人質司法になる原因の一つではあるのだが、羹に懲りてなますを吹く愚を犯さないで欲しい。RT
  @taniyama: こういうのがあると身柄判断が厳しくなる。 RT @47news:
  入院で勾留停止の被告が逃走 大阪、1カ月後確保 \url{http://t.co/ezBwWsSt} 
\end{itemize}

※ @kk\_hironoのアカウントがブロックされ,リツイートに失敗したツイート

\begin{itemize}
\tightlist
\item
  TW fukazawas(深澤諭史) 日時:2012/10/23 22:03:17 URL:
  \url{https://twitter.com/fukazawas/status/260728099346472960} 
  \textgreater{} @motoken\_tw @taniyama @47news
  逃亡も問題ですし,刑罰法規の迅速な適用というのも重大な利益です。ただ,逃げても捕まえれば,ある程度の取り返しはつくでしょうが,誤認逮捕や長期勾留は,そのあとにいくら釈放しても,日本社会では取り返しがつかないのです。
\end{itemize}

※ @kk\_hironoのアカウントがブロックされ,リツイートに失敗したツイート

\begin{itemize}
\tightlist
\item
  TW fukazawas(深澤諭史) 日時:2012/10/23 16:19:13 URL:
  \url{https://twitter.com/fukazawas/status/260641512336003072} 
  \textgreater{}
  訴状を書いて,証拠を整理していて思うのですが,民事訴訟では刑事訴訟みたいに,原告が被告を自分の手元に閉じ込めて,「請求原因事実を認めれば家に帰してやる。和解で2割引いてやる。さあ認めろ」とかできないから大変です。
\end{itemize}

※ @kk\_hironoのアカウントがブロックされ,リツイートに失敗したツイート

\begin{itemize}
\tightlist
\item
  TW fukazawas(深澤諭史) 日時:2012/10/23 14:39:45 URL:
  \url{https://twitter.com/fukazawas/status/260616481073684481} 
  \textgreater{} @nosaibaninko @ustht
  非難もそうなのですが,刑訴法の改正はもはや時代の要請だと思います。検察・警察の不適切な捜査は問題ですが,より大きな問題は,一方当事者が不適切に振る舞った瞬間に,えん罪という結果が簡単に生じてしまう刑訴法です。
\end{itemize}

※ @kk\_hironoのアカウントがブロックされ,リツイートに失敗したツイート

\begin{itemize}
\tightlist
\item
  TW fukazawas(深澤諭史) 日時:2012/10/22 22:23:57 URL:
  \url{https://twitter.com/fukazawas/status/260370912979857408} 
  \textgreater{} @BarlKarth 了解です。
\end{itemize}

※ @kk\_hironoのアカウントがブロックされ,リツイートに失敗したツイート

\begin{itemize}
\tightlist
\item
  TW
  yjochi(弁護士落合洋司🌸感染拡大を招く東京(頭狂)オリンピック中止!🌸)
  日時:2012/10/22 13:54:30 URL:
  \url{https://twitter.com/yjochi/status/260242704607371264} 
  \textgreater{}
  野田内閣の支持率を2で割ると、ほぼ、平清盛の視聴率、か。わかりやすい。
\end{itemize}

※ @kk\_hironoのアカウントがブロックされ,リツイートに失敗したツイート

\begin{itemize}
\tightlist
\item
  TW fukazawas(深澤諭史) 日時:2012/10/22 13:58:27 URL:
  \url{https://twitter.com/fukazawas/status/260243698590310402} 
  \textgreater{} @yjochi
  最終回も同じ時期になりそう?ですね。ドラマはこれから見せ場がありそうですが,内閣にはあるのでしょうか。
\end{itemize}

※ @kk\_hironoのアカウントがブロックされ,リツイートに失敗したツイート

\begin{itemize}
\tightlist
\item
  TW fukazawas(深澤諭史) 日時:2012/10/22 22:03:31 URL:
  \url{https://twitter.com/fukazawas/status/260365769840414720} 
  \textgreater{} @BarlKarth
  ちょっと話がそれますが,開示された被疑者のP面を証拠請求されたら,不同意にされた事ってありますか?
\end{itemize}

※ @kk\_hironoのアカウントがブロックされ,リツイートに失敗したツイート

\begin{itemize}
\tightlist
\item
  TW motoken\_tw(モトケン) 日時:2012/10/23 22:05:30 URL:
  \url{https://twitter.com/motoken\_tw/status/260728657084022785} 
  \textgreater{}
  裁判官に、裏切られた、という思いを生じさせることが問題なんです。RT
  @fukazawas: @motoken\_tw @taniyama @47news
  逃亡も問題ですし,刑罰法規の迅速な適用というのも重大な利益です。ただ,逃げても捕まえれば,ある程度の取り返しはつくでしょうが
\end{itemize}

※ @kk\_hironoのアカウントがブロックされ,リツイートに失敗したツイート

\begin{itemize}
\tightlist
\item
  TW fukazawas(深澤諭史) 日時:2012/10/23 22:06:39 URL:
  \url{https://twitter.com/fukazawas/status/260728947543785472} 
  \textgreater{} @motoken\_tw @taniyama @47news
  なるほど。根は深いですね。
\end{itemize}

 深澤諭史弁護士の高島章弁護士(新潟県弁護士会)、落合洋司弁護士(東京弁護士会)とのTwitterのやりとりというのも珍しく感じたのですが、そういう時代があったことを発見しました。

 どうも深澤諭史弁護士とモトケンこと矢部善朗弁護士(京都弁護士会)のツイートのやりとりにある谷山智光弁護士のツイートが見当たらなかったのですが、近く、弁護士の宴と一緒に取り上げたいと思います。

 弁護士が妖怪としての本性をまだストレートに見せていた時代、という感想がありました。これぐらいで切り上げておこうと思います。次があるので。

\begin{itemize}
\tightlist
\item
  〈〈〈 2021/06/06 15:08:56 Linux Emacs: 〈〈〈
\end{itemize}

\hypertarget{ux516cux5224ux8abfux66f8ux306bux3064ux3044ux3066googleux3067ux691cux7d22ux3057ux3066ux3044ux308bux3068ux51faux3066ux304dux305fux548cux4e45ux5cfbux4e09ux306eux96e8ux6708ux8358ux6bbaux4ebaux4e8bux4ef6}{%
\paragraph{公判調書についてGoogleで検索していると出てきた、和久峻三の「雨月荘殺人事件」}\label{ux516cux5224ux8abfux66f8ux306bux3064ux3044ux3066googleux3067ux691cux7d22ux3057ux3066ux3044ux308bux3068ux51faux3066ux304dux305fux548cux4e45ux5cfbux4e09ux306eux96e8ux6708ux8358ux6bbaux4ebaux4e8bux4ef6}}

\begin{itemize}
\tightlist
\item
  〉〉〉 Linux Emacs: 2021/06/21 17:15:40 〉〉〉
\end{itemize}

:CATEGORIES: @kanazawabengosi \#金沢弁護士会 @JFBAsns
日本弁護士連合会(日弁連) \#法務省 @MOJ\_HOUMU \#深澤諭史弁護士
\#被告発人岡田進弁護士

f:id:hirono\_hideki:20210623195924j:image

 予定では、ぽぽひとの正体とも呼ばれた柴田収弁護士のツイートと、そのツイートの内容に触発され作成した深澤諭史弁護士のツイートのまとめ記事について取り上げるつもりでした。状況次第では変化に対応し、急遽舵取りをする可能性もあるので、ツイートと記事を先に紹介しておきます。

〉〉〉 kk\_hironoのリツイート 〉〉〉

\begin{itemize}
\tightlist
\item
  RT
  kk\_hirono(刑事告発・非常上告_金沢地方検察庁御中)|themis\_okayama(弁護士 柴田収@DV・モラハラ離婚案件がメイン)
  日時:2021-06-21 17:21/2021/06/18 20:31 URL:
  \url{https://twitter.com/kk\_hirono/status/1406890044426657798} 
  \url{https://twitter.com/themis\_okayama/status/1405850540190232576} 
  \textgreater{}
  DVやモラハラをする人は「感情をコントロールできない」と評されることがありますが、彼ら(彼女ら)が職場の上司とか重要な取引先とかに怒りの感情をぶつけることはありません。
  怒りをぶつけていい相手を冷静に選んでいます。
\end{itemize}

 固定されたツイートになっていますが、昨夜気がついたところ6月18日のツイートでした。つい先日ですが、現時点でリツイートの数が1.2万、いいねが4.1万となっています。この反響の大きさも重視しました。

\begin{itemize}
\tightlist
\item
  2021年06月21日09時04分の登録:
  REGEXP:''病気''/深澤諭史(@fukazawas)の検索(2013-02-22〜2020-12-21/2021年06月21日09時04分の記録113件)
  \url{https://kk2020-09.blogspot.com/2021/06/regexpfukazawas2013-02-222020-12.html} 
\item
  2021年06月21日09時05分の登録:
  REGEXP:''医者''/深澤諭史(@fukazawas)の検索(2013-09-08〜2020-06-06/2021年06月21日09時05分の記録113件)
  \url{https://kk2020-09.blogspot.com/2021/06/regexpfukazawas2013-09-082020-06.html} 
\item
  2021年06月21日09時06分の登録:
  REGEXP:''病院''/深澤諭史(@fukazawas)の検索(2013-02-22〜2021-01-17/2021年06月21日09時06分の記録144件)
  \url{https://kk2020-09.blogspot.com/2021/06/regexpfukazawas2013-02-222021-01.html} 
\end{itemize}

 だいぶん前から公判調書を取り上げ告発事件の核心に迫る予定だったのですが、その場その場の出来事や発見もあり、意識が他に集中したり、記録の作成の作業を優先させたため、なかなか取り掛かることが出来ずにいました。

 前にも少しGoogleで調べていたのですが、公判調書に関する資料というのは意外に少なく、それも古いものが多いようです。裁判の書面の書式がA4の横書きに変更されてから20年ほど経つと思うので、その横書きの書面も参考にしたかったのですが、ほとんどが縦書きで手書きともなっているようです。

\begin{itemize}
\tightlist
\item
  あの人気ミステリー作家が完全再現した裁判の中身!あなたが判事、判決を下す『雨月荘殺人事件』和久峻三【変な本
  \#26】 - YouTube \url{https://t.co/7nVxulG7Iz}  42 回視聴•2021/05/08
\end{itemize}

 そんな中で発見したのが上記の動画です。最近アップロードされたYouTube動画で5月8日となっています。動画なのでわかりやすかったのですが、本の中身が袋とじとなり、そのまま裁判書類の体裁となっていました。この後もう少し調べて確認したいところですが、1989年となっていたように思います。

 私が子供の時代、小学生の低学年になるので昭和40年代の終わりの方だと思いますが、子供向けの雑誌によく付録がついていたことを思い出し、それの大人版のようにも思えました。とても珍しく思ったのですが、それ以上に機縁のようなものを感じました。

 和久峻三という作家のことは、平成4年に拘置所に入る前から知っていた数少ない作家で、テレビでは赤かぶ検事奮戦記というドラマが有名でしたが、そういうドラマ自体をみることは少なかったと思います。

 連続ドラマだったのか2時間もののサスペンスドラマだったのかも記憶はないですが、2時間ドラマというのは、当時落ち着きのなかった私には、我慢して視聴を続けるのも苦痛で、それ以上に集中力にも掛けたのか、内容や筋、流れを理解することが出来ず、ストレスにもなったような憶えがあります。

 本もなかなか集中して読むことが出来ず、理解も不十分だったのですが、金沢刑務所の拘置所では平成4年5月28日から平成6年3月17日まで1年9ヶ月以上ずっと独居房で生活し、じっとしていることには慣れるようになりました。なお、途中1月の精神鑑定がありました。

 拘置所の独居房では雨が続くと3週間ほど一度も外に出ないこともありました。晴れた日は外で30分ほどの運動がありましたが、他は着替えを入れて全部で15分の入浴が週に2回か3回あり、入浴のない免業日意外が運動でした。雨のときは室内でストレッチの放送です。

 そんな中で、和久峻三の本も数冊読んだのですが、忘れないのは官本にあった本で、小説ではない法律関係のくだけた感じの本でした。本の題名は憶えていないのですが、前から調べておきたいという考えはありました。特に印象的だったのが被告訴人池田宏美に似た短髪の女性のイラストです。

\begin{itemize}
\tightlist
\item
  和久峻三 - Wikipedia \url{https://t.co/GQMC5fS6Jy} 
\end{itemize}

 著作の数に驚いたのですが、小説以外の著書だけでも結構な数があります。ちょうど1年ほど前、平成5年11月28日付の手書きの書面で、平成5年11月28日付の手書きの書面で長文の引用部分を発見していたので、1年ぶりに調べ直すきっかけになりそうです。

 時刻は6月22日12時35分です。昨夜はまたいろいろと発見があって記録の作成をしておきたいところなのですが、「深澤先生」というのがキーワードになって、様々な発見につながっていきました。Twitterのトレンドに出たようですが、テレビドラマの話題のようでした。

 午前中、こたつの布団を取り払ったところ、置き場所がなく部屋の中がさらに散らかって、気分的に落ち着きがないのですが、平成5年11月28日付の手書きの書面にあった和久峻三の本のことも気になって頭が離れません。

 時刻は6月23日20時07分です。昨日の午後から中断をしていたと思いますが、平成5年11月28日付の手書きの書面を探していて、見つけることが出来なかったのですが、探したのをきっかけに思いがけない発見がありました。たぶん父親と母親の結婚式の写真になるのですが、生まれて初めて目にしたものです。

 本日6月23日は、午後に歯医者に行ってきました。正確なことはわからないですが3年ぶりぐらいになりそうな気がします。家に戻ると、母親が入院している病院から電話があり、介護の靴を買ってきてほしいと頼まれ、夕方に宇出津新港に行き、アルプで靴を買ってきました。

 何時頃になるのか憶えていませんが、今朝はモトケンこと矢部善朗弁護士(京都弁護士会)のタイムラインで大きな発見があり、そこから始まっています。

\begin{itemize}
\tightlist
\item
  〈〈〈 2021/06/23 20:16:07 Linux Emacs: 〈〈〈
\end{itemize}

\hypertarget{section-5}{%
\paragraph{}\label{section-5}}

\hypertarget{section-6}{%
\paragraph{}\label{section-6}}

\hypertarget{ux88abux544aux767aux4ebaux6728ux68a8ux677eux55e3}{%
\subsubsection{被告発人木梨松嗣}\label{ux88abux544aux767aux4ebaux6728ux68a8ux677eux55e3}}

\hypertarget{ux6700ux9ad8ux88c1ux3067ux5deeux3057ux623bux3057ux306bux306aux308aux6709ux7f6aux5224ux6c7aux304cux78baux5b9aux3057ux305fux3089ux3057ux3044ux304bux3059ux304bux306aux60c5ux5831ux3092ux898bux304bux3051ux305fux6ff1ux7530ux6b66ux5f8bux88c1ux5224ux9577ux306eux5f37ux5236ux308fux3044ux305bux3064ux7121ux7f6aux5224ux6c7aux3068abemaprimeux306eux30b4ux30b8ux30e9}{%
\paragraph{最高裁で差し戻しになり有罪判決が確定したらしいかすかな情報を見かけた濱田武律裁判長の強制わいせつ無罪判決と,AbemaPrimeのゴジラ}\label{ux6700ux9ad8ux88c1ux3067ux5deeux3057ux623bux3057ux306bux306aux308aux6709ux7f6aux5224ux6c7aux304cux78baux5b9aux3057ux305fux3089ux3057ux3044ux304bux3059ux304bux306aux60c5ux5831ux3092ux898bux304bux3051ux305fux6ff1ux7530ux6b66ux5f8bux88c1ux5224ux9577ux306eux5f37ux5236ux308fux3044ux305bux3064ux7121ux7f6aux5224ux6c7aux3068abemaprimeux306eux30b4ux30b8ux30e9}}

\begin{itemize}
\tightlist
\item
  〉〉〉 Linux Emacs: 2021/05/15 09:00:54 〉〉〉
\end{itemize}

:CATEGORIES: @kanazawabengosi \#金沢弁護士会 @JFBAsns
日本弁護士連合会(日弁連) \#法務省 @MOJ\_HOUMU \#被告発人木梨松嗣弁護士

 昨日の5月14日は,午前10時48分頃に図書館から電話があり,昼過ぎに取り寄せた桶川ストーカー殺人事件の清水潔氏の本を受け取り,Aコープ能都店で食パンとイカリングのフライだけ買って家に戻りました。地元のパン屋の食パンですが,食パンは初めて買ったかもしれません。

 図書館では今月いっぱいの緊急事態宣言が出たとかで,1時間以内とかパソコンの利用が出来ないとか制限が出ていました。平成4年の9月と10月の北國新聞縮小版を係の人にお願いしたのですが,その窓口のようなところの横にある棚の上の方に初めての発見がありました。

 見つけたのは北國新聞縮小版のDVDだったのですが,平成27年からになっていたように思います。館内のパソコンでしか見れないが緊急事態宣言が出ているのでパソコンの利用は出来ないといわれました。前から置いてあったらしいのですが,妙なタイミングで見つけたものです。

 平成4年の9月と10月の北國新聞縮小版を調べたのは,蛸島事件の無罪判決のことがあって,濱田武律裁判長が時期外れの転任になったきっかけのようなニュースがなかったのかと確認しておきたかったからです。

 その蛸島事件もインターネット上の情報がとても乏しいのですが,無罪判決が出した翌日ぐらいに,転任させられたという新聞記事がありました。

\begin{itemize}
\tightlist
\item
  ./kk\_hirono2021-05-15\_091732.csv:2021-02-07 16:01:27
  ``「珠洲市蛸島町の学童殺しに無罪判決をした金沢地裁・家裁七尾支部の田中武一判事(五九)=写真=は四日付け最高裁辞令で三重県津地裁・家裁伊勢支部判事に転任する。」とあります。''
  \url{https://twitter.com/kk\_hirono/status/1358309884790415360} 
\item
  ./kk\_hirono2021-05-15\_091732.csv:2021-02-07 15:54:48
  ``問題がありそうなので,ここではファイル名しかご紹介できないのですが,「2021-01-23\_164657_.jpg」が,昭和44年6月4日の北國新聞の撮影になります。小さい記事ですが,見出しに,「蛸島無罪判決の田中判事が転任 津地裁・家裁の伊勢支部へ」とあります。''
  \url{https://twitter.com/kk\_hirono/status/1358308210713665537} 
\end{itemize}

 蛸島事件の無罪判決は昭和44年6月3日となっているのですが,その翌日の4日付で最高裁辞令で三重県津地裁・家裁伊勢支部判事に転任とあります。歴史な大事件を発見したように思ったのですが,奥能登の図書館の片隅で,偶然が重なったのか見つけることが出来た情報でした。

\begin{quote}
《引用の始まり》
\end{quote}

\begin{quote}
生年月日 S7.4.1出身大学 京大退官時の年齢 61 歳叙勲
H14年春・勲二等瑞宝章H6.3.1 依願退官H4.11.20 ~ H6.2.28
大阪高裁2刑判事S63.12.1 ~ H4.11.19
名古屋高裁金沢支部刑事部部総括S62.4.1 ~ S63.11.30
大阪高裁3刑判事S58.4.1 ~ S62.3.31 京都地裁1刑部総括S54.4.1 ~
S58.3.31 和歌山地裁刑事部部総括S51.4.1 ~ S54.3.31
和歌山地家裁田辺支部長S47.4.1 ~ S51.3.31 大分地裁刑事部部総括S44.4.16
~ S47.3.31 大阪地家裁判事S42.4.16 ~ S44.4.15
和歌山地家裁新宮支部判事S42.4.6 ~ S42.4.15
神戸地家裁姫路支部判事S40.4.1 ~ S42.4.5 神戸地家裁姫路支部判事補S37.4.1
~ S40.3.31 徳島家地裁判事補S34.4.20 ~ S37.3.31 大阪地家裁判事補S32.4.6
~ S34.4.19 旭川地家裁判事補
\end{quote}

\begin{quote}
《引用の終わり》
\end{quote}

\begin{itemize}
\tightlist
\item
  浜田武律裁判官(9期)の経歴 \textbar{}
  弁護士山中理司のブログ \url{https://yamanaka-bengoshi.jp/2019/03/09/hamada3/n} 
\end{itemize}

 前にも見ているページですが,司法修習が9期という一桁はとてもめずらしく感じました。大正生まれで生きている可能性は乏しいと想像していたのですが,これも初めの発見のような感覚で昭和7年4月1日という生年月日がありました。

 名古屋高裁金沢支部刑事部部総括の任期が平成4年11月19日までとなっています。翌日の20日からは大阪高裁2刑判事での任期が始まっていますが,ちょうどこのあたりのタイミングで,私の第2回の控訴審公判は,被告発人小島裕史裁判長に変わったのです。

 濱田武律裁判長の審理は1回しか記憶にないので初公判になると思いますが,被告発人小島裕史裁判長に変わった第2回公判が平成4年11月20日以降というのは,参考になる,時期の絞り込みが出来る情報となりました。

 そういえば,濱田武律裁判長より経歴を何度も見ている被告発人小島裕史裁判長の全任地が記憶にないのですが,ついでに調べておきたいと思います。

\begin{quote}
《引用の始まり》
\end{quote}

\begin{quote}
生年月日 S12.3.10出身大学 金沢大退官時の年齢 65 歳叙勲
H19年秋・瑞宝重光章H14.3.10 定年退官H12.9.1 ~ H14.3.9
名古屋高裁2刑部総括H10.2.21 ~ H12.8.31 名古屋家裁所長H8.7.15 ~
H10.2.20 秋田地家裁所長H5.1.13 ~ H8.7.14 名古屋高裁金沢支部長H4.11.20
~ H5.1.12 名古屋高裁金沢支部民事部部総括S61.4.7 ~ H4.11.19
名古屋地裁4刑部総括S58.4.1 ~ S61.4.6 司研刑裁教官S55.4.1 ~ S58.3.31
岐阜地家裁判事S54.4.1 ~ S55.3.31 名古屋高裁判事S51.4.1 ~ S54.3.31
名古屋地裁判事S50.4.1 ~ S51.3.31 名古屋高裁金沢支部判事S47.4.10 ~
S50.3.31 金沢家地裁判事S47.4.1 ~ S47.4.9 金沢家地裁判事補S43.4.10 ~
S47.3.31 岐阜家地裁判事補S40.4.16 ~ S43.4.9
富山家地裁高岡支部判事補S37.4.10 ~ S40.4.15 名古屋地家裁判事補
\end{quote}

\begin{quote}
《引用の終わり》
\end{quote}

\begin{itemize}
\tightlist
\item
  小島裕史裁判官(14期)の経歴 \textbar{}
  弁護士山中理司のブログ \url{https://yamanaka-bengoshi.jp/2019/02/24/kojima14/n} 
\end{itemize}

 自分の頭がどうかしていたのかと思うぐらいの驚きの発見がありました。情報が追加されたという可能性もあるかと思いますが,生年月日
S12.3.10という,これは見覚えのある生年月日の真下に,「出身大学
金沢大」とあるのです。

 被告発人木梨松嗣弁護士が金沢大学の出身であることも年末か今年の初めの方に知ったと思います。宇出津新港に高級パンの店が出来た後になると思います。この高級パンの店のことがきっかけで,テレビでいしかわ四高記念公園を見たことが始まりでした。

\begin{itemize}
\tightlist
\item
  ./kk\_hirono2021-05-15\_093305.csv:2020-12-11 16:49:28
  ``テレビ金沢の「となりのテレ金ちゃん」という番組ですが,番組の合間に,能登町宇出津で高級パンの店が明日オープンと出て,仏壇に遺影の写真があり,昔,見た顔の面影があると感じました。番組の内容はネットの記事にもなっているのとほぼ同じでした。''
  \url{https://twitter.com/kk\_hirono/status/1337303471620382721} 
\end{itemize}

 正直なところ今年に入ってからと考えていたのですが,12月11日のツイートに明日オープンとあり,そういえばそんな時期だったと思い出しました。12月といえば,「さらばシベリア鉄道」の歌詞を思い出しますが,能登は歌の雰囲気に合うような季節です。

 初めにインターネットで被告発人小島裕史裁判長のことを調べた頃,同姓同名の弁護士が2人いたかもしれないのですが,情報の多い法律事務所が八王子市付近にあったように思います。そういえば小島裕史でtwilog-serch-post
はまだ作成していなかったかもしれません。

\begin{itemize}
\tightlist
\item
  2021年05月15日10時01分の登録:
  「小島裕史」を@hirono\_hideki @kk\_hirono @s\_hironoで検索 301件の該当 2021-05-15\_10:00の記録
  \url{https://kk2020-09.blogspot.com/2021/05/hironohidekikkhironoshirono3012021-05.html} 
\end{itemize}

2010-06-02 21:50:44 ``小島法律事務所(東京都練馬区) -
弁護士・法律事務所データ \url{http://goo.gl/Zojo} 
同姓同名の可能性もあると思いますが、小島裕史(ひろし)と読むのか、ずっと(ゆうじ)だと思っていた。コーポって、アパートが法律事務所なのかな?''
\url{https://twitter.com/hirono\_hideki/status/15248960183} 

2011-12-02 18:24:50
``小島裕史裁判長はそのまま検索しただけで、なんとか法律事務所というのが出ていました。うるおぼえですが八王子あたりだったかもです。RT
@amneris84
裁判官検索って便利ですね。退官後も分かるともっといいんだけど。''
\url{https://twitter.com/hirono\_hideki/status/142534681185890304} 

 2010年5月15日の検索結果のツイートが東京都練馬区ですが,2011年12月2日のツイートでも勘違いしていたのか「うるおぼえですが八王子あたりだったかもです。」となっていました。非公式の引用はジャーナリストの江川紹子氏のツイートのようです。

 なお,さきほどご紹介した蛸島事件で無罪判決を出して伊勢に飛ばされた裁判長ですが,田中武一判事は同じ石川県の小松市となっていました。

\begin{itemize}
\tightlist
\item
  ./hirono\_hideki2021-05-15\_100903.csv:2021-02-08 10:16:28 ``-
  1163:2021-02-08\_10:16:18
  告発に至る経緯・奥能登の蛸島事件と,弁護士列車編\#\#\#\#
  小松市出身という蛸島事件の田中武一裁判長,蝶屋町出身という同じく弁護人の梨木作次郎弁護士,根上町出身という森喜朗氏
  \url{https://hirono-hideki.hatenadiary.jp/entry/2021/02/08/101615''} 
  \url{https://twitter.com/hirono\_hideki/status/1358585456632360960} 
\end{itemize}

 小松市塩田町出身と北國新聞縮小版の記事にあったのですが,珠洲市の塩田のこともあるので海の近いとばかり思っていたのですが,山間で,ハニベ岩窟院の近くだったと思います。ここに行った夜,片山津温泉で母親が心臓発作を起こし,救急車で大聖寺の病院に運ばれました。

 昨夜は,まだ早めの時間で夕方に近かったような気もしますが,星野事件について調べていて,山中事件の情報がぞろぞろと出てきました。梨木作次郎弁護士が弁護団の団長となっていたように思います。

 不思議なことですが,山中事件は昨年の5月頃,宮城県の松山事件と同時進行で徹底的に調べたのですが,山中事件と梨木作次郎弁護士を結びつける情報は見なかったと思うのです。どちらもtwilog-serch-post
を作成します。

\begin{itemize}
\item
  2021年05月15日10時01分の登録:
  「小島裕史」を@hirono\_hideki @kk\_hirono @s\_hironoで検索 301件の該当 2021-05-15\_10:00の記録
  \url{https://kk2020-09.blogspot.com/2021/05/hironohidekikkhironoshirono3012021-05.html} 
\item
  2021年05月15日10時27分の登録:
  「山中事件」を@hirono\_hideki @kk\_hirono @s\_hironoで検索 62件の該当 2021-05-15\_10:27の記録
  \url{https://kk2020-09.blogspot.com/2021/05/hironohidekikkhironoshirono622021-05.html} 
\item
  2021年05月15日10時28分の登録:
  「星野事件」を@hirono\_hideki @kk\_hirono @s\_hironoで検索 11件の該当 2021-05-15\_10:28の記録
  \url{https://kk2020-09.blogspot.com/2021/05/hironohidekikkhironoshirono112021-05.html} 
\item
  2021年05月15日10時28分の登録:
  「蛸島事件」を@hirono\_hideki @kk\_hirono @s\_hironoで検索 218件の該当 2021-05-15\_10:28の記録
  \url{https://kk2020-09.blogspot.com/2021/05/hironohidekikkhironoshirono2182021-05.html} 
\item
  2021年05月15日10時29分の登録:
  「梨木作次郎」を@hirono\_hideki @kk\_hirono @s\_hironoで検索 94件の該当 2021-05-15\_10:28の記録
  \url{https://kk2020-09.blogspot.com/2021/05/hironohidekikkhironoshirono942021-05.html} 
\item
  星野事件 - Google 検索 \url{https://t.co/V9rKObx9bN} 
\end{itemize}

 Googleの検索結果で5ページ目まで開きましたが星野事件は見つかりませんでした。星野リゾートが多いですが,昨夜は1件のPDFファイルと,他の星野という人の冤罪事件に関する情報もちらほらありました。死刑囚だったかもしれません。亡くなったという見出しも1つ見かけました。

\begin{itemize}
\tightlist
\item
  無実で39年 獄壁をこえた愛と革命―星野文昭・暁子の闘い \textbar{}
  星野さんをとり戻そう!全国再審連絡会議 \textbar 本 \textbar{} 通販
  \textbar{} Amazon \url{https://t.co/ZKnIxBDYgc} 
\end{itemize}

 「星野 冤罪」とGoogle検索のキーワードを変更しています。ずいぶん前からたまに見かけることのあった事件ですが,本が出ていたとは知りませんでした。検索結果には渋谷暴動事件ともあったように思います。

\begin{itemize}
\tightlist
\item
  渋谷暴動事件で無期懲役 獄中で「冤罪」訴える71歳男性の家族らが高松市で会見
  \textbar{} KSBニュース \textbar{} KSB瀬戸内海放送
  \url{https://t.co/HkFGZZwcKh} 
\end{itemize}

 記事にニュース動画が出てきて視聴しましたが,死刑囚ではなく無期懲役でした。1971年というのも被害者安藤文さんの生まれた1970年の翌年で感慨深さがありますが,暴動という混乱の中の事件で,無実でみせしめにされたという可能性は本当にあるのかもしれません。

\begin{itemize}
\tightlist
\item
  無実の政治囚
  星野文昭さん逝去から1年 国賠に勝利し新たな闘いの発展へ 星野文昭さんを取り戻そう!全国再審連絡会議 金山克巳
  - 週刊『前進』 \url{https://t.co/HeAtmk2bEU}  発行日: 2020年5月25日 第3134号
  弾圧との闘い
\end{itemize}

\begin{quote}
《引用の始まり》
\end{quote}

\begin{quote}
体勢を立て直した隊員らが戻ると、N巡査は真っ黒になってうずくまっていた。顔の識別が難しいほどの全身火傷を負ったN巡査は新潟県から父と兄が駆け付けてきた数時間後の翌15日21時25分、死亡が確認された[12][13]。身体の一部が炭化するほど激しく損傷していたため、母親には対面させられなかった[14]。現場道路のアスファルトは変色しており、火勢の凄まじさを物語っていた[15]。中核派はN巡査の他にも、新潟県警機動隊員3人に同様の暴力を繰り返し、全治2週間から治癒期間不明の熱傷等の重症を負わせた[16]。

活動家には女性も居り、犯行後マスクと軍手を外し、Gパンを脱ぎ捨て下に履いていたミニスカートで逃走するなど、捜査が及ばないように予め周到な準備を行っていた[17]。

中核派は『前進』で次のように報じた[18]。
\end{quote}

\begin{quote}
《引用の終わり》
\end{quote}

\begin{itemize}
\tightlist
\item
  渋谷暴動事件 -
  Wikipedia \url{https://ja.wikipedia.org/wiki/\%E6\%B8\%8B\%E8\%B0\%B7\%E6\%9A\%B4\%E5\%8B\%95\%E4\%BA\%8B\%E4\%BB\%B6n} 
\end{itemize}

 学生運動の1つだったようですが,壮絶で生々しい描写がありました。学生と警察あるいは学生と権力との衝突が,これまでの認識や理解を超えて激しいのですが,これも時代背景の一つで,裁判でもそれに近い激しい対立があったのかもしれません。まだ年を確認していませんが,昭和40年台と見ています。

 昭和46年11月14日とありました。浅間山荘事件の報道はかすかに記憶にあるのですが,日本赤軍とか海外でのハイジャック事件も大きなニュースになっていたように思います。

 昨夜,星野という名前は余り聞かないと有名人のことなど思い浮かべていたのですが,そういえば銀河鉄道999の主人公の少年の名前が星野鉄郎だったと気が付きました。今朝,さきほどの検索では,星野源という著名人の名前もありました。

 「逃げるは恥だが役に立つ」というドラマが大流行していましたが,正月の特番で少しだけ見て,最終回の神社でのバザールのような場面が印象的でした。名前が思い出せないですが,リーガルハイの女性弁護士か事務員役で,初めに注目するようになった女優です。

 そういえばガッキーと呼ばれていました。新垣ゆいだと思いますが,漢字がはっきりしません。新垣結衣なのか。

 今朝は,AbemaPrimeで映画のゴジラを少しだけ飛ばしながら視聴したのですが,AbemaPrimeを見たのは「リーガルハイ」の放送がないか探すのも1つの目的となっています。

 コメント欄をみると,ゴジラのシリーズで最初の作品だったようです。やがて50年近くも前のことなのでぼんやりとしか記憶はないですが,小学生の頃,宇出津にあった映画館でゴジラを観ました。

 AbemaPrimeの視聴を始めて5つ目ぐらいに映画「日本沈没」を視聴したのですが,ほとんど記憶にないような内容でした。宇出津小学校の6年生の時,学校でこの映画を観に行ったのですが,先生がよくこの映画を子供に魅せたと思える内容で,時代の違いも感じました。

 映画を観に行った回数は少ないですが,テレビではよく戦争のドラマがある時代でした。昭和50年代になると戦争ドラマの数も減っていったように思いますが,法廷で濱田武律裁判長を見たとき,昭和の時代の戦争映画か戦争ドラマでみた軍人の大将の役のように思えました。

 裁判官というのはニュースでもドラマでも見かけますが,椅子には座っているのでしょうが直立不動で常に前を見ているというイメージがあります。どういう仕草だったのかはっきりとは思い出せないですが,濱田武律裁判長は,服を揺らし身を乗り出すような動きをしたのが印象的でした。

 初公判の最初で,私をとても珍しそうに見ながら,「心神喪失,だったら無罪じゃないか」などと言いました。戦争ドラマだけではなく,ウルトラマンの地球防衛軍などの中にも,濱田武律裁判長のような出演者がいたように記憶にあり,そういうのもとても印象深く憶えています。

 つい最近まで,あのまま濱田武律裁判長が裁判長のままで,被告発人小島裕史裁判長にかわっていなければ,もっとましな審理になっていたとも考えていたのですが,星野事件の無罪判決を見ると,幻影が打ち砕かれたような思いにもなりました。

\begin{itemize}
\tightlist
\item
  \url{https://t.co/hEEXK0Lxkh:}  ゴジラを観る \textbar{} Prime Video
  \url{https://t.co/riOBXKr7KE} 
\end{itemize}

 コメントではなくレビューとなっていました。ゴジラが水爆実験の放射能汚染から誕生したことはずっと前から知っていましたが,このレビューにあるように政治的メッセージ性があるとは聞いたことがなかったし,考えたこともありませんでした。

 終戦から9年後の映画で空襲を再現しているというレビューのコメントもありましたが,まさにそれを強く感じたのが映画「日本沈没」の場面でした。見たことのないほどリアルで,これは実際に空襲を経験した人の実体験が反映されていると思いました。

 Amazonプライムビデオでは,シン・ゴジラも視聴できるようになっていましたが,数年前,人からDVDを借りて視聴したことがありました。深澤諭史弁護士のTwitterのアイコンがシン・ゴジラを模したものとなり,その後,変更がありません。

\begin{lstlisting}
py37_env ❯ tun fukazawas 1
\end{lstlisting}

\begin{itemize}
\tightlist
\item
  RT fukazawas(深澤諭史)|shimadayusuke66(島田雄左) 日時:2021-04-26
  07:41/2021-04-23 18:57 URL:
  \url{https://twitter.com/fukazawas/status/1386450393157169156} 
  \url{https://twitter.com/shimadayusuke66/status/1385533372798173191} 
  \textgreater{}
  弁護士や司法書士の資格取得後、以前は、丁稚奉公したら独立するのスタンダードでした。でも、今は独立に限らず、専門性を高めたり事務所の総合化などに伴い、色んなキャリアプランが増えたように感じます。そう考えると、資格者の役割も多様化してるので、資格取得後のキャリアプランは無限大です。
\end{itemize}

 コマンドで確認しましたが,現時点でも2021-04-26
07:41のリツイート以来,深澤諭史弁護士のタイムラインは更新がないようです。休眠状態ともなっているのですが,理由,原因は不明のままです。

 前にもずいぶん前から止まっているTwitterアカウントを見ているのですが,1つが伊藤真弁護士,もう1つが久保利英明弁護士で,久保利英明弁護士のTwitterアカウントは,他のYouTubeなどと一緒に,割と最近になって再始動を始めています。

〉〉〉 kk\_hironoのリツイート 〉〉〉

\begin{itemize}
\tightlist
\item
  RT
  kk\_hirono(刑事告発・非常上告_金沢地方検察庁御中)|ito\_\_makoto(伊藤真)
  日時:2021-05-15 12:00/2012/05/10 10:48 URL:
  \url{https://twitter.com/kk\_hirono/status/1393400971481403393} 
  \url{https://twitter.com/ito\_\_makoto/status/200401889781743616} 
  \textgreater{}
  論究ジュリスト2012春号「特集憲法最高裁判例を読み直す」を読んだ。安念潤司先生の「憲法訴訟論とは何だったのか、これから何であり得るか」に最大の共感を覚え、かつこういう先生の下で学べる学生をうらやましく思った。受験生も自分を客観視するために必読と思う。
\end{itemize}

 上記が最終更新となっている伊藤真弁護士のツイートですが,今の私のリツイートで106となりました。フォロワーの数が意外に少なく,9,979となっています。フォロー中が41です。アイコンは本人の顔写真ですが,認証マークはなく,弁護士で認証マークがあるのは,ごくわずかです。

 これまで取り上げる必要があるのか迷いもあった久保利英明弁護士の弁護活動があるのですが,これも昭和40年代で,その舞台というのも同じ石川県の金沢でした。ネットの情報で金沢事件として見かけることもありますが,金沢仁検事による暴行事件も金沢事件となっていたように思います。

 金沢仁検事の暴行事件は,金沢刑務所の拘置所にいるとき週刊誌で記事を読んだと記憶にあります。鬼検事というのは昭和から平成の初めによく聞いた言葉でしたが,実際の暴力事件というのは聞いたのも始めてで,週刊誌の小説から飛び出してきたような事件でした。

 今でもたまにネットで見かけることのある金沢仁検事の暴行事件ですが,拘置所の独居房にいたときという記憶しかなく,インターネットが普及していない時代でもあったと思います。検索すればすぐに情報は見つかるでしょうが,検索をするのは初めてになるかもしれません。

\begin{itemize}
\tightlist
\item
  創価大学卒の元検事・金沢仁さんは -
  創価学会員として、一体どのような\ldots{} - Yahoo!知恵袋
  \url{https://t.co/LyTdw6V2BD} 
\end{itemize}

 上記のページのベストアンサーに「1993年11月29日
金沢検事、懲戒免職処分を受け、逮捕される。(容疑を認める)。」とありました。

 平成5年というのは思ったより古く感じたのですが,同じ頃になるのか,週刊誌に創価学会が会員を多数,法律家にしているという記事がありました。法曹です。週刊誌では他にも創価学会のゴシップ記事が多い時代でした。

 ベストアンサーの続きには,朝日新聞の引用開始としたあとに,小説のような暴行の様子があるのですが,引用の終了が見当たらないので,わかりづらくなっています。

 金沢仁検事の暴行事件は,意外に検索結果の情報が少ないのですが,ほとんどが創価学会と関連付けられているようです。ゼネコンというのも今は滅多にみかけないですが,それこそ耳にタコが出来るほど,見聞きする時代がありました。

 予定では次に,珠洲市三崎町と被告発人木梨松嗣弁護士を取り上げるつもりでしたのですが,前座というかたちになるのか,久保利英明弁護士と金沢事件を先に取り上げておきたいと思います。多少金沢大学との関連もあります。

\begin{itemize}
\tightlist
\item
  〈〈〈 2021/05/15 12:33:10 Linux Emacs: 〈〈〈
\end{itemize}

\hypertarget{ux5e73ux62106ux5e7411ux670810ux65e5ux304b11ux65e5ux306bux798fux4e95ux5211ux52d9ux6240ux3078ux5c4aux3044ux305fux3068ux30ceux30fcux30c8ux3067ux8a18ux8f09ux3092ux898bux3064ux3051ux305fux88abux544aux767aux4ebaux6728ux68a8ux677eux55e3ux5f01ux8b77ux58ebux306eux4e00ux4ef6ux8a18ux9332ux88abux544aux767aux4ebaux9577ux8c37ux5dddux7d18ux4e4bux5f01ux8b77ux58ebux306eux8a34ux72b6ux306eux4e00ux9031ux9593ux5f8c}{%
\paragraph{平成6年11月10日か11日に福井刑務所へ届いたとノートで記載を見つけた被告発人木梨松嗣弁護士の一件記録,被告発人長谷川紘之弁護士の訴状の一週間後}\label{ux5e73ux62106ux5e7411ux670810ux65e5ux304b11ux65e5ux306bux798fux4e95ux5211ux52d9ux6240ux3078ux5c4aux3044ux305fux3068ux30ceux30fcux30c8ux3067ux8a18ux8f09ux3092ux898bux3064ux3051ux305fux88abux544aux767aux4ebaux6728ux68a8ux677eux55e3ux5f01ux8b77ux58ebux306eux4e00ux4ef6ux8a18ux9332ux88abux544aux767aux4ebaux9577ux8c37ux5dddux7d18ux4e4bux5f01ux8b77ux58ebux306eux8a34ux72b6ux306eux4e00ux9031ux9593ux5f8c}}

\begin{itemize}
\tightlist
\item
  〉〉〉 Linux Emacs: 2021/05/19 14:02:32 〉〉〉
\end{itemize}

:CATEGORIES: @kanazawabengosi \#金沢弁護士会 @JFBAsns
日本弁護士連合会(日弁連) \#法務省 @MOJ\_HOUMU \#被告発人木梨松嗣弁護士
\#被告発人長谷川紘之弁護士 \#福井刑務所

 2,3週間ほど前になるのか家の中で福井刑務所でのノートを見つけました。今日になって引っ張り出し外でホコリを落としたのですが,12冊ほどあるノートのうち以外に早く目的の情報を探し出すことが出来ました。

〉〉〉 kk\_hironoのリツイート 〉〉〉

\begin{itemize}
\tightlist
\item
  RT
  kk\_hirono(刑事告発・非常上告_金沢地方検察庁御中)|s\_hirono(非常上告-最高検察庁御中\_ツイッター)
  日時:2021-05-19 14:07/2021/05/19 11:23 URL:
  \url{https://twitter.com/kk\_hirono/status/1394882441421070337} 
  \url{https://twitter.com/s\_hirono/status/1394841025135276033} 
  \textgreater{}
  2021-05-19\_105810_福井刑務所 ノート使用許可書 平成6年4月18日 雑記用.jpg
  \url{https://t.co/rmBsgvyAT9} 
\end{itemize}

〉〉〉 kk\_hironoのリツイート 〉〉〉

\begin{itemize}
\tightlist
\item
  RT
  kk\_hirono(刑事告発・非常上告_金沢地方検察庁御中)|s\_hirono(非常上告-最高検察庁御中\_ツイッター)
  日時:2021-05-19 14:07/2021/05/19 11:23 URL:
  \url{https://twitter.com/kk\_hirono/status/1394882459192283139} 
  \url{https://twitter.com/s\_hirono/status/1394841010232909825} 
  \textgreater{}
  2021-05-19\_105748_福井刑務所 ノート使用許可書 平成6年4月18日 雑記用.jpg
  \url{https://t.co/CK6K1CxHGt} 
\end{itemize}

〉〉〉 kk\_hironoのリツイート 〉〉〉

\begin{itemize}
\tightlist
\item
  RT
  kk\_hirono(刑事告発・非常上告_金沢地方検察庁御中)|s\_hirono(非常上告-最高検察庁御中\_ツイッター)
  日時:2021-05-19 14:07/2021/05/19 11:23 URL:
  \url{https://twitter.com/kk\_hirono/status/1394882473394270208} 
  \url{https://twitter.com/s\_hirono/status/1394840994630148100} 
  \textgreater{}
  2021-05-19\_105734_福井刑務所 ノート使用許可書 平成6年4月18日 雑記用.jpg
  \url{https://t.co/erXdhj7tdk} 
\end{itemize}

〉〉〉 kk\_hironoのリツイート 〉〉〉

\begin{itemize}
\tightlist
\item
  RT
  kk\_hirono(刑事告発・非常上告_金沢地方検察庁御中)|s\_hirono(非常上告-最高検察庁御中\_ツイッター)
  日時:2021-05-19 14:07/2021/05/19 11:22 URL:
  \url{https://twitter.com/kk\_hirono/status/1394882492444725248} 
  \url{https://twitter.com/s\_hirono/status/1394840979761295362} 
  \textgreater{}
  2021-05-19\_105708_福井刑務所 ノート使用許可書 平成6年4月18日 雑記用.jpg
  \url{https://t.co/ZMzE2RShG1} 
\end{itemize}

〉〉〉 kk\_hironoのリツイート 〉〉〉

\begin{itemize}
\tightlist
\item
  RT
  kk\_hirono(刑事告発・非常上告_金沢地方検察庁御中)|s\_hirono(非常上告-最高検察庁御中\_ツイッター)
  日時:2021-05-19 14:07/2021/05/19 11:22 URL:
  \url{https://twitter.com/kk\_hirono/status/1394882503886864387} 
  \url{https://twitter.com/s\_hirono/status/1394840965295140864} 
  \textgreater{}
  2021-05-19\_105611_福井刑務所 ノート使用許可書 平成6年4月18日 雑記用.jpg
  \url{https://t.co/4rhfQLMRTt} 
\end{itemize}

〉〉〉 kk\_hironoのリツイート 〉〉〉

\begin{itemize}
\tightlist
\item
  RT
  kk\_hirono(刑事告発・非常上告_金沢地方検察庁御中)|s\_hirono(非常上告-最高検察庁御中\_ツイッター)
  日時:2021-05-19 14:08/2021/05/19 11:22 URL:
  \url{https://twitter.com/kk\_hirono/status/1394882525428736000} 
  \url{https://twitter.com/s\_hirono/status/1394840950774534147} 
  \textgreater{}
  2021-05-19\_102839_外でホコリを落とした犯罪精神医学,UNIX関係の勉強の本.jpg
  \url{https://t.co/vhka60BHeV} 
\end{itemize}

〉〉〉 kk\_hironoのリツイート 〉〉〉

\begin{itemize}
\tightlist
\item
  RT
  kk\_hirono(刑事告発・非常上告_金沢地方検察庁御中)|s\_hirono(非常上告-最高検察庁御中\_ツイッター)
  日時:2021-05-19 14:08/2021/05/19 11:22 URL:
  \url{https://twitter.com/kk\_hirono/status/1394882546945576960} 
  \url{https://twitter.com/s\_hirono/status/1394840935733682177} 
  \textgreater{}
  2021-05-19\_101941_外でホコリを落とした福井刑務所のノート雑記帳.jpg
  \url{https://t.co/GrBh3aFwhr} 
\end{itemize}

〉〉〉 kk\_hironoのリツイート 〉〉〉

\begin{itemize}
\tightlist
\item
  RT
  kk\_hirono(刑事告発・非常上告_金沢地方検察庁御中)|s\_hirono(非常上告-最高検察庁御中\_ツイッター)
  日時:2021-05-19 14:13/2021/05/19 11:22 URL:
  \url{https://twitter.com/kk\_hirono/status/1394883907909152775} 
  \url{https://twitter.com/s\_hirono/status/1394840834403561476} 
  \textgreater{} 2021-05-19\_100809_.jpg \url{https://t.co/34KkNAu3pk} 
\end{itemize}

 @s\_hirono
ノートの使用許可書が平成6年4月18日になっています。これは2週間の新入教育が終わって2,3日後になるのかと思います。南寮3階6号室というのは雑居房です。

 時刻は5月20日15時22分です。確認すると2021-05-19
14:21からの中断となっていました。宇出津新港に買い物に出かけ,釣り公園の辺りを散歩したりしていましたが,そのまま再開することはありませんでした。出先でスマホからの投稿は考えていました。

 @s\_hirono
写真で上の右から2つ目の「Dabian辞典」という本が,きっかけでttyref,ttygifというコマンドを知りました。再生が長くなるとGIF動画の作成に失敗したのですが,これがヒントとなり,他の方法で記録化することにしました。

 @s\_hirono
ノートの内容でこの平成6年4月18日が南寮3階6室への転房だとわかりました。2週間の侵入教育,配役審査を経て,第2工場の出役が始まった日になると思います。

 @s\_hirono
「11月6日(日)雨 金曜日,210号室に転房になった。書類も入っていた。」とあります。当時の福井刑務所は南寮と北寮という棟があって,南寮は雑居房,北寮は独居房でした。

 @s\_hirono
210号室とありますが,北寮2階の10号室のことだと思います。通路の手前で201号室から始まっていたと思いますが,受刑中は6号室辺りより奥に行くことがなかったと思っていました。北寮3階にいた方が長かったとも思います。

 @s\_hirono
「11月8日(または9日)母面会,5000円差入」とあります。調べたところ平成6年は8日が火曜日,9日が水曜日となっていました。

 @s\_hirono
11月14日(月)に,「11日に届いたらしい一件記録の通知を受けた」とあります。これが母親経由で被告発人木梨松嗣弁護士から届いた裁判の記録になります。

 @s\_hirono
11月11日は金曜日なのですが,この日に母親の面会があったものと勘違いしていました。面会は8日か9日となっていました。郵送で届いたのが11日,通知が休み明けの月曜日14日になります。

 @s\_hirono
被告発人木梨松嗣弁護士から送られたものを送るという話は,手紙などではなく直接母親から話を聞いたように記憶にあったので,それが8日か9日の面会になりそうです。

 なお,当時の福井刑務所は,新入教育が第1工場と決まっていて,新入教育の終了と同時で配役審査で配役される工場が決まるのですが,私はずっと第2工場でした。一度,懲罰を受けていますが2工場に戻っています。懲罰が終わることを罰明けと呼んでいましたが,他の工場に移動が多いと聞きました。

 その第2工場は数が一番多く少ないときで60人ぐらいから多いときで75人ぐらいだったと思いますが,雑居房の場合,第2工場の3,4級生は南寮3階,2級生は南寮4階と決まっていました。福井刑務所に1級生はいませんでしたが,準1級生はいて,北寮4階で開放的な生活をしていました。

 独居房に鍵はなく,第2工場で最初に雑役をしていた人が準1級生でしたが,刑務所の外の工場に通いで仕事に行っている受刑者が多いと聞いていました。親和寮などと呼ばれていました。一度だけ中に入ったことがありました。

 福井刑務所での母親の面会ですが,あまり憶えてはいないものの2,3ヶ月に1回はあったように思います。福井刑務所の敷地内は入り組んだ迷路のようになっていてよくわからない場所に面会室がありました。2級生になってからは遮蔽板のない畳の部屋で面会をしたように憶えています。

 ノートに大体の日付の記載があったものの内容に具体性がありませんでした。初めて独居房に移った日は,工場から戻った時点で,洗濯かごに入った被告発人長谷川紘之弁護士の書面が置いてありましたが,その夜に,けっこう印象に残る日本映画がありました。

 少年少女が離島のような場所で冒険をする映画だったと思います。時折,書面から目を移していたので映画の内容は記憶にもないですが,その映画のことと相まって,独居房での初日のことはよく憶えています。独居房でテレビを見るのも初めてでしたが,雑居房と同じ大きさでした。

 整理しますと,平成6年11月4日金曜日に,私は北寮2階210号室に転房したことになります。11月に入ってすぐの免業日の午睡で,半袖シャツ1枚で寝ていた人のことをよく憶えていたのですが,土日ではなく,11月3日の祭日であったと思われます。転房の前日というのも意外でした。

 天気は良かったですが11月に入って肌寒さもあったのに,真夏と同じような格好で午睡をしていました。年配の受刑者で,小柄な人でしたが強盗か殺人未遂と話していました。名古屋の人です。心臓に持病があるらしくニトロを服用しているとも話していました。ニトロの薬は初めて聞く話でした。

 あまりはっきり思い出せないのですが,郵送などで差入があった場合,通知がありました。帳簿に名前を書いて指印を押していたように思います。この差入のあった物品が領置品になるのですが,手元の房内に入れるには,仮下げという願箋が必要でした。これは拘置所でも同じです。

 土日祭日の免業日は仮下げの交付もないのですが,拘置所では仮下げの願箋を出せる日が月水金と決まっていたように思います。受刑者であれば工場で雑役の人に頼んでいつでも願箋を出せたかもしれません。雑役と呼ばれていましたが,掃除洗濯専門の雑役とは違い,刑務官の秘書のような仕事でした。

 秘書というのもよくわからないですが,事務の仕事で経理などもしていたようです。工場内で事務所のような一角になっていました。計算工とも呼ばれることがあったかもしれません。受刑者の世話役でもあります。私がいる間に3人ぐらい替わりましたが,今でもよく憶えています。

 少し思い出したのですが,願箋は雑役の人にもらって居房に戻ってから記入し,朝の工場出役のときに,岡持ちという木の箱に入れていたように思います。脱衣場の手前で,置いてある岡持ちの箱に工場に持っていきたいものを入れ,脱衣場では工場用の衣服に着替えていました。

 3.4ヶ月か前,日弁連が出どころで,この受刑者の脱衣場での検査のことも問題にされていました。昔は,カンカン踊りとも呼ばれていたようですが,何年も経ってから見かけた話題だったので,今でも問題になるのかと思ったのですが,検査が緩くはなっているようでした。

 2級生の場合は,パンツとシャツで,シャツを上にまくりあげて刑務官に見せていたように記憶にあります。ホリエモンこと堀江貴文氏が服役した頃,刑務所の生活がネットで話題になっていて,その頃には監獄法が別に法律に変わっていて,制限もずいぶん緩和されているような話でした。

 仮下げの願箋を出して11月4日には入っているので,11月に入ってから被告発人長谷川紘之弁護士の書面が届いたという通知があったのだと思いますが,私の記憶では通知から4,5日あったように思います。いきなり雑居房から独居房へ転房というのが大きく,その間に数日あったと思います。

 福井刑務所のノートには他にも意外な大きな発見があったのですが,これは被告発人松平日出男と関連付けて記述します。昨日5月19日,覚醒剤密輸の未遂で無期懲役という判例を見つけたのですが,わずか5枚というのが驚きでした。これも被告発人松平日出男に関連付けます。

\begin{itemize}
\tightlist
\item
  〈〈〈 2021/05/20 17:13:50 Linux Emacs: 〈〈〈
\end{itemize}

\hypertarget{ux9577ux539fux609fux5f01ux8b77ux58ebux3068ux7c73ux7530ux5f18ux5e78ux5f01ux8b77ux58ebux3068ux306e2ux3064ux306eux7d44ux307fux5408ux308fux305bux304cux3042ux308bux88abux544aux767aux4ebaux6728ux68a8ux677eux55e3ux5f01ux8b77ux58ebux306eux6cd5ux5f8bux4e8bux52d9ux6240ux540dux91d1ux6ca2ux5e02ux5927ux624bux753a734ux3068ux3044ux3046ux756aux5730ux306eux4f4fux6240ux306eux4e0dux601dux8b70ux306aux7591ux554fux70b9}{%
\paragraph{長原悟弁護士と米田弘幸弁護士との2つの組み合わせがある被告発人木梨松嗣弁護士の法律事務所名,金沢市大手町7−34という番地の住所の不思議な疑問点}\label{ux9577ux539fux609fux5f01ux8b77ux58ebux3068ux7c73ux7530ux5f18ux5e78ux5f01ux8b77ux58ebux3068ux306e2ux3064ux306eux7d44ux307fux5408ux308fux305bux304cux3042ux308bux88abux544aux767aux4ebaux6728ux68a8ux677eux55e3ux5f01ux8b77ux58ebux306eux6cd5ux5f8bux4e8bux52d9ux6240ux540dux91d1ux6ca2ux5e02ux5927ux624bux753a734ux3068ux3044ux3046ux756aux5730ux306eux4f4fux6240ux306eux4e0dux601dux8b70ux306aux7591ux554fux70b9}}

\begin{itemize}
\tightlist
\item
  〉〉〉 Linux Emacs: 2021/05/26 13:45:18 〉〉〉
\end{itemize}

:CATEGORIES: @kanazawabengosi \#金沢弁護士会 @JFBAsns
\#日本弁護士連合会(日弁連) \#法務省 @MOJ\_HOUMU
\#被告発人木梨松嗣弁護士

 昨夜,被告発人木梨松嗣弁護士の法律事務所について調べたのですが,その奉納\さらば弁護士鉄道・泥棒神社の物語(@hirono\_hideki)のツイートをまずリツイートしてご紹介したいと思います。

〉〉〉 kk\_hironoのリツイート 〉〉〉

\begin{itemize}
\tightlist
\item
  RT
  kk\_hirono(刑事告発・非常上告_金沢地方検察庁御中)|hirono\_hideki(奉納\さらば弁護士鉄道・泥棒神社の物語)
  日時:2021-05-26 13:49/2021/05/26 02:07 URL:
  \url{https://twitter.com/kk\_hirono/status/1397414478119989249} 
  \url{https://twitter.com/hirono\_hideki/status/1397237835099475971} 
  \textgreater{} - 1386:2021-05-26\_02:07:05 \#告発状 \#\#\#\#
  2021年5月26日未明,大きな発見となった石川県羽咋市の平鍛造株式会社と被告発人木梨松嗣弁護士との接点,評議員に長原悟弁護士,理事に米田弘幸弁護士という関係性
  \url{https://t.co/9K52X9Ixpb} 
\end{itemize}

〉〉〉 kk\_hironoのリツイート 〉〉〉

\begin{itemize}
\tightlist
\item
  RT
  kk\_hirono(刑事告発・非常上告_金沢地方検察庁御中)|hirono\_hideki(奉納\さらば弁護士鉄道・泥棒神社の物語)
  日時:2021-05-26 13:49/2021/05/26 00:41 URL:
  \url{https://twitter.com/kk\_hirono/status/1397414496352694272} 
  \url{https://twitter.com/hirono\_hideki/status/1397216212854218753} 
  \textgreater{} - 会社概要 \textbar{} 会社案内 \textbar{} 平鍛造(R) /
  平鍛造株式会社 - Taira Forging \url{https://t.co/eyGk82EOFE} 
\end{itemize}

〉〉〉 kk\_hironoのリツイート 〉〉〉

\begin{itemize}
\tightlist
\item
  RT
  kk\_hirono(刑事告発・非常上告_金沢地方検察庁御中)|hirono\_hideki(奉納\さらば弁護士鉄道・泥棒神社の物語)
  日時:2021-05-26 13:49/2021/05/26 00:30 URL:
  \url{https://twitter.com/kk\_hirono/status/1397414508461641728} 
  \url{https://twitter.com/hirono\_hideki/status/1397213544974888964} 
  \textgreater{} - 役員一覧 \textbar{} 公益財団法人 平昭七記念財団
  \url{https://t.co/WrK88kjU9h} 
\end{itemize}

〉〉〉 kk\_hironoのリツイート 〉〉〉

\begin{itemize}
\tightlist
\item
  RT
  kk\_hirono(刑事告発・非常上告_金沢地方検察庁御中)|hirono\_hideki(奉納\さらば弁護士鉄道・泥棒神社の物語)
  日時:2021-05-26 13:49/2021/05/26 00:12 URL:
  \url{https://twitter.com/kk\_hirono/status/1397414552086540289} 
  \url{https://twitter.com/hirono\_hideki/status/1397209057509978119} 
  \textgreater{} - 長原悟(ながはらさとる)弁護士の刑事事件対応情報
  \textbar{} あなたのみかた \url{https://t.co/iXdrTGMvyk} 
\end{itemize}

〉〉〉 kk\_hironoのリツイート 〉〉〉

\begin{itemize}
\tightlist
\item
  RT
  kk\_hirono(刑事告発・非常上告_金沢地方検察庁御中)|hirono\_hideki(奉納\さらば弁護士鉄道・泥棒神社の物語)
  日時:2021-05-26 13:50/2021/05/26 00:00 URL:
  \url{https://twitter.com/kk\_hirono/status/1397414740951900160} 
  \url{https://twitter.com/hirono\_hideki/status/1397206020867375112} 
  \textgreater{} - 1385:2021-05-26\_00:00:38 \#告発状 \#\#\#\#
  福井刑務所で平成6年11月11日受付となっていた副本と赤い判子のある甲号証(二)と(四)の書類,被告発人長谷川紘之弁護士からの原告側提出書証,ノートとの整合性
  \url{https://t.co/RNknnDgTWR} 
\end{itemize}

〉〉〉 kk\_hironoのリツイート 〉〉〉

\begin{itemize}
\tightlist
\item
  RT
  kk\_hirono(刑事告発・非常上告_金沢地方検察庁御中)|hirono\_hideki(奉納\さらば弁護士鉄道・泥棒神社の物語)
  日時:2021-05-26 13:50/2021/05/25 23:58 URL:
  \url{https://twitter.com/kk\_hirono/status/1397414766075809792} 
  \url{https://twitter.com/hirono\_hideki/status/1397205507975356429} 
  \textgreater{} -
  日本弁護士連合会・金沢弁護士会・法テラスの共催で、「遺言の日」記念行事を実施しました。(平成29年4月17日)|法テラス
  \url{https://t.co/cJoTBta2bJ} 
\end{itemize}

 思ったほど情報がなかったというのが率直な感想ですが,それだけ調べた情報の読み込みに専念し,集中していたということかもしれません。改めてGoogleで,長原悟弁護士と米田弘幸弁護士について調べてみます。

\begin{itemize}
\item
  長原 悟弁護士(長原法律事務所)に法律相談 -
  石川県金沢市、北陸鉄道浅野川線 上諸江駅 \textbar{} Legalus
  \url{https://t.co/rysA75svJw}  ¥\n 〒920-0027 石川県金沢市駅西新町3-1-10
  NEWSビル102
\item
  石川県金沢市駅西新町3-1-10 NEWSビル102 - Google マップ
  \url{https://t.co/qrYgkRtlCS} 
\item
  長原悟 - 株式会社アイ・オー・データ機器 (2016年6月期) 役員の略歴
  (健全!どんぶり会計β版) \url{https://t.co/c1Wa1Twoif}  ¥\n 1968年誕生 ¥\n
  2000年4月弁護士登録 ¥\n 2000年4月木梨・長原法律事務所(現任) ¥\n
  2016年9月監査役就任(現任)
\end{itemize}

 昨夜より格段に内容のある具体的な情報が見つかりました。被告発人木梨松嗣弁護士と同年代という可能性も想定していたため,2000年4月の弁護士登録というのは弁護士としての経歴も比較的浅めですが,そこから比較的若年層を想定していたところ,昭和43年生まれと出てきました。

 何月生まれとはないですが,昭和43年生まれというのは小倉秀夫弁護士と同じ年になると思います。そして小倉秀夫弁護士の弁護士登録は1990年代の初めの方だったように思います。

\begin{itemize}
\tightlist
\item
  小倉 秀夫|スタッフ紹介|事務所紹介|東京平河法律事務所
  \url{https://t.co/2wBrvxSp4g}  ¥\n 1968年生・東京都 ¥\n ¥\n
  1991 司法試験合格 ¥\n 1992 早稲田大学法学部卒業 ¥\n
  1994 司法修習終了(46期) ¥\n 1994 弁護士登録(東京弁護士会)
\end{itemize}

 確認しました。小倉秀夫弁護士の生年はやはり昭和43年でしたが,弁護士登録が1994年ということで,長原悟弁護士の2000年とは6年違い,それほど大きな違いには感じません。

 昨夜も書きましたが,なぜ被告発人木梨松嗣弁護士と同じ法律事務所の長原という弁護士を調べなかったのか自分でも不思議だったのですが,長原悟という下の名前まで確認したのも余り記憶にないことでした。twilog-serch-post
で調べて確認します。

 木梨長原法律事務所にはかなり該当がありそうなので,2つの検索でやってみます。 

\begin{itemize}
\tightlist
\item
  2021年05月26日14時09分の登録:
  「長原」を@hirono\_hideki @kk\_hirono @s\_hironoで検索 28件の該当 2021-05-26\_14:09の記録
  \url{https://kk2020-09.blogspot.com/2021/05/hironohidekikkhironoshirono282021-05\_26.html} 
\item
  2021年05月26日14時09分の登録:
  「長原.+弁護士」を@hirono\_hideki @kk\_hirono @s\_hironoで検索 13件の該当 2021-05-26\_14:08の記録
  \url{https://kk2020-09.blogspot.com/2021/05/hironohidekikkhironoshirono132021-05\_26.html} 
\end{itemize}

2014-01-01 18:24:20 ``木梨・長原法律事務所(石川県金沢市) -
弁護士・法律事務所データベース \url{http://ow.ly/sbzq4}  木梨 松嗣(男性)長原
悟(男性)竹内 昭夫(男性)
所属弁護士が3人ということは今回初めて知りました。弁護士番号をみるとかなり年の離れた世代のようです。''
\url{https://twitter.com/hirono\_hideki/status/418311718482509824} 

2021-05-26 00:12:52 ``- 長原悟(ながはらさとる)弁護士の刑事事件対応情報
\textbar{} あなたのみかた \url{https://t.co/iXdrTGMvyk''} 
\url{https://twitter.com/hirono\_hideki/status/1397209057509978119} 

 2014年1月1日のツイートに,「木梨 松嗣(男性)長原 悟(男性)竹内
昭夫(男性)
所属弁護士が3人」とありますが,これは見つけたページのテキストをそのままコピペしただけのようです。竹内昭夫という弁護士も見覚えがないですが,どちらも続けて調べた形跡がありません。

\begin{itemize}
\tightlist
\item
  弁護士を探す(お住まいの地域から探す)|金沢弁護士会
  \url{https://t.co/G6jbEYog3t} 
\end{itemize}

 竹内昭夫という弁護士は金沢弁護士会にいないようですが,大阪弁護士会に1つ情報がありました。

 「株式会社アイ・オー・データ機器 (2016年6月期)
役員」という情報が見つかった長原悟弁護士ですが,同じページに「
2000年4月木梨・長原法律事務所(現任)」とあるので,初めの弁護士登録のときから被告発人木梨松嗣弁護士の法律事務所で弁護士をしていたようです。

 最近は情報を見かけていないアイ・オー・データ機器ですが,2009年3月まで羽咋市に住んでいた頃に,金沢市の企業であることは知っていたと思います。なにか製品を買ったこともありました。

 現在は独立している情報で,長原法律事務所の住所が石川県金沢市駅西新町3-1-10
NEWSビル102となっていました。よく見ると駅西新町となっていますが,これは初めて見たように思います。ずっと前からあって範囲の広いのが駅西本町でした。

 市場急配センターの平成4年から同じ事務所も平成4年当時は金沢市二口町だったのですが,平成10年代の中頃に住所が現在の駅西本町となっていました。何年か前に市場急配センターの事務所は近くに移転しており,現在も残る前の建物はGoogleマップで研修所などと表示されていました。

 市場急配センターの前の事務所のすぐ横にある金沢中央卸売市場は住所が金沢市西念で,この西念というのも西念と西念町があったように思いますが,けっこう広い範囲になっていて,駅西新町とある住所も以前は西念だったのかもしれません。または,金沢市南新保です。

 駅西合同庁舎のすぐ近くに長原法律事務所がありますが,金沢中央卸売市場の仕事で近くはよく通行していたのですが,いつの間にか出来ていた大きな建物で,少なくとも昭和59年当時にはなかったと思います。

 金沢市の合同庁舎は他に神田か新神田にあるのですが,そちらは昭和59年当時にあったように思います。労働基準局の入る庁舎で,平成9年に一度,市場急配センターの労災問題として相談に行っています。

 平成9年の7月に入ってすぐに私は宇出津から被告発人大網健二に世話をしてもらった金沢市北安江の借家に住むようになったのですが,それから間もない頃のようにも思うのですが,駅西合同庁舎に行って人権擁護局の人と話をしたこともありました。

 平成11年の2月頃には,被告発人大網健二につられて同じ駅西合同庁舎に入ったのですが,そのときは不動産かなにかの登記の話となっていました。登記のことで弁護士の仕事もありそうですが,駅西に法律事務所は見たことがなく,ネットで知った鹿島弁護士が金石街道沿いの事務所だったぐらいです。

 もうずいぶん長くアカウントをみていないですが,プロフィールに寛容な,とあったように記憶します。

〉〉〉 kk\_hironoのリツイート 〉〉〉

\begin{itemize}
\tightlist
\item
  RT
  kk\_hirono(刑事告発・非常上告_金沢地方検察庁御中)|kk\_hirono(刑事告発・非常上告_金沢地方検察庁御中)
  日時:2021-05-26 14:47/2019/07/29 19:17 URL:
  \url{https://twitter.com/kk\_hirono/status/1397429072729317379} 
  \url{https://twitter.com/kk\_hirono/status/1155784487122300928} 
  \textgreater{}
  今朝の初めの予定では、福永活也弁護士を取り上げるつもりでした。鹿島啓一弁護士を優先させたのは、より身近な金沢弁護士会の所属で活動しているということもあります。
\end{itemize}

〉〉〉 kk\_hironoのリツイート 〉〉〉

\begin{itemize}
\item
  RT
  kk\_hirono(刑事告発・非常上告_金沢地方検察庁御中)|s\_hirono(非常上告-最高検察庁御中\_ツイッター)
  日時:2021-05-26 14:47/2019/07/29 21:38 URL:
  \url{https://twitter.com/kk\_hirono/status/1397429185279258627} 
  \url{https://twitter.com/s\_hirono/status/1155819882698592258} 
  \textgreater{} 2019-07-29-164542\_寛容な鹿島啓一 - Google 検索.jpg
  \url{https://t.co/mIUtXuMaFR} 
\item
  2021年05月26日14時48分の登録:
  「鹿島啓一」を@hirono\_hideki @kk\_hirono @s\_hironoで検索 61件の該当 2021-05-26\_14:48の記録
  \url{https://kk2020-09.blogspot.com/2021/05/hironohidekikkhironoshirono612021-05\_26.html} 
\item
  \begin{enumerate}
  \def\labelenumi{(\arabic{enumi})}
  \setcounter{enumi}{2}
  \tightlist
  \item
    プロフィール / Twitter \url{https://t.co/B9hlwDg6YU}  ¥\n
    このアカウントは存在しません
  \end{enumerate}
\end{itemize}

 鹿島啓一弁護士のTwitterアカウントは消滅していたようです。

 2019年7月26日にこの鹿島啓一弁護士について集中的に調べているのですが,これはなんとなく残っていた記憶どおりで,この当日か翌日辺りに,島根県安来市の奇跡を思い出したことをきっかけに,恋路海岸に行ったように思います。

 時刻は15時14分です。2019年7月27日に撮影した写真をみていたのですが,この日は特別な日で,他に忘れていた発見もありました。前からまとまった写真の見やすい記事を作成しておきたいと考えいたので,この機会にやっておきたいと思います。

 時刻は20時01分です。まもなく皆既月食が始まるらしいので忘れないようにしたいところですが,夕方に大きな発見がありました。まさかと思ったありえない場所から出てきたものがあります。その後,宇出津新港に買い物に行っていました。30分ほどか前に戻っています。

 時刻は20時10分です。しばらく外にいたのですが,空に何も見えませんでした。港の方がいくぶん明るいのはいつものことです。夕方は雲がなかったか,少なかったように思うのですが,星空は見えませんでした。

 データの並び順が間違っていることに気がついたので,次のようにデータベースのデータを削除しオートインクリメントの番号も修正しました。

\begin{lstlisting}
mysql blog -e 'DELETE FROM blogger_pfoto WHERE entry LIKE "%2019-07-27_恋路〜珠洲〜町野%"\G'
mysql blog -e 'ALTER TABLE blogger_pfoto AUTO_INCREMENT=75235;'
\end{lstlisting}

 そして次のようにデータベースへの登録をやり直しました。

\begin{lstlisting}
pfoto-db-blogger-list_2021-05.rb \url{https://kk2020-09.blogspot.com/2021/05/202105262019-07-27.html} 
mysql blog -e 'SELECT * FROM blogger_pfoto order by id DESC limit 1\G'
 *************************** 1. row ***************************
         ID: 75325
      fname: 2019-07-27_172157_恋路〜珠洲〜町野_.jpg
        url: \url{https://kk2020-09.blogspot.com/2021/05/202105262019-07-27.html#91} 
      entry: 奉納\危険生物・弁護士脳汚染除去装置\金沢地方検察庁御中_2020: 2021年05月26日の記録:写真資料:2019-07-27_恋路〜珠洲〜町野
   category: 写真
      ctime: 2019-07-27 17:21:57
create_time: 2021-05-26 21:54:02
\end{lstlisting}

 時刻は22時03分です。見出しを確認したところ「金沢市大手町7−34という番地の住所の不思議な疑問点」という後段の部分が未着手でしたが,その前にもう一人の被告発人木梨松嗣弁護士の法律事務所の弁護士,米田弘幸弁護士についても取り上げていなかったと思います。

 米田というのは人の名前として他に見たり,聞いた覚えはなく,地名として羽咋市の周辺で見かけていたような気がします。

\begin{itemize}
\tightlist
\item
  石川県 米田 - Google マップ \url{https://t.co/ZX13Rrv8Pa} 
\end{itemize}

 石川県に米田という地名はありませんでした。少し思い出したのですが,米出ではなかったかと思います。

\begin{itemize}
\tightlist
\item
  米出 I.C. - Google マップ \url{https://t.co/81Mm6K5ad9}  ¥\n 〒929-1331
  石川県羽咋郡宝達志水町米出 米出IC
\end{itemize}

 Googleマップで見ると,赤い線で囲まれた範囲の大半が,能登カントリークラブとなっています。羽咋市に住んでいるとき,当時の能登有料道路を白尾インターで乗り,夜の22時か23時以降に米出インターで降りると,料金所が閉鎖され料金がいらなかったように思います。

 米出インターで降りると,ゴルフ場のホテルのような細長く高い建物があったのですが,それもよく印象に残っています。今も建物がそのまま残っているのか不明ですが,昭和58年頃には既に建物があったように思います。被告発人木梨松嗣弁護士辺りは常連なのかと想像しました。

 平成3年12月22日頃,被告発人大網健二と昼に会ったときもゴルフの話があり,河北潟でなんというのかわかりませんがゴルフの玉を打った後,津幡町の打ちっぱなしのゴルフ場に行ったことも記憶にあります。最初の方に,金石街道沿いパチンコオークラの隣あたりの花屋でバラの花の配達を頼みました。

 その花屋に被告発人大網健二が同行していたのか今は記憶にないのですが,そのあと国道8号線バイパス沿いで,石川中央病院の近くの大きな喫茶店に入り,被告発人大網健二と一緒に店にいたことははっきり記憶にあります。

\begin{itemize}
\item
  2021年05月26日22時06分の登録:
  「米田弘幸」を@hirono\_hideki @kk\_hirono @s\_hironoで検索 14件の該当 2021-05-26\_22:06の記録
  \url{https://kk2020-09.blogspot.com/2021/05/hironohidekikkhironoshirono142021-05.html} 
\item
  2021年05月26日22時07分の登録:
  「米田」を@hirono\_hideki @kk\_hirono @s\_hironoで検索 47件の該当 2021-05-26\_22:07の記録
  \url{https://kk2020-09.blogspot.com/2021/05/hironohidekikkhironoshirono472021-05.html} 
\item
  2019-10-02 04:33:38 ``米田弘幸 弁護士 - Google 検索
  \url{https://t.co/J8IWP39SCX''}  \url{https://t.co/SyZUeq4O56} 
\end{itemize}

 記憶になかったですが,米田弘幸弁護士の方は,2019年10月2日にGoogleで調べた形跡がありました。

 米田弘幸弁護士をGoogleで調べた形跡はあるものの,これという情報は見つからず,今年の1月2日のツイートでは,「思い出せないですが,この米田ではなかった気がしますし,米田弘幸という名前は今見ても初めて目にしたように思える名前です。」というぐらいに忘れていたようです。

 平鍛造の関係で「評議員に長原悟弁護士,理事に米田弘幸弁護士」という関係性しか,昨夜のGoogleの検索で記憶にないのですが,改めてGoogleで米田弘幸弁護士を調べてみます。

\begin{itemize}
\tightlist
\item
  日本弁護士連合会・金沢弁護士会・法テラスの共催で、「遺言の日」記念行事を実施しました。(平成29年4月17日)|法テラス
  \url{https://t.co/FDtLZLvItb} 
\end{itemize}

 そういえば,昨夜も上記のページで米田弘幸弁護士が講師という講演会の写真をみていたのですが,ページに米田弘幸弁護士の名前が見当たりませんでした。あるのは写真です。講師として名前があり,その下の写真に弁護士バッチらしいものをつけた,それらしい人物の写真があります。

 率直に真面目そうで頭の良さそうな人物に見えます。前にどこかで見たような顔写真でもあるのですが,思い出せずにいます。ページ内検索をしても米田で結果は0ですが,なぜGoogleの検索に出てきたのは不思議です。「日本弁護士連合会・金沢弁護士会・法テラスの共催で、「遺言の日」」とか。

 時刻は22時40分です。さきほど外に出るとぼやけた満月のような月が見えたので,スマホで撮影をしてきたのですが,雲が厚いのか月の明かりが特別強いのか,生まれてこのかた見たことのないようなお月様に見えました。朧月夜とは違っているような気がします。

\begin{itemize}
\tightlist
\item
  朧月夜 - Google 検索 \url{https://t.co/CrMtbpSkJb} 
\end{itemize}

 よく考えてみると朧月夜の意味がわかっていなかったので,Googleで検索をしてみましたが,写真を見たところでは,ちょっとわからない感じで,雲が少し差し込んだだけの月夜に見ます。

\begin{itemize}
\item
  朧月・朧月夜の意味、由来って何か知ってます?いつ使う言葉なの?
  \textbar{} 雑学トレンディ \url{https://t.co/YMlRmsngHH} 
  朧月とは、霧や靄(もや)などに包まれて霞んで見える月のことです。 ¥\n
  ¥\n なので、朧月夜は朧月がでている夜のこととなります。
\item
  朧月・朧月夜の意味、由来って何か知ってます?いつ使う言葉なの?
  \textbar{} 雑学トレンディ \url{https://t.co/YMlRmsngHH} 
  朧気に見える月・・・朧月というわけです。 ¥\n ¥\n
  先程の「朧月夜」の歌詞にも''霞ふかし''という言葉がありますよね。 ¥\n
  ¥\n まさに朧月夜がどういうものか、情景を歌っているのです。
\item
  朧月・朧月夜の意味、由来って何か知ってます?いつ使う言葉なの?
  \textbar{} 雑学トレンディ \url{https://t.co/YMlRmsngHH} 
  意味は霧や靄などに包まれて霞んで見える月ですが、実は本当は季節が決められているのです。
  ¥\n ¥\n それは「春」です。 ¥\n ¥\n
  そう、最初にご紹介した歌でも「菜の花畠に」「春の風」といった言葉が
\end{itemize}

 そういえば,昨日の夕方は一時的でしたが嵐のような風が吹き,あとになって,あれが夕嵐なのかと思いました。

\begin{itemize}
\tightlist
\item
  朧月夜(日本の童謡) - YouTube \url{https://t.co/DMweMsHAlA} 
\end{itemize}

 このYouTube動画のタイトルで,朧が月に龍と書く漢字なのだと気が付きました。数日前にスーパーで「おぼろ豆腐」というのが気になって,今日もどんたく宇出津店で1つ小さいのを買ってきたのですが,手に取ると輪島市町野町の谷内豆腐店のようでした。

 冷蔵庫から出してきて確認すると,商品名がそのまま「おぼろ月夜」となっていました。

\begin{itemize}
\tightlist
\item
  谷内の豆腐 おぼろ月夜 - Google 検索 \url{https://t.co/vwaVDQ4ZoY} 
\end{itemize}

 そのものずばりの商品は見つかりませんでした。

〉〉〉 kk\_hironoのリツイート 〉〉〉

\begin{itemize}
\tightlist
\item
  RT
  kk\_hirono(刑事告発・非常上告_金沢地方検察庁御中)|estefoods(さいはての谷内のおとうふ【公式】)
  日時:2021-05-26 23:02/2021/05/26 16:20 URL:
  \url{https://twitter.com/kk\_hirono/status/1397553775024300041} 
  \url{https://twitter.com/estefoods/status/1397452597380849665} 
  \textgreater{} 小木のお客さんからの注文で、 味付がんもどき(大)50
  味付がんもどき(小)50 味付厚揚28 味付油揚28 味付すし揚げ10
  があり、冷凍してイカ釣り漁船に乗せて持っていくそうです🙄
  長旅にお供させて頂いて嬉しいです😊
\end{itemize}

〉〉〉 kk\_hironoのリツイート 〉〉〉

\begin{itemize}
\tightlist
\item
  RT
  kk\_hirono(刑事告発・非常上告_金沢地方検察庁御中)|estefoods(さいはての谷内のおとうふ【公式】)
  日時:2021-05-26 23:04/2021/04/25 07:54 URL:
  \url{https://twitter.com/kk\_hirono/status/1397554187605512194} 
  \url{https://twitter.com/estefoods/status/1386091116177461249} 
  \textgreater{} 輪島市下山町で熊が捕獲されたみたい🙄
  \url{https://t.co/8YzAzJLROS} 
\end{itemize}

 輪島で,捕獲されたクマが逃げたという能登町の告知放送があったように思います。

 おぼろ月夜ということで思い出したのは,まだ曲名がはっきり思い出せないですが,輪島市町野町になるのかその近くの金蔵の風景でした。「まどいせん」という歌詞は思い出しています。原曲は蛍の光と同じで,ヨーロッパあるいは同じイングランドであったかもしれません。

 最初に聞いたのか意識するようになったのが金沢刑務所の拘置所,未決にいるときで夕方に,その音楽が流れていました。曲名がわかったのはずっとあとです。

\begin{itemize}
\tightlist
\item
  童謡・唱歌 遠き山に日は落ちて byうたこ - YouTube
  \url{https://t.co/tSTSyybhRS} 
\end{itemize}

 思い出した歌詞でGoogle検索すると,すぐに入力候補が出てきたのですが,他の曲名があったような気もします。

\begin{itemize}
\tightlist
\item
  【軍歌】勇敢なる水兵 - Yukannarusuihei - - YouTube
  \url{https://t.co/FWnr3zXPMb} 
\end{itemize}

 YouTubeの自動再生で始まりました。

\begin{itemize}
\tightlist
\item
  66期司法修習の終了者名簿 \textbar{} 弁護士山中理司のブログ
  \url{https://t.co/6nEfJS7SZp} 
\end{itemize}

 米田弘幸という名前があるのですが,「米田 弘幸」となっています。同姓同名の可能性は否定できないですが,その確立は低く,被告発人木梨松嗣弁護士と同じ法律事務所の米田弘幸弁護士は,66期司法修習生の可能性が高そうです。今最新の登録弁護士で73期ではないかと思います。

\begin{itemize}
\item
  加藤 俊治 \textbar{} 弁護士山中理司のブログ \url{https://t.co/HVYasTkTfg} 
  いつも当サイトをご覧頂きありがとうございます。「加藤 俊治」で検索しましたがページが見つかりませんでした。
\item
  加藤俊治 \textbar{} 弁護士山中理司のブログ \url{https://t.co/c8pGFP4JE4}  88
  加藤俊治 44 期 1966年7月26日 53 歳 2019年3月11日 宮崎地検検事正 (
  最高検検事 )
\end{itemize}

 これまで裁判官の検索で度々お世話になってきたサイトですが,司法修習生でも検索できると知り,やってみたところ,これも同姓同名の可能性は完全に否定できないものの,加藤俊治検事と思われる情報が見つかりました。本日,不思議な発見で,不起訴処分通知書を発見しています。

\begin{itemize}
\item
  下平豪 \textbar{} 弁護士山中理司のブログ \url{https://t.co/wuw5joH96M} 
  いつも当サイトをご覧頂きありがとうございます。「下平豪」で検索しましたがページが見つかりませんでした。
\item
  浜崎一 \textbar{} 弁護士山中理司のブログ \url{https://t.co/Am1xkGLaau} 
  いつも当サイトをご覧頂きありがとうございます。「浜崎一」で検索しましたがページが見つかりませんでした。
\item
  下平 \textbar{} 弁護士山中理司のブログ \url{https://t.co/IkZ1xUY9UY} 
\item
  深澤諭史 \textbar{} 弁護士山中理司のブログ \url{https://t.co/9Iv6cbRoAR} 
\end{itemize}

 時刻は23時34分です。時間が進むのが早く感じますが,外に出たところ,さらに月が薄く霞んで見えました。

\begin{itemize}
\item
  海 - YouTube \url{https://t.co/GVdggog1yV} 
\item
  唱歌 海 - YouTube \url{https://t.co/CbAL779Bui} 
\item
  朝の音楽 役場 - YouTube \url{https://t.co/zjkKv0SWOS} 
\item
  文部省唱歌 とんび - YouTube \url{https://t.co/amdIqpYiOI} 
\item
  【童謡】とんび - YouTube \url{https://t.co/FGuTJGpQSi} 
\item
  【高齢者・認知症の人が喜ぶ!】なつかしい童謡・唱歌集 - YouTube
  \url{https://t.co/TolhsWmMct}  みなと(13:44~)
\item
  港 文部省唱歌 尋常小学3年 - YouTube \url{https://t.co/dLpmXxu5ur} 
\end{itemize}

 時刻は5月27日10時49分です。起きたとき,パソコンの電源が入ったまま,モニターの電源は落ちたままでしたが全く反応しない状態となっていました。KDEのログイン時に出るエラーもそのままなので,このあとUbuntuのシステムをインストールし直すことにしました。準備も出来ました。

\begin{lstlisting}
py37_env ❯ sudo tune2fs -l /dev/sda3 | grep 'Filesystem created'
Filesystem created:       Mon Jul 13 09:16:05 2020
py37_env ❯ sudo tune2fs -l /dev/nvme0aie6 | grep 'Filesystem created'
Filesystem created:       Mon Jul 13 09:16:00 2020
\end{lstlisting}

 デバイス名を少し変更したのですが,今のLinuxのシステムが昨年7月13日9時16分頃に作成されたものと確認しました。何月にパソコンを買ったのかも記憶になかったのですが,定額給付金で買えたパソコンです。

\begin{itemize}
\item
  〈〈〈 2021/05/27 11:05:50 Linux Emacs: 〈〈〈
\item
  〉〉〉 Linux Emacs: 2021/06/01 11:32:35 〉〉〉
\end{itemize}

 告発状の記述としては2021/05/27
11:05:50のツイート以来の再開となります。この間、Ubuntuを新規にインストールしデータの移行もしています。パーティションが別になるデバイスはマウントポイントの再設定で従来どおり使えています。パソコンはずいぶん調子が良くなりました。

 昨日の5月31日は、昼前に宇出津新港の職業安定所に行ったあと、車に乗せてもらって珠洲に行っていました。鉢ケ崎海水浴場まで行ったのですが、蛸島町になるはずかと思います。その少し先が珠洲市三崎町になります。

 蛸島についても書いておきたいことがあったのですが、蛸島町の隣になるこの珠洲市三崎町というのは、私が長い間、被告発人木梨松嗣弁護士との関係で勘違いをされられ続けてきた場所でもあります。

 その前に、被告発人木梨松嗣弁護士の法律事務所について、記述を終わらせておきたいのですが、Googleマップのストリートビューで最初に建物を見た頃から、記憶ある場所とは違っていると感じていました。

 現在の金沢地方裁判所の正面入口というのはテレビでも出入りの様子を見たことがないのでよくわからないのですが、久保利英明弁護士が出てきたニュース映像では、横の方に正面入口があるように思えました。

〉〉〉 kk\_hironoのリツイート 〉〉〉

\begin{itemize}
\tightlist
\item
  RT
  kk\_hirono(刑事告発・非常上告_金沢地方検察庁御中)|s\_hirono(非常上告-最高検察庁御中\_ツイッター)
  日時:2021-06-01 11:56/2021/06/01 11:53 URL:
  \url{https://twitter.com/kk\_hirono/status/1399560578662686728} 
  \url{https://twitter.com/s\_hirono/status/1399559780599943170} 
  \textgreater{}
  2021-06-01-115236\_金沢地方裁判所 - Google 検索 - Google Chrome.jpg
  \url{https://t.co/h8ioIVMkTx} 
\end{itemize}

 Googleで金沢地方裁判所を調べると、見たことのない角度から撮影した写真が出てきたのですが、上記のスクリーンショットにある写真です。これをみたとき、正面入口は以前の建物と同じ位置になるのかとおもったのですが、ストリートビューで確認すると、撮影された位置がよくわからなくなってきました。

\begin{itemize}
\tightlist
\item
  金沢地方裁判所 案内 - Google 検索 \url{https://t.co/GzQyqHWg3f} 
\end{itemize}

 金沢地方裁判所の正面出入り口を確認したいだけなのですが、それだけでも情報が簡単に見つかりません。建物の中には案内の図面のようなものがありそうですが、ネットに情報は出されていないのかもしれません。

\begin{itemize}
\tightlist
\item
  金沢地方裁判所金沢簡易裁判所 \textbar{} 裁判所 \url{https://t.co/sBzfMZUILQ} 
\end{itemize}

 上記のページに金沢地方裁判所の建物と周辺の見取り図がありますが、イラストの正面出入り口が金沢城公園に向いています。前の金沢地方裁判所の建物というのは、この見取り図だと建物の左手、兼六園下交差点の角に敷地への出入り口がありました。

 まるで古い病院のような建物に見えて、昭和の時代の宇出津病院とも似ていたのですが、割と広い駐車場の奥に金沢地方裁判所の建物の正面出入り口がありました。

 また、現在は金沢城公園となっていますが、平成9年8月頃はまだ造成工事が終わったばかりのような状況で、U字溝の埋設に仕事に何度か行ったことがありました。道路も未舗装で、土がむき出しの状態がほとんどであったように記憶にあります。

 見取り図では、兼六園下交差点から検察庁前交差点に向かうと、右折禁止とあり、ほぼ直角に右に曲がっているのですが、この曲がったところの道路の左側に被告発人木梨松嗣弁護士の法律事務所がありました。

 見取り図では右に曲がった先で、次に左に大きく曲がるところに信号機があって味噌蔵町とあります。平成3年当時も広くてほぼ直角に曲がる道路でした。私の記憶では信号機がなかったのですが、いつの間にか信号機ができていたのかと初めて知りました。

 味噌蔵町という地名は少し聞いたことがあったのですが、金沢市内のどのあたりになるのかはしらずにいました。これは近年、旧地名として復活したものかもしれません。

 ストリートビューでみると、たしかに信号機があって味噌蔵町と標識があります。その味噌蔵町の交差点の停止線から手前に2つ目の建物が現在の被告発人木梨松嗣弁護士の法律事務所の建物となっています。

 私の記憶では同じ通りの同じ並びになりますが、逆に金沢地方裁判所から近い方で、2つ目か3つ目の建物であったという記憶です。刺繍のような大きな看板が壁面に彫り込まれていたような記憶もあります。

 まるで暴力団組長の自宅兼事務所という印象もあったのですが、ビルで2階建てであったように思います。ガレージのような大きな駐車場も印象的だったのですが、車は見ておらず、シャッターになっていたようにも思います。

 事務所の出入り口は向かって右手にありましたが、建物正面の左側3分の2から4分の3はガレージのような駐車場になっていたという記憶です。まるで要塞のようにも見える堅牢そうな建物で、かなり目立っていました。

 先日、被告発人木梨松嗣弁護士の法律事務所の住所と番地が平成4年当時と全く同じだと確認できたのですが、現在の建物と違っていることは、最初にストリートビューを見た頃から感じていました。10年ほど経つように思います。

 平成9年から平成11年の間は、金沢地方裁判所への用事で、その被告発人木梨松嗣弁護士の法律事務所の前を車で通ることがありましたが、基本的に裁判所の用事以外で通りかかる道ではなかったと思います。

 一方通行の進入禁止となっている金沢地方裁判所前の道路ですが、市内配達をしていた頃の記憶では、この進入禁止となっている方向から入っていたように思います。突き当りの右手が金沢地方検察庁ですが、その向かいあたりに配達先がありました。

 建物はNTTだったように思います。ほぼ毎回配達があったような気もするのですが果物を届けていました。建物の中に入ったときの記憶はないのですが、今考えると、なかに病院によくある売店があったのかと思います。

 現在と同じ進入禁止だと、さきほどの味噌蔵町の信号機の先にある、大手町交差点を左折し金沢地方検察庁前に至ることになります。交差点の大手町は今、Googleマップで見たところですが、記憶になり交差点の名前になります。

 その大手町交差点の右側がGoogleマップに「卯辰山公園線」という表示もありますが、卯辰山に向かう道路です。市内配達では一週間に1回ぐらいだったと思いますが、卯辰山のサニーランドという動物園に配達があり、逆に卯辰山の方から降りてくることがほとんどだったと記憶します。

 サニーランドに向かうとき卯辰山は鈴見町の方から登っていました。田井町の交差点にある小森という商店のことは、令和3年3月31日付告発状にも記述をしたように思います。また、その近くで被告発人若杉幸平弁護士の自宅のような建物をストリートビューで見ています。

 さきほどの検察庁前の交差点ですが、金沢地方裁判所の方から来るとその手前の右側に被告発人若杉幸平弁護士の法律事務所の建物がありました。若杉幸平弁護士の法律事務所は、前回以来調べていたいのですが、かれこれ2年ほど経つかもしれません。

 被告発人木梨松嗣弁護士の法律事務所に入ったことは一度だけありました。平成9年の秋だったように思います。その建物の内部も印象的だったのですが、これは別のところで記述をしたいと思います。

\begin{itemize}
\tightlist
\item
  金沢地方裁判所 - Wikiwand \url{https://t.co/q4t02m706H} 
  2010年9月まで使用された旧庁舎(2010年4月4日撮影)
\end{itemize}

 「金沢地方裁判所 昭和」とGoogle検索したところ、上記のページにある写真がみつかりました。建物が3階建てです。私の記憶では2階までしかなかったのですが、名古屋高裁金沢支部もあるのに2階建てにしては敷地面積が狭すぎるという疑問はありました。

 建物の外見というか建築のデザインですが、これまで以上に石川県立水産高校本校の建物に似ていると思いました。石川県内では他にも同じような建物を見かけるのですが、施工業者が同じとも考えられます。テレビで見た現在の富山地裁の建物も同じであったように思います。

\begin{itemize}
\tightlist
\item
  富山地裁建て替え新年度に着手 最高裁が敷地調査 高岡支部は設計業務|社会|富山のニュース|富山新聞
  \url{https://t.co/IHXrReQ329} 
\end{itemize}

 テレビで見たときほど似ているとは思わなかったですが、北国らしい雰囲気の古い建物とは思います。富山地裁の場所も前にGoogleマップで調べたように思うのですが、ほとんど憶えておらず、仕事で行ったような場所ではなかったように思います。

\begin{itemize}
\tightlist
\item
  富山市公設地方卸売市場関連店舗・その他富山中央市場冷蔵(株) から
  富山地方・家庭・簡易裁判所 - Google マップ \url{https://t.co/GKZLMzswk6} 
\end{itemize}

 最初、Googleマップの上の方を岐阜県高山市方面、下の方を海だと思いこんでいたので混乱しました。富山市内の地図はたまに見ると、前とはまったく違っているように思えるのですが、富山城というのも数年前テレビの番組で見て初めて知りました。

 その富山城の向かいあたりに桜木町がありますが、辺田の浜に住んでいる頃、テレビのCMでよく見た地名でした。実際に行った記憶ははっきりしないのですが、飲み屋の多い繁華街と聞いており、テレビのCMもそのようなものでした。夜のとばりがなんとかというナレーションが記憶に残っています。

 これまでに何度か書いていますが、辺田の浜の家では、石川テレビが映らず、富山放送や北日本放送が映っていました。他にあるのはMRO北陸放送とNHKだけだったと思うので、テレビの半分以上は富山湾の対岸になる富山県の放送を見ていたように思います。

 その辺田の浜の家から宇出津の小棚木に引っ越してきたのが昭和50年4月ですが、石川テレビは映るようになったものの富山放送は映らなかったと思いますし、1チャンネルだったと記憶にある北日本放送というのも、宇出津の家では見た記憶が残っていません。

\begin{itemize}
\tightlist
\item
  北日本放送|KNB WEB {[}テレビ{]} 1ch/{[}ラジオ{]} AM738kHz/FM90.2MHz
  \url{https://t.co/bkVEi3EWB4} 
\end{itemize}

 現在の北日本放送というのはテレビではなくラジオ放送というイメージもあったのですが、ただいま放送中としてミヤネ屋が出ています。石川県ではテレビ金沢の放送になるのですが、そのテレビ金沢が開局したのも平成2年ぐらいだったように思います。

 2,3ヶ月ほど前に2階で見つけ、近くに置いていた金沢市内の地図があったことを思い出し調べたのですが、最初の閲読許可証が平成6年2月23日から同年3月4日となっていました。未決57番となっていますが、判決が確定したのがちょうどその頃になると思います。

 福井刑務所では所持許可証として平成6年12月15日から訴訟終結までの間となっていて、「訴訟用書類として認む」とあります。

 「1992年第24刷発行 エリアマップ/旺文社 金沢市」とあります。この地図を見て初めて気がついたのですが、間に道路が一本あって、それが大手町の6番と7番の境界となっています。これは現在も同じなのかもしれません。

 大手町交差点と検察庁前交差点の間には、2本の路地があることがわかったのですが、どちらの路地も裁判所前の一方通行の道路と道幅が変わらないように見えます。大手町7−34の34の場所は変わったように思えてならないのですが、7番は変わりそうにありません。

\begin{itemize}
\tightlist
\item
  金沢(その1) - YouTube \url{https://t.co/ulvsVZoYf4} 
\end{itemize}

 上記の動画の再生4分11秒辺りから4分25秒あたりにかけ、ちょうど被告発人木梨松嗣弁護士の法律事務所の前の通りが撮影されていますが、それらしい建物は見当たりません。元は個人医院ではなのかと思っていた現在の法律事務所の建物も見当たりません。

 不鮮明で信号機横の標識の文字が読み取りませんが、今の味噌蔵町の信号機で、ちょうど赤信号で停止したのに気が付きました。

 もう一つ気になっていたのが、その味噌蔵町のカーブの角辺りにあったラーメン店で、上記の動画でも店の存在が確認できなかったのですが、平成3年のたぶん4月の終わり頃、金沢市場輸送で被告発人安田敏の採用が決まったあとに、立ち寄った店という記憶なのです。

 一応ラーメン屋としましたが、中華料理店であったのかもしれず、記憶にあるのは餃子を食べたようなことです。たぶん被告発人安田敏に勧められて注文したように思います。

 金沢市場輸送の会社から金沢市花里町の被告発人安田敏のアパートに向かう途中だったと思います。面接は19時を過ぎていたと思いますが完全に暗くなっていました。遅い時間に変わった場所に店があるという印象もありました。

 平成16年あたりの告訴状か告発状でもそのまま記載したと思うのですが、被告発人安田敏の供述調書の住所が「金沢市花里町壱五の五番地藤村アパート二階一号室」となっていました。タウエさんに被告発人か被告訴人の住所をできるだけ特定するように言われていました。

\begin{itemize}
\tightlist
\item
  〒920-0951 石川県金沢市花里町15−5 - Google マップ
  \url{https://t.co/wh3dYLFWKp} 
\end{itemize}

 建物の写真が出てきたのですが、平成3年当時と同じ建物かもしれません。左側に階段があってそこだけ新しいアパートのようになっていますが、その2階のへやが被告発人安田敏のアパートでした。正確には妻となった女性のアパートで、後に転がり込んだという話でした。

 金沢市内から向かうと途中に団地のような建物が左手にあったと記憶します。官舎と聞いていました。警察の官舎としてニュースを見かけたような記憶もあります。同僚の警察官が女性の部屋にベランダから不法侵入したような話だったように思います。ネットで見かけただけかもしれません。

\begin{itemize}
\tightlist
\item
  15'7 石川県警 警察官舎不法侵入:巡査長、容疑で書類送検 金沢中署 -
  警察官の不祥事・組織 犯罪集団の日本の警察 \url{https://t.co/pXgFXLgZce} 
  金沢市花里町の4階建て警察官舎で、男性警察官が入居する1階部屋のベランダに正当な理由なく侵入したとしている。
\end{itemize}

 それらしい情報が見つかりましたが、2015年7月12日の投稿のようです。ニュース記事の引用のような部分に、「送検容疑は5月20日午後10時45分ごろ、金沢市花里町の4階建て警察官舎で、」とあるのですが、私の記憶では10年ほど前だったように思います。

 男性警察官が男性警察官の部屋に侵入したとありますが、女性問題が絡んでいたとも思われます。

\begin{itemize}
\item
  奉納\さらば弁護士鉄道・泥棒神社の物語さんはTwitterを使っています
  「北陸農政局係長が住居侵入で逮捕とか。わいせつ目的という。NHKの石川県内ニュース845。花里の官舎の住居侵入の警察官も依願退職とその前にやっていた。すっかり忘れていた。」
  / Twitter \url{https://t.co/sqMHUg5TNJ} 
\item
  ./kk\_hirono2021-06-01\_152503.csv:2015-05-22 16:00:05
  ``金沢市内の花里という地名ですが、平成3,4年当時、被告訴人YSNが住んでいた場所であり、向かい側に団地のような建物が並んでいて、被告訴人YSNが官舎だと話していたことで記憶にあったのですが、警察なのかという細かい話は聞かなかったのか、記憶には残っていません。''
  \url{https://twitter.com/kk\_hirono/status/601643677161664513} 
\item
  ./kk\_hirono2021-06-01\_152503.csv:2015-05-22 15:57:26
  ``銀行で見た北國新聞の記事では「花里の官舎」という文字が、初めに強く気を引いた記事になっていました。花里に官舎があるということは知っていたのですが、裁判所や検察庁の官舎なのかと考えることがあったので、警察の官舎だと確認できたのが意外でした。''
  \url{https://twitter.com/kk\_hirono/status/601643012280557568} 
\end{itemize}

 銀行で北國新聞を見たという記憶もないのですが、銀行に新聞をみかけなくなったのは新型コロナウィルスの問題が起きてからかもしれません。

 そういえば、石川県の緊急事態宣言が延長されたのかニュースを見ていないのですが、延長されていなければ昨日の5月いっぱいで終わっているのかもしれません。パソコンでの北國新聞の記事DVDの閲覧ですが、平成27年以降となっていたように思います。

\begin{itemize}
\tightlist
\item
  能登中央図書館|コンセールのと|能登町観光・地域交流センター
  \url{https://t.co/VUB96tKhOa} 
\end{itemize}

 写真で確認したのですが、この前借りてきた3冊の本が5月23日でした。14日を過ぎていることも考えたのですが今日で9日目のようです。

 味噌蔵町の交差点の辺りで夜に餃子を食べたという記憶のある店ですが、おでん屋のような飲み屋に近い雰囲気があったようにも思います。やたらと強く印象に残る店でしたが、平成3年のことなのでさすがに記憶は衰えています。

 餃子といえば、被告発人安田敏の案内で小立野の有名な店に行ったこともあったのですが、話題になっていた店で、その後、テレビでも見かけたように思いますが、テレビで見たのはここ10年以内かもしれません。番号のような名前の店でしたが、小立野と餃子で調べれば情報は見つかりそうです。

\begin{itemize}
\tightlist
\item
  第7ギョーザの歴史「金沢ギョーザ物語」柏野幸一と辰子からはじまる金沢のギョーザの物語!金沢では餃子ではなくギョーザなのだ。
  \textbar{} 金沢マニアックマガジン ビューティーホクリク
  \url{https://t.co/qR21x33y3Z}  第七・八ギョーザの店 (金沢市小立野)
\end{itemize}

 他にも番号付きの店があるようですが、たぶん第七餃子で、他の番号は記憶にありません。

 金沢カレーの方は全国的な知名度となりましたが、金沢餃子というのは、個人的に聞いたことがなかったと思います。小立野の店だけが一時期やたらと話題になっていました。小立野は学生街のような雰囲気があり、たぶん住所が金沢市宝町となっていた金沢大学病院も近くにあります。

\begin{itemize}
\item
  明治32年 加賀・金沢・兼六園方面への旅 \textasciitilde1899
  Kaga,Kanazawa,Kenrokuen\textasciitilde{} - YouTube
  \url{https://t.co/bHAlCyUi4S} 
\item
  最近の出来事|法務専攻(法科大学院)|法学研究科|大学院|金沢大学
  \url{https://t.co/3i6cBVOY8g} 
  金沢大学法曹会会長 山越茂先生,金沢大学法曹会副会長 木梨松嗣先生の4名の先生方にお越しいただき,祝辞を賜りました。
\item
  石川県金沢市大手町 (172010700) \textbar{}
  国勢調査町丁・字等別境界データセット \url{https://t.co/yaZMXE69Su} 
\item
  「地番」と「番地」の違いとは?地番の調べ方と活用法 -
  ベンチャーサポート不動産株式会社 \url{https://t.co/8wEHkBlXSD} 
\item
  住所の表し方で、○番地と○番○号のどちらが正しいですか \textbar{}
  松山市 よくある質問と回答集 \url{https://t.co/Z5fFkwn2dL} 
\item
  住居表示についてよくある質問 松本市ホームページ
  \url{https://t.co/K8c58w0HOu} 
\item
  住居表示の申請について \url{https://t.co/TXnZczHZpG} 
\item
  住居表示の決め方 \url{https://t.co/YpRQMpe0Kz} 
\end{itemize}

〉〉〉 kk\_hironoのリツイート 〉〉〉

\begin{itemize}
\item
  RT
  kk\_hirono(刑事告発・非常上告_金沢地方検察庁御中)|sakamobi(sakamobi.com)
  日時:2021-06-01 19:49/2021/06/01 12:29 URL:
  \url{https://twitter.com/kk\_hirono/status/1399679467195273218} 
  \url{https://twitter.com/sakamobi/status/1399568640932933635} 
  \textgreater{} 【画像あり】【朗報】中川翔子さん、免許更新した結果
  \url{https://t.co/38f2859JFo} 
  ホンマは「薔子(しょうこ)」って名前にしたかったけど薔の字が名前で使える漢字でなかったので平仮名。しかも「ょ」の字が大きかったので「しようこ」と戸籍に登録された😂😂😂
  ※リンク先に画像あり
\item
  金沢市 大手町7−34 - Google 検索 \url{https://t.co/EVHPpGiLp3} 
\item
  金沢市大手町7−34 - Google 検索 \url{https://t.co/Kmf6XsmPzd} 
\end{itemize}

 時刻は20時00分です。どうにも納得のいかないのが金沢市大手町7−34という被告発人木梨松嗣弁護士の法律事務所の住所です。図書館に行ったのですが過去の金沢市内の住宅地図を探し出すことはできませんでした。

 そのまま宇出津新港のどんたく宇出津店に買い物に行くつもりだったのですが、Aコープ能都店の前で玉ねぎをもらったことがきっかけで、Aコープ能都店で半額になっていた紅鮭と他に買い物のをしていったん家に戻り、迷ったのですがもう一度でかけて予定通りどんたく宇出津店で買い物をしてきました。

 その前に、ホームセンタームサシで、焼き魚用の網を買いました。何年かぶりに魚焼き用の網を買ったのですが、なんとかセンサー対応となっていました。値段は1500円弱だったと思います。値のはる商品でしたが数年間、フライパンに百均のクッキングシートのようなものを使って魚を焼いていました。

 使ってみないとわからないですが、網で焼く魚は数年ぶりです。フライパンだと手間は掛からないのですが味が今一つで、魚を焼いて食べる機会も少なくなっていました。今一つではないぐらい味が落ちていたと思います。前から網は欲しかったのですが、いよいよ思い切って購入したことになります。

 まだ外が薄明るいうちに家に戻ったのですが、すぐに気になっていた被告発人木梨松嗣弁護士の法律事務所の住所について調べました。見れば見るほどなっとくのいかない医院のような建物の写真が出てくるのですが、私の記憶にあるイメージはずばり暴力団組長の自宅兼事務所という威圧的なイメージです。

 なにか住居表示という専門用語もでてきたのですが、御庁つまり金沢地方検察庁の住所も当初は手書きで、金沢市大手町6番15号としていたことを憶えています。ネットの情報では6−15以外に見かけていないように思うのですが、さっそく確認しておきます。

\begin{itemize}
\item
  金沢地方検察庁 \url{https://t.co/eV1FVKCSw5}  〒920-0912
  石川県金沢市大手町6番15号 ¥\n 電話:076-221-3161(代表)
\item
  金沢地方検察庁 - Wikipedia \url{https://t.co/qeGv0NVUCi} 
  所在地石川県金沢市大手町6番15号 ¥\n 金沢法務合同庁舎 ¥\n
  北緯36度34分4秒 ¥\n 東経136度39分47.1秒
\item
  金沢地方検察庁(金沢市/省庁・国の機関)の電話番号・住所・地図|マピオン電話帳
  \url{https://t.co/XJhwKbjM0k}  住所〒920-0912 ¥\n 石川県金沢市大手町6−15
\item
  金沢地方検察庁 - 金沢市 / 国の機関 / ウィキペディア - goo地図
  \url{https://t.co/9AOfI9DzG7}  お店/施設名 ¥\n 金沢地方検察庁 ¥\n ¥\n 住所
  ¥\n 石川県金沢市大手町6
\item
  金沢地方検察庁 (金沢市|法務省\textbar 代表:076-221-3161) -
  インターネット電話帳ならgooタウンページ \url{https://t.co/mP2RtAD0AJ}  ¥\n
  住所 ¥\n 石川県 金沢市 大手町6-15
\item
  金沢地方検察庁の周辺地図・アクセス・電話番号|国機関(法務省)|乗換案内NEXT
  \url{https://t.co/csntMTlHBY}  {[}住所{]}石川県金沢市大手町6−15 ¥\n
  {[}業種{]}国機関(法務省) ¥\n {[}電話番号{]}076-221-3161
\end{itemize}

 ネットで調べたところ6番15号が2つ6−15が4つGoogleの検索結果の1ページ目に出てきました。

 出かける前にストリートビューのタイムラインで確認したところタイムラインが2014年から始まり、現在と同じ場所の建物です。3階建ての住宅のようにも見える建物ですが、車庫やガレージのようなものはなく出入り口が左側にあり、私には昔見た個人病院、なんとか医院の建物に見えてしかたありません。

 味噌蔵町の交差点にやたらと近いのも私の記憶とはかけ離れています。

 本日、調べて初めて知った味噌蔵町の交差点は思っていた以上に緩く大きなカーブとなっているのですが、どう考えても路上駐車は躊躇われる場所です。

 被告発人大網健二に被告発人若杉幸平弁護士の法律事務所に連れて行かれた以外は、自分で運転する車以外で金沢地方裁判所の周辺に行った記憶はないのですが、有料駐車場を利用したような記憶があるのは小堀秀行弁護士の法律事務所に行ったときぐらいです。

 兼六園下交差点の側に立体型のような駐車場がありました。前にも一度、同じ駐車場を利用したような記憶があったのですが、小将町に食堂があってその家の2階に前妻の女友達が住み込みで働き、ちょうど結婚するとか結婚したという話を聞いたところでした。

 米米CLUBとかいうバンドの曲が大流行した頃で、その前妻の女友達が大ファンという話だったかもしれません。金沢地方裁判所のすぐ近くですが、そういうこともあって金沢地方裁判所の近くに行くとよく米米CLUBのことを思い出していました。大ヒットは2曲あったと思います。

\begin{itemize}
\item
  米米CLUB - Wikipedia \url{https://t.co/ptQud9zplW} 
  1990年に「浪漫飛行」がシングルカット後にミリオンセラー、日本航空(JAL)の沖縄キャンペーンのCMソングに使用された。
\item
  米米CLUB - Wikipedia \url{https://t.co/ptQud9zplW} 
  1992年、フジテレビ月9ドラマ『素顔のままで』の主題歌に「君がいるだけで」が起用され、同曲は累計売上289.5万枚を記録し、同年の第34回日本レコード大賞を受賞するなどの大ヒットを飛ばすと、米米CLUBの知名度は一気に上昇し、ファン層は拡大、一方で
\end{itemize}

 大ヒットとして記憶にあるのは「浪漫飛行」と「君がいるだけで」の2曲だけですが、ドラマの主題歌やCMソングのことは記憶にないものの、1992年という「君がいるだけで」を聞いていた可能性が確定的に高いです。1992年は平成2年です。

 金沢地方裁判所の周辺というのは、ぶらぶらとドライブすることは滅多になかった場所だと思いますが、金沢市場輸送や市場急配センターでの市内配達では、受け持ちの小立野・片町コースとして必ず通行していた配達先のある場所でした。東京ストア横山店というのもよく行きました。

\begin{itemize}
\item
  東京ストアー - Wikipedia \url{https://t.co/66SoRdzbyh}  横山町店 -
  金沢市、2009年(平成21年)4月{[}15{]}7日に閉店
\item
  東京ストアー横山店〔close〕 店舗情報と周辺状況【AJSM】
  \url{https://t.co/M1jvj4PF7Q}  金沢市横山町14-3
\item
  〒920-0922 石川県金沢市横山町14−3 - Google マップ
  \url{https://t.co/zoNp4gsBKB} 
\end{itemize}

 この東京ストアー横山店は金沢市内でも珍しく感じていたのですが、店舗が長屋のような建物で周辺が広場になっていて、昼間しか行ったことがなかったですが、盆踊りの会場のようなイメージがあり、周辺の町並みともイメージの異なる一角で、貧困層が多く住んでいそうな感じがしていました。

 その東京ストア横山店から大手町の交差点の辺りに抜けるような狭い路地があって、一方通行だったのか2トントラックの車幅ぎりぎりのような古い町並みの中にある路地でした。

 そこで、大学生のような若者が飛び出してきて、急ブレーキを踏み、長男が前に投げ出されるかっこうで、降りてその若者を怒鳴りつけたことは、手書きの書面にも書いたと記憶にあります。

 東京ストアで記憶にあるのは金沢市の繁華街片町の近くの竪町の店舗と、西南部店でした。西南部店は仕事より買い物に何度か行った記憶があるのですが、金沢市場輸送では、その東京ストアの定期便を東京からやっていました。神田の市場から大田区の市場へと移転しました。

 どちらかが青果と野菜にわかれていたと思いますが、丸一青果とヤマリだったと思います。平成2年頃には積み込み先が板橋の市場に変わり、早朝に長野県上田市辺りの山の上から高原野菜を積みようになり、おろし場所も金沢中央卸売市場の近くから松任市旭工業団地の新しい倉庫に移転していました。

 過去に何度も記述していると思いますが、東京ストアの定期便で行き荷の中心だった高岡市内のアルミサッシも卸先が、当初の埼玉県桶川市から茨城県水海道市に新しい配送センターに変わっていました。たぶんですが三協アルミだったように思います。

 今も同じかどうかわからず、近年は聞いたこともないのですが、金沢市の竪町というのは歩行者天国のような商店街で若者向けの服屋が多かったとも思います。被害者安藤文さん親友という高校の時の同級生の供述調書にも事件の数日前、たぶん前の日曜日に一緒に片町で買い物をしたという話がありました。

\begin{itemize}
\tightlist
\item
  竪町 (金沢市) - Wikipedia \url{https://t.co/7xU9gAh8pP} 
\end{itemize}

 軽く目を通したところ江戸時代からの歴史があるようなことが記述にありました。少なくとも昭和の時代は、東京の原宿に近いイメージや店があったように思います。郊外に衣料品の店舗を含めた大型店舗ができる前の話です。

\begin{itemize}
\item
  竪町 (金沢市) - Wikipedia \url{https://t.co/7xU9gAh8pP} 
  戦後、街灯の設置や夜店の開催など買物客の集客の整備が進められ、1968年(昭和43年)以降、竪町通りにはユニー(旧ほていや)や長崎屋などが竪町に店舗を構えるようになり、片町商店街と並び商店街として位置づけられる。
\item
  竪町 (金沢市) - Wikipedia \url{https://t.co/7xU9gAh8pP} 
  1966年(昭和41年)、住居表示制度の実施により、竪町の一部が片町一丁目となり現在に至る。
\end{itemize}

 もしかすると、東京ストアの竪町と、長崎屋という店を勘違いしていたかもしれません。毎回ではなかったと思いますが、竪町によく行く配達先が市内配達にありました。スーパーというよりデパートに近かったと思いますが、階数の高い建物で買い物に入ったことはなかったと思います。

\begin{itemize}
\tightlist
\item
  竪町・テミス跡地はザイマックスがホテルを軸に再開発を計画!|金沢まちゲーション
  \url{https://t.co/n2QmSB8cg6}  ¥\n
  1993年に長崎屋金沢店が「ラパーク金沢」として西泉へ移転すると、商業施設「アメリカンマインドテミス」として再出発しました。
\end{itemize}

 ラパーク金沢が西泉というのは思い出したのですが、長崎屋と関係があったとはしらず、ラパーク金沢の店舗に入ったこともなかったと思います。ただ、平成9年の秋頃、被告発人大網健二に頼まれた早朝の引っ越しの手伝いが、西泉のその辺りだったことは記憶にあります。

 ラパーク金沢という店名もはっきりとは記憶にないですが、その数年後になるのか金沢市内の郊外、西インターと御経塚の間にサティという大きな商業施設ができていました。平成10年の夏、犀川まつりの当日、昼に関係者KYNと一緒に服を買いに行ったことはよく憶えています。

 夜は香林坊のビルの屋上でビアガーデンでした。アムズ株式会社の接待のような感じにもなっていました。犀川まつりは花火大会がメインで、8月の第一土曜日の夜だったと思います。最初の思い出が昭和56年で、そのときだけになるのか犀川沿いを関係者OSNと歩きました。被告発人大網健二の兄です。

 時刻は21時47分です。買ってきた網で焼いている紅鮭の切り身をひっくり返してきましたが、簡単にひっくり返すことができました。ときどき脂が落ちるような音が聞こえていましたが、けっこう強火に近い火力でセンサーが反応することなく焼けている様子です。

 一年後の犀川まつりが、被害者安藤文さんの自宅に向かったときで、橋の上でけっこうな渋滞が発生していました。横目に花火を見たとも思うのですが、若宮大橋だったと思います。平成4年当時は金石街道から影も形もなかった新しい道路上の犀川に架かる橋でした。

 その若宮大橋の少し手前にあるのがパソコンの館という店舗でしたが、パソコンの専門店というのは平成9年当時、金沢市内で他にはなかったか、数はごく少数だったと思います。平成9年の4月にノートパソコンを買いましたが、しばらく前に領収書を見つけています。58万円ほどの買い物だったと思います。

 父親の建てた辺田の浜の家を処分して買えた買い物でしたが、この売買に協力したのも被告発人大網健二でした。被告発人大網健二の思惑とは方向がずれたようですが、被告発人大網健二に世話になったことは間違いのないことです。

\begin{itemize}
\tightlist
\item
  〈〈〈 2021/06/01 22:05:11 Linux Emacs: 〈〈〈
\end{itemize}

\hypertarget{section-7}{%
\paragraph{}\label{section-7}}

\hypertarget{ux88abux544aux767aux4ebaux5c0fux5cf6ux88d5ux53f2ux88c1ux5224ux9577}{%
\subsubsection{被告発人小島裕史裁判長}\label{ux88abux544aux767aux4ebaux5c0fux5cf6ux88d5ux53f2ux88c1ux5224ux9577}}

\hypertarget{ux61b2ux6cd5ux6539ux6b63ux306fux4e0dux8981ux3092ux9023ux547cux3059ux308bux30e2ux30c8ux30b1ux30f3ux3053ux3068ux77e2ux90e8ux5584ux6717ux5f01ux8b77ux58ebux4eacux90fdux5f01ux8b77ux58ebux4f1aux3068ux88abux7528ux8005ux306eux6c42ux511fux6a29ux3092ux8a8dux3081ux305fux6700ux9ad8ux88c1ux5224ux4f8bux88abux544aux767aux4ebaux5c0fux5cf6ux88d5ux53f2ux88c1ux5224ux9577ux306eux4f1aux793eux306eux8cacux4efbux5224ux65ad}{%
\paragraph{憲法改正は不要を連呼するモトケンこと矢部善朗弁護士(京都弁護士会)と被用者の求償権を認めた最高裁判例、被告発人小島裕史裁判長の会社の責任判断}\label{ux61b2ux6cd5ux6539ux6b63ux306fux4e0dux8981ux3092ux9023ux547cux3059ux308bux30e2ux30c8ux30b1ux30f3ux3053ux3068ux77e2ux90e8ux5584ux6717ux5f01ux8b77ux58ebux4eacux90fdux5f01ux8b77ux58ebux4f1aux3068ux88abux7528ux8005ux306eux6c42ux511fux6a29ux3092ux8a8dux3081ux305fux6700ux9ad8ux88c1ux5224ux4f8bux88abux544aux767aux4ebaux5c0fux5cf6ux88d5ux53f2ux88c1ux5224ux9577ux306eux4f1aux793eux306eux8cacux4efbux5224ux65ad}}

\begin{itemize}
\tightlist
\item
  〉〉〉 Linux Emacs: 2021/06/24 10:46:41 〉〉〉
\end{itemize}

:CATEGORIES: @kanazawabengosi \#金沢弁護士会 @JFBAsns
日本弁護士連合会(日弁連) \#法務省 @MOJ\_HOUMU
\#モトケンこと矢部善朗弁護士(京都弁護士会) \#被告発人小島裕史裁判長

f:id:hirono\_hideki:20210623195924j:image

〉〉〉 kk\_hironoのリツイート 〉〉〉

\begin{itemize}
\tightlist
\item
  RT
  kk\_hirono(刑事告発・非常上告_金沢地方検察庁御中)|nhk\_news(NHKニュース)
  日時:2021-06-24 10:56/2021/06/23 15:11 URL:
  \url{https://twitter.com/kk\_hirono/status/1407880335312887816} 
  \url{https://twitter.com/nhk\_news/status/1407581947195318272} 
  \textgreater{} 【速報 JUST IN 】夫婦別姓認めない民法の規定は合憲
  最高裁大法廷 判断 \#nhk\_news \url{https://t.co/B5uS6LGh1B} 
\end{itemize}

 もとは上記の「夫婦別姓認めない民法の規定は合憲 最高裁大法廷
判断」というニュースですが、判断というのが気になって他に調べたところ、大法廷決定となっていました。大法廷なので当然に口頭弁論を経ている判決かと思ったのですが、違っていたようです。

 モトケンこと矢部善朗弁護士(京都弁護士会)が短時間でツイートを連発していましたが、これまでにない積極性が行動となっています。内容はともかくそちらの方が気になったのですが、次のようなまとめ記事が作成できたほどです。

\begin{itemize}
\tightlist
\item
  2021年06月24日10時32分の登録:
  REGEXP:''憲法改正は不要。''/データベース登録済みツイートの検索:2021-06-24〜2021-06-24/2021年06月24日10時32分の記録:ユーザ・投稿:1/12件
  \url{https://kk2020-09.blogspot.com/2021/06/regexp2021-06-242021-06.html} 
\end{itemize}

 もう一つ「再掲」のまとめ記事を作成しました。時間の範囲を昨日からと絞っています。

\begin{itemize}
\tightlist
\item
  2021年06月24日11時18分の登録:
  REGEXP:''再掲''/データベース登録済みツイートの検索:2021-06-24〜2021-06-24/2021年06月24日11時18分の記録:ユーザ・投稿:1/9件
  \url{https://kk2020-09.blogspot.com/2021/06/regexp2021-06-242021-06-2420210624111819.html} 
\end{itemize}

 もともと賛否の大きな問題だと思いますが、弁護士商売の収穫作業にも思えるモトケンこと矢部善朗弁護士(京都弁護士会)の過剰な反応です。見込みがあっての言動なのかと思いますが、それ以上に影響力について考えさせられるところです。

 主義主張や表現の自由という問題以上に、弁護士商売の個人的な打算がすこぶる強く感じられるのがモトケンこと矢部善朗弁護士(京都弁護士会)のTwitterの利用法で、その影響力を再検討していたところのタイミングでした。

※ @kk\_hironoのアカウントがブロックされ,リツイートに失敗したツイート

\begin{itemize}
\tightlist
\item
  TW motoken\_tw(モトケン) 日時:2021/06/23 18:35:11 URL:
  \url{https://twitter.com/motoken\_tw/status/1407633321530593283} 
  \textgreater{}
  この最高裁決定に関して、別姓制を採用するためには憲法改正が必要だと考えている人が相当数いるようなのですが、最高裁は選択的夫婦別姓制は違憲とは言ってなくて、国会で決めればいいと言ってますので、憲法改正は不要です。法律改正をすればいいだけ。
  \url{https://t.co/dbqX7WujIR} 
\end{itemize}

 私は、本件告発事件と非常上告の手続きを同時に進めているつもりなのですが、非常上告は憲法問題で、弁護士と裁判官の違法行為を主因とする人権保障も謳った刑事裁判における適正手続きの問題になります。

 モトケンこと矢部善朗弁護士(京都弁護士会)はその渦中において、重大深刻な影響を本件告発事件と非常上告の利害関係者に与え、私の権利実現を阻害し続けてきた張本人になります。

 暫く前から、モトケンこと矢部善朗弁護士(京都弁護士会)と小倉秀夫弁護士の名誉毀損での刑事告訴の必要性を再認識したことは、Twitterやブログで表明しており、スクリーンショットや写真による資料も公開しています。

 私の告発事件と非常上告に対して、これはモトケンこと矢部善朗弁護士(京都弁護士会)に限った話ではないですが、早い段階でモトケンこと矢部善朗弁護士(京都弁護士会)が他のインターネットを利用する弁護士らに与えた影響というのも大きいと考えています。

 小倉秀夫弁護士もモトケンこと矢部善朗弁護士(京都弁護士会)の影響を受けて、同じく名誉毀損の刑事責任追求を表明されるはめになった、あるいは巻き添えをくうかたちになったのかもしれないですが、同じ軌道を辿った経緯があります。

 そういえば、最近でも小倉秀夫弁護士がモトケンこと矢部善朗弁護士(京都弁護士会)のツイートに、引用ツイートやあるいは返信をしていたと思うのですが、以前のような意見の応酬という展開は見ていないように思います。

 モトケンこと矢部善朗弁護士(京都弁護士会)の方で消極的な対応をしていたような印象が残っているのですが、そのあたりのところもこれから作成するまとめ記事で確認をしておきたいと思います。なお本稿の主軸は被告発人小島裕史裁判長です。

\begin{itemize}
\tightlist
\item
  2021年06月24日11時57分の登録:
  REGEXP:''@chosakukenho''/モトケン(@motoken\_tw)の検索(2020-11-03〜2021-03-17/2021年06月24日11時57分の記録25件)
  \url{https://kk2020-09.blogspot.com/2021/06/regexpchosakukenhomotokentw2020-11.html} 
\item
  2021年06月24日11時59分の登録:
  REGEXP:''モトケン''/小倉秀夫(@chosakukenho)の検索(2020-10-01〜2021-02-16/2021年06月24日11時59分の記録4件)
  \url{https://kk2020-09.blogspot.com/2021/06/regexpchosakukenho2020-10-012021-02.html} 
\end{itemize}

 小倉秀夫弁護士の方からモトケンこと矢部善朗弁護士(京都弁護士会)のツイートへのアプローチという形態しか印象になかったのですが、小倉秀夫弁護士からモトケンこと矢部善朗弁護士(京都弁護士会)のメンション@motoken\_twというのはゼロでした。

 まだ初見ですが、次のツイートもモトケンこと矢部善朗弁護士(京都弁護士会)のツイートを引用した小倉秀夫弁護士のツイートになります。

\begin{itemize}
\item
  TW chosakukenho(小倉秀夫) 日時: 2020/11/17 09:05:33 URL:
  \url{https://twitter.com/chosakukenho/status/1328489416013213696} 
  \textgreater{}
  甘利さんとか、疑惑を追求されたときにご病気になってその説明をうやむやにできたのは、偶然の奇跡だったのですね。
  \url{https://t.co/pHcNBjUHuA} 
\item
  TW motoken\_tw(モトケン) 日時: 2020/11/17 09:19:12 URL:
  \url{https://twitter.com/motoken\_tw/status/1328492851840180224} 
  \textgreater{} @chosakukenho
  相変わらず、「自説の理由付け」だけはお上手ですね。\\
  \textgreater{} 知らない人は知らないでしょうが、知る人ぞ知る、ですよw
\end{itemize}

 大人の対応にもみえるモトケンこと矢部善朗弁護士(京都弁護士会)のツイートですが、返信先にある小倉秀夫弁護士のツイートにもモトケンこと矢部善朗弁護士(京都弁護士会)のツイートの引用があります。

 憲法問題のツイートが出てきましたが、今度は自衛隊が出てきました。これは少し見覚えのあるやりとりです。

\begin{itemize}
\item
  TW motoken\_tw(モトケン) 日時: 2021/02/16 10:27:34 URL:
  \url{https://twitter.com/motoken\_tw/status/1361487352942653441} 
  \textgreater{} @chosakukenho
  解釈改憲の余地を減らして、自衛隊が何ができて何ができないかを明確にするため。\\
  \textgreater{} これまで何回も言ってますよね。
\item
  TW chosakukenho(小倉秀夫) 日時: 2021/02/16 10:17:19 URL:
  \url{https://twitter.com/chosakukenho/status/1361484770635472897} 
  \textgreater{}
  モトケン先生は、自衛隊に何をさせたくて憲法九条を変えたいと思っているのですか。
  \url{https://t.co/vzycPN6KpU} 
\end{itemize}

 モトケンこと矢部善朗弁護士(京都弁護士会)が小倉秀夫弁護士のことを曲解と評価するツイートが目立っていたのですが、夫婦別姓問題につながるモトケンこと矢部善朗弁護士(京都弁護士会)のツイートが出てきました。

※ @kk\_hironoのアカウントがブロックされ,リツイートに失敗したツイート

\begin{itemize}
\tightlist
\item
  TW motoken\_tw(モトケン) 日時:2021/03/17 08:33:15 URL:
  \url{https://twitter.com/motoken\_tw/status/1371967831655419904} 
  \textgreater{} @chosakukenho
  重箱の隅をつついて本末転倒の論陣を張る。\\
  \textgreater{} さすがです。\\
  \textgreater{}
  ほとんどの夫婦が夫の姓を選ぶように機能している現行制度のほうが、よっぽど「加盟(家名ですよね)を子々孫々残したいという封建主義的発想を残置させるもの」でしょう。\\
  \textgreater{}
  選択的夫婦別姓制はその根本思想において家名を否定している。
\end{itemize}

〉〉〉 kk\_hironoのリツイート 〉〉〉

\begin{itemize}
\tightlist
\item
  RT
  kk\_hirono(刑事告発・非常上告_金沢地方検察庁御中)|chosakukenho(小倉秀夫)
  日時:2021-06-24 12:18/2021/03/17 08:28 URL:
  \url{https://twitter.com/kk\_hirono/status/1407901012879282183} 
  \url{https://twitter.com/chosakukenho/status/1371966562115100674} 
  \textgreater{}
  婚姻している夫婦の子どもに夫婦のどちらかの氏を付けられるようにするならばそこに不均衡が生じます。それでもなお、夫婦のどちらかの氏を付けられるようにするというのは、加盟を子々孫々残したいという封建主義的発想を残置させるものでしかありません。
  \url{https://t.co/WRDPxEzfdZ} 
\end{itemize}

※ @kk\_hironoのアカウントがブロックされ,リツイートに失敗したツイート

\begin{itemize}
\tightlist
\item
  TW motoken\_tw(モトケン) 日時:2021/03/17 08:57:01 URL:
  \url{https://twitter.com/motoken\_tw/status/1371973811050999808} 
  \textgreater{} @chosakukenho >先祖代々受け継いできた、\\
  \textgreater{}\\
  \textgreater{} 何を先祖代々受け継いできたと言うのですか?
\end{itemize}

〉〉〉 kk\_hironoのリツイート 〉〉〉

\begin{itemize}
\tightlist
\item
  RT
  kk\_hirono(刑事告発・非常上告_金沢地方検察庁御中)|chosakukenho(小倉秀夫)
  日時:2021-06-24 12:21/2021/03/17 08:54 URL:
  \url{https://twitter.com/kk\_hirono/status/1407901730650550278} 
  \url{https://twitter.com/chosakukenho/status/1371973188662435841} 
  \textgreater{}
  先祖代々受け継いできた、名前の一部を構成するものでしょうね。
  \url{https://t.co/ItpeP3SaJx} 
\end{itemize}

※ @kk\_hironoのアカウントがブロックされ,リツイートに失敗したツイート

\begin{itemize}
\tightlist
\item
  TW motoken\_tw(モトケン) 日時:2021/03/17 09:10:40 URL:
  \url{https://twitter.com/motoken\_tw/status/1371977247456849922} 
  \textgreater{} @chosakukenho つまり氏(姓)ですね。\\
  \textgreater{}
  そうすると、同姓婚というのは、姓を変える側に先祖代々受け継いできた氏を捨てろと言うことになりますね(離婚による復氏の可能性はあるとしても)。\\
  \textgreater{}
  そして、大半の婚姻において女性側が先祖代々受け継いできた氏を捨てることになる。\\
  \textgreater{} すごく家父長制的ですね。
\end{itemize}

※ @kk\_hironoのアカウントがブロックされ,リツイートに失敗したツイート

\begin{itemize}
\item
  TW motoken\_tw(モトケン) 日時:2021/03/17 09:57:35 URL:
  \url{https://twitter.com/motoken\_tw/status/1371989052556075012} 
  \textgreater{} @chosakukenho
  >家父長制を持ち込みたいということにはなり得ませんね。\\
  \textgreater{}\\
  \textgreater{} 家父長的価値観やその価値観に沿う運用\\
  \textgreater{}\\
  \textgreater{} 家父長制を持ち込むこととの違いがわかりますよね。ね?
\item
  (25/25) TW motoken\_tw(モトケン) 日時:2021-03-17 09:57:35 +0900
  URL: \url{https://twitter.com/motoken\_tw/status/1371989052556075012} 
\end{itemize}

 時刻は12時29分です。台所で昼食の支度に取り掛かりましたが、昨夜の残り物の野菜炒めを温め直し始めたところです。モトケンこと矢部善朗弁護士(京都弁護士会)が家父長的価値観に批判的というのも忘れていました。

 モトケンこと矢部善朗弁護士(京都弁護士会)がいう家父長的価値観というのも、勝手に作り上げた粘土細工のようなもので、社会の実態に即しているとは考えにくいのですが、他にもつながるマジックワードというか弁護士商売の武器にはなりそうです。

 えきなんローヤ‐のタイムラインで、大西洋一弁護士のツイートを見たのがきっかけですが、・・・と書いてからタイムラインでツイートを探したのですが、ツイートの数が多くなかなか辿り着けませんでした。次のツイートから始まっています。

〉〉〉 kk\_hironoのリツイート 〉〉〉

\begin{itemize}
\tightlist
\item
  RT
  kk\_hirono(刑事告発・非常上告_金沢地方検察庁御中)|o2441(スラ弁(弁護士大西洋一))
  日時:2021-06-24 13:15/2021/06/23 21:42 URL:
  \url{https://twitter.com/kk\_hirono/status/1407915301308702720} 
  \url{https://twitter.com/o2441/status/1407680543177723905} 
  \textgreater{} 文章力ある人がうらやましいわ。 \url{https://t.co/d6ttwYwD9F} 
\end{itemize}

〉〉〉 kk\_hironoのリツイート 〉〉〉

\begin{itemize}
\tightlist
\item
  RT
  kk\_hirono(刑事告発・非常上告_金沢地方検察庁御中)|waniwani\_law(昼夜大逆転ワニ)
  日時:2021-06-24 13:16/2021/06/23 18:21 URL:
  \url{https://twitter.com/kk\_hirono/status/1407915575825690630} 
  \url{https://twitter.com/waniwani\_law/status/1407629799397289990} 
  \textgreater{} 草野裁判官、すごく煽りスキル高そう
  \url{https://t.co/7AR2EoISoG} 
\end{itemize}

〉〉〉 kk\_hironoのリツイート 〉〉〉

\begin{itemize}
\item
  RT
  kk\_hirono(刑事告発・非常上告_金沢地方検察庁御中)|waniwani\_law(昼夜大逆転ワニ)
  日時:2021-06-24 13:16/2021/06/24 01:25 URL:
  \url{https://twitter.com/kk\_hirono/status/1407915587221692417} 
  \url{https://twitter.com/waniwani\_law/status/1407736459004575746} 
  \textgreater{}
  草野判事、めっちゃ伸びてる。特に宣伝することもないので、草野判事が作ったとも言える判例を宣伝しときます。
  \url{https://t.co/4qrKsDKXp8} 
\item
  089270\_hanrei.pdf \url{https://t.co/fPdve0gxt2} 
  平成30年(受)第1429号 債務確認請求本訴,求償金請求反訴事件 ¥\n
  令和2年2月28日 第二小法廷判決
\end{itemize}

 ずいぶん前に見かけていたと思っていたのですが、調べて確認したところ、やはり福山通運でした。確認できなかったのは死亡事故を起こした運転手が女性ということです。私の記憶では女性なのですが、それもあったので強く印象に残っていました。

 よくみると、第二小法廷判決とあります。最高裁で口頭弁論が開かれたことを意味するはずですが、控訴審が大阪高裁となっていました。最高裁のことはほとんど記憶になかったのですが、破棄差し戻しとありました。

 夫婦別姓などの社会問題を別に、民事裁判で個別の最高裁判例を見たのも珍しかったのですが、民事裁判で差し戻しというのは更に珍しく思いました。さらに民事裁判の方が当事者にかかる金銭的負担が大きいと思われます。

\begin{itemize}
\tightlist
\item
  ./hirono\_hideki2021-06-24\_132502.csv:2020-02-28 17:18:17
  ``勤務中の事故、会社に負担請求可 最高裁が初判断 | 共同通信
  \url{https://this.kiji.is/606002621051569249?c=39550187727945729}  2020/2/28
  14:57 (JST)2/28 15:09 (JST)updated
  原告は運送大手、福山通運(広島県福山市)のドライバーだった女性。''
  \url{https://twitter.com/hirono\_hideki/status/1233305406350344192} 
\end{itemize}

 ページタイトルに共同通信とありますが、URLはリンク切れとなっていました。

\begin{itemize}
\tightlist
\item
  勤務で事故、会社に請求可 被害者への賠償で初判断 最高裁、使用者責任明示
  -- 株式会社 労働開発研究会 \url{https://t.co/JW2e2lSTL1} 
\end{itemize}

 上記の記事に、大阪府吹田市での死亡事故とありました。

\begin{itemize}
\item
  〈〈〈 2021/06/24 13:47:50 Linux Emacs: 〈〈〈
\item
  〉〉〉 Linux Emacs: 2021/06/24 15:10:02 〉〉〉
\end{itemize}

 歯医者に行き、Aコープ能都店で買い物をして戻ってきました。待合室のテレビでミヤネ屋の放送があり、小笠原諸島に台風が近づいているという天気予報があり、お天気キャスターのような女性の姿を見て、ずいぶん長くまともにミヤネ屋の放送をみることがなかったのだと実感しました。

 ミヤネ屋の放送が始まる直前の予告だったと思いますが、たしか小田原市で、19頭とかの猿の集団に駆除決定が出たというようなニュースがありました。珍しいニュースに思えたのですが、小田原市といえば村松謙弁護士を思い出します。

\begin{itemize}
\tightlist
\item
  40年以上も・・・農作物荒らし人を威嚇
  ``サルの群れ''を駆除へ 神奈川・小田原(2021年6月22日放送「news
  every.」より) - YouTube \url{https://t.co/VQwrEyoGXz} 
\end{itemize}

 時刻は6月26日13時11分です。丸2日近く中断していました。6月24日は17時過ぎぐらいに用事で漆原に行っていました。昨日の25日は、新品のお札をもらうために早めに興能信用金庫に行き、他の用事も済ませてきました。

 そのための準備も済ませているので、そろそろWindowsパソコンで決算報告書の作成に取り掛かりたいところですが、もう少し区切りをつけておきたいと思います。

\begin{quote}
《引用の始まり》
\end{quote}

\begin{quote}
【事故と補償の概要】

・2010年7月26日、運転者A(女性)は信号のない交差点を右折中に、交差点に入ってきた自転車と接触、自転車の乗員は転倒し、死亡しました。

・Aの務める運送会社は大手企業ですが、任意保険には加入していませんでした。

~

・この事故をめぐって、運送会社は亡くなった被害者の相続人の1人(次男)に治療費や和解金を支払いましたが、もう1人の相続人(長男)はこれを不服として加害者側を提訴しました。

・民事訴訟により、追加で1,552万円余りの損害賠償が認められ、運転者はこれを相続人に支払いました。

 

・運転者は、会社の事業執行としてトラックを運転中に起こした交通事故に関し、第三者に加えた損害を賠償したことにより会社への求償権を取得した等の主張で、使用者に対して求償金等の支払いを求めました。~
\end{quote}

\begin{quote}
《引用の終わり》
\end{quote}

\begin{itemize}
\tightlist
\item
  社員から企業への事故の逆求償を認める -
  人と車の安全な移動をデザインするシンク出版株式会社
  \url{https://www.think-sp.com/2020/04/02/hanrei-kyushoken-2/} 
\end{itemize}

 もう少し取り上げておきたいと考えていた逆求償の最高裁判決ですが、これまでに見たことのなかった事実関係が見つかりました。福山通運が死亡事故の対応をしなかったとは考えにくいということで気になっていたのですが、遺族の次男とは和解が成立していたとのことです。

 死亡した自転車の乗員の年齢というのは記載がないのですが、ページ内検索をしても「逸失利益」や「慰謝料」の該当がなく、追加で1,552万円余りの損害賠償の法的な内訳が気になるところです。

 死亡した被害者の損害賠償請求権を遺族が相続したという明記も見当たらないような気がします。遺族の長男と次男に個別の請求権が認められたとは考え難く、子供を事故で失った親の慰謝料でも数百万という判例を見たような記憶があります。

 私の告発事件も市場急配センターという使用者責任の絡みもあると考えるのですが、法人自体が負う安全配慮義務という問題もあり、計画性や故意との関係性というのも考慮すべき事情と考えてきました。

 そのあたりの市場急配センターの会社の責任を明確に否定したのが被告発人小島裕史裁判長の平成5年9月7日付判決になります。私の責任転嫁だと一蹴し、反省が不十分だと評価していました。

\begin{quote}
《引用の始まり》
\end{quote}

\begin{quote}
甲号証(四)\_341_平成4年(う)第52号 判決 平成5年9月7日宣告 小島裕史裁判長\_06.jpg
\end{quote}

\begin{quote}
《引用の終わり》
\end{quote}

\begin{itemize}
\tightlist
\item
  奉納\危険生物・弁護士脳汚染除去装置\金沢地方検察庁御中\_2020:
  2021年05月24日の記録:写真資料:甲号証(四)\_判決(9枚) 平成4年(う)第52号 平成5年9月7日宣告 被告発人小島裕史裁判長
  \url{https://kk2020-09.blogspot.com/2021/05/202105249452597.html} 
\end{itemize}

 上記の被告発人小島裕史裁判長による判決文の写真撮影部分には、弁護人に多数の手紙を発信、その内容には精神的に不安定だったことを回想し、などとあります。

 これも運送会社の事故の問題と密接に関わるのですが、長距離トラック運転手の仕事には支障を及ぼし、身の危険と大型車による事故で他人や家族を巻き込む大惨事の発生の可能性も真剣に危惧していました。

 この被告発人小島裕史裁判長の判決文は、前提として被告発人木梨松嗣弁護士の弁護活動の成果が反映されているのですが、どこがどのように問題かという以前に、事実関係の主要部分の大半を無視して切り捨てたものとなっています。

 なお、私が被告発人木梨松嗣弁護士に郵送した便箋の手紙というのも、記録として保存され、被告発人木梨松嗣弁護士が名古屋高裁金沢支部に提出していたようです。証拠になるのか素人にはよくわからないところもありますが、検事の異議もなく採用された様子です。

 その手紙には被告発人木梨松嗣弁護士の判子があるものとないものがあって、途中で気になって写真撮影を始めたところ、内容を確認していくと、ずいぶん重要な内容が記載されていたことに気が付きました。それまではゴミの扱いで、存在自体忘れていたところです。

 いずれ取り上げて、ご説明をしておきたいと考えておりますが、次のように公開済みの記録資料となっています。

\begin{itemize}
\tightlist
\item
  甲号証(四)\_002_被告発人木梨松嗣弁護士への手紙001.jpg
  \url{https://kk2020-09.blogspot.com/2021/05/20210524.html\#2} 
\item
  甲号証(四)\_003_被告発人木梨松嗣弁護士への手紙002.jpg
  \url{https://kk2020-09.blogspot.com/2021/05/20210524.html\#3} 
\item
  甲号証(四)\_004_被告発人木梨松嗣弁護士への手紙003.jpg
  \url{https://kk2020-09.blogspot.com/2021/05/20210524.html\#4} 
\item
  甲号証(四)\_005_被告発人木梨松嗣弁護士への手紙004.jpg
  \url{https://kk2020-09.blogspot.com/2021/05/20210524.html\#5} 
\item
  甲号証(四)\_006_被告発人木梨松嗣弁護士への手紙005.jpg
  \url{https://kk2020-09.blogspot.com/2021/05/20210524.html\#6} 
\item
  甲号証(四)\_007_被告発人木梨松嗣弁護士への手紙006.jpg
  \url{https://kk2020-09.blogspot.com/2021/05/20210524.html\#7} 
\item
  甲号証(四)\_008_被告発人木梨松嗣弁護士への手紙007.jpg
  \url{https://kk2020-09.blogspot.com/2021/05/20210524.html\#8} 
\item
  甲号証(四)\_009_被告発人木梨松嗣弁護士への手紙008.jpg
  \url{https://kk2020-09.blogspot.com/2021/05/20210524.html\#9} 
\item
  甲号証(四)\_010_被告発人木梨松嗣弁護士への手紙009.jpg
  \url{https://kk2020-09.blogspot.com/2021/05/20210524.html\#10} 
\item
  甲号証(四)\_011_被告発人木梨松嗣弁護士への手紙010.jpg
  \url{https://kk2020-09.blogspot.com/2021/05/20210524.html\#11} 
\item
  甲号証(四)\_012_被告発人木梨松嗣弁護士への手紙011.jpg
  \url{https://kk2020-09.blogspot.com/2021/05/20210524.html\#12} 
\item
  甲号証(四)\_013_被告発人木梨松嗣弁護士への手紙012.jpg
  \url{https://kk2020-09.blogspot.com/2021/05/20210524.html\#13} 
\item
  甲号証(四)\_014_被告発人木梨松嗣弁護士への手紙013.jpg
  \url{https://kk2020-09.blogspot.com/2021/05/20210524.html\#14} 
\item
  甲号証(四)\_015_被告発人木梨松嗣弁護士への手紙014.jpg
  \url{https://kk2020-09.blogspot.com/2021/05/20210524.html\#15} 
\item
  甲号証(四)\_016_被告発人木梨松嗣弁護士への手紙015.jpg
  \url{https://kk2020-09.blogspot.com/2021/05/20210524.html\#16} 
\item
  甲号証(四)\_017_被告発人木梨松嗣弁護士への手紙016.jpg
  \url{https://kk2020-09.blogspot.com/2021/05/20210524.html\#17} 
\item
  甲号証(四)\_018_被告発人木梨松嗣弁護士への手紙017.jpg
  \url{https://kk2020-09.blogspot.com/2021/05/20210524.html\#18} 
\item
  甲号証(四)\_019_被告発人木梨松嗣弁護士への手紙018.jpg
  \url{https://kk2020-09.blogspot.com/2021/05/20210524.html\#19} 
\item
  甲号証(四)\_020_被告発人木梨松嗣弁護士への手紙019.jpg
  \url{https://kk2020-09.blogspot.com/2021/05/20210524.html\#20} 
\item
  甲号証(四)\_021_被告発人木梨松嗣弁護士への手紙020.jpg
  \url{https://kk2020-09.blogspot.com/2021/05/20210524.html\#21} 
\item
  甲号証(四)\_022_被告発人木梨松嗣弁護士への手紙021.jpg
  \url{https://kk2020-09.blogspot.com/2021/05/20210524.html\#22} 
\item
  甲号証(四)\_023_被告発人木梨松嗣弁護士への手紙022.jpg
  \url{https://kk2020-09.blogspot.com/2021/05/20210524.html\#23} 
\item
  甲号証(四)\_024_被告発人木梨松嗣弁護士への手紙023.jpg
  \url{https://kk2020-09.blogspot.com/2021/05/20210524.html\#24} 
\item
  甲号証(四)\_025_被告発人木梨松嗣弁護士への手紙024.jpg
  \url{https://kk2020-09.blogspot.com/2021/05/20210524.html\#25} 
\item
  甲号証(四)\_026_被告発人木梨松嗣弁護士への手紙025.jpg
  \url{https://kk2020-09.blogspot.com/2021/05/20210524.html\#26} 
\item
  甲号証(四)\_027_被告発人木梨松嗣弁護士への手紙026.jpg
  \url{https://kk2020-09.blogspot.com/2021/05/20210524.html\#27} 
\item
  甲号証(四)\_028_被告発人木梨松嗣弁護士への手紙027.jpg
  \url{https://kk2020-09.blogspot.com/2021/05/20210524.html\#28} 
\item
  甲号証(四)\_029_被告発人木梨松嗣弁護士への手紙028.jpg
  \url{https://kk2020-09.blogspot.com/2021/05/20210524.html\#29} 
\item
  甲号証(四)\_030_被告発人木梨松嗣弁護士への手紙029.jpg
  \url{https://kk2020-09.blogspot.com/2021/05/20210524.html\#30} 
\item
  甲号証(四)\_031_被告発人木梨松嗣弁護士への手紙030.jpg
  \url{https://kk2020-09.blogspot.com/2021/05/20210524.html\#31} 
\item
  甲号証(四)\_032_被告発人木梨松嗣弁護士への手紙031.jpg
  \url{https://kk2020-09.blogspot.com/2021/05/20210524.html\#32} 
\item
  甲号証(四)\_033_被告発人木梨松嗣弁護士への手紙032.jpg
  \url{https://kk2020-09.blogspot.com/2021/05/20210524.html\#33} 
\item
  甲号証(四)\_034_被告発人木梨松嗣弁護士への手紙033.jpg
  \url{https://kk2020-09.blogspot.com/2021/05/20210524.html\#34} 
\item
  甲号証(四)\_035_被告発人木梨松嗣弁護士への手紙034.jpg
  \url{https://kk2020-09.blogspot.com/2021/05/20210524.html\#35} 
\item
  甲号証(四)\_036_被告発人木梨松嗣弁護士への手紙035.jpg
  \url{https://kk2020-09.blogspot.com/2021/05/20210524.html\#36} 
\item
  甲号証(四)\_037_被告発人木梨松嗣弁護士への手紙036.jpg
  \url{https://kk2020-09.blogspot.com/2021/05/20210524.html\#37} 
\item
  甲号証(四)\_038_被告発人木梨松嗣弁護士への手紙037.jpg
  \url{https://kk2020-09.blogspot.com/2021/05/20210524.html\#38} 
\item
  甲号証(四)\_039_被告発人木梨松嗣弁護士への手紙038.jpg
  \url{https://kk2020-09.blogspot.com/2021/05/20210524.html\#39} 
\item
  甲号証(四)\_040_被告発人木梨松嗣弁護士への手紙039.jpg
  \url{https://kk2020-09.blogspot.com/2021/05/20210524.html\#40} 
\item
  甲号証(四)\_041_被告発人木梨松嗣弁護士への手紙040.jpg
  \url{https://kk2020-09.blogspot.com/2021/05/20210524.html\#41} 
\item
  甲号証(四)\_042_被告発人木梨松嗣弁護士への手紙041.jpg
  \url{https://kk2020-09.blogspot.com/2021/05/20210524.html\#42} 
\item
  甲号証(四)\_043_被告発人木梨松嗣弁護士への手紙042.jpg
  \url{https://kk2020-09.blogspot.com/2021/05/20210524.html\#43} 
\item
  甲号証(四)\_044_被告発人木梨松嗣弁護士への手紙043.jpg
  \url{https://kk2020-09.blogspot.com/2021/05/20210524.html\#44} 
\item
  甲号証(四)\_045_被告発人木梨松嗣弁護士への手紙044.jpg
  \url{https://kk2020-09.blogspot.com/2021/05/20210524.html\#45} 
\item
  甲号証(四)\_046_被告発人木梨松嗣弁護士への手紙045.jpg
  \url{https://kk2020-09.blogspot.com/2021/05/20210524.html\#46} 
\item
  甲号証(四)\_047_被告発人木梨松嗣弁護士への手紙046.jpg
  \url{https://kk2020-09.blogspot.com/2021/05/20210524.html\#47} 
\item
  甲号証(四)\_048_被告発人木梨松嗣弁護士への手紙047.jpg
  \url{https://kk2020-09.blogspot.com/2021/05/20210524.html\#48} 
\item
  甲号証(四)\_049_被告発人木梨松嗣弁護士への手紙048.jpg
  \url{https://kk2020-09.blogspot.com/2021/05/20210524.html\#49} 
\item
  甲号証(四)\_050_被告発人木梨松嗣弁護士への手紙049.jpg
  \url{https://kk2020-09.blogspot.com/2021/05/20210524.html\#50} 
\item
  甲号証(四)\_051_被告発人木梨松嗣弁護士への手紙050.jpg
  \url{https://kk2020-09.blogspot.com/2021/05/20210524.html\#51} 
\item
  甲号証(四)\_052_被告発人木梨松嗣弁護士への手紙051.jpg
  \url{https://kk2020-09.blogspot.com/2021/05/20210524.html\#52} 
\item
  甲号証(四)\_053_被告発人木梨松嗣弁護士への手紙052.jpg
  \url{https://kk2020-09.blogspot.com/2021/05/20210524.html\#53} 
\item
  甲号証(四)\_054_被告発人木梨松嗣弁護士への手紙053.jpg
  \url{https://kk2020-09.blogspot.com/2021/05/20210524.html\#54} 
\item
  甲号証(四)\_055_被告発人木梨松嗣弁護士への手紙054.jpg
  \url{https://kk2020-09.blogspot.com/2021/05/20210524.html\#55} 
\item
  甲号証(四)\_056_被告発人木梨松嗣弁護士への手紙055.jpg
  \url{https://kk2020-09.blogspot.com/2021/05/20210524.html\#56} 
\item
  甲号証(四)\_057_被告発人木梨松嗣弁護士への手紙056.jpg
  \url{https://kk2020-09.blogspot.com/2021/05/20210524.html\#57} 
\item
  甲号証(四)\_058_被告発人木梨松嗣弁護士への手紙057.jpg
  \url{https://kk2020-09.blogspot.com/2021/05/20210524.html\#58} 
\item
  甲号証(四)\_059_被告発人木梨松嗣弁護士への手紙058.jpg
  \url{https://kk2020-09.blogspot.com/2021/05/20210524.html\#59} 
\item
  甲号証(四)\_060_被告発人木梨松嗣弁護士への手紙059.jpg
  \url{https://kk2020-09.blogspot.com/2021/05/20210524.html\#60} 
\item
  甲号証(四)\_061_被告発人木梨松嗣弁護士への手紙060.jpg
  \url{https://kk2020-09.blogspot.com/2021/05/20210524.html\#61} 
\item
  甲号証(四)\_062_被告発人木梨松嗣弁護士への手紙061.jpg
  \url{https://kk2020-09.blogspot.com/2021/05/20210524.html\#62} 
\item
  甲号証(四)\_063_被告発人木梨松嗣弁護士への手紙062.jpg
  \url{https://kk2020-09.blogspot.com/2021/05/20210524.html\#63} 
\item
  甲号証(四)\_064_被告発人木梨松嗣弁護士への手紙063.jpg
  \url{https://kk2020-09.blogspot.com/2021/05/20210524.html\#64} 
\item
  甲号証(四)\_065_被告発人木梨松嗣弁護士への手紙064.jpg
  \url{https://kk2020-09.blogspot.com/2021/05/20210524.html\#65} 
\item
  甲号証(四)\_066_被告発人木梨松嗣弁護士への手紙065.jpg
  \url{https://kk2020-09.blogspot.com/2021/05/20210524.html\#66} 
\item
  甲号証(四)\_067_被告発人木梨松嗣弁護士への手紙066.jpg
  \url{https://kk2020-09.blogspot.com/2021/05/20210524.html\#67} 
\item
  甲号証(四)\_068_被告発人木梨松嗣弁護士への手紙067.jpg
  \url{https://kk2020-09.blogspot.com/2021/05/20210524.html\#68} 
\item
  甲号証(四)\_069_被告発人木梨松嗣弁護士への手紙068.jpg
  \url{https://kk2020-09.blogspot.com/2021/05/20210524.html\#69} 
\item
  甲号証(四)\_070_被告発人木梨松嗣弁護士への手紙069.jpg
  \url{https://kk2020-09.blogspot.com/2021/05/20210524.html\#70} 
\item
  甲号証(四)\_071_被告発人木梨松嗣弁護士への手紙070.jpg
  \url{https://kk2020-09.blogspot.com/2021/05/20210524.html\#71} 
\item
  甲号証(四)\_072_被告発人木梨松嗣弁護士への手紙071.jpg
  \url{https://kk2020-09.blogspot.com/2021/05/20210524.html\#72} 
\item
  甲号証(四)\_073_被告発人木梨松嗣弁護士への手紙072.jpg
  \url{https://kk2020-09.blogspot.com/2021/05/20210524.html\#73} 
\item
  甲号証(四)\_074_被告発人木梨松嗣弁護士への手紙073.jpg
  \url{https://kk2020-09.blogspot.com/2021/05/20210524.html\#74} 
\item
  甲号証(四)\_075_被告発人木梨松嗣弁護士への手紙074.jpg
  \url{https://kk2020-09.blogspot.com/2021/05/20210524.html\#75} 
\item
  甲号証(四)\_076_被告発人木梨松嗣弁護士への手紙075.jpg
  \url{https://kk2020-09.blogspot.com/2021/05/20210524.html\#76} 
\item
  甲号証(四)\_077_被告発人木梨松嗣弁護士への手紙076.jpg
  \url{https://kk2020-09.blogspot.com/2021/05/20210524.html\#77} 
\item
  甲号証(四)\_078_被告発人木梨松嗣弁護士への手紙077.jpg
  \url{https://kk2020-09.blogspot.com/2021/05/20210524.html\#78} 
\item
  甲号証(四)\_079_被告発人木梨松嗣弁護士への手紙078.jpg
  \url{https://kk2020-09.blogspot.com/2021/05/20210524.html\#79} 
\item
  甲号証(四)\_080_被告発人木梨松嗣弁護士への手紙079.jpg
  \url{https://kk2020-09.blogspot.com/2021/05/20210524.html\#80} 
\item
  甲号証(四)\_081_被告発人木梨松嗣弁護士への手紙080.jpg
  \url{https://kk2020-09.blogspot.com/2021/05/20210524.html\#81} 
\item
  甲号証(四)\_082_被告発人木梨松嗣弁護士への手紙081.jpg
  \url{https://kk2020-09.blogspot.com/2021/05/20210524.html\#82} 
\item
  甲号証(四)\_083_被告発人木梨松嗣弁護士への手紙082.jpg
  \url{https://kk2020-09.blogspot.com/2021/05/20210524.html\#83} 
\item
  甲号証(四)\_084_被告発人木梨松嗣弁護士への手紙083.jpg
  \url{https://kk2020-09.blogspot.com/2021/05/20210524.html\#84} 
\item
  甲号証(四)\_085_被告発人木梨松嗣弁護士への手紙084.jpg
  \url{https://kk2020-09.blogspot.com/2021/05/20210524.html\#85} 
\item
  甲号証(四)\_086_被告発人木梨松嗣弁護士への手紙085.jpg
  \url{https://kk2020-09.blogspot.com/2021/05/20210524.html\#86} 
\item
  甲号証(四)\_087_被告発人木梨松嗣弁護士への手紙086.jpg
  \url{https://kk2020-09.blogspot.com/2021/05/20210524.html\#87} 
\item
  甲号証(四)\_088_被告発人木梨松嗣弁護士への手紙087.jpg
  \url{https://kk2020-09.blogspot.com/2021/05/20210524.html\#88} 
\item
  甲号証(四)\_089_被告発人木梨松嗣弁護士への手紙088.jpg
  \url{https://kk2020-09.blogspot.com/2021/05/20210524.html\#89} 
\item
  甲号証(四)\_090_被告発人木梨松嗣弁護士への手紙089.jpg
  \url{https://kk2020-09.blogspot.com/2021/05/20210524.html\#90} 
\item
  甲号証(四)\_091_被告発人木梨松嗣弁護士への手紙090.jpg
  \url{https://kk2020-09.blogspot.com/2021/05/20210524.html\#91} 
\item
  甲号証(四)\_092_被告発人木梨松嗣弁護士への手紙091.jpg
  \url{https://kk2020-09.blogspot.com/2021/05/20210524.html\#92} 
\item
  甲号証(四)\_093_被告発人木梨松嗣弁護士への手紙092.jpg
  \url{https://kk2020-09.blogspot.com/2021/05/20210524.html\#93} 
\item
  甲号証(四)\_094_被告発人木梨松嗣弁護士への手紙093.jpg
  \url{https://kk2020-09.blogspot.com/2021/05/20210524.html\#94} 
\item
  甲号証(四)\_095_被告発人木梨松嗣弁護士への手紙094.jpg
  \url{https://kk2020-09.blogspot.com/2021/05/20210524.html\#95} 
\item
  甲号証(四)\_096_被告発人木梨松嗣弁護士への手紙095.jpg
  \url{https://kk2020-09.blogspot.com/2021/05/20210524.html\#96} 
\item
  甲号証(四)\_097_被告発人木梨松嗣弁護士への手紙096.jpg
  \url{https://kk2020-09.blogspot.com/2021/05/20210524.html\#97} 
\item
  甲号証(四)\_098_被告発人木梨松嗣弁護士への手紙097.jpg
  \url{https://kk2020-09.blogspot.com/2021/05/20210524.html\#98} 
\item
  甲号証(四)\_099_被告発人木梨松嗣弁護士への手紙098.jpg
  \url{https://kk2020-09.blogspot.com/2021/05/20210524.html\#99} 
\item
  甲号証(四)\_100_被告発人木梨松嗣弁護士への手紙099.jpg
  \url{https://kk2020-09.blogspot.com/2021/05/20210524.html\#100} 
\item
  甲号証(四)\_101_被告発人木梨松嗣弁護士への手紙100.jpg
  \url{https://kk2020-09.blogspot.com/2021/05/20210524.html\#101} 
\item
  甲号証(四)\_102_被告発人木梨松嗣弁護士への手紙101.jpg
  \url{https://kk2020-09.blogspot.com/2021/05/20210524.html\#102} 
\item
  甲号証(四)\_103_被告発人木梨松嗣弁護士への手紙102.jpg
  \url{https://kk2020-09.blogspot.com/2021/05/20210524.html\#103} 
\item
  甲号証(四)\_104_被告発人木梨松嗣弁護士への手紙103.jpg
  \url{https://kk2020-09.blogspot.com/2021/05/20210524.html\#104} 
\item
  甲号証(四)\_105_被告発人木梨松嗣弁護士への手紙104.jpg
  \url{https://kk2020-09.blogspot.com/2021/05/20210524.html\#105} 
\item
  甲号証(四)\_106_被告発人木梨松嗣弁護士への手紙105.jpg
  \url{https://kk2020-09.blogspot.com/2021/05/20210524.html\#106} 
\item
  甲号証(四)\_107_被告発人木梨松嗣弁護士への手紙106.jpg
  \url{https://kk2020-09.blogspot.com/2021/05/20210524.html\#107} 
\item
  甲号証(四)\_108_被告発人木梨松嗣弁護士への手紙107.jpg
  \url{https://kk2020-09.blogspot.com/2021/05/20210524.html\#108} 
\item
  甲号証(四)\_109_被告発人木梨松嗣弁護士への手紙108.jpg
  \url{https://kk2020-09.blogspot.com/2021/05/20210524.html\#109} 
\item
  甲号証(四)\_110_被告発人木梨松嗣弁護士への手紙109.jpg
  \url{https://kk2020-09.blogspot.com/2021/05/20210524.html\#110} 
\item
  甲号証(四)\_111_被告発人木梨松嗣弁護士への手紙110.jpg
  \url{https://kk2020-09.blogspot.com/2021/05/20210524.html\#111} 
\item
  甲号証(四)\_112_被告発人木梨松嗣弁護士への手紙111.jpg
  \url{https://kk2020-09.blogspot.com/2021/05/20210524.html\#112} 
\item
  甲号証(四)\_113_被告発人木梨松嗣弁護士への手紙112.jpg
  \url{https://kk2020-09.blogspot.com/2021/05/20210524.html\#113} 
\item
  甲号証(四)\_114_被告発人木梨松嗣弁護士への手紙113.jpg
  \url{https://kk2020-09.blogspot.com/2021/05/20210524.html\#114} 
\item
  甲号証(四)\_115_被告発人木梨松嗣弁護士への手紙114.jpg
  \url{https://kk2020-09.blogspot.com/2021/05/20210524.html\#115} 
\item
  甲号証(四)\_116_被告発人木梨松嗣弁護士への手紙115.jpg
  \url{https://kk2020-09.blogspot.com/2021/05/20210524.html\#116} 
\item
  甲号証(四)\_117_被告発人木梨松嗣弁護士への手紙116.jpg
  \url{https://kk2020-09.blogspot.com/2021/05/20210524.html\#117} 
\item
  甲号証(四)\_118_被告発人木梨松嗣弁護士への手紙117.jpg
  \url{https://kk2020-09.blogspot.com/2021/05/20210524.html\#118} 
\item
  甲号証(四)\_119_被告発人木梨松嗣弁護士への手紙118.jpg
  \url{https://kk2020-09.blogspot.com/2021/05/20210524.html\#119} 
\item
  甲号証(四)\_120_被告発人木梨松嗣弁護士への手紙119.jpg
  \url{https://kk2020-09.blogspot.com/2021/05/20210524.html\#120} 
\item
  甲号証(四)\_121_被告発人木梨松嗣弁護士への手紙120.jpg
  \url{https://kk2020-09.blogspot.com/2021/05/20210524.html\#121} 
\item
  甲号証(四)\_122_被告発人木梨松嗣弁護士への手紙121.jpg
  \url{https://kk2020-09.blogspot.com/2021/05/20210524.html\#122} 
\item
  甲号証(四)\_123_被告発人木梨松嗣弁護士への手紙122.jpg
  \url{https://kk2020-09.blogspot.com/2021/05/20210524.html\#123} 
\item
  甲号証(四)\_124_被告発人木梨松嗣弁護士への手紙123.jpg
  \url{https://kk2020-09.blogspot.com/2021/05/20210524.html\#124} 
\item
  甲号証(四)\_125_被告発人木梨松嗣弁護士への手紙124.jpg
  \url{https://kk2020-09.blogspot.com/2021/05/20210524.html\#125} 
\item
  甲号証(四)\_126_被告発人木梨松嗣弁護士への手紙125.jpg
  \url{https://kk2020-09.blogspot.com/2021/05/20210524.html\#126} 
\item
  甲号証(四)\_127_被告発人木梨松嗣弁護士への手紙126.jpg
  \url{https://kk2020-09.blogspot.com/2021/05/20210524.html\#127} 
\item
  甲号証(四)\_128_被告発人木梨松嗣弁護士への手紙127.jpg
  \url{https://kk2020-09.blogspot.com/2021/05/20210524.html\#128} 
\item
  甲号証(四)\_129_被告発人木梨松嗣弁護士への手紙128.jpg
  \url{https://kk2020-09.blogspot.com/2021/05/20210524.html\#129} 
\item
  甲号証(四)\_130_被告発人木梨松嗣弁護士への手紙129.jpg
  \url{https://kk2020-09.blogspot.com/2021/05/20210524.html\#130} 
\item
  甲号証(四)\_131_被告発人木梨松嗣弁護士への手紙130.jpg
  \url{https://kk2020-09.blogspot.com/2021/05/20210524.html\#131} 
\item
  甲号証(四)\_132_被告発人木梨松嗣弁護士への手紙131.jpg
  \url{https://kk2020-09.blogspot.com/2021/05/20210524.html\#132} 
\item
  甲号証(四)\_133_被告発人木梨松嗣弁護士への手紙132.jpg
  \url{https://kk2020-09.blogspot.com/2021/05/20210524.html\#133} 
\item
  甲号証(四)\_134_被告発人木梨松嗣弁護士への手紙133.jpg
  \url{https://kk2020-09.blogspot.com/2021/05/20210524.html\#134} 
\item
  甲号証(四)\_135_被告発人木梨松嗣弁護士への手紙134.jpg
  \url{https://kk2020-09.blogspot.com/2021/05/20210524.html\#135} 
\item
  甲号証(四)\_136_被告発人木梨松嗣弁護士への手紙135.jpg
  \url{https://kk2020-09.blogspot.com/2021/05/20210524.html\#136} 
\item
  甲号証(四)\_137_被告発人木梨松嗣弁護士への手紙136.jpg
  \url{https://kk2020-09.blogspot.com/2021/05/20210524.html\#137} 
\item
  甲号証(四)\_138_被告発人木梨松嗣弁護士への手紙137.jpg
  \url{https://kk2020-09.blogspot.com/2021/05/20210524.html\#138} 
\item
  甲号証(四)\_139_被告発人木梨松嗣弁護士への手紙138.jpg
  \url{https://kk2020-09.blogspot.com/2021/05/20210524.html\#139} 
\item
  甲号証(四)\_140_被告発人木梨松嗣弁護士への手紙139.jpg
  \url{https://kk2020-09.blogspot.com/2021/05/20210524.html\#140} 
\item
  甲号証(四)\_141_被告発人木梨松嗣弁護士への手紙140.jpg
  \url{https://kk2020-09.blogspot.com/2021/05/20210524.html\#141} 
\item
  甲号証(四)\_142_被告発人木梨松嗣弁護士への手紙141.jpg
  \url{https://kk2020-09.blogspot.com/2021/05/20210524.html\#142} 
\item
  甲号証(四)\_143_被告発人木梨松嗣弁護士への手紙142.jpg
  \url{https://kk2020-09.blogspot.com/2021/05/20210524.html\#143} 
\item
  甲号証(四)\_144_被告発人木梨松嗣弁護士への手紙143.jpg
  \url{https://kk2020-09.blogspot.com/2021/05/20210524.html\#144} 
\item
  甲号証(四)\_145_被告発人木梨松嗣弁護士への手紙144.jpg
  \url{https://kk2020-09.blogspot.com/2021/05/20210524.html\#145} 
\item
  甲号証(四)\_146_被告発人木梨松嗣弁護士への手紙145.jpg
  \url{https://kk2020-09.blogspot.com/2021/05/20210524.html\#146} 
\item
  甲号証(四)\_147_被告発人木梨松嗣弁護士への手紙146.jpg
  \url{https://kk2020-09.blogspot.com/2021/05/20210524.html\#147} 
\item
  甲号証(四)\_148_被告発人木梨松嗣弁護士への手紙147.jpg
  \url{https://kk2020-09.blogspot.com/2021/05/20210524.html\#148} 
\item
  甲号証(四)\_149_被告発人木梨松嗣弁護士への手紙148.jpg
  \url{https://kk2020-09.blogspot.com/2021/05/20210524.html} 
\item
  甲号証(四)\_150_被告発人木梨松嗣弁護士への手紙149.jpg
  \url{https://kk2020-09.blogspot.com/2021/05/20210524.html\#150} 
\item
  甲号証(四)\_151_被告発人木梨松嗣弁護士への手紙150.jpg
  \url{https://kk2020-09.blogspot.com/2021/05/20210524.html\#151} 
\item
  甲号証(四)\_152_被告発人木梨松嗣弁護士への手紙151.jpg
  \url{https://kk2020-09.blogspot.com/2021/05/20210524.html\#152} 
\item
  甲号証(四)\_153_被告発人木梨松嗣弁護士への手紙152.jpg
  \url{https://kk2020-09.blogspot.com/2021/05/20210524.html\#153} 
\item
  甲号証(四)\_154_被告発人木梨松嗣弁護士への手紙153.jpg
  \url{https://kk2020-09.blogspot.com/2021/05/20210524.html\#154} 
\item
  甲号証(四)\_155_被告発人木梨松嗣弁護士への手紙154.jpg
  \url{https://kk2020-09.blogspot.com/2021/05/20210524.html\#155} 
\item
  甲号証(四)\_156_被告発人木梨松嗣弁護士への手紙155.jpg
  \url{https://kk2020-09.blogspot.com/2021/05/20210524.html\#156} 
\item
  甲号証(四)\_157_被告発人木梨松嗣弁護士への手紙156.jpg
  \url{https://kk2020-09.blogspot.com/2021/05/20210524.html\#157} 
\item
  甲号証(四)\_158_被告発人木梨松嗣弁護士への手紙157.jpg
  \url{https://kk2020-09.blogspot.com/2021/05/20210524.html\#158} 
\item
  甲号証(四)\_159_被告発人木梨松嗣弁護士への手紙158.jpg
  \url{https://kk2020-09.blogspot.com/2021/05/20210524.html\#159} 
\item
  甲号証(四)\_160_被告発人木梨松嗣弁護士への手紙159.jpg
  \url{https://kk2020-09.blogspot.com/2021/05/20210524.html\#160} 
\item
  甲号証(四)\_161_被告発人木梨松嗣弁護士への手紙160.jpg
  \url{https://kk2020-09.blogspot.com/2021/05/20210524.html\#161} 
\item
  甲号証(四)\_162_被告発人木梨松嗣弁護士への手紙161.jpg
  \url{https://kk2020-09.blogspot.com/2021/05/20210524.html\#162} 
\item
  甲号証(四)\_163_被告発人木梨松嗣弁護士への手紙162.jpg
  \url{https://kk2020-09.blogspot.com/2021/05/20210524.html\#163} 
\item
  甲号証(四)\_164_被告発人木梨松嗣弁護士への手紙163.jpg
  \url{https://kk2020-09.blogspot.com/2021/05/20210524.html\#164} 
\item
  甲号証(四)\_165_被告発人木梨松嗣弁護士への手紙164.jpg
  \url{https://kk2020-09.blogspot.com/2021/05/20210524.html\#165} 
\item
  甲号証(四)\_166_被告発人木梨松嗣弁護士への手紙165.jpg
  \url{https://kk2020-09.blogspot.com/2021/05/20210524.html\#166} 
\item
  甲号証(四)\_167_被告発人木梨松嗣弁護士への手紙166.jpg
  \url{https://kk2020-09.blogspot.com/2021/05/20210524.html\#167} 
\item
  甲号証(四)\_168_被告発人木梨松嗣弁護士への手紙167.jpg
  \url{https://kk2020-09.blogspot.com/2021/05/20210524.html\#168} 
\item
  甲号証(四)\_169_被告発人木梨松嗣弁護士への手紙168.jpg
  \url{https://kk2020-09.blogspot.com/2021/05/20210524.html\#169} 
\item
  甲号証(四)\_170_被告発人木梨松嗣弁護士への手紙169.jpg
  \url{https://kk2020-09.blogspot.com/2021/05/20210524.html\#170} 
\item
  甲号証(四)\_171_被告発人木梨松嗣弁護士への手紙170.jpg
  \url{https://kk2020-09.blogspot.com/2021/05/20210524.html\#171} 
\item
  甲号証(四)\_172_被告発人木梨松嗣弁護士への手紙171.jpg
  \url{https://kk2020-09.blogspot.com/2021/05/20210524.html\#172} 
\item
  甲号証(四)\_173_被告発人木梨松嗣弁護士への手紙172.jpg
  \url{https://kk2020-09.blogspot.com/2021/05/20210524.html\#173} 
\item
  甲号証(四)\_174_被告発人木梨松嗣弁護士への手紙173.jpg
  \url{https://kk2020-09.blogspot.com/2021/05/20210524.html\#174} 
\item
  甲号証(四)\_175_被告発人木梨松嗣弁護士への手紙174.jpg
  \url{https://kk2020-09.blogspot.com/2021/05/20210524.html\#175} 
\item
  甲号証(四)\_176_被告発人木梨松嗣弁護士への手紙175.jpg
  \url{https://kk2020-09.blogspot.com/2021/05/20210524.html\#176} 
\item
  甲号証(四)\_177_被告発人木梨松嗣弁護士への手紙176.jpg
  \url{https://kk2020-09.blogspot.com/2021/05/20210524.html\#177} 
\item
  甲号証(四)\_178_被告発人木梨松嗣弁護士への手紙177.jpg
  \url{https://kk2020-09.blogspot.com/2021/05/20210524.html\#178} 
\item
  甲号証(四)\_179_被告発人木梨松嗣弁護士への手紙178.jpg
  \url{https://kk2020-09.blogspot.com/2021/05/20210524.html\#179} 
\item
  甲号証(四)\_180_被告発人木梨松嗣弁護士への手紙179.jpg
  \url{https://kk2020-09.blogspot.com/2021/05/20210524.html\#180} 
\item
  甲号証(四)\_181_被告発人木梨松嗣弁護士への手紙180.jpg
  \url{https://kk2020-09.blogspot.com/2021/05/20210524.html\#181} 
\item
  甲号証(四)\_182_被告発人木梨松嗣弁護士への手紙181.jpg
  \url{https://kk2020-09.blogspot.com/2021/05/20210524.html\#182} 
\item
  甲号証(四)\_183_被告発人木梨松嗣弁護士への手紙182.jpg
  \url{https://kk2020-09.blogspot.com/2021/05/20210524.html\#183} 
\item
  甲号証(四)\_184_被告発人木梨松嗣弁護士への手紙183.jpg
  \url{https://kk2020-09.blogspot.com/2021/05/20210524.html\#184} 
\item
  甲号証(四)\_185_被告発人木梨松嗣弁護士への手紙184.jpg
  \url{https://kk2020-09.blogspot.com/2021/05/20210524.html\#185} 
\item
  甲号証(四)\_186_被告発人木梨松嗣弁護士への手紙185.jpg
  \url{https://kk2020-09.blogspot.com/2021/05/20210524.html\#186} 
\item
  甲号証(四)\_187_被告発人木梨松嗣弁護士への手紙186.jpg
  \url{https://kk2020-09.blogspot.com/2021/05/20210524.html\#187} 
\item
  甲号証(四)\_188_被告発人木梨松嗣弁護士への手紙187.jpg
  \url{https://kk2020-09.blogspot.com/2021/05/20210524.html\#188} 
\item
  甲号証(四)\_189_被告発人木梨松嗣弁護士への手紙188.jpg
  \url{https://kk2020-09.blogspot.com/2021/05/20210524.html\#189} 
\item
  甲号証(四)\_190_被告発人木梨松嗣弁護士への手紙189.jpg
  \url{https://kk2020-09.blogspot.com/2021/05/20210524.html\#190} 
\item
  甲号証(四)\_191_被告発人木梨松嗣弁護士への手紙190.jpg
  \url{https://kk2020-09.blogspot.com/2021/05/20210524.html\#191} 
\item
  甲号証(四)\_192_被告発人木梨松嗣弁護士への手紙191.jpg
  \url{https://kk2020-09.blogspot.com/2021/05/20210524.html\#192} 
\item
  甲号証(四)\_193_被告発人木梨松嗣弁護士への手紙192.jpg
  \url{https://kk2020-09.blogspot.com/2021/05/20210524.html\#193} 
\item
  甲号証(四)\_194_被告発人木梨松嗣弁護士への手紙193.jpg
  \url{https://kk2020-09.blogspot.com/2021/05/20210524.html\#194} 
\item
  甲号証(四)\_195_被告発人木梨松嗣弁護士への手紙194.jpg
  \url{https://kk2020-09.blogspot.com/2021/05/20210524.html\#195} 
\item
  甲号証(四)\_196_被告発人木梨松嗣弁護士への手紙195.jpg
  \url{https://kk2020-09.blogspot.com/2021/05/20210524.html\#196} 
\item
  甲号証(四)\_197_被告発人木梨松嗣弁護士への手紙196.jpg
  \url{https://kk2020-09.blogspot.com/2021/05/20210524.html\#197} 
\item
  甲号証(四)\_198_被告発人木梨松嗣弁護士への手紙197.jpg
  \url{https://kk2020-09.blogspot.com/2021/05/20210524.html\#198} 
\item
  甲号証(四)\_199_被告発人木梨松嗣弁護士への手紙198.jpg
  \url{https://kk2020-09.blogspot.com/2021/05/20210524.html\#199} 
\item
  甲号証(四)\_200_被告発人木梨松嗣弁護士への手紙199.jpg
  \url{https://kk2020-09.blogspot.com/2021/05/20210524.html\#200} 
\item
  甲号証(四)\_201_被告発人木梨松嗣弁護士への手紙200.jpg
  \url{https://kk2020-09.blogspot.com/2021/05/20210524.html\#201} 
\item
  甲号証(四)\_202_被告発人木梨松嗣弁護士への手紙201.jpg
  \url{https://kk2020-09.blogspot.com/2021/05/20210524.html\#202} 
\item
  甲号証(四)\_203_被告発人木梨松嗣弁護士への手紙202.jpg
  \url{https://kk2020-09.blogspot.com/2021/05/20210524.html\#203} 
\item
  甲号証(四)\_204_被告発人木梨松嗣弁護士への手紙203.jpg
  \url{https://kk2020-09.blogspot.com/2021/05/20210524.html\#204} 
\item
  甲号証(四)\_205_被告発人木梨松嗣弁護士への手紙204.jpg
  \url{https://kk2020-09.blogspot.com/2021/05/20210524.html\#205} 
\item
  甲号証(四)\_206_被告発人木梨松嗣弁護士への手紙205.jpg
  \url{https://kk2020-09.blogspot.com/2021/05/20210524.html\#206} 
\item
  甲号証(四)\_207_被告発人木梨松嗣弁護士への手紙206.jpg
  \url{https://kk2020-09.blogspot.com/2021/05/20210524.html\#207} 
\item
  甲号証(四)\_208_被告発人木梨松嗣弁護士への手紙207.jpg
  \url{https://kk2020-09.blogspot.com/2021/05/20210524.html\#208} 
\item
  甲号証(四)\_209_被告発人木梨松嗣弁護士への手紙208.jpg
  \url{https://kk2020-09.blogspot.com/2021/05/20210524.html\#209} 
\item
  甲号証(四)\_210_被告発人木梨松嗣弁護士への手紙209.jpg
  \url{https://kk2020-09.blogspot.com/2021/05/20210524.html\#210} 
\item
  甲号証(四)\_211_被告発人木梨松嗣弁護士への手紙210.jpg
  \url{https://kk2020-09.blogspot.com/2021/05/20210524.html\#211} 
\item
  甲号証(四)\_212_被告発人木梨松嗣弁護士への手紙211.jpg
  \url{https://kk2020-09.blogspot.com/2021/05/20210524.html\#212} 
\item
  甲号証(四)\_213_被告発人木梨松嗣弁護士への手紙212.jpg
  \url{https://kk2020-09.blogspot.com/2021/05/20210524.html\#213} 
\item
  甲号証(四)\_214_被告発人木梨松嗣弁護士への手紙213.jpg
  \url{https://kk2020-09.blogspot.com/2021/05/20210524.html\#214} 
\item
  甲号証(四)\_215_被告発人木梨松嗣弁護士への手紙214.jpg
  \url{https://kk2020-09.blogspot.com/2021/05/20210524.html\#215} 
\item
  甲号証(四)\_216_被告発人木梨松嗣弁護士への手紙215.jpg
  \url{https://kk2020-09.blogspot.com/2021/05/20210524.html\#216} 
\item
  甲号証(四)\_217_被告発人木梨松嗣弁護士への手紙216.jpg
  \url{https://kk2020-09.blogspot.com/2021/05/20210524.html\#217} 
\item
  甲号証(四)\_218_被告発人木梨松嗣弁護士への手紙217.jpg
  \url{https://kk2020-09.blogspot.com/2021/05/20210524.html\#218} 
\item
  甲号証(四)\_219_被告発人木梨松嗣弁護士への手紙218.jpg
  \url{https://kk2020-09.blogspot.com/2021/05/20210524.html\#219} 
\item
  甲号証(四)\_220_被告発人木梨松嗣弁護士への手紙219.jpg
  \url{https://kk2020-09.blogspot.com/2021/05/20210524.html\#220} 
\item
  甲号証(四)\_221_被告発人木梨松嗣弁護士への手紙220.jpg
  \url{https://kk2020-09.blogspot.com/2021/05/20210524.html\#221} 
\item
  甲号証(四)\_222_被告発人木梨松嗣弁護士への手紙221.jpg
  \url{https://kk2020-09.blogspot.com/2021/05/20210524.html\#222} 
\item
  甲号証(四)\_223_被告発人木梨松嗣弁護士への手紙222.jpg
  \url{https://kk2020-09.blogspot.com/2021/05/20210524.html\#223} 
\item
  甲号証(四)\_224_被告発人木梨松嗣弁護士への手紙223.jpg
  \url{https://kk2020-09.blogspot.com/2021/05/20210524.html\#224} 
\item
  甲号証(四)\_225_被告発人木梨松嗣弁護士への手紙224.jpg
  \url{https://kk2020-09.blogspot.com/2021/05/20210524.html\#225} 
\item
  甲号証(四)\_226_被告発人木梨松嗣弁護士への手紙225.jpg
  \url{https://kk2020-09.blogspot.com/2021/05/20210524.html\#226} 
\item
  甲号証(四)\_227_被告発人木梨松嗣弁護士への手紙226.jpg
  \url{https://kk2020-09.blogspot.com/2021/05/20210524.html\#227} 
\item
  甲号証(四)\_228_被告発人木梨松嗣弁護士への手紙227.jpg
  \url{https://kk2020-09.blogspot.com/2021/05/20210524.html\#228} 
\item
  甲号証(四)\_229_被告発人木梨松嗣弁護士への手紙228.jpg
  \url{https://kk2020-09.blogspot.com/2021/05/20210524.html\#229} 
\item
  甲号証(四)\_230_被告発人木梨松嗣弁護士への手紙229.jpg
  \url{https://kk2020-09.blogspot.com/2021/05/20210524.html\#230} 
\item
  甲号証(四)\_231_被告発人木梨松嗣弁護士への手紙230.jpg
  \url{https://kk2020-09.blogspot.com/2021/05/20210524.html\#231} 
\item
  甲号証(四)\_232_被告発人木梨松嗣弁護士への手紙231.jpg
  \url{https://kk2020-09.blogspot.com/2021/05/20210524.html\#232} 
\item
  甲号証(四)\_233_被告発人木梨松嗣弁護士への手紙232.jpg
  \url{https://kk2020-09.blogspot.com/2021/05/20210524.html\#233} 
\item
  甲号証(四)\_234_被告発人木梨松嗣弁護士への手紙233.jpg
  \url{https://kk2020-09.blogspot.com/2021/05/20210524.html\#234} 
\item
  甲号証(四)\_235_被告発人木梨松嗣弁護士への手紙234.jpg
  \url{https://kk2020-09.blogspot.com/2021/05/20210524.html\#235} 
\item
  甲号証(四)\_236_被告発人木梨松嗣弁護士への手紙235.jpg
  \url{https://kk2020-09.blogspot.com/2021/05/20210524.html\#236} 
\item
  甲号証(四)\_237_被告発人木梨松嗣弁護士への手紙236.jpg
  \url{https://kk2020-09.blogspot.com/2021/05/20210524.html\#237} 
\item
  甲号証(四)\_238_被告発人木梨松嗣弁護士への手紙237.jpg
  \url{https://kk2020-09.blogspot.com/2021/05/20210524.html\#238} 
\item
  甲号証(四)\_239_被告発人木梨松嗣弁護士への手紙238.jpg
  \url{https://kk2020-09.blogspot.com/2021/05/20210524.html\#239} 
\item
  甲号証(四)\_240_被告発人木梨松嗣弁護士への手紙239.jpg
  \url{https://kk2020-09.blogspot.com/2021/05/20210524.html\#240} 
\item
  甲号証(四)\_241_被告発人木梨松嗣弁護士への手紙240.jpg
  \url{https://kk2020-09.blogspot.com/2021/05/20210524.html\#241} 
\item
  甲号証(四)\_242_被告発人木梨松嗣弁護士への手紙241.jpg
  \url{https://kk2020-09.blogspot.com/2021/05/20210524.html\#242} 
\item
  甲号証(四)\_243_被告発人木梨松嗣弁護士への手紙242.jpg
  \url{https://kk2020-09.blogspot.com/2021/05/20210524.html\#243} 
\item
  甲号証(四)\_244_被告発人木梨松嗣弁護士への手紙243.jpg
  \url{https://kk2020-09.blogspot.com/2021/05/20210524.html\#244} 
\item
  甲号証(四)\_245_被告発人木梨松嗣弁護士への手紙244.jpg
  \url{https://kk2020-09.blogspot.com/2021/05/20210524.html\#245} 
\item
  甲号証(四)\_246_被告発人木梨松嗣弁護士への手紙245.jpg
  \url{https://kk2020-09.blogspot.com/2021/05/20210524.html\#246} 
\item
  甲号証(四)\_247_被告発人木梨松嗣弁護士への手紙246.jpg
  \url{https://kk2020-09.blogspot.com/2021/05/20210524.html\#247} 
\item
  甲号証(四)\_248_被告発人木梨松嗣弁護士への手紙247.jpg
  \url{https://kk2020-09.blogspot.com/2021/05/20210524.html\#248} 
\item
  甲号証(四)\_249_被告発人木梨松嗣弁護士への手紙248.jpg
  \url{https://kk2020-09.blogspot.com/2021/05/20210524.html\#249} 
\item
  甲号証(四)\_250_被告発人木梨松嗣弁護士への手紙249.jpg
  \url{https://kk2020-09.blogspot.com/2021/05/20210524.html\#250} 
\item
  甲号証(四)\_251_被告発人木梨松嗣弁護士への手紙250.jpg
  \url{https://kk2020-09.blogspot.com/2021/05/20210524.html\#251} 
\item
  甲号証(四)\_252_被告発人木梨松嗣弁護士への手紙251.jpg
  \url{https://kk2020-09.blogspot.com/2021/05/20210524.html\#252} 
\item
  甲号証(四)\_253_被告発人木梨松嗣弁護士への手紙252.jpg
  \url{https://kk2020-09.blogspot.com/2021/05/20210524.html\#253} 
\item
  甲号証(四)\_254_被告発人木梨松嗣弁護士への手紙253.jpg
  \url{https://kk2020-09.blogspot.com/2021/05/20210524.html\#254} 
\item
  甲号証(四)\_255_被告発人木梨松嗣弁護士への手紙254.jpg
  \url{https://kk2020-09.blogspot.com/2021/05/20210524.html\#255} 
\item
  甲号証(四)\_256_被告発人木梨松嗣弁護士への手紙255.jpg
  \url{https://kk2020-09.blogspot.com/2021/05/20210524.html\#256} 
\item
  甲号証(四)\_257_被告発人木梨松嗣弁護士への手紙256.jpg
  \url{https://kk2020-09.blogspot.com/2021/05/20210524.html\#257} 
\item
  甲号証(四)\_258_被告発人木梨松嗣弁護士への手紙257.jpg
  \url{https://kk2020-09.blogspot.com/2021/05/20210524.html\#258} 
\item
  甲号証(四)\_259_被告発人木梨松嗣弁護士への手紙258.jpg
  \url{https://kk2020-09.blogspot.com/2021/05/20210524.html\#259} 
\item
  甲号証(四)\_260_被告発人木梨松嗣弁護士への手紙259.jpg
  \url{https://kk2020-09.blogspot.com/2021/05/20210524.html\#260} 
\item
  甲号証(四)\_261_被告発人木梨松嗣弁護士への手紙260.jpg
  \url{https://kk2020-09.blogspot.com/2021/05/20210524.html\#261} 
\item
  甲号証(四)\_262_被告発人木梨松嗣弁護士への手紙261.jpg
  \url{https://kk2020-09.blogspot.com/2021/05/20210524.html\#262} 
\item
  甲号証(四)\_263_被告発人木梨松嗣弁護士への手紙262.jpg
  \url{https://kk2020-09.blogspot.com/2021/05/20210524.html\#263} 
\item
  甲号証(四)\_264_被告発人木梨松嗣弁護士への手紙263.jpg
  \url{https://kk2020-09.blogspot.com/2021/05/20210524.html\#264} 
\item
  甲号証(四)\_265_被告発人木梨松嗣弁護士への手紙264.jpg
  \url{https://kk2020-09.blogspot.com/2021/05/20210524.html\#265} 
\item
  甲号証(四)\_266_被告発人木梨松嗣弁護士への手紙265.jpg
  \url{https://kk2020-09.blogspot.com/2021/05/20210524.html\#266} 
\item
  甲号証(四)\_267_被告発人木梨松嗣弁護士への手紙266.jpg
  \url{https://kk2020-09.blogspot.com/2021/05/20210524.html\#267} 
\item
  甲号証(四)\_268_被告発人木梨松嗣弁護士への手紙267.jpg
  \url{https://kk2020-09.blogspot.com/2021/05/20210524.html\#268} 
\item
  甲号証(四)\_269_被告発人木梨松嗣弁護士への手紙268.jpg
  \url{https://kk2020-09.blogspot.com/2021/05/20210524.html\#269} 
\item
  甲号証(四)\_270_被告発人木梨松嗣弁護士への手紙269.jpg
  \url{https://kk2020-09.blogspot.com/2021/05/20210524.html\#270} 
\item
  甲号証(四)\_271_被告発人木梨松嗣弁護士への手紙270.jpg
  \url{https://kk2020-09.blogspot.com/2021/05/20210524.html\#271} 
\item
  甲号証(四)\_272_被告発人木梨松嗣弁護士への手紙271.jpg
  \url{https://kk2020-09.blogspot.com/2021/05/20210524.html\#272} 
\item
  甲号証(四)\_273_被告発人木梨松嗣弁護士への手紙272.jpg
  \url{https://kk2020-09.blogspot.com/2021/05/20210524.html\#273} 
\item
  甲号証(四)\_274_被告発人木梨松嗣弁護士への手紙273.jpg
  \url{https://kk2020-09.blogspot.com/2021/05/20210524.html\#274} 
\item
  甲号証(四)\_275_被告発人木梨松嗣弁護士への手紙274.jpg
  \url{https://kk2020-09.blogspot.com/2021/05/20210524.html\#275} 
\item
  甲号証(四)\_276_被告発人木梨松嗣弁護士への手紙275.jpg
  \url{https://kk2020-09.blogspot.com/2021/05/20210524.html\#276} 
\item
  甲号証(四)\_277_被告発人木梨松嗣弁護士への手紙276.jpg
  \url{https://kk2020-09.blogspot.com/2021/05/20210524.html\#277} 
\item
  甲号証(四)\_278_被告発人木梨松嗣弁護士への手紙277.jpg
  \url{https://kk2020-09.blogspot.com/2021/05/20210524.html\#278} 
\item
  甲号証(四)\_279_被告発人木梨松嗣弁護士への手紙278.jpg
  \url{https://kk2020-09.blogspot.com/2021/05/20210524.html\#279} 
\item
  甲号証(四)\_280_被告発人木梨松嗣弁護士への手紙279.jpg
  \url{https://kk2020-09.blogspot.com/2021/05/20210524.html\#280} 
\item
  甲号証(四)\_281_被告発人木梨松嗣弁護士への手紙280.jpg
  \url{https://kk2020-09.blogspot.com/2021/05/20210524.html\#281} 
\item
  甲号証(四)\_282_被告発人木梨松嗣弁護士への手紙281.jpg
  \url{https://kk2020-09.blogspot.com/2021/05/20210524.html\#282} 
\item
  甲号証(四)\_283_被告発人木梨松嗣弁護士への手紙282.jpg
  \url{https://kk2020-09.blogspot.com/2021/05/20210524.html\#283} 
\item
  甲号証(四)\_284_被告発人木梨松嗣弁護士への手紙283.jpg
  \url{https://kk2020-09.blogspot.com/2021/05/20210524.html\#284} 
\item
  甲号証(四)\_285_被告発人木梨松嗣弁護士への手紙284.jpg
  \url{https://kk2020-09.blogspot.com/2021/05/20210524.html\#285} 
\item
  甲号証(四)\_286_被告発人木梨松嗣弁護士への手紙285.jpg
  \url{https://kk2020-09.blogspot.com/2021/05/20210524.html\#286} 
\item
  甲号証(四)\_287_被告発人木梨松嗣弁護士への手紙286.jpg
  \url{https://kk2020-09.blogspot.com/2021/05/20210524.html\#287} 
\item
  甲号証(四)\_288_被告発人木梨松嗣弁護士への手紙287.jpg
  \url{https://kk2020-09.blogspot.com/2021/05/20210524.html\#288} 
\item
  甲号証(四)\_289_被告発人木梨松嗣弁護士への手紙288.jpg
  \url{https://kk2020-09.blogspot.com/2021/05/20210524.html\#289} 
\item
  甲号証(四)\_290_被告発人木梨松嗣弁護士への手紙289.jpg
  \url{https://kk2020-09.blogspot.com/2021/05/20210524.html\#290} 
\item
  甲号証(四)\_291_被告発人木梨松嗣弁護士への手紙290.jpg
  \url{https://kk2020-09.blogspot.com/2021/05/20210524.html\#291} 
\item
  甲号証(四)\_292_被告発人木梨松嗣弁護士への手紙291.jpg
  \url{https://kk2020-09.blogspot.com/2021/05/20210524.html\#292} 
\item
  甲号証(四)\_293_被告発人木梨松嗣弁護士への手紙292.jpg
  \url{https://kk2020-09.blogspot.com/2021/05/20210524.html\#293} 
\item
  甲号証(四)\_294_被告発人木梨松嗣弁護士への手紙293.jpg
  \url{https://kk2020-09.blogspot.com/2021/05/20210524.html\#294} 
\end{itemize}

 222から292の間を飛ばしてツイートしましたが、ファイル名と写真が対応していない可能性もありそうです。一から撮影をやり直した方がよいとも考えていますが、個別に確認した上で取り上げる分には支障がなさそうです。

 告発事件の本体である被告発人木梨松嗣弁護士や被告発人小島裕史裁判長の記録資料について取り上げたところですが、深澤諭史弁護士とうの字についても取り上げておきたいことがあり、時間が経つほど作業の効率が悪くなるので、さっそくやっておきたいところです。

\hypertarget{section-8}{%
\paragraph{}\label{section-8}}

\hypertarget{ux88abux544aux767aux4ebaux9577ux8c37ux5dddux7d18ux4e4bux5f01ux8b77ux58eb}{%
\subsubsection{被告発人長谷川紘之弁護士}\label{ux88abux544aux767aux4ebaux9577ux8c37ux5dddux7d18ux4e4bux5f01ux8b77ux58eb}}

\hypertarget{ux5211ux88c1ux30b5ux30a4ux592aux306eux30bfux30a4ux30e0ux30e9ux30a4ux30f3ux304cux304dux3063ux304bux3051ux3067ux4fddux967aux91d1ux6bbaux4ebaux306eux61f2ux5f7919ux5e74ux306eux5224ux6c7aux6587ux821eux9db4ux5973ux5b50ux9ad8ux6821ux751fux6bbaux5bb3ux4e8bux4ef6ux306eux88abux5bb3ux8005ux306eux7d30ux5dddux6cbbux5f01ux8b77ux58ebux306eux30d6ux30edux30b0ux767aux898bux307eux3067}{%
\paragraph{刑裁サイ太のタイムラインがきっかけで,保険金殺人の懲役19年の判決文,舞鶴女子高校生殺害事件の被害者の細川治弁護士のブログ発見まで}\label{ux5211ux88c1ux30b5ux30a4ux592aux306eux30bfux30a4ux30e0ux30e9ux30a4ux30f3ux304cux304dux3063ux304bux3051ux3067ux4fddux967aux91d1ux6bbaux4ebaux306eux61f2ux5f7919ux5e74ux306eux5224ux6c7aux6587ux821eux9db4ux5973ux5b50ux9ad8ux6821ux751fux6bbaux5bb3ux4e8bux4ef6ux306eux88abux5bb3ux8005ux306eux7d30ux5dddux6cbbux5f01ux8b77ux58ebux306eux30d6ux30edux30b0ux767aux898bux307eux3067}}

\begin{itemize}
\tightlist
\item
  〉〉〉 Linux Emacs: 2021/05/16 08:39:30 〉〉〉
\end{itemize}

:CATEGORIES: @kanazawabengosi \#金沢弁護士会 @JFBAsns
日本弁護士連合会(日弁連) \#法務省 @MOJ\_HOUMU
\#被告発人長谷川紘之弁護士

 本告発状の見出しの階層レベル3が「被告発人長谷川紘之弁護士」で,その下のレベル4では初めての項目追加となります。項目というかエントリーの見出しには長谷川紘之弁護士を含めていませんが,長谷川紘之弁護士と被害者安藤文さん家族との関わりを軸に取り上げていきます。

 まず,これまでの流れを奉納\さらば弁護士鉄道・泥棒神社の物語(@hirono\_hideki)のツイートで見ていきたいと思います。

〉〉〉 kk\_hironoのリツイート 〉〉〉

\begin{itemize}
\tightlist
\item
  RT
  kk\_hirono(刑事告発・非常上告_金沢地方検察庁御中)|hirono\_hideki(奉納\さらば弁護士鉄道・泥棒神社の物語)
  日時:2021-05-16 08:46/2021/05/16 08:18 URL:
  \url{https://twitter.com/kk\_hirono/status/1393714460687167489} 
  \url{https://twitter.com/hirono\_hideki/status/1393707273495613440} 
  \textgreater{} 舞鶴事件被害者遺族のコメント \textbar{}
  弁護士細川治のそりゃないぜ!日記 \url{https://t.co/blCq44JmYk} 
  一昨日、偶然にも、母親は京都府警から美穂さんの遺品等の還付を受けていました。またその際には、京都府警担当者から、舞鶴事件についての再捜査の状況等についても説明を受けたとのことでした。
\end{itemize}

〉〉〉 kk\_hironoのリツイート 〉〉〉

\begin{itemize}
\tightlist
\item
  RT
  kk\_hirono(刑事告発・非常上告_金沢地方検察庁御中)|hirono\_hideki(奉納\さらば弁護士鉄道・泥棒神社の物語)
  日時:2021-05-16 08:46/2021/05/16 08:15 URL:
  \url{https://twitter.com/kk\_hirono/status/1393714480391987200} 
  \url{https://twitter.com/hirono\_hideki/status/1393706633130233856} 
  \textgreater{} 舞鶴事件最高裁決定時のコメント \textbar{}
  弁護士細川治のそりゃないぜ!日記 \url{https://t.co/3IJa7uFfAj} 
  その後、被害者遺族の母親と叔母(と小職)が最高検、京都府警から事件の総括などについて説明を受けていますが、京都府警からの説明後に記者会見を行っています。
\end{itemize}

〉〉〉 kk\_hironoのリツイート 〉〉〉

\begin{itemize}
\tightlist
\item
  RT
  kk\_hirono(刑事告発・非常上告_金沢地方検察庁御中)|hirono\_hideki(奉納\さらば弁護士鉄道・泥棒神社の物語)
  日時:2021-05-16 08:46/2021/05/16 08:13 URL:
  \url{https://twitter.com/kk\_hirono/status/1393714506035994626} 
  \url{https://twitter.com/hirono\_hideki/status/1393706216480677890} 
  \textgreater{} 中勝美被疑者逮捕報道と、舞鶴事件の扱い \textbar{}
  弁護士細川治のそりゃないぜ!日記 \url{https://t.co/stBhr2PqVA} 
  純粋な感想ですが、京都新聞が母親のコメントに触れていないのが意外に感じました。
  無罪が確定していることをどのように解釈するか、ということで各社のスタンスが分かれたのでしょう。
\end{itemize}

〉〉〉 kk\_hironoのリツイート 〉〉〉

\begin{itemize}
\tightlist
\item
  RT
  kk\_hirono(刑事告発・非常上告_金沢地方検察庁御中)|hirono\_hideki(奉納\さらば弁護士鉄道・泥棒神社の物語)
  日時:2021-05-16 08:46/2021/05/16 08:08 URL:
  \url{https://twitter.com/kk\_hirono/status/1393714524746747907} 
  \url{https://twitter.com/hirono\_hideki/status/1393704970612920320} 
  \textgreater{} 舞鶴事件、放送を見ての感想など。 \textbar{}
  弁護士細川治のそりゃないぜ!日記 \url{https://t.co/tcHirVqKLL} 
\end{itemize}

〉〉〉 kk\_hironoのリツイート 〉〉〉

\begin{itemize}
\tightlist
\item
  RT
  kk\_hirono(刑事告発・非常上告_金沢地方検察庁御中)|yamashiromisora(弁護士細川治@みそら)
  日時:2021-05-16 08:46/2021/05/07 12:22 URL:
  \url{https://twitter.com/kk\_hirono/status/1393714555839082496} 
  \url{https://twitter.com/yamashiromisora/status/1390507339728396290} 
  \textgreater{} 舞鶴事件のその後 ⇒ \url{https://t.co/6H1EEcxwCM}  \#アメブロ
  @ameba\_officialより
\end{itemize}

〉〉〉 kk\_hironoのリツイート 〉〉〉

\begin{itemize}
\tightlist
\item
  RT
  kk\_hirono(刑事告発・非常上告_金沢地方検察庁御中)|hirono\_hideki(奉納\さらば弁護士鉄道・泥棒神社の物語)
  日時:2021-05-16 08:47/2021/05/16 08:04 URL:
  \url{https://twitter.com/kk\_hirono/status/1393714572540796931} 
  \url{https://twitter.com/hirono\_hideki/status/1393703803631783936} 
  \textgreater{}
  「舞鶴高1年女子」の母「一番恐れていたことが現実になった」・・・被害者の代理人が会見
  - 産経WEST \url{https://t.co/M373yKr63s} 
  小杉さんの母親(44)の代理人を務める細川治弁護士が5日、京都市内で会見を開き、「驚きとともに憤りを感じた。一番恐れていたことが現実となってしまった。捜査機関には、
\end{itemize}

〉〉〉 kk\_hironoのリツイート 〉〉〉

\begin{itemize}
\tightlist
\item
  RT
  kk\_hirono(刑事告発・非常上告_金沢地方検察庁御中)|hirono\_hideki(奉納\さらば弁護士鉄道・泥棒神社の物語)
  日時:2021-05-16 08:47/2021/05/16 08:02 URL:
  \url{https://twitter.com/kk\_hirono/status/1393714596473573377} 
  \url{https://twitter.com/hirono\_hideki/status/1393703241477627906} 
  \textgreater{}
  首相は「どこかの神社のお猿さん」 立民・安住氏、日光の三猿なぞらえ批判
  - 産経ニュース \url{https://t.co/SM5pnVN2Q5} 
\end{itemize}

〉〉〉 kk\_hironoのリツイート 〉〉〉

\begin{itemize}
\tightlist
\item
  RT
  kk\_hirono(刑事告発・非常上告_金沢地方検察庁御中)|hirono\_hideki(奉納\さらば弁護士鉄道・泥棒神社の物語)
  日時:2021-05-16 08:47/2021/05/16 08:00 URL:
  \url{https://twitter.com/kk\_hirono/status/1393714619311550466} 
  \url{https://twitter.com/hirono\_hideki/status/1393702974807896070} 
  \textgreater{}
  妻殺害認定の鍵は「砂」 両親の思いと裏腹に審理は高裁へ(1/3ページ) -
  産経ニュース \url{https://t.co/MNRd7Jmexr}  2021.4.26 11:00社会裁判 Twitter
  反応 Facebook
\end{itemize}

〉〉〉 kk\_hironoのリツイート 〉〉〉

\begin{itemize}
\tightlist
\item
  RT
  kk\_hirono(刑事告発・非常上告_金沢地方検察庁御中)|hirono\_hideki(奉納\さらば弁護士鉄道・泥棒神社の物語)
  日時:2021-05-16 08:49/2021/05/16 07:57 URL:
  \url{https://twitter.com/kk\_hirono/status/1393715215389192194} 
  \url{https://twitter.com/hirono\_hideki/status/1393702179978891265} 
  \textgreater{}
  【自殺撮影】JR函館線苗穂駅 自殺を撮影、ネット公開か 飛び込み死亡の男女(10〜20代)
  : まとめダネ! \url{https://t.co/tvqW8hY4O3} 
\end{itemize}

〉〉〉 kk\_hironoのリツイート 〉〉〉

\begin{itemize}
\tightlist
\item
  RT
  kk\_hirono(刑事告発・非常上告_金沢地方検察庁御中)|hirono\_hideki(奉納\さらば弁護士鉄道・泥棒神社の物語)
  日時:2021-05-16 08:49/2021/05/16 07:54 URL:
  \url{https://twitter.com/kk\_hirono/status/1393715233273778177} 
  \url{https://twitter.com/hirono\_hideki/status/1393701306003431425} 
  \textgreater{}
  人気芸能人の資産はどれくらい?ギャラから年収に意外な副業まで - -
  Travallin \url{https://t.co/EZFufzF2Up} 
\end{itemize}

〉〉〉 kk\_hironoのリツイート 〉〉〉

\begin{itemize}
\tightlist
\item
  RT
  kk\_hirono(刑事告発・非常上告_金沢地方検察庁御中)|hirono\_hideki(奉納\さらば弁護士鉄道・泥棒神社の物語)
  日時:2021-05-16 08:49/2021/05/16 07:51 URL:
  \url{https://twitter.com/kk\_hirono/status/1393715257533636609} 
  \url{https://twitter.com/hirono\_hideki/status/1393700486902026243} 
  \textgreater{}
  人気芸能人の資産はどれくらい?ギャラから年収に意外な副業まで -
  Travallin \url{https://t.co/LMzyyLerP5} 
\end{itemize}

〉〉〉 kk\_hironoのリツイート 〉〉〉

\begin{itemize}
\tightlist
\item
  RT
  kk\_hirono(刑事告発・非常上告_金沢地方検察庁御中)|hirono\_hideki(奉納\さらば弁護士鉄道・泥棒神社の物語)
  日時:2021-05-16 08:49/2021/05/16 07:49 URL:
  \url{https://twitter.com/kk\_hirono/status/1393715273329299456} 
  \url{https://twitter.com/hirono\_hideki/status/1393700061498929154} 
  \textgreater{} 【水難事故に見せかけ妻を殺害】和歌山県白浜町
  野田孝史容疑者(29)を逮捕 : まとめダネ! \url{https://t.co/bi88bB5rTi} 
\end{itemize}

〉〉〉 kk\_hironoのリツイート 〉〉〉

\begin{itemize}
\tightlist
\item
  RT
  kk\_hirono(刑事告発・非常上告_金沢地方検察庁御中)|hirono\_hideki(奉納\さらば弁護士鉄道・泥棒神社の物語)
  日時:2021-05-16 08:49/2021/05/16 07:43 URL:
  \url{https://twitter.com/kk\_hirono/status/1393715291293552645} 
  \url{https://twitter.com/hirono\_hideki/status/1393698501075161088} 
  \textgreater{} 090225\_hanrei.pdf \url{https://t.co/s4RWb2Kr9t} 
  平成30(わ)142  殺人 令和3年3月23日  和歌山地方裁判所
\end{itemize}

〉〉〉 kk\_hironoのリツイート 〉〉〉

\begin{itemize}
\tightlist
\item
  RT
  kk\_hirono(刑事告発・非常上告_金沢地方検察庁御中)|o2441(スラ弁(弁護士大西洋一))
  日時:2021-05-16 08:50/2021/05/14 12:58 URL:
  \url{https://twitter.com/kk\_hirono/status/1393715317793116162} 
  \url{https://twitter.com/o2441/status/1393053166178955267} 
  \textgreater{} ん?どゆこと? \url{https://t.co/AZJKZSlzO4} 
\end{itemize}

〉〉〉 kk\_hironoのリツイート 〉〉〉

\begin{itemize}
\tightlist
\item
  RT
  kk\_hirono(刑事告発・非常上告_金沢地方検察庁御中)|kose\_usanee(kose)
  日時:2021-05-16 08:50/2021/04/16 06:25 URL:
  \url{https://twitter.com/kk\_hirono/status/1393715357450326018} 
  \url{https://twitter.com/kose\_usanee/status/1382807363611107329} 
  \textgreater{}
  天音総合法律なんちゃらから何回も電話かかってきたり、「減額診断が〜」とかショートメールきたけど、全く心当たり無いので検索したら同じ様な被害うけてる人けっこうおるな。
\end{itemize}

〉〉〉 kk\_hironoのリツイート 〉〉〉

\begin{itemize}
\tightlist
\item
  RT
  kk\_hirono(刑事告発・非常上告_金沢地方検察庁御中)|hirono\_hideki(奉納\さらば弁護士鉄道・泥棒神社の物語)
  日時:2021-05-16 08:50/2021/05/16 07:22 URL:
  \url{https://twitter.com/kk\_hirono/status/1393715378392469507} 
  \url{https://twitter.com/hirono\_hideki/status/1393693198552162304} 
  \textgreater{} - 法律事務所からの急な電話、ショートメールについて -
  弁護士ドットコム 犯罪・刑事事件 \url{https://t.co/fzYIBnl0Gi} 
\end{itemize}

〉〉〉 kk\_hironoのリツイート 〉〉〉

\begin{itemize}
\tightlist
\item
  RT
  kk\_hirono(刑事告発・非常上告_金沢地方検察庁御中)|hirono\_hideki(奉納\さらば弁護士鉄道・泥棒神社の物語)
  日時:2021-05-16 08:50/2021/05/16 07:21 URL:
  \url{https://twitter.com/kk\_hirono/status/1393715403889598467} 
  \url{https://twitter.com/hirono\_hideki/status/1393693037620916227} 
  \textgreater{} -
  天音総合法律事務所ってとこからメールがきててこの間お問い合わせいただい\ldots{}
  - Yahoo!知恵袋 \url{https://t.co/mD8GI754zN} 
\end{itemize}

〉〉〉 kk\_hironoのリツイート 〉〉〉

\begin{itemize}
\item
  RT
  kk\_hirono(刑事告発・非常上告_金沢地方検察庁御中)|o2441(スラ弁(弁護士大西洋一))
  日時:2021-05-16 08:50/2021/05/14 18:19 URL:
  \url{https://twitter.com/kk\_hirono/status/1393715475842928640} 
  \url{https://twitter.com/o2441/status/1393133754005131264} 
  \textgreater{}
  でも、「国破れて三部あり」の藤山裁判官、そんな変な風に飛ばされてた印象もないから、理屈がきちんとしてれば平気なんじゃないかな。あ、でも、公安事件で無罪出すと異動というのはあるような気もしなくもないかな。いずれにせよ、判決の理由次第な気がするな。
  \url{https://t.co/i0K3TQAVbx} 
\item
  〈〈〈 2021/05/16 09:43:24 Linux Emacs: 〈〈〈
\end{itemize}

\hypertarget{ux4e8bux4ef6ux5831ux9053ux306fux30deux30b9ux30b3ux30dfux306eux30a8ux30f3ux30bfux30e1ux3068ux3044ux3046ux533fux540dux5f01ux8b77ux58ebux5211ux88c1ux30b5ux30a4ux592aux548cux6b4cux5c71ux306eux4fddux967aux91d1ux76eeux7684ux6bbaux4ebaux4e8bux4ef6ux3068ux4f50ux85e4ux5927ux548cux5f01ux8b77ux58eb}{%
\paragraph{事件報道はマスコミのエンタメという匿名弁護士刑裁サイ太,和歌山の保険金目的殺人事件と佐藤大和弁護士}\label{ux4e8bux4ef6ux5831ux9053ux306fux30deux30b9ux30b3ux30dfux306eux30a8ux30f3ux30bfux30e1ux3068ux3044ux3046ux533fux540dux5f01ux8b77ux58ebux5211ux88c1ux30b5ux30a4ux592aux548cux6b4cux5c71ux306eux4fddux967aux91d1ux76eeux7684ux6bbaux4ebaux4e8bux4ef6ux3068ux4f50ux85e4ux5927ux548cux5f01ux8b77ux58eb}}

\begin{itemize}
\tightlist
\item
  〉〉〉 Linux Emacs: 2021/05/16 09:46:44 〉〉〉
\end{itemize}

:CATEGORIES: @kanazawabengosi \#金沢弁護士会 @JFBAsns
日本弁護士連合会(日弁連) \#法務省 @MOJ\_HOUMU \#刑裁サイ太
\#佐藤大和弁護士 \#被告発人長谷川紘之弁護士

\begin{itemize}
\tightlist
\item
  1371:2021-05-16\_09:44:05 \#告発状 \#\#\#\#
  刑裁サイ太のタイムラインがきっかけで,保険金殺人の懲役19年の判決文,舞鶴女子高校生殺害事件の被害者の細川治弁護士のブログ発見まで
  \url{https://hirono-hideki.hatenadiary.jp/entry/2021/05/16/094402} 
\end{itemize}

 刑裁サイ太のタイムラインで裁判所のサイトの判例へのリンクがあり,URLでPDFファイルの指定部分を削除すると次のページが表示されました。トップページになると思います。

\begin{itemize}
\tightlist
\item
  裁判所 - Courts in Japan \url{https://www.courts.go.jp/index.html} 
\end{itemize}

 ページの下の方に,下級裁判所判例速報というリンクがあって,そこからみつけたのが和歌山の保険金殺人の事件でした。懲役19年となっていましたが,しばらく前,ネットのニュースで見かけて判決のことは知っていました。少し調べてもいたと思います。

 そういえば比較的最近,刑裁サイ太がマスコミを皮肉るのに娯楽としたツイートをしていたように思い出したのですが,娯楽ではそれらしいツイートが見つからず,ku3のまとめ記事を作成して,それらしい時期のツイートを眺めていったところ,「エンタメ」になっていたことを発見しました。

 エンタメが私の脳内で娯楽に置き換わって記憶されていたようです。娯楽というよりはエンタメの方が,私が和歌山の事件に抱いていたイメージとマッチするのですが,和歌山の保険金目的殺人事件はもう1つ別にあり,これもしばらく前,被疑者の女性が逮捕され大きなニュースになっていました。

 ブログ記事としての記録は次の流れとなっています。

\begin{itemize}
\tightlist
\item
  2021年05月16日09時06分の登録:
  REGEXP:''娯楽''/サイ太(@uwaaaa)の検索(2016-03-30〜2019-04-11/2021年05月16日09時06分の記録6件)
  \url{https://kk2020-09.blogspot.com/2021/05/regexpuwaaaa2016-03-302019-04.html} 
\item
  2021年05月16日09時11分の登録:
  「娯楽」を@hirono\_hideki @kk\_hirono @s\_hironoで検索 85件の該当 2021-05-16\_09:11の記録
  \url{https://kk2020-09.blogspot.com/2021/05/hironohidekikkhironoshirono852021-05\_16.html} 
\item
  2021年05月16日09時14分の登録:
  「サイ太」を@hirono\_hideki @kk\_hirono @s\_hironoで検索 4754件の該当 2021-05-16\_09:13の記録
  \url{https://kk2020-09.blogspot.com/2021/05/hironohidekikkhironoshirono47542021-05.html} 
\item
  2021年05月16日09時28分の登録:
  \ystk @lawkus\橋本崇載氏、YouTubeでは精神的に参ってたとか請求書が付いてなかったからとか婚姻費用不払に悪気はなかったような言い訳してたけど、普通に兵糧
  \url{https://kk2020-09.blogspot.com/2021/05/ystklawkusyoutube.html} 
\item
  2021年05月16日09時28分の登録:
  @k\_sawmen(泥濘ノ魔王獣マガサイケ)のツイート ''.*'' 3223/3223:2021-01-26\_1606〜2021-05-16\_0921 2021年05月16日09時27分の記録
  \url{https://kk2020-09.blogspot.com/2021/05/ksawmen322332232021-01-2616062021-05.html} 
\item
  2021年05月16日09時30分の登録:
  \リーチ一発ツモ裏1 @luckymangan\子供の学費について、泡姫やAV嬢になって作れ=養育費支払わないことを肯定するのも、子供のことを思っての言動なんでしょうか・・・
  \url{https://kk2020-09.blogspot.com/2021/05/luckymanganav.html} 
\item
  2021年05月16日09時31分の登録:
  \豚野郎 @butayar0\これで笑える橋本八段邪悪だわ。相手方困窮するとしてそのことがそんなにうれしいの?なにが共同親権だよ。
  \url{https://kk2020-09.blogspot.com/2021/05/butayar0\_16.html} 
\item
  2021年05月16日09時33分の登録:
  \kozakana-sakanako @KSakanako\「弁護士を排除したい」かつ「シェルターを敵視する」って、もうそれはDV加害者の考えですよ。
  \url{https://kk2020-09.blogspot.com/2021/05/kozakana-sakanakoksakanakodv.html} 
\item
  2021年05月16日09時37分の登録:
  \佐藤大和/レイ法律事務所代表弁護士 @yamato\_lawyer\「守秘義務があるため回答できない」とすると不満そうにし、他に事実確認もせず一方的に依頼者に不利な記事が
  \url{https://kk2020-09.blogspot.com/2021/05/yamatolawyer.html} 
\item
  2021年05月16日09時39分の登録:
  REGEXP:''エンタメ''/サイ太(@uwaaaa)の検索(2013-11-14〜2021-04-30/2021年05月16日09時39分の記録13件)
  \url{https://kk2020-09.blogspot.com/2021/05/regexpuwaaaa2013-11-142021-04.html} 
\item
  2021年05月16日09時40分の登録:
  REGEXP:''神話裁判''/データベース登録済みツイートの検索:2021-05-11〜2021-05-14/2021年05月16日09時40分の記録:ユーザ・投稿:2/2件
  \url{https://kk2020-09.blogspot.com/2021/05/regexp2021-05-112021-05-1420210516094022.html} 
\end{itemize}

 刑裁サイ太のツイートを調べていると関連した法クラのツイートが出てきて,けっこう久しぶりに見る橋本崇載氏の話題が出てきましたが,最初の方にYouTube動画を見ていて,とてもエンタメ要素が強い問題になっていると感じていました。将棋の八段ということもそれまではしらなかったです。

 いくつかスクリーンショットの記録も行っているので,そちらもご紹介しておきます。私自身の記憶の確認と整理にもなります。

\begin{itemize}
\item
  TW s\_hirono(非常上告-最高検察庁御中\_ツイッター) 日時: 2021-05-16
  08:45 URL: \url{https://twitter.com/s\_hirono/status/1393714308291305472} 
  \textgreater{}
  2021-05-16-073900\_不更に被害者を被保険者とする別の死亡保険契約を締結したほか,本件前日夕方に「保険金(災害死亡保険金)が支払われない」旨のタイトルのインターネ.jpg
  \url{https://t.co/3igHa2Pe7W} 
\item
  TW s\_hirono(非常上告-最高検察庁御中\_ツイッター) 日時: 2021-05-16
  09:07 URL: \url{https://twitter.com/s\_hirono/status/1393719629772517377} 
  \textgreater{}
  2021-05-16-090519\_''娯楽'' (from:uwaaaa) - Twitter検索 / Twitter.jpg
  \url{https://t.co/x0m5vISC0V} 
\item
  TW s\_hirono(非常上告-最高検察庁御中\_ツイッター) 日時: 2021-05-16
  09:07 URL: \url{https://twitter.com/s\_hirono/status/1393719702422048769} 
  \textgreater{}
  2021-05-16-090526\_''娯楽'' (from:uwaaaa) - Twitter検索 / Twitter.jpg
  \url{https://t.co/9y00acQfZh} 
\item
  TW s\_hirono(非常上告-最高検察庁御中\_ツイッター) 日時: 2021-05-16
  09:32 URL: \url{https://twitter.com/s\_hirono/status/1393726027336015873} 
  \textgreater{}
  2021-05-16-092505\_サイ太@uwaaaa·2018年11月26日刑事手続についての国民の誤解には,エンタメの影響力も否定できないところだと僕は思います。そういう.jpg
  \url{https://t.co/TkzGkTlSuv} 
\item
  TW s\_hirono(非常上告-最高検察庁御中\_ツイッター) 日時: 2021-05-16
  09:32 URL: \url{https://twitter.com/s\_hirono/status/1393726101118013440} 
  \textgreater{}
  2021-05-16-092901\_泥濘ノ魔王獣マガサイケさんがリツイートystk@lawkus·5時間橋本崇載氏、YouTubeでは精神的に参ってたとか請求書が付いてなかっ.jpg
  \url{https://t.co/8knhNt3wcL} 
\item
  TW s\_hirono(非常上告-最高検察庁御中\_ツイッター) 日時: 2021-05-16
  09:33 URL: \url{https://twitter.com/s\_hirono/status/1393726173541068803} 
  \textgreater{}
  2021-05-16-093035\_泥濘ノ魔王獣マガサイケさんがリツイートリーチ一発ツモ裏1@luckymangan·10時間橋本元プロは子供のことを思っていると評価していた.jpg
  \url{https://t.co/BQ8I4mOQUB} 
\item
  TW s\_hirono(非常上告-最高検察庁御中\_ツイッター) 日時: 2021-05-16
  09:33 URL: \url{https://twitter.com/s\_hirono/status/1393726246513647623} 
  \textgreater{}
  2021-05-16-093135\_泥濘ノ魔王獣マガサイケさんがリツイート豚野郎@butayar0·10時間これで笑える橋本八段邪悪だわ。相手方困窮するとしてそのことがそんな.jpg
  \url{https://t.co/rAyRdZE8Nz} 
\item
  TW s\_hirono(非常上告-最高検察庁御中\_ツイッター) 日時: 2021-05-16
  09:40 URL: \url{https://twitter.com/s\_hirono/status/1393728118246952962} 
  \textgreater{}
  2021-05-16-093409\_泥濘ノ魔王獣マガサイケさんがリツイートkozakana-sakanako@KSakanako·20時間「弁護士を排除したい」かつ「シェルタ.jpg
  \url{https://t.co/asAOsUMBmG} 
\item
  TW s\_hirono(非常上告-最高検察庁御中\_ツイッター) 日時: 2021-05-16
  09:41 URL: \url{https://twitter.com/s\_hirono/status/1393728190858678274} 
  \textgreater{}
  2021-05-16-093535\_佐藤大和/レイ法律事務所代表弁護士@yamato\_lawyer毎日、メディアでは、全国各地の逮捕ニュース等を報道しているが、世界の刑事裁判制.jpg
  \url{https://t.co/1x9WWh47y9} 
\item
  TW s\_hirono(非常上告-最高検察庁御中\_ツイッター) 日時: 2021-05-16
  09:41 URL: \url{https://twitter.com/s\_hirono/status/1393728263789236232} 
  \textgreater{}
  2021-05-16-093744\_佐藤大和/レイ法律事務所代表弁護士@yamato\_lawyer·5月13日報道機関からの質問(取材)に対して「どこからの情報ですか」と確認す.jpg
  \url{https://t.co/bzAV54G9wF} 
\item
  TW s\_hirono(非常上告-最高検察庁御中\_ツイッター) 日時: 2021-05-16
  09:41 URL: \url{https://twitter.com/s\_hirono/status/1393728336354889730} 
  \textgreater{}
  2021-05-16-093840\_佐藤大和/レイ法律事務所代表弁護士さんがリツイート小木とまい『神話裁判』1巻5/14発売@ogitomai·5月12日ご紹介有難うございま.jpg
  \url{https://t.co/zDj7ZiOLg9} 
\item
  神話裁判 - 小木とまい / 第1話 \textbar{} MAGCOMI
  \url{https://t.co/6F194q0p7u}  ストーカーに誘拐犯、神話の神々は犯罪者だらけ!?
  敏腕弁護士は法で彼らを守れるのか? 怒涛の法廷コメディ開廷!
\end{itemize}

 ツイートにあった画像にも「ストーカーに誘拐犯、神話の神々は犯罪者だらけ!?
敏腕弁護士は法で彼らを守れるのか?」などとありましたが,これがいわゆる紀州のドンファン事件での弁護士の活躍とオーバーラップしました。

 今確認すると,強く印象に残っていた刑裁サイ太のツイートは4月30日となっていました。4月30日にどのような報道があったのか憶えていないですが,これも紀州のドンファン事件と関連がありそうには目星をつけています。

\begin{itemize}
\tightlist
\item
  (13/13) TW uwaaaa(サイ太) 日時:2021-04-30 17:48:28 +0900 URL:
  \url{https://twitter.com/uwaaaa/status/1388052621512953856\textgreater} {}
  逮捕は刑罰じゃないし,事件報道はエンタメじゃない。
\end{itemize}

 次が4月の終わりから5月の初めの刑裁サイ太のツイートの記録です。

\begin{itemize}
\tightlist
\item
  2021年04月25日14時07分の登録:
  \サイ太 @uwaaaa\灯火規制で治安が悪化した盛り場に足を踏み入れた女性が被害にあった場合,誰が一番悪いとジャッジするんでしょうかね
  \url{https://kk2020-09.blogspot.com/2021/04/uwaaaa\_25.html} 
\item
  2021年04月28日22時47分の登録:
  \サイ太 @uwaaaa\ずるい!ぼくも所属タレントさんとお近づきになりたい!
  \url{https://kk2020-09.blogspot.com/2021/04/uwaaaa\_28.html} 
\item
  2021年04月28日22時48分の登録:
  \サイ太 @uwaaaa\あれ,出入りする弁護士全員に声かけてるみたいですね。ああいう根性は見習いたい。
  \url{https://kk2020-09.blogspot.com/2021/04/uwaaaa\_63.html} 
\item
  2021年04月28日22時49分の登録:
  \サイ太 @uwaaaa\またそっち系の人から「弁護士を訴えるビジネス」とか言われちゃう
  \url{https://kk2020-09.blogspot.com/2021/04/uwaaaa\_6.html} 
\item
  2021年04月29日08時15分の登録:
  \サイ太 @uwaaaa\なぜ俺は毎度毎度大型連休前に緊急対応が必要な案件をツモってしまうのか
  \url{https://kk2020-09.blogspot.com/2021/04/uwaaaa\_29.html} 
\item
  2021年04月30日14時01分の登録:
  \サイ太 @uwaaaa\また弁護士名簿を定期的のDLしておく癖がネトストに役立ってしまった
  \url{https://kk2020-09.blogspot.com/2021/04/uwaaaadl.html} 
\item
  2021年04月30日15時04分の登録:
  \サイ太 @uwaaaa\「最新の季刊刑事弁護で、被疑者国選で釈放された後でも報告期限の14日以内に示談したら国選報酬加算される」という話がありましたが,当職も類例を経験
  \url{https://kk2020-09.blogspot.com/2021/04/uwaaaa14.html} 
\item
  2021年05月01日09時44分の登録:
  \サイ太 @uwaaaa\逮捕は刑罰じゃないし,事件報道はエンタメじゃない。
  \url{https://kk2020-09.blogspot.com/2021/05/uwaaaa.html} 
\item
  2021年05月03日08時48分の登録:
  \サイ太 @uwaaaa\ああいう人が弁護士登録して、誰が依頼するんだろうというのはある
  \url{https://kk2020-09.blogspot.com/2021/05/uwaaaa\_3.html} 
\item
  2021年05月03日22時46分の登録:
  \サイ太 @uwaaaa\ああいう人が弁護士登録して、誰が依頼するんだろうというのはある
  \url{https://kk2020-09.blogspot.com/2021/05/uwaaaa\_7.html} 
\item
  2021年05月05日21時22分の登録:
  \サイ太 @uwaaaa\ということでyoutuber弁護士をまとめてるんですけど,頭がおかしくなりそうになった
  \url{https://kk2020-09.blogspot.com/2021/05/uwaaaayoutuber.html} 
\item
  2021年05月05日21時22分の登録:
  \サイ太 @uwaaaa\ゴ3ネタブログを更新しました。今回は弁護士youtuberをまとめました。弁護士youtuberをまとめてみた結果wwwwwwww
  \url{https://kk2020-09.blogspot.com/2021/05/uwaaaayoutuberyoutuber.html} 
\end{itemize}

 「ああいう人が弁護士登録して、誰が依頼するんだろうというのはある」で名指しされているのは,大坪弘道元大阪地検特捜部長のことだと思います。他にも似たような酷評のツイートが法クラには散見されました。エンタメ要素の強い皮肉り方で,いわぽんこと岩田圭只弁護士のツイートも。

\begin{itemize}
\tightlist
\item
  2021年05月03日18時42分の登録:
  \いわぽん @yiwapon\登録請求を繰り返していただけでも呆れてしまうが、法曹界から永久追放しないとダメな性質の事案なんじゃないですかね。
  \url{https://kk2020-09.blogspot.com/2021/05/yiwapon.html} 
\item
  2021年05月03日21時47分の登録:
  %@yiwapon いわぽん%捜査情報のリークをこれだけ野放しにしておいたままなのに裁判に市民感覚を反映しましょうとかいうのはかなり理解しがたい。
  \url{https://kk2020-09.blogspot.com/2021/05/yiwapon\_3.html} 
\end{itemize}

 一部の弁護士とジャーナリストによって,エンタメのショービジネスの如く大悪人に祭り上げられたのが,前田恒彦元特捜部主任検事や大坪弘道元大阪地検特捜部長という私の見立てなのですが,あれよあれよと実刑判決や有罪判決が出て,弁護士鉄道の歴史に不朽の足跡を残しました。

\begin{itemize}
\tightlist
\item
  ./s\_hirono2021-05-16\_094243.csv:2017-04-30 21:54:20
  ``2017年04月30日18時12分26秒/記録資料/深澤諭史さんがリツイート> Kay> 依頼者の利益のためであれば弁護士がどれだけ相手方に対して残酷になれるか知らない人は多い。
  \url{https://www.youtube.com/watch?v=qi1BTx2zpQo''} 
  \url{https://twitter.com/s\_hirono/status/858665825296728064} 
\end{itemize}

 twilog-serch
04月30日,という検索で出てきたのですが,見覚えのない記録だと思ったところ2017年の4月30日でした。先月の4月30日という思い込みがあったのですが,そういえば宇出津新港の職業安定所に行った日だったことを思い出しました。

 「依頼者の利益のためであれば弁護士がどれだけ相手方に対して残酷になれるか知らない人は多い。」という内容のツイートは深澤諭史弁護士のリツイートとして記録したものですが,依頼者の利益のためならば責任を問われることがないという信仰が一部の弁護士にはあるようです。

 それもこれも弁護士のエンタメ要素に関連がありそうです。もはや弁護士という職業自体が,エンタメつまりエンターテインメントあるいはエンターテイナーというマジック,パフォーマンス要素が満載という感じです。目的の判決が確定すれば完成品,弁護士神話の殿堂入りです。

 すべての刑事事件がそうではないですが,無実であっても大きな物語を経て,弁護士の手柄と警察,検察の落ち度や悪意が強調されるのが常態化しているパターンかと思います。無実や不当判決であって弁護士に無視され,放置されている事件の方が,星の数ほど無数にあるかもしれません。

 弁護士鉄道の夜,弁護士鉄道の星空のようなものです。

\begin{itemize}
\item
  ./hirono\_hideki2021-05-16\_100902.csv:2021-05-01 03:44:29
  ``2021-04-30\_20:05
  奉納\\#危険生物・弁護士脳汚染除去装置\\#金沢地方検察庁御中\_2020:
  REGEXP:''京アニ放火殺人事件容疑者に主治医''/データベース登録済みツイートの検索:2021-04-29〜2021-04-30/2021年04月30日20時04分の記録:ユーザ・投稿:16/22件
  \url{https://kk2020-09.blogspot.com/2021/04/regexp2021-04-292021-04.html}  "
  \url{https://twitter.com/hirono\_hideki/status/1388202613514989569} 
\item
  (01/22) TW news\_type\_c(NEWS JAPAN R) 日時: 2021-04-29 15:42:07
  +0900 URL:
  \url{https://twitter.com/news\_type\_c/status/1387658435458748416\textgreater} {}
  京アニ放火殺人事件容疑者に主治医・上田敬博が伝えたこと「俺はおまえに向き合う。絶対に逃げるな」
  - 社会 - ニュース|週プレNEWS[週刊プレイボーイのニュースサイト]
  {[}35コメント{]}\textgreater{}
  『週プレNEWS』は、集英社「週刊プレイ・・・ \url{https://t.co/EcShYZfDTv} 
\end{itemize}

 たぶんTwitterのトレンドで見かけて,読んだ記事ですが最初のニュースサイトのツイートが4月29日15時42分と記録されていました。

\begin{itemize}
\tightlist
\item
  (04/22) TW lawkus(ystk) 日時: 2021-04-29 18:02:39 +0900 URL:
  \url{https://twitter.com/lawkus/status/1387693804824629254\textgreater} {}
  京アニ放火殺人事件容疑者に主治医・上田敬博が伝えたこと「俺はおまえに向き合う。絶対に逃げるな」(週プレNEWS)\textgreater{}
  \#Yahooニュース\textgreater{} \url{https://t.co/d6PTaitXN3\textgreater} {}
  救命した仕事そのものは掛け値なしに立派だが、患・・・
  \url{https://t.co/eHNRjMWWlO} 
\end{itemize}

 気がついたのですが,「救命した仕事そのものは掛け値なしに立派だが、患・・・」となっています。ツイートの全文が収録されていないですが,textという変数の値のようです。full\_textに修正してあるものと思っていたのですが,なっていないようです。

\begin{itemize}
\tightlist
\item
  TW lawkus(ystk) 日時: 2021/04/29 18:02:39 URL:
  \url{https://twitter.com/lawkus/status/1387693804824629254} 
  \textgreater{}
  京アニ放火殺人事件容疑者に主治医・上田敬博が伝えたこと「俺はおまえに向き合う。絶対に逃げるな」(週プレNEWS)\\
  \textgreater{} \#Yahooニュース\\
  \textgreater{} \url{https://t.co/d6PTaitXN3} 
  \textgreater{}
  救命した仕事そのものは掛け値なしに立派だが、患者との会話の内容をマスコミにべらべら喋ってしまうのはまずいでしょ。
\end{itemize}

 時刻は11時49分です。今まで気が付かなかったのもどうかと思うのですが,スクリプトの修正をしました。ツイートを取得するメソッドの引数に:tweet\_mode
=\textgreater{}``extended'',を追加し,取得する変数をfull\_textに変更するだけです。

 「患者との会話の内容をマスコミにべらべら喋ってしまうのはまずいでしょ。」という三浦義隆弁護士のツイートの,「べらべら」がきになったのですが,「ぺらぺら」だと思ったのが「べらべら」でした。弁護士魂を感じる他者攻撃性です。

 単なる批判ではなく,依頼者の客層及び資質を絞り込み,経営上の業務の効率性を図っていると思われるのも三浦義隆弁護士らしい特徴で,深澤諭史弁護士にも似た傾向がありました。

※ @kk\_hironoのアカウントがブロックされ,リツイートに失敗したツイート

\begin{itemize}
\tightlist
\item
  TW lawkus(ystk) 日時:2021/05/16 08:25:59 URL:
  \url{https://twitter.com/lawkus/status/1393709275365593089} 
  \textgreater{}
  モラハラ=弁護士の洗脳説を唱える方々がいるようだが、離婚相談でモラハラを主張する相談者の何割かは証拠的に無理そうな事案なので、そういう場合はむしろモラハラを争点にしない方向で俺は話してるし、そういう弁護士は多いと思う。それでも離婚の意思が固いなら離婚調停の依頼自体は受けるけどね。
\end{itemize}

 顧客の獲得を第一にTwitterをやっていると感じる三浦義隆弁護士ですが,とりわけ徹底していると感じるところがあります。今,タイムラインで目についたツイートを1つリツイートをして失敗しましたが,これも他の弁護士とは違うスタンスを感じました。

 奉納\さらば弁護士鉄道・泥棒神社の物語(@hirono\_hideki)ではブロックされていない三浦義隆弁護士のTwitterアカウントですが,ブロックしている事自体,認識も自覚もないのかもしれません。私は弁護士業界さらには司法制度全般に影響の及ぶ問題として手続きを進め記録をしています。

\begin{itemize}
\tightlist
\item
  TW lawkus(ystk) 日時: 2021/05/16 04:21:10 URL:
  \url{https://twitter.com/lawkus/status/1393647662612709378} 
  \textgreater{}
  橋本崇載氏、YouTubeでは精神的に参ってたとか請求書が付いてなかったからとか婚姻費用不払に悪気はなかったような言い訳してたけど、普通に兵糧攻めしてる自覚あるのね。
  \url{https://t.co/EiW4p54tcv} 
\end{itemize}

 先程も少し見かけていた上記の三浦義隆弁護士のツイートですが,ずっと長い間忘れていた古い曲の歌詞が頭に浮かんで来ました。その思い出した歌詞の一部で検索したところ,曲名は全くみおぼえのない曲が見つかりました。感じたテーマ性は弁護士と人との間の溝あるいは闇です。

〉〉〉 kk\_hironoのリツイート 〉〉〉

\begin{itemize}
\tightlist
\item
  RT
  kk\_hirono(刑事告発・非常上告_金沢地方検察庁御中)|hirono\_hideki(奉納\さらば弁護士鉄道・泥棒神社の物語)
  日時:2021-05-16 12:13/2021/05/16 12:13 URL:
  \url{https://twitter.com/kk\_hirono/status/1393766642677551110} 
  \url{https://twitter.com/hirono\_hideki/status/1393766440990232582} 
  \textgreater{} 野坂昭如 黒の舟唄 歌詞 - 歌ネット
  \url{https://t.co/9CEpFeiwpt} 
\end{itemize}

〉〉〉 kk\_hironoのリツイート 〉〉〉

\begin{itemize}
\tightlist
\item
  RT
  kk\_hirono(刑事告発・非常上告_金沢地方検察庁御中)|hirono\_hideki(奉納\さらば弁護士鉄道・泥棒神社の物語)
  日時:2021-05-16 12:13/2021/05/16 12:07 URL:
  \url{https://twitter.com/kk\_hirono/status/1393766651493974016} 
  \url{https://twitter.com/hirono\_hideki/status/1393765102185156610} 
  \textgreater{} 野坂昭如「黒の舟唄」(1974年ライヴ) - YouTube
  \url{https://t.co/hczmuMZBff} 
\end{itemize}

 この野坂昭如という著名人は,拘置所にいる時に読んでいた週刊誌の連載コラムで名前を知り記憶にあったのですが,後になってネットでいくつか情報を見るようになりました。どちらもずいぶん個性の強そうな人物ですが,大島渚監督との殴り合いというのもあったと思います。

\begin{itemize}
\item
  大島渚 vs 野坂昭如 殴り合い【放送事故】 - ニコニコ動画
  \url{https://t.co/j1zEbrNtLz} 
\item
  【復刻】大島渚監督「祝う会」で野坂昭如氏と乱闘 - おくやみ :
  日刊スポーツ \url{https://t.co/4yqkJ3JUcC}  {[}2015年12月10日10時49分{]} ¥\n
  【1990年10月24日付紙面から】
\end{itemize}

 2015年の記事となっていましたが,もっと前の出来事と思っていたところ最後に平成2年10月24日とありました。大島渚監督の訃報というのもずいぶん前にあったように思います。

\begin{quote}
《引用の始まり》
\end{quote}

\begin{quote}
生い立ち[編集]

1945年の神戸大空襲で養父を、下の妹を疎開先の福井県で栄養失調で亡くす。父は土木技師で戦後に新潟県副知事を務めた野坂相如(すけゆき)[4]。当時野坂家の住いは東京市麹町区隼町だったが、産み月近くなって両親が別居。昭如は神奈川県鎌倉市小町で誕生した。実母ぬいは自身を産んだ2月後に死別。生後半年で神戸の張満谷(はりまや)家へ養子に出される。

11歳の時、戸籍謄本を偶然に見て、自分が養子であることを知り、後には妹2人も別々に養子として入る。

その後、上の妹を病気で、1945年の神戸大空襲で養父を、下の妹を疎開先の福井県春江町(坂井市)で栄養失調で亡くした。後に福井県で妹を亡くした経験から贖罪のつもりで『火垂るの墓』を記した。終戦時から大阪府守口市などを2年間転々とする。

なお、『火垂るの墓』の後、「空襲で父母をなくした」と長らく詐称していたが、養父は実際に空襲で行方不明となっていたが、養母は大怪我をしながら生きており、元から一緒に暮らしていた養祖母も健在だった。(1973年発表の「アドリブ自叙伝」で告白)。
\end{quote}

\begin{quote}
《引用の終わり》
\end{quote}

\begin{itemize}
\tightlist
\item
  野坂昭如 -
  Wikipedia \url{https://ja.wikipedia.org/wiki/\%E9\%87\%8E\%E5\%9D\%82\%E6\%98\%AD\%E5\%A6\%82n} 
\end{itemize}

 『火垂るの墓』のことは記憶にあったのですが,確信が持てず,関係性もよくわかっていませんでした。映画はたしかテレビで視聴しています。2,3年ほど前でしょうか。

\begin{itemize}
\tightlist
\item
  ./s\_hirono2021-05-16\_094243.csv:2018-04-24 12:20:54
  ``2018-04-13\_224140_テレビの画面・火垂るの墓.jpg
  \url{http://pic.twitter.com/iDj8HT5T9D''} 
  \url{https://twitter.com/s\_hirono/status/988618757852426240} 
\end{itemize}

〉〉〉 kk\_hironoのリツイート 〉〉〉

\begin{itemize}
\item
  RT
  kk\_hirono(刑事告発・非常上告_金沢地方検察庁御中)|nabeteru1Q78(渡辺輝人)
  日時:2021-05-16 12:33/2021/05/16 12:27 URL:
  \url{https://twitter.com/kk\_hirono/status/1393771583047622659} 
  \url{https://twitter.com/nabeteru1Q78/status/1393769980227768321} 
  \textgreater{}
  元気になったのなら、もりかけ桜、財務省の資料隠滅に、自分自身の汚職。表舞台で早く説明すべきだろう。
  / 2件のコメント \url{https://t.co/qz3aioSJYs} 
  ``安倍前首相に復権の兆し 再々登板に渦巻く警戒と熱視線:朝日新聞デジタル''
  \url{https://t.co/DANOSKwR5K} 
\item
  おもちゃのチャチャチャ - Wikipedia \url{https://t.co/axVN88F6cq} 
  「おもちゃのチャチャチャ」は、野坂昭如作詞、吉岡治補作詞、越部信義作曲の日本の童謡である。曲名は、「おもちゃ」と「チャチャチャ」をかけている{[}1{]}。
\end{itemize}

 家にレコードのようなものはなかったですが,昭和40年代によく耳にした曲として記憶にあります。その「おもちゃのチャチャチャ」が野坂昭如作詞とは知らず意外でした。昭和5年生まれと確認しています。

\begin{itemize}
\tightlist
\item
  Japanese Children's Song - 童謡 - 3D Omocha no ChaChaCha - 3D
  おもちゃのチャチャチャ - YouTube \url{https://t.co/dYqVO7aP7c}  ¥\n
  19,623,786 回視聴•2018/06/03
\end{itemize}

 再生回数が2千万回近くになっていますが,ずいぶん久しぶりに聴いた曲です。アニメの3D映像というのも余り見た覚えがなく珍しく思いました。

\begin{itemize}
\tightlist
\item
  野坂昭如  ソクラテス サントリーゴールド900 1976年 - YouTube
  \url{https://t.co/cCw2pIyGQp} 
\end{itemize}

 何年か前にソクラテスで始まるCMが野坂昭如氏の出演だったと知ったのですが,子供の頃,かなり長い間,それこそ毎日のようにテレビに流れていたと記憶にあります。上記のYouTube動画には1976年とありますが,昭和46年から始まったCMだったのでしょう。

 野坂昭如氏はシャンソン歌手ともありますが,昨日あたり,思い出していた曲があって,シャンソンなのかわからないですが,昭和40年代のフランスの曲の雰囲気で,日本人が作曲したとあったと思います。

 昨日の午後は,しばらく清水潔氏の桶川ストーカー殺人事件の本を読んでいたのですが,事件の発生が平成11年の秋で,ちょうどその頃に金沢刑務所の拘置所でよく流れていた曲でした。ネットで探せば出てきますが,他では耳にしたことのない曲です。

\begin{itemize}
\tightlist
\item
  新谷のり子/フランシーヌの場合  (1969年) - YouTube
  \url{https://t.co/VjuKjbbE7b} 
\end{itemize}

 「hatena-log-search
フランシーヌ」の結果はなしでした。平成11年からより記憶の新しい時期ですが,羽咋市に住んでいた頃,はてなダイアリーに記述することはなかったようです。ときどき思い出していたと思いますが,曲を調べることもなかったのかもしれません。

\begin{itemize}
\tightlist
\item
  ./kk\_hirono2021-05-16\_123304.csv:2018-12-18 23:48:22 ``»
  フランシーヌの場合 / 新谷のり子 - YouTube
  \url{https://www.youtube.com/watch?v=SFhTkurFoDQ''} 
  \url{https://twitter.com/kk\_hirono/status/1075040079549218816} 
\end{itemize}

 ちょっと納得がいかないのですが,2018年12月18日が「フランシーヌ」の最初のツイートとなっているようです。もっと前からYouTubeで曲も聴いていたと思うのですが,他に記録は残っていないようです。

 昨日は夜になって大きな発見があったのですが,それも報道について深く考えさせられる今までにない情報でした。文春オンラインの記事になっていたと思います。意外に反応の乏しい記事だったかもしれません。

\begin{itemize}
\item
  2021年05月15日22時29分の登録:
  REGEXP:''文春''/データベース登録済みツイートの検索:2021-05-13〜2021-05-15/2021年05月15日22時28分の記録:ユーザ・投稿:22/64件
  \url{https://kk2020-09.blogspot.com/2021/05/regexp2021-05-132021-05\_15.html} 
\item
  (56/64) TW hirono\_hideki(奉納\さらば弁護士鉄道・泥棒神社の物語)
  日時: 2021-05-15 20:06:46 +0900 URL:
  \url{https://twitter.com/hirono\_hideki/status/1393523245228257281\textgreater} {}
  -
  一家4人を博多港に沈める凶行 犯人の実家を中国河南省に訪ねると「こんな小さな子まで・・・」
  \textbar{} 未解決事件を追う \textbar{} 文春オンライン
  \url{https://t.co/4gLQfEFfQy} 
\end{itemize}

 キーワードが文春だけだと,文春文庫や文春新書が含まれてしまうと気がついたのですが,やはり私のツイート以外に,その記事に関するツイートはありませんでした。

 今,twitterAPI-search-lawList-mydql-add.rb
を「文春オンライン」でやり直したところですが,やはり8500のリミットに達していました。

\begin{itemize}
\tightlist
\item
  2021年05月16日09時44分の登録:
  REGEXP:''佐藤大和''/データベース登録済みツイートの検索:2013-01-17〜2021-05-16/2021年05月16日09時42分の記録:ユーザ・投稿:42/169件
  \url{https://kk2020-09.blogspot.com/2021/05/regexp2013-01-172021-05.html} 
\end{itemize}

 昨夜作成したまとめ記事と思っていたのですが,今日の9時42分の記録とあります。まだ開いていないと思いますが,和歌山でページ内検索をしていきます。

\begin{itemize}
\tightlist
\item
  (153/169) RT
  hirono\_hideki(奉納\さらば弁護士鉄道・泥棒神社の物語)|JapanPinapple(JAPAN-PINAPPLE?)
  日時:2021-04-28 23:04:13 +0900/2021-04-28 12:13:00 +0900 URL:
  \url{https://twitter.com/hirono\_hideki/status/1387407306493554697} 
  \url{https://twitter.com/JapanPinapple/status/1387243540573425670\textgreater} {}
  当時から元妻か家政婦が怪しいと言われていました\textgreater\textgreater{}
  とうとう元妻が逮捕されたのですね\textgreater{}
  迷宮入りかと思っていました\textgreater{}
  和歌山県警も意地を見せたという事ですね\textgreater\textgreater{}
  やはり今回も佐藤大和弁護士が\textgreater{}
  代理人を引き受けるのでしょうか\textgreater\textgreater{}
  元妻を殺人容・・・ \url{https://t.co/Ai1wPHu6Tl} 
\end{itemize}

 埋め込みツイートが表示されませんが,和歌山で該当したのは上記の1件だけで,それも私のリツイートでした。元のJapanPinappleのツイートが4月28日12時13分となっています。

 ニュースサイトのツイートかと思ったのですが,内容を読むとそうではなさそうです。記事へのリンクがあるようなのでこれを開いてみます。短縮URLなのでさっぱりわからないですが,Yahooニュースだとリンク切れの可能性が高いです。昨日今日とけっこう出くわしています。

\begin{itemize}
\tightlist
\item
  殺人容疑で元妻逮捕 「紀州のドン・ファン」死亡―体内から覚せい剤・和歌山県警:時事ドットコム
  \url{https://t.co/RmYmh9Z2Fh}  2021年04月28日13時40分
\end{itemize}

 4月28日の13時40分という記事ですが,28日に逮捕とあるので,逮捕は午前中だったのかもしれません。最初にニュースを知った時のことを憶えていませんが,今頃になって逮捕というのは意外でしたし,同じ感想の声が多くありました。

 紀州の事件の家政婦について調べたのですが,やはり嫌疑はあるものの逮捕などはされていないようです。4月28日の逮捕であれば,近日中には起訴あるいは不起訴が決まりそうです。起訴されたという情報も見かけませんでした。

\begin{itemize}
\tightlist
\item
  〈〈〈 2021/05/16 15:23:00 Linux Emacs: 〈〈〈
\end{itemize}

\hypertarget{ux821eux9db4ux4e8bux4ef6ux306eux4eacux7530ux8fbaux5e02ux306eux7d30ux5dddux6cbbux5f01ux8b77ux58ebux88abux5bb3ux8005ux5b89ux85e4ux6587ux3055ux3093ux5bb6ux65cfux3068ux88abux544aux767aux4ebaux9577ux8c37ux5dddux7d18ux4e4bux5f01ux8b77ux58ebux306eux95a2ux4fc2ux6027}{%
\paragraph{舞鶴事件の京田辺市の細川治弁護士,被害者安藤文さん家族と被告発人長谷川紘之弁護士の関係性}\label{ux821eux9db4ux4e8bux4ef6ux306eux4eacux7530ux8fbaux5e02ux306eux7d30ux5dddux6cbbux5f01ux8b77ux58ebux88abux5bb3ux8005ux5b89ux85e4ux6587ux3055ux3093ux5bb6ux65cfux3068ux88abux544aux767aux4ebaux9577ux8c37ux5dddux7d18ux4e4bux5f01ux8b77ux58ebux306eux95a2ux4fc2ux6027}}

\begin{itemize}
\tightlist
\item
  〉〉〉 Linux Emacs: 2021/05/16 15:27:33 〉〉〉
\end{itemize}

:CATEGORIES: @kanazawabengosi \#金沢弁護士会 @JFBAsns
日本弁護士連合会(日弁連) \#法務省 @MOJ\_HOUMU
\#被告発人長谷川紘之弁護士

 これからtwilog-serch
で調べてみたいと思いますが,細川治弁護士は本日2021年5月16日,初めて知った弁護士だと思います。なぜ今頃の発見というのが最近けっこう多いのですが,犯罪被害者と弁護士との関係性についてあらためて考える機会ともなりました。

 違いはよくわからないのですが,少なからず弁護士細川治などと,弁護士を先に付ける表記を見かけることがあります。「細川治弁護士のそりゃないぜ!日記」だと,自分が弁護士であることを強調した印象もあるので,どちらかといえばへりくだった表現になるのかもしれません。

 「弁護士細川治のそりゃないぜ!日記」というブログ名も,これまでにないある種の衝撃を受けたのですが,カバー写真とも言うのかヘッダ画像が,青い空と一面に広がる緑の畑,これは京都の名産である茶畑になるのではと思いました。

\begin{itemize}
\tightlist
\item
  2021年05月16日15時32分の登録:
  「(細川治\textbar 細川治弁護士)」を@hirono\_hideki @kk\_hirono @s\_hironoで検索 22件の該当 2021-05-16\_15:32の記録
  \url{https://kk2020-09.blogspot.com/2021/05/hironohidekikkhironoshirono222021-05.html} 
\end{itemize}

〉〉〉 kk\_hironoのリツイート 〉〉〉

\begin{itemize}
\tightlist
\item
  RT
  kk\_hirono(刑事告発・非常上告_金沢地方検察庁御中)|s\_hirono(非常上告-最高検察庁御中\_ツイッター)
  日時:2021-05-16 15:41/2021/05/16 15:34 URL:
  \url{https://twitter.com/kk\_hirono/status/1393818881366523904} 
  \url{https://twitter.com/s\_hirono/status/1393817208162197507} 
  \textgreater{}
  2021-05-16-153342\_舞鶴事件のその後 | 弁護士細川治のそりゃないぜ!日記.jpg
  \url{https://t.co/6VmCwFpAue} 
\end{itemize}

 細川治弁護士の法律事務所の住所が京田辺市となっていました。聞いたことのある地名でしたが,Googleマップで調べると,同じ京都府でも舞鶴市とは真反対の方向で,奈良市に近い場所となっていました。

 京都のお茶の産地で有名なのが宇治市だと思いますが,以前,テレビで茶畑をみるようなことがありました。はっきりとは憶えていないですが,1,2年ほど前だったように思います。

 なにがきっかけだったのか憶えていないですが,暗闇祭りについて調べたことがあり,その最初のきっかけとなったのも京都府で,奈良県との県境に近かったので,この京田辺市ではなかったかと思います。

\begin{lstlisting}
py37_env ❯ twilog-serch  京田辺市
\end{lstlisting}

\begin{itemize}
\tightlist
\item
  ./kk\_hirono2021-05-16\_153143.csv:2021-05-16 15:27:27 ``\#\#\#\#
  舞鶴事件の京田辺市の細川治弁護士,被害者安藤文さん家族と被告発人長谷川紘之弁護士の関係性''
  \url{https://twitter.com/kk\_hirono/status/1393815339599745030} 
\item
  ./kk\_hirono2021-05-16\_153143.csv:2019-01-15 12:32:29
  ``大野病院事件のことは、数日前にも思い出し、考えていました。京田辺市の産婦人科のニュースが大きかったのですが、同じ京都府に住むらしいモトケンこと矢部善朗弁護士(京都弁護士会)が全く話題にしていないように思えたのも不思議なことでした。少なくともツイートは未確認です。''
  \url{https://twitter.com/kk\_hirono/status/1085016847227838464} 
\item
  ./kk\_hirono2021-05-16\_153143.csv:2019-01-09 02:53:30
  ``とても期待したニュースは録画再生をしても見つかりそうもなかったので、内容を確認せずに全て消去しました。日付が変わって、朝から京田辺市の医療事故での和解の報道があるかもしれないので、ニュース報道以上の報道を期待し、2つ情報番組の録画を入れました。''
  \url{https://twitter.com/kk\_hirono/status/1082696814405443584} 
\item
  ./kk\_hirono2021-05-16\_153143.csv:2019-01-09 02:29:33
  ``時刻は2時27分となっています。1月9日です。日付が変わっています。京都府京田辺市の産婦人科医院の医療事故について、Twitterの検索を見ていたのですが、これは記憶にあると思う記事を見つけました。和解でニュースになったのとは別の問題のようです。''
  \url{https://twitter.com/kk\_hirono/status/1082690788159438848} 
\item
  ./kk\_hirono2021-05-16\_153143.csv:2019-01-08 17:24:12
  ``京田辺市の無痛分娩医療事故和解のニュースは、映像がミヤネ屋の全国ニュースで見たものとほぼ同じだと思いましたが、画面右上に括弧書きとして(現在は休院)などと出ていました。Mozcでは「きゅういん」として変換候補がなかったようです。一文字ずつ変換しました。''
  \url{https://twitter.com/kk\_hirono/status/1082553545616805888} 
\item
  ./kk\_hirono2021-05-16\_153143.csv:2019-01-08 17:20:48
  ``時刻は17時18分です。先ほどのCMのあと、梅鉢紋のような5項目のニュースの見出しが出てきて、最初に京都府京田辺市の無痛分娩での医療事故の和解のニュース、次が民主党の会派のニュース、その次にやはり島根県隠岐市での北朝鮮の木造船漂着のニュースでした。''
  \url{https://twitter.com/kk\_hirono/status/1082552687583850496} 
\end{itemize}

 無痛分娩での医療事故の和解のニュースは,同じ京都府でも場所を長岡京市と取り違えて記憶していたようです。昭和59年に金沢市場輸送の4トン保冷車の仕事で,よく夜中に油揚げを駐車場で積み替えていたのが,その長岡京市だったと思います。夜中のだだっぴろい場所でした。

 暗闇祭りは京都府宇治田原町となっていました。

\begin{itemize}
\item
  ./kk\_hirono2021-05-16\_153143.csv:2021-01-10 09:10:51
  ``他に同じ問題の情報がないのか調べたのですが,時間を掛けずに調べた範囲では見当たらなかったと思います。暗闇祭りについては,東京都府中市の「くらやみ祭り」の情報が多く出てきました。''
  \url{https://twitter.com/kk\_hirono/status/1348059694481301504} 
\item
  ./kk\_hirono2021-05-16\_153143.csv:2021-01-10 09:08:12
  ``■暗闇祭り■京都府宇治田原町の旧家による政財官界への接待の話題■少女出産■
  \textbar{} ママの交流掲示板 \textbar{} ママスタコミュニティ
  \url{http://mamastar.jp/bbs/comment.do?topicId=3529157''} 
  \url{https://twitter.com/kk\_hirono/status/1348059027385712642} 
\item
  ./kk\_hirono2021-05-16\_153143.csv:2021-01-10 09:06:32
  ``暗闇祭りというのも初めて知ったと思うのですが,変わった記事で「TVやネットでおなじみのお茶の産地・京都府綴喜郡宇治田原町のお話を、一緒に楽しくしましょう^^」という本文の他は,コメント欄に情報が多くあります。''
  \url{https://twitter.com/kk\_hirono/status/1348058606671826945} 
\item
  ./kk\_hirono2021-05-16\_153143.csv:2021-01-10 09:02:14 ``RT
  @passerby\_IP: @9jtCdbGf3lih8Fe
  こちらによれば、そのような「性犯罪」が、閉鎖的な町で、昔からの伝統、地元の風習と言う名で、現代でも行われているそうです。
  ■暗闇祭り■京都府宇治田原町の旧家による政財官界への接待の話題■少女出産■
  \url{http://mamastar.jp/bbs/comment.do?topicId=3529157''} 
  \url{https://twitter.com/kk\_hirono/status/1348057525527080962} 
\item
  宇治田原町 から 弁護士法人みそら総合 - Google マップ
  \url{https://t.co/Urq9FFZbUE} 
\end{itemize}

 調べてみると細川治弁護士の法律事務所から宇治田原町は車で20分,距離が10.5キロとなっていました。ブログでは「山城みそら法律事務所」となっていますが,Googleマップでは「弁護士法人みそら総合」に置き換わったような様子です。

\begin{itemize}
\tightlist
\item
  弁護士法人みそら総合 - 京都府京田辺市 - 弁護士ドットコム
  \url{https://t.co/k9KtDNVXVb}  ¥\n 弁護士ドットコム登録弁護士数1名 ¥\n ¥\n
  細川 治 弁護士(京都弁護士会)
\end{itemize}

 弁護士数1名となっているのでおかしいと思ったのですが,「弁護士ドットコム登録弁護士数1」ということで他の所属弁護士は弁護士ドットコムに未登録という意味なのかと理解しました。

 さきほどのまとめ記事を開いて見ると,細川治弁護士の名前のあるツイートが2014年に6つありました。私の方で忘れていたようですが,2014年11月5日のツイートから本日2021年5月16日のツイートまではけっこう長い空白となっています。

2014-07-13 10:36:00
``2014-07-13-103551\_一審京都地裁の公判から被害者参加した母親の代理人を務める細川治弁護士が京都市内で会見し、コメントを読み上げた.jpg
\url{http://pic.twitter.com/gPr6PttnTt''} 
\url{https://twitter.com/s\_hirono/status/488134718076887040} 

 今月に入ってからだと思いますが,舞鶴女子高校生殺害事件のニュース記事を見るようになりました。同じ日の少し先に見たのが大津市園児死傷事故のニュース記事で,こちらも報道は久しぶりでしたが,事故から2年目という節目で検察審査会に申し立てをするということでした。

\begin{itemize}
\tightlist
\item
  2021年05月16日16時22分の登録:
  「舞鶴.*事件」を@hirono\_hideki @kk\_hirono @s\_hironoで検索 267件の該当 2021-05-16\_16:22の記録
  \url{https://kk2020-09.blogspot.com/2021/05/hironohidekikkhironoshirono2672021-05.html} 
\end{itemize}

2021-05-07 13:52:01
``「生きていれば、どんなすてきな大人に」 舞鶴女子高生殺害事件、未解決のまま13年 遺族の苦悩深く|社会|地域のニュース|京都新聞
\url{https://t.co/eWqY7gwomr''} 
\url{https://twitter.com/hirono\_hideki/status/1390529834430832640} 

 5月7日のツイートを調べていたら,同じ京都府での次のニュースがありました。これはリツイートのようですが,最初の発見だったのかは不明です。

\begin{itemize}
\tightlist
\item
  ./kk\_hirono2021-05-16\_162236.csv:2021-05-07 11:06:41 ``RT
  @okinahimeji: 京都府警のひき逃げ事件捜査 弁護士が異例の立ち会い|NHK
  関西のニュース \url{https://t.co/oac4euEVpI''} 
  \url{https://twitter.com/kk\_hirono/status/1390488226129801216} 
\end{itemize}

 どうもツイートのテキストだけでは分かりづらく,他のアカウントのツイートも混じっているので,Twilogで調べてみます。

〉〉〉 kk\_hironoのリツイート 〉〉〉

\begin{itemize}
\tightlist
\item
  RT
  kk\_hirono(刑事告発・非常上告_金沢地方検察庁御中)|hirono\_hideki(奉納\さらば弁護士鉄道・泥棒神社の物語)
  日時:2021-05-16 16:32/2021/05/07 14:00 URL:
  \url{https://twitter.com/kk\_hirono/status/1393831739173179396} 
  \url{https://twitter.com/hirono\_hideki/status/1390532016618102784} 
  \textgreater{}
  大津園児事故、直進車不起訴に不服申し立て 遺族ら「注意義務怠った」|社会|地域のニュース|京都新聞
  \url{https://t.co/AdsaIlr76P}  2021年5月7日 10:18
\end{itemize}

 ツイートの時刻は5月7日の14時00分ですが,記事の配信時刻が同日の10時18分とあり,リンクを開いて確認しています。やはり記憶どおりの京都新聞の記事です。

\begin{itemize}
\tightlist
\item
  奉納\さらば弁護士鉄道・泥棒神社の物語(@hirono\_hideki)/2021年05月07日
  - Twilog \url{https://t.co/arRmiioltG} 
\end{itemize}

 すでにリツイート済みかもしれないのですが,ページ内検索で最初に舞鶴が出たのが13時52分のツイートで,次の京都新聞の記事へのリンクがあります。

\begin{itemize}
\tightlist
\item
  「生きていれば、どんなすてきな大人に」 舞鶴女子高生殺害事件、未解決のまま13年 遺族の苦悩深く|社会|地域のニュース|京都新聞
  \url{https://t.co/MNOkGgOmbl}  2021年5月7日 11:00
\end{itemize}

 記事の配信時刻が11時00分でした。私の記憶では大津市園児死傷事故の京都新聞の記事を読んでいるとき,右手のメニューで舞鶴の記事の見出しを見たと思うのです。なぜかツイートでは,舞鶴のツイートが13時52分,大津のツイートが14時00分と順番が逆になっているようです。

 大津市園児死傷事故のニュースは,Twitterのトレンドで知りました。これははっきり憶えています。

2021年05月16日16時45分の実行記録: twitterAPI-search-lawList-mydql-add.rb
``舞鶴 事件'' ツイート数:16/2413 リツイート数:9/2413
トータル:47``舞鶴 事件''の該当: hirono\_hideki 6/2件 kk\_hirono 8/7件
s\_hirono 2/0件

 TwitterAPIの検索の射程というのは意外に短いのですが,次の5月9日7時28分の私のツイートが最後の取得のようです。非常上告-最高検察庁御中\_ツイッター(@s\_hirono)のツイートでスクリーンショットの記録。「舞鶴事件で、小坂井久先生の控訴趣意書を」とあります。

TW s\_hirono(非常上告-最高検察庁御中\_ツイッター)2021/05/09 07:28:55
\url{https://twitter.com/s\_hirono/status/13911581980951592982021-05-09-072737\_} 舞鶴事件で、小坂井久先生の控訴趣意書を見せてもらっていなかったら、無罪は取れていなかったと思います。.jpg
\url{https://t.co/1xtyz1OuzH} 

 射程を21日に指定しまとめ記事を作成しました。処理の過程のメッセージを見ると,文春オンラインと私のアカウントの他は,細川治弁護士のTwitterアカウントだけでした。そういえば,リストに登録済みだったみたいなので,これも調べて確認しておきたいと考えていました。

\begin{itemize}
\tightlist
\item
  2021年05月16日16時51分の登録:
  REGEXP:''舞鶴.*事件''/データベース登録済みツイートの検索:2021-05-06〜2021-05-16/2021年05月16日16時51分の記録:ユーザ・投稿:5/33件
  \url{https://kk2020-09.blogspot.com/2021/05/regexp2021-05-062021-05\_16.html} 
\end{itemize}

 本日2021年8月16日前というのは,次の1件がデータベースにありました。京アニ,京都アニメーション放火殺人事件の弁護士無料相談会というツイートのようです。

\begin{itemize}
\item
  2019年08月07日13時35分の登録:
  \みそらの弁護士 @yamashiromisora\アメブロを更新しました。
  『京都アニメーション事件、弁護士会無料相談。』 \#京都アニメーション
  \#無料相談
  \url{http://hirono2014sk.blogspot.com/2019/08/yamashiromisora.html} 
\item
  ごぶさた、おしん。 \textbar{} 弁護士細川治のそりゃないぜ!日記
  \url{https://t.co/qLeQeIDPw2} 
\end{itemize}

 ku3を実行するのに,一つURLをコピーしたツイートにリンクがあったのが上記の記事です。「無事に出産したあと、お馴染みの奈良岡朋子のナレーションが、不吉な未来を告げる。」とあるので,やはりNHK連続テレビ小説のことと思われます。

 2019年頃のTwitterのまとめ記事は,埋め込みツイートの数が現在の50件の倍,100件となっているので埋め込みツイートが表示されないまま何時までもページの読み込みが続いていました。このあたりはTwitterの仕様変更もあるようですが,以前は問題なくすぐに表示されていました。

 Twitterの細川治弁護士のページで「1,712件のツイート」とありますが,ku3では1,711件の取得となっていました。削除されたツイートなのか不明ですが,3200件前後より少ないという条件はあるものの,満額に近くのは初めてかもしれません。

\begin{itemize}
\tightlist
\item
  2021年05月16日17時00分の登録:
  @yamashiromisora(弁護士細川治@みそら)のツイート ''.*'' 1711/1711:2013-01-27\_2152〜2021-05-07\_1222 2021年05月16日17時00分の記録
  \url{https://kk2020-09.blogspot.com/2021/05/yamashiromisora171117112013-01.html} 
\end{itemize}

 時刻は17時17分です。いつの間にか17時を過ぎていてちょっと驚きました。16時台は一度も時間を見ていないかもしれません。外の天気もどんより曇ったままなので,夕陽で時間の変化がわかるということもありません。

 本日は,細川治弁護士のついて時間を割いて取り上げてきましたが,1つ大きな動機となることがありました。手を広げすぎたので探しづらいというのはありそうですが,幸い,細川治弁護士がブログの方で,検索をするように勧めていて,アメブロの検索結果は見やすく要約付きで表示されていました。

\begin{itemize}
\tightlist
\item
  【舞鶴】のブログ記事検索結果|Ameba検索 \url{https://t.co/dFxTAWTcAt} 
\end{itemize}

\begin{quote}
《引用の始まり》
\end{quote}

\begin{quote}
また、会見においてお話ししたことを少し補足させていただきます。

一昨日、偶然にも、母親は京都府警から美穂さんの遺品等の還付を受けていました。またその際には、京都府警担当者から、舞鶴事件についての再捜査の状況等についても説明を受けたとのことでした。

このようなタイミングで、逮捕の事実と、その事件の内容に触れたことで、知らせを受けたときは相当動揺され、鼓動が止まらないような感覚になったと話されていました。
\end{quote}

\begin{quote}
《引用の終わり》
\end{quote}

\begin{itemize}
\tightlist
\item
  舞鶴事件被害者遺族のコメント \textbar{}
  弁護士細川治のそりゃないぜ!日記 \url{https://ameblo.jp/yamashiromisora/entry-11948907181.htmln} 
\end{itemize}

 2014-11-06
15:34:42が記事の投稿時刻となっています。「一昨日、偶然にも、母親は京都府警から美穂さんの遺品等の還付を受けていました。」とあるので,一昨日というのは2014年11月4日のことかと思います。

 中勝美被疑者逮捕という報道は印象的に記憶にあるのですが,翌日か翌々日になって,わいせつ目的という報道が出て,その頃には社会の関心も薄れ,ひと目に触れにくくなっているという状況でした。

\begin{itemize}
\tightlist
\item
  2021年05月16日17時32分の登録:
  「\^{}2014-11-0{[}4-6{]}」を@hirono\_hideki @kk\_hirono @s\_hironoで検索 74件の該当 2021-05-16\_17:32の記録
  \url{https://kk2020-09.blogspot.com/2021/05/2014-11-04-6hironohidekikkhironoshirono.html} 
\end{itemize}

2014-11-05 20:07:18 ``RT @norishiroyukiya:
現行犯の今回は無罪はねーでしょうな。裁判官や弁護士も連座制を適用しなくちゃ割に合わんな。
RT @hirono\_hideki 舞鶴女子高生殺害事件で無罪確定へ \url{http://ow.ly/z1Is9} 
この裁判で、弁護を担当した小坂井久弁護士によりますと、中勝美さんは「無実・・・''
\url{https://twitter.com/hirono\_hideki/status/529953094894108672} 

2014-11-05 20:07:22 ``RT @norishiroyukiya:
小坂井久弁護士はせめて、中勝美容疑者に何十回も滅多刺しにされた女性を見舞ってほしいですね。
法的責任がなければ何もしなくて良いのかね?RT @hirono\_hideki
主任弁護人を務めた小坂井久弁護士(大阪弁護士会)は10日、大阪市内で記者会見し、「刑事裁判の原則にのっとった・・・''
\url{https://twitter.com/hirono\_hideki/status/529953112522764290} 

2014-11-05 20:08:55 ``\url{http://www.fnn-news.com:} 
女子高校生殺害事件で\ldots{} - Mozilla Firefox \url{http://ow.ly/DQLjN} 
女子高校生殺害事件で無罪確定の男、38歳女性刺したとして逮捕''
\url{https://twitter.com/hirono\_hideki/status/529953501959684096} 

2014-11-05 20:12:51
``今回の事件について、若狭弁護士は「舞鶴の事件で無罪になった件は、今後、それがもう1回、審理をやり直されることはない。実は疑わしかったんだということで、(今回の事件の量刑に)考慮することはできない」と話した。
\url{http://ow.ly/DQLXd''} 
\url{https://twitter.com/hirono\_hideki/status/529954491605737472} 

 昼の午後にテレビで見たニュースとして記憶にあったのですが,ツイートは20時07分が最初となっているようです。今ほど,ニュースをツイートにすることはなく,投稿する時間というのも無頓着であったように思います。タブで開いたままになっていたページをツイートとするとか。

\begin{itemize}
\tightlist
\item
  【衝撃事件の核心】誰もが恐れていた「再犯」・・・メッタ刺しの凶行に蘇った「舞鶴女子高生殺害」の記憶(1/4ページ)
  - 産経WEST \url{https://t.co/LUcc0vjs9A} 
  5日午前8時40分ごろ、大阪市北区兎我野(とがの)町の雑居ビル。自分のいた部屋に突然、飛び込んできた2人の様子に驚いた男性は、とっさに廊
\end{itemize}

 文字数の関係でツイートが途切れていますが,とっさに廊下に出て110番したとあり,やはり11月5日の午前8時40分頃に発生した事件でした。「一昨日、偶然にも、母親は京都府警から美穂さんの遺品等の還付を受けていました。」の翌日であることを確認したことになります。

 しかし,なぜ京都府警から遺品等の還付なのか気になります。私も被疑者として押収品の還付を受けていますが,それは金沢地方検察庁から福井刑務所に郵送のあったものと思います。

 まだ寒い季節ではなく天気の良い日の午後であったように記憶にあるのですが,福井刑務所の中の受刑者の領置品を置いた広い部屋に,一人の刑務官に案内されたような記憶です。平成9年1月18日が満期出所でしたが,その2,3ヶ月前だったかもしれません。

 なぜ今頃なのか不思議に思い,答えは見つからなかったのですが,血のついたシャツやズボン,カセットテープケースの破片もあったかと思います。自宅に郵送したという記憶もないので,その場で処分をしてもらったのかもしれません。細々と数があったかもしれませんが,記憶が薄れています。

 京都府警が被害者の遺品等をあずかっていたというのは,無罪判決の確定で事件が解決しておらず,再捜査の必要があったとも考えられますが,それを遺族に還付した翌日に,無罪判決になって釈放されていた被疑者が,別の殺人未遂事件を起こし,現行犯で逮捕されるというのは偶然なのかということです。

 偶然なのか否か,その答えは誰にもないと思いますが,巡る因果ということなのかもしれません。

2021-05-16 08:04:15
``「舞鶴高1年女子」の母「一番恐れていたことが現実になった」・・・被害者の代理人が会見
- 産経WEST \url{https://t.co/M373yKr63s} 
小杉さんの母親(44)の代理人を務める細川治弁護士が5日、京都市内で会見を開き、「驚きとともに憤りを感じた。一番恐れていたことが現実となってしまった。捜査機関には、''
\url{https://twitter.com/hirono\_hideki/status/1393703803631783936} 

\begin{itemize}
\tightlist
\item
  「舞鶴高1年女子」の母「一番恐れていたことが現実になった」・・・被害者の代理人が会見
  - 産経WEST \url{https://t.co/7UICDiiMmQ}  2014.11.5 19:28
\end{itemize}

 本日最初の細川治弁護士の名前が出るツイートとリンクにある産経のニュース記事ですが,確認すると・・・。2021年5月5日と思いこんでいたのが,見直すと2014.11.5
19:28となっていました。2014年11月5日です。

 今月の初めの方に,京都新聞の舞鶴女子高校生殺害事件の記事を見ていたので,同時期の産経ニュースの記事という思い込みがありました。ページ内のリンクに数年前の古い記事があることは珍しくないですが,舞鶴の事件を調べていたわけではなかったと思います。Twilogで確認します。

\begin{itemize}
\tightlist
\item
  首相は「どこかの神社のお猿さん」 立民・安住氏、日光の三猿なぞらえ批判
  - 産経ニュース \url{https://t.co/RU6J9PdwoB} 
\end{itemize}

 直前のツイートにリンクのある記事です。念のため舞鶴でページ内検索をしましたが該当なしです。「舞鶴高1年女子」の母,という記事は産経WESTで,こちらは産経ニュースとなっているという違いもあります。

 2つのツイートをTwitterAPIで取得し並べてみます。

\begin{itemize}
\item
  TW hirono\_hideki(奉納\さらば弁護士鉄道・泥棒神社の物語) 日時:
  2021/05/16 08:02:01 URL:
  \url{https://twitter.com/hirono\_hideki/status/1393703241477627906} 
  \textgreater{}
  首相は「どこかの神社のお猿さん」 立民・安住氏、日光の三猿なぞらえ批判
  - 産経ニュース \url{https://t.co/SM5pnVN2Q5} 
\item
  TW hirono\_hideki(奉納\さらば弁護士鉄道・泥棒神社の物語) 日時:
  2021/05/16 08:04:15 URL:
  \url{https://twitter.com/hirono\_hideki/status/1393703803631783936} 
  \textgreater{}
  「舞鶴高1年女子」の母「一番恐れていたことが現実になった」・・・被害者の代理人が会見
  - 産経WEST \url{https://t.co/M373yKr63s} 
  小杉さんの母親(44)の代理人を務める細川治弁護士が5日、京都市内で会見を開き、「驚きとともに憤りを感じた。一番恐れていたことが現実となってしまった。捜査機関には、
\end{itemize}

 「どこかの神社のお猿さん」のツイートが8時2分1秒,次の「舞鶴高1年女子」の母「一番恐れていたことが現実になった」,のツイートが8時4分15秒で,その間は2分14秒です。

\begin{itemize}
\tightlist
\item
  万引疑いNHK副局長逮捕 - 産経ニュース \url{https://t.co/T6iVANKeFU} 
  逮捕容疑は15日午前11時40分ごろ、旭川市のリサイクルショップで水筒など3点(計770円相当)を盗んだ疑い。
\end{itemize}

 「どこかの神社のお猿さん」の記事のページで,右手にランキング1位となっているのを見つけた記事です。見出しにNHK副局長逮捕とのみありますが,記事には旭川放送局とあります。女子中学生が自殺した問題で本日も混迷が伝わる記事の見出しを見かけていた北海道旭川市です。

 文春オンラインのまとめ記事で見かけたように思いますが,リンクは開いて読んでいません。ツイートに要約は表示されていたかと思います。

\begin{itemize}
\tightlist
\item
  《遺族は「違和感しかない」》イジメ調査の第三者委員会リストに``問題人物'' 旭川14歳女子凍死
  \textbar{} 文春オンライン \url{https://t.co/QK1GRztrxb} 
\end{itemize}

 記事の見出しにはありませんが,見出しとヘッダ画像の間にある#の番号のリンクの意味が今になって理解できました。「旭川14歳少女イジメ凍死事件
\#13」というサブタイトルのようなものでシリーズ化されているようです。

 弁護士が表に現れてこないことで気になっている事件でもあります。

 半角の#がつくとツイートではハッシュタグになってしまいそうですが,これはわかりやすいまとめ方だと思いました。

\begin{itemize}
\tightlist
\item
  〈〈〈 2021/05/16 18:32:57 Linux Emacs: 〈〈〈
\end{itemize}

\hypertarget{ux88abux5bb3ux8005ux5b89ux85e4ux6587ux3055ux3093ux5bb6ux65cfux3068ux88abux544aux767aux4ebaux9577ux8c37ux5dddux7d18ux4e4bux5f01ux8b77ux58ebux306eux95a2ux4fc2ux6027ux821eux9db4ux4e8bux4ef6ux3067ux4eacux7530ux8fbaux5e02ux306eux7d30ux5dddux6cbbux5f01ux8b77ux58ebux304bux3089ux601dux3046ux3053ux3068}{%
\paragraph{被害者安藤文さん家族と被告発人長谷川紘之弁護士の関係性,舞鶴事件で京田辺市の細川治弁護士から思うこと}\label{ux88abux5bb3ux8005ux5b89ux85e4ux6587ux3055ux3093ux5bb6ux65cfux3068ux88abux544aux767aux4ebaux9577ux8c37ux5dddux7d18ux4e4bux5f01ux8b77ux58ebux306eux95a2ux4fc2ux6027ux821eux9db4ux4e8bux4ef6ux3067ux4eacux7530ux8fbaux5e02ux306eux7d30ux5dddux6cbbux5f01ux8b77ux58ebux304bux3089ux601dux3046ux3053ux3068}}

\begin{itemize}
\tightlist
\item
  〉〉〉 Linux Emacs: 2021/05/18 20:54:39 〉〉〉
\end{itemize}

:CATEGORIES: @kanazawabengosi \#金沢弁護士会 @JFBAsns
日本弁護士連合会(日弁連) \#法務省 @MOJ\_HOUMU
\#被告発人長谷川紘之弁護士

 被告発人長谷川紘之弁護士をレベル3の項目に,レベル4として「「金沢地方裁判所
昭和48年(ワ)329号
判決」に名前が出てきた被告発人長谷川紘之弁護士,これまで余り気にすることがなかった年齢」を追加したのですが,見出しだけだったこの項目を先に片付けます。

 「舞鶴事件で京田辺市の細川治弁護士から思うこと」と見出しの後段にありますが,Googleで被告発人長谷川紘之弁護士について調べていると,それどころではないような情報が出てきて謎が深まりました。

 安藤健次郎さんがなぜ,被告発人長谷川紘之弁護士を被害者安藤文さんの訴訟代理人にしたのか私にはまったくわからず,安藤健次郎さん本人に尋ねてこともありません。

 記憶が曖昧になっていますが,平成11年7月頃,安藤健次郎さんが私と会って話すことを約束しながら,私には友達を連れてくるように条件を出しながら,後になって弁護士を同伴させるつもりだったと聞いたような憶えがあります。これは本人ではなく金沢中警察署の担当刑事から聞いたように思います。

 その平成11年の安藤健次郎さんに対する傷害事件で,取調べを担当したのは井口か猪口という名前の警部補だったと思うのですが,ずいぶん前から記憶がはっきりしなくなっています。一つ記憶にあるのは,当時の能美郡になるのか,関連付けて記憶した地名があることです。

 Googleマップで調べますが,その石川県内の地名というのは,本日集中的に取り上げた被告発人長谷川紘之弁護士の妻が被害に遭った強盗事件,福井刑務所では加害者本人の話という伝聞として強姦もしているという事件なのですが,犯人が当日の午前中に,わいせつ事件を起こしたと新聞で見た地名です。

 何年も前に宇出津の図書館で北國新聞縮小版で記事を見たという記憶なのですが,それでも最後に記事をみたのは,現在のコンセールのとに図書館が移転した後のことのであったように思います。

 記憶が曖昧になっているので断定的なことは言えませんが,強制わいせつでも未遂事件になっていたように思います。陰茎という男性器を女性の膣内に挿入して既遂になるという当時の強姦罪とは違い,どのように想像しても素人の私には強制わいせつの未遂というのが状況としてイメージ出来ませんでした。

 いくつか余罪がありながら早く刑事裁判の判決が確定し同じ福井刑務所に服役したと聞いていました。詳細は,私に金沢合同法律事務所をお勧めで教えてくれた同じ第2工場の受刑者のこと,その受刑者も強制わいせつで有罪となり,再審請求をすると言っていたこととして過去に記述があるかもしれません。

 当時の石川郡野々市町,現在の野々市市の運送会社で,運転手として同じ会社の女性事務員を被害者とした事件と話を聞いていたように思うのですが,私の傷害・準強姦被告事件に似ているだけではなく,中西運輸商で接点のある運送会社でした。

 その野々市の運送会社は今でも記憶にありますが,再審請求などの書面には繰り返し詳細な記述をしているはずですが,ネットでは触れたことがなかったかもしれません。今からその運送会社名で検索をしてみますが,結果だけ次にお知らせします。

 求人情報のたぐいは野々市市の運送会社としてGoogleの検索結果にぞろぞろと出てきました。他にも会社の形態としてずいぶん久しぶりに見かけた情報があるのですが,特定を避けるため記述はしません。御庁つまり金沢地方検察庁には廃棄されていない限り,山ほど情報があるはずです。

 運送の前の,漢字は2文字で,読みは3文字だと聞いていましたが,同じ3文字の読みでも読み方が違っているような情報を,Google検索の結果にある要約で見かけました。福井県敦賀市の国道8号線沿い,急な上り坂の手前にあるドライブインで3人で食事をしたことは,今もしっかり記憶にあります。

 犯罪被害者の代理人というのは,ここ10年ほどの間に出てきたと思いますが,これは最近見かけなくなった法改正によるものと思います。時間も掛かるのであえて調べるようなことはしませんが,被害者参加制度などとなっていたと思います。

 被告発人長谷川紘之弁護士が被害者安藤文さんの原告訴訟代理人となった平成6年当時は,そのような制度など影も形もなく,法改正のきっかけとなったのが,今は全く情報を見かけなくなった岡村弁護士の妻殺害事件と,こちらも最近は情報を見かけていない光市母子殺害事件でした。

 被告発人古川龍一裁判官が単独で判決を出したのが平成7年中だったと思いますが,本日も公開された書面の写真ファイルを確認しています。被告発人長谷川紘之弁護士の訴状の方をまだ確認していないのですが,刑事裁判の有罪判決の事実認定をそのまま引き写しで援用した内容でした。

 最近になって,安藤健次郎さんは弁護士に対する根深い不信感から試すつもりで,被告発人長谷川紘之弁護士を選んで被害者安藤文さんの訴訟代理人とした可能性があるように思えてきました。それが最も有力な事実経過の解釈とも,その場の当事者として考えています。

 令和3年3月31日付告発状でも取り上げたように思いますが,平成6年11月の初め頃,金沢地方裁判所民事A係経由で,訴状と証拠番号のような書面が届き,その一週間か10日後ぐらいに,私の母親を経由して一件記録と表紙にあった書類が郵送で私に届きました。

 被告発人木梨松嗣弁護士から届いた一見記録は1つの編綴で,被告発人長谷川紘之弁護士の書類は2つの編綴となっていたと思いますが,分量は被告発人木梨松嗣弁護士の方が相対的に6割程度となっていたように思います。

 被告発人長谷川紘之弁護士の書類は2つの編綴となった書面は,温泉宿の脱衣所によくあるような大きさの衣類を入れる用途の駕籠で,左右どちらも上の部分が駕籠の上辺から3,4センチあるいは4,5センチはみ出す高さとなっていたように思います。

 家庭では洗濯かごとして使われているものかもしれません。ホームセンターに行けば置いてありそうですが,今までホームセンターで思い出すこともなかったので,どれぐらいの大きさなのか確認はしていませんが,だいたいの大きさは決まっていたように思います。

 手元に圧縮された779枚のA4用紙があるのですが,上下になっているコピー用紙が入っていたと思われるダンボールの厚みを含めても,その1つの山で4倍はあったように思えます。ただ,当時はほとんどがB4用紙の袋とじだったので,A4用紙とはいくらか違った分量になるのかもしれません。

 編綴されていた書面はずっと前にばらしており,散逸もしているので,実際,どれほどの分量であったのか正確なことはわからなくなっていますが,上述が私に出来るご説明としたの目安にやります。

 被告発人長谷川紘之弁護士経由で郵送された書面と被告発人木梨松嗣弁護士から郵送された書面は,刑事裁判の記録としてほぼ重複していたのですが,違いは被告発人長谷川紘之弁護士から郵送された書類には,病院の診断書や病院の領収書が多数含まれていたことです。

 他に,被害者安藤文さんの意識が戻ったあとに谷内孝志警部補が聞き取りを行い作成した調書というのも,被告発人長谷川紘之弁護士から郵送された書面にのみ綴られていたかもしれません。それも長い間見つかっていないのですが,捨てるはずもなく,家の中で見つからないとおかしいのです。

 私の願望だったのかもしれないのですが,谷内孝志警部補の作成した起訴後の調書は,後悔が色濃くにじみ出て,全く別人の作成した調書という印象を受けたのですが,真正に作成されたものと思い込んでいたところ,今思うところ,事実を捻じ曲げた保身のためのデタラメの調書だったのかもしれません。

 時刻は22時14分です。今しがたスマホで撮影をしたのですが,どんたく宇出津店で輪島市町野町の茶碗豆腐を買ってきました。茶碗豆腐は母親が買ってきたのを食べた記憶があるのですが,自分が買ってきた可能性を含めても10年ぶりに近くなっていると思います。

 谷内という名前は奥能登のかなり多い苗字だと思いますが,金沢西警察署での取調べで輪島市出身と私の質問に答えていた谷内孝志警部補は,その豆腐屋とも親戚の可能性があるとも考えています。4月の初めに町野で川沿いの桜並木を見た後,看板を見かけ,それから気になっていました。

 谷内孝志警部補についてはまるっきり情報がありませんが,個人的に自殺している可能性もあるとずっと前から考えていました。1992年に40歳前後と聞いたので,健在であれば昭和27年頃の生まれで現在70歳前後という計算になります。石川県警察の方は退職しているでしょう。

 平成11年の安藤健次郎さんに対する傷害事件のときも基本は同じでしたが,平成4年4月1日の夜に金沢西警察署に出頭後,自首調書が作成された後は,翌日の4月2日からずっと谷内孝志警部補のみの取調べと供述調書の作成でした。

 この谷内孝志警部補の功名心と手柄の出世欲が,初動捜査を大きく歪めたと私はずっと前から考えているのですが,最近は不思議と見かけない漁夫の利という言葉のように,それを弁護士商売に悪用しようとしたのが被告発人木梨松嗣弁護士で,同調者が被告発人長谷川紘之弁護士とも見ています。

 福井刑務所の満期出所後に私が頼ってくることを当然のごとく期待していたとも推測される被告発人木梨松嗣弁護士の動きですが,被告発人大網健二との関係性が重要なところで,二股の選択肢として,私が金沢刑務所の拘置所で発狂したり,出所後の悪友で覚醒剤中毒になることも期待していたはず。

 市場急配センターの殺人未遂事件を社会的に抹殺しようとした動機,犯行態様に疑問を挟む余地はなく,前代未聞の悪質さがあるので,他の弁護士や裁判官と同様,一律に無期懲役,二度とは社会に戻れない弁護士鉄道の末路としてのパスポートを渡すのが,刑事罰として妥当というところです。

 あと,平成9年の2月頃に掛けた電話で,被告発人岡田進弁護士を被告発人長谷川紘之弁護士と取り違えたような記事もネットで書いていたと最近になって確認しているのですが,やはり被告発人岡田進弁護士の可能性が高いと思います。これも被告発人岡田進弁護士をメインに後日取り上げます。

 能登町にも宇出津に隣接した辺田の浜に谷内自動車がありますが,谷内という名前は奥能登でかなり多い苗字だと重ます。谷内孝志警部補は「やち」と読みますが,あるいは矢内であったのか記憶は定かでないものの,昭和55年頃に「やない」という読み方も宇出津の人で聞いています。

\begin{itemize}
\tightlist
\item
  谷内さんの名字の由来や読み方、全国人数・順位|名字検索No.1/名字由来net|日本人の苗字・姓氏99\%を掲載!!
  \url{https://t.co/LNEwsq13KS} 
\end{itemize}

 上記のページで確認したところ予想通り石川県が1位でしたがおよそ3,200人で,市町村別では金沢市が820人,輪島市が580人,能登等が240人となっていますが,およそは省略しました。金沢市に多いというのは意外な結果でしたが,能登から流れている可能性のあると思います。

 なんどでもエンドレスに繰り返したい気持ですが,平成4年当時金沢西警察署にいた谷内孝志警部補,谷内孝志警部補,谷内孝志警部補です。おみしおりなく,というような言葉もあったような。

 自殺をしてくれという気持はさらさらなく,被告発人も長谷川紘之弁護士も真相解明のため生きていてくれたらと願うぐらいですが,自殺などしてくれたら迷惑なだけです。責任のとりかたは重要ですが,ここではっきり書いておきます。他の被告発人もまったく同じです。

 前にも金沢弁護士会の今年度の会長となった塩梅という苗字について調べていますが,たぶん同じサイトで,今回は比率が多い地域TOP5という項目があるのに気が付きました。それによると市町村別で石川県輪島市が1位,比率が2.153%となっています。100人に2人以上という計算です。

 能登町では鵜川の久田船長と同じ久田という苗字が多いのですが,宇出津の人で親戚関係はないという話も聞いています。輪島市の場合はより範囲も広いので,谷内孝志警部補と親戚関係のない谷内という苗字の人も多いと思います。それでも谷内孝志警部補と連呼したくなります。

 元警察官としてのこの上のない恥さらしとも思いますが,その点は江村正之検察官もおなじです。江村正之検察官については茨城県下妻市の江村法律事務所,江村正之弁護士としても取り上げていると思いますが,こちらも自殺はするな責任をとれ,首を洗って待っていろ,という気持です。

\begin{itemize}
\tightlist
\item
  〈〈〈 2021/05/18 23:01:41 Linux Emacs: 〈〈〈
\end{itemize}

\hypertarget{ux91d1ux6ca2ux5730ux65b9ux88c1ux5224ux6240-ux662dux548cuxff14uxff18ux5e74ux30efuxff13uxff12uxff19ux53f7-ux5224ux6c7aux306bux540dux524dux304cux51faux3066ux304dux305fux88abux544aux767aux4ebaux9577ux8c37ux5dddux7d18ux4e4bux5f01ux8b77ux58ebux3053ux308cux307eux3067ux4f59ux308aux6c17ux306bux3059ux308bux3053ux3068ux304cux306aux304bux3063ux305fux5e74ux9f62}{%
\paragraph{「金沢地方裁判所 昭和48年(ワ)329号
判決」に名前が出てきた被告発人長谷川紘之弁護士,これまで余り気にすることがなかった年齢}\label{ux91d1ux6ca2ux5730ux65b9ux88c1ux5224ux6240-ux662dux548cuxff14uxff18ux5e74ux30efuxff13uxff12uxff19ux53f7-ux5224ux6c7aux306bux540dux524dux304cux51faux3066ux304dux305fux88abux544aux767aux4ebaux9577ux8c37ux5dddux7d18ux4e4bux5f01ux8b77ux58ebux3053ux308cux307eux3067ux4f59ux308aux6c17ux306bux3059ux308bux3053ux3068ux304cux306aux304bux3063ux305fux5e74ux9f62}}

\begin{itemize}
\tightlist
\item
  〉〉〉 Linux Emacs: 2021/05/19 06:13:11 〉〉〉
\end{itemize}

:CATEGORIES: @kanazawabengosi \#金沢弁護士会 @JFBAsns
日本弁護士連合会(日弁連) \#法務省 @MOJ\_HOUMU
\#被告発人長谷川紘之弁護士

\begin{quote}
《引用の始まり》
\end{quote}

\begin{quote}
原告 大谷健

右訴訟代理人弁護士 手取屋三千夫

同 北尾強也

同 野村侃靱

同 長谷川紘之

同 山腰茂

被告 社会福祉法人恩賜財団済生会

右代表者理事 真柄要助

右訴訟代理人弁護士 岩上勇二

同 田中幹則
\end{quote}

\begin{quote}
《引用の終わり》
\end{quote}

\begin{itemize}
\tightlist
\item
  金沢地方裁判所 昭和48年(ワ)329号 判決 -
  大判例 \url{https://daihanrei.com/l/\%E9\%87\%91\%E6\%B2\%A2\%E5\%9C\%B0\%E6\%96\%B9\%E8\%A3\%81\%E5\%88\%A4\%E6\%89\%80\%20\%E6\%98\%AD\%E5\%92\%8C\%EF\%BC\%94\%EF\%BC\%98\%E5\%B9\%B4\%EF\%BC\%88\%E3\%83\%AF\%EF\%BC\%89\%EF\%BC\%93\%EF\%BC\%92\%EF\%BC\%99\%E5\%8F\%B7\%20\%E5\%88\%A4\%E6\%B1\%BAn} 
\end{itemize}

 被告発人長谷川紘之弁護士が原告代理人として名を連ねる判例ですが,「金沢地方裁判所
昭和48年(ワ)329号
判決」という事件番号があるものの,判決の出た日が確認できません。長期に渡る裁判で,数年後に被告発人長谷川紘之弁護士が参加した可能性もありそうです。

 被告が「社会福祉法人恩賜財団済生会」となっており,初めは医療事故かと思ったのですが,少し読むと勤務していた医師との雇用契約のトラブルのような感じです。

 昔ほど裁判は長引くものというイメージがあり,一昨日あたりもアスベストで画期的な最高裁の初判決が出たというニュースがあり驚いていました。アスベストの健康被害が社会問題になったのは,もうずっと前という印象がありました。

\begin{itemize}
\tightlist
\item
  建設アスベスト 国と企業の責任認める 最高裁が初判決 \textbar{}
  NHKニュース
  \url{https://www3.nhk.or.jp/news/html/20210517/k10013035241000.html} 
\end{itemize}

 記事にざっと目を通すと,アスベストの裁判が始まったのは平成20年の13年前とありますが,「これを踏まえて2006年、被害を救済するための法律「アスベスト健康被害救済法」が施行されました。」ともあります。2年前の平成18年には法律が出来ていたと知りました。

\begin{itemize}
\tightlist
\item
  石川県済生会金沢病院 - Google マップ \url{https://t.co/o15Eb1HTki} 
  〒920-0353 石川県金沢市赤土町ニ13−6
\end{itemize}

 この金沢市の郊外にある済生会病院ですが,平成11年8月8日の安藤健次郎さんに対する傷害事件で逮捕される数日前,建築建設現場への派遣の仕事で,近くに現場に行き,ちょうど少し前に電話で関係者KYが入院していると聞いたこの病院に行き,一階のロビーで話をしました。

 関係者KYが鉄工所と呼んでいた安藤健次郎さんを殴った話をし,今度一緒に会ってもらうことになるかもしれないと話したところ,彼はいよいよやってくれたかと満足そうな表情を浮かべているように見えました。

 詳しいことは記憶にないですが,内蔵を悪くしたらしく,本人の様子を見ても精神的なストレスが原因のように見え,最初病室で会った時も普段よりおとなしい感じでしたが,私の話を聞くと,やれやれようやく良い方向に向いてくれたと安堵をしたように見えました。

 被告発人長谷川紘之弁護士の顔写真というのは北國新聞の法律相談コーナーで掲載された白黒の写真を見ただけですが,磯野波平のような頭髪で頭頂部に毛がなかったと思います。それでもけっこう若い年齢に見える顔立ちで,年齢と頭髪のギャップの大きさがかなり珍しく感じました。

 被告発人長谷川紘之弁護士の名前を知ったのは福井刑務所にいた平成6年11月なので,出所するまで北國新聞を見ることはなく,法律相談で掲載された写真は,金沢刑務所の拘置所で直に北國新聞で見た被告発人木梨松嗣弁護士と同時期だったので,平成5年頃の記事と思います。

 つまり被告発人長谷川紘之弁護士の写真は北國新聞縮小版で見た可能性が高いのですが,写真も小さくなりインクのにじみのようなものも出て,本来の紙面より不鮮明でした。特に劣化が顕著だと思った写真が,これも同時期の同じコーナーの写真で,被告発人岡田進弁護士でした。

\begin{itemize}
\tightlist
\item
  〈〈〈 2021/05/19 06:55:41 Linux Emacs: 〈〈〈
\end{itemize}

\hypertarget{ux751fux5b58ux304cux78baux8a8dux3067ux304dux306aux3044ux88abux544aux767aux4ebaux9577ux8c37ux5dddux7d18ux4e4bux5f01ux8b77ux58ebux30e9ux30b8ux30aaux304bux306aux3056ux308fux96fbux8a71ux306aux3093ux3067ux3082ux76f8ux8ac7ux5ba4-ux6708ux66dcux65e5-ux6cd5ux5f8bux56f0ux308aux4e8bux76f8ux8ac7ux306eux30dbux30fcux30e0ux30daux30fcux30b8ux306bux51faux6f14ux60c5ux5831}{%
\paragraph{生存が確認できない被告発人長谷川紘之弁護士,ラジオかなざわ電話なんでも相談室 月曜日 法律・困り事相談のホームページに出演情報}\label{ux751fux5b58ux304cux78baux8a8dux3067ux304dux306aux3044ux88abux544aux767aux4ebaux9577ux8c37ux5dddux7d18ux4e4bux5f01ux8b77ux58ebux30e9ux30b8ux30aaux304bux306aux3056ux308fux96fbux8a71ux306aux3093ux3067ux3082ux76f8ux8ac7ux5ba4-ux6708ux66dcux65e5-ux6cd5ux5f8bux56f0ux308aux4e8bux76f8ux8ac7ux306eux30dbux30fcux30e0ux30daux30fcux30b8ux306bux51faux6f14ux60c5ux5831}}

\begin{itemize}
\tightlist
\item
  〉〉〉 Linux Emacs: 2021/05/19 07:17:58 〉〉〉
\end{itemize}

:CATEGORIES: @kanazawabengosi \#金沢弁護士会 @JFBAsns
日本弁護士連合会(日弁連) \#法務省 @MOJ\_HOUMU
\#被告発人長谷川紘之弁護士

 昨夜,Google
Chromeのブラウザで被告発人長谷川紘之弁護士の検索をしていたのですが,見出しにある番組出演情報を見つけ,今朝になって履歴から探したところ見つからず,スクリーンショットの記録を頼りに探すと,タイトルにFMがないことがわかりました。

 昨夜見ていたはずのページが見つからず,履歴まで残っていないと思ったので,これは心霊現象なのかと考えも頭をよぎりました。ラジオかなざわ,で終了している番組のようですが,ホームページはそのまま残されているようです。番組表で確認しました。

〉〉〉 kk\_hironoのリツイート 〉〉〉

\begin{itemize}
\tightlist
\item
  RT
  kk\_hirono(刑事告発・非常上告_金沢地方検察庁御中)|s\_hirono(非常上告-最高検察庁御中\_ツイッター)
  日時:2021-05-19 07:27/2021/05/19 07:26 URL:
  \url{https://twitter.com/kk\_hirono/status/1394781795636285441} 
  \url{https://twitter.com/s\_hirono/status/1394781509354070016} 
  \textgreater{}
  2021-05-19-071417\_電話なんでも相談室 - Google Chrome.jpg
  \url{https://t.co/yxBT3Rbnag} 
\end{itemize}

〉〉〉 kk\_hironoのリツイート 〉〉〉

\begin{itemize}
\tightlist
\item
  RT
  kk\_hirono(刑事告発・非常上告_金沢地方検察庁御中)|s\_hirono(非常上告-最高検察庁御中\_ツイッター)
  日時:2021-05-19 07:27/2021/05/19 07:26 URL:
  \url{https://twitter.com/kk\_hirono/status/1394781809133584385} 
  \url{https://twitter.com/s\_hirono/status/1394781437069467654} 
  \textgreater{}
  2021-05-19-071331\_Google Chromeの検索「電話なんでも相談室」 1件の検索結果.jpg
  \url{https://t.co/dK4eTgiBqF} 
\end{itemize}

〉〉〉 kk\_hironoのリツイート 〉〉〉

\begin{itemize}
\tightlist
\item
  RT
  kk\_hirono(刑事告発・非常上告_金沢地方検察庁御中)|s\_hirono(非常上告-最高検察庁御中\_ツイッター)
  日時:2021-05-19 07:27/2021/05/19 07:26 URL:
  \url{https://twitter.com/kk\_hirono/status/1394781823125712898} 
  \url{https://twitter.com/s\_hirono/status/1394781364415713282} 
  \textgreater{}
  2021-05-19-071046\_Google Chromeの検索「ラジオ」 7件の検索結果.jpg
  \url{https://t.co/B7f3HI2NYo} 
\end{itemize}

〉〉〉 kk\_hironoのリツイート 〉〉〉

\begin{itemize}
\tightlist
\item
  RT
  kk\_hirono(刑事告発・非常上告_金沢地方検察庁御中)|s\_hirono(非常上告-最高検察庁御中\_ツイッター)
  日時:2021-05-19 07:28/2021/05/18 23:31 URL:
  \url{https://twitter.com/kk\_hirono/status/1394782000616120321} 
  \url{https://twitter.com/s\_hirono/status/1394662026044051457} 
  \textgreater{}
  2021-05-18-204126\_電話なんでも相談室 - Google Chrome.jpg
  \url{https://t.co/L2cgjeb0oP} 
\end{itemize}

〉〉〉 kk\_hironoのリツイート 〉〉〉

\begin{itemize}
\tightlist
\item
  RT
  kk\_hirono(刑事告発・非常上告_金沢地方検察庁御中)|s\_hirono(非常上告-最高検察庁御中\_ツイッター)
  日時:2021-05-19 07:28/2021/05/18 23:31 URL:
  \url{https://twitter.com/kk\_hirono/status/1394782015933730817} 
  \url{https://twitter.com/s\_hirono/status/1394661953126105089} 
  \textgreater{}
  2021-05-18-204033\_--- FM78。0MHz ラジオかなざわ --- - Google Chrome.jpg
  \url{https://t.co/NNIxVceSTo} 
\end{itemize}

〉〉〉 kk\_hironoのリツイート 〉〉〉

\begin{itemize}
\tightlist
\item
  RT
  kk\_hirono(刑事告発・非常上告_金沢地方検察庁御中)|s\_hirono(非常上告-最高検察庁御中\_ツイッター)
  日時:2021-05-19 07:28/2021/05/18 23:31 URL:
  \url{https://twitter.com/kk\_hirono/status/1394782032409018376} 
  \url{https://twitter.com/s\_hirono/status/1394661879792881666} 
  \textgreater{}
  2021-05-18-204022\_--- FM78。0MHz ラジオかなざわ --- - Google Chrome.jpg
  \url{https://t.co/rZwSq2iE5P} 
\end{itemize}

〉〉〉 kk\_hironoのリツイート 〉〉〉

\begin{itemize}
\tightlist
\item
  RT
  kk\_hirono(刑事告発・非常上告_金沢地方検察庁御中)|s\_hirono(非常上告-最高検察庁御中\_ツイッター)
  日時:2021-05-19 07:28/2021/05/18 23:31 URL:
  \url{https://twitter.com/kk\_hirono/status/1394782048083054595} 
  \url{https://twitter.com/s\_hirono/status/1394661806581248000} 
  \textgreater{}
  2021-05-18-203817\_--- FM78。0MHz ラジオかなざわ --- - Google Chrome.jpg
  \url{https://t.co/LgWBeu2VUd} 
\end{itemize}

〉〉〉 kk\_hironoのリツイート 〉〉〉

\begin{itemize}
\item
  RT
  kk\_hirono(刑事告発・非常上告_金沢地方検察庁御中)|s\_hirono(非常上告-最高検察庁御中\_ツイッター)
  日時:2021-05-19 07:28/2021/05/18 23:30 URL:
  \url{https://twitter.com/kk\_hirono/status/1394782065686503427} 
  \url{https://twitter.com/s\_hirono/status/1394661734019784705} 
  \textgreater{}
  2021-05-18-203433\_みなさんからの質問・疑問を解決するスペシャリスト紹介月曜日法律・困り事相談長谷川 紘之(はせがわ ひろゆき) 弁護士・長谷川法律事務所西 徹.jpg
  \url{https://t.co/iGEpSwKtJF} 
\item
  ラジオかなざわ 電話なんでも相談室 - Google 検索
  \url{https://t.co/K6l4HHvhkO} 
\end{itemize}

 検索結果のトップは,訪問済みのリンクで,リンクを開いて確認しましたが,やはりさきほどの被告発人長谷川紘之弁護士の出演情報があるホームページでした。

 ラジオ番組終了の情報があるかと思って検索をしたのですが,昭和59年から平成4年4月1日まで他の仕事や市内配達の仕事をしていた期間はあるものの,長距離トラックの仕事で,深夜によく聴いていたのが,今は知らない人がほとんどかもしれないですが,いすゞ歌うヘッドライトでした。

\begin{itemize}
\item
  いすゞ歌うヘッドライト〜コックピットのあなたへ〜 - Wikipedia
  \url{https://t.co/mwAXtAPAzO}  (のちに日野ミッドナイトグラフィティ
  走れ!歌謡曲となり2021年放送終了)の成功に対抗し、走れ!歌謡曲スタートの6年後の1974年から2001年にかけてTBSラジオなどJRN系列の各放送局で放送していた音楽番組。
\item
  いすゞ 歌うヘッドライトOP~ラッシュアワー◆夜明けの仲間たち(矯正版 -
  YouTube \url{https://t.co/Qj02NZxjnv} 
\end{itemize}

 他のいすゞ
歌うヘッドライトのYouTube動画で,再生できない時間を置いてお試し下さい,などというエラーが出ていたのですが,上記の動画は再生が出来ました。そのままラジオのような音声ですが,静止画像が長い感覚で切り替わるので,再生できたのかもしれません。

 しかし,他の動画も音声のみの静止画であったように思います。30年ほど前にタイムスリップしたようで懐かしいですが,番組は平成13年に終了していたとWikipediaにあります。

\begin{itemize}
\tightlist
\item
  いすゞ 歌うヘッドライト 最終回 - YouTube \url{https://t.co/JPxJiA0JMe} 
  125,151 回視聴•2011/10/22
\end{itemize}

 ブラウザの戻るボタンをクリックすると再生が始まりました。バックにエンディングの曲が流れています。午前5時が番組の終了時間で,その前にエンディングの曲が流れていましたが,そういう時間にトラックを運転しているのは疲れもあってか,景色が幻想的に思えることもありました。

 ラジオは拘置所でもよく聴いていましたが,自分で選択できるような番組ではなかったです。FM放送というのは余り聴かなかったのですが,場所によってはAM放送の音声が乱れ,FMでは受信する音声が安定しているのでしばらく聴いていることはありました。

 4,5年前に輪島市内のホームセンターに行ったとき,ポケットサイズのラジオがあって安いのを買ってきて,海釣りに行った時にヘッドフォンで聴くことがあったのですが,雑音が多く音声が聞き取りづらいので,数回使っただけで全く使わなくなりました。

 FMラジオのイメージは音楽放送のみだったのですが,被告発人長谷川紘之弁護士が出演するような相談コーナーがあったとは,全く知らず,想像にもしませんでした。

 番組終了のお知らせがないままホームページが残っているというのが不思議なのですが,他のラジオ番組のことも知らないので,それ以上のことはわかりません。

 放送の時間帯が「月曜~木曜 14:00~15:25 土曜 15:00~16:25」と明記されているので,番組が終了していれば勘違いする視聴者もいそうですが,何事もなかったからこの状態で続いているのでしょう。なお,次がラジオかなざわ,のトップページです。

\begin{itemize}
\tightlist
\item
  --- FM78.0MHz ラジオかなざわ --- \url{https://t.co/euGc0odcrc} 
\end{itemize}

 なお,先程も書いたように個人的にFM放送というのは余り視聴しなかったのですが,市場急配センターで市内配達をしていたときは,交通情報もあるのでラジオはつけっぱなしにしていました。自分が好きな曲が聴けるカセットテープがついているのは4t車以上でした。

\begin{itemize}
\tightlist
\item
  〈〈〈 2021/05/19 08:11:28 Linux Emacs: 〈〈〈
\end{itemize}

\hypertarget{ux798fux4e95ux5211ux52d9ux6240ux3067ux5e73ux62106ux5e7411ux670811ux65e5ux53d7ux4ed8ux3068ux306aux3063ux3066ux3044ux305fux526fux672cux3068ux8d64ux3044ux5224ux5b50ux306eux3042ux308bux7532ux53f7ux8a3cux4e8cux3068ux56dbux306eux66f8ux985eux88abux544aux767aux4ebaux9577ux8c37ux5dddux7d18ux4e4bux5f01ux8b77ux58ebux304bux3089ux306eux539fux544aux5074ux63d0ux51faux66f8ux8a3cux30ceux30fcux30c8ux3068ux306eux6574ux5408ux6027}{%
\paragraph{福井刑務所で平成6年11月11日受付となっていた副本と赤い判子のある甲号証(二)と(四)の書類,被告発人長谷川紘之弁護士からの原告側提出書証,ノートとの整合性}\label{ux798fux4e95ux5211ux52d9ux6240ux3067ux5e73ux62106ux5e7411ux670811ux65e5ux53d7ux4ed8ux3068ux306aux3063ux3066ux3044ux305fux526fux672cux3068ux8d64ux3044ux5224ux5b50ux306eux3042ux308bux7532ux53f7ux8a3cux4e8cux3068ux56dbux306eux66f8ux985eux88abux544aux767aux4ebaux9577ux8c37ux5dddux7d18ux4e4bux5f01ux8b77ux58ebux304bux3089ux306eux539fux544aux5074ux63d0ux51faux66f8ux8a3cux30ceux30fcux30c8ux3068ux306eux6574ux5408ux6027}}

\begin{itemize}
\tightlist
\item
  〉〉〉 Linux Emacs: 2021/05/25 11:52:39 〉〉〉
\end{itemize}

:CATEGORIES: @kanazawabengosi \#金沢弁護士会 @JFBAsns
\#日本弁護士連合会(日弁連) \#法務省 @MOJ\_HOUMU
\#被告発人長谷川紘之弁護士 \#被告発人木梨松嗣弁護士
\#被告発人古川龍一裁判官

 2021年5月22日22時50分頃に家の中で発見した書類になります。甲号証についてネットで調べたところ原告側提出書証とあったので,これからそのように記述することにします。

 具体的なご説明を別途するところですが,本日5月25日の朝,次の写真資料をブログのまとめ記事として投稿,公開しました。この経緯がとても重要であるという判断です。まったく予想にしなかった大きな発見が含まれていました。

\begin{itemize}
\tightlist
\item
  奉納\危険生物・弁護士脳汚染除去装置\金沢地方検察庁御中\_2020:
  2021年05月25日の記録:写真資料:2021-05-23\_甲号証(二)(四),家の中から発見に至った経緯の記録
  \url{https://kk2020-09.blogspot.com/2021/05/202105252021-05-23.html\#108} 
\end{itemize}

 次の方がわかりやすいかもしれません。

\begin{itemize}
\tightlist
\item
  2021年05月25日09時16分の登録:
  2021年05月25日の記録:写真資料:2021-05-23\_甲号証(二)(四),家の中から発見に至った経緯の記録
  \url{https://kk2020-09.blogspot.com/2021/05/202105252021-05-23.html} 
\end{itemize}

 雑記帳のノートの該当箇所は,次の記録としています。

 と思ったのですが,別の目的でまとめた雑記帳ノートの初めの部分でした。被告発人松平日出男の供述調書を写経のように手書きで書き写してあるのですが,ここで必要なのは同じ雑記帳ノートの11月の記載部分です。これは写真2枚だけになるのでアルバムとしては作成していません。

 写真2枚というよりノートの2ページ分です。すでに投稿で公開済みのものがあると想うのですが,さきほどパソコンで作成したものを非常上告-最高検察庁御中\_ツイッター(@s\_hirono)にアップロードします。見開きの2ページ分ですが,きれいに撮影できていました。

〉〉〉 kk\_hironoのリツイート 〉〉〉

\begin{itemize}
\tightlist
\item
  RT
  kk\_hirono(刑事告発・非常上告_金沢地方検察庁御中)|s\_hirono(非常上告-最高検察庁御中\_ツイッター)
  日時:2021-05-25 12:31/2021/05/25 12:30 URL:
  \url{https://twitter.com/kk\_hirono/status/1397032467857481728} 
  \url{https://twitter.com/s\_hirono/status/1397032343244730373} 
  \textgreater{}
  2021-05-20\_141423_福井刑務所 ノート雑記帳_平成6年11月6日から11月14日の記載部分.jpg
  \url{https://t.co/LSjDohRh1J} 
\end{itemize}

 上記のノートの撮影部分に,左側のページの上から2つ目の赤字になりますが,11月6日(日)とあります。ここに「金曜日,210号室に転房になった。書類も入っていた。」と記載があります。

 これが本来,原告側代理人だった被告発人長谷川紘之弁護士が,金沢地方裁判所A係(被告発人古川龍一裁判官単独)を経由して福井刑務所に郵送してきた原告側提出書証になるはずなのですが,5月22日に発見した現物には11月11日受付という福井刑務所の印があります。

 ノートに,その11月11日の日付に「期日呼出,申出書」とのみ記載がありますが,期日の呼出として思い当たるのは安藤健次郎さんの証人尋問のようなものです。これは金沢地方裁判所に出頭するもので,福井刑務所に連行を願い出たのですが不許可となりました。

 次に右側のページですが,赤字の日付があるのは11月14日(日)だけです。ここに,「11日に届いたらしい一件記録の通知を受けた。再審申立書と趣意書を提出」と記載があります。

 これが私の母親を経由して被告発人木梨松嗣弁護士から郵送された一件記録となるのですが,これは1つに編綴された書面の表紙に「一件記録」とありました。まだ綴りをばらさないときに撮影した写真があるはずなのですが,まだ見つけていません。羽咋市のアパートで撮影したものと思います。

 時刻は12時50分です。今,OneDriveで2003とあるフォルダをダウンロードしているのですが,けっこう時間がかかっています。zipのアーカイブとしてダウンロードしていますが,思いの外サイズが大きく時間がかかっています。

 被告発人長谷川紘之弁護士の原告側提出書証は甲号証(二)と甲号証(四)が同じ箱にそれだけ入って見つかっているのですが,これは(一)から(五)まであったと思います。これを参考に「証拠番号5」と題した書面を作成したことがあるのですが,安藤健次郎さんに郵送した書面と思います。

 この証拠番号5は,テキストファイルとしてあちこちに保存した憶えがあるのですが,もう何年も開いて読んでいません。

 時刻は12時58分です。さきほどダウンロードが完了しました。Zipアーカイブで672.8MBとあります。

 時刻は14時27分です。昼食もしていたのですが,ダウンロードし展開したフォルダで,find
. -name '*.JPG' \textbar{} xargs mv
--target-directory=./000 というようなコマンドを実行し,写真のまとめ記事まで作成,投稿していました。

\begin{itemize}
\tightlist
\item
  2021年05月25日14時22分の登録:
  2021年05月25日の記録:写真資料:平成15年(2003年)の写真
  \url{https://kk2020-09.blogspot.com/2021/05/20210525152003.html} 
\end{itemize}

 一件記録の表紙の写真はなかったのですが,平成15年(た)第1号 決定の写真が1枚ありましいた。それもA4用紙1枚の全部が撮影されたものではなく,2枚目にあるものと推定する裁判官の名前を撮影したものがなぜかありませんでした。

 同じ写真は2,3週間ほど前にも見かけていたと思うのですが,その時の発見として重なって下にある書面が平成11年の安藤健次郎さんの傷害事件の判決の書面です。存在を確認できたかたちですが,家で現物はまだ見つかっていません。

 また,金沢検察審査会からの封筒の写真もあったのですが,中の書面を撮影した写真がありませんでした。封筒だけ撮影することはないはずなのですが,なぜか中身の書面の写真がありません。

\begin{itemize}
\item
  2003-06-02\_131158_平成15年(2003年)の写真 平成15年(た)第1号 決定 再審請求棄却.jpg
  \url{https://kk2020-09.blogspot.com/2021/05/20210525152003.html\#41} 
\item
  2003-06-02\_131706_平成15年(2003年)の写真 金沢検察審査会 封筒.jpg
  \url{https://kk2020-09.blogspot.com/2021/05/20210525152003.html\#44} 
\end{itemize}

 時刻は15時12分です。告発状のHTMLを見つけたのですが,内容が同じものがあったり,文字化けの問題もあって修正していました。最終的に,kokuhatu\_050505.html
kokuhatu\_2005\_11\_02\_2ch.html
kokuhatujyou\_050117.html という3つのHTMLファイルに絞り込みました。

 まだ少ししか目を通していないのですが,いずれも平成17年の告発状で,現在匿名化している関係者KYNと被告発人大網健二の兄である関係者OSNの2人も被告発人になっていて,表題部に住所の記載もありました。

 時刻は15時46分です。honnbunn.070308 honnbunn.070705.dat honnbunn.vim
honnbunn20070401 mailq\_060720.txt
orange-strange.txt というようなテキストファイルが見つかったので,内容のあるテキストを次の3つの記事としてブログに投稿しました。

\begin{itemize}
\tightlist
\item
  2021年05月25日15時26分の登録: No.12 廣野秀樹 さんのコメント
  \textbar{} 2007年07月05日 00:34 \textbar{} CID 65516 \textbar{}
  (Top)
  \url{https://kk2020-09.blogspot.com/2021/05/no12-20070705-0034-cid-65516-top.html} 
\item
  2021年05月25日15時34分の登録: honnbunn.070308
  \url{https://kk2020-09.blogspot.com/2021/05/honnbunn070308.html} 
\item
  2021年05月25日15時40分の登録: ■{[}日記{]}告訴状についてのコメント
  15:45 このエントリーを含むブックマーク
  \url{https://kk2020-09.blogspot.com/2021/05/1545.html} 
\end{itemize}

 コメント欄でのやり取りをテキストとして保存したファイルでしたが,スレッドの階層なのかと思いますが,やたらとインデントが繰り返され表示幅がないと折返しで見づらくなっていたので,インデントを削除して整形し直した部分があります。

\begin{itemize}
\tightlist
\item
  吉崎・次場弥生公園 - Google マップ \url{https://t.co/ZP9TX5YSps}  ¥\n
  〒925-0021 石川県羽咋市 吉崎町ウ19番地1
\end{itemize}

 OneDriveで羽咋市に住んでいた頃のファイルを探しているのですが,今度は,羽咋市内の公園で撮影した訴訟記録の撮影写真がまとまって見つかりました。ずいぶん前にも探したことがあったのですが,見つからなかったものです。Bloggerでのまとめ記事の作成に取り掛かるところです。

\begin{itemize}
\tightlist
\item
  2021年05月25日17時17分の登録:
  2021年05月25日の記録:写真資料:羽咋市内の公園(吉崎・次場弥生公園)で2007年10月6日に撮影した訴訟記録の写真(63枚)
  \url{https://kk2020-09.blogspot.com/2021/05/20210525200710663.html} 
\end{itemize}

 時刻は17時56分です。2,30分ほど前から遠くに雷が鳴り出しているのですが,急に強い風が吹き始め時刻を見ると17時55分でした。今日は宇出津新港のどんたく宇出津店に買い物に行く予定だったのですが,この様子だと諦めるほかないと考えています。天気予報は見ていません。

\begin{itemize}
\tightlist
\item
  2021年05月25日17時50分の登録:
  2021年05月25日の記録:写真資料:2003-02-27\_被告発人大網健二が事務所とした西宅建株式会社の小屋
  \url{https://kk2020-09.blogspot.com/2021/05/202105252003-02-27.html} 
\end{itemize}

 すでに何度かアップロードなどをしているNテックの写真ですが,ppコマンドのためにデータベースに新規に追加しました。住所は金沢市桂町となっていたように思います。丸西水産輸送の道路をはさんだ真向かいでした。

 ガッキーこと新垣結衣と星野源の結婚が数日前にネットで話題になっていましたが,逃げるは恥だが役に立つ,の最終回で印象に残った場面は,先日も少し取り上げたところですが,神社のバザールというより,あとで思い出した言葉で,フリーマーケットでした。

 被告発人大網健二の本陣不動産株式会社だと思いますが後輩社員が,その小屋に来て,驚く様子がなかったのも不思議だったのですが,今度どこかでフリーマーケットをするという話を被告発人大網健二にしていたのがとても印象に残っていました。前にも記述しているはずです。

 また,2003年2月27日となっている西宅建株式会社の小屋を夜に撮影した写真ですが,ずっと前の私の記憶では,宇出津のあばれ祭りのときに金沢に行き,そのときに撮影したように思うのです。第一金土になったのは平成17年だったように思うので,7月の7日か8日とあばれ祭りは決まっていました。

 一方で,平成14年である2002年の11月の終わり頃に,名古屋に移り住んでいるという被告発人大網健二に車から形態で電話をしながら金沢に向かい,西宅建株式会社の小屋というより隣にあったチェリー商事の倉庫から冬タイヤを取ってきたという記憶があり,それから余り間がなかったとも思います。

\begin{itemize}
\tightlist
\item
  2003-02-27\_172158_被告発人大網健二が事務所とした西宅建株式会社の小屋.jpg
  \url{https://kk2020-09.blogspot.com/2021/05/202105252003-02-27.html\#5} 
\end{itemize}

 上記のリンクで開かれる位置にある写真が,チェリー商事の倉庫になります。その高齢の社長のような人とは,被告発人大網健二といるときに一度会ったことがありました。被告発人大網健二が倉庫を譲り受けるような話も少し聞いていました。

 NテックとKECというプレートがありますが,このKECも被告発人大網健二がNテックを試験営業などと話しているときに,自分の故人会社の名前だと私に話し,3つの頭文字だと説明をしていましたが,健二がK以外は忘れました。なお,Nテックは能登のテクニック集団と言っていました。

 この被告発人大網健二がNテックの事務所で,ここで不動産やリフォームの契約を行うと本気で話していた西宅建株式会社の小屋ですが,被告発人大網健二は机を一つ置き,固定電話も設置していました。小屋の中はカビも生えた土間で,半分ぐらいのスペースに,西宅建の軽トラが入っていました。

 時刻は18時30分です。炊飯の用意をしていました。ちょうど前に買った5kgのお米がなくなるところで,タッパの残りを入れると2合半近くになっていました。食べた残りは冷凍保存することも考えています。

 忘れそうになっていたのですが,さきほど気になった発見がありました。チェリー商事の倉庫にあるプレートを撮影した時刻は17時21分ですが,最初の撮影も17時20分48秒。その写真も外はすっかり夜で暗くなっていますが,2月27日であれば,いくらか明るさが残っている時間に思えたのです。

〉〉〉 kk\_hironoのリツイート 〉〉〉

\begin{itemize}
\tightlist
\item
  RT
  kk\_hirono(刑事告発・非常上告_金沢地方検察庁御中)|s\_hirono(非常上告-最高検察庁御中\_ツイッター)
  日時:2021-05-25 18:44/2021/05/25 18:43 URL:
  \url{https://twitter.com/kk\_hirono/status/1397126319578570752} 
  \url{https://twitter.com/s\_hirono/status/1397126192721907715} 
  \textgreater{} 2015-02-01 17.33.13.jpg \url{https://t.co/UlTBzB7vq5} 
\end{itemize}

 1つパソコン内から写真ファイルを探し出したのですが,2015年2月1日17時33分13秒という写真です。コマンドを使って調べたところ撮影の機種はAndroidの最初に買ったスマホで,「Camera
model : HTL21」となっていました。夕焼け空が一部に見えます。2月1日です。

〉〉〉 kk\_hironoのリツイート 〉〉〉

\begin{itemize}
\tightlist
\item
  RT
  kk\_hirono(刑事告発・非常上告_金沢地方検察庁御中)|s\_hirono(非常上告-最高検察庁御中\_ツイッター)
  日時:2021-05-25 18:55/2021/05/25 18:54 URL:
  \url{https://twitter.com/kk\_hirono/status/1397129185533956096} 
  \url{https://twitter.com/s\_hirono/status/1397128957909094406} 
  \textgreater{} 2016-02-25\_17.58.07.jpg \url{https://t.co/JNYFHnH9Bs} 
\end{itemize}

〉〉〉 kk\_hironoのリツイート 〉〉〉

\begin{itemize}
\tightlist
\item
  RT
  kk\_hirono(刑事告発・非常上告_金沢地方検察庁御中)|s\_hirono(非常上告-最高検察庁御中\_ツイッター)
  日時:2021-05-25 18:55/2021/05/25 18:53 URL:
  \url{https://twitter.com/kk\_hirono/status/1397129209265278982} 
  \url{https://twitter.com/s\_hirono/status/1397128804154281992} 
  \textgreater{} 2016-02-25\_17.23.05-2.jpg \url{https://t.co/Y1x9asxqnf} 
\end{itemize}

〉〉〉 kk\_hironoのリツイート 〉〉〉

\begin{itemize}
\tightlist
\item
  RT
  kk\_hirono(刑事告発・非常上告_金沢地方検察庁御中)|s\_hirono(非常上告-最高検察庁御中\_ツイッター)
  日時:2021-05-25 18:55/2021/05/25 18:51 URL:
  \url{https://twitter.com/kk\_hirono/status/1397129220803817472} 
  \url{https://twitter.com/s\_hirono/status/1397128252095799296} 
  \textgreater{}
  2016-02-07\_17.31.11-1_能登半島宇出津港恵比寿の堤防・ヤリイカ釣り.jpg
  \url{https://t.co/7esWi3rlF7} 
\end{itemize}

〉〉〉 kk\_hironoのリツイート 〉〉〉

\begin{itemize}
\tightlist
\item
  RT
  kk\_hirono(刑事告発・非常上告_金沢地方検察庁御中)|s\_hirono(非常上告-最高検察庁御中\_ツイッター)
  日時:2021-05-25 18:55/2021/05/25 18:51 URL:
  \url{https://twitter.com/kk\_hirono/status/1397129237497204737} 
  \url{https://twitter.com/s\_hirono/status/1397128152309133313} 
  \textgreater{}
  2016-02-07\_17.25.00_能登半島宇出津港恵比寿の堤防・ヤリイカ釣り.jpg
  \url{https://t.co/fnYdkZMiVU} 
\end{itemize}

 他にも調べたのですが,2016年2月25日の写真は,17時58分でもまだ空に明るさが残っています。場所は九十九湾ですが,この場所で出会った金沢の人に,被告発人長谷川紘之弁護士が,確か肺がんで,死んだと聞きました。金沢弁護士会の会長だったとも最初に話していました。

 能登町の方が金沢市内より緯度が高いので日没は遅めのように思います。また,西に日が沈むので平地の多い金沢市内では海の方に日が沈み西日がとても眩しいということもあります。能登町は能登半島の内浦で山側に日が沈みますが,空の上の方はしばらく日没後も明るいことがあります。

 なお,西宅建株式会社の小屋のある金沢市桂町は,海の近くというか金沢港の近くです。金沢港自体が入り江の奥のような場所になりますが,海からは遮るもののない平地です。坂がありません。

\begin{itemize}
\tightlist
\item
  (株)金港倉庫 本社 - Google マップ \url{https://t.co/tO26k4KioY}  ¥\n
  〒920-0334 石川県金沢市桂町イ53
\end{itemize}

 調べたところ平成15年当時まだ西宅建株式会社の小屋があった場所は,現在,金港倉庫本社となっているようです。最初,Googleマップで全然違った場所を探していたのですが,丸西観光で検索して場所の特定が出来ました。ずいぶん細い道路に見えますが,平成11年当時は金沢港に行く主要道でした。

 金沢港の周辺も広い新しい道路がいくつも出来ているので,実際にその新しい道路を車で運転していない私には,平成4年当時の道路状況の記憶が強いだけに,とても分かりづらくなっています。金沢港の周辺はイワシの運搬やスルメイカの積み込みなどでよく行っていました。

 そういえば,撮影したカメラの確認をしていなかったですが,2003年であれば,最初に手にしたデジカメでワープロソフト一太郎のジャストシステム社が,いきなり宇出津の家に送ってきた最強のレッドキャンペーンのプレゼントというデジカメだと思います。

 今度はGoogleフォトの情報で確認しましたが撮影機種は「TOSHIBA
PDR-T15」となっていました。デジカメは時刻のズレが嫌だったのですが,日付までズレたという経験,少なくとも記憶はありません。1時間単位でもなかったと思います。10分以内の誤差です。

\begin{itemize}
\tightlist
\item
  プレスリリース (2002.11.12) \textbar{} ニュース&トピックス \textbar{}
  東芝 \url{https://t.co/vwXXLkisne} 
  「sora T15」と、ハローキティをデザインした専用のフェースパッドを同梱したハローキティ
  モデル「sora T15(KT)」*を商品化し、11月下旬より発売します。
\end{itemize}

 soraという商品名は記憶にあるものですが,平成14年11月下旬より発売とあります。最近,その平成14年の12月に撮影したという写真も見かけたというか,写真ファイルが持っている情報に気がついたということがあります。記述もしていると思います。

 デジカメは持っていても余り使うことがなかったのですが,本日見つけた羽咋市内の公園で撮影した写真でも風景を撮影した写真というのは存在せず,なぜか最後に空を撮影した写真が3枚連続で残されていました。空が気になって撮影したことは,少し心当たりがあるとも思いました。

 もう探していた一件記録の表紙の写真は見つかりそうにないですが,また1つ発見がありました。これは新たな項目として記述をして置きますが,平成25年の段階で,被告発人木梨松嗣弁護士一人を被告訴人とした告訴状の下書きがありました。LibreOfficeのodtファイルです。

 ずいぶん脱線してしまったのですが,ノートとの整合性です。

 再掲になりますが,ノートの該当箇所を撮影した写真ファイルのツイートです。

\begin{itemize}
\tightlist
\item
  TW s\_hirono(非常上告-最高検察庁御中\_ツイッター) 日時: 2021/05/25
  12:30:40 URL:
  \url{https://twitter.com/s\_hirono/status/1397032343244730373} 
  \textgreater{}
  2021-05-20\_141423_福井刑務所 ノート雑記帳_平成6年11月6日から11月14日の記載部分.jpg
  \url{https://t.co/LSjDohRh1J} 
\end{itemize}

 さきほどどこまで書いたのかよく憶えていないのですが,調べる方が手間なので最初から記述します。

 さきほど記述したことを思い出したのですが,右側のページの11月14日の一件記録の記載です。そこに11日に届いたらしい,とあるのですが,羽咋市の公園で撮影した甲号証(一)も福井刑務所の受付がその11月11日となっていました。

 残念なことに母親経由で被告発人木梨松嗣弁護士から郵送された一件記録の表紙の現物も撮影した写真ファイルも見つからないのですが,その表紙にも同じ福井刑務所の受付の印があるはずです。

 ここで気がついたのですが,twilog-serch-post
で一件記録をキーワードにしたまとめ記事を作成すれば,そこからアップロードした写真ファイルへのリンクが見つかるかもしれません。これは時期的にhatena-log-search-post
でもやってみる価値はあります。

\begin{itemize}
\item
  2021年05月25日19時54分の登録:
  「一件記録」を過去のはてなダイアリーの記事から検索
  \url{https://kk2020-09.blogspot.com/2021/05/blog-post\_25.html} 
\item
  2021年05月25日19時55分の登録:
  「一件記録」を@hirono\_hideki @kk\_hirono @s\_hironoで検索 159件の該当 2021-05-25\_19:54の記録
  \url{https://kk2020-09.blogspot.com/2021/05/hironohidekikkhironoshirono1592021-05.html} 
\item
  アルバム アーカイブ -
  一件記録・写真/H04-10-26\_事前準備書\_私選弁護人木梨松嗣弁護士
  \url{https://t.co/gdj2c7Ru3v} 
\end{itemize}

 これはまだ見つけていなかった書面の写真になりそうですが,2枚だけでした。平成4年10月26日付となっていますが,精神鑑定の請求と被告人質問の所要時間を1時間としています。最近,新たな資料を見つける前まで,被告発人木梨松嗣弁護士が精神鑑定を言い出したのは平成5年1月以降とも考えていました。

 そういえば,まだ書いていないかと思うのですが,昨日辺り,控訴審の初公判の期日が平成4年11月10日だったと書面の写真を見つけてわかりました。それまでは10月中に初公判があったと考え9月中の可能性も否定できないと考えていました。

 それというのも控訴審の法廷で被害者安藤文さんの兄が暴れたのは,初公判より後の公判で,初めて被告発人小島裕史裁判長が審理したときでもないと考えていました。前にも記述しているはずですが,そのときの天候が割合強く印象に残っているからです。

 法廷で初めて被告発人小島裕史裁判長に会ったときは夕方に近い時間帯で外は薄暗くなり法廷は照明がついていたようにも思います。当時の名古屋高裁金沢支部の法廷は同じ建物の2階で金沢地方裁判所の法廷も同じでしたが,法廷の窓側に通路があり,通路の窓から光が差し込んでいました。

 前にも同じことを書いていると思いますが,初めて安藤健次郎さんの姿を法廷で見たのも隣に座った被害者安藤文さんの兄と思われる人が一緒で,最初から興奮した様子で私を睨み付けていました。白い厚手のジャンバーを着ていたと記憶にあります。真冬のような服装です。

 それが昼過ぎか午前中のお昼近くであったと私には記憶にあるのですが,太陽が頭上近くにあるような光のさし具合であったと記憶にあるのですが,当時の金沢地方裁判所の建物の2階の法廷は,そういう外の光の差し込み具合が,他の建物にないぐらいはっきりした造りとなっていました。

 今,思い出したのが,夕方に近いその法廷をスケッチしたような場面が,アニメの「ダム・キーパー」にありました。

\begin{itemize}
\tightlist
\item
  ダム・キーパー \textbar{} キッズの動画・DVD - TSUTAYA/ツタヤ
  \url{https://t.co/19rXGSIIfF} 
\end{itemize}

 小さい画像しかないですが,DVDのカバーになっているようです。放課後の教室の描写に見えますが,これが陽射しが差し込む夕方の金沢地方裁判所の法廷や,昭和56年8月28日,いきなり金沢家庭裁判所から連れて行かれた金沢少年鑑別所の夕方の情景とずいぶん重なって思えたものです。

 ダム・キーパーの製作年が2014年とありますが,アカデミー賞の短編アニメの候補としてテレビで話題になっていたのが2月で,買ったばかりの今も使っている録画機器で情報番組の録画をしました。同じく長編アニメの候補となっていたのが「かぐや姫の物語」でした。

 また,その前の年の11月になると思うのですが,こちらも今もそのまま使っている地デジ対応の薄型テレビ19インチですが,これを宇出津の仙人町にある家電店で買いました。3万3千円ぐらいだったように思います。

 そういえば,昨夜のイチケイのカラスは,最初の頃をそのまま視聴していて,あとは台所でナスとシメジとウィンナーの炒めものを作りながら音声だけ聞いていたのですが,思った以上に面白そうなドラマに思えました。

 音声だけでしたが,万引きの他に,どちらかが記憶を失い,どちらの供述が信用できるのかという話が聞こえていました。終わりの方は,台所で作ったものを食べながら直に視聴していたのですが,書記官が疑われた痴漢の犯人が女性だったという聞いたことのない話の結末になっていました。

 そのあとも報道ステーション,ニュースzero,月曜から夜ふかし,そのあとの浜ちゃんの番組を視聴していたのですが,長い時間テレビをみたのもずいぶん久しぶりのことで,初めて見るテレビCMが多いというのも一つの感想でした。

 報道ステーションでは,テレビを背中にパソコンを見ているとき,背後に「ごとうけんじ」という声が聞こえ,どこかで聞いたことのある名前だと思ったのですが,テレビで本人の姿を見て衝撃を受けました。ジャーナリストとも呼ばれる後藤謙次氏です。それがテレビに集中するきっかけになりました。

\begin{itemize}
\tightlist
\item
  ダム・キーパー \textbar{} キッズの動画・DVD - TSUTAYA/ツタヤ
  \url{https://t.co/19rXGSIIfF}  ¥\n
  巨大なダムによって世界中を覆う大気汚染を免れている小さな街。ある少年が一人でダムを管理していたが、ある日、彼の人生を変える出会いが訪れる。堤大介、ロバート・コンドウが初監督した短編アニメーション。
\end{itemize}

 上記のツイートにあるページの引用部分は,「この作品のあらすじ・みどころ」とあるのですが,このダム・キーパーという作品との出会いが,私の弁護士脳汚染除去装置というテーマに与えた影響も大きいです。

 iTunesで動画を購入して視聴したのですが,300円ぐらいだったように思います。たぶん残高は残っていたと思うのですが,アカウントが不明になってしまい,最近になって同じiTunesから購入して視聴したのが「長崎ぶらぶら節」でした。令和3年3月31日付告発状の提出後のことです。

 コンビニに行ったところまで記憶にあるのですが,そのあとどうやってiTunesの決済をしたのか記憶にありません。Amazonプライムビデオの入会はそのあとになるのか,これも記憶が今ひとつはっきりしません。

〉〉〉 kk\_hironoのリツイート 〉〉〉

\begin{itemize}
\tightlist
\item
  RT
  kk\_hirono(刑事告発・非常上告_金沢地方検察庁御中)|s\_hirono(非常上告-最高検察庁御中\_ツイッター)
  日時:2021-05-25 20:55/2021/05/25 20:54 URL:
  \url{https://twitter.com/kk\_hirono/status/1397159407612432387} 
  \url{https://twitter.com/s\_hirono/status/1397159236824535044} 
  \textgreater{}
  2021-04-22\_150022_映画「長崎ぶらぶら節」 iTunesで購入して視聴.jpg
  \url{https://t.co/lHwwlu5qH8} 
\end{itemize}

 上記のツイートにある4月22日15時00分22秒がiTunesで「長崎ぶらぶら節」の視聴を始めてまもない時間ですが,その前に外で撮影した写真がなかったので,もしかするとコンビニには行っていないのかもしれません。

 映画「長崎ぶらぶら節」は最初の方と終わりの方に蛍の場面が出てくるのですが,ネットで公開されている動画には見ることがなかった場面で,金沢刑務所で視聴した記憶にも残っていませんでした。いずれ記述をしたいと思いますが,刑務官に電源を切られた場面ははっきり確認出来ませんでした。

 5月22日の夜の甲号証(二)と(四)の発見ですが,心当たりのある場所を探しまくった後,一応のつもりであるはずがないと思っていた場所を探しそこで見つけたことになります。これは良かったのですが,逆に言えば(一)と(三),それと一件記録の発見は可能性が遠のいたことになります。

 家の中に泥棒でも入らない限りなくなるはずはなく,私が置き場所を忘れている可能性が高いのですが,これはやはり片付けをしながら探す必要性があるのかもしれません。これまでも偶然とは思えないような発見があるので,そのうち出てくる可能性というのもあるのかもしれません。

\begin{itemize}
\tightlist
\item
  アルバム アーカイブ - 一件記録・写真/その他/新法律学辞典
  \url{https://t.co/UZZStfAaeA} 
\end{itemize}

 こんなのを撮影しアップロードしていたのかと思ったのですが,福井刑務所の私本閲読許可証がたくさん出てきて,3ヶ月の期間の始まりと終わり,それと北3−2,北3−3という独居房の番号が見えます。これは福井刑務所の北寮3階2房と同じく3房を意味します。

\begin{itemize}
\tightlist
\item
  再審請求\_金沢地方裁判所御中 \url{https://t.co/tZYpIbvZL5} 
\end{itemize}

 一時期更新していたBloggerのブログの1つで,写真ファイルを扱いやすいデザインですが,動画の音声ファイルも多いようですが,リンク切れとなっている画像もいくつか散見されますが,リンク切れとなった理由,原因は不明です。今は音声ファイルに全く関心がありません。会話の録音ですが。

\begin{itemize}
\tightlist
\item
  平成4年4月20日付被告発人・被告訴人YSKの供述調書(テキスト)
  \textbar{} 再審請求\_金沢地方裁判所御中 \url{https://t.co/Z4PrxJ5x3b} 
\end{itemize}

 被告発人安田繁克の名前をYSKと匿名にしていますが,これは他にもここ数日よく見かけているのですが,時期や原因を特定するには至っていません。時期が2009年(平成21年)9月に近ければ,大家警部補に言われたことが大きいのは確実です。

 他にも被告発人の供述調書の文字起こしのテキストデータが一通り揃っているようですが,被告発人松平日出男の平成4年4月2日付の供述調書というのは,以前に探しても見つからず,数日前になって2,3,写真ファイルなどの情報を見つけたところです。甲号証に写しがあったかと思います。

 大家警部補によるマイナスダメージというのも大きなものがあったとあらためて思い返すのですが,被害者安藤文さんのことを電話で話し始めたとき,「わしに金沢まで行ってこいちゅうが?」と言われたのを,より深く真剣に考えるべきであったとも今は思います。

 当時は,大家警部補の一番上の上司だと思っていた平成21年9月当時の石川県警本部長に対する不満がより強くありました。名前の方を忘れていますが,ネットでは最初に鹿児島県出身の情報を見て,最後に調べたときは違っていたと記憶にありますが,どこかの大学の教授とかにもなっていました。

\begin{itemize}
\tightlist
\item
  元検ブログの常連からのコメント(2008年6月24日) \textbar{}
  再審請求\_金沢地方裁判所御中 \url{https://t.co/l34Y5iHVDt} 
\end{itemize}

 令和3年3月31日付告発状でも取り上げたモトケンこと矢部善朗弁護士(京都弁護士会)のブログからの使者ジェイのコメントを取り上げた記事ですが,リンクにある画像が読み込みにとても時間のかかるフォト蔵の画像でした。

 このフォト蔵も一時期,写真やスクリーンショットの画像のアップロードしてよくつかった時期があったのですが,割と最近にも当時と変わりのないデータは閲覧しています。なんとなく愛着もあったフォト蔵ですが,他に見かけることはなく,時間が止まったような不思議な存在感があります。

 フォト蔵には1062件の画像・写真ファイルがあったのですが,忘れていたスクリーンショットの画像が少なくなかったものの一件記録の表紙の写真は残念ながら見つかりませんでした。

\begin{itemize}
\tightlist
\item
  奉納\危険生物・弁護士脳汚染除去装置\金沢地方検察庁御中\_2020:
  「一件記録」を@hirono\_hideki @kk\_hirono @s\_hironoで検索 159件の該当 2021-05-25\_19:54の記録
  \url{https://kk2020-09.blogspot.com/2021/05/hironohidekikkhironoshirono1592021-05.html} 
\end{itemize}

 ow.lyという短縮URLがのきなみリンク切れとなっているようです。

2014-07-03 13:51:28
``厚い紙の表紙のようなものに挟まれた一綴りの書面でした。表紙の部分には「一件記録」とかいう記載もあったように記憶しています。これは以前、デジカメで撮影し、インターネットでも公開していたようにも思うのですが、現在のところ私自身確認を出来ていません。''
\url{https://twitter.com/kk\_hirono/status/484560028611776512} 

〉〉〉 kk\_hironoのリツイート 〉〉〉

\begin{itemize}
\tightlist
\item
  RT
  kk\_hirono(刑事告発・非常上告_金沢地方検察庁御中)|hirono\_hideki(奉納\さらば弁護士鉄道・泥棒神社の物語)
  日時:2021-05-25 22:17/2021/05/25 22:16 URL:
  \url{https://twitter.com/kk\_hirono/status/1397179926445449220} 
  \url{https://twitter.com/hirono\_hideki/status/1397179750246879233} 
  \textgreater{} 撮影日時: 2007:09:01 \url{https://t.co/jZjklG73zT} 
\end{itemize}

 上記のツイートにあるリンクのアルバムに,本日他では見かけなかった写真がありましたが,中でも平成12年という私本閲読許可証がありました。呼称番号が87番となっているようですが,余り記憶にはない番号で,平成4年のときは57番となっていたような記憶が残っています。

 他に服役中の呼称番号として記憶にあるのは139と272です。139の方は本日も何度か写真ファイルで見かけています。87番は金沢刑務所の未決つまり拘置所での呼称番号になります。

 クラウドというネットにアップロードした写真を調べまくったのですが一件記録の表紙の写真は見つかりそうになく,たぶん同じ時にその表紙の下の1枚目にあった新宿バス放火事件の判例も一緒に撮影しているはずなのですが,そちらも見つかっていません。

\begin{itemize}
\tightlist
\item
  能登半島九十九湾にて手で捕まえた37センチのアジ\_2011年03月30日 -
  Google フォト \url{https://t.co/AObedccwvs} 
\end{itemize}

 ずいぶん前に,サイズが縮小されExif情報も消えたFacebookの投稿写真しか見つけることが出来なかったと記憶にある,「能登半島九十九湾にて手で捕まえた37センチのアジ\_2011年03月30日」の写真がGoogleフォトで見つかりました。

\begin{itemize}
\tightlist
\item
  2015年4月3日_図書館の取り寄せで借りてきた「八甲田山から還ってきた男」
  - Google フォト \url{https://t.co/965IQ4hAA6} 
\end{itemize}

 もう一つ意外な発見がありました。金沢刑務所の拘置所の官本で読んだと記憶にある福島泰蔵大尉の天上天下で始まる記載ですが,宇出津図書館で借りてきた八甲田山の本では見つけることが出来ず,気になっていた箇所になります。2015年とあるので,コンセールのとに移転する前になりそうです。

\begin{itemize}
\tightlist
\item
  八甲田山死の彷徨 (新潮文庫) \textbar{} 次郎, 新田 \textbar 本
  \textbar{} 通販 \textbar{} Amazon \url{https://t.co/y9NDiugFa8} 
\end{itemize}

 コンセールのとの図書館で借りてきて調べたのは,上記の新田次郎の本になると思います。有名な作家で,「鉄道員」を,これも平成11年の安藤健次郎さんに対する傷害事件で金沢刑務所の拘置所にいるときに読んだ記憶があるのですが,他に収録された短編の方が印象深さがありました。

\begin{itemize}
\tightlist
\item
  八甲田山から還ってきた男―雪中行軍隊長・福島大尉の生涯 (文春文庫)
  \textbar{} 高木 勉 \textbar 本 \textbar{} 通販 \textbar{} Amazon
  \url{https://t.co/hLNRxopdOI} 
\end{itemize}

 表紙には見覚えがないですが,長年気にかけてきた八甲田山の福島泰蔵大尉の本は,上記の本だったようです。個人的に一つ謎がとけて少し満足です。「八甲田山死の彷徨」を調べてあるはずのものがないと,フラストレーションを高めていました。

\begin{itemize}
\tightlist
\item
  個別「20090718222314」の写真、画像 - hirono\_hideki's fotolife
  \url{https://t.co/dec2reSJI4} 
\end{itemize}

 あまりつかっていなかったはてなフォトで見つけた2009年7月18日のUbuntuのディスクトップのスクリーンショットですが,これもウィンドウマネージャはKDEになるものと思います。どこで見つけたのか記憶にないですが,うる星やつらのラムちゃんが壁紙になっています。

 時刻は23時45分です。ご飯を炊いたまままだ夕食の支度もしていないのですが,意外な写真を見つけて調べていました。一件記録の表紙と同じく撮影自体は記憶にあったものですが,意外な場所で見つかりました。Googleフォトには見当たりませんでした。

 撮影した順序が分かりづらいという問題があるのですが,平成9年1月18日に福井刑務所を満期出所したすぐあとのノートでのメモの記載です。一月ほど軽く調べたのですがあると思っていた場所で見つかりませんでした。

 読みづらい文字で判然としなかったのですが,横にある電話番号を調べると,被告発人木梨松嗣弁護士の法律事務所と確認できました。法律事務所の名前は,パートナーと思われる相方の弁護士の苗字で2度変わっているのですが,電話番号は平成9年当時と変わっていないようです。

\begin{itemize}
\tightlist
\item
  〈〈〈 2021/05/26 00:00:00 Linux Emacs: 〈〈〈
\end{itemize}

\hypertarget{ux5e745ux670826ux65e5ux672aux660eux5927ux304dux306aux767aux898bux3068ux306aux3063ux305fux77f3ux5dddux770cux7fbdux548bux5e02ux306eux5e73ux935bux9020ux682aux5f0fux4f1aux793eux3068ux88abux544aux767aux4ebaux6728ux68a8ux677eux55e3ux5f01ux8b77ux58ebux3068ux306eux63a5ux70b9ux8a55ux8b70ux54e1ux306bux9577ux539fux609fux5f01ux8b77ux58ebux7406ux4e8bux306bux7c73ux7530ux5f18ux5e78ux5f01ux8b77ux58ebux3068ux3044ux3046ux95a2ux4fc2ux6027}{%
\paragraph{2021年5月26日未明,大きな発見となった石川県羽咋市の平鍛造株式会社と被告発人木梨松嗣弁護士との接点,評議員に長原悟弁護士,理事に米田弘幸弁護士という関係性}\label{ux5e745ux670826ux65e5ux672aux660eux5927ux304dux306aux767aux898bux3068ux306aux3063ux305fux77f3ux5dddux770cux7fbdux548bux5e02ux306eux5e73ux935bux9020ux682aux5f0fux4f1aux793eux3068ux88abux544aux767aux4ebaux6728ux68a8ux677eux55e3ux5f01ux8b77ux58ebux3068ux306eux63a5ux70b9ux8a55ux8b70ux54e1ux306bux9577ux539fux609fux5f01ux8b77ux58ebux7406ux4e8bux306bux7c73ux7530ux5f18ux5e78ux5f01ux8b77ux58ebux3068ux3044ux3046ux95a2ux4fc2ux6027}}

:CATEGORIES: @kanazawabengosi \#金沢弁護士会 @JFBAsns
\#日本弁護士連合会(日弁連) \#法務省 @MOJ\_HOUMU
\#被告発人木梨松嗣弁護士 \#市場急配センター

 5月25日から26日に日付をまたぐタイミングで,「これまで不思議と意識することがなく調べることがなかった,長原悟弁護士と米田弘幸弁護士」というようなタイトルで新規項目を作成するつもりだったのですが,調べ始めると,驚く発見に至りました。

 私は平成14年11月の下旬から平成21年3月15日まで羽咋市内の社員寮のような派遣会社のアパートに一人住まいをしていたという縁もあるのですが,その羽咋市の平鍛造という会社は,金沢市場輸送と市場急配センターの仕事で2回行ったことがあり,どちらもとても印象に残る仕事でした。

 被告発人木梨松嗣弁護士が羽咋高校の卒業生であることも昨年の12月10日過ぎ以降にネットで知ったのですが,きっかけは金沢の旧制第四高等学校の記念公園を紹介したテレビ金沢の夕方の番組でした。以前は,石川県立中央公園という名称でした。

 まず,被告発人木梨松嗣弁護士が金沢大学の卒業生であることがわかり,続いて羽咋高校の卒業生であることも検索結果から情報が出てきたのですが,金沢大学の方は確か,昭和41年の入学で昭和45年の卒業生となっていたと思います。これが年齢を推定する唯一の情報ともなっています。

 浪人したという可能性は否定できませんが,大学への入学を18歳とすると,戦後生まれで昭和23年前後の生まれと推定されます。身近に同じ年頃の生まれで元気な人がいるのですが,石川県の選挙管理委員会の委員長としてテレビで見た被告発人木梨松嗣弁護士は,その年齢より年老いて見えました。

 10年以上は前になると思いますが,ネットで被告発人木梨松嗣弁護士を調べ始めた頃から,法律事務所の名称は共同経営のように2つの名前があり,どちらが先なのかもよくわからなかったのですが,今ネットで調べても木梨と長原,木梨と米田という組み合わせがあります。

 先程も書いたように不思議と意識をすることがなく相方の弁護士について調べることがなかったのですが,被告発人木梨松嗣弁護士の同年代の古参弁護士という想像する一方で,金沢刑務所の拘置所にいるとき一緒に接見に来た新人弁護士という可能性も頭の片隅にはありました。

 正確には弁護士になる前の司法修習生だったと思いますが,たぶん被告発人木梨松嗣弁護士と一緒に接見に来た数日後に,拘置所で購読していた北國新聞の記事に,司法修習生として研修を被告発人木梨松嗣弁護士受けていることが紹介され,高松町の出身とあったことがとても印象に残っていました。

 この高松町というのは現在のかほく市ですが,平成4年から5年当時はまだまだ河北郡高松町でした。この石川県河北郡は現在もそのまま残っていて津幡町と内灘町が河北郡で,河北郡からかほく市になったのが高松市の他に宇ノ気町と七塚町になります。

 七塚町は海沿いで金沢市に近い内灘町にも隣接していると思うのですが,この旧河北郡七塚町の出身と聞いていたのが被告発人松平日出男になります。被告発人木梨松嗣弁護士と被告発人松平日出男の接点というのは確認できませんが,友人知人を介した接点は十分にあり得ることと思います。

 そのかほく市は現在も羽咋郡宝達志水町と隣接しているのですが,私がテレビで比較的近年に知ったところによると,この宝達志水町とかほく市の境が,能登地方と加賀地方の境目になるとのことです。テレビでは名前を忘れましたが,ある川を境界として紹介していました。

 この宝達志水町も以前は同じ羽咋郡として金沢に近い方が押水町,羽咋市に近い方が志雄町となっていたのですが,今も同じだと思うのですが,羽咋市を挟んだ飛び地となっている羽咋郡で,能登半島の先端に向かった北側は以前,羽咋郡で志賀町と富来町がありました。

 富来町の方が志賀町に吸収されたのですが,その志賀町に隣接したのが鳳至郡門前町で,だいたい同じ頃になるかと思いますが,その門前町も輪島市に吸収合併され現在に至ります。なお,志賀町には全国的にも知名度が高いと思われる志賀原子力発電所があります。

 私が生まれ住む能登町に比較すれば,かほく市も羽咋市も大きな町で人口も多いのですが,どちらも人口5万人台の七尾市よりは,町の規模も小さく人口も少なかったと思います。ちょっと調べて確認をしておきます。

\begin{itemize}
\tightlist
\item
  羽咋市(はくいし)住民人口/羽咋市公式ホームページ
  \url{https://t.co/5pnsVpuxg3}  2021年5月1日現在の人口に関するデータです。
  ¥\n ¥\n 総人口 ¥\n 20,753人 (うち 外国人 149人)
\end{itemize}

 2,3日前,イカの駅,つくモール,イカのモニュメントでTwitterの検索をしていたときの情報で能登町は人口が現在1万6千人となっていましたが,能登町の発足当時は2万人弱となっていたように思います。羽咋市の方も人口減があるのかわからないですが,意外に人口差がないと思いました。

 羽咋市の場合は,平成14年当時にもイオンのマックスバリュという24時間営業のスーパーがあったのですが,富山県の氷見市の方からも来客が多いと聞いたことがありました。数年前に羽咋市よりは近いかもしれない同じ富山県の小矢部市に大きなショッピングモールが出来ています。

 人口が20,753人というのはとても少ないと思われそうですが,個人的な感覚としては5年ちょっと生活をしていた体験でけっこう大きな町という印象が未だに残っています。

 なお,人口1万6千人という能登町の中でもより狭い範囲の私が生まれ住む宇出津でも,まったく見知らぬ人が多いものだと近年になって実感するというこれも実体験があります。

 平鍛造については,さきほどtwilog-serch-post
のまとめ記事を作成していますが,該当は思ったより少ないという結果でした。まだ投稿されたページを開いていませんが,記憶にあるのはイワシの運搬の運転手総出で小松市のコマニーまで工場の備品のようなものを運んだこと。

 それと平成3年の12月の前半と思いますが,城西運輸の仕事で,その羽咋市内の平鍛造の工場から山形県天童市辺りに荷物を運んだことです。工業団地のような場所で,近くにある自衛隊の基地で道を尋ねました。

 小松市のコマニーの方は,他にも製品を積み込みに行ったことがあったのですが,その主力製品というのが間仕切りで,市場急配センターの2階事務所でもトイレ・台所の仕切りとして印象深かったのがその間仕切りになります。

\begin{quote}
《引用の始まり》
\end{quote}

\begin{quote}
役員一覧

顧問顧問 竹中 博康 石川県副知事

評議員評議員 稲村 建男 石川県議会議員評議員 宮村 栄一
学校法人金沢医科大学理事評議員 鈴木 泰信 技術士(金属部門)評議員
柳生 好春 株式会社 日本エルデイアイ代表取締役・教育評論家評議員
長原 悟 弁護士 木梨・長原法律事務所評議員 福島 隆太
羽咋丸善株式会社 専務取締役

役員代表理事 平 美都江 羽昨丸善株式会社 代表取締役社長理事 米田 弘幸
弁護士 木梨・長原法律事務所理事 前田 ユキ子 ボランティア活動家理事
横井 敏弘 弁理士 みさき国際特許事務所理事 若山 雄一郎
税理士 税理士法人山田&パートナーズ金沢事務所理事 佐藤 未史
羽昨丸善株式会社 取締役監事 島田 二郎 税理士 島田二郎税理士事務所監事
沖野 陽平 弁護士 沖野陽平法律事務所
\end{quote}

\begin{quote}
《引用の終わり》
\end{quote}

\begin{itemize}
\tightlist
\item
  役員一覧 \textbar{} 公益財団法人
  平昭七記念財団 \url{http://taira-zaidan.jp/member.htmln} 
\end{itemize}

 長原悟弁護士の検索結果でしたが,「公益財団法人
平昭七記念財団」というページが出てきたときは理解が出来ず,満蒙開拓団の歴史に近いものを想定しました。平鍛造というのは,そのまま人物の名前と考えていたこともあったのですが,鍛造というのは鉄鋼業の技術のことのようでした。

\begin{itemize}
\tightlist
\item
  〈〈〈 2021/05/26 02:06:29 Linux Emacs: 〈〈〈
\end{itemize}

\hypertarget{ux88abux544aux767aux4ebaux53e4ux5dddux9f8dux4e00ux88c1ux5224ux5b98}{%
\subsubsection{被告発人古川龍一裁判官}\label{ux88abux544aux767aux4ebaux53e4ux5dddux9f8dux4e00ux88c1ux5224ux5b98}}

\hypertarget{ux88abux544aux767aux4ebaux53e4ux5dddux9f8dux4e00ux88c1ux5224ux5b98ux306eux95a2ux4e0eux672cux4ef6ux6bbaux4ebaux672aux9042ux4e8bux4ef6ux306bux4e0eux3048ux305fux5f71ux97ffux88abux544aux767aux4ebaux53e4ux5dddux9f8dux4e00ux88c1ux5224ux5b98ux306eux7d4cux6b74}{%
\paragraph{被告発人古川龍一裁判官の関与,本件殺人未遂事件に与えた影響:被告発人古川龍一裁判官の経歴}\label{ux88abux544aux767aux4ebaux53e4ux5dddux9f8dux4e00ux88c1ux5224ux5b98ux306eux95a2ux4e0eux672cux4ef6ux6bbaux4ebaux672aux9042ux4e8bux4ef6ux306bux4e0eux3048ux305fux5f71ux97ffux88abux544aux767aux4ebaux53e4ux5dddux9f8dux4e00ux88c1ux5224ux5b98ux306eux7d4cux6b74}}

\begin{itemize}
\tightlist
\item
  〉〉〉 Linux Emacs: 2021/05/19 08:40:57 〉〉〉
\end{itemize}

:CATEGORIES: @kanazawabengosi \#金沢弁護士会 @JFBAsns
日本弁護士連合会(日弁連) \#法務省 @MOJ\_HOUMU \#被告発人古川龍一裁判官
\#被告発人長谷川紘之弁護士

\begin{quote}
《引用の始まり》
\end{quote}

\begin{quote}
生年月日 S27.6.6出身大学 早稲田大退官時の年齢 48 歳H13.4.24
依願退官H9.4.1 ~ H13.4.23 福岡高裁2刑判事(H13.3.30戒告)H6.4.13 ~
H9.3.31 金沢地家裁判事H4.4.1 ~ H6.4.12 金沢地家裁判事補H2.4.1 ~
H4.3.31 最高裁刑事局付H1.4.1 ~ H2.3.31 東京海上火災保険(研修)H1.3.27
~ H1.3.31 東京地裁判事補S61.4.1 ~ H1.3.26
青森地家裁弘前支部判事補S59.4.13 ~ S61.3.31 東京地裁判事補
\end{quote}

\begin{quote}
《引用の終わり》
\end{quote}

\begin{itemize}
\tightlist
\item
  古川龍一裁判官(36期)の経歴 \textbar{}
  弁護士山中理司のブログ \url{https://yamanaka-bengoshi.jp/2019/03/31/hurukawa36/n} 
\end{itemize}

 今朝は5時30分頃に目が覚めそのまま起きているのですが,起きてしばらくしてから上記のページを開こうとすると,読み込みに長い時間が掛かった後,接続を拒否されたというようなエラーが出ていました。

 裁判官の名前で検索すると上位に出てくることの多いブログで,ずいぶん前から利用させてもらっているのですが,昨日,他の裁判官のページを表示したままタブにあり,ページにあるブログ内検索で被告発人古川龍一裁判官を検索をしたところ,上記のエラーが出ていたのです。

 Firefoxのブラウザでも試していたのですが検索を実行からリンクのページの読み込みが始まったまま放置し,Google
Chromeのブラウザで他のページを閲覧していたところ,どれぐらい時間があったのかわからないですが,ブラウザをFirefoxに切り替えした時点で,ページが表示されていました。

 この被告発人古川龍一裁判官の経歴のページは,これまでに何度も閲覧しているのですが,ここでも新たな気が付きがありました。

 平成4年4月1日から平成6年4月12日まで金沢地家裁判事補で,続けて平成6年4月13日から平成9年3月31日まで金沢地家裁判事となっています。判事補の特例があることは知っていて一通り情報を読んでいるのですが,単独で裁判が出来たのか憶えていません。

\begin{quote}
《引用の始まり》
\end{quote}

\begin{quote}
依頼者の方から「担当裁判官お若い方ですね」と言われることがあります。 私自身、民事事件の単独事件の担当をはじめたのは29歳の時です。当然判事補でした。実際、若いですね。

 昭和23年に定められた「判事補の職権の特例等に関する法律」という法律があります。 1条に「判事補で裁判所法42条1項各号に掲げる職の1または2以上にあつて、その年数を通算して5年以上になる者のうち、最高裁判所の指名する者は、当分の間、判事補としての職権の制限を受けないものとし、同法第29条3項(同法第31条の5で準用する場合を含む)及び36条の規定の適用については、その属する地方裁判所又は家庭裁判所の判事の権限を有するものとする」。 指名された判事補は「特例判事補」といわれます。

 判事補任官5年経過すれば(6年目から)、留学中や出向中の裁判官を除き、最高裁判所は全員を特例判事補に指名します。留学中や出向中の判事補は、帰国したり、出向先から戻ったときに、最高裁判所から、特例判事補に指名されます。

 何のことはない、判事補10年と定まっていますが、実質的に判事補5年で判事の仕事をすることになります。 なお「当分の間」とありますが、60年以上続いています。常識的に考えて経験が5年あれば単独はできますね。 制限は、高等裁判所(特殊事件を除き3名の合議体)の裁判長になれないこと、3人の裁判官のうち2名同時に入れないことくらいです。
\end{quote}

\begin{quote}
《引用の終わり》
\end{quote}

\begin{itemize}
\tightlist
\item
  特例判事補 \url{https://www.nishino-law.com/publics/index/45/detail=1/b\_id=77/r\_id=651/n} 
\end{itemize}

 西野法律事務所とありますが,わかりやすい説明がみつかりました。判事補の経験5年で単独で裁判ができるとあります。

 次に,裁判官としての初任地が昭和59年4月13日からの東京地裁ですが,次の任地が「S61.4.1
~ H1.3.26
青森地家裁弘前支部判事補」となっています。余り重要なことでないかもしれないですが,モトケンこと矢部善朗弁護士(京都弁護士会)の任地と重なる可能性はありそうです。

 青森県の弘前市も東北自動車道の通過だけではなく,一度,北九州市行きのリンゴを積みに行ったことがあるのですが,市内の中心部ではなく,のどかな風景で一昔前にタイムスリップしたような感覚が印象に残っています。

 たぶん,調べてみると弘前市も人工が多そうで20万人はいそうに予想します。

 Googleの検索で17.74万
(2015年)という結果でした。人口密度はわからないですが,モトケンこと矢部善朗弁護士(京都弁護士会)も山菜採りに行き,山で遭難しそうになったとかブログで書きながら,のんびりした生活をしていたような様子でした。

 ひところ,判検交流というのが一部の弁護士の批判としてネットの話題になっていたのですが,ネットで裁判所の支部の建物をみていてもこじんまりとしていて,人間関係の密度も高まりそうに思えていました。まあ,なかにはかえって険悪な間柄になるということもあるのでしょう。

 いちおうの可能性ですが,ただ,被告発人古川龍一裁判官とモトケンこと矢部善朗弁護士(京都弁護士会)の組み合わせというのも意外なものを感じました。私が知らないだけの法律家同士の接点というのはいくつかあるのだと予想はしていますし,親戚関係もありうることです。

\begin{itemize}
\tightlist
\item
  〈〈〈 2021/05/19 09:19:00 Linux Emacs: 〈〈〈
\end{itemize}

\hypertarget{ux88abux544aux767aux4ebaux53e4ux5dddux9f8dux4e00ux88c1ux5224ux5b98ux306eux95a2ux4e0eux672cux4ef6ux6bbaux4ebaux672aux9042ux4e8bux4ef6ux306bux4e0eux3048ux305fux5f71ux97ffux798fux4e95ux5211ux52d9ux6240ux3067ux306eux51faux5f35ux5c0bux554fux59ffux3082ux5c0bux554fux3082ux306aux304bux3063ux305fux539fux544aux8a34ux8a1fux4ee3ux7406ux4ebaux306eux88abux544aux767aux4ebaux9577ux8c37ux5dddux7d18ux4e4bux5f01ux8b77ux58eb}{%
\paragraph{被告発人古川龍一裁判官の関与,本件殺人未遂事件に与えた影響:福井刑務所での出張尋問,姿も尋問もなかった原告訴訟代理人の被告発人長谷川紘之弁護士}\label{ux88abux544aux767aux4ebaux53e4ux5dddux9f8dux4e00ux88c1ux5224ux5b98ux306eux95a2ux4e0eux672cux4ef6ux6bbaux4ebaux672aux9042ux4e8bux4ef6ux306bux4e0eux3048ux305fux5f71ux97ffux798fux4e95ux5211ux52d9ux6240ux3067ux306eux51faux5f35ux5c0bux554fux59ffux3082ux5c0bux554fux3082ux306aux304bux3063ux305fux539fux544aux8a34ux8a1fux4ee3ux7406ux4ebaux306eux88abux544aux767aux4ebaux9577ux8c37ux5dddux7d18ux4e4bux5f01ux8b77ux58eb}}

\begin{itemize}
\tightlist
\item
  〉〉〉 Linux Emacs: 2021/05/19 09:22:35 〉〉〉
\end{itemize}

:CATEGORIES: @kanazawabengosi \#金沢弁護士会 @JFBAsns
日本弁護士連合会(日弁連) \#法務省 @MOJ\_HOUMU
\#被告発人長谷川紘之弁護士 \#被告発人古川龍一裁判官

 そういえば,昨日,はてなブログでPicasaウェブアルバムのまとめ記事を見つけたのですが,記事のタイトルが分かりづらいものでした。今からページを探し出し,HTMLをコピーして,Bloggerのブログに転載をしておきたいと思います。

 URLがリンクになっていなかったので,次のようにリンクを作成したのですが,久しぶりに込み入った正規表現を使ったので,多少手間取りました。

\begin{lstlisting}
py37_env ❯ cat x|sed -E 's%(https?:\/\/[^<]+?)%<a target="_blank" href="\1">\1</a>%gm' |xsel -ib
\end{lstlisting}

\begin{itemize}
\item
  2021年05月19日09時31分の登録: Picasa ウェブ
  アルバムに登録済みの再審請求に関する資料リスト(2011/11/18現在)
  \url{https://kk2020-09.blogspot.com/2021/05/picasa-20111118.html} 
\item
  アルバム アーカイブ -
  平成6年7月5日付民事訴状(原告訴訟代理人長谷川紘之弁護士作成)(写真 6
  枚) \url{https://t.co/a5p7nda3sb} 
\end{itemize}

 昨日気にかけていた被告発人長谷川紘之弁護士の訴状が見つかりました。平成6年7月5日付とありました。私の手元に来たのはその年の11月に入ってからです。訴訟記録の房内所持は許可が必要ですが,その条件というのが雑居房から独居房への転房でした。

 時刻は10時33分です。家の中でゴミを置くような場所に長年放置していた福井刑務所でのノートを引っ張り出し,外でホコリを落としていました。UNIXの勉強をした本と,どこに置いたのかわからなくなっていた「犯罪精神医学」の本も見つかりました。

 細かい雨が当たって,いつでも本降りになりそうな空模様ですが,昨日の夕方も小雨の中,宇出津新港に買い物に行き,帰りはけっこうな本降りとなって濡れて帰りました。水曜日と勘違いしたまま出掛け,どんたく宇出津店でレジを済ませた後になって気が付きました。

 その前に店内で,本日ポイント5倍という放送があったので,毎週火曜日から水曜日に変更になったのかと勘違いしていたのですが,レジの後,買い物袋に詰め込んでいたときは,間違って放送が流れたのかとも考えていました。

 時刻は11時38分です。洗濯をしていました。外は雨が降り出していました。ノートの一部を写真撮影し,Twitterにアップロードしたのですが,いくつか確認できた事実経過がありました。簡単すぎてもう少し書けなかったのかと不満もあるのですが,目立ってわかりやすいとも思いました。

 さきに被告発人古川龍一裁判官の出張尋問のことですが,期日を指定した書面などアップロードしたファイルがあるように思います。

 あるはずだと思うのですが見つかりませんでした。被告発人古川龍一裁判官の出張尋問には予納金とかが必要ということで,1万2千円ぐらいの支払いをしたと記憶にあり,その書面ははっきり写真撮影していた記憶があります。

 被告発人古川龍一裁判官の民事裁判の判決が平成7年7月27日であることは今,パソコンにあるファイルで確認しました。出張尋問は4月から6月の間で暖かくなった季節という記憶はあります。少しだけ福井刑務所の建物から出て,隣の建物に歩いて移動しました。

 刑務所と言ってもよくトラックの仕事でいった配送センターのような建物でした。たぶん事務棟になるのとは思います。小さい建物に2階で被告発人古川龍一裁判官の出張尋問を受けました。周りにけっこう人がいたのですが,話をしたのは私と被告発人古川龍一裁判官だけだったと思います。

 極端に頭が禿げ上がったり,刑務官以外に帽子をかぶった人はいなかったと思うので被告発人長谷川紘之弁護士の姿はなかったはずですが,少なくとも被告発人古川龍一裁判官以外の尋問というのはありませんでした。

\begin{itemize}
\tightlist
\item
  2021年02月19日14時49分の登録:
  「古川龍一(裁判官)?」を@hirono\_hideki @kk\_hirono @s\_hironoで検索
  \url{https://kk2020-09.blogspot.com/2021/02/hironohidekikkhironoshirono\_34.html} 
\end{itemize}

 2月19日に作成したtwilog-serch-post
のまとめ記事がありましたが,まだワンライナーで実行していた頃かもしれません。新たにまとめ記事を作成しても代わり映えはないと思います。

 被告発人古川龍一裁判官については,あまり重要視せず,思い出したり考えたりしない時期も多かったので,古い記憶は早い段階で抜け落ちていったという自覚がありました。手書きで作成していた福井刑務所時代の書面であれば,臨場感のある具体性があったかもしれません。

 前にも書いているはずですが,私がしばらく前に郵送で提出した書面を被告発人古川龍一裁判官は私の前で,判決文を読み上げるように読んでいました。自分で作成した書面を裁判官が読みあがるのは,妙な体験ではありました。

 質問された内容も答えた内容も思い出せないですが,割と好意的な対応をされているようには感じ,法廷とは全然違った雰囲気で,かなり近い距離で差し向かいで話をしていたと記憶にあります。

 また,その前に金沢地方裁判所の法廷で安藤健次郎さんの証人尋問のような通知が届いていたのですが,福井刑務所に出廷を願い出たものの不許可となっていました。他の受刑者だったのか民事裁判ではほとんど許可されないという話も聞いたように思います。

 その安藤健次郎さんの尋問の内容というのも私には伝わっていなかったように思います。すべて金沢西警察署で作成された供述調書で事実認定がされた感じでした。さきほど被告発人古川龍一裁判官の判決文が入ったフォルダを見たのですが,思ったより多い枚数がありました。

 もともと金沢西警察署のずさんな捜査の落ち度を認め,いずれ再審請求になることを予想していたのかもしれないですが,民事裁判と同じ頃に,被告発人古川龍一裁判官が合議体の1人になった刑事の再審請求でも請求棄却の決定書が届いていたと記憶にあります。

 もともと被告発人古川龍一裁判官の名前は,再審請求で見ていたような気もするのですが,私の再審請求は全て同じパターンで,再審請求書を提出した後,しばらくして求意見書が届き,意見書を提出してから上申書を何度か送り,そのうちいきなり再審請求棄却の決定書が届いていました。

 ネットで再審請求の話題は,最近でこそ少なくなりましたが,大崎事件があと一歩で再審開始になる頃までは注目度も高く情報量も多かったと思います。それが最高裁で棄却され,しばらくの間は批判が沸き起こっていましたが,その後は沈静化し,下火になったという印象です。

 よほど関心が薄いのか,Twitterで再審を検索しても古いツイートが出たり,全体的に数が少ないと思います。その下火になる前でも,求意見書というのは,ほとんど見かけた憶えがありません。私自身もしばらく使っていませんでした。

 2021年05月19日12時24分の実行記録:
twitterAPI-search-lawList-mydql-add.rb ``求意見書'' ツイート数:2/2420
リツイート数:0/2420 トータル:2 ¥\n ``求意見書''の該当: hirono\_hideki
0/0件 kk\_hirono 2/0件 s\_hirono 0/0件

\begin{itemize}
\tightlist
\item
  2021年05月19日12時25分の登録:
  REGEXP:''求意見書''/データベース登録済みツイートの検索:2021-05-19〜2021-05-19/2021年05月19日12時25分の記録:ユーザ・投稿:1/1件
  \url{https://kk2020-09.blogspot.com/2021/05/regexp2021-05-192021-05-1920210519122511.html} 
\end{itemize}

 つい先程のツイートが他に2つはデータベースに入っているはずなのですが,まとめ記事に出たのは1件だけでした。twitterAPI-search-lawList-mydql-add.rb
の実行結果のツイートと思ったのですが,それが違っていました。rの実行前かもしれません。

 twilog-serch-post とhatena-log-search-post
で求意見書をまとめ記事にしますが,hatena-log-search-post
はけっこう数があると思います。求意見書を主要なテーマにしていた時期があります。

 時刻は13時07分です。個人的に大きく感じる2つがありましたが,渡辺輝人弁護士がイマドキの弁護士らしい,ツイートを安倍晋三元首相のツイートに向けています。すぐ忘れるようですが,この渡辺輝人弁護士も自由法曹団常任幹事とプロフィールにあります。

〉〉〉 kk\_hironoのリツイート 〉〉〉

\begin{itemize}
\tightlist
\item
  RT
  kk\_hirono(刑事告発・非常上告_金沢地方検察庁御中)|nabeteru1Q78(渡辺輝人)
  日時:2021-05-19 13:09/2021/05/18 18:38 URL:
  \url{https://twitter.com/kk\_hirono/status/1394867869498765316} 
  \url{https://twitter.com/nabeteru1Q78/status/1394588184986591237} 
  \textgreater{}
  あなたは報道機関の正当な取材行為に対して干渉する前に、モリカケ桜、財務省の資料隠滅、桜を見る会前夜祭の有権者買収、ジャパンライフの詐欺事件で広告塔を果たした件、数々の疑惑の説明が先。
  \url{https://t.co/vPhlhTuNIb} 
\end{itemize}

〉〉〉 kk\_hironoのリツイート 〉〉〉

\begin{itemize}
\tightlist
\item
  RT
  kk\_hirono(刑事告発・非常上告_金沢地方検察庁御中)|AbeShinzo(安倍晋三)
  日時:2021-05-19 13:10/2021/05/18 17:36 URL:
  \url{https://twitter.com/kk\_hirono/status/1394868027699597314} 
  \url{https://twitter.com/AbeShinzo/status/1394572582058217478} 
  \textgreater{} 朝日、毎日は極めて悪質な妨害愉快犯と言える。
  防衛省の抗議に両社がどう答えるか注目。 \url{https://t.co/Q0ofuKioNQ} 
\end{itemize}

〉〉〉 kk\_hironoのリツイート 〉〉〉

\begin{itemize}
\tightlist
\item
  RT
  kk\_hirono(刑事告発・非常上告_金沢地方検察庁御中)|KishiNobuo(岸信夫)
  日時:2021-05-19 13:10/2021/05/18 08:49 URL:
  \url{https://twitter.com/kk\_hirono/status/1394868073853702149} 
  \url{https://twitter.com/KishiNobuo/status/1394440062125805572} 
  \textgreater{} 自衛隊大規模接種センター予約の報道について。
  今回、朝日新聞出版AERAドット及び毎日新聞の記者が不正な手段により予約を実施した行為は、本来のワクチン接種を希望する65歳以上の方の接種機会を奪い、貴重なワクチンそのものが無駄になりかねない極めて悪質な行為です。
\end{itemize}

 AERAドットというニュースサイトのようなものは見覚えがあるのですが,朝日新聞出版とは知らなかったように思います。もう1つのTwitterのトレンドで見かけたニュースですが,元県議の事務局長を逮捕とあります。始まりに従軍慰安婦問題があったように思うのですが,人の人生が狂ったかも。

〉〉〉 kk\_hironoのリツイート 〉〉〉

\begin{itemize}
\item
  RT
  kk\_hirono(刑事告発・非常上告_金沢地方検察庁御中)|asahicom(朝日新聞デジタル)
  日時:2021-05-19 13:15/2021/05/19 08:02 URL:
  \url{https://twitter.com/kk\_hirono/status/1394869174933942279} 
  \url{https://twitter.com/asahicom/status/1394790614819119104} 
  \textgreater{} リコール不正、元県議の事務局長を逮捕 署名偽造容疑
  \url{https://t.co/Xllf9JoSgU} 
  愛知県の大村秀章知事へのリコール署名偽造事件で、愛知県警は19日、運動団体事務局長で元県議の田中孝博容疑者(59)を地方自治法違反(署名偽造)の疑いで逮捕。事務局関係者ら数人も同容疑で逮捕する方針。
  \url{https://t.co/VR57bFbRBi} 
\item
  2021年05月19日12時30分の登録:
  「求意見書」を@hirono\_hideki @kk\_hirono @s\_hironoで検索 22件の該当 2021-05-19\_12:30の記録
  \url{https://kk2020-09.blogspot.com/2021/05/hironohidekikkhironoshirono222021-05\_19.html} 
\item
  2021年05月19日12時31分の登録:
  「求意見書」を過去のはてなダイアリーの記事から検索
  \url{https://kk2020-09.blogspot.com/2021/05/blog-post\_19.html} 
\item
  2021年05月19日12時34分の登録:
  \小倉秀夫 @chosakukenho\1年目に,岩田合同の大先生と某事件を共同受任させていただき、勉強させていただきました。
  \url{https://kk2020-09.blogspot.com/2021/05/chosakukenho\_19.html} 
\item
  2021年05月19日12時47分の登録:
  「岩田宙造」を@hirono\_hideki @kk\_hirono @s\_hironoで検索 1件の該当 2021-05-19\_12:47の記録
  \url{https://kk2020-09.blogspot.com/2021/05/hironohidekikkhironoshirono12021-05.html} 
\end{itemize}

 小倉秀夫弁護士のツイートがきっかけで,岩田合同法律事務所や岩田宙造弁護士のことを知ったのですが,即身仏のような立派な銅像があり,紹介文に巨星とありました。私が今まで見てきたところ,合同法律事務所で,自由法曹団の関係以外は見たことがないのですが,未確認です。

 岩田合同法律事務所は弁護士の数も多かったのですが,1人として見覚えのある名前はなく,弁護士業界の広さを感じました。各弁護士はカードのようなリンクになっていて,リンクのページを開くと顔写真がありましたが,いずれも白黒写真になっていたのもこだわりというか流儀の強さを感じました。

 「
「求意見書」を過去のはてなダイアリーの記事から検索」にざっと目を通しましたが,だいたい記憶にあるような内容で,余り詳細にも具体的にも書いていないと感じました。告訴状や告発状の本文に力を入れていた時期かもしれません。再審請求は平成15年で終わっています。

 この平成15年の再審請求の決定書も家の中で探し出さなければいけないのですが,まったく予想外だった平成14年(た)1号の同じく再審請求の棄却決定書が見つかっています。それが伊東一廣裁判長でした。名前だけは新聞で見たような記憶だったのですが,平成15年も同じ可能性が高いです。

\begin{itemize}
\tightlist
\item
  〈〈〈 2021/05/19 13:28:44 Linux Emacs: 〈〈〈
\end{itemize}

\hypertarget{ux88abux544aux767aux4ebaux53e4ux5dddux9f8dux4e00ux88c1ux5224ux5b98ux306eux95a2ux4e0eux672cux4ef6ux6bbaux4ebaux672aux9042ux4e8bux4ef6ux306bux4e0eux3048ux305fux5f71ux97ff}{%
\paragraph{被告発人古川龍一裁判官の関与,本件殺人未遂事件に与えた影響:}\label{ux88abux544aux767aux4ebaux53e4ux5dddux9f8dux4e00ux88c1ux5224ux5b98ux306eux95a2ux4e0eux672cux4ef6ux6bbaux4ebaux672aux9042ux4e8bux4ef6ux306bux4e0eux3048ux305fux5f71ux97ff}}

\hypertarget{section-9}{%
\paragraph{}\label{section-9}}

\hypertarget{ux88abux544aux767aux4ebaux82e5ux6749ux5e78ux5e73ux5f01ux8b77ux58eb}{%
\subsubsection{被告発人若杉幸平弁護士}\label{ux88abux544aux767aux4ebaux82e5ux6749ux5e78ux5e73ux5f01ux8b77ux58eb}}

\hypertarget{ux5e7405ux670807ux65e5ux5927ux6d25ux5712ux5150ux6b7bux4ea1ux4e8bux6545ux306eux88abux5bb3ux8005ux5f01ux8b77ux56e3ux306eux30cbux30e5ux30fcux30b9ux3067ux601dux3044ux51faux3057ux305fux88abux544aux767aux4ebaux82e5ux6749ux5e78ux5e73ux5f01ux8b77ux58ebux306eux5e73ux62106ux5e747ux670820ux65e5ux4ed8ux4ebaux6a29ux6551ux6e08ux7533ux7acbux56deux7b54ux66f8ux3068ux691cux5bdfux5be9ux67fbux4f1a}{%
\paragraph{2021年05月07日,大津園児死亡事故の被害者弁護団のニュースで思い出した,被告発人若杉幸平弁護士の平成6年7月20日付,人権救済申立回答書と検察審査会}\label{ux5e7405ux670807ux65e5ux5927ux6d25ux5712ux5150ux6b7bux4ea1ux4e8bux6545ux306eux88abux5bb3ux8005ux5f01ux8b77ux56e3ux306eux30cbux30e5ux30fcux30b9ux3067ux601dux3044ux51faux3057ux305fux88abux544aux767aux4ebaux82e5ux6749ux5e78ux5e73ux5f01ux8b77ux58ebux306eux5e73ux62106ux5e747ux670820ux65e5ux4ed8ux4ebaux6a29ux6551ux6e08ux7533ux7acbux56deux7b54ux66f8ux3068ux691cux5bdfux5be9ux67fbux4f1a}}

\begin{itemize}
\tightlist
\item
  〉〉〉 Linux Emacs: 2021/05/07 14:44:36 〉〉〉
\end{itemize}

:CATEGORIES: @kanazawabengosi \#金沢弁護士会 @JFBAsns
日本弁護士連合会(日弁連) \#法務省 @MOJ\_HOUMU \#被告発人若杉幸平弁護士
\#検察審査会 \#大津園児死亡事故

\begin{itemize}
\tightlist
\item
  1346:2021-05-07\_14:07:45 \#告発状 \#\#\#\#
  櫻井光政弁護士のテレビドラマ「イチケイのカラス」に関するツイートがきっかけで間違いに気がついた左陪席の位置
  \url{https://hirono-hideki.hatenadiary.jp/entry/2021/05/07/140742} 
\end{itemize}

 上記のエントリーで針路が新潟小2女児殺害事件のことを調べる方向に振れ,新潟方面からのぼる朝日の写真を探したのですが,AmazonPhotoからダウンロードした写真が数枚あって,検察審査会のポスターの写真があり,そのすぐあとに見つけたのが大津園児死亡事故の新着ニュースでした。

 埋め込みツイートのこともあり,別のエントリーとしてTwitterにアップロードし掲載するつもりでいたのですが,まずはその写真をアップロードします。さきほど,人権救済の回答書の写真を追加しましたが,jpegとなっていた拡張子を変更し,ファイル名も変えています。

\begin{itemize}
\item
  2002-12-29\_163215_人権救済平成6年.jpg \url{https://t.co/tjlC6jtfzD} 
\item
  2020-11-22\_064422_.jpg \url{https://t.co/DQYtyskJQZ} 
\item
  2020-11-22\_105848_.jpg \url{https://t.co/dZNatFmZo5} 
\item
  2020-11-22\_114920_.jpg \url{https://t.co/r31Fj62D69} 
\item
  2020-11-22\_131925_.jpg \url{https://t.co/UByYYnBW6h} 
\item
  2020-11-22\_164908_.jpg \url{https://t.co/Mejw1UNTht} 
\item
  2020-11-22\_173631_.jpg \url{https://t.co/R37PfvftKQ} 
\end{itemize}

 撮影時刻以外の情報をファイル名につけるのを忘れていたのですが,この撮影時刻というのは写真ファイルにあるExif情報から取得した時刻をプログラムで置き換えたファイル名になります。人権救済平成6年の撮影日時が2002年12月だったのは意外でした。

 2002年というのは平成14年になりますが,たぶんこれが初めてのデジカメで,ワープロソフト一太郎のジャストシステム社からいきなり郵送されてきたデジカメで,最強のレッドキャンペーンとなっていたように記憶にあります。

 デジカメが送られてきた時期は秋で11月頃だったとも思うのですが,平成15年のことかと思っていました。それから1,2年して金沢市の県立中央病院の近く,たぶんヤマダ電機でデジカメを買ったように思います。

 2020-11-22\_164908_.jpgの写真が検察審査会の十一面観音像のポスターですが,午後4時49分という時刻で外がずいぶん暗くなっていると思いました。11月22日なので最も日暮れが早いらしい冬至よりは一月ほど前になります。

 宇出津新港のアルプというショッピングタウンの建物にある食堂です。客として行ったのではなかったのですが,その前回,客として行ったときはポスターがなかったように思います。客として店に入ることは少ないのですが,最初に検察審査会のポスターを見たのは3年ほど前かと思います。

 撮影した写真もあるはずですが,その頃は写真のファイル名に細かく情報を入れていたように思います。だとすれば,すぐに見つけることもできそうです。

\begin{itemize}
\tightlist
\item
  2019-02-13\_125201_宇出津新港 アルプの食堂 全国検察審査会連合会ポスター.jpg
  \url{http://hirono2014sk.blogspot.com/2019/03/2019031118452019-01-241753042019-03.html\#20190213125201} 
\end{itemize}

 同じ2019年2月13日の写真が並んでいますが,アルプの2階にあったゲームセンターでパチスロ機の写真がありました。ちょうど1年ほど前になるのか閉店し撤去となっています。ホームセンタームサシでセラミックヒーターを買ってきたのもそのあとだったようです。

 薄いピンク色のセラミックヒーターは現在あるのと同じですが,前は同じ製品の違う色で白だったと思うのですが,温風の吹き出し口を塞いだまま寝てしまい電気系統が壊れたので買い替えをしたのでした。

 何日前になるのか久しぶりに同じ食堂に入ったのですが,検察審査会のポスターはなくなっていました。その日の朝刊に,珠洲市で初の新型コロナウィルス感染者という記事があったことはよく憶えています。写真も撮影しました。

 写真は別の機会にご紹介をしたいと思いますが,4月30日で,職業安定所に金沢地方検察庁からの郵送物を持ってハンコを貰いに行った日のことでした。何か特別なことが食事の前後にあったとは憶えていたのですが,これもホームセンターでの買い物のことかと考えていました。

\begin{itemize}
\tightlist
\item
  2021-04-30\_11-33-42\_000.jpeg ¥\n \url{https://t.co/g5djopfWgI} 
\end{itemize}

 AmazonPhotoの写真を1枚,共有リンクにしたのですが,ほとんど解体された旧能登町役場の写真です。まだ瓦礫が残っているようですが,今頃は瓦礫もなくなっているかもしれません。その頃はよく前を通っていたのですが,5日ほど前を通って解体現場を見ていないように思います。

 追加でAmazonPhotoに写真のアップロードをしたのですが,どうも4月30日が最後に撮影した旧能登町役場解体工事現場の写真だったようです。そのあと夕方に一度撮影したような気がしていたのですが,写真は見当たりませんでした。4月30日より前だったのかもしれません。

\begin{lstlisting}
py37_env ❯ head -n 20 ~/twilog/hirono_hideki2021-05-07_143302.csv|sed 's/^/- /'
\end{lstlisting}

\begin{itemize}
\tightlist
\item
  2021-05-07 14:27:36
  ``2021年05月07日14時12分の実行記録¥\ntwitterAPI-search-lawList-mydql-add.rb''大津園児事故``¥\nツイート数:16/2407
  リツイート数:2/2407 トータル:904¥\nhirono\_hideki
  3/2件¥\nkk\_hirono 0/0件¥\ns\_hirono 0/0件''
  \url{https://twitter.com/hirono\_hideki/status/1390538787722121218} 
\item
  2021-05-07 14:07:55 ``- 1346:2021-05-07\_14:07:45 \#告発状 \#\#\#\#
  櫻井光政弁護士のテレビドラマ「イチケイのカラス」に関するツイートがきっかけで間違いに気がついた左陪席の位置
  \url{https://t.co/Fbuj3ppZYz''} 
  \url{https://twitter.com/hirono\_hideki/status/1390533834765045760} 
\item
  2021-05-07 14:01:27
  ``大津園児事故、直進車不起訴に不服申し立て 遺族ら「注意義務怠った」|社会|地域のニュース|京都新聞
  \url{https://t.co/AdsaIlr76P} 
  直進車の女性(64)=大津市=を不起訴とした大津地検の処分を不服とし、被害園児の遺族や保護者らが7日、大津検察審査会に審査を申し立てた。''
  \url{https://twitter.com/hirono\_hideki/status/1390532206292996101} 
\item
  2021-05-07 14:00:42
  ``大津園児事故、直進車不起訴に不服申し立て 遺族ら「注意義務怠った」|社会|地域のニュース|京都新聞
  \url{https://t.co/AdsaIlr76P}  2021年5月7日 10:18''
  \url{https://twitter.com/hirono\_hideki/status/1390532016618102784} 
\item
  2021-05-07 13:59:26
  ``困ったときは、弁護士に相談を!あなたのまちの京都弁護士会【相続/交通事故/借金/離婚/雇用関係】
  <特集PR>|PR|京都新聞 \url{https://t.co/ye2LZYR2zJ} 
  社会正義の実現のために¥\n~京都弁護士会が作る人権救済基金~''
  \url{https://twitter.com/hirono\_hideki/status/1390531698429874179} 
\item
  2021-05-07 13:59:12
  ``困ったときは、弁護士に相談を!あなたのまちの京都弁護士会【相続/交通事故/借金/離婚/雇用関係】
  <特集PR>|PR|京都新聞 \url{https://t.co/ye2LZYR2zJ} 
  これまでの取り扱い件数は71件になります(2017年10月現在)。''
  \url{https://twitter.com/hirono\_hideki/status/1390531638153539586} 
\item
  2021-05-07 13:57:48
  ``困ったときは、弁護士に相談を!あなたのまちの京都弁護士会【相続/交通事故/借金/離婚/雇用関係】
  <特集PR>|PR|京都新聞 \url{https://t.co/ye2LZYR2zJ''} 
  \url{https://twitter.com/hirono\_hideki/status/1390531287375433730} 
\item
  2021-05-07 13:53:04
  ``舞鶴女子高生殺害事件、未解決のまま13年 遺族の苦悩深く|社会|地域のニュース|京都新聞
  \url{https://t.co/eWqY7gwomr} 
  2008年5月に舞鶴市で東舞鶴高浮島分校1年の小杉美穂さん=当時(15)=が殺害された事件で、美穂さんの母親が7日までに、京都新聞社の取材に初めて応じた。''
  \url{https://twitter.com/hirono\_hideki/status/1390530095584976899} 
\item
  2021-05-07 13:52:01
  ``「生きていれば、どんなすてきな大人に」 舞鶴女子高生殺害事件、未解決のまま13年 遺族の苦悩深く|社会|地域のニュース|京都新聞
  \url{https://t.co/eWqY7gwomr''} 
  \url{https://twitter.com/hirono\_hideki/status/1390529834430832640} 
\item
  2021-05-07 13:51:14 ``RT @kanazawa\_rinji:
  「石川非常事態宣言」の発出¥\n(対象期間:5月25日まで)¥\n¥\n\url{https://t.co/rGhzp7tlfk} 
  \url{https://t.co/UjZLpIz57v''} 
  \url{https://twitter.com/hirono\_hideki/status/1390529636770086914} 
\item
  2021-05-07 13:51:03
  ``祇園祭、今年も規模縮小 山鉾巡行は代替、コロナ対策で|観光|地域のニュース|京都新聞
  \url{https://t.co/z4p1xrERGC''} 
  \url{https://twitter.com/hirono\_hideki/status/1390529589303123968} 
\item
  2021-05-07 13:50:13 ``RT @nkuma1314:
  亡くなった園児は可哀想なんだけどさ。¥\n¥\nコレ、直進車は側面からぶつけられてる事故で、¥\n直進車が交差点の侵入後に右折車にぶつけられてるから、¥\n注意するも何もねぇのよ。¥\n¥\nそもそも直進車優先だし。¥\n¥\n大津園児事故、直進車不起訴に不服申し立て 遺族ら「注意義務怠った」
  ¥\n\url{https://t.co/ozY6PXksGc''} 
  \url{https://twitter.com/hirono\_hideki/status/1390529378556071936} 
\item
  2021-05-07 13:49:37 ``RT @Imerdf:
  無理やりの右折にぶつかった直進は被害者だ、踏んだり蹴ったりもいいとこだ。弁護の入れ知恵もいい加減にしろよ。¥\n大津園児事故、直進車不起訴に不服申し立て 遺族ら「注意義務怠った」''
  \url{https://twitter.com/hirono\_hideki/status/1390529229658329088} 
\item
  2021-05-07 13:49:08 ``(15) 大津園児事故 - Twitter検索 / Twitter
  \url{https://t.co/tlewkeRfvQ''} 
  \url{https://twitter.com/hirono\_hideki/status/1390529106538754051} 
\item
  2021-05-07 11:41:25 ``2021-05-07\_11:40
  奉納\\#危険生物・弁護士脳汚染除去装置\\#金沢地方検察庁御中\_2020:
  「(ポートアイランド\textbar 港島)」を@hirono\_hideki @kk\_hirono @s\_hironoで検索 7件の該当 2021-05-07\_11:40の記録
  \url{https://t.co/g9dCQI9sLk''} 
  \url{https://twitter.com/hirono\_hideki/status/1390496964765122563} 
\item
  2021-05-07 11:41:18 ``2021-05-07\_11:39
  奉納\\#危険生物・弁護士脳汚染除去装置\\#金沢地方検察庁御中\_2020:
  「高槻市」を@hirono\_hideki @kk\_hirono @s\_hironoで検索 28件の該当 2021-05-07\_11:39の記録
  \url{https://t.co/kNnnyXCxzZ''} 
  \url{https://twitter.com/hirono\_hideki/status/1390496938328350741} 
\item
  2021-05-07 11:41:12 ``2021-05-07\_11:39
  奉納\\#危険生物・弁護士脳汚染除去装置\\#金沢地方検察庁御中\_2020:
  「黒木華」を@hirono\_hideki @kk\_hirono @s\_hironoで検索 36件の該当 2021-05-07\_11:38の記録
  \url{https://t.co/YnfbCtzB43''} 
  \url{https://twitter.com/hirono\_hideki/status/1390496911862288386} 
\item
  2021-05-07 11:41:06 ``2021-05-07\_07:24
  奉納\\#危険生物・弁護士脳汚染除去装置\\#金沢地方検察庁御中\_2020:
  REGEXP:''旭川.*少女''/データベース登録済みツイートの検索:2021-05-04〜2021-05-07/2021年05月07日07時24分の記録:ユーザ・投稿:2/40件
  \url{https://t.co/KUdhQNoD6P} "
  \url{https://twitter.com/hirono\_hideki/status/1390496885337595904} 
\item
  2021-05-07 10:58:11 ``ショートカットを学ぶ \url{https://t.co/q63CwGoFdh''} 
  \url{https://twitter.com/hirono\_hideki/status/1390486086103691264} 
\item
  2021-05-07 10:30:56 ``危機一髪! クジラの彫刻が電車救う
  オランダ 写真5枚 国際ニュース:AFPBB News \url{https://t.co/ljfBy2rrWS''} 
  \url{https://twitter.com/hirono\_hideki/status/1390479227712000000} 
\end{itemize}

 上記は奉納\さらば弁護士鉄道・泥棒神社の物語(@hirono\_hideki)のタイムラインで最新20件のツイートになります。はてなのブログでも埋め込みツイートの表示にはならないと思いますが,全体の流れはわかりやすくなっているかと思います。

\begin{itemize}
\tightlist
\item
  2021年05月07日14時28分の登録:
  REGEXP:''大津園児事故''/データベース登録済みツイートの検索:2020-01-16〜2021-05-07/2021年05月07日14時27分の記録:ユーザ・投稿:17/30件
  \url{https://kk2020-09.blogspot.com/2021/05/regexp2020-01-162021-05.html} 
\end{itemize}

 追加の作業を忘れていたBloggerのブログ記事になります。これまで大津園児死亡事故としていたのですが,本日のTwitterのトレンドでは大津園児事故となっていて,そちらで調べました。

\begin{itemize}
\tightlist
\item
  (07/30) TW ichifuna\_law(弁護士
  髙橋裕樹(アトム市川船橋法律事務所代表)) 日時: 2020-02-17 15:43:00
  +0900 URL:
  \url{https://twitter.com/ichifuna\_law/status/1229295206471614466\textgreater} {}
  被害者の方々が望むような被告人の反省も期待通りの判決も現実的にはかなわないことが圧倒的に多い\textgreater{}
  政治利用したり被害感情を煽るだけの弁護士も結構いるのでそれは問題だと思います\textgreater{}
  遺族ら「納得できない」「反省の姿見せるべき」大津園児事故、・・・
  \url{https://t.co/7Ixl0r3Jby} 
\end{itemize}

 実写版の漫画に出てくるイメージがとても強い髙橋裕樹弁護士で,アトム市川船橋法律事務所代表となっています。千葉県市川市のことと思いますが,そういえば昨夜の発見になるのか,千葉市の弁護士が矯正性交等致傷の未遂で再逮捕されたというニュースがありました。

 他の弁護士らの反応を含め個別に取り上げることも検討していたのですが,最初の逮捕のニュースが4月の前半にあって,私自身がニュースのツイートのリツイートをしていたのですが,リツイートをしたみにおぼえがなくって不思議に思った事件でもあります。

\begin{itemize}
\tightlist
\item
  (14/30) TW NEWS\_JAPAN\_S(NEWS JAPAN) 日時: 2021-05-07 10:45:13
  +0900 URL:
  \url{https://twitter.com/NEWS\_JAPAN\_S/status/1390482820787810309\textgreater} {}
  【滋賀】大津園児事故、直進車不起訴に不服申し立て
  遺族ら「注意義務怠った」\textgreater{}
  大津市で2019年5月、散歩中の保育園児2人が死亡し、園児ら14人が重軽傷を負った交通事故で、自動車運転処罰法違反(過失致死傷)の疑いで書類送検された直進・・・
  \url{https://t.co/y3KqyEp15l} 
\end{itemize}

 さきほど気がついたのですが,大津園児死亡事故は2019年5月という時期で,先程の新潟小2女児殺害事件の1年後の同時期にあたります。電池のプラスとマイナスのような方向性の違いはありますが,弁護士が異常に加熱した生物的化学反応を起こしたという点に共通性を感じていました。

 直進車の不起訴は,個人的に被疑者が警察関係者の縁故者である可能性も想像したのですが,不起訴処分に異を唱える発言はほどんど見かけなかったように思います。先程最初の方に見たツイートには,直進車が側面に衝突を受けたという話があり,初めて聞いたように思いました。

 右折車の女性被告は,不運が重なった事故と言いながら,実際に出会ったこともないというストーカー事件で併合審理され,大西直樹裁判長が禁錮4年6月という判決を出していました。控訴したものの間もなく取り下げたというニュースもありましたが,その頃には報道もすっかり下火でした。

 2021-05-07
10:45:13が最初に記録されている上記のNEWS\_JAPAN\_Sのツイートの時刻ですが,私がTwitterのトレンドで知った直後のツイートは13時49分となっています。16時24分となった今になって気がついたのですが,14時前後の情報番組でテレビをつけておけばよかったです。

 検察審査会といえば,確か小沢一郎氏の陸山会事件のことで,ジャーナリストの江川紹子氏らが批判の矛先を向けて監視をしていたような時期があったと思うのですが,ニュースで見たのも久しぶりです。しばらく前にも見かけたように思いますが,ちょっと思い出せません。

 2021年05月07日16時29分の実行記録 ¥\n
twitterAPI-search-lawList-mydql-add.rb ``大津園児事故'' ¥\n
ツイート数:31/2407 リツイート数:2/2407 トータル:1410 ¥\n
hirono\_hideki 5/2件 ¥\n kk\_hirono 11/0件 ¥\n s\_hirono 0/0件

 テレビの報道もみていないのですが,大津園児事故をキーワードにした,TwitterAPIの検索はトータルで1410件という乏しい数です。

 被害者弁護団や弁護士の宣伝となることを忌避する意識が働いているのかと,邪推のような考えももたげてくるのですが,このところ報道の格差というのもかなり強く意識するようになっています。

 報道の成果として大きかったのが森友学園問題やその関連ですが,過去に遡るとえん罪となった村木厚子氏の郵便不正事件があって,検察批判の一大キャンペーンとなっていました。ジャーナリストの江川紹子氏と郷原信郎弁護士の活躍がとりわけ大きかったと思います。

 令和3年3月31日付告発状でも少し取り上げて導入部としておいたつもりですが,その村木厚子氏のの郵便不正事件の関連でも同告発状の提出意向,大きな動きがあり,これは弁護士らの反応が期待を超えた大きなものとして,記録することに成功しています。弁護士鉄道の記録資料です。

 刑裁サイ太のタイムラインに「ソクられる」というのが出てきて,どうも以前見かけていたソクラテスメソッドのことのようなのですが,気になりながら調べておらず正確な意味がわからないソクラテスでした。

 古代ヨーロッパにソクラテスという有名な哲学者がいたように思います。昭和50年代の前半になるのか,ソクラテスに始まりサルトルやニーチェが出てくるテレビCMがありました。ウィスキーのCMだったと思います。

 最近はみかけないですが,ソクラテスメソッドと同じ頃に,同じような法クラのアカウントがよくクソテラスともツイートをしていましたが,これは法テラスを揶揄した言葉に間違いなさそうでした。

\begin{itemize}
\tightlist
\item
  ソクラテスメソッド|Bunny100\%|note \url{https://t.co/ejycHkpN1X} 
\end{itemize}

 調べてみると,意外な結果で,法クラの弁護士のロースクールを批判した造語とは違っていたようです。さきほどみかけた「ソクられる」は,無駄なことをさせられる学習法という言い回しがあったように思います。

〉〉〉 kk\_hironoのリツイート 〉〉〉

\begin{itemize}
\tightlist
\item
  RT
  kk\_hirono(刑事告発・非常上告_金沢地方検察庁御中)|ponponazarashi(ぽんぽん🍋🍃)
  日時:2021-05-07 17:18/2021/05/06 18:23 URL:
  \url{https://twitter.com/kk\_hirono/status/1390581814708764672} 
  \url{https://twitter.com/ponponazarashi/status/1390235857940926466} 
  \textgreater{} ちなみに、法科大学院生は『ソクられる』恐怖があるため、
  司法試験対策としては深すぎる論点等にも時間を割きすぎてしまい、
  『効率的な勉強をしなければ!』というスイッチを上手く入れられない
  →『この程度の勉強で大丈夫なわけない・・・』という完璧主義に陥りやすい🤔?
  という話も出ました👀✍️
\end{itemize}

 最初に記録されていたのが上記のリツイートのツイートですが,見覚えのないプロフィールの名前とアイコンで,変更された可能性があるのかもしれません。ブロックされているアカウントの可能性が高そうに思っていたのですが,リツイートは出来ました。

 d\textbar grep @ponponazarashi の結果がありませんでした。

\begin{lstlisting}
py37_env ❯ twilog-serch  @ponponazarashi  
\end{lstlisting}

\begin{itemize}
\tightlist
\item
  ./hirono\_hideki2021-05-07\_170903.csv:2021-05-03 19:51:15 ``RT
  @ponponazarashi:
  5/3憲法記念日の本日、約8年お付き合いしてきた方と入籍致しました。
  コロナ禍のため、顔合わせ・結納・新婚旅行すべての行事がキャンセル、婚姻届もひとりで提出となりましたが、家族皆健康で新たな門出を迎えることができました。今後もどうぞよろしくお願い致します💐''
  \url{https://twitter.com/hirono\_hideki/status/1389170684287356935} 
\item
  ./kk\_hirono2021-05-07\_172225.csv:2021-05-07 17:22:11
  ``d\textbar grep @ponponazarashi の結果がありませんでした。''
  \url{https://twitter.com/kk\_hirono/status/1390582720753926144} 
\item
  ./kk\_hirono2021-05-07\_172225.csv:2021-05-07 17:18:35 ``RT
  @ponponazarashi:
  ちなみに、法科大学院生は『ソクられる』恐怖があるため、¥\n司法試験対策としては深すぎる論点等にも時間を割きすぎてしまい、¥\n『効率的な勉強をしなければ!』というスイッチを上手く入れられない¥\n→『この程度の勉強で大丈夫なわけない・・・』という完璧主義に陥りやすい🤔?¥\n¥\nという話も出ました👀✍️''
  \url{https://twitter.com/kk\_hirono/status/1390581814708764672} 
\end{itemize}

 「フルタイムワーカーとして働きながら令和2年司法試験に4回目で合格(東大ロー未修卒・74期奈良)/未修」などとTwitterのプロフィールにあるので,5月3日にツイートを見かけ,リストに登録したアカウントだったようです。

〉〉〉 kk\_hironoのリツイート 〉〉〉

\begin{itemize}
\tightlist
\item
  RT
  kk\_hirono(刑事告発・非常上告_金沢地方検察庁御中)|1JnrSmgg0NLUiKU(小さい弁護士)
  日時:2021-05-07 17:27/2021/05/06 22:51 URL:
  \url{https://twitter.com/kk\_hirono/status/1390584148079747075} 
  \url{https://twitter.com/1JnrSmgg0NLUiKU/status/1390303157243170816} 
  \textgreater{}
  法科大学院の勉強を熱心にするあまり、効率的な勉強をするという視点に欠け、3年生の夏になってようやく初めて過去問を解いてみたところ、全く解けない現実を認めたくないという心理から、さらに深すぎる論点に時間を割くという負のスパイラル
  \url{https://t.co/w2M2nVwfO3} 
\end{itemize}

 見たことのあるようなアイコンで水木しげるの漫画に出てくるキャラクターのようですが,登録済みと思いながらコマンドを実行したところ,リストへの新規追加となりました。

〉〉〉 kk\_hironoのリツイート 〉〉〉

\begin{itemize}
\tightlist
\item
  RT
  kk\_hirono(刑事告発・非常上告_金沢地方検察庁御中)|real\_metamon(めたもん)
  日時:2021-05-07 17:30/2021/05/06 23:06 URL:
  \url{https://twitter.com/kk\_hirono/status/1390584705594978305} 
  \url{https://twitter.com/real\_metamon/status/1390306863980302342} 
  \textgreater{}
  深い論点意外にも、通説・判例・実務がある中で、反対説をソクラテス対策に学ばなければいけないときがありますが、それが試験対策にどれだけ意義があるかとかも
  \url{https://t.co/3r8gDen4TR} 
\end{itemize}

 プロフィールをみただけで判断は難しいのですが,司法試験受験者は基本,対象外の方針なのでリストへの登録は見送りました。

\begin{itemize}
\tightlist
\item
  (18/31) TW un\_co\_the2nd(うの字を名乗る?物) 日時: 2021-05-07
  11:15:03 +0900 URL:
  \url{https://twitter.com/un\_co\_the2nd/status/1390490328818278405\textgreater} {}
  ソクられるって、ソクラテスされるってことか。合点
\end{itemize}

 twitterAPI-search-lawList-mydql-add.rb
で自動登録されたうの字のツイートです。

〉〉〉 kk\_hironoのリツイート 〉〉〉

\begin{itemize}
\tightlist
\item
  RT
  kk\_hirono(刑事告発・非常上告_金沢地方検察庁御中)|babel0101(anonymity)
  日時:2021-05-07 17:35/2021/05/07 12:09 URL:
  \url{https://twitter.com/kk\_hirono/status/1390586166412414984} 
  \url{https://twitter.com/babel0101/status/1390503975435210757} 
  \textgreater{} 「ソクられる」というのはかなり定着した感。
\end{itemize}

 ブロックされていることを確認するつもりでリツイートを試みたのですが,リツイートが出来たので驚きました。奉納\さらば弁護士鉄道・泥棒神社の物語(@hirono\_hideki)ではブロックされていると思います。確認してみます。

▶ ブロックされたツイート%babel0101(anonymity)%2021/05/07 12:09:16%
\url{https://twitter.com/babel0101/status/1390503975435210757\textgreater} {}
「ソクられる」というのはかなり定着した感。

\begin{itemize}
\tightlist
\item
  深澤諭史(@fukazawas)さんの返信があるツイート / Twitter
  \url{https://twitter.com/fukazawas/with\_replies} 
\end{itemize}

 未だ更新がないことを確認した深澤諭史弁護士のTwitterアカウントです。ソクラテスメソッドというのは深澤諭史弁護士が反応しそうなツイートだったので確認をしてみました。

 平成6年7月20日付となっている被告発人若杉幸平弁護士の回答書ですが,人権救済申立が6月20日と6月26日付とされています。短期間になぜ2つあるのか不明ですが,ちょうど一月で回答があったようです。

 もう少し長く2ヶ月ぐらいが間があったように思うのですが,金沢弁護士会に人権救済申立をしたのと同じ頃に,たぶん同じ頃だったと思うのですが,検察審査会にも不起訴処分の不服申立てを郵送し,回答がありました。標題も記憶にないですが,まだましな結論が理解が出来る内容でした。

 デジカメで撮影した写真があると思っていたのですが,パソコンには見当たりませんでした。捨てるはずもないのでそのうち見つかるかと思います。

 また,これまでの告訴告発で,金沢地方検察庁が検察審査会への申立を期待しているのかと考えたこともあるのですが,確か2017年3月31日付,の不起訴処分通知書は,私が告訴状としていたものを,告発状として処理できないなどと書いてありました。処理だったのか言葉は憶えていません。

\begin{itemize}
\tightlist
\item
  〈〈〈 2021/05/07 17:49:57 Linux Emacs: 〈〈〈
\end{itemize}

\hypertarget{ux88abux544aux767aux4ebaux82e5ux6749ux5e78ux5e73ux5f01ux8b77ux58ebux306bux3064ux3044ux3066ux8abfux3079ux3066ux307fux308bux82e5ux6749ux6cd5ux5f8bux4e8bux52d9ux6240ux306f2012ux5e74ux304bux3089ux66f4ux65b0ux306eux306aux3044ux30b9ux30c8ux30eaux30fcux30c8ux30d3ux30e5ux30fcux306eux4f4fux5b85ux5730ux91d1ux6ca2ux5e02ux7530ux4e95ux753a4-5}{%
\paragraph{被告発人若杉幸平弁護士について調べてみる〜若杉法律事務所は2012年から更新のないストリートビューの住宅地(金沢市田井町4-5)}\label{ux88abux544aux767aux4ebaux82e5ux6749ux5e78ux5e73ux5f01ux8b77ux58ebux306bux3064ux3044ux3066ux8abfux3079ux3066ux307fux308bux82e5ux6749ux6cd5ux5f8bux4e8bux52d9ux6240ux306f2012ux5e74ux304bux3089ux66f4ux65b0ux306eux306aux3044ux30b9ux30c8ux30eaux30fcux30c8ux30d3ux30e5ux30fcux306eux4f4fux5b85ux5730ux91d1ux6ca2ux5e02ux7530ux4e95ux753a4-5}}

\begin{itemize}
\tightlist
\item
  〉〉〉 Linux Emacs: 2021/06/02 11:07:15 〉〉〉
\end{itemize}

:CATEGORIES: @kanazawabengosi \#金沢弁護士会 @JFBAsns
日本弁護士連合会(日弁連) \#法務省 @MOJ\_HOUMU \#被告発人若杉幸平弁護士

\begin{itemize}
\item
  2021年06月02日11時18分の登録:
  「若杉幸平」を@hirono\_hideki @kk\_hirono @s\_hironoで検索 127件の該当 2021-06-02\_11:18の記録
  \url{https://kk2020-09.blogspot.com/2021/06/hironohidekikkhironoshirono1272021-06.html} 
\item
  アルバム アーカイブ - 資料写真/若杉幸平弁護士 \url{https://t.co/QAMaHH8siK} 
  2014/03/26 • 一般公開
\end{itemize}

〉〉〉 kk\_hironoのリツイート 〉〉〉

\begin{itemize}
\tightlist
\item
  RT
  kk\_hirono(刑事告発・非常上告_金沢地方検察庁御中)|kk\_hirono(刑事告発・非常上告_金沢地方検察庁御中)
  日時:2021-06-02 11:29/2019/01/05 17:15 URL:
  \url{https://twitter.com/kk\_hirono/status/1399916164369719298} 
  \url{https://twitter.com/kk\_hirono/status/1081464166710300672} 
  \textgreater{}
  同じ若杉幸平弁護士の法律事務所の検索で、初めにこのストリートビューの静止写真をみたのは、けっこう前だったと思いますが、そのときは何かの間違いではないかと思い、それほど深くは考えませんでした。それでも元の金沢地方検察庁の横の事務所は、抜け殻のように写されていました。
\end{itemize}

〉〉〉 kk\_hironoのリツイート 〉〉〉

\begin{itemize}
\tightlist
\item
  RT
  kk\_hirono(刑事告発・非常上告_金沢地方検察庁御中)|s\_hirono(非常上告-最高検察庁御中\_ツイッター)
  日時:2021-06-02 11:30/2018/12/31 11:57 URL:
  \url{https://twitter.com/kk\_hirono/status/1399916395656220676} 
  \url{https://twitter.com/s\_hirono/status/1079572173088419840} 
  \textgreater{} 2018-12-31-114433\_若杉幸平 弁護士(若杉法律事務所) -
  石川県金沢市 - 弁護士ジャパン.jpg \url{https://t.co/uZisId7Oop} 
\end{itemize}

〉〉〉 kk\_hironoのリツイート 〉〉〉

\begin{itemize}
\tightlist
\item
  RT
  kk\_hirono(刑事告発・非常上告_金沢地方検察庁御中)|hirono\_hideki(奉納\さらば弁護士鉄道・泥棒神社の物語)
  日時:2021-06-02 11:32/2018/02/23 00:59 URL:
  \url{https://twitter.com/kk\_hirono/status/1399916820572692481} 
  \url{https://twitter.com/hirono\_hideki/status/966703884197445635} 
  \textgreater{}
  あらためて金沢地方検察庁の不起訴処分通知書の写真。数日前に処分の宮下検事からスマホの形態に電話があった。通知書が届いて電話をすると、移動でいなくなったと言われ、女性検事が対応。
  2017-04-04\_16.07.09.jpg 2017-04-04\_16.07.32.jpg
  2017-04-04\_16.09.15.jpg 2017-04-04\_16.56.08.jpg
  \url{https://t.co/rwjROVlAoH} 
\end{itemize}

〉〉〉 kk\_hironoのリツイート 〉〉〉

\begin{itemize}
\tightlist
\item
  RT
  kk\_hirono(刑事告発・非常上告_金沢地方検察庁御中)|hirono\_hideki(奉納\さらば弁護士鉄道・泥棒神社の物語)
  日時:2021-06-02 11:32/2018/02/23 14:57 URL:
  \url{https://twitter.com/kk\_hirono/status/1399916833356935168} 
  \url{https://twitter.com/hirono\_hideki/status/966914986038632451} 
  \textgreater{}
  横にもう一枚紙があるので、調べたところ。もう一枚の書面の写真があった。木梨松嗣弁護士、岡田進弁護士、長谷川紘之弁護士、若杉幸平弁護士、古川龍一裁判官も不起訴処分となっていた。2枚に分かれていたとは記憶になかった。
  \url{https://t.co/jkUgKBiKUA} 
\end{itemize}

〉〉〉 kk\_hironoのリツイート 〉〉〉

\begin{itemize}
\tightlist
\item
  RT
  kk\_hirono(刑事告発・非常上告_金沢地方検察庁御中)|hirono\_hideki(奉納\さらば弁護士鉄道・泥棒神社の物語)
  日時:2021-06-02 11:32/2018/02/23 15:06 URL:
  \url{https://twitter.com/kk\_hirono/status/1399916872305319937} 
  \url{https://twitter.com/hirono\_hideki/status/966917144624558080} 
  \textgreater{}
  平成28年7月8日付の告発、となっている。告訴状として提出したはず。
\end{itemize}

 被告発人若杉幸平弁護士の法律事務所の検索で金沢市田井町の住居が出てきたのはGoogleマップやストリートビューとばかり思っていたのですが、「弁護士ジャパン」という見覚えのないサイトというかネットの検索サービスだったようです。

 2018年12月31日のスクリーンショットを確認しましたが、これから始める被告発人若杉幸平弁護士の法律事務所の検索はそれ以来になるかと思われます。

\begin{itemize}
\item
  若杉幸平 法律事務所 - Google 検索 \url{https://t.co/jJDZCTxwPO} 
\item
  若杉 幸平弁護士(若杉法律事務所)に法律相談 -
  石川県金沢市、北陸鉄道石川線 野町駅 \textbar{} Legalus
  \url{https://t.co/11KPGLpFj5}  若杉法律事務所 ¥\n 〒920-0924
  石川県金沢市田井町4-5
\item
  若杉 幸平|若杉法律事務所|金沢弁護士会に所属|弁護士データベース
  \url{https://t.co/oeDd6Xfv8B}  ¥\n 事務所名 ¥\n 若杉法律事務所 ¥\n 住所
  ¥\n 石川県 金沢市田井町4-5
\item
  若杉幸平弁護士と被告訴人OSN、被告訴人OKN兄弟との関係、勧められた面談 :
  金沢地方検察庁御中 \url{https://t.co/JNAAKKEk7G} 
\item
  若杉幸平弁護士と被告訴人OSN、被告訴人OKN兄弟との関係、勧められた面談 :
  金沢地方検察庁御中 \url{https://t.co/JNAAKKEk7G}  ¥\n
  上記参考資料にあるとおり、若杉幸平弁護士の法律事務所は現在、丸の内法律事務所という名称になっていました。以前は若杉幸平法律事務所ではなかったかとも思いますが、いつの
\item
  奉納\危険生物・弁護士脳汚染除去装置\金沢地方検察庁御中:
  2018年12月30日16:30記録\法務検察・石川県警察宛\写真資料:2018-12-17\_220936〜2018-12-29\_194901:244件
  \url{https://t.co/8PaoJYITt8}  35件目 »
  2018-12-20\_162646_図書館のパソコン Google検索「若杉幸平 弁護士」.jpg
\item
  弁護士を探す(弁護士名から探す)|金沢弁護士会 \url{https://t.co/5Iy0QTOVpL} 
  ¥\n 若杉 幸平  わかすぎ こうへい ¥\n 若杉法律事務所 ¥\n ¥\n
  〒920-0924 金沢市田井町4-5 ¥\n
  TEL:076-223-5765 FAX:076-223-5765
\end{itemize}

 最終的に金沢弁護士会のホームページで確認となったのですが、やはり被告発人若杉幸平弁護士の法律事務所の住所が金沢市太町となっていました。続けて調べてみますが、住宅地の中にある一般住宅だと思います。

\begin{itemize}
\tightlist
\item
  〒920-0924 石川県金沢市田井町4−5 - Google マップ
  \url{https://t.co/H1ff3ozCez} 
\end{itemize}

 ストリートビューでは以前見たのと同じと思われる住宅が出てきました。すっかり忘れていたのですが、さきほど少し見かけていた過去のツイートにあったように、日傘を手にした女性が家から出てくる静止画となっていました。

 前はたまたまなのかとあまり気にかけずにいたようですが、家から出てくるタイミングでのストリートビューの写真というのは他に見たことがなく、同じ場所で1枚だけ撮影しているとも考えにくいので、差し替えなど検討しなかったのかと疑問を感じました。

 ストリートビューをよく見ると、右下に撮影日が2012年9月とあるのですが、左上のメニューにタイムラインが見当たらず、この写真1枚だけのようです。

\begin{itemize}
\item
  田井町ショッピングセンター - Google マップ \url{https://t.co/F5bZxkz7K6} 
\item
  県道27号 - Google マップ \url{https://t.co/PqFppDiAYO} 
\end{itemize}

 さきほどストリートビューで見つけた場所ですが、県道27号というページタイトルになっているストリートビューは、2012年、2018年、2019年と3つのタイムラインの写真があります。

 田井町ショッピングセンターとありますが、ここに市内配達でよく行った、とりわけ印象深く記憶に残っている小森という店があった場所だと思います。伝票にも小森とだけ書いてあったように思いますが、果物を中心にした商店という様子でした。

 市内配達では金沢市内に数店舗ある「マルエー」「ニュー三久」などスーパーはまとまった量の荷物だったのですが、そのスーパー以上に荷物が多かったのが、この小森という店で、老夫婦が二人でやっているようでした。仲買の番号が415だったようにも思います。

 小立野の金沢大学病院の方から来ると、田井町の交差点の手前、左側にその小森という店があったのですが、田井町という金沢市内の住所は他に聞くこともなく、交差点の標識としてよく憶えていた地名でした。

\begin{itemize}
\tightlist
\item
  若杉法律事務所 金沢 - Google 検索 \url{https://t.co/z0zbnclnhJ} 
\end{itemize}

 どうもネット上の情報が極端に少ないようです。隠れなんとかという飲食店の話はテレビで見かけたことがあるのですが、文字通り隠れ法律事務所のような被告発人若杉幸平弁護士の若杉法律事務所です。

 被告発人若杉幸平弁護士の顔というのもぼんやりぼやけた記憶となっていますが、とにかく真面目そうで頭が良さそうという印象でした。年齢も若く見えたのですが、ツイートの検索では、ずいぶん早い時期に金沢弁護士会の会長となっていたようです。

 弁護士会の会長に年齢制限があるとは思えませんが、とりわけ経験や人脈がものをいいそうな弁護士という仕事柄、30代の弁護士会会長というのは考えにくく、統率力も必要と思われるので40代でも微妙という気がします。

\begin{itemize}
\tightlist
\item
  所属弁護士|大手町法律事務所(石川県金沢市) \url{https://t.co/sO7sVFfxJq} 
\end{itemize}

 そういえば先程、若杉法律事務所の検索で、塩梅修弁護士の名前があったのですが、金沢市出身となっていました。4月の初めは珠洲市に多い名前ということを調べて確認していましたが、今年度の金沢弁護士会の会長という塩梅修弁護士です。

 いくらか若いときの写真を使っているようにも思えるのですが、昭和36年金沢市生まれとあります。単純計算で60歳になります。

 被告発人木梨松嗣弁護士も金沢弁護士会の会長として金沢弁護士会のホームページで会長声明を読んでいるのですが、本件告発事件に関係した弁護士で、最初に金沢弁護士会の会長になった、あるいはなっていたと知ったのが、この被告発人若杉幸平弁護士だったと思います。

 昨日、金沢地方裁判所前の一方通行の道路と書きましたが、その一方通行の侵入口が検察庁前の交差点で、交差点から入ってすぐ右手にあったのが被告発人若杉幸平弁護士の法律事務所で、一方通行の道路を挟んで金沢地方検察庁の建物と向き合っていました。

 私が市内配達の仕事をしていた頃から見覚えがあって気になる感じの建物がその被告発人若杉幸平弁護士の法律事務所が2階にある建物だったのですが、1階の方も店舗になっていたように思います。

\hypertarget{ux9053ux8defux3092ux631fux3093ux3067ux91d1ux6ca2ux5730ux65b9ux691cux5bdfux5e81ux306eux6a2aux306bux3042ux3063ux305fux88abux544aux767aux4ebaux82e5ux6749ux5e78ux5e73ux5f01ux8b77ux58ebux306eux6cd5ux5f8bux4e8bux52d9ux6240ux306eux5efaux7269ux3068ux91d1ux6ca2ux5730ux691cux306eux6b74ux53f2}{%
\paragraph{道路を挟んで金沢地方検察庁の横にあった被告発人若杉幸平弁護士の法律事務所の建物と、金沢地検の歴史}\label{ux9053ux8defux3092ux631fux3093ux3067ux91d1ux6ca2ux5730ux65b9ux691cux5bdfux5e81ux306eux6a2aux306bux3042ux3063ux305fux88abux544aux767aux4ebaux82e5ux6749ux5e78ux5e73ux5f01ux8b77ux58ebux306eux6cd5ux5f8bux4e8bux52d9ux6240ux306eux5efaux7269ux3068ux91d1ux6ca2ux5730ux691cux306eux6b74ux53f2}}

\begin{itemize}
\tightlist
\item
  〉〉〉 Linux Emacs: 2021/06/02 13:18:45 〉〉〉
\end{itemize}

:CATEGORIES: @kanazawabengosi \#金沢弁護士会 @JFBAsns
日本弁護士連合会(日弁連) \#法務省 @MOJ\_HOUMU \#被告発人若杉幸平弁護士
\#金沢地方検察庁

 まずGoogleマップで金沢地方検察庁の場所を開こうとしたのですが、Googleの検索結果のページにあった金沢地方検察庁の建物の色がこれまでになく赤く見えて気になり、2度ほどリンクを開くと、検事正のご挨拶のページが出てきました。

\begin{quote}
《引用の始まり》
\end{quote}

\begin{quote}
令和元年11月8日付けで金沢地方検察庁の検事正を拝命いたしました植村誠です。 前任は鳥取地検の検事正を務めてまいりました。 出身は東京都で,平成2年に任官し,その後は東京と全国各地の検察庁との間の転勤を繰り返してきました。 金沢地検での勤務は初めてですが,平成10年4月から平成12年3月までの2年間,隣県の富山地検で三席検事を務めていました。
\end{quote}

\begin{quote}
《引用の終わり》
\end{quote}

\begin{itemize}
\tightlist
\item
  検事正紹介:金沢地方検察庁
  \url{http://www.kensatsu.go.jp/kakuchou/kanazawa/page1000005.html} 
\end{itemize}

 最終更新日が2019年11月18日となっています。植村誠検事正とありますが、前にも見たこと読んだことのある内容でした。

 ただ、今回すごく気になったのは、「平成10年4月から平成12年3月までの2年間,隣県の富山地検で三席検事を務めていました。」という部分です。

 ネットで三席検事という言葉を見たのもずいぶんと久しぶりに思ったのですが、この植村誠検事正は昭和38年生まれとあるので、昭和39年11月生まれの私とは1つ違いの年長になります。

 年齢を1つ読み替えるだけなのでとてもわかり易いのですが、よく見ると平成2年に札幌地検で検事任官とあります。私は平成4年の4月の時点で27歳だったのですが、2年引いて1年プラスすると、概算ですが26歳で検事に任官したことになりそうです。

 富山地検の三席検事になったのが平成10年とあるので検事任官から8年目ということがわかり、26歳から8年目で、これも概算になりますが34歳で三席検事。これはずいぶん若く感じます。

 もう何年もネットで三席検事という言葉を見かけていないので、制度改革でなくなっている可能性もあるぐらいに思えるのですが、ずっと前に調べたときもはっきりした役職という情報は見かけなかったように憶えています。

 地方検察庁のトップが検事正で次が次席検事、次席検事はマスコミ対応など広報的な役割も担うと聞きますが、どちらも決済官として、取り調べや法廷など現場に出ることはないと聞きます。そして現場に出る検事のトップが三席検事という話でした。

 三席という検事を最初に聞いたのは福井刑務所にいるときだったと思いますが、それも江村正之検察官がその三席検事だったという話を聞きました。

 金沢地方検察庁のホームページで検事正紹介のページには、右手に「金沢地検の紹介」というメニューがあって、「検事正紹介」の項目の下に「金沢地検の歴史」があることに気が付きました。なお、その下は「仕事・組織」「アクセス」となっています。

 ちょうど検事正紹介のページを開く前に、金沢地方検察庁の現在の建物のことが気になっていたのですが、私が最初に見た頃から現在と同じ建物で、そこに検察庁の建物があると認識した時期ははっきりしないのですが、平成2年頃に一度、未納付だった反則金を支払いに行きました。

 金沢市場輸送まで金沢地方検察庁を名乗る数人が来て、私がいる前で私の名前を出して所在を尋ねたので、私はすぐに飛び出して金沢地方検察庁に向かい、支払いをしたのです。

 断片的なショートムービーのような記憶しかないですが、金沢地方検察庁の建物のかなり上の階で、部屋に一人でいるような人物に背後から声を掛けたのです。まるでドラマの俳優の演技を見るような場面だったのですが、驚いた様子で軽く笑って対応をしていました。

 今考えると、検事ではなく副検事だった可能性が高そうですが、検事であった可能性が高いとも思われ、会ったときの第一印象や表情、そして逮捕された被疑者とは立場がまるで違ったものの、前に応対したのも江村正之検察官ではなかったのかと考えたことがありました。

 金沢地方検察庁の建物というのは警察や金沢刑務所から連れて行かれた以外は、中に入った回数も少なく反則金の納付以外は3回ぐらいしか思い出せないのですが、一度、建物の中を自由に歩き回ったことがありました。

 あてもなく金沢地方検察庁の庁舎内を歩き回ったわけではなく、トイレか部屋を探していたように思います。3階より上の階で、一つ上の階から降りてきたような気もするのですが、建物の正面側の通路に細かく仕切られたような副検事の部屋が並んでいました。

 けっこう数があってほとんどの部屋がドアを開けっ放しにして、通路を歩く私を訝しい目で見ていたのが印象的でした。副検事の部屋だとわかったのは、表札のようなものが掛けられていたからです。同じ間取りで狭い取調室のような部屋でした。

 通路というか廊下は、卯辰山の方角から歩いてきたのですが、右手に副検事の部屋が並んでいました。部屋の奥は建物正面の窓になるはずなのですが、一面が壁の密室のように見えたのも野性的な警戒心のような副検事らの反応と相まって異様なものだったと印象にあります。

 他に記憶にあるのは、金沢地方検察庁の建物の正面に小さな駐車場のスペースがあって、そこに自分の軽四を停めていたのですが、帰り際に車に乗り込み、エンジンを掛けたか掛けようとしたタイミングで、守衛のような人が慌てて駆け寄ってきました。

 あまり憶えていないのですが、庁舎に入る許可証のようなものを受取り、それを返さないまま帰ろうとしたのかと思います。何の用事で金沢地方検察庁に行ったのか記憶にないのですが、羽咋市に住んでいるときで、金沢地方検察庁での滞在時間もその時は短かった気がします。

 金沢地方検察庁の庁舎内の構造に関心があったわけではないのですが、振り返ると、入った部屋は1つの例外を除き、建物の裏側の部屋でいずれも陽当りが良かったように思います。

 1つの例外というのが下平豪検事が教官のような司法修習生の実習室のような部屋で、横に細長い部屋でした。考えてみると遠塚さんと話し、告訴状か告発状を手渡した捜査官室も、他に見たことのないような細長さを感じる部屋でした。奥行きが狭く感じられたとも言えます。

 奥行きが広く感じられたのは、反則金のことで入った部屋でした。金沢地方検察庁の建物の正面から右側の面に部屋があったようなイメージなのですが、この面に記憶がある部屋というのは、呼び出した参考人の待合室のような部屋ぐらいです。

 少なくとも平成18年10月の時点では、玄関口での受付さえ済ませば、事実上自由に金沢地方検察庁の庁舎内を歩き回ることができたと思います。この先進入禁止などの立て札を見たような記憶もありません。

〉〉〉 kk\_hironoのリツイート 〉〉〉

\begin{itemize}
\item
  RT
  kk\_hirono(刑事告発・非常上告_金沢地方検察庁御中)|nobuogohara(郷原信郎【長いものには巻かれない・権力と戦う弁護士】)
  日時:2021-06-02 14:37/2021/06/02 14:15 URL:
  \url{https://twitter.com/kk\_hirono/status/1399963387090685952} 
  \url{https://twitter.com/nobuogohara/status/1399957743453097984} 
  \textgreater{}
  「検察審査会の正義」で議員辞職に追い込まれた菅原一秀氏、「秘書にハメられた」についても説明を
  \#BLOGOS \url{https://t.co/gSYeR59mcy} 
\item
  金沢地検の歴史:金沢地方検察庁 \url{https://t.co/bSEtkLQGZ7}  ¥\n
  昭和26年12月   金沢市上胡桃町27の1番地(現在の金沢市大手町6番15号)
  ¥\n           に庁舎新営 ¥\n
  昭和57年 3月   旧庁舎跡地に金沢法務合同庁舎新営
\end{itemize}

 ブラウザの拡張機能からツイートしたタイミングで、郷原信郎弁護士のツイートが出てきました。

 昭和57年3月に「金沢法務合同庁舎新営」とあります。建物の建設にどれぐらいの期間が掛かったのか気になるのですが、前年の昭和56年8月28日の午後に、私はすぐ近くの金沢家庭裁判所に行っています。

 当時は広坂交差点の角にあった金沢中警察署から護送車に乗せられ、着いた先がその金沢家庭裁判所だったのですが、どこに行くという説明もなかったと思いますし、着いた先の金沢家庭裁判所でもほとんど説明がなく、理解できたのはこれから少年鑑別所に入るということでした。

 金沢少年鑑別所は小立野にあったのですが、広い通りからは離れた閑静な場所でした。2,3年か前にGoogleマップで場所を調べたのですが、場所自体は昭和56年当時と変わっていなかった気がします。

 今でも歴史的な体験をしたように思い出すのですが、当時の金沢少年鑑別所の建物は、今なら移築で歴史遺産になるような建物だったと思います。古い時代の田舎の分校のようでもあったのですが、独居房の扉が分厚い鉄扉で、土蔵のようでした。それでも陽当りは良かったと思います。

 28日以内に審判があって家に戻れるか少年院に行くか決まると言われていたのですが、鑑別所にいたのは26日だったと思います。試験観察でしたが、半年以内には次の審判の結果が出たと思います。保護観察もつかなかったので、不処分のようなものだったとかと思います。

 記憶にあるのは同じ昭和56年の11月の20日頃になると思いますが、七尾市の家庭裁判所で調査官に会いました。令和3年3月31日付告発状には記述したように思いますが、待合室で、被告発人安田敏とその母親と会ったときのことです。

 もう一度、審判で金沢家庭裁判所に行ったという記憶もないので、書面のような通知が届いて終わりになったのかと今は考えますが、思い出す以前にあまり考えたことがなかった気がします。

\begin{itemize}
\tightlist
\item
  金沢少年鑑別所 - Google マップ \url{https://t.co/w2k20SMJfu}  〒920-0942
  石川県金沢市小立野5丁目2−14 金沢少年鑑別所
\end{itemize}

 場所が昭和56年当時と変わっていないと思ったのは隣の学校の建物のことですが、ずっと「ろう学校」だと思っていたのが「石川県立盲学校」になっていることに気が付きました。「ろう」も漢字になっていたはずですが、変換候補に見当たりませんでした。

 「ろうあ者」でも変換候補が出ないですが、差別語として使われなくなっている可能性がありそうです。耳が聞こえず普通に喋れない子供の学校と聞いたのですが、それも鑑別所の中で初めて知ったことでした。

 その石川県立盲学校の隣に、石川県立金沢商業高等学校の建物がありますが、あまり聞いた憶えのない高校名で、校名に金沢とあるのに石川県立というのは、とてもめずらしく感じました。航空写真で見てもずいぶん新しそうな建物です。

 金沢少年鑑別所に入る広めの道路に209という番号があり、県道なのか聞いたことはないものの金沢市の市道なのかと迷ったところ、芝原石引町線とあるのを調べたところ県道であることがわかりました。

 ずいぶんと様変わりしたものだと思うのですが、昭和56年当時は、車もほとんど通らないような道で、神社の参道で町外れの火葬場に向かう道のような鬱蒼とした印象がありました。

 近年、ネットで知った情報だと思いますが、この金沢市小立野は寺院も多いようです。また、金沢地方裁判所の前の兼六園下の交差点から上り坂になる高台でもあったのですが、小立野台地と呼ばれることも知りました。

 金沢で小立野といえば、金沢大学病院を思い浮かべる人が多いかと思いますが、これもネットで見かけた情報では住所が金沢市宝町となっていました。

\begin{itemize}
\tightlist
\item
  宝町 - Google マップ \url{https://t.co/1uceLKJTZT} 
\end{itemize}

 市内配達の仕事で電柱によく宝町という表示を見たと記憶にあるのですが、高台の下の方の方も宝町を見たように思っていました。Googleマップの範囲はすべて高台になっているようです。同じ高台の下で隣接していると思っていたのが天神町になります。

 小立野の高台から天神町に坂を降りて田井町に向かうというのは市内配達の基本的パターンでしたが、たまに出る配達先でコースを変えることもありました。この市内配達は昭和61年と平成3年にやっていますが、短い期間だったものの、他と比較して憶えていることが多い気がします。

\begin{itemize}
\tightlist
\item
  金沢監獄中央看守所・監房|エリア別建造物|文化財|村内の楽しみ方|博物館明治村
  \url{https://t.co/xN13ZrUgQl} 
\end{itemize}

 湯涌温泉の近くにあったのは江戸村ではなかったのかと思ったのですが、明治村の住所は愛知県犬山市とありました。古い建物が県外に移築されたという話は聞いたことがなかったので、てっきり金沢にある場所と思い込んでいました。

\begin{itemize}
\tightlist
\item
  〒920-0942 石川県金沢市小立野5丁目 - Google マップ
  \url{https://t.co/XcrllU7MAT} 
\end{itemize}

 明治時代の金沢監獄と思われる場所を調べたところ、前の金沢少年鑑別所の検索では気が付かなかった、金沢大学の鶴間キャンパスというのが出てきました。角間キャンパスというのは割合よく見かけるのですが、鶴間というのは地名としても記憶にありません。

 鶴間キャンパスの住所も金沢市小立野5丁目となっているので、鶴間というのは地名ではなさそうです。大学でなんとかキャンパスというのは、すべて地名という固定観念がありました。

\begin{itemize}
\tightlist
\item
  金沢大学 - Wikipedia \url{https://t.co/noUODnXTwA}  ¥\n
  鶴間キャンパス{[}編集{]} ¥\n 使用学域:医薬保健学域保健学類 ¥\n
  交通アクセス:金沢駅から北鉄バス医王山線、また野々市金大線、市立病院線「如来寺前」下車。
\end{itemize}

 Googleで鶴間キャンパスを検索しても賃貸物件のような情報がほとんどで、他に医学部という情報ぐらいしか見当たらなかったのですが、よく考えると鶴間というのも江戸時代の旧地名になるのかもしれないと思えてきました。

\begin{itemize}
\item
  鶴間坂 (金沢市旭町3丁目-小立野5丁目)の由来、写真、地図、コラムなど :
  金沢の坂道 \url{https://t.co/aVNb8A2U7O} 
\item
  上り うっさか 下(お)り つるま ― 鶴間坂 <上>
  \url{https://t.co/hMgeW5BWMi} 
  それまでの牛坂、人は親しみを込めて「うっさか」と呼んだこの坂が「つるま」となるのは瑞祥地名にこだわったからにほかならない。この坂の場合、どうして鶴間坂になったかという経緯から入らなければならない。
\item
  上り うっさか 下(お)り つるま ― 鶴間坂 <上>
  \url{https://t.co/hMgeW5BWMi} 
  旭坂は旭町からきたものだろう。旭町の誕生は昭和11年(1936)。小立野台東端の裾に長く伸びた牛坂村は、降り注ぐ朝日を受けて(崎浦村を経て)旭町になった。
\item
  2021年06月02日16時24分の登録:
  「牛坂」を@hirono\_hideki @kk\_hirono @s\_hironoで検索 9件の該当 2021-06-02\_16:24の記録
  \url{https://kk2020-09.blogspot.com/2021/06/hironohidekikkhironoshirono92021-06.html} 
\end{itemize}

 牛坂という地名は、いえ人名を含めた言葉自体を見かけたことがないように思ったのですが、過去のツイートに牛坂八幡神社がありました。金沢市旭町とあります。

 たまにある配達先で小立野から旭町の方に降りることがあったのですが、決まって十全病院の前を通ったように思います。電柱にもよく十全病院というブリキの看板のようなものを見かけていたのですが、異様に思える古い建物の病院でした。

\begin{itemize}
\tightlist
\item
  2021年06月02日16時32分の登録:
  「十全病院」を@hirono\_hideki @kk\_hirono @s\_hironoで検索 24件の該当 2021-06-02\_16:32の記録
  \url{https://kk2020-09.blogspot.com/2021/06/hironohidekikkhironoshirono242021-06.html} 
\end{itemize}

 十全病院のことは、その場を市内配達のトラックで通るときぐらいしか思い出すことがなく、人に話したことも話を聞いたり情報を見かけたこともなかったのですが、とにかく強烈な印象が残っている古い病院の建物でした。

\begin{itemize}
\tightlist
\item
  十全病院 - Google マップ \url{https://t.co/zj3qt8aUz5}  〒920-1155
  石川県金沢市田上本町カ45−1
\end{itemize}

 十全病院については前にも調べているのですが、田上のことは記憶にありませんでした。十全病院を調べたきっかけは、金沢市近郊における精神病治療の歴史を紐解くことにつながったのですが、拘束で死亡させたという損害賠償の裁判がきっかけでした。

 控訴審で裁判の結果がひっくり返ったように記憶にありますが、その後の報道というのは見ておらず、判決が確定しているのかも不明です。ときわ病院だったと思い出しましたが、卯辰山の麓の常盤町が始まりであったと憶えています。

 令和3年3月31日付告発状で書いたのか憶えていないですが、金沢刑務所の拘置所での生活では、2階の浴場の窓から遠くに見えるヨーロッパ調の三角屋根に見える建物がとても印象的で、ちょうどその方角に現在の十全病院の建物があるようです。

 一人用の小さい浴場の小さい窓から遠くに見える風景だったので、ほんとうに三角屋根の建物だったのか不明ですが、いずれ社会に戻って自由になったら現地に行って確かめてみたいと思いながら行くことはなかったのですが、仕事では何度かその辺りに行っています。

\begin{itemize}
\item
  五木寛之の金沢を歩く (1) (2) \url{https://t.co/zHJbgbwzZd} 
  左の写真が金沢での最初の住まい、金沢市小立野五丁目の「東山荘」です。2棟立っており、どちらかはわかりませんが、二階の一番奥の部屋に住んでいたようです。「・・・私は、その刑務所の真裏のアパートの一室で、
\item
  五木寛之の金沢を歩く (1) (2) \url{https://t.co/zHJbgbwzZd} 
  配偶者である玲子さんの父親は岡良一氏(医学博士)で、昭和14年社会党で石川県議に初当選し、以後衆議院議員6期、金沢市長を2期務められた、金沢では有名な方です。もともとは十全病院の医院長だったようで、この東山荘の近くに病院があり
\end{itemize}

 五木寛之という人は名のしれた作家だと思いますが、石川県のローカル番組でも名前を見ることがあります、これまで何度か、なかにし礼と取り違えて勘違いしたことがあったのですが、有名な作家だと思っていたのが、名前の漢字を確認する検索で、作詞家だったことに気が付きました。

 作家でもあることを確認するために本をつけて調べたところ、「赤い月」という書名が出てきましたが、これは金沢刑務所の拘置所にいるとき、週刊誌の連載で何度か読んだような記憶があります。それもけっこう衝撃的な内容だったような気がします。満州が舞台だったような。

\begin{itemize}
\tightlist
\item
  赤い月 - Wikipedia \url{https://t.co/DpZyw6jtfK}  ¥\n
  『赤い月』(あかいつき)は、なかにし礼の小説。『週刊新潮』で1999年から2000年まで連載された。2001年に単行本化され、100万部に迫るロングセラーとなった{[}1{]}。
\end{itemize}

 1999年から2000年の連載というのは平成11年から平成12年になりますが、安藤健次郎さんの事件での二度目の金沢刑務所の拘置所での生活は、母親からの現金の差し入れも余りなかったように思うので、週刊誌を注文することも少なかったように思います。

 「戦前・戦中の旧満州を舞台に、一人の女性の生き方を描いた物語で、なかにしの実際の体験をもとにした自伝的小説である。」という部分があるぐらいで、内容についてほとんど記載がないのもWikipediaのページでは珍しく感じましたが、もともとミステリー性のようなものを感じていました。

 満州といえば、歴史的に深い関わりのがるのが731部隊と金沢大学ですが、戦後の石川県の医療界で活躍した人も721部隊の関係者が多かったという情報を読んでいます。令和3年3月31日付告発状の作成に取り掛かっている時期で、記述した部分もあるのかもしれません。

 私自身が、被告発人木梨松嗣弁護士と金沢大学の名誉教授で、個人的には石川県を代表する精神病医療という認識の松原病院の会長でもある山口成良金沢大学教授の人体実験にされたような事実関係もあるのですが、一ヶ月でしたがその精神鑑定で入院したのも金沢大学病院の精神科閉鎖病棟でした。

 100万部に迫るベストセラーというのもすごいと思いましたが、20年ほど前でもそれだけ満州に関心のある人がいたことにもなるのかもしれません。金沢刑務所の拘置所にいる頃、「大地の子」の本が読みたく映画もみたかったのですが、宇出津の図書館に貸出のDVDがあると、割と最近になって気が付きました。

 ページ内リンクを開くと、「この作品記事はあらすじの作成が望まれています。ご協力ください。(使い方)」とありました。私も告訴状の実質的意義と目的として、「さらば弁護士鉄道」と「弁護士泥棒神社の物語」を記録しているつもりです。

 この弁護士鉄道は、シベリア鉄道、能登鉄道の他、南満州鉄道の歴史的な要素も多分に含んでいます。

\begin{itemize}
\tightlist
\item
  赤い月(上) (文春文庫) \textbar{} なかにし 礼 \textbar 本 \textbar{}
  通販 \textbar{} Amazon \url{https://t.co/WEkyQwd1wz}  ¥\n
  2000年「長崎ぶらぶら節」で第122回直木賞を受賞(本データはこの書籍が刊行された当時に掲載されていたものです)
\end{itemize}

 ちょっと忘れていたのですが、そういえば映画「長崎ぶらぶら節」も原作が、なかにし礼となっていました。拘置所ではないですが、金沢刑務所での受刑中に独居房のテレビで視聴した映画でした。

 もうずいぶん前から作詞家や作曲家で有名になる人の名前は見かけていないのですが、昭和の時代は、ときに曲を歌う歌手以上に注目され話題になったのも作詞家や作曲家だったと思います。コンピュータでの制作が進歩しすぎて、趣が薄れたということもありそうな気がします。

 本当は被告発人若杉幸平弁護士をメインに始めた項目なのですが、隣の建物の金沢地方検察庁からずっと離れた内容の記述を行い、いくらか接近したのが金沢市田井町になります。

 パソコンの時計に秒数が表示されていることに気がついたのですが、17時45分です。時計に目を向けたときが44分でしたが、それで図書館が閉館する時間が間近だということに気が付きました。昨日の今頃の時間も図書館にいたのですが、金沢市内の住宅地図に軽く目を通していました。

 被告発人木梨松嗣弁護士の金沢市大手町7−34のごく周辺の住宅地図だけ見たのですが、係の人に金沢市内の古い住宅地図は2017年ぐらいしかないと言われました。資料で古いものは倉庫にしまってあるらしく、北國新聞の縮小版も平成6年とかになるとお願いして倉庫から持ってきてもらっています。

 Googleマップのストリートビューでは2012年の状況しか確認できなかったのですが、図書館にある金沢市の住宅地図なら被告発人若杉幸平弁護士の金沢市田井町の住所がただの住宅地のままなのか、金沢弁護士会のホームページにもあった若杉法律事務所の表示になっているのか確認できそうです。

 大病を患い弁護士としての活動を大幅に縮小したことで金沢地方検察庁の横にあった貸事務所を引き払ったという可能性もありそうですが、Googleにおける被告発人若杉幸平弁護士に関する情報は、異常と思えるほどの少なさ、乏しさで、専門の業者が依頼を受けて活動しているのかと勘ぐりたいぐらいです。

 繰り返しになりますが、平成11年の2月頃に被告発人大網健二を介して被告発人若杉幸平弁護士と会ったのが最初で最後になるのですが、かなり若く精力的に見えたという印象が残っています。ベテランで30代に見えたということはなかったですが、たぶん実際の年齢よりは若く見えているのだと思いました。

 令和3年3月31日付告発状には書いていないように思いますが、羽咋市に住んでいる頃、夜に金沢の東インターの近くのガソリンスタンドに入ったとき、暴力団対策のチラシのようなものに被告発人若杉幸平弁護士の名前を見たことがありました。平成16年から18年ぐらいの間であったように思います。

 Googleマップの表示を金沢市宝町のままにしていても、マウスの中ボタンのスクロールで少し拡大表示すると名前の表示が出てくるのが「越田商店」です。今までほとんど意識することのなかった建物というか店の名前ですが、この表示の建物に2階にあったのが若杉法律事務所だったと思います。

 さきほどストリートビューで拡大した表示をすると、ただのタイル張りではなさそうに見えた建物の外壁ですが、金沢地方検察庁の建物とも似たような模様と色になります。レンガのような色と模様になります。

 レンガで割と思い出すのが、731部隊の本で有名な「悪魔の飽食」ですが、なかの古い資料のような写真に、崩れたレンガのような建物があったと記憶しています。前回、731部隊について調べたときは、資料が偽物であると指摘され、手直しで編集し直したものが再出版になったような情報がありました。

 先程から少し思い出したように思いながら、まだ確認をしていないのですが、昨年2020年12月の10日過ぎ、テレビで石川四高の記念公園を見たことがきっかけで、他と同時進行的に731部隊について調べたような気がします。やたらと詳しい関係者の資料がでてきました。

 時刻は18時24分です。今日は珍しく起きた時間が朝の10時頃で、少ししてから昨日、どんたく宇出津店で買ってきたパンを2つ食べたのですが、一つは手作り感のある焼きそばパンで、もう一つがランチパックのようなシリーズのミートソースだったのですが、すごくお腹がいっぱいになりました。

 たいした量ではなかったのですが、不思議なほどお腹がいっぱいで、少し苦しくなり、昼は何も食べずに過ぎています。早めの夕食を考えていたのですが、今もあまり食べたいとは思いません。

\begin{itemize}
\tightlist
\item
  越田商店 - Google マップ \url{https://t.co/nuC4IYQvBP}  〒920-0937
  石川県金沢市丸の内6
\end{itemize}

 過去のツイートでも見かけていたことを思い出したのですが、被告発人若杉幸平弁護士の法律事務所があったと思われる建物の住所が金沢市丸の内となっていました。裁判所と同じ丸の内という住所ですが、裁判所以外の建物というのは数が少なそうです。

\begin{itemize}
\tightlist
\item
  丸の内 - Google マップ \url{https://t.co/Cua6oNIG6K} 
\end{itemize}

 実際にGoogleマップで丸の内を検索すると、金沢城公園がすっぽり中に入ったのですが、黒門前緑地とかある近江町市場の近くの一角も、住所が丸の内になるというのは意外でした。それと金沢城公園の敷地が、尾山神社と隣接し、背中合わせのようになっているのも意外な発見でした。

 尾山神社は、前を歩いて通ったような記憶しかないのですが、石川県では有名な前田家ゆかりの神社で観光のスポットにもなっているようです。前を歩いて通っても中に入ろうと考えたことはなかったのですが、周辺の町並みにも特徴があったと思います。

 越田商店のストリートビューを確認すると撮影日が2019年7月となっていました。過剰と思わるほど隙間のないカーテンがしまり、一階の店が営業を続けているとは思えないのですが、アルミサッシのガラス窓に「たばこ」と大きなシールが赤字で貼ってある様子です。

 もともと何を売っている店なのか気になることがあったのですが、金沢中央卸売市場の中にある高瀬商店と似たような品揃えのある店なのかもしれないと考えたことはありました。

 高瀬商店は間口が狭く、奥行きはそこそこあった小さな店でしたが、揚げ物の惣菜とかおにぎりとか手軽に食べられるものが揃えてあり、カップ麺にお湯を入れてくれたようにも思います。大人の駄菓子屋のような雰囲気もあったのですが、奥の方には週刊誌の漫画本も揃えてあったような記憶があります。

 近くの検察庁や裁判所の職員が、カップ麺にお湯を入れてもらって食べるとは、想像もできなかったですが、その場所に適したニーズで便利屋にも近いような雑多なものを売っていそうな店という印象がありました。

 建物が3階建てになっていることも昨日辺りに同じGoogleマップのストリートビューで気がついたのですが、被告発人若杉幸平弁護士と話をしたのは、かなり雑然とした印象も受けた正面の窓のある部屋でした。

 奥の方が訪問者の待合室のようにもなっていたのですが、右手に机ではなくラックで組んだようなパソコンの前に座り、タイピングを続ける女性の姿があり、それも印象的でした。タイピングの技術も目をみはるようなものがあったと思うのですが、集中力を含めたその場の雰囲気が異質な世界に映りました。

 ずっと前には、東電OL殺害事件の被害者の顔写真に、どこか似ている感じの女性だったという表現で記述をしたという記憶もあるのですが、卓越した事務作業の能力ともに、ミステリー性を感じるところが強く、それが謎の多いとされた事件との共通項になっていたとも思います。

 たぶん貸事務所だったのだと思いますが、2階にある被告発人若杉幸平弁護士の法律事務所は、個性的なこだわり感の強い、喫茶店や小さなレストランのような雰囲気も感じられました。実際に、そのような店に入ったような体験もあったので、それと重なって見えたのだと思います。

 本件告発事件の一つの鍵を握るのが、この被告発人若杉幸平弁護士で、まるで古本屋にあるミステリー小説の世界のようですが、雑然と置かれた書類のようなものも、古本屋に置かれた資料のように見えたのだと思います。部屋の中がかなり薄暗く感じられたのも印象的でした。

 事務員のような女性がパソコンの前に座る部屋は対照的に明るかったのですが、少なくとも女性事務員がいた部屋の左手と奥は、一面がガラス窓でカーテンなど最初からなかったような気がするぐらいです。陽光の差し込みぐらいでへ眩しさで仕事に支障もありそうですが、何かと印象深い法律事務所でした。

 今改めて考えると、女性事務員は被告発人若杉幸平弁護士の妻だった可能性も一応ありそうですが、法律事務所内に人は被告発人若杉幸平弁護士とその女性の2人だけではなかったと思います。

 被告発人若杉幸平弁護士の法律事務所に男性がいたという記憶はないのですが、他に女性事務員がいたとしても他に印象に強すぎることがあり、記憶に残らなかったのかもしれません。他に弁護士らしい人の姿を見たという記憶もありません。けっこうワンマンにも思える雰囲気が感じられました。

 Googleマップで上空からの写真を見ると、さほど奥行きのない建物で、やはり部屋は2つだけだったのかと思ったのですが、撮影された写真が歪んでいるのかとも思うのですが、隣の金沢地方検察庁の建物と比較すると、建物の横幅が5分の2には相当しそうなスペースです。

 たまたまなんでしょうが金沢地方検察庁の建物は真下に裏側の駐車場があり、大きなガラスで2部屋分に相当するスペースの長さが駐車場の白線で乗用車5台半分に相当するとわかります。越田商店の建物の長さと同じぐらいに見えます。

 今まで考えたことがなかったのですが、駐車場での一台分の幅の長さを調べると2.5メートルという情報がありました。5台半だと13.7メートルほどになりそうです。

 Googleマップで上空から見ると、建物の一辺の長さが金沢地方検察庁の建物と変わらないように思えた金沢弁護士会の建物ですが、余り拡大しない状態だと建物に「法テラス石川」とだけ表示されているのですが、かなり拡大すると金沢弁護士会という表示が一緒になりました。

 横並びで見ると金沢地方検察庁の建物と重なる部分があるのですが、私の拙い記憶では、今の金沢弁護士会の建物の辺りに金沢家庭裁判所の建物があり、金沢地方裁判所の建物との間に、平屋のお堂のような金沢弁護士会の古い建物があったような記憶です。

 その金沢弁護士会の古い建物は、日本で最後に陪審員裁判が開かれた法廷としてネットのニュースになっていました。もう10年ほど前になると思いますが、それでも2009年3月15日に羽咋市から戻った後に、見たニュースであったように思います。

\begin{itemize}
\tightlist
\item
  【印刷用】最後の陪審法廷跡、姿消す/来年度取り壊し、金沢 \textbar{}
  全国ニュース \textbar{} 四国新聞社 \url{https://t.co/T6UuEoc7jT}  2008/12/27
  17:09
\end{itemize}

 内容は同じかと思いますが、ページのデザインが記憶とは全く違い、やたらと白地のスペースが多い珍しい記事です。記事の日付が2008年12月27日となっていますが、私が検索で見つけたのはだいぶん後日であったとも考えられます。それでも2010年4月にTwitterを始める前かもしれません。

 久しぶりに見た前の金沢弁護士会の建物の写真は、思っていたよりずいぶん小さく見えたのですが、屋根が瓦に見えるものの外壁がモルタルなのか、それともコンクリート造りの建物に瓦屋根を乗せているだけなのか微妙に思える写真です。

 今回は、いっそう、火葬場にある礼拝所のようなお堂の建物に思えてきました。中に入ったのは1回だけでしたが、窓口で話をして、そこから行った先が、なんとか美智子弁護士の法律事務所でした。ときどき思い出していたのですが、上の名前が思い出せなくなっています。

 Googleで「金沢弁護士会 美智子」という検索を実行する直前に思い出し、検索で確認した形になるのですが、畠山美智子弁護士でした。畠山というのは能登の羽咋市の辺りで由緒のある名前と聞いたこともあるのですが、室町時代か戦国時代の守護大名の名前になっていたと思います。

 個人的な感覚になりますが、羽咋市は能登というイメージがあまりなく、金沢市にも近いですが河北郡やかほく市と同じグループというイメージがあります。実際は、羽咋市より金沢市に近い羽咋郡宝達志水町から能登地方になるという話です。

 奥能登では畠山という名前を1つの例外を除いて聞いた覚えがないのですが、その1つの例外というのも婿養子に行ったという話でした。けっこう本件告発事件とも接点があり、参考になる人物と思うのですが、プライバシーの問題もあり、ここでは深入りしません。書面の記録にはあるはずです。

 昨日Googleマップをみていて思ったのですが、金沢市大手町というのは、思っていた以上にずいぶん広い範囲になっていました。全国的にも大手町という地名は、官庁街のようなイメージがあるのですが、住宅地の住所もかなり大手町となっていそうな感じでした。

\begin{itemize}
\tightlist
\item
  2021年06月02日20時17分の登録:
  「畠山美智子」を@hirono\_hideki @kk\_hirono @s\_hironoで検索 19件の該当 2021-06-02\_20:17の記録
  \url{https://kk2020-09.blogspot.com/2021/06/hironohidekikkhironoshirono192021-06.html} 
\end{itemize}

 そのままテレビのドラマに出てきそうなおばさん弁護士という印象のあった畠山美智子弁護士でしたが、完全に私を軽くあしらったようなその場しのぎの対応が際立っていました。かんたんに言えば、どうでもいいから帰ってくれ、という感じでしたが、事務員の姿はなく二人っきりだったような記憶です。

 割と広い部屋でしたが、住所が大手町となっていたような記憶もあるものの、市内配達の仕事でけっこう行ったことのある近江町市場の近くという感じでした。近江町市場も北陸新幹線の開通で、知名度が全国的になったという感じですが、仕事で行っていた頃は、余り特別な場所という気はしませんでした。

 近江町市場で買い物をしたという記憶はなく、唯一が平成3年の11月頃、被告発人安田敏と二人で入った店で回転寿司だったような気がします。個人的には、少し離れた場所にある横安江町のアーケード街の方が印象に残っています。

 詳しいことはしりませんが、金沢駅の方に向かって歩いているときに、東別院、西別院という大きなお寺を見たという体験もあるので、もともとは寺院の参道として発展した歴史もあるのかもしれないと考えた横安江町のアーケード街ですが、映画「不敵にわらう男」のロケ地にもなっていました。

\begin{itemize}
\tightlist
\item
  拳銃無頼帖 不敵に笑う男 \textbar{} 映画 \textbar{} 日活
  \url{https://t.co/uCgbqXSGvr}  製作年:1960公開年月日:1960/8/6
\end{itemize}

 「笑う」という漢字は「嗤う」なのかもしれないと思ったのですが、製作年が昭和35年で、公開が同年8月6日となっています。

 金沢の横安江町アーケード街の女性専門の洋服店から、近くでお祭りをやっているというようなセリフから、いきなり宇出津のあばれ祭りが出てくるのですが、7月の7日8日と決まっていた時代と思います。

 その間に、香林坊で太鼓を叩く場面があったような気もするのですが、吉永小百合の恋人役が靴磨きをする場面も香林坊となっていながら、横安江町とは違い、まったく記憶にない香林坊の風景でした。香林坊というよりは竪町に近いような風景に思えました。

 「拳銃無頼帖 不敵に笑う男」は、令和3年3月31日付告発状の郵送の後、Amazonプライムビデオで視聴したのですが、部分的には以前にYouTubeでも動画をみていました。映画の筋に全く関わりがなく、宇出津のあばれ祭りと小木港のとも旗祭りが出てきたことは、全部を視聴したことで理解ができました。

 「拳銃無頼帖 不敵に笑う男」は、令和3年3月31日付告発状の郵送の後、Amazonプライムビデオで視聴したのですが、部分的には以前にYouTubeでも動画をみていました。映画の筋に全く関わりがなく、宇出津のあばれ祭りと小木港のとも旗祭りが出てきたことは、全部を視聴したことで理解ができました。

 前にもGoogleの検索結果で一度は見たことのあるページだと思ったのですが、ロケ地の説明文に「金沢刑務所(=本編冒頭タイトルバック。「北陸刑務所」という設定)」とあることに気が付きました。県外の刑務所の建物と思っていたのですが、小立野5丁目にあったという金沢監獄なのかもしれません。

\begin{itemize}
\tightlist
\item
  \url{https://t.co/hEEXK0Lxkh:}  拳銃無頼帖 不敵に笑う男を観る \textbar{}
  Prime Video \url{https://t.co/bkDnAAPHw1}  ¥\n 1時間23分 ¥\n 1960
\end{itemize}

 はっきりした記憶はないのですが、Amazonプライムに会員登録したのが4月の10日すぎで、「拳銃無頼帖 不敵に笑う男」を視聴したのは4月14日辺りになるかと思います。視聴期間の目安がまだわかっていないのですが、今も視聴できることを確認しました。

 ただの昔のどこかの刑務所の建物と思っていたのが、金沢刑務所の昔の建物だと知って一気に注目度も高まったのですが、刑務所の塀や外壁の殆どがレンガ造りになっていたことに気が付きました。どことなく海に近いような場所だと思っていたのですが、松の木で防風林をイメージしていたようです。

 北陸刑務所を出所した次の場面が海岸線を走る機関車の映像でトンネルの上の山に白い灯台が見えますが、これもどこか県外のロケ地と思っていたものが、県外には違いないですが能登半島になる富山県の雨晴海岸に思えてきました。

 昨年になるのか一昨年になるのか思い出せないですが、7月か8月に雨晴海岸に行ったとき、線路と電車を見ていたのですが、どこまで行く電車なのかと考えながら、七尾駅に富山方面への乗り換えはなく、鉄道もつながっていないはずだと漠然と考えていました。

\begin{itemize}
\tightlist
\item
  雨晴海岸 - Google マップ \url{https://t.co/nVAOATXPFz} 
\end{itemize}

 雨晴海岸が高岡市になることを確認しましたが、電車の線路は氷見駅が終着駅と始発駅になっているようです。3ヶ月ほど前になるのかTwitterのトレンドでも雨晴海岸が能登半島になるのかと話題になっていました。

 距離的に近いだけあって雨晴海岸の写真は立山連峰の冠雪がくっきりきれいに見えていますが、一昨日の5月31日月曜日に、蛸島港に行ったとき、一部でしたが立山連峰の冠雪が見えて、珠洲市内から立山連峰を見たのは初めてのように思いました。

 ネットの写真では高性能のカメラで撮影したと思える珠洲市内から富山県内の夜景が見える写真も数年前に見たことがあり、珠洲市の方が富山湾にせり出す地形となっているので、対岸との距離も短く近くに見えやすのかと考えていたのですが、珠洲市内から立山連峰のことは考えたこともなかったのです。

 Twitterでは富山市内からの立山連峰の景色もトレンドで見かけたことが2,3回あるのですが、長距離トラックの仕事で富山県内を走行することは多かったのですが、遠くの山の方に目を向けることはなかったのか、立山連峰を見たという記憶は全く残っていません。

 宇出津からも立山連峰が見えることは多いですが、空気が澄んだ寒い季節にきれいに見えることが多いとは聞いていました。平成3年の11月頃から平成4年の3月は、七尾市から国道160号線の海岸線をよく通行していたのですが、より身近に見えたはずの立山連峰を見たという記憶が、これも全く残っていません。

 ただ、その国道160号線の海岸線を走りながらこの海の向こうにはシベリアがあって、太田裕美の歌にあるシベリア鉄道があってヨーロッパにつながっているのかと思いを馳せることはあったと記憶に残っています。

 時刻は21時33分です。冷蔵庫に日曜日の昼辺りにすりおろしお酢をまぜていた山芋を見つけたのですが、変色などしておらずそのまま食べられそうです。2回ぐらい食べようと思いながら忘れていたのですが、さすがに口にするのはリスクがあると思い捨てるつもりです。

 すりおろした山芋は昼にすったものを夜に食べたことはあったと思いますが、一日置いたものを食べたことはなかったと思います。大きくなく変色していないものをみつかけると買ってくることのある山芋ですが、お好み焼きに入れることが多い食材です。

 お好み焼きを作るのも2ヶ月に1回ぐらいのペースだと思いますが、安く仕上がる反面、食べ過ぎたり、洗い物に時間がかかるという難点があって、これまで敬遠しがちでもありました。

 雨晴海岸へは、朝に宇出津から出発した車に乗せてもらって金沢市内から向かったのですが、途中で小矢部市のアウトレットとかいう大きな店に立ち寄り、七尾市に向かったときは自動車専用道路でしたが、その前に国道160号線沿いの氷見警察署を助手席から眺めていました。かなり古い建物でした。

 弁護士鉄道の歴史には記念碑のような冤罪事件がその富山県氷見市であり、強姦事件として福井刑務所で服役したあとでの冤罪発覚だったと思いますが、弁護士らが横から手柄を横取りしたようなかたちで、確か鳥取県警で連続強姦事件の被疑者が自供したことでの発覚が端緒にあったと憶えています。

 その富山氷見強姦冤罪事件は、まるで弁護士一座の地方巡業講演で、チンドン屋のラッパ吹きが集まって曲芸をしているように思える弁護士鉄道の歴史的一幕がありました。

 現在の私の記憶のままで書くとまだ10年は経っていないように思うのですが、その富山氷見強姦冤罪事件が発覚し、テレビでニュースになっていた時期は、2年か3年ほど続けて、NHKの20時45分から始まるニュース石川845という番組だったようにも思いますが、富山市のチンドンコンクールをみていました。

\begin{itemize}
\tightlist
\item
  〈〈〈 2021/06/02 21:59:03 Linux Emacs: 〈〈〈
\end{itemize}

\hypertarget{section-10}{%
\paragraph{}\label{section-10}}

\hypertarget{ux88abux544aux767aux4ebaux677eux5e73ux65e5ux51faux7537}{%
\subsubsection{被告発人松平日出男}\label{ux88abux544aux767aux4ebaux677eux5e73ux65e5ux51faux7537}}

\hypertarget{ux88abux544aux767aux4ebaux6c60ux7530ux5b8fux7f8e}{%
\subsubsection{被告発人池田宏美}\label{ux88abux544aux767aux4ebaux6c60ux7530ux5b8fux7f8e}}

\hypertarget{ux88abux544aux767aux4ebaux6885ux91ceux535aux4e4b}{%
\subsubsection{被告発人梅野博之}\label{ux88abux544aux767aux4ebaux6885ux91ceux535aux4e4b}}

\hypertarget{ux88abux544aux767aux4ebaux5b89ux7530ux7e41ux514b}{%
\subsubsection{被告発人安田繁克}\label{ux88abux544aux767aux4ebaux5b89ux7530ux7e41ux514b}}

\hypertarget{ux88abux544aux767aux4ebaux5b89ux7530ux654f}{%
\subsubsection{被告発人安田敏}\label{ux88abux544aux767aux4ebaux5b89ux7530ux654f}}

\hypertarget{ux88abux544aux767aux4ebaux6771ux6e21ux597dux4fe1}{%
\subsubsection{被告発人東渡好信}\label{ux88abux544aux767aux4ebaux6771ux6e21ux597dux4fe1}}

\hypertarget{ux88abux544aux767aux4ebaux591aux7530ux654fux660e}{%
\subsubsection{被告発人多田敏明}\label{ux88abux544aux767aux4ebaux591aux7530ux654fux660e}}

\hypertarget{ux88abux544aux767aux4ebaux6d5cux53e3ux5353ux4e5f}{%
\subsubsection{被告発人浜口卓也}\label{ux88abux544aux767aux4ebaux6d5cux53e3ux5353ux4e5f}}

\hypertarget{ux88abux544aux767aux4ebaux5927ux7db2ux5065ux4e8c}{%
\subsubsection{被告発人大網健二}\label{ux88abux544aux767aux4ebaux5927ux7db2ux5065ux4e8c}}

\hypertarget{ux88abux5bb3ux8005ux5b89ux85e4ux6587ux5f53ux664221ux6b73}{%
\subsubsection{被害者安藤文(当時21歳)}\label{ux88abux5bb3ux8005ux5b89ux85e4ux6587ux5f53ux664221ux6b73}}

\hypertarget{ux5b89ux85e4ux5065ux6b21ux90ceux3055ux3093}{%
\subsubsection{安藤健次郎さん}\label{ux5b89ux85e4ux5065ux6b21ux90ceux3055ux3093}}

\hypertarget{ux5e73ux621011ux5e74ux306eux50b7ux5bb3ux4e8bux4ef62021ux5e746ux670822ux65e5ux306bux898bux3064ux3051ux305fux4e0aux544aux5be9ux6700ux9ad8ux88c1ux56fdux9078ux5f01ux8b77ux4ebaux5c71ux53e3ux6cbbux592bux5f01ux8b77ux58ebux304bux3089ux306eux624bux7d19}{%
\paragraph{平成11年の傷害事件、2021年6月22日に見つけた上告審(最高裁)国選弁護人、山口治夫弁護士からの手紙}\label{ux5e73ux621011ux5e74ux306eux50b7ux5bb3ux4e8bux4ef62021ux5e746ux670822ux65e5ux306bux898bux3064ux3051ux305fux4e0aux544aux5be9ux6700ux9ad8ux88c1ux56fdux9078ux5f01ux8b77ux4ebaux5c71ux53e3ux6cbbux592bux5f01ux8b77ux58ebux304bux3089ux306eux624bux7d19}}

安藤健次郎さん/平成11年の傷害事件、2021年6月22日に見つけた上告審(最高裁)国選弁護人、山口治夫弁護士からの手紙

告発の事実/
被告発人らの関与と役割及び具体的犯罪事実/安藤健次郎さん/平成11年の傷害事件、2021年6月22日に見つけた上告審(最高裁)国選弁護人、山口治夫弁護士からの手紙

\begin{itemize}
\tightlist
\item
  告発の事実/
  被告発人らの関与と役割及び具体的犯罪事実/安藤健次郎さん/平成11年の傷害事件、2021年6月22日に見つけた上告審(最高裁)国選弁護人、山口治夫弁護士からの手紙
  \textbar{} 告発・非常上告\_2021\金沢地方検察庁御中
  \url{http://hirono-hideki.info/wp/?p=1179} 
\end{itemize}

:CATEGORIES: @kanazawabengosi \#金沢弁護士会 @JFBAsns
日本弁護士連合会(日弁連) \#法務省 @MOJ\_HOUMU \#安藤健次郎さん
\#山口治夫弁護士

\begin{itemize}
\item
  〉〉〉 Linux Emacs: 2021/07/16 09:47:24 〉〉〉
\item
  TW kk\_hirono(刑事告発・非常上告_金沢地方検察庁御中) 日時:
  2021/07/15 10:10:35 URL:
  \url{https://twitter.com/kk\_hirono/status/1415478870682214403} 
  \textgreater{} - 告発の事実/
  被告発人らの関与と役割及び具体的犯罪事実/安藤健次郎さん/平成11年の傷害事件、2021年6月22日に見つけた上告審(最高裁)国選弁護人、山口治夫弁護士からの手紙
  \textbar{} 告発・非常上告\_2021\金沢地方検察庁御中
  \url{https://t.co/St4cdRtuGJ} 
\end{itemize}

 昨日の最後のツイートの時刻を確認したのですが、まる一日近い中断になっていたことを確認しました。タイムラインを開く前に通知を開いたのですが、そこにおすすめツイートがあって、返信となっているツイートを含めいくつかリツイートしました。

\begin{quote}
《引用の始まり》
\end{quote}

\begin{quote}
ガツ@gatsu73法律家・大阪の弁護士(73期) / 同志社法 / 京大ロー /
法律・司法試験・司法修習・就活関連のことをつぶやいてます /
DMも見たら返します 質問箱
https://peing.net/ja/naranoshika\_?event=0\ldots 大阪gatsulaw.blog.jp2016年11月からTwitterを利用しています658
フォロー中2,943 フォロワー
\end{quote}

\begin{quote}
《引用の終わり》
\end{quote}

\begin{itemize}
\tightlist
\item
  ガツさん (@gatsu73) / Twitter \url{https://twitter.com/gatsu73} 
\end{itemize}

 上記のアカウントのツイートですが、前にも通知で「おすすめツイート」を見たかもしれません。フォロー返ししている弁護士アカウントなのかと思っていたのですが、そういうことではなさそうです。

〉〉〉 kk\_hironoのリツイート 〉〉〉

\begin{itemize}
\tightlist
\item
  RT
  kk\_hirono(刑事告発・非常上告_金沢地方検察庁御中)|gatsu73(ガツ)
  日時:2021-07-16 10:07/2021/07/15 20:50 URL:
  \url{https://twitter.com/kk\_hirono/status/1415840582413406209} 
  \url{https://twitter.com/gatsu73/status/1415639831820914689} 
  \textgreater{} さすが \textgreater{}
  【速報】「表現の不自由展かんさい」予定通り7月16日から開催へ 施設側の即時抗告を大阪高裁が棄却(MBSニュース)
  - Yahoo!ニュース \url{https://t.co/DhQhWzHgdV} 
\end{itemize}

 タイムラインに2つ目にあるツイートをリツイートしましたが、「さすが」と論評を加えた、気になっていた社会問題のツイートでした。大阪での表現の不自由展です。開催すれば弁護士の仕事が増えそうだと考えていました。警察もです。

〉〉〉 kk\_hironoのリツイート 〉〉〉

\begin{itemize}
\tightlist
\item
  RT
  kk\_hirono(刑事告発・非常上告_金沢地方検察庁御中)|FaXa9n(マッドハンド)
  日時:2021-07-16 10:13/2021/07/15 20:12 URL:
  \url{https://twitter.com/kk\_hirono/status/1415842063606706181} 
  \url{https://twitter.com/FaXa9n/status/1415630370104512515} 
  \textgreater{} マドハンド \url{https://t.co/XTMPdA1wPo} 
  \url{https://t.co/R0RysTNWox} 
\end{itemize}

 今度は通知に「マッドハンドさんにフォローされました」が入っていました。直前に図書館から電話があったのですが、予約した本が一冊入ったという連絡で、予約した2冊のうち、PHPの方ということでした。この本はじっくり時間を掛けて読む必要があると考えている方でした。

 通知で見かけて最初にリツイートをしたのは次のツイートです。

\begin{itemize}
\tightlist
\item
  TW gatsu73(ガツ) 日時: 2021/07/15 16:45:41 URL:
  \url{https://twitter.com/gatsu73/status/1415578299921690624} 
  \textgreater{} え?なにが懲戒事由なのかまじでわからないんだけど。
  \url{https://t.co/hVxC3Cs8An} 
\end{itemize}

 撮影した書面と思っていたのですが、見直すと上に「公告」とあり、見出しのような部分にも「懲戒処分の公告」とあります。これが日弁連の自由と正義のページであれば、初めて目にしたことになるかもしれません。1ページが2つの段組みで線囲いしてあるのも珍しく思いました。

 言葉を思い出したのですが、枠線と呼ばれていたように思います。線囲いは文字列の修飾に使われていたかもしれません。HTMLではインライン要素とブロック要素の違いになりそうです。

 依頼者と協議しないまま報酬の割合を5%から8%にしたということですが、受領した解決金から報酬金を差し引き残額を一方的に振り込んだとあります。弁護士の報酬というのは報酬基準が撤廃され事実上青天井なっているとも聞きますが、弁護士にも言い分はあるのでしょう。戒告で済んでいるようですが。

 これから満を持して取り上げるところだったのが、山口治夫弁護士になります。しばらく前にも集中的に取り上げたことがありますが、ネットで意外な情報が見つかったことを中心にご紹介したかと思います。

\begin{lstlisting}
base ❯ hatena-log-search  山口治夫弁護士
\end{lstlisting}

\begin{itemize}
\tightlist
\item
  ./hatena-diary\_20111118:{[}再審請求の資料{]}平成12年7月31日付\_山口治夫弁護士(東京弁護士会)からの手紙
  - 廣野秀樹 - Picasa ウェブ アルバム
\item
  ./hatena-diary\_20111118:\emph{1321607785}{[}再審請求の資料{]}平成12年7月31日付\_山口治夫弁護士(東京弁護士会)からの手紙
  - 廣野秀樹 - Picasa ウェブ アルバム\textbackslash end\{lstlisting\}
\end{itemize}

 はてなブログでは2件のみの該当で、それも2011年と比較的新しいものになります。2006年12月から始まっているはずなので、その間、一度も山口治夫弁護士の名前を「はてなダイヤリー」に記述しなかったことになりそうです。

\begin{itemize}
\tightlist
\item
  2021年07月16日10時57分の登録:
  「山口治夫弁護士」を@hirono\_hideki @kk\_hirono @s\_hironoで検索 151件の該当 2021-07-16\_10:57の記録
  \url{https://kk2020-09.blogspot.com/2021/07/hironohidekikkhironoshirono1512021-07.html} 
\end{itemize}

 Twitterの方は151件となっています。内容を確認していないですが、何年間も山口治夫弁護士の名前を記述しないことがあったと思います。重要視しない期間が長く多かったことをそのまま意味します。

2011-11-16 05:27:52
""``記事の中でいろいろな司法関係者の意見が紹介されています。山口治夫弁護士は、「弁護士も容疑者を追及する側に回ったら、現在の刑事司法は成り立たない」と言っていますが、誰も弁護士に容疑者追及に協力しろとは言っていま\ldots{}''``\ldots{}
\url{http://tmblr.co/ZuVOrwBzjObM''} 
\url{https://twitter.com/hirono\_hideki/status/136540945322745856} 

\begin{itemize}
\tightlist
\item
  hirono-hideki,
  記事の中でいろいろな司法関係者の意見が紹介されています。山口治夫弁護士は、「弁護士も容疑者を追及する側\ldots{}
  \url{https://t.co/myWlm928BR} 
\end{itemize}

 ページタイトルには含まれていないですが、tumblrというネットサービスでした。一時期使っていたのですが、軽く5年以上は全く投稿していないように思います。全文が記事の引用のようですが、一番下に引用先のリンクがあり、「「黙秘の勧め」は捜査妨害」とあります。

 2,3日前にも少し見かけていたように思うのですが、余り意識しておらず、記憶が減退していることもあるのか、山口治夫弁護士と黙秘という組み合わせは、新局面の扉が開かれたような新鮮さを感じます。リンクはまだ開いておらず、どういうページなのか想像も出来ません。

\begin{quote}
《引用の始まり》
\end{quote}

\begin{quote}
 11月13日の産経新聞「ヒ素保険金事件 弁護団に疑問の声」と言う特集記事を読みました。折角の特集でしたが弁護士業界に対する遠慮が目立ち、物足りない印象は否めませんでした。弁護士を批判するのに弁護士業界の反論にスペースをとりすぎです。

 弁護士にも真相解明の「消極的な協力義務はある」と言う見出しになっていますが、協力義務があるかどうかの問題ではありません。特集記事のポイントがずれています。問題は弁護士が被疑者に「黙秘を勧める」ことは正当な弁護活動の範囲か、それともその範囲を逸脱した捜査の妨害か、捜査の妨害であるとすれば、弁護士による捜査の妨害は許されるかどうかと言うことです。この点に的を絞って掘り下げた議論をすべきであろうと思います。弁護士業界の人の意見だけではなく、現行の司法制度の枠にとらわれずもっと一般国民の意見を紹介すべきだと思います。

 記事の中でいろいろな司法関係者の意見が紹介されています。山口治夫弁護士は、「弁護士も容疑者を追及する側に回ったら、現在の刑事司法は成り立たない」と言っていますが、誰も弁護士に容疑者追及に協力しろとは言っていません。そして、現在の司法制度が成り立たなくなると言うことは、弁護士の捜査妨害批判に対する反論にはなりません。
\end{quote}

\begin{quote}
《引用の終わり》
\end{quote}

\begin{itemize}
\tightlist
\item
  「黙秘の勧め」は捜査の妨害
  \url{http://www.kcn.ne.jp/\textasciitilde}  ca001/B10.htm
\end{itemize}

 平成10年11月18日とあるのが投稿日のようです。ヒ素とあるので、30秒ほど和歌山カレー事件のことかと思ったのですが、和歌山カレー事件は保険金目的とはなっていなかったはずです。夫の方は保険金目的の事件で服役しているので、違っていると断定することもできそうにないです。

 先月になるのか関空連絡橋のことがあり、それがきっかけで和歌山カレー事件について新たに知る情報が一気に増えたのですが、その中に林真須美死刑囚の夫の服役が、滋賀刑務所とあったように思います。滋賀刑務所の存在はしっていましたが、服役したという話は初めてだったと思います。

\begin{itemize}
\tightlist
\item
  新聞の宅配問題 \url{http://www.kcn.ne.jp/\textasciitilde}  ca001/index.html
\end{itemize}

 ずいぶん懐かしい感じのデザインのホームページですが、ここにも平成10年6月6日という日付がページフッターのような位置にあり、10年以上更新されていない感じです。URLにもプロバイダーが提供するホームページスペースのようなユーザーの記号があります。

 \textasciitilde ca001とあるのがその部分ですが、最近は滅多にみかけないものです。

 時刻は11時31分です。ちょうど能登町の告知放送が終わったところですが、今度は恋路でのツキノワグマの目撃情報でした。7月16日と聞こえましたが、時刻の方は聞き取ることが出来ませんでした。14日の19時30分が同じ能登町の藤波でしたが、随分離れた場所で、恋路は珠洲市に隣接しています。

\begin{quote}
《引用の始まり》
\end{quote}

\begin{quote}
クマが確認された場所では、15日朝から猟友会や町の職員が設置された捕獲用のおりを確認するなどし、警戒を強めています。

クマは、14日最初に発見されてから2時間後、3キロほど離れた場所でも確認されています。

また15日も午前9時ごろに町内で別の個体とみられるクマの目撃情報がありました。

能登町では、防災無線を使って住民に注意するよう呼びかけています。
\end{quote}

\begin{quote}
《引用の終わり》
\end{quote}

\begin{itemize}
\tightlist
\item
  能登で相次ぐ 石川・能登町でクマを目撃(MRO北陸放送) - Yahoo!ニュース
  \url{https://news.yahoo.co.jp/articles/155371b1304d32e831aed5f6fc50253dff089013} 
\end{itemize}

 「クマは、14日最初に発見されてから2時間後、3キロほど離れた場所でも確認されています。」と「また15日も午前9時ごろに町内で別の個体とみられるクマの目撃情報がありました。」という2件の目撃情報は、能登町の告知放送になく、テレビもつけていないので知りませんでした。

\begin{itemize}
\tightlist
\item
  道横切り山の中へ\ldots ドラレコに『クマ』の姿 体長約1mで子グマか
  石川・能登町で4件の目撃情報 \textbar{} 石川テレビニュース \textbar{}
  石川テレビ放送 \url{https://t.co/UIthUqke0g}  \url{https://t.co/GFBDTBNlBN} 
\end{itemize}

 動画の一部は、昨日の夕方辺りにみていたのですが、思っていた場所と雰囲気が違って見えます。辺田の浜から矢波に行く山沿いの道路で、農免道路と呼ばれていたかもしれません。昭和56年頃にはバイクで通行をしていました。上り坂の右手に倉庫のような建物があるのが心当たりのないところです。

 ドライブレコーダーにクマが撮影されたという話も初めて見かけたように思うのですが、そういう記録が残る時代になったものとあらためて思いました。

\begin{quote}
《引用の始まり》
\end{quote}

\begin{quote}
映像は、14日午後5時15分ごろ、能登町藤波の道路を走っていた地元の社会福祉法人の車のドライブレコーダーによってとらえられたものです。黒い動物が、道路を右側から左側に走って渡る様子が確認できます。動物が目撃されたのは、能登町内の海に近い場所ですが、道路の周辺は木に囲まれています。以前は、奥能登地方にはクマは生息していないと考えられていましたが、このところ目撃が相次ぐようになり、能登町でも平成28年以降、毎年、数件のクマの目撃情報が寄せられています。
\end{quote}

\begin{quote}
《引用の終わり》
\end{quote}

\begin{itemize}
\tightlist
\item
  奥能登でも相次ぐ クマの目撃情報|NHK 石川県のニュース
  \url{https://www3.nhk.or.jp/lnews/kanazawa/20210715/3020008718.html} 
\end{itemize}

 能登町の告知放送では14日の19時半頃と聞こえていたように思っていたのですが、NHKの石川県ニュースでは17時15分頃とあります。能登町でのクマの目撃は平成28年以降というのも初めて知った情報ですが、告知放送で最初に聞いたと記憶にあるのが、旧柳田村の笹川でした。

\begin{lstlisting}
base ❯ time twilog-serch 笹川|grep 熊
\end{lstlisting}

\begin{itemize}
\tightlist
\item
  ./kk\_hirono2021-07-16\_115003.csv:2018-12-12 03:42:48
  ``昨年だったように思いますが、能登町の笹川で熊の目撃があり、今年の秋だったと思いますが、能登町の藤ノ瀬で熊の目撃があったという告知放送がありました。それはテレビのニュースにはなっていないようでした。実際に見た人がいるのでしょうが、撮影などの記録は出来なかったようです。''
  \url{https://twitter.com/kk\_hirono/status/1072562361150640128twilog-serch} 
  笹川 0.02s user 0.02s system 101\% cpu 0.039 total
\end{itemize}

 「wc -l hirono\_hideki2021-07-16\_095002.csv
kk\_hirono2021-07-16\_120003.csv
s\_hirono2021-07-16\_110004.csv」というコマンドの実行結果が「472751
合計」なのですが、そのうちに1件の該当で、調べるのに掛かった時間が0.039秒という速さです。

\begin{itemize}
\item
  恋路 クマ - Twitter検索 / Twitter \url{https://t.co/UAo20nCH1s} 
\item
  恋路 ツキノワグマ - Twitter検索 / Twitter \url{https://t.co/qhROGxz1ti} 
\end{itemize}

〉〉〉 kk\_hironoのリツイート 〉〉〉

\begin{itemize}
\tightlist
\item
  RT
  kk\_hirono(刑事告発・非常上告_金沢地方検察庁御中)|jichitai\_ISKW(自治体情報-石川県)
  日時:2021-07-16 12:35/2016/06/29 09:24 URL:
  \url{https://twitter.com/kk\_hirono/status/1415877768626249731} 
  \url{https://twitter.com/jichitai\_ISKW/status/747948749633163266} 
  \textgreater{} 能登町/ クマにご注意!!:
   6月26日(日)午後1時頃、笹川地内県道宇出津町野線沿いで、また、6月27日(月)午後6時15分頃、国重と源平の境界付近の町道でツキノワグマと思われる目撃情報が寄せられました。\ldots{}
  \url{https://t.co/9A2qqAmdyr} 
\end{itemize}

 次は別のアカウントの通知におすすめで出てきたツイートになります。ちょくちょく見かけている匿名弁護士アカウントですが、間違いなくブロックされていると思います。

※ @kk\_hironoのアカウントがブロックされ,リツイートに失敗したツイート

\begin{itemize}
\tightlist
\item
  TW gimu13(gimu13) 日時:2021/07/15 18:20:23 URL:
  \url{https://twitter.com/gimu13/status/1415602130304389123} 
  \textgreater{} 委任契約書に書いてあるんだよね。何が問題なの?\\
  \textgreater{} 本当に訳分からんのだが\ldots。 \url{https://t.co/coBjB9LWfW} 
\end{itemize}

 引用されているツイートが最初にご紹介した公告の写真付きツイートでした。その懲戒処分の公告には、理由らしき文中に、委任契約書には8%から5%の間で決めるとあったようなのですが、説明や協議を行わず、一方的に8%で報酬を差し引いた差額を振り込んだとあります。

 もう一度読み直すと、「事件終了後、協議で8%から5%まで変更を認めることとされており」などとなっていました。遺産分割交渉事件とありますが、依頼者の得た経済的利益が、仮に1億円だとすれば、弁護士報酬は8百万円から5百万円になりそうです。

 そういえば、今日はもう一つ気になる内容のツイートを見かけていました、どこのタイムラインだったか憶えていないのですが、数時間前のことなので探すのに手間取ることはなさそうです。

〉〉〉 kk\_hironoのリツイート 〉〉〉

\begin{itemize}
\tightlist
\item
  RT
  kk\_hirono(刑事告発・非常上告_金沢地方検察庁御中)|FaXa9n(マッドハンド)
  日時:2021-07-16 14:37/2021/07/16 00:28 URL:
  \url{https://twitter.com/kk\_hirono/status/1415908439084068870} 
  \url{https://twitter.com/FaXa9n/status/1415694657183391746} 
  \textgreater{} 大正? \url{https://t.co/Y0si1sjztV} 
\end{itemize}

〉〉〉 kk\_hironoのリツイート 〉〉〉

\begin{itemize}
\tightlist
\item
  RT
  kk\_hirono(刑事告発・非常上告_金沢地方検察庁御中)|yatsuhashidayo(やつはし)
  日時:2021-07-16 14:37/2021/07/15 23:55 URL:
  \url{https://twitter.com/kk\_hirono/status/1415908457195073543} 
  \url{https://twitter.com/yatsuhashidayo/status/1415686577880064010} 
  \textgreater{}
  デカい事件が来ても定時で帰る先生がいて、「どうやって事件を回しているんだろ」と思う。
  そういう先生が本当に優秀なんだと思う。
  そんでね、その先生の棚にある記録を手に取って、委任契約書の日付を見ると。。
\end{itemize}

 引用したツイートが「大正?」で、そのアカウントのタイムラインでした。初めて目にしたアカウントではなさそうに思ったのですが、リストには未登録でした。「やつはし」というアカウントは、これまでにけっこうな数、見かけてきましたが、ブックマークなどしておらず、かなり久々です。

\begin{itemize}
\tightlist
\item
  宮前区 - Google マップ \url{https://t.co/R2LK7W5P35} 
\end{itemize}

 弁護士の懲戒処分ということで、川崎市の法律事務所のパワハラで退会命令が出たことを思い出したのですが、国道246号線が川崎市内を通過しているのか気になって調べてみました。思ったより東京湾の海から離れた場所ですが、川崎市宮前区とありました。

 この川崎市宮前区は、川崎市内のどの辺りになるのか調べることはなかったのですが、トンネル内の殺人事件の記事で見かけたのが最初の地名であったように思います。

\begin{itemize}
\item
  2021年07月16日14時51分の登録:
  「宮前区」を@hirono\_hideki @kk\_hirono @s\_hironoで検索 8件の該当 2021-07-16\_14:51の記録
  \url{https://kk2020-09.blogspot.com/2021/07/hironohidekikkhironoshirono82021-07.html} 
\item
  2021年07月16日14時52分の登録:
  「川崎大師」を@hirono\_hideki @kk\_hirono @s\_hironoで検索 15件の該当 2021-07-16\_14:52の記録
  \url{https://kk2020-09.blogspot.com/2021/07/hironohidekikkhironoshirono152021-07.html} 
\item
  2021年07月16日14時52分の登録:
  「(国道246|246)」を@hirono\_hideki @kk\_hirono @s\_hironoで検索 7件の該当 2021-07-16\_14:52の記録
  \url{https://kk2020-09.blogspot.com/2021/07/246hironohidekikkhironoshirono72021-07.html} 
\item
  奉納\さらば弁護士鉄道・泥棒神社の物語(@hirono\_hideki)/2016年01月16日
  - Twilog \url{https://t.co/2VCwdK1J8A} 
\end{itemize}

2016-01-16 23:26:44
``小室さんの母・久美子さん(52)は十六日、川崎市宮前区の自宅前で「何事も前向きで一生懸命頑張る娘でした。友人に囲まれ、やりたいことをやってきた。やり残したことはないと信じています」と話した。
\url{http://ow.ly/X9d5X''} 
\url{https://twitter.com/hirono\_hideki/status/688366781736554497} 

 例の婚約騒動の小室さんかと思ったのですが、2016年から5年越しの話題とは考えにくく、Twilogで調べたところ、軽井沢のスキーバス事故でした。他のツイートには「同級生や親戚、地域の人など、およそ1200人に参列」とありますが、事故を起こした運転手との違いの大きさが際立っていました。

 2件目があるはずと思っていた事件の内容のツイートになります。

2018-03-02 17:22:11 ``川崎通り魔、殺人罪で起訴 37歳男を横浜地検 -
西日本新聞 \url{https://www.nishinippon.co.jp/nnp/national/article/398320/} 
川崎市宮前区のトンネルで2006年9月、帰宅途中の黒沼由理さん=当時(27)=が刺殺された通り魔事件で、横浜地検は2日、殺人罪で無職鈴木洋一容疑者(37)を起訴した。''
\url{https://twitter.com/hirono\_hideki/status/969488021702627328} 

 この川崎通り魔事件は、岡山で起きた死刑執行事件と共通性を感じ色々と調べていましたが、岡山の事件では岡山市の弁護士として浜崎一検事の名前を見たように憶えています。今考えると同姓同名の別人という可能性もありそうですが、そうである可能性の方が高いと思います。

 金沢刑務所の事務棟のような建物の2階の部屋で、若い事務官を1人同行させていた浜崎一検事で、告発か告訴の調書を作成したのですが、終わりの方で自分の手が震えだし、止まらなくなりました。後にも先にもない身体症状のような経験でしたが、若い検察事務官を身を乗り出す勢いで驚いていました。

 そういえば今朝、Twitterのトレンドに連続テレビ小説まれ、で横浜での恋愛関係の若者を演じていた俳優の名前があり、リンクを開いてタイムラインのツイートを見ていくと、一年ほど前に自殺で話題になった若手俳優の名前と顔写真があり、いくらか、あの時の検察事務官に似ているように思えました。

〉〉〉 kk\_hironoのリツイート 〉〉〉

\begin{itemize}
\tightlist
\item
  RT
  kk\_hirono(刑事告発・非常上告_金沢地方検察庁御中)|eigacom(映画.com)
  日時:2021-07-16 15:32/2021/07/14 12:00 URL:
  \url{https://twitter.com/kk\_hirono/status/1415922147545948161} 
  \url{https://twitter.com/eigacom/status/1415144192980918275} 
  \textgreater{}
  黒崎博監督の情熱がアメリカのスタッフを刺激 「映画 太陽の子」舞台裏映像
  \#映画太陽の子 \url{https://t.co/YfBWKIR9pn} 
\end{itemize}

 俳優の名前がどちらも正確に思い出せないのですが、一緒に見かけていた「太陽の子」で検索しました。日米合作映画とありますが、朝には気が付かなかったもので、戦争当時の場面は、別の話題と関連したものと思っていました。

 どうも「沈まぬ太陽」と「大地の子」の記憶が混同していたようです。「沈まぬ太陽」は間接的な話題を見た記憶しかないのですが、御巣鷹山の墜落事故を題材にしているようでした。同じ作者で、山崎豊子だったと思いますが、盗作疑惑や裁判沙汰でも晩年は話題を見かけていました。

 山崎豊子の本は、金沢刑務所の拘置所で何か1冊を読んだように思うのですが、それで同じ頃、大きな話題となっていた「大地の子」を読みたいとずっと思っていたのですが、まだ手にしたこともない状態です。

\begin{itemize}
\tightlist
\item
  黒崎博監督の情熱がアメリカのスタッフを刺激 「映画 太陽の子」舞台裏映像
  : 映画ニュース - 映画.com \url{https://t.co/3bw1belo8w} 
\end{itemize}

 戦争映画の話題自体がとても珍しく感じたのですが、初公開の新作映画のはずなのに1年ほど前に自殺した俳優の名前があるようです。名前を取り違えているのか、少し不安になってきましたが、上記の記事には自殺のことはなかったと思います。名前に春とある男性の名前がとても珍しいと印象にありました。

\begin{itemize}
\tightlist
\item
  三浦春馬 - Wikipedia \url{https://t.co/hlKpGfi7Ut}  三浦 春馬(みうら
  はるま、1990年(平成2年)4月5日{[}2{]} -
  2020年(令和2年)7月18日{[}3{]})は、日本の俳優、歌手。茨城県土浦市出身{[}4{]}。最終所属はアミューズ{[}5{]}。
\end{itemize}

 自殺の報道があった日のことは印象的に憶えているのですが、テレビでみたあと、昼過ぎにAコープ能都店に買い物にいって、レジでおばさんらが近所の出来事のように井戸端会議の様相で話していたのですが、私はその俳優の名前も知らなかったので、最初は事態が飲み込めずにいました。

\begin{itemize}
\item
  \begin{enumerate}
  \def\labelenumi{(\arabic{enumi})}
  \setcounter{enumi}{9}
  \tightlist
  \item
    『映画 太陽の子』予告編 2021年8月6日(金)公開 - YouTube
    \url{https://t.co/whKVkL9ptB} 
  \end{enumerate}
\end{itemize}

 数年前、若手俳優の主演で戦艦大和を設計した人物の映画が話題となっていました。アルキメデスのなんとかという題名だったように思います。金沢刑務所の拘置所で戦艦大和の設計者の分厚い本を官本で借りたことがあったのですが、1,2ページ読んで、読むのをやめました。

 書き出しの資料の細かさに圧倒されたような記憶が残っています。

\begin{quote}
《引用の始まり》
\end{quote}

\begin{quote}
コロナ禍ということがあり、やむを得ないのだろうが、ファンたちにはもやもやしたものが残るに違いない。というのも、まだ、彼のお墓はもちろん、いまだに納骨もできていないというのである。週刊文春によれば、遺産を巡って実母と事務所側が一時対立し、さらに、今年(2021年)の1月に、春馬が幼い頃に離婚した実父が亡くなったため、遺産分割協議が終わっていないこともあり、実母によれば、「もう少し心の整理をつけるまでお時間を頂ければ\ldots\ldots」ということのようだ。
\end{quote}

\begin{quote}
《引用の終わり》
\end{quote}

\begin{itemize}
\tightlist
\item
  三浦春馬
  間もなく一周忌と映画「太陽の子」公開なのに\ldots『納骨』できない母親のあるわだかまり――ほか3編(2021年7月15日)|BIGLOBEニュース
  \url{https://news.biglobe.ne.jp/entertainment/0715/jc\_210715\_2183268936.html} 
\end{itemize}

 気になって調べていたことが記述されていたのですが、遺産分割協議が終わらず、お墓はもちろん納骨もできていない、というのは知りませんでした。あとで確認すると記事のタイトルに「ほか3編」とあったのですが、小見出しで全く違ったような話に切り替わり、不思議に思っていました。

 「この映画がクランクアップした約2か月後の昨年の正月、親しく付き合っていた会社社長と靖国神社に初詣に行ったという。いい出したのは春馬からで、資料館(遊就館)に行って、特攻隊たちの家族に宛てた手紙を見たかったそうだが、心が重くなり途中で出てきたそうだ。」とあります。

\begin{itemize}
\tightlist
\item
  映画撮影のクランクアップから劇場公開までの期間について。 -
  制作\ldots{} - Yahoo!知恵袋 \url{https://t.co/uIeMHetwRp} 
\end{itemize}

 撮影が終わった後の編集に時間がかかるというのは、これまで考えになかったことですが、今回の告発状の作成にも似たところがあって、参考になりました。

 そろそろ項目の区切りをつけたいところですが、見出しに対応させる必要もあり、6月22日のことについて絞り込んで行きたいと思います。昨日になりますか、写真資料の記事を作成しました。同じ写真は、WordPressのブログにもギャラリーにしてありますが、ファイル名との対応は分かりづらいはずです。

\begin{itemize}
\tightlist
\item
  奉納\危険生物・弁護士脳汚染除去装置\金沢地方検察庁御中\_2020:
  2021年07月15日の記録:写真資料:2021年07月15日作成の写真資料
  \url{https://t.co/nKKvFzdM0W} 
\end{itemize}

 6月18日から始まっていますが、宇出津新港のホームセンタームサシで掃除機を買ってきました。その時、半年は経っていないように思うのですが宇出津新港のアルプの2階にあったゲームセンターが完全撤去され、違ったゲームセンターが出来ているのに気がついたのですが、全く話は聞いていませんでした。

\begin{itemize}
\tightlist
\item
  奉納\危険生物・弁護士脳汚染除去装置\金沢地方検察庁御中\_2020:
  2021年07月15日の記録:写真資料:2021年07月15日作成の写真資料
  2021-06-22\_163707_一時的にアカウントがロックと表示されたモトケンこと矢部善朗弁護士(京都弁護士会)のタイムライン.jpg
  \url{https://kk2020-09.blogspot.com/2021/07/2021071520210715.html\#7} 
\end{itemize}

 その前にも何かあったように思うのですが、モトケンこと矢部善朗弁護士(京都弁護士会)のTwitterを開くと、ロックされているという表示が出て、スクリーンショットで見ると16時37分から3分後の16時40分には、通常通りに表示されたスクリーンショットが記録されています。

 これは本当に一度限りの一時的な現象だったのですが、その数日前になるのか、小倉秀夫弁護士の新しいアカウントが一時的に凍結され、本人も理由がわからないというツイートがあったように、普通に戻ったのですが、それまで普通に表示されていた前の小倉秀夫弁護士のアカウントに変化がありました。

 たぶん現在も同じだと思いますが、これは状態が継続しているはずです。

\begin{itemize}
\item
  \begin{enumerate}
  \def\labelenumi{(\arabic{enumi})}
  \setcounter{enumi}{3}
  \tightlist
  \item
    プロフィール / Twitter \url{https://t.co/5WfsMkIElU}  ¥\n
    アカウントは凍結されています ¥\n
    Twitterでは、Twitterルールに違反しているアカウントを凍結しています
  \end{enumerate}
\end{itemize}

 ちょっと驚いたのですが、今ブックマークを開くと、小倉秀夫弁護士の前のアカウントは完全な凍結となっていました。

〉〉〉 kk\_hironoのリツイート 〉〉〉

\begin{itemize}
\tightlist
\item
  RT
  kk\_hirono(刑事告発・非常上告_金沢地方検察庁御中)|chosakukenho(小倉秀夫)
  日時:2021-07-16 16:50/2021/07/16 15:05 URL:
  \url{https://twitter.com/kk\_hirono/status/1415941944102903811} 
  \url{https://twitter.com/chosakukenho/status/1415915426089881605} 
  \textgreater{}
  こんな発言をしたら一発退場になる国ってどこのことを言うのだろう。
  \url{https://t.co/FP8fxY9Vir} 
\end{itemize}

 1時間前の表示になっていますが、小倉秀夫弁護士の新しいアカウントの最新ツイートをリツイートしました。何事もなく表示されています。

 スクリーンショットを記録しているはずですが、すぐに見つからない状態です。

 さきほど「twilog-serch
このアカウントは\textbar tac」という検索で出てこなかったのですが、スクリーンショットのファイルが1つ見つかりました。Twilogの方で探してみます。

〉〉〉 kk\_hironoのリツイート 〉〉〉

\begin{itemize}
\tightlist
\item
  RT
  kk\_hirono(刑事告発・非常上告_金沢地方検察庁御中)|s\_hirono(非常上告-最高検察庁御中\_ツイッター)
  日時:2021-07-16 17:10/2021/07/01 19:24 URL:
  \url{https://twitter.com/kk\_hirono/status/1415946887379619843} 
  \url{https://twitter.com/s\_hirono/status/1410544847073599501} 
  \textgreater{}
  2021-07-01-183730\_小倉秀夫@Hideo\_Ogura注意: このアカウントは一時的に制限されていますこのアカウントは不審な行為が確認されています。表示してもよろ.jpg
  \url{https://t.co/5qXjnuRE9Z} 
\end{itemize}

〉〉〉 kk\_hironoのリツイート 〉〉〉

\begin{itemize}
\tightlist
\item
  RT
  kk\_hirono(刑事告発・非常上告_金沢地方検察庁御中)|s\_hirono(非常上告-最高検察庁御中\_ツイッター)
  日時:2021-07-16 17:10/2021/07/01 08:52 URL:
  \url{https://twitter.com/kk\_hirono/status/1415946932585852928} 
  \url{https://twitter.com/s\_hirono/status/1410385811485315075} 
  \textgreater{}
  2021-07-01-083041\_小倉秀夫@Hideo\_Ogura注意: このアカウントは一時的に制限されていますこのアカウントは不審な行為が確認されています。表示してもよろ.jpg
  \url{https://t.co/rujIguvtG2} 
\end{itemize}

〉〉〉 kk\_hironoのリツイート 〉〉〉

\begin{itemize}
\item
  RT
  kk\_hirono(刑事告発・非常上告_金沢地方検察庁御中)|s\_hirono(非常上告-最高検察庁御中\_ツイッター)
  日時:2021-07-16 17:10/2021/06/18 17:46 URL:
  \url{https://twitter.com/kk\_hirono/status/1415947011212283905} 
  \url{https://twitter.com/s\_hirono/status/1405809069714071553} 
  \textgreater{}
  2021-06-18-174400\_@km0bakeこのアカウントは存在しませんキーワードを変えて検索してみてください。.jpg
  \url{https://t.co/ziY11YhClQ} 
\item
  非常上告-最高検察庁御中\_ツイッター(@s\_hirono)/「このアカウントは」の検索結果
  - Twilog \url{https://t.co/Dmyy1mAE0G} 
\end{itemize}

 一つ間違えてリツイートをしましたが、削除された村松謙弁護士のTwitterアカウントでした。この一時的な制限は、確認ボタンのようなものをクリックすることで、通常と思われるタイムラインの表示に切り替わっていました。

\begin{itemize}
\item
  2021年07月16日17時18分の登録:
  REGEXP:''小倉秀夫''/データベース登録済みツイートの検索:2021-07-01〜2021-07-16/2021年07月16日17時17分の記録:ユーザ・投稿:6/47件
  \url{https://kk2020-09.blogspot.com/2021/07/regexp2021-07-012021-07\_16.html} 
\item
  (06/47) RT @YoWatShiinaEsq(渡邉葉)|bewizyou1(郡司真子Masako
  GUNJI) 日時:2021-07-01 06:11:50 +0900/2021-06-30 21:17:00 +0900
  URL: \url{https://twitter.com/YoWatShiinaEsq/status/1410345358178144256} 
  \url{https://twitter.com/bewizyou1/status/1410210873881235458\textgreater} {}
  小倉秀夫凍結\textgreater{}
  このレベルのミソジニー法クラたくさんいるから、全部凍結お願いします。\textgreater{}
  #女子校プールの水になりたい弁護士に抗議します\textgreater{}
  プールの水弁護士一括凍結お願いします。 \url{https://t.co/CH6ZlgFGPZ} 
\end{itemize}

 見覚えのないアカウントと思ったのですが、ブロックされていました。リツイートされている郡司真子Masako
GUNJIというアカウントの方が印象深く記憶にあります。

\begin{itemize}
\item
  \begin{enumerate}
  \def\labelenumi{(\arabic{enumi})}
  \setcounter{enumi}{3}
  \tightlist
  \item
    郡司真子Masako GUNJIさん (@bewizyou1) / Twitter
    \url{https://t.co/zNUu3WGQ4k}  ¥\n ブロックされています ¥\n
    @bewizyou1さんのフォローやツイートの表示はできません。詳細はこちら
  \end{enumerate}
\end{itemize}

 奉納\さらば弁護士鉄道・泥棒神社の物語(@hirono\_hideki)の方だけかとも思ったのですが、確認すると告発\市場急配センター殺人未遂事件\金沢地方検察庁・石川県警察御中(@kk\_hirono)でブロックされていました。

\begin{itemize}
\item
  2021年07月16日17時30分の登録:
  「@YoWatShiinaEsq」を@hirono\_hideki @kk\_hirono @s\_hironoで検索 17件の該当 2021-07-16\_17:30の記録
  \url{https://kk2020-09.blogspot.com/2021/07/yowatshiinaesqhironohidekikkhironoshiro.html} 
\item
  奉納\危険生物・弁護士脳汚染除去装置\金沢地方検察庁御中:
  %@YoWatShiinaEsq 渡邉葉%ワシントンポスト紙:¥\n伊藤詩織氏レイプ事件の報道についての読者のコメント¥\n「日本社会は素晴らしい面もあるが、極めて病的で虐待
  \url{https://t.co/TXZd6bWjwf} 
\item
  奉納\さらば弁護士鉄道・泥棒神社の物語(@hirono\_hideki)/「米国に住んでいる日本語話者の弁護士がソーシャルメディアのプロフィールで弁護士と書くことがどう軽犯罪法に抵触するでしょ」の検索結果
  - Twilog \url{https://t.co/fuBgkLRvcC} 
\end{itemize}

 初めての発見かもしれないですが、リツイートしたツイートのアカウントにその後ブロックされると、リツイートしたツイートのURLではツイートを開けないようです。

 時刻は7月19日12時36分になります。

\begin{itemize}
\tightlist
\item
  TW kk\_hirono(刑事告発・非常上告_金沢地方検察庁御中) 日時:
  2021-07-16 17:49 URL:
  \url{https://twitter.com/kk\_hirono/status/1415956719327350786\textgreater} {}
  奉納\さらば弁護士鉄道・泥棒神社の物語(@hirono\_hideki)/「米国に住んでいる日本語話者の弁護士がソーシャルメディアのプロフィールで弁護士と書くことがどう軽犯罪法に抵触するでしょ」の検索結果
  - Twilog \url{https://t.co/fuBgkLRvcC} 
\end{itemize}

 確認したところ上記の7月16日17時49分のツイートからの中断となっていました。新規レンタルサーバーのお試し期間が7月15日に終了し、7月17日に七尾市のセブンイレブンのコンビニまで支払いに行ってきたのですが、DNSが更新されず、まだドメインが使えない状態となっています。

 更新に2,3日かかることがあるとも言われるDNSですが、私の経験上、設定が反映されるのにこれほど時間が掛かったことはなく、他にも設定を疑ったのですが、問い合わせでWhois情報に問題はなく正しく登録されているとのことでした。

 「nslookup hirono-hideki.info」というコマンドの実行結果が、「** server
can't find hirono-hideki.info: NXDOMAIN」となっています。

 時刻は7月20日13時53分です。午前中にようやく独自ドメインが使えるようになったのですが、その直後に、キーボードが壊れていることに気が付きました。

 dと入力するとD1と入力されたり、スペースキーでは音源の切り替えになっていました。別のキーボードを探して付け替えたところ、キーボードの故障だということがわかりました。

 新規サーバーの方は、お試し期間が終了する7月15日以前の状態に復旧できたようです。これが普及するまでは他のことに集中して取り組む気持ちに慣れなかったのですが、この間にもいろいろなことがあり、偶然とは思えないような大きな状況変化ともなっています。

 6月22日のことになりますが、この日の午後に山口治夫弁護士の手紙を新たに見つけたことになるはずなのですが、撮影したはずと思っていたその場の写真が1枚も見つからず、手紙を入れた袋もすぐには見つけることが出来ず、探して見つけるまでは幻を見たのかと不安な気持ちになっていました。

 7月15日付になりますが、次の写真資料のブログ記事を作成していました。

\begin{quote}
《引用の始まり》
\end{quote}

\begin{quote}
2021-06-18\_144411_新しく出来ていた宇出津新港アルプ2階のゲームセンター.jpg2021-06-18\_144419_.jpg2021-06-18\_153123_旧石川県立水産高校(能都高校)のグランド.jpg2021-06-18\_154150_宇出津新港ホームセンターむさし 掃除機.jpg2021-06-18\_164832_勝ってきた掃除機.jpg2021-06-21\_205942_もらった舳倉島のハチメ 塩焼き.jpg2021-06-22\_163707_一時的にアカウントがロックと表示されたモトケンこと矢部善朗弁護士(京都弁護士会)のタイムライン.jpg2021-06-22\_163819_一時的にアカウントがロックと表示されたモトケンこと矢部善朗弁護士(京都弁護士会)のタイムライン.jpg2021-06-22\_164046_一時的にアカウントがロックと表示されたモトケンこと矢部善朗弁護士(京都弁護士会)のタイムライン.jpg2021-06-22\_173005_家の中で発見した父母の結婚写真等.jpg2021-06-22\_173258_家の中で発見した父母の結婚写真等.jpg2021-06-22\_174244_家の中で発見した父母の結婚写真等.jpg2021-06-23\_181608_電話があった後、宇出津新港のアルプで勝ってきた介護用シューズ.jpg2021-06-24\_191533_宇出津八坂神社.jpg2021-06-24\_204813_宇出津漆原 大平川と山の上の満月.jpg2021-06-25\_142525_一度も開かず図書館に返却したHTML・CSSの本3冊.jpg2021-06-25\_150604_コンセールのと あばれ祭りの展示.jpg2021-06-28\_125054_Amazonで購入した和歌山カレー事件「再審申立書」.jpg2021-06-28\_140524_介護用シューズを届けに行く能登町上町の珠洲道路.jpg2021-06-29\_163114_宇出津新港 職業安定所.jpg2021-06-30\_183639_珠洲市ほたるの里.jpg2021-06-30\_202656_蛍.jpg2021-07-13\_135319_.jpg2021-07-13\_143721_摂取予約の確定.jpg2021-07-13\_181311_北國銀行宇出津支店ATM.jpg2021-07-13\_191115_宇出津港.jpg2021-07-13\_191118_.jpg2021-07-13\_191127_.jpg2021-07-13\_191909_いしりぽん酢.jpg2021-07-14\_023852_.png2021-07-14\_103930_旧能登鉄道のトンネル 珠洲方面.jpg2021-07-14\_103947_旧能登鉄道のトンネル 珠洲方面.jpg2021-07-14\_154313_令和3年6月22日に発見した山口治夫弁護士の手紙と拘置所のノートを発見した場所\_ココナッツサブレ.jpg2021-07-14\_154325_令和3年6月22日に発見した山口治夫弁護士の手紙と拘置所のノートを発見した場所\_ココナッツサブレ.jpg2021-07-14\_154448_令和3年6月22日に発見した山口治夫弁護士の手紙と拘置所のノートを発見した場所.jpg2021-07-14\_154452_令和3年6月22日に発見した山口治夫弁護士の手紙と拘置所のノートを発見した場所.jpg2021-07-14\_155653_令和3年6月22日に発見した山口治夫弁護士の手紙と拘置所のノート(デジカメ撮影)\_01.jpg2021-07-14\_155715_令和3年6月22日に発見した山口治夫弁護士の手紙と拘置所のノート(デジカメ撮影)\_02.jpg2021-07-14\_155725_令和3年6月22日に発見した山口治夫弁護士の手紙と拘置所のノート(デジカメ撮影)\_03.jpg2021-07-14\_155746_令和3年6月22日に発見した山口治夫弁護士の手紙と拘置所のノート(デジカメ撮影)\_04.jpg2021-07-14\_155753_令和3年6月22日に発見した山口治夫弁護士の手紙と拘置所のノート(デジカメ撮影)\_05.jpg2021-07-14\_155808_令和3年6月22日に発見した山口治夫弁護士の手紙と拘置所のノート(デジカメ撮影)\_06.jpg2021-07-14\_155828_令和3年6月22日に発見した山口治夫弁護士の手紙と拘置所のノート(デジカメ撮影)\_07.jpg2021-07-14\_155833_令和3年6月22日に発見した山口治夫弁護士の手紙と拘置所のノート(デジカメ撮影)\_08.jpg2021-07-14\_155919_令和3年6月22日に発見した山口治夫弁護士の手紙と拘置所のノート(デジカメ撮影)\_09.jpg2021-07-14\_155939_令和3年6月22日に発見した山口治夫弁護士の手紙と拘置所のノート(デジカメ撮影)\_10.jpg2021-07-14\_160025_令和3年6月22日に発見した山口治夫弁護士の手紙と拘置所のノート(デジカメ撮影)\_11.jpg2021-07-14\_160031_令和3年6月22日に発見した山口治夫弁護士の手紙と拘置所のノート(デジカメ撮影)\_12.jpg2021-07-14\_160102_令和3年6月22日に発見した山口治夫弁護士の手紙と拘置所のノート(デジカメ撮影)\_13.jpg2021-07-14\_160115_令和3年6月22日に発見した山口治夫弁護士の手紙と拘置所のノート(デジカメ撮影)\_14.jpg2021-07-14\_160121_令和3年6月22日に発見した山口治夫弁護士の手紙と拘置所のノート(デジカメ撮影)\_15.jpg2021-07-14\_160141_令和3年6月22日に発見した山口治夫弁護士の手紙と拘置所のノート(デジカメ撮影)\_16.jpg2021-07-14\_160151_令和3年6月22日に発見した山口治夫弁護士の手紙と拘置所のノート(デジカメ撮影)\_17.jpg2021-07-14\_160219_令和3年6月22日に発見した山口治夫弁護士の手紙と拘置所のノート(デジカメ撮影)\_18.jpg2021-07-14\_160257_令和3年6月22日に発見した山口治夫弁護士の手紙と拘置所のノート(デジカメ撮影)\_19.jpg2021-07-14\_160306_令和3年6月22日に発見した山口治夫弁護士の手紙と拘置所のノート(デジカメ撮影)\_20.jpg2021-07-14\_160353_令和3年6月22日に発見した山口治夫弁護士の手紙と拘置所のノート(デジカメ撮影)\_21.jpg2021-07-14\_160401_令和3年6月22日に発見した山口治夫弁護士の手紙と拘置所のノート(デジカメ撮影)\_22.jpg2021-07-14\_160555_令和3年6月22日に発見した山口治夫弁護士の手紙と拘置所のノート(デジカメ撮影)\_23.jpg2021-07-14\_160601_令和3年6月22日に発見した山口治夫弁護士の手紙と拘置所のノート(デジカメ撮影)\_24.jpg2021-07-14\_160625_令和3年6月22日に発見した山口治夫弁護士の手紙と拘置所のノート(デジカメ撮影)\_25.jpg2021-07-14\_160636_令和3年6月22日に発見した山口治夫弁護士の手紙と拘置所のノート(デジカメ撮影)\_26.jpg2021-07-14\_160700_令和3年6月22日に発見した山口治夫弁護士の手紙と拘置所のノート(デジカメ撮影)\_27.jpg2021-07-14\_160705_令和3年6月22日に発見した山口治夫弁護士の手紙と拘置所のノート(デジカメ撮影)\_28.jpg2021-07-14\_160759_令和3年6月22日に発見した山口治夫弁護士の手紙と拘置所のノート(デジカメ撮影)\_29.jpg2021-07-14\_160807_令和3年6月22日に発見した山口治夫弁護士の手紙と拘置所のノート(デジカメ撮影)\_30.jpg2021-07-14\_160812_令和3年6月22日に発見した山口治夫弁護士の手紙と拘置所のノート(デジカメ撮影)\_31.jpg2021-07-14\_160818_令和3年6月22日に発見した山口治夫弁護士の手紙と拘置所のノート(デジカメ撮影)\_32.jpg2021-07-14\_161137_令和3年6月22日に発見した山口治夫弁護士の手紙と拘置所のノート(デジカメ撮影)\_33.jpg2021-07-14\_161149_令和3年6月22日に発見した山口治夫弁護士の手紙と拘置所のノート(デジカメ撮影)\_34.jpg2021-07-14\_161156_令和3年6月22日に発見した山口治夫弁護士の手紙と拘置所のノート(デジカメ撮影)\_35.jpg2021-07-14\_161201_令和3年6月22日に発見した山口治夫弁護士の手紙と拘置所のノート(デジカメ撮影)\_36.jpg2021-07-14\_161207_令和3年6月22日に発見した山口治夫弁護士の手紙と拘置所のノート(デジカメ撮影)\_37.jpg2021-07-14\_161249_令和3年6月22日に発見した山口治夫弁護士の手紙と拘置所のノート(デジカメ撮影)\_38.jpg2021-07-14\_161947_平成12年3月30日付 上告審における弁護人選任について(照会書).jpg2021-07-14\_162001_平成12年1月17日付 弁護人選任通知.jpg2021-07-14\_162013_平成11年10月14日付 公判期日召喚状 裁判官 小川賢司.jpg2021-07-14\_162028_平成12年9月20日付 最高裁決定\_1.jpg2021-07-14\_162038_平成12年9月20日付 最高裁決定\_2.jpg2021-07-14\_162059_平成12年8月30日付 最高裁決定 上告棄却\_1.jpg2021-07-14\_162109_平成12年8月30日付 最高裁決定 上告棄却\_2.jpg2021-07-14\_162131_平成12年9月1日付 山口治夫弁護士からの手紙\_1.jpg2021-07-14\_162137_平成12年9月1日付 山口治夫弁護士からの手紙\_2.jpg2021-07-14\_162201_平成12年9月1日付 山口治夫弁護士からの手紙\_3.jpg2021-07-14\_162218_平成12年9月1日付 山口治夫弁護士からの手紙\_4.jpg2021-07-14\_162254_平成12年6月6日付 山口治夫弁護士からの手紙.jpg2021-07-14\_162319_平成12年6月6日付 山口治夫弁護士からの手紙.jpg2021-07-14\_181055_辺田の浜、旧ガソリンスタンド前.jpg2021-07-14\_181137_辺田の浜、神目神社.jpg2021-07-14\_182847_.jpg2021-07-14\_182958_宇出津崎山口交差点.jpg2021-07-14\_195230_銭湯のテレビ.jpg
\end{quote}

\begin{quote}
《引用の終わり》
\end{quote}

\begin{itemize}
\tightlist
\item
  奉納\危険生物・弁護士脳汚染除去装置\金沢地方検察庁御中\_2020:
  2021年07月15日の記録:写真資料:2021年07月15日作成の写真資料
  \url{https://kk2020-09.blogspot.com/2021/07/2021071520210715.html} 
\end{itemize}

 16時ぐらいかに母親のいる病院から電話があり、夕方に出掛けて介護用シューズを買ってきた日に、病院からの電話がある前にちょっとしたことがあったという記憶はあったので、その日に、山口治夫弁護士からの手紙などを発見したものとも考えていました。

 結局、写真が残っていないので確認のしようがないのですが、母親と父親の結婚式の写真を見つけた場所は、山口治夫弁護士からの手紙などを発見した場所のすぐ近くで、前後していたように思うのですが、それが結婚式の写真が残る6月22日になります。

 一番はっきりと撮影したと記憶にあったのが、山口治夫弁護士からの手紙などの近くにあったココナッツサブレというお菓子の写真になるのですが、このココナッツサブレは特別な思い出のあるお菓子で、最初に知ったのが昭和56年の金沢少年鑑別所での物品購入でした。

 山口治夫弁護士の2通の手紙は、拘置所の雑記帳などと一緒にレジ袋に入っていたと思うのですが、なぜ同じ袋に入れてその場所に置いていたのか記憶になく、部屋の模様替えなどを考えると、その場所に置いたのが5年以上前という可能性はないと思うのです。

 山口治夫弁護士の手紙は同じくまとめたものを家の中の全く別の場所で見つけているのですが、そちらの方は2010年ごろには写真撮影し、ネットで公開する状態にしていたことがわかっています。

 内容をみると、後に見つけた方が、より重要と思えるもので、上告審の上告棄却決定が届いた直後の手紙のようでした。上告棄却決定の直後に山口治夫弁護士から手紙をもらったこともすっかり記憶になく、いつの間にか忘れていたことになりそうです。

 最高裁の異議申し立てに対する決定書もありますが、おぼろげに憶えているのは、金沢刑務所の拘置所の2階の担当だった刑務官が、異議申し立て書を便箋で書けと指示したことです。おそらくですが、認書作成願いを頼んだところで、便箋で書けと指示を受けたように思います。

 裁判所や検察庁に出す書面は、全罫紙と大型封筒を使っていました。まず、認書作成願いの願箋を出して許可をもらい、郵送するときも大型封筒使用願いという願箋を出すことになっていました。

 このときの担当の刑務官の名前は思い出せないのですが、小柄で筋肉質のような体型で、割と年配に見えましたが、何よりの特徴は、話し言葉が能登弁で、それも宇出津にかなり近い方言でした。宇出津とは少し違っているようにも思え、柳田村の可能性が高そうに思えていました。

 車で10分ほど走ると、かなり方言に違いが出てくるのが当時の奥能登の特徴で、これは地元で生活をしていないと聞き分けの出来ない特徴かもしれません。現在は方言自体が薄れていますが、年代や男女の違いもあって、個人個人でもけっこう違いを感じることがありました。

 金沢刑務所の刑務官は全員が国家公務員だと聞いていましたが、けっこうな数、かなり強い能登の方言で話す刑務官がいました。今考えると、特徴的な金沢弁を話す刑務官というのが記憶にないぐらいです。

 山口治夫弁護士の手紙の内容については、ここでは触れないことにしたいと思います。次に、モトケンこと矢部善朗弁護士(京都弁護士会)について取り上げておきたいと思うのですが、その中で、問題のつながりなどを指摘しておきたいとは考えています。

\begin{itemize}
\tightlist
\item
  〈〈〈 2021/07/20 14:50:09 Linux Emacs: 〈〈〈
\end{itemize}

\hypertarget{ux5c0fux5c71ux7530ux572dux543eux6c0fux306bux53b3ux3057ux3044ux30e2ux30c8ux30b1ux30f3ux3053ux3068ux77e2ux90e8ux5584ux6717ux5f01ux8b77ux58ebux4eacux90fdux5f01ux8b77ux58ebux4f1aux306bux601dux3046ux672cux4ef6ux544aux767aux4e8bux4ef6ux306bux4e0eux3048ux305fux5f71ux97ff}{%
\paragraph{小山田圭吾氏に厳しいモトケンこと矢部善朗弁護士(京都弁護士会)に思う、本件告発事件に与えた影響}\label{ux5c0fux5c71ux7530ux572dux543eux6c0fux306bux53b3ux3057ux3044ux30e2ux30c8ux30b1ux30f3ux3053ux3068ux77e2ux90e8ux5584ux6717ux5f01ux8b77ux58ebux4eacux90fdux5f01ux8b77ux58ebux4f1aux306bux601dux3046ux672cux4ef6ux544aux767aux4e8bux4ef6ux306bux4e0eux3048ux305fux5f71ux97ff}}

安藤健次郎さん/小山田圭吾氏に厳しいモトケンこと矢部善朗弁護士(京都弁護士会)に思う、本件告発事件に与えた影響

告発の事実/
被告発人らの関与と役割及び具体的犯罪事実/安藤健次郎さん/小山田圭吾氏に厳しいモトケンこと矢部善朗弁護士(京都弁護士会)に思う、本件告発事件に与えた影響

\begin{itemize}
\tightlist
\item
  告発の事実/
  被告発人らの関与と役割及び具体的犯罪事実/安藤健次郎さん/小山田圭吾氏に厳しいモトケンこと矢部善朗弁護士(京都弁護士会)に思う、本件告発事件に与えた影響
  \textbar{} 告発・非常上告\_2021\金沢地方検察庁御中
  \url{http://hirono-hideki.info/wp/?p=1299} 
\end{itemize}

:CATEGORIES: @kanazawabengosi \#金沢弁護士会 @JFBAsns
日本弁護士連合会(日弁連) \#法務省 @MOJ\_HOUMU \#安藤健次郎さん
\#モトケンこと矢部善朗弁護士(京都弁護士会)

\begin{itemize}
\tightlist
\item
  〉〉〉 Linux Emacs: 2021/07/20 15:19:00 〉〉〉
\end{itemize}

 これからモトケンこと矢部善朗弁護士(京都弁護士会)のタイムラインを遡って、リツイートを行いますがモトケンこと矢部善朗弁護士(京都弁護士会)のツイートは全てブロックされていると思います。関連のあるツイートもリツイートするかもしれません。

 その前に、小山田をキーワードに含む記録の確認です。意外なことに7月16日から記録がありました。モトケンこと矢部善朗弁護士(京都弁護士会)のタイムラインにおける開始地点を見定めるのが目的でした。

\begin{lstlisting}
base ❯ d|grep 小山田
\end{lstlisting}

\begin{itemize}
\tightlist
\item
  2021年07月16日07時17分の登録:
  REGEXP:''小山田圭吾''/データベース登録済みツイートの検索:2021-07-16〜2021-07-16/2021年07月16日07時17分の記録:ユーザ・投稿:8/13件
  \url{https://kk2020-09.blogspot.com/2021/07/regexp2021-07-162021-07.html} 
\item
  2021年07月17日06時26分の登録:
  \??? @un\_co\_the2nd\小山田圭吾はなぜイジメ記事問題を無視し続けるのか
  \url{https://kk2020-09.blogspot.com/2021/07/uncothe2nd\_17.html} 
\item
  2021年07月18日01時40分の登録:
  \高橋雄一郎 @kamatatylaw\小山田圭吾氏がでてきて謝罪会見して収益の一部を障害者団体に寄付する宣言をすれば,所詮パフォーマンスだけど,批判は和らぎ,現場スタッ
  \url{https://kk2020-09.blogspot.com/2021/07/kamatatylaw\_18.html} 
\item
  2021年07月18日19時12分の登録:
  \ystk @lawkus\Twitter論壇が馬鹿ばかりなのは知ってたけど、「五輪は小山田を使うな」を「小山田に音楽その他の社会的活動をさせるな」と混同するほどの馬鹿は
  \url{https://kk2020-09.blogspot.com/2021/07/ystklawkustwitter.html} 
\item
  2021年07月19日08時45分の登録:
  REGEXP:''小山田圭吾''/データベース登録済みツイートの検索:2021-07-16〜2021-07-19/2021年07月19日08時44分の記録:ユーザ・投稿:35/57件
  \url{https://kk2020-09.blogspot.com/2021/07/regexp2021-07-162021-07\_19.html} 
\item
  2021年07月19日15時29分の登録:
  \モトケン @motoken\_tw\引導を渡した感じだな。小山田圭吾の問題に加藤官房長官「主催者である組織委員会において適切に対応していただきたい」
  \url{https://kk2020-09.blogspot.com/2021/07/motokentw\_3.html} 
\item
  2021年07月19日20時47分の登録:
  REGEXP:''小山田''/データベース登録済みツイートの検索:2016-01-17〜2021-07-19/2021年07月19日20時45分の記録:ユーザ・投稿:59/99件
  \url{https://kk2020-09.blogspot.com/2021/07/regexp2016-01-172021-07.html} 
\item
  2021年07月19日22時09分の登録:
  \弁護士戸舘圭之オフィシャル/とってぃ/袴田事件弁護団 @todateyoshiyuki\私は小山田圭吾さんの件は彼が少年時代にしていた行いじたいは過去の未熟な子ども時代
  \url{https://kk2020-09.blogspot.com/2021/07/todateyoshiyuki\_19.html} 
\item
  2021年07月19日22時25分の登録:
  \高橋雄一郎 @kamatatylaw\小山田圭吾氏は辞任したのだね。政権が動くとはやいね。
  \url{https://kk2020-09.blogspot.com/2021/07/kamatatylaw\_19.html} 
\end{itemize}

 モトケンこと矢部善朗弁護士(京都弁護士会)に限定して調べると、7月19日からでしたが、7月16日にはネットで話題となっていたようなので、7月15日辺りまでタイムラインを遡って、開始地点を決めたいと思います。

\begin{itemize}
\tightlist
\item
  2021年07月20日15時26分の登録:
  REGEXP:''小山田''/モトケン(@motoken\_tw)の検索(2021-07-19〜2021-07-20/2021年07月20日15時26分の記録6件)
  \url{https://kk2020-09.blogspot.com/2021/07/regexpmotokentw2021-07-192021-07.html} 
\end{itemize}

 それでは、モトケンこと矢部善朗弁護士(京都弁護士会)のタイムラインに移って作業を始めます。

〉〉〉 kk\_hironoのリツイート 〉〉〉

\begin{itemize}
\tightlist
\item
  RT
  kk\_hirono(刑事告発・非常上告_金沢地方検察庁御中)|takuramix(タクラミックス)
  日時:2021-07-20 15:37/2021/07/18 04:21 URL:
  \url{https://twitter.com/kk\_hirono/status/1417373011007602689} 
  \url{https://twitter.com/takuramix/status/1416478205326729218} 
  \textgreater{}
  小山田圭吾氏の辞任を求めようとは思わない。私がそんなものを求める必要性も無い。
  ただ、五輪の筋として、あの起用は無いだろ?というだけの事で、日程から考えて、多分辞任も解任も無い。
  だが、それで分かるのは「五輪の建前の軽さ」だって事。
  当事者たちがそれで良いというのなら、そういう事。
\end{itemize}

〉〉〉 kk\_hironoのリツイート 〉〉〉

\begin{itemize}
\tightlist
\item
  RT
  kk\_hirono(刑事告発・非常上告_金沢地方検察庁御中)|BigHopeClasic(BigHopeClasic)
  日時:2021-07-20 15:42/2021/07/19 15:18 URL:
  \url{https://twitter.com/kk\_hirono/status/1417374233802338304} 
  \url{https://twitter.com/BigHopeClasic/status/1417005784362217472} 
  \textgreater{} @motoken\_tw なお組織委員会の回答は「知るかボケ」でした
  \url{https://t.co/m4XB90eiKI} 
\end{itemize}

〉〉〉 kk\_hironoのリツイート 〉〉〉

\begin{itemize}
\tightlist
\item
  RT
  kk\_hirono(刑事告発・非常上告_金沢地方検察庁御中)|o2441(スラ弁(弁護士大西洋一))
  日時:2021-07-20 15:42/2021/07/19 15:05 URL:
  \url{https://twitter.com/kk\_hirono/status/1417374274344558595} 
  \url{https://twitter.com/o2441/status/1417002561106706432} 
  \textgreater{}
  もう選び直す時間がないからやむを得ない、ご理解いただきたいとでも言えばいいのに、正面から肯定しちゃうと燃料投下になっちゃうよね。
  小山田圭吾氏の留任 組織委があらためて強調「貢献は大きなもの」(日刊スポーツ)
  \#Yahooニュース \url{https://t.co/eH2DNdgl1X} 
\end{itemize}

〉〉〉 kk\_hironoのリツイート 〉〉〉

\begin{itemize}
\tightlist
\item
  RT
  kk\_hirono(刑事告発・非常上告_金沢地方検察庁御中)|HirokoKonishi(小西寛子)
  日時:2021-07-20 15:42/2021/07/19 17:24 URL:
  \url{https://twitter.com/kk\_hirono/status/1417374357475655680} 
  \url{https://twitter.com/HirokoKonishi/status/1417037679816966145} 
  \textgreater{} @motoken\_tw 彼らの発言はわざとなのか本気なのか
\end{itemize}

〉〉〉 kk\_hironoのリツイート 〉〉〉

\begin{itemize}
\tightlist
\item
  RT
  kk\_hirono(刑事告発・非常上告_金沢地方検察庁御中)|spell\_of\_enigma(spell)
  日時:2021-07-20 15:43/2021/07/19 18:32 URL:
  \url{https://twitter.com/kk\_hirono/status/1417374551663517696} 
  \url{https://twitter.com/spell\_of\_enigma/status/1417054646992916484} 
  \textgreater{} @motoken\_tw
  よりによってパラリンピックだからなあ\ldots{}
\end{itemize}

〉〉〉 kk\_hironoのリツイート 〉〉〉

\begin{itemize}
\tightlist
\item
  RT
  kk\_hirono(刑事告発・非常上告_金沢地方検察庁御中)|Xephyr2009(Xephyr)
  日時:2021-07-20 15:43/2021/07/19 18:42 URL:
  \url{https://twitter.com/kk\_hirono/status/1417374646530318342} 
  \url{https://twitter.com/Xephyr2009/status/1417057324196188161} 
  \textgreater{} 今大事なのは目前に差し迫った仕事を完遂する事。
  それ以外にあるか?と思うんだが。 \url{https://t.co/6Jg72zfmiM} 
\end{itemize}

〉〉〉 kk\_hironoのリツイート 〉〉〉

\begin{itemize}
\tightlist
\item
  RT
  kk\_hirono(刑事告発・非常上告_金沢地方検察庁御中)|yagi\_okamoto(じゃい活系YouTuber♪-おれはジャイアンさまだ!-)
  日時:2021-07-20 15:44/2021/07/19 19:13 URL:
  \url{https://twitter.com/kk\_hirono/status/1417374767674318853} 
  \url{https://twitter.com/yagi\_okamoto/status/1417065012007358468} 
  \textgreater{} @motoken\_tw
  犯罪に対して、警察が動かずに、その代わり、皆で非難するのがいいんですか
  動かないから、しょうがないので、非難しましょうなら分かりますけど
  「関係ない」は、凄いな
\end{itemize}

〉〉〉 kk\_hironoのリツイート 〉〉〉

\begin{itemize}
\tightlist
\item
  RT
  kk\_hirono(刑事告発・非常上告_金沢地方検察庁御中)|tikkutakku(tikutaku)
  日時:2021-07-20 15:45/2021/07/19 23:00 URL:
  \url{https://twitter.com/kk\_hirono/status/1417375022415421442} 
  \url{https://twitter.com/tikkutakku/status/1417122069486063624} 
  \textgreater{} @motoken\_tw
  いいか悪いかで言えば悪いに決まってるんだけど、30年も前の話でここまで人格否定されるってのもなかなかに酷い話だ。だったら25年前に記事が出たときに徹底的に潰しておけよ。今更感有りすぎる。再起不能じゃん(´・ω・`)
\end{itemize}

〉〉〉 kk\_hironoのリツイート 〉〉〉

\begin{itemize}
\tightlist
\item
  RT
  kk\_hirono(刑事告発・非常上告_金沢地方検察庁御中)|gondotomohiko(ゴンドウトモヒコ)
  日時:2021-07-20 15:46/2021/07/18 11:38 URL:
  \url{https://twitter.com/kk\_hirono/status/1417375452918779905} 
  \url{https://twitter.com/gondotomohiko/status/1416588224445652994} 
  \textgreater{} [先日の削除したツィートに関して]
  余計な一言のために油に火を注ぐようなことをして大変申し訳ありませんでした。
  いじめ、暴力に関しては断固反対です。
  謝罪文を支持し過去を払拭できる活動を一メンバーとして非力ながらも協力していきたいと思います。
\end{itemize}

〉〉〉 kk\_hironoのリツイート 〉〉〉

\begin{itemize}
\tightlist
\item
  RT
  kk\_hirono(刑事告発・非常上告_金沢地方検察庁御中)|tikkutakku(tikutaku)
  日時:2021-07-20 15:48/2021/07/20 03:26 URL:
  \url{https://twitter.com/kk\_hirono/status/1417375908369866754} 
  \url{https://twitter.com/tikkutakku/status/1417189229700927504} 
  \textgreater{} @motoken\_tw
  25年もほっといて今になって「償ってない」はちょっと酷いと思いますよ、もちろん彼に非があることは分かりますがやり過ぎというかそこまで追い詰めるほどみんな聖人君子なのかと思います(´・ω・`)
\end{itemize}

〉〉〉 kk\_hironoのリツイート 〉〉〉

\begin{itemize}
\tightlist
\item
  RT
  kk\_hirono(刑事告発・非常上告_金沢地方検察庁御中)|tikkutakku(tikutaku)
  日時:2021-07-20 15:49/2021/07/20 10:44 URL:
  \url{https://twitter.com/kk\_hirono/status/1417376022144577538} 
  \url{https://twitter.com/tikkutakku/status/1417299389702311954} 
  \textgreater{} @motoken\_tw
  償っていないと他人の生活の糧を奪おうとする人がそれを言うのですか?
\end{itemize}

〉〉〉 kk\_hironoのリツイート 〉〉〉

\begin{itemize}
\tightlist
\item
  RT
  kk\_hirono(刑事告発・非常上告_金沢地方検察庁御中)|tikkutakku(tikutaku)
  日時:2021-07-20 15:49/2021/07/20 10:52 URL:
  \url{https://twitter.com/kk\_hirono/status/1417376096991924228} 
  \url{https://twitter.com/tikkutakku/status/1417301295363362816} 
  \textgreater{} @motoken\_tw
  最初から「ここまで攻め立てることはないのでは」と言っております。
\end{itemize}

〉〉〉 kk\_hironoのリツイート 〉〉〉

\begin{itemize}
\item
  RT
  kk\_hirono(刑事告発・非常上告_金沢地方検察庁御中)|white\_boards(渋犬)
  日時:2021-07-20 15:50/2021/07/20 12:37 URL:
  \url{https://twitter.com/kk\_hirono/status/1417376320799993857} 
  \url{https://twitter.com/white\_boards/status/1417327862302928897} 
  \textgreater{}
  まぁそうなんだけど、この件に関しては、いじめてた頃は未成年だったって点に触れてる人見たことないな
  \url{https://t.co/Ma0hZKGQDg} 
\item
  〉〉〉 アカウント(@motoken\_tw)は,@kk\_hironoをブロックしています。リツイートできませんでした。
  〉〉〉 ¥\n ¥\n \url{https://t.co/WWGOvyMu5u} 
\item
  〉〉〉 アカウント(@motoken\_tw)は,@kk\_hironoをブロックしています。リツイートできませんでした。
  〉〉〉 ¥\n ¥\n \url{https://t.co/bcFQT4qWyp} 
\item
  〉〉〉 アカウント(@motoken\_tw)は,@kk\_hironoをブロックしています。リツイートできませんでした。
  〉〉〉 ¥\n ¥\n \url{https://t.co/jq8J2JPYri} 
\item
  〉〉〉 アカウント(@motoken\_tw)は,@kk\_hironoをブロックしています。リツイートできませんでした。
  〉〉〉 ¥\n ¥\n \url{https://t.co/gkJOopPx2n} 
\item
  〉〉〉 アカウント(@motoken\_tw)は,@kk\_hironoをブロックしています。リツイートできませんでした。
  〉〉〉 ¥\n ¥\n \url{https://t.co/TiwOq8tSQV} 
\item
  〉〉〉 アカウント(@motoken\_tw)は,@kk\_hironoをブロックしています。リツイートできませんでした。
  〉〉〉 ¥\n ¥\n \url{https://t.co/0IolcikLZm} 
\item
  〉〉〉 アカウント(@motoken\_tw)は,@kk\_hironoをブロックしています。リツイートできませんでした。
  〉〉〉 ¥\n ¥\n \url{https://t.co/uXfrQwUje4} 
\item
  〉〉〉 アカウント(@motoken\_tw)は,@kk\_hironoをブロックしています。リツイートできませんでした。
  〉〉〉 ¥\n ¥\n \url{https://t.co/SMgVCL3O8q} 
\item
  〉〉〉 アカウント(@motoken\_tw)は,@kk\_hironoをブロックしています。リツイートできませんでした。
  〉〉〉 ¥\n ¥\n \url{https://t.co/eeJ2MwVxAS} 
\item
  〉〉〉 アカウント(@motoken\_tw)は,@kk\_hironoをブロックしています。リツイートできませんでした。
  〉〉〉 ¥\n ¥\n \url{https://t.co/CaS4qNf2bE} 
\item
  〉〉〉 アカウント(@motoken\_tw)は,@kk\_hironoをブロックしています。リツイートできませんでした。
  〉〉〉 ¥\n ¥\n \url{https://t.co/jwwnm8o8lj} 
\item
  〉〉〉 アカウント(@motoken\_tw)は,@kk\_hironoをブロックしています。リツイートできませんでした。
  〉〉〉 ¥\n ¥\n \url{https://t.co/APPzqX0lGO} 
\item
  〉〉〉 アカウント(@motoken\_tw)は,@kk\_hironoをブロックしています。リツイートできませんでした。
  〉〉〉 ¥\n ¥\n \url{https://t.co/yRyhVX5CK9} 
\item
  〉〉〉 アカウント(@motoken\_tw)は,@kk\_hironoをブロックしています。リツイートできませんでした。
  〉〉〉 ¥\n ¥\n \url{https://t.co/NbJ2OV1q0K} 
\item
  〉〉〉 アカウント(@motoken\_tw)は,@kk\_hironoをブロックしています。リツイートできませんでした。
  〉〉〉 ¥\n ¥\n \url{https://t.co/ioOrJjzihd} 
\item
  〉〉〉 アカウント(@motoken\_tw)は,@kk\_hironoをブロックしています。リツイートできませんでした。
  〉〉〉 ¥\n ¥\n \url{https://t.co/xnqWyqZx03} 
\item
  〉〉〉 アカウント(@motoken\_tw)は,@kk\_hironoをブロックしています。リツイートできませんでした。
  〉〉〉 ¥\n ¥\n \url{https://t.co/c6fPyERTC5} 
\item
  〉〉〉 アカウント(@motoken\_tw)は,@kk\_hironoをブロックしています。リツイートできませんでした。
  〉〉〉 ¥\n ¥\n \url{https://t.co/TWT1WGi5TF} 
\item
  〉〉〉 アカウント(@motoken\_tw)は,@kk\_hironoをブロックしています。リツイートできませんでした。
  〉〉〉 ¥\n ¥\n \url{https://t.co/wgtTqRVuw3} 
\item
  〉〉〉 アカウント(@motoken\_tw)は,@kk\_hironoをブロックしています。リツイートできませんでした。
  〉〉〉 ¥\n ¥\n \url{https://t.co/MUWL8uUDKW} 
\item
  〉〉〉 アカウント(@motoken\_tw)は,@kk\_hironoをブロックしています。リツイートできませんでした。
  〉〉〉 ¥\n ¥\n \url{https://t.co/9Da5oFWVMN} 
\item
  〉〉〉 アカウント(@motoken\_tw)は,@kk\_hironoをブロックしています。リツイートできませんでした。
  〉〉〉 ¥\n ¥\n \url{https://t.co/P3mKQQUqR7} 
\item
  〉〉〉 アカウント(@motoken\_tw)は,@kk\_hironoをブロックしています。リツイートできませんでした。
  〉〉〉 ¥\n ¥\n \url{https://t.co/bvbCD2Y1Iw} 
\item
  〉〉〉 アカウント(@motoken\_tw)は,@kk\_hironoをブロックしています。リツイートできませんでした。
  〉〉〉 ¥\n ¥\n \url{https://t.co/6Mm3XmyVut} 
\item
  〉〉〉 アカウント(@motoken\_tw)は,@kk\_hironoをブロックしています。リツイートできませんでした。
  〉〉〉 ¥\n ¥\n \url{https://t.co/AO5yDq9Djs} 
\item
  〉〉〉 アカウント(@motoken\_tw)は,@kk\_hironoをブロックしています。リツイートできませんでした。
  〉〉〉 ¥\n ¥\n \url{https://t.co/sFMLblaaN9} 
\item
  〉〉〉 アカウント(@motoken\_tw)は,@kk\_hironoをブロックしています。リツイートできませんでした。
  〉〉〉 ¥\n ¥\n \url{https://t.co/qf1brFsJca} 
\item
  〉〉〉 アカウント(@motoken\_tw)は,@kk\_hironoをブロックしています。リツイートできませんでした。
  〉〉〉 ¥\n ¥\n \url{https://t.co/ChdRaPmzr7} 
\item
  〉〉〉 アカウント(@motoken\_tw)は,@kk\_hironoをブロックしています。リツイートできませんでした。
  〉〉〉 ¥\n ¥\n \url{https://t.co/qpd1iitj9G} 
\end{itemize}

 以上42件のリツイートを試みたことになりますが、次がブロックされていてリツイートに失敗したツイートの内容になります。すべてモトケンこと矢部善朗弁護士(京都弁護士会)のツイートのようです。

※ @kk\_hironoのアカウントがブロックされ,リツイートに失敗したツイート

\begin{itemize}
\tightlist
\item
  TW motoken\_tw(モトケン) 日時:2021/07/15 08:21:40 URL:
  \url{https://twitter.com/motoken\_tw/status/1415451458875445250} 
  \textgreater{} @nowhereman134
  あなたは、人間の欲望と行動に関する理解が決定的に不足している。\\
  \textgreater{} 話にならんレベル。
\end{itemize}

※ @kk\_hironoのアカウントがブロックされ,リツイートに失敗したツイート

\begin{itemize}
\tightlist
\item
  TW motoken\_tw(モトケン) 日時:2021/07/15 15:04:49 URL:
  \url{https://twitter.com/motoken\_tw/status/1415552916375494658} 
  \textgreater{} @runeruneradio @Luksemburg
  >やっぱり世間はそういうものなのでしょうか?\\
  \textgreater{}\\
  \textgreater{} 最近\\
  \textgreater{} 世に偏見の種はつきまじ\\
  \textgreater{} と言いましたが、そんなものと言えばそんなもの。\\
  \textgreater{} でも、偏見の持ち主を減らすことはできると思ってます。
\end{itemize}

※ @kk\_hironoのアカウントがブロックされ,リツイートに失敗したツイート

\begin{itemize}
\tightlist
\item
  TW motoken\_tw(モトケン) 日時:2021/07/15 17:28:13 URL:
  \url{https://twitter.com/motoken\_tw/status/1415589003772915713} 
  \textgreater{}
  Pの時もBのときも、覚醒剤常習者と話をするときに、覚醒剤を使うのはこれが最後にしときや、と言うけど、本音の気持ちとしては、出所後に覚醒剤を使うのが1日でも2日でも遅れたらいいな、程度に思ってたし、今も思ってる。
  \url{https://t.co/DJJ6UqxecY} 
\end{itemize}

※ @kk\_hironoのアカウントがブロックされ,リツイートに失敗したツイート

\begin{itemize}
\tightlist
\item
  TW motoken\_tw(モトケン) 日時:2021/07/16 08:41:50 URL:
  \url{https://twitter.com/motoken\_tw/status/1415818923698626560} 
  \textgreater{}
  目の前の悲惨な事件事故を見て、それを重く処罰する理屈を提唱する人がいるのだが、その理屈を自分や自分の家族に適用されたらどうなるか、ということを考えない人が多い。
  \url{https://t.co/uXgl1OfIqL} 
\end{itemize}

※ @kk\_hironoのアカウントがブロックされ,リツイートに失敗したツイート

\begin{itemize}
\tightlist
\item
  TW motoken\_tw(モトケン) 日時:2021/07/18 07:54:07 URL:
  \url{https://twitter.com/motoken\_tw/status/1416531689786908672} 
  \textgreater{} >「五輪の建前の軽さ」\\
  \textgreater{}\\
  \textgreater{} これに尽きるかな。 \url{https://t.co/CIkRO2sKAu} 
\end{itemize}

※ @kk\_hironoのアカウントがブロックされ,リツイートに失敗したツイート

\begin{itemize}
\tightlist
\item
  TW motoken\_tw(モトケン) 日時:2021/07/18 11:38:32 URL:
  \url{https://twitter.com/motoken\_tw/status/1416588165112991744} 
  \textgreater{}
  公的な行事における「建前」というのはとても大事なことだと思う。\\
  \textgreater{}
  最も根本的な建前と真っ向から衝突する事態は断固として拒否する、という姿勢が不可欠。\\
  \textgreater{}
  それによって式典が潰れようと、拒否すべきものを容認すれば、行事全体が汚される。
  \url{https://t.co/TTam3dj8Si} 
\end{itemize}

※ @kk\_hironoのアカウントがブロックされ,リツイートに失敗したツイート

\begin{itemize}
\tightlist
\item
  TW motoken\_tw(モトケン) 日時:2021/07/18 17:34:56 URL:
  \url{https://twitter.com/motoken\_tw/status/1416677857057599491} 
  \textgreater{}
  出版業界には倫理観というものを期待してはいけないんだろうな。\\
  \textgreater{}
  一般化してはいけないのかも知れないけど、そういう話がいくらでも可視化されるツイッター。
\end{itemize}

※ @kk\_hironoのアカウントがブロックされ,リツイートに失敗したツイート

\begin{itemize}
\tightlist
\item
  TW motoken\_tw(モトケン) 日時:2021/07/19 09:13:46 URL:
  \url{https://twitter.com/motoken\_tw/status/1416914123779018759} 
  \textgreater{} @sakamotomasayuk @popohito
  私は「廃すべきではない」とまでは言わないが、「冤罪による死刑の可能性(現実にはすでにあったかも知れないが)」は死刑を廃止する決定的な理由にはならない、と言った記憶がある。
\end{itemize}

※ @kk\_hironoのアカウントがブロックされ,リツイートに失敗したツイート

\begin{itemize}
\tightlist
\item
  TW motoken\_tw(モトケン) 日時:2021/07/19 10:12:04 URL:
  \url{https://twitter.com/motoken\_tw/status/1416928795781591040} 
  \textgreater{} コロナもそうだけど選手の熱中症も心配。\\
  \textgreater{}
  元々の大元からいろいろ無理や不祥事があった今回の東京五輪。\\
  \textgreater{}
  今回の小山田問題を含めて運営側の危機管理能力には深刻な危機感を覚える。
\end{itemize}

※ @kk\_hironoのアカウントがブロックされ,リツイートに失敗したツイート

\begin{itemize}
\tightlist
\item
  TW motoken\_tw(モトケン) 日時:2021/07/19 15:22:16 URL:
  \url{https://twitter.com/motoken\_tw/status/1417006858607333378} 
  \textgreater{} @BigHopeClasic ほほー。\\
  \textgreater{} 政府のメンツを潰しましたね。
\end{itemize}

※ @kk\_hironoのアカウントがブロックされ,リツイートに失敗したツイート

\begin{itemize}
\tightlist
\item
  TW motoken\_tw(モトケン) 日時:2021/07/19 17:26:12 URL:
  \url{https://twitter.com/motoken\_tw/status/1417038047690907651} 
  \textgreater{} @HirokoKonishi
  言葉の上で繕えばいい、という意味で、わざとで本気。
\end{itemize}

※ @kk\_hironoのアカウントがブロックされ,リツイートに失敗したツイート

\begin{itemize}
\tightlist
\item
  TW motoken\_tw(モトケン) 日時:2021/07/19 17:02:29 URL:
  \url{https://twitter.com/motoken\_tw/status/1417032078588141569} 
  \textgreater{}
  組織委は何が問題なのかわかってない、ということがわかった。\\
  \textgreater{} 貢献の問題ではない。\\
  \textgreater{}\\
  \textgreater{}
  小山田圭吾氏の留任 組織委があらためて強調「貢献は大きなもの」(日刊スポーツ)\\
  \textgreater{} \#Yahooニュース\\
  \textgreater{} \url{https://t.co/59RpwnaEDH} 
\end{itemize}

※ @kk\_hironoのアカウントがブロックされ,リツイートに失敗したツイート

\begin{itemize}
\tightlist
\item
  TW motoken\_tw(モトケン) 日時:2021/07/19 18:33:21 URL:
  \url{https://twitter.com/motoken\_tw/status/1417054946818551808} 
  \textgreater{} @spell\_of\_enigma
  パラリンピックでなかったら、勝手にしろ、と言う。
\end{itemize}

※ @kk\_hironoのアカウントがブロックされ,リツイートに失敗したツイート

\begin{itemize}
\tightlist
\item
  TW motoken\_tw(モトケン) 日時:2021/07/19 18:52:57 URL:
  \url{https://twitter.com/motoken\_tw/status/1417059877101346820} 
  \textgreater{} @Xephyr2009 大いにある。\\
  \textgreater{} パラリンピックの理念を汚さないこと。
\end{itemize}

※ @kk\_hironoのアカウントがブロックされ,リツイートに失敗したツイート

\begin{itemize}
\tightlist
\item
  TW motoken\_tw(モトケン) 日時:2021/07/19 19:18:16 URL:
  \url{https://twitter.com/motoken\_tw/status/1417066250295922689} 
  \textgreater{} @yagi\_okamoto
  刑事責任を当ためには、警察が捜査して、検察が起訴して、裁判所が有罪判決をして、その判決が確定する必要がある。\\
  \textgreater{} いじめを非難するのに、判決が確定する必要があるのかね?
\end{itemize}

※ @kk\_hironoのアカウントがブロックされ,リツイートに失敗したツイート

\begin{itemize}
\tightlist
\item
  TW motoken\_tw(モトケン) 日時:2021/07/19 19:38:51 URL:
  \url{https://twitter.com/motoken\_tw/status/1417071428638101504} 
  \textgreater{} いったいこの会見はなんだったんだ。\\
  \textgreater{} 最も批判されるべきは、小山田氏ではなく組織委だ。\\
  \textgreater{} \url{https://t.co/IeH3NWPbem}  \url{https://t.co/zBj2Ek2V2K} 
\end{itemize}

※ @kk\_hironoのアカウントがブロックされ,リツイートに失敗したツイート

\begin{itemize}
\tightlist
\item
  TW motoken\_tw(モトケン) 日時:2021/07/19 19:52:22 URL:
  \url{https://twitter.com/motoken\_tw/status/1417074832047439877} 
  \textgreater{}
  私には日本も日本人も好きだし、誇れるところがたくさんあると思っているのだが、オリパラの組織委は全く誇れない。\\
  \textgreater{} 国辱もの。
\end{itemize}

※ @kk\_hironoのアカウントがブロックされ,リツイートに失敗したツイート

\begin{itemize}
\tightlist
\item
  TW motoken\_tw(モトケン) 日時:2021/07/19 19:58:41 URL:
  \url{https://twitter.com/motoken\_tw/status/1417076422561079298} 
  \textgreater{} 私を批判するまとも理由がない人が使う定番ですね。\\
  \textgreater{} 「ヤメ検」\\
  \textgreater{} ブロックしますw \url{https://t.co/uaGrFrcUqI} 
\end{itemize}

※ @kk\_hironoのアカウントがブロックされ,リツイートに失敗したツイート

\begin{itemize}
\tightlist
\item
  TW motoken\_tw(モトケン) 日時:2021/07/19 23:37:56 URL:
  \url{https://twitter.com/motoken\_tw/status/1417131598596640772} 
  \textgreater{} @tikkutakku
  パラリンピックでなければここまで問題にならなかった。\\
  \textgreater{}
  彼も、辞任したことでようやくけじめをつけられたのだと思う。\\
  \textgreater{} それを認める人が多ければ再起不能にならない。
\end{itemize}

※ @kk\_hironoのアカウントがブロックされ,リツイートに失敗したツイート

\begin{itemize}
\tightlist
\item
  TW motoken\_tw(モトケン) 日時:2021/07/19 15:13:37 URL:
  \url{https://twitter.com/motoken\_tw/status/1417004680996036613} 
  \textgreater{} 引導を渡した感じだな。\\
  \textgreater{}\\
  \textgreater{}
  小山田圭吾の問題に加藤官房長官「主催者である組織委員会において適切に対応していただきたい」
  \url{https://t.co/ZxiHg58zVh} 
\end{itemize}

※ @kk\_hironoのアカウントがブロックされ,リツイートに失敗したツイート

\begin{itemize}
\tightlist
\item
  TW motoken\_tw(モトケン) 日時:2021/07/19 23:58:17 URL:
  \url{https://twitter.com/motoken\_tw/status/1417136717732089856} 
  \textgreater{}
  小山田問題は、ツイッターなどのSNSがない時代なら、ここまで大きな問題にはならなかったかも知れない。\\
  \textgreater{} しかし、今は瞬時に情報が共有される。\\
  \textgreater{} そして、みんなが意見を述べる。\\
  \textgreater{}
  時代についてこれない組織委は、そういう流れをなめてたんだろうな。
\end{itemize}

※ @kk\_hironoのアカウントがブロックされ,リツイートに失敗したツイート

\begin{itemize}
\tightlist
\item
  TW motoken\_tw(モトケン) 日時:2021/07/20 07:44:47 URL:
  \url{https://twitter.com/motoken\_tw/status/1417254115600572417} 
  \textgreater{} 垢消ししてもこの記録は一生(死んでも)残るんだよね。\\
  \textgreater{} ネット活動を再開すれば常につきまとう。\\
  \textgreater{} 実名アカウントの怖いところ。\\
  \textgreater{}\\
  \textgreater{}
  小山田圭吾のいとこ謝罪 辞任発表に「正義を振りかざす皆さん、良かったですねー!」ツイート(東スポWeb)\\
  \textgreater{} \#Yahooニュース\\
  \textgreater{} \url{https://t.co/k10Z0z86Qq} 
\end{itemize}

※ @kk\_hironoのアカウントがブロックされ,リツイートに失敗したツイート

\begin{itemize}
\tightlist
\item
  TW motoken\_tw(モトケン) 日時:2021/07/20 08:34:55 URL:
  \url{https://twitter.com/motoken\_tw/status/1417266732838768641} 
  \textgreater{} 「油に火を注ぐ」の元ネタはこれだったのか?
  \url{https://t.co/giSKtYhG4l} 
\end{itemize}

※ @kk\_hironoのアカウントがブロックされ,リツイートに失敗したツイート

\begin{itemize}
\tightlist
\item
  TW motoken\_tw(モトケン) 日時:2021/07/19 17:17:17 URL:
  \url{https://twitter.com/motoken\_tw/status/1417035803616714755} 
  \textgreater{} 彼のやったいじめは障害者に対する悪質な犯罪行為だ。\\
  \textgreater{}
  犯罪者でも自分の罪を悔い反省し償いをしてきたのならパラリンピックを支える資格があると思う。\\
  \textgreater{} しかし、彼にそのような事実はあるのか?\\
  \textgreater{} 批判されて形ばかりの謝罪をしただけ。\\
  \textgreater{} 彼のどこに「高い倫理観」があるのか?\\
  \textgreater{} 反吐が出そうな会見だ。 \url{https://t.co/zBj2Ek2V2K} 
\end{itemize}

※ @kk\_hironoのアカウントがブロックされ,リツイートに失敗したツイート

\begin{itemize}
\tightlist
\item
  TW motoken\_tw(モトケン) 日時:2021/07/20 07:16:50 URL:
  \url{https://twitter.com/motoken\_tw/status/1417247083430506498} 
  \textgreater{} @tikkutakku
  償いというのは、他人からしろと言われてするものですか?
\end{itemize}

※ @kk\_hironoのアカウントがブロックされ,リツイートに失敗したツイート

\begin{itemize}
\tightlist
\item
  TW motoken\_tw(モトケン) 日時:2021/07/20 10:47:23 URL:
  \url{https://twitter.com/motoken\_tw/status/1417300069313773570} 
  \textgreater{} @tikkutakku 論点ずらしをする人は嫌いなんだよ。
\end{itemize}

※ @kk\_hironoのアカウントがブロックされ,リツイートに失敗したツイート

\begin{itemize}
\tightlist
\item
  TW motoken\_tw(モトケン) 日時:2021/07/20 10:54:37 URL:
  \url{https://twitter.com/motoken\_tw/status/1417301891495563266} 
  \textgreater{} @tikkutakku その答えは最初に言ってる。
\end{itemize}

※ @kk\_hironoのアカウントがブロックされ,リツイートに失敗したツイート

\begin{itemize}
\tightlist
\item
  TW motoken\_tw(モトケン) 日時:2021/07/20 11:24:56 URL:
  \url{https://twitter.com/motoken\_tw/status/1417309520011751424} 
  \textgreater{} @MiRi\_zzzzz
  私が関与した盗撮事案で被害者が自衛していた事案はなかった。
\end{itemize}

※ @kk\_hironoのアカウントがブロックされ,リツイートに失敗したツイート

\begin{itemize}
\tightlist
\item
  TW motoken\_tw(モトケン) 日時:2021/07/20 12:39:23 URL:
  \url{https://twitter.com/motoken\_tw/status/1417328257624510468} 
  \textgreater{} @white\_boards インタビューを受けたのはいくつだっけ?
\end{itemize}

 まだテレビでは報道をみていないのですが、大きな問題となっているようです。発端は平成6年の雑誌のインタビュー記事にあったようですが、随分前の問題が今頃に取り沙汰されるというのも珍しいケースで、いくつか条件が重なって反応を起こしたのだとは思います。

 私の告発事件も平成4年の傷害・準強姦被告事件に始まりますが、平成5年に起こったのが山形マット死事件で、私は拘置所の中で週刊誌の記事を読んでいたのですが、その少年事件の担当検事だったのがモトケンこと矢部善朗弁護士(京都弁護士会)ということです。

 問題となったインタビュー記事の抜粋のようなものを読んでいますが、その中には山形マット死事件のこともありました。この山形マット死事件は、異例の経過を辿るのですが民事裁判は現在も続いているのではと思われます。

 モトケンこと矢部善朗弁護士(京都弁護士会)はインタビューのような記事で、理解を得られず勾留延長を断念したなどと述べていましたが、勾留延長の必要性を説くのも担当検事の責任ではないかと単純に考えていました。ずいぶん前から1つだけ残っていた記事です。

 今もネットで見つかるのか探してみたいと思います。2年ほど前には見つけていたはずと思います。

\begin{itemize}
\tightlist
\item
  ★阿修羅♪:一瞬アクセスが集中して読込失敗。またはページが見つかりません。404
  Not Found \url{https://t.co/WFNDSTEqAc} {]}(\url{https://t.co/WFNDSTEqAc} 
\end{itemize}

 次のリンクを開こうとしました。

\url{http://www.asyura2.com/0401/nihon11/msg/819.html} {]}(\url{http://www.asyura2.com/0401/nihon11/msg/819.html} 

 URLの指定の間違えに気がついたので、開き直しをしたのですが、見覚えのある内容が表示されました。

\begin{quote}
《引用の始まり》
\end{quote}

\begin{quote}
<山形マット死事件>「捜査は不十分だった」 検事が心情明かす

 山形県新庄市立明倫中学校のマット死事件(93年1月)で、当時、山形地検で主任検察官として捜査にかかわった矢部善朗弁護士(48)=京都弁護士会所属=が取材に応じ「取り調べ時間が10日しかなく、捜査は到底十分とは言えなかった」と初めて心情を明かした。逮捕・補導された7少年が事件に関与したか否かの判断は、少年審判とその抗告審で割れ、捜査の不備や少年審判の事実認定の不安定さが指摘されていた。

 矢部さんによると、7少年(当時14~12歳)のうち、逮捕・送検された14歳の3人を取り調べた。3人とも警察で容疑を認めていたが、10日間の身柄拘束中、1人は否認に転じるなど供述に食い違いが生じた。矢部さんは10日間の延長を望んだが、少年の保護・育成の観点から地検内で同意を得られなかったという。

 矢部さんは「(嫌疑をかけられた少年が)7人おり、否認もいた。成人なら10日間なんてあり得ない。20日間取り調べられなかったことが最後まで尾を引いている」と語った。だが一方で、その後の追加捜査が抗告審で「7人全員関与」の判断を導いたとの考えも示した。【山根真紀】

(毎日新聞)[3月4日6時37分更新]
\end{quote}

\begin{quote}
《引用の終わり》
\end{quote}

\begin{itemize}
\tightlist
\item
  <山形マット死事件>「捜査は不十分だった」 検事が心情明かす(毎日新聞)[3月4日6時37分更新]
  まさちゃん \url{http://www.asyura2.com/0401/nihon11/msg/819.html} 
\end{itemize}

 これまで気が付かなかったのか、少なくとも意識することがなかったものと思いますが、「まさちゃん」が投稿者のお名前となっているようです。文字が小さく分かりづらいですが、「投稿者
まさちゃん 日時 2004 年 3 月 04 日 08:37:18」ともページにありました。

 担当検事ではなく、主任検察官とあります。こういうインタビューで受け答えをすることをリークなどと蔑み批判していたのもモトケンこと矢部善朗弁護士(京都弁護士会)になりますが、改めて読むと、青少年の保護育成に熱心な弁護士という印象を受けます。

\begin{itemize}
\item
  2021年07月20日16時29分の登録:
  「山形.*検事」を@hirono\_hideki @kk\_hirono @s\_hironoで検索 32件の該当 2021-07-20\_16:28の記録
  \url{https://kk2020-09.blogspot.com/2021/07/hironohidekikkhironoshirono322021-07.html} 
\item
  2021年07月20日16時30分の登録:
  「リーク.*モトケン」を@hirono\_hideki @kk\_hirono @s\_hironoで検索 40件の該当 2021-07-20\_16:30の記録
  \url{https://kk2020-09.blogspot.com/2021/07/hironohidekikkhironoshirono402021-07.html} 
\item
  2021年07月20日16時31分の登録:
  「モトケン.*リーク」を@hirono\_hideki @kk\_hirono @s\_hironoで検索 21件の該当 2021-07-20\_16:30の記録
  \url{https://kk2020-09.blogspot.com/2021/07/hironohidekikkhironoshirono212021-07.html} 
\item
  奉納\危険生物・弁護士脳汚染除去装置\金沢地方検察庁御中:
  %@motoken\_tw モトケン%こういう情報をリークする意図は何なのか教えてくれないかな、警察の誰かさん。馬鹿かお前ら。RT
  @47news: 複数の刃物で一斗缶遺体切断か 部位により形状異なる
  \url{https://t.co/6S6MwCEXFx} 
\end{itemize}

 思ったような情報が見つからなかったこともあり、モトケンこと矢部善朗弁護士(京都弁護士会)のツイートで「リーク」をキーワードにしたまとめ記事を神域に作成することにします。

\begin{itemize}
\item
  2021年07月20日16時41分の登録:
  REGEXP:''リーク''/モトケン(@motoken\_tw)の検索(2010-09-04〜2020-06-28/2021年07月20日16時41分の記録98件)
  \url{https://kk2020-09.blogspot.com/2021/07/regexpmotokentw2010-09-042020-06.html} 
\item
  (01/98) TW motoken\_tw(モトケン) 日時:2010-09-04 09:52:00 +0900
  URL: \url{https://twitter.com/motoken\_tw/status/22934658063\textgreater} {}
  その一文を明記するようになっただけ進歩 RT @powerpc970:
  新聞記事で気になるのは「県警への取材によると」の一文。簡単に捜査情報を「お互いの都合よく」リークする馴れ合い関係が怖い
\end{itemize}

 2010年9月4日のツイートで最初に記録されていたものですが、これは初めて見たように感じた内容のツイートです。マスコミにも厳しいのがモトケンこと矢部善朗弁護士(京都弁護士会)ですが、厳しいどころではないものを最近も見かけています。

 今年2021年に入ってから記録されているものです。

\begin{itemize}
\tightlist
\item
  2021年01月15日21時00分の登録:
  \モトケン @motoken\_tw\ある人について、大規模SNSが垢BANし、マスコミが発言を報じなくなれば、憲法が保障する言論の自由は実質的に失われる。そして、SNSや
  \url{http://kk2020-09.blogspot.com/2021/01/motokentwsnsbansns.html} 
\item
  2021年02月22日08時51分の登録:
  \モトケン @motoken\_tw\安全と安心はたしかに違うのだが、科学的に安全ならば多くの人が安心するような記事を書くのがマスコミとして正しいあり方だと思う。でも、科学
  \url{https://kk2020-09.blogspot.com/2021/02/motokentw\_82.html} 
\item
  2021年03月23日08時48分の登録:
  REGEXP:''マスコミ''/モトケン(@motoken\_tw)の検索(2010-03-02〜2021-03-22/2021年03月23日08時48分の記録1015件)
  \url{https://kk2020-09.blogspot.com/2021/03/regexpmotokentw2010-03-022021-03.html} 
\item
  2021年03月23日13時36分の登録:
  \モトケン @motoken\_tw\マスコミが、刑事訴訟法のごく基本的な理解をしようともせず独善的な論理で自己正当化をしているので、法律家(特に報道被害を実感している実務
  \url{https://kk2020-09.blogspot.com/2021/03/motokentw\_62.html} 
\item
  2021年03月27日21時17分の登録:
  \モトケン @motoken\_tw\「『容疑者とは被疑者を犯人と決めつけている言葉だ』というイメージ」を作ってきたのはマスコミですよ。今井さんが記者になる前からの話ですか
  \url{https://kk2020-09.blogspot.com/2021/03/motokentw\_40.html} 
\item
  2021年03月28日10時05分の登録:
  \モトケン @motoken\_tw\原発事故で心身を崩した人は、放射線被曝が原因なのか朝日などのマスコミによる危険煽り記事が原因なのか他に原因があるのかを検証したのかな?
  \url{https://kk2020-09.blogspot.com/2021/03/motokentw\_28.html} 
\item
  2021年04月05日15時27分の登録:
  \モトケン @motoken\_tw\マスコミも、いわゆるジャーナリストも、被害者やその遺族も、多くのツイッターアカウントも、被疑者被告人に自白を求めますね。
  \url{https://kk2020-09.blogspot.com/2021/04/motokentw\_11.html} 
\item
  2021年05月16日19時30分の登録:
  \モトケン @motoken\_tw\根本的には、マスコミの記事の信頼性が低下して、読者の中に第一次資料に当たりたいというニーズが増えてきているということだと思う。そういう
  \url{https://kk2020-09.blogspot.com/2021/05/motokentw\_49.html} 
\item
  2021年05月19日12時55分の登録:
  \モトケン @motoken\_tw\マスコミの常識は社会の非常識、ということが如実にわかるリプとコメント。
  \url{https://kk2020-09.blogspot.com/2021/05/motokentw\_68.html} 
\item
  2021年06月13日19時59分の登録:
  \モトケン @motoken\_tw\言論の自由度ランキングなら世界最高水準だと思いますよ、日本は。報道の自由の観点でも、単にマスコミが無能で腰抜けなだけでしょう。
  \url{https://kk2020-09.blogspot.com/2021/06/motokentw\_71.html} 
\item
  2021年06月19日11時02分の登録:
  \モトケン @motoken\_tw\被告の主張は、なかなか大胆な主張だと思うが、マスコミの記事は基本的に信用できないので、被告の準備書面を読みたい気がする。私なら、依頼者
  \url{https://kk2020-09.blogspot.com/2021/06/motokentw\_19.html} 
\item
  2021年06月27日10時33分の登録:
  \モトケン @motoken\_tw\こういう問題が生じる原因の一つとして、ワクチンの危険を煽るマスコミの影響が極めて大きいと思う。接種を拒否する職員もそういう職員が出てく
  \url{https://kk2020-09.blogspot.com/2021/06/motokentw\_27.html} 
\item
  2021年06月27日10時34分の登録:
  \モトケン @motoken\_tw\自社の儲け(経済的利益)でしょうね。かなり以前から、週刊誌の(デマをデマと認めない)スキャンダリズムが、新聞を含むマスコミ全体に広がっ
  \url{https://kk2020-09.blogspot.com/2021/06/motokentw\_45.html} 
\item
  2021年07月02日23時19分の登録:
  REGEXP:''マスコミ''/モトケン(@motoken\_tw)の検索(2010-03-02〜2021-06-27/2021年07月02日23時19分の記録1036件)
  \url{https://kk2020-09.blogspot.com/2021/07/regexpmotokentw2010-03-022021-06.html} 
\item
  2021年07月02日23時35分の登録:
  REGEXP:''マスコミ''/モトケン(@motoken\_tw)の検索(2021-01-15〜2021-06-27/2021年07月02日23時35分の記録29件)
  \url{https://kk2020-09.blogspot.com/2021/07/regexpmotokentw2021-01-152021-06.html} 
\end{itemize}

 「REGEXP:''リーク''/モトケン(@motoken\_tw)の検索(2010-09-04〜2020-06-28/2021年07月20日16時41分の記録98件)」に戻ります。

\begin{itemize}
\item
  (06/98) TW motoken\_tw(モトケン) 日時:2010-10-03 10:04:00 +0900
  URL: \url{https://twitter.com/motoken\_tw/status/26222378576\textgreater} {}
  第三者機関による検証の際には、検察によるリークの当否も対象にすべきだと思います。RT
  @kazemachiroman:
  こういう報道自体がどうなんだろうね。~「\ldots」と言われたと最高検に供述していることが捜査関係者の話で分かった~asahi
  \url{http://bit.ly/aUBB0i}  target="\_blank"\textgreater \url{http://bit.ly/aUBB0i} 
\item
  (14/98) TW motoken\_tw(モトケン) 日時:2010-11-08 09:00:00 +0900
  URL:
  \url{https://twitter.com/motoken\_tw/status/1423759915352064\textgreater} {}
  要するに、リーク問題の本質はメディアの甘えと怠慢ですね。RT
  @kazu1961omi:
  いわゆるリークが問題とされるのは、メディアが権力の意向に沿った情報拡散をすることかと。メディアの存在意義のひとつは、権力が独占している情報を取ってきて、みんなに伝えることだと思ってます。
\item
  (26/98) TW motoken\_tw(モトケン) 日時:2011-09-02 20:57:00 +0900
  URL:
  \url{https://twitter.com/motoken\_tw/status/109595911541370880\textgreater} {}
  捜査情報をいくらマスコミにリークしても問題にならないけどな。>捜査資料漏えい事件、捜査1課警部懲戒免職
  : 社会 : YOMIURI ONLINE(読売新聞) \url{http://t.co/dKtlsqi} 
  target="\_blank"\textgreater \url{http://t.co/dKtlsqi}  via @yomiuri\_online
\item
  (29/98) TW motoken\_tw(モトケン) 日時:2013-02-11 11:52:00 +0900
  URL:
  \url{https://twitter.com/motoken\_tw/status/300799480247250944\textgreater} {}
  ブログを更新しました。【ヤベラボ】:
  どうしてリークするのか?【遠隔操作ウイルス事件】 \url{http://t.co/h0MaPQ7T} 
  target="\_blank"\textgreater \url{http://t.co/h0MaPQ7T} 
\item
  (47/98) TW motoken\_tw(モトケン) 日時:2013-03-10 01:17:00 +0900
  URL:
  \url{https://twitter.com/motoken\_tw/status/310424065322151937\textgreater} {}
  遠隔操作事件で最近警察のリーク報道が少ないのは、リークしたがる幹部連中が理解できないレベルの話で捜査が進んでいるからではないかと憶測w
\end{itemize}

 探していた内容のツイートを見つけました。次のツイートです。

\begin{itemize}
\item
  (65/98) TW motoken\_tw(モトケン) 日時:2018-04-06 09:40:00 +0900
  URL:
  \url{https://twitter.com/motoken\_tw/status/982055370448187392\textgreater} {}
  捜査機関(当然、公務員)による捜査情報のリークが国家公務員法や地方公務員法の守秘義務違反になるかどうかは、けっこう興味深い問題。\textgreater{}
  いろんな方面に波及しますからね。
\item
  (73/98) TW motoken\_tw(モトケン) 日時:2018-04-07 10:11:00 +0900
  URL:
  \url{https://twitter.com/motoken\_tw/status/982425707736371200\textgreater} {}
  @analoggaii
  検察に限らず、警察発表や警察によるリークなど、犯罪報道における日常的な問題です。
\item
  (75/98) TW motoken\_tw(モトケン) 日時:2018-04-07 11:22:00 +0900
  URL:
  \url{https://twitter.com/motoken\_tw/status/982443410756845569\textgreater} {}
  検察がリークした、といって検察を非難している人たちは、検察が一切の捜査情報を出さなくなったら、検察は仕事をしていない、といって非難するんだろうな。\textgreater{}
  検察としてどっちが健全かは議論があると思うが、御都合主義の人たちはどっちに転んでも非難する。\textgreater{}
  なお、検察を警察に置き換えても同じ。
\item
  (87/98) TW motoken\_tw(モトケン) 日時:2018-08-20 12:46:00 +0900
  URL:
  \url{https://twitter.com/motoken\_tw/status/1031386843839778816\textgreater} {}
  @sagechin\_MR
  毎日の記事なんかは警察の誘導的虚偽リークの疑いがありますよ。
\item
  (98/98) TW motoken\_tw(モトケン) 日時:2020-06-28 10:05:00 +0900
  URL:
  \url{https://twitter.com/motoken\_tw/status/1277045523556388864\textgreater} {}
  @fujibook611
  マスコミに節度とか同種事犯の抑止などというものを期待するのは無理。連中は無節操の極み。\textgreater{}
  こういう情報をリークする警察が問題。\textgreater{}
  前職だったころは冒陳でどこまで書くかにも気を使っていたのに。
\end{itemize}

 ブログ記事の埋め込みツイートの数が心配になっています。モトケンこと矢部善朗弁護士(京都弁護士会)が、被害者安藤文さん、安藤健次郎さん家族、私や母親、親戚に与えた深刻重大な影響については、折りに触れて個別具体的なご指摘をしていきたいと思います。

 小山田圭吾氏の件で、モトケンこと矢部善朗弁護士(京都弁護士会)の法律家としてのスタンスや人柄が再確認できたのですが、さらにパワーアップしているとも感じました。撒き餌の効果で、条件にかなった依頼者を見つけやすい、出会いやすいという経験則があるのかもしれません。

\begin{itemize}
\tightlist
\item
  〈〈〈 2021/07/20 17:19:26 Linux Emacs: 〈〈〈
\end{itemize}

\hypertarget{xxxxx}{%
\paragraph{xxxxx}\label{xxxxx}}

\hypertarget{ux53c2ux8003ux8cc7ux6599}{%
\section{参考資料}\label{ux53c2ux8003ux8cc7ux6599}}

\hypertarget{ux5f01ux8b77ux58eb}{%
\subsection{弁護士}\label{ux5f01ux8b77ux58eb}}

\hypertarget{ux6df1ux6fa4ux8aedux53f2ux5f01ux8b77ux58ebux7b2cux4e8cux6771ux4eacux5f01ux8b77ux58ebux4f1a}{%
\subsubsection{深澤諭史弁護士(第二東京弁護士会)}\label{ux6df1ux6fa4ux8aedux53f2ux5f01ux8b77ux58ebux7b2cux4e8cux6771ux4eacux5f01ux8b77ux58ebux4f1a}}

\hypertarget{ux7a00ux306bux3053ux3046ux3044ux3046ux4ebaux3044ux307eux3059ux304cux5f01ux8b77ux58ebux306eux4ed5ux4e8bux306eux90aaux9b54ux3092ux3057ux3066ux6575ux306bux5869ux3092ux9001ux308bux3060ux3051ux306aux3093ux3067ux3059ux3051ux308cux3069ux3082ux306dux3047ux3068ux3044ux3046ux6df1ux6fa4ux8aedux53f2ux5f01ux8b77ux58ebux306eux30c4ux30a4ux30fcux30c8}{%
\paragraph{「(・∀・)稀にこういう人いますが、弁護士の仕事の邪魔をして、敵に塩を送るだけなんですけれどもねぇ。。。。」という深澤諭史弁護士のツイート}\label{ux7a00ux306bux3053ux3046ux3044ux3046ux4ebaux3044ux307eux3059ux304cux5f01ux8b77ux58ebux306eux4ed5ux4e8bux306eux90aaux9b54ux3092ux3057ux3066ux6575ux306bux5869ux3092ux9001ux308bux3060ux3051ux306aux3093ux3067ux3059ux3051ux308cux3069ux3082ux306dux3047ux3068ux3044ux3046ux6df1ux6fa4ux8aedux53f2ux5f01ux8b77ux58ebux306eux30c4ux30a4ux30fcux30c8}}

\begin{itemize}
\tightlist
\item
  〉〉〉 Linux Emacs: 2021/04/20 08:04:29 〉〉〉
\end{itemize}

:CATEGORIES: @kanazawabengosi \#金沢弁護士会 @JFBAsns
日本弁護士連合会(日弁連) \#法務省 @MOJ\_HOUMU \#深澤諭史弁護士

\begin{itemize}
\item
  TW fukazawas(深澤諭史) 日時: 2021-04-19 19:26 URL:
  \url{https://twitter.com/fukazawas/status/1384091146134032387\textgreater} {}
  (・∀・)基本、相手方の悪性格の主張にこだわった時点で、負け筋ですね。
  ¥\n
  (;・∀・)「証拠からも明白な真実によれば、こっちが悪いのは明白なので、相手の悪性を主張しまくることでなんとかしたい(できるわけない)」ってことですので。
  \url{https://t.co/JwGubQHuBn} 
\item
  TW fukazawas(深澤諭史) 日時: 2021-04-19 19:19 URL:
  \url{https://twitter.com/fukazawas/status/1384089185460441099\textgreater} {}
  (・∀・)稀にこういう人いますが、弁護士の仕事の邪魔をして、敵に塩を送るだけなんですけれどもねぇ。。。。
  \url{https://t.co/ItfvtIDTcC} 
\item
  RT fukazawas(深澤諭史)|KR31917261(KR🐶) 日時:2021-04-19
  19:18/2021-04-19 19:17 URL:
  \url{https://twitter.com/fukazawas/status/1384089004245590021} 
  \url{https://twitter.com/KR31917261/status/1384088870220800003\textgreater} {}
  なんか依頼者からのメールで相手方への罵詈雑言と当職への不満が長文で送られてきたな。辞任すっぞ?
\item
  TW fukazawas(深澤諭史) 日時: 2021-04-19 16:21 URL:
  \url{https://twitter.com/fukazawas/status/1384044578571882503\textgreater} {}
  (^ω^)弁護士会からFATF第4次対日相互審査対応ワーキンググループの委員選任通知がきたお・・・・。
\end{itemize}

 昨日4月19日の19時19分となっている深澤諭史弁護士の「敵に塩を送るだけなんですけれどもねぇ」というツイートですが,夕方の早めの時間に読んでいたように思いました。その30分程前に,たまたま上杉謙信の塩の記事を読んでいただけに,おやっと思う発見でした。

〉〉〉 kk\_hironoのリツイート 〉〉〉

\begin{itemize}
\tightlist
\item
  RT
  kk\_hirono(刑事告発・非常上告_金沢地方検察庁御中)|KR31917261(KR🐶)
  日時:2021-04-20 08:12/2021/04/19 19:17 URL:
  \url{https://twitter.com/kk\_hirono/status/1384283686246383616} 
  \url{https://twitter.com/KR31917261/status/1384088870220800003} 
  \textgreater{}
  なんか依頼者からのメールで相手方への罵詈雑言と当職への不満が長文で送られてきたな。辞任すっぞ?
\end{itemize}

〉〉〉 kk\_hironoのリツイート 〉〉〉

\begin{itemize}
\tightlist
\item
  RT
  kk\_hirono(刑事告発・非常上告_金沢地方検察庁御中)|hirono\_hideki(奉納\さらば弁護士鉄道・泥棒神社の物語)
  日時:2021-04-20 08:12/2021/04/19 20:06 URL:
  \url{https://twitter.com/kk\_hirono/status/1384283709252112389} 
  \url{https://twitter.com/hirono\_hideki/status/1384101191584423940} 
  \textgreater{} ▶
  ブロックされたツイート%fukazawas(深澤諭史)%2021/04/19 19:19:11%
  \url{https://t.co/4LZJNAZrEO}  \textgreater{}
  (・∀・)稀にこういう人いますが、弁護士の仕事の邪魔をして、敵に塩を送るだけなんですけれどもねぇ。。。。
  \url{https://t.co/AEiWVqPqr0} 
\end{itemize}

〉〉〉 kk\_hironoのリツイート 〉〉〉

\begin{itemize}
\tightlist
\item
  RT
  kk\_hirono(刑事告発・非常上告_金沢地方検察庁御中)|hirono\_hideki(奉納\さらば弁護士鉄道・泥棒神社の物語)
  日時:2021-04-20 08:12/2021/04/19 19:51 URL:
  \url{https://twitter.com/kk\_hirono/status/1384283748825399301} 
  \url{https://twitter.com/hirono\_hideki/status/1384097388311105536} 
  \textgreater{} 上杉謙信の美談「敵に塩を送る」実は打算だった \textbar{}
  リーダーシップ・教養・資格・スキル \textbar{} 東洋経済オンライン
  \textbar{} 経済ニュースの新基準 \url{https://t.co/mybRek8RJ5} 
  定価を厳守させるとも伝えている。そして、この決して値段を変えさせないという謙信のスタンスは、どの編纂史料でも一致している。
\end{itemize}

〉〉〉 kk\_hironoのリツイート 〉〉〉

\begin{itemize}
\tightlist
\item
  RT
  kk\_hirono(刑事告発・非常上告_金沢地方検察庁御中)|hirono\_hideki(奉納\さらば弁護士鉄道・泥棒神社の物語)
  日時:2021-04-20 08:12/2021/04/19 19:48 URL:
  \url{https://twitter.com/kk\_hirono/status/1384283830710800388} 
  \url{https://twitter.com/hirono\_hideki/status/1384096477710929927} 
  \textgreater{} 上杉謙信の美談「敵に塩を送る」実は打算だった \textbar{}
  リーダーシップ・教養・資格・スキル \textbar{} 東洋経済オンライン
  \textbar{} 経済ニュースの新基準 \url{https://t.co/wZoe5UdDr2} 
\end{itemize}

〉〉〉 kk\_hironoのリツイート 〉〉〉

\begin{itemize}
\tightlist
\item
  RT
  kk\_hirono(刑事告発・非常上告_金沢地方検察庁御中)|hirono\_hideki(奉納\さらば弁護士鉄道・泥棒神社の物語)
  日時:2021-04-20 08:12/2021/04/19 19:42 URL:
  \url{https://twitter.com/kk\_hirono/status/1384283859626328064} 
  \url{https://twitter.com/hirono\_hideki/status/1384095117896601606} 
  \textgreater{} 「29歳元アイドル」会社員になって痛感した無力 \textbar{}
  リーダーシップ・教養・資格・スキル \textbar{} 東洋経済オンライン
  \textbar{} 経済ニュースの新基準 \url{https://t.co/SZ9bMNGN07} 
  とはいえ現状を嘆いても仕方がない。私は人生がうまくいかない理由を環境とか、誰かのせいにするのは嫌いだ。
\end{itemize}

 昨日は,16時過ぎぐらいに出掛けて,バイクのオイル交換をしに行き,そのあとAコープ能都店で少し買い物をして銭湯に行き,家に戻ってパソコンをつけて,時計をみたところ,ちょうど19:00となっていたことを憶えています。

 
「29歳元アイドル」会社員になって痛感した無力」という記事は,2,3日前からTwitterのトレンドに入っていたように思うのですが,気になったので読んでみました。読み終えた頃にページ内の一覧に「上杉謙信の美談「敵に塩を送る」実は打算だった」を見つけたように思います。

 武田信玄の名前も以前ほど見かけなくなっていますが,山梨県を代表するような歴史上の人物であることを思い出しました。ずいぶんと残酷なことをやったと以前,本で読んだようなことを思い出したのですが,ネットでは見かけていない情報かもしれません。

 1つは,妊娠した女性の腹を開いたという話で,もう一つは敵の武将の妻らを売春婦にしたというような話だったと思います。歴史の本は買うことも図書館で借りたこともないので,官本で読んだはずです。少し調べて確認をしてみます。

\begin{itemize}
\tightlist
\item
  武田信虎 悪逆非道の戦国武将 妊婦の腹裂き \textbar{}
  猫奴隷とあるじさま達 ~時々馬・うずら~ \url{https://t.co/qSZx6rruMw} 
\end{itemize}

 有名な話だったと確認しましたが,武田信玄ではなく,武田信玄の父親の武田信虎とあります。それもデマの可能性を指摘されています。

\begin{quote}
《引用の始まり》
\end{quote}

\begin{quote}
ではその比較する大名とは、謙信とは最大のライバルであった甲斐の武田信玄です。彼にもちゃんと人身売買した記録が残されています。資料の出所は、妙法寺というお寺の記録です。

「男女生取成候、悉甲州ヘ引越申候、去程ニ二貫三貫五貫十貫ニテモ親類アル人ハ承ヶ申候」

この現代語はというと、

「(武田軍は)戦によって生け取った男女の悉くを、甲州(甲斐)へと連れて行きました。二貫・三貫・五貫・十貫という値段であっても、親類親族がいれば身請けを許可していました」

ってなります。

ここでいう身請けとは、武田軍によって捕虜となった者の親族・親類には、武田のふっかけた値段を払えば開放しますってことです。

捕虜っていうのは、簡単に言うと「奴隷」ですね。

この奴隷となってしまった場合、それは悲惨な生活が待っています。先ず自由がなくて、まともな衣食住の生活は送れません。男の奴隷なら、甲斐では金山等の採掘に従事させられ、酷使されて捨てられます。女性ともなれば売春婦のような身分に置かれます。
\end{quote}

\begin{quote}
《引用の終わり》
\end{quote}

\begin{itemize}
\tightlist
\item
  人身売買/越後の虎
  \url{http://26.pro.tok2.com/\textasciitilde}  yataro/sy-hitouri.html
\end{itemize}

 上杉謙信が人身売買を許可していたという話が出てきたのですが,武田信玄も人身売買をやっていたという話になって,武田信玄の人身売買の値段が上杉謙信の100倍だったという話が出てきました。

\begin{quote}
《引用の始まり》
\end{quote}

\begin{quote}
だが実際の戦場では、「乱取り」という、「奴隷狩り」が行われていた。甲斐国の戦国大名である武田氏の事例から見てみよう。戦国初期の甲斐国の生活や世相を記録した『妙法寺記』という史料には、武田信虎(信玄の父)の軍勢が他国の「足弱」(あしよわ)百人ばかりを獲っていった、という記述がある。

「足弱」とは、女性や老人、子供のような弱い立場にある人々のことだと考えられている。つまり、武田氏の軍勢は戦争のどさくさに紛れ、「戦利品」として他国の民を強奪したのだ。このように、戦場において人、物資を奪うことを「乱取り」といい、戦場で得た物は自分の資産にすることができた。人や物資は売ることも可能だったので、戦争に参加することは、経済的に豊かになれるという「うまみ」があったのだ。
\end{quote}

\begin{quote}
《引用の終わり》
\end{quote}

\begin{itemize}
\tightlist
\item
  あの有名武将も奴隷を認めていた!
  合戦の目的は「奴隷狩り」!? ドラマでは描かれない戦国時代の実態
  \textbar{} ダ・ヴィンチニュース \url{https://ddnavi.com/news/312424/a/} 
\end{itemize}

 もう一つ記事の引用をしましたが,『妙法寺記』という記録が論拠となっていて,それも武田信玄ではなく武田信虎のこととして記録されているようです。

 少し思い出したのですが,武田信玄が父の武田信虎を国外追放したという話があります。武田信玄といえば,治水工事で堤防を作ったことでも有名で,これは高い評価を受けているように思います。

\begin{quote}
《引用の始まり》
\end{quote}

\begin{quote}
信玄堤は、増水した河川のエネルギーを内側に押し込めるのではなく、あらかじめ堤防を分断させてエネルギーを分散させる方法である。400年前、武田信玄は、堤防に切り込みをつくらせて、増水した川の水を湿地や田んぼ、遊水池に流れだす仕組みにした。切り込みが斜めになっているので、雨がやめば、川の外に分散した水は再び元の川に戻っていく。
\end{quote}

\begin{quote}
《引用の終わり》
\end{quote}

\begin{itemize}
\tightlist
\item
  「決壊しない堤防」をつくった武田信玄の発想法に学べ \textbar{} Forbes
  JAPAN(フォーブス ジャパン)
  \url{https://forbesjapan.com/articles/detail/30203/2/1/1} 
\end{itemize}

 山梨といえば地域的な特性として,日本住血吸虫のこともあるのですが,山梨県甲府市が深澤諭史弁護士の出身地という情報もあります。弁護士という職業に対する絶対的な信仰のようなものを深澤諭史弁護士に感じております。

\begin{itemize}
\tightlist
\item
  〈〈〈 2021/04/20 09:01:54 Linux Emacs: 〈〈〈
\end{itemize}

\hypertarget{ux4f0aux85e4ux587eux8b1bux5e2bux306eux30c4ux30a4ux30fcux30c8ux3092ux5f15ux7528ux3057ux53f8ux6cd5ux5236ux5ea6ux6539ux9769ux306eux5931ux6557ux3092ux61b2ux6cd5ux5b66ux8005ux4f50ux85e4ux5e78ux6cbbux6c0fux306bux95a2ux9023ux4ed8ux3051ux308bux6df1ux6fa4ux8aedux53f2ux5f01ux8b77ux58ebux306eux30c4ux30a4ux30fcux30c8}{%
\paragraph{伊藤塾講師のツイートを引用し,司法制度改革の失敗を憲法学者佐藤幸治氏に関連付ける深澤諭史弁護士のツイート}\label{ux4f0aux85e4ux587eux8b1bux5e2bux306eux30c4ux30a4ux30fcux30c8ux3092ux5f15ux7528ux3057ux53f8ux6cd5ux5236ux5ea6ux6539ux9769ux306eux5931ux6557ux3092ux61b2ux6cd5ux5b66ux8005ux4f50ux85e4ux5e78ux6cbbux6c0fux306bux95a2ux9023ux4ed8ux3051ux308bux6df1ux6fa4ux8aedux53f2ux5f01ux8b77ux58ebux306eux30c4ux30a4ux30fcux30c8}}

\begin{itemize}
\tightlist
\item
  〉〉〉 Linux Emacs: 2021/04/22 13:07:05 〉〉〉
\end{itemize}

:CATEGORIES: @kanazawabengosi \#金沢弁護士会 @JFBAsns
日本弁護士連合会(日弁連) \#法務省 @MOJ\_HOUMU \#深澤諭史弁護士
\#司法制度改革 \#法科大学院 \#伊藤塾

\begin{itemize}
\item
  TW fukazawas(深澤諭史) 日時: 2021/04/22 09:15:49 URL:
  \url{https://twitter.com/fukazawas/status/1385024504519290901} 
  \textgreater{} 第180回国会 法務委員会 第4号\\
  \textgreater{} \url{https://t.co/hnrNZkgFAs} 
  \textgreater{} 佐藤幸治先生\\
  \textgreater{}
  具体的な方策を至急に出して・・・私の口からはちょっと申しにくい、しかももう書斎生活に戻っている身ですから正確な状況を把握していないこともありますので、この辺でお許しいただければと思います。
  \url{https://t.co/7aMNnp4EL3} 
\item
  RT fukazawas(深澤諭史)|0kazakikei(岡崎 敬) 日時:2021-04-22
  09:15/2021-04-20 12:37 URL:
  \url{https://twitter.com/fukazawas/status/1385024306749460489} 
  \url{https://twitter.com/0kazakikei/status/1384350417505587206} 
  \textgreater{}
  多様な人材を確保するためのはずの法科大学院制度なのに、志望者の絶対数が激減してしまっているのでは、何のための制度改革だったんだということになる。\\
  \textgreater{}\\
  \textgreater{}
  何が問題だったのか、どうすればいいかということが、制度を作り維持している立場の人たちから、全く聞こえてこない。これを、無責任という。
  \url{https://t.co/XCku2dlMqo} 
\end{itemize}

 深澤諭史弁護士のタイムラインではリツイートの後に,リツイートしたツイートの引用ツイートが並んでいるだけです。これまでだと周辺には,司法制度改革を皮肉って批判するツイートが散りばめられていたという印象なのですが,最近は全体的に自重気味とも感じられるところです。

\begin{itemize}
\item
  2021年04月22日11時50分の登録:
  REGEXP:''佐藤幸治''/深澤諭史(@fukazawas)の検索(2013-01-31〜2021-04-22/2021年04月22日11時50分の記録107件)
  \url{https://kk2020-09.blogspot.com/2021/04/regexpfukazawas2013-01-312021-04.html} 
\item
  (003/107) TW fukazawas(深澤諭史) 日時:2013-08-18 22:29:00 +0900
  URL:
  \url{https://twitter.com/fukazawas/status/369088652405510144\textgreater} {}
  @donetter\_LP
  市場原理ってそういうものですよね。売り手だけではなくて買い手も競争すると。正確な情報に基づく合理的な判断を出来ない人が,買い手・売り手を問わずに悲惨な目に遭うことを目指したのが,佐藤幸治先生が実現しようとした平成司法改革の素晴らしき成果です。
\item
  (006/107) TW fukazawas(深澤諭史) 日時:2014-05-31 18:38:00 +0900
  URL:
  \url{https://twitter.com/fukazawas/status/472673586885296128\textgreater} {}
  佐藤幸治先生は,平成の司法改革で導入された裁判員制度のおかげで,調書中心から公判中心になった!っておっしゃっているそうです。\textgreater\textgreater{}
  ですが,先生ご自身は,予備校教育に問題があるって断じたとき,直接話も聞かなかったそうですね。\textgreater\textgreater{}
  まずは,ご自身から直接主義を採用すべきではないでしょうか
\item
  (008/107) TW fukazawas(深澤諭史) 日時:2014-10-28 14:37:00 +0900
  URL:
  \url{https://twitter.com/fukazawas/status/526970889359732737\textgreater} {}
  医療の世界も,国民に開かれ信用されるように,大手術には,くじ引きで選ばれた「医療員」が手術に参加するようにしてはどうか。\textgreater{}
  専門知識の無い一般国民の感覚で内臓を切り裂いて,よりよき医療を実現しよう。\textgreater{}
  患者第一号は,類似の制度導入に尽力された佐藤幸治先生にお願いしよう。\textgreater{}
  (・∀・)
\item
  (010/107) TW fukazawas(深澤諭史) 日時:2015-04-19 19:04:00 +0900
  URL:
  \url{https://twitter.com/fukazawas/status/589731208059166721\textgreater} {}
  佐藤幸治教授に足りないもの \url{http://t.co/LenMr7XwAQ} 
  target="\_blank"\textgreater \url{http://t.co/LenMr7XwAQ\textgreater} {}
  「制度設計者として無責任のそしりを免れない。弁護士会など他人に責任転嫁しているうちは、「世間知らずが祭り上げられて浮かれたあげく、晩節を汚した」との評価を受け続けるだろう。」
\item
  (011/107) TW fukazawas(深澤諭史) 日時:2015-06-28 12:22:00 +0900
  URL:
  \url{https://twitter.com/fukazawas/status/614997243226009600\textgreater} {}
  平成の司法改革,佐藤幸治先生や高橋宏志先生らが夢見た,新しい公選弁護のあり方ですね。\textgreater{}
  (#・∀・)両先生もさぞやお喜びでしょう。 \url{https://t.co/TkcXuB5njC} 
\item
  (012/107) TW fukazawas(深澤諭史) 日時:2015-07-22 18:49:00 +0900
  URL:
  \url{https://twitter.com/fukazawas/status/623791961539919873\textgreater} {}
  >RT\textgreater{}
  自由競争とか,市場原理とか,無邪気にいっている先生,特に佐藤幸治先生とかには,百回くらい読んで頂きたい記事ですね。\textgreater{}
  (#・∀・)
\item
  (017/107) RT fukazawas(深澤諭史)|mstk\_Horiguchi(ほりぐち)
  日時:2016-03-16 12:00:00 +0900/2016-03-16 11:56:00 +0900 URL:
  \url{https://twitter.com/fukazawas/status/709937242035216384} 
  \url{https://twitter.com/mstk\_Horiguchi/status/709936436582035456\textgreater} {}
  成仏理論ができてからもう10年か。時が経つのは早いな。\textgreater{}
  高橋宏志 大先生も佐藤幸治
  大先生も功徳が足りないのかまだ成仏されていないけども
\end{itemize}

 ついつい引用掲載の数が増えてしまうのですが,次は時系列を逆に最近のものから眺めていきたいと思います。

\begin{itemize}
\item
  (103/107) TW fukazawas(深澤諭史) 日時:2021-01-16 08:40:00 +0900
  URL:
  \url{https://twitter.com/fukazawas/status/1350226258026663937\textgreater} {}
  佐藤幸治先生、「市民に寄り添う弁護士」ってなに?\textgreater{}
  これまでは、そういう弁護士がいなかったの?\textgreater{}
  どういう根拠でいっているの?\textgreater{}
  また、以前みたいに、「直接見聞きしていないけれども、いろんなものを通じて知っているつもり」って話ですか?・・・
  \url{https://t.co/avJ2teNkWT} 
\item
  (102/107) TW fukazawas(深澤諭史) 日時:2020-12-30 08:05:00 +0900
  URL:
  \url{https://twitter.com/fukazawas/status/1344057059319246848\textgreater} {}
  「旧司法試験は一発試験で弊害がある」との学者の先生による主張 -
  Schulze BLOG \url{https://t.co/YiYFUg7hiW\textgreater} {}
  佐藤幸治「私の口からはちょっと申しにくい、しかももう書斎生活に戻っている身ですから正確な状況を把握していないこともありますので、この辺でお許しいただければと思います。」
\item
  (092/107) TW fukazawas(深澤諭史) 日時:2020-03-05 09:24:00 +0900
  URL:
  \url{https://twitter.com/fukazawas/status/1235360508011778048\textgreater} {}
  佐藤幸治先生は,旧試験が「資質の確保に重大な影響」があると断言しておきながら,具体的にどこがどう問題なのか,問われて,中身についてはほぼ答えられなかったですよね。\textgreater{}
  予備校についても,問題があると言いながら,結局直接話は聞いていない・・・
  \url{https://t.co/lhvTb9F6AJ} 
\item
  (089/107) TW fukazawas(深澤諭史) 日時:2019-12-17 08:53:00 +0900
  URL:
  \url{https://twitter.com/fukazawas/status/1206724117652590592\textgreater} {}
  国家等に対して膨大な債務を負担し,生業を営むに窮する状況に追い込まれた弁護士が,どうやって,憲法違反を主張する訴訟等を行えるのか。\textgreater{}
  佐藤幸治先生からは,機会があれば,どうか休み休みご教示にあずかりたいと思っています。\textgreater{}
  (・∀・) \url{https://t.co/NXaKu1KEJU} 
\item
  (088/107) TW fukazawas(深澤諭史) 日時:2019-12-17 08:48:00 +0900
  URL:
  \url{https://twitter.com/fukazawas/status/1206722891493957632\textgreater} {}
  人類の英知・立憲主義、悲劇の背景を忘れるな 佐藤幸治・京大名誉教授:朝日新聞デジタル
  \url{https://t.co/IoV9Uw7gf9\textgreater} {}
  (#・∀・)あれだけ,憲法違反を主張する立場の在野法曹たる弁護士を弱体化させる政策を推進しておい・・・
  \url{https://t.co/dohpF6mfgU} 
\end{itemize}

 2019年10月19日のツイート辺りで読み込みをやめたのですが,「第180回国会 法務委員会」に関連したツイートが多く,深澤諭史弁護士にはよほどのこだわりがあるようです。

 すべての諸悪の根源を司法制度改革に求めているようですが,なぜ司法制度改革が必要とされたのか,時代背景に対する考察や問題点の洗い出しのようなものは全くうかがえず,結果が悪く弁護士が疲弊したので時代を巻き戻せと深澤諭史弁護士は正論を述べている感覚のようです。

 見ている世界,見えている世界の違いもあるのでしょうが,深澤諭史弁護士を批判するようなツイートはみたこともなく,同調的に指示し,稼ぎ頭のように評価する弁護士は散見されます。代表的なのが小倉秀夫弁護士の「寿司債権」です。

\begin{itemize}
\item
  2018年04月21日10時59分の登録:
  \小倉秀夫 @Hideo\_Ogura\売れっ子→寿司債権者集会。RT @fukazawas:
  朝から取材に対応するメールを書いている当職・・・。
  \url{http://hirono2014sk.blogspot.com/2018/04/hideoogurart-fukazawas\_21.html} 
\item
  2021年02月06日17時22分の登録:
  REGEXP:''寿司債権''/小倉秀夫(@Hideo\_Ogura)の検索(2017-01-19〜2019-04-20/2021年02月06日17時21分の記録17件)
  \url{http://kk2020-09.blogspot.com/2021/02/regexphideoogura2017-01-192019-04.html} 
\item
  〈〈〈 2021/04/22 13:42:05 Linux Emacs: 〈〈〈
\end{itemize}

\hypertarget{ux88c1ux5224ux54e1ux5236ux5ea6ux3092ux76aeux8089ux308bux6df1ux6fa4ux8aedux53f2ux5f01ux8b77ux58ebux81eaux8eabux306bux3088ux308bux904eux53bbux306eux30c4ux30a4ux30fcux30c8ux306eux30eaux30c4ux30a4ux30fcux30c8ux304bux3089ux518dux767aux898bux306bux81f3ux3063ux305fux65b0ux6f5fux5973ux5150ux6bbaux5bb3ux4e8bux4ef6ux306bux304aux3051ux308bux9ed9ux79d8ux6a29ux3068ux3046ux306eux5b57ux3068ux306eux30b3ux30e9ux30dcux306bux3088ux308bux30d2ux30e3ux30c3ux30cfux30fcux306eux30c4ux30a4ux30fcux30c81}{%
\paragraph{裁判員制度を皮肉る深澤諭史弁護士自身による過去のツイートのリツイートから再発見に至った、新潟女児殺害事件における黙秘権とうの字とのコラボによるヒャッハーのツイート(1)}\label{ux88c1ux5224ux54e1ux5236ux5ea6ux3092ux76aeux8089ux308bux6df1ux6fa4ux8aedux53f2ux5f01ux8b77ux58ebux81eaux8eabux306bux3088ux308bux904eux53bbux306eux30c4ux30a4ux30fcux30c8ux306eux30eaux30c4ux30a4ux30fcux30c8ux304bux3089ux518dux767aux898bux306bux81f3ux3063ux305fux65b0ux6f5fux5973ux5150ux6bbaux5bb3ux4e8bux4ef6ux306bux304aux3051ux308bux9ed9ux79d8ux6a29ux3068ux3046ux306eux5b57ux3068ux306eux30b3ux30e9ux30dcux306bux3088ux308bux30d2ux30e3ux30c3ux30cfux30fcux306eux30c4ux30a4ux30fcux30c81}}

\begin{itemize}
\tightlist
\item
  〉〉〉 Linux Emacs: 2021/06/26 15:08:19 〉〉〉
\end{itemize}

:CATEGORIES: @kanazawabengosi \#金沢弁護士会 @JFBAsns
日本弁護士連合会(日弁連) \#法務省 @MOJ\_HOUMU \#深澤諭史弁護士
\#黙秘権 \#死刑判決 \#うの字

\begin{itemize}
\item
  RT fukazawas(深澤諭史)|koutaros2(弁護士 川口 洸太朗)
  日時:2021-06-25 19:38/2021-06-23 19:28 URL:
  \url{https://twitter.com/fukazawas/status/1408374127677689858} 
  \url{https://twitter.com/koutaros2/status/1407646701079330823} 
  \textgreater{}
  年間300冊読書する弁護士が薦める、読まないと人生損するベスト10冊\\
  \textgreater{} \url{https://t.co/W9Xap0wx9t}  \url{https://t.co/CYburmGWfQ} 
\item
  RT fukazawas(深澤諭史)|attorney\_sasaki(弁護士佐々木さくら)
  日時:2021-06-25 20:28/2021-06-25 18:08 URL:
  \url{https://twitter.com/fukazawas/status/1408386588036386825} 
  \url{https://twitter.com/attorney\_sasaki/status/1408351425323671560} 
  \textgreater{}
  最高裁が発行している市民向けの裁判員制度のパンフレット。\\
  \textgreater{}
  懲役刑か、無期か、死刑か、本当にこの人が有罪なのかを決めるのに、こういう雰囲気になるんですかねぇ。\\
  \textgreater{} 市民に対しても被告人に対しても侮辱的すぎませんか。
  \url{https://t.co/iXFt2CQAo4} 
\item
  RT fukazawas(深澤諭史)|cho\_seiho(CHO Seiho/趙誠峰)
  日時:2021-06-25 20:28/2021-06-25 19:47 URL:
  \url{https://twitter.com/fukazawas/status/1408386614695399429} 
  \url{https://twitter.com/cho\_seiho/status/1408376386171981827} 
  \textgreater{} これはひどい。。\\
  \textgreater{} いい勉強になったって。。\\
  \textgreater{}
  勉強じゃないんだよ。そのあなたの判断で人の生死、自由がかかってるんです。
  \url{https://t.co/0sAcdAvgaY} 
\item
  RT
  fukazawas(深澤諭史)|harrier0516osk(向原総合法律事務所 弁護士向原)
  日時:2021-06-25 20:28/2021-06-25 20:11 URL:
  \url{https://twitter.com/fukazawas/status/1408386637994729473} 
  \url{https://twitter.com/harrier0516osk/status/1408382359888482309} 
  \textgreater{}
  裁判員制度について漫画を作ったので、近々noteにUPします。noteのアカウント取ったばかりなので使い方がよくわかってないのと、いま非常に忙しいので、7月に入ったころを目処にUPしたいと思います。
  \url{https://t.co/L3YOaf8sEu} 
\item
  RT fukazawas(深澤諭史)|fukazawas(深澤諭史) 日時:2021-06-25
  20:29/2019-11-23 11:28 URL:
  \url{https://twitter.com/fukazawas/status/1408386985836695553} 
  \url{https://twitter.com/fukazawas/status/1198065660171096064} 
  \textgreater{} 裁判員裁判三傑\\
  \textgreater{}
  ①障害者は受け入れ先がないので求刑より長く刑務所入れるのが健全な社会常識\\
  \textgreater{} ②無罪っぽいんで刑を軽くします。\\
  \textgreater{}
  ③否認する被告人に真実話さないと償えないと問う裁判所←NEW!!
\end{itemize}

 昨夜はいつの間にか横になって寝ていて、目が覚めたのが夜中の2時50分ぐらいだったのですが、妙に重苦しいものを感じていました。寝る前に見ていた深澤諭史弁護士のツイートが原因だと思ったので、調べることにしました。

 時刻は15時27分です。深澤諭史弁護士のタイムラインで思いがけない発見があったのですが、深澤諭史弁護士の講演というツイートに、返信があって、気になって調べたところ「新潟県司法書士会 会長(現職)」とありました。

 Twitterアカウントは鈴木利益という名前で、フォロワーが4、フォロー中が2となっています。ずいぶん少ないですが、2011年10月からTwitterを利用しています、とあります。まず名前が変わっていると思ったのですが、としえき、と読むようです。

 プロフィールにも新潟加茂市とあるのですが、名前で調べた情報も新潟県加茂市で司法書士事務所をされている人物と同姓同名となっています。その前に2020年4月19日のツイートを見ていて、新潟県加茂市の中学校の同窓会中止という内容でした。

 「ツイート」では最上位の最新のツイートですが、「ツイートと返信」のタイムラインでは、1時間前と表示されている深澤諭史弁護士への返信ツイートが最上位で最新となっています。その次が4月19日の同窓会中止のツイートです。

 新潟県加茂市が新潟県のどの辺りなのかわからなかったのですが、Googleマップをみると三条市と五泉市の間にありました。

 新潟県加茂市という地名は余り記憶になかったのですが、三条市の国道8号線から五泉市内を通過した国道49号線というのは金沢市場輸送でよく通行した道路で、Googleマップでみると国道290号線のようです。

 国道49号線にぶつかったところの交差点の信号機に「安田」とあったような記憶もあるのですが、福島県郡山市や福島市に向かうときは決まって通行する道路だったと思います。その途中には会津若松市や猪苗代湖がありました。

 新潟県の長岡市から新潟市内は国道8号線と北陸自動車道がほぼ並行して通っていたという記憶ですが、新潟市から来ると長岡インターの近くで、北陸道と関越道が分岐していました。

 なぜなのか不思議に思っていたのですが、後述するうの字のヒャッハーのツイートを初めに読んだとき、夕方の明るい時間だったとも記憶にあるのですが、なぜか新潟市内ではなく、その北陸道と関越道の分岐点あたりの情景が頭に浮かんで離れずにいました。

※ @kk\_hironoのアカウントがブロックされ,リツイートに失敗したツイート

\begin{itemize}
\tightlist
\item
  TW fukazawas(深澤諭史) 日時:2021/06/26 14:29:26 URL:
  \url{https://twitter.com/fukazawas/status/1408658643684265989} 
  \textgreater{} (・∀・)講演の休憩なう!\\
  \textgreater{} (^ω^)だお!
\end{itemize}

〉〉〉 kk\_hironoのリツイート 〉〉〉

\begin{itemize}
\tightlist
\item
  RT
  kk\_hirono(刑事告発・非常上告_金沢地方検察庁御中)|ToshiekiSuzuki(鈴木利益)
  日時:2021-06-26 15:53/2021/06/26 14:32 URL:
  \url{https://twitter.com/kk\_hirono/status/1408679834406948869} 
  \url{https://twitter.com/ToshiekiSuzuki/status/1408659394519203840} 
  \textgreater{} @fukazawas ご講演ありがとうございます!
\end{itemize}

 深澤諭史弁護士が新潟県の司法書士会で講演をしている可能性がありますが、これが事実だと地方出張の依頼まで受けて深澤諭史弁護士が講演をしているという社会活動の一面を垣間見たことになります。

 深澤諭史弁護士の新潟女児殺害事件に関連した一連のツイートは、午前中にまとめ記事を作成してあります。今回もスクリーンショットとデジカメの写真が混じっています。

\begin{itemize}
\item
  2021年06月26日11時47分の登録:
  2021年06月26日の記録:写真資料:新潟の女児殺害事件の黙秘権問題や、うの字とのヒャッハーにつながった深澤諭史弁護士のツイート
  \url{https://kk2020-09.blogspot.com/2021/06/20210626.html} 
\item
  〈〈〈 2021/06/26 16:03:13 Linux Emacs: 〈〈〈
\end{itemize}

\hypertarget{ux88c1ux5224ux54e1ux5236ux5ea6ux3092ux76aeux8089ux308bux6df1ux6fa4ux8aedux53f2ux5f01ux8b77ux58ebux81eaux8eabux306bux3088ux308bux904eux53bbux306eux30c4ux30a4ux30fcux30c8ux306eux30eaux30c4ux30a4ux30fcux30c8ux304bux3089ux518dux767aux898bux306bux81f3ux3063ux305fux65b0ux6f5fux5973ux5150ux6bbaux5bb3ux4e8bux4ef6ux306bux304aux3051ux308bux9ed9ux79d8ux6a29ux3068ux3046ux306eux5b57ux3068ux306eux30b3ux30e9ux30dcux306bux3088ux308bux30d2ux30e3ux30c3ux30cfux30fcux306eux30c4ux30a4ux30fcux30c82}{%
\paragraph{裁判員制度を皮肉る深澤諭史弁護士自身による過去のツイートのリツイートから再発見に至った、新潟女児殺害事件における黙秘権とうの字とのコラボによるヒャッハーのツイート(2)}\label{ux88c1ux5224ux54e1ux5236ux5ea6ux3092ux76aeux8089ux308bux6df1ux6fa4ux8aedux53f2ux5f01ux8b77ux58ebux81eaux8eabux306bux3088ux308bux904eux53bbux306eux30c4ux30a4ux30fcux30c8ux306eux30eaux30c4ux30a4ux30fcux30c8ux304bux3089ux518dux767aux898bux306bux81f3ux3063ux305fux65b0ux6f5fux5973ux5150ux6bbaux5bb3ux4e8bux4ef6ux306bux304aux3051ux308bux9ed9ux79d8ux6a29ux3068ux3046ux306eux5b57ux3068ux306eux30b3ux30e9ux30dcux306bux3088ux308bux30d2ux30e3ux30c3ux30cfux30fcux306eux30c4ux30a4ux30fcux30c82}}

\begin{itemize}
\tightlist
\item
  〉〉〉 Linux Emacs: 2021/06/26 16:34:53 〉〉〉
\end{itemize}

:CATEGORIES: @kanazawabengosi \#金沢弁護士会 @JFBAsns
日本弁護士連合会(日弁連) \#法務省 @MOJ\_HOUMU \#深澤諭史弁護士
\#うの字 \#黙秘権 \#死刑判決

 6月24日になると思いますが、気になる死刑判決のネットニュースがありました。同じ日には和歌山カレー事件の再審請求取り下げのニュースもあったと思うのですが、この2つのニュースをテレビで確認するため録画機器の状態を点検していました。

 時刻は16時37分です。録画はうまくいっているはずなので、これから再生をして内容を確認しておきたいと思います。

 時刻は17時00分です。24日のテレビはNEWS7で死刑判決のニュースがありましたが、主要項目ではなくその他のニュースとなっていました。NEWS9では放送が確認できませんでした。

 同じ24日のかがのとイブニングで能登町が項目にあったのですが、やなぎだ植物公園の菖蒲園と九十九湾でした。柳田のブルーベリーの栽培は38年前からということでずいぶん前ですが、知ったのは15年か20年ほど前であったように思います。

 さすがに死刑判決が全国ニュースにならないということはなかったようですが、ネットでも余り注目されるニュースではなさそうでした。弁護士の反応を記録したのですが、その数も少なそうでした。

\begin{itemize}
\tightlist
\item
  2021年06月26日04時45分の登録:
  REGEXP:''ヒャッハ''/深澤諭史(@fukazawas)の検索(2013-10-10〜2020-03-10/2021年06月26日04時45分の記録24件)
  \url{https://kk2020-09.blogspot.com/2021/06/regexpfukazawas2013-10-102020-03.html} 
\item
  2021年06月26日11時47分の登録:
  2021年06月26日の記録:写真資料:新潟の女児殺害事件の黙秘権問題や、うの字とのヒャッハーにつながった深澤諭史弁護士のツイート
  \url{https://kk2020-09.blogspot.com/2021/06/20210626.html} 
\item
  2021年06月26日11時58分の登録:
  REGEXP:''死刑判決''/データベース登録済みツイートの検索:2021-06-24〜2021-06-26/2021年06月26日11時58分の記録:ユーザ・投稿:21/41件
  \url{https://kk2020-09.blogspot.com/2021/06/regexp2021-06-242021-06\_26.html} 
\end{itemize}

\begin{lstlisting}

SELECT * FROM tw_user_tweet WHERE tw_date BETWEEN '2021-06-23 11:57' AND '2021-06-26 11:57' AND tweet REGEXP "死刑判決" ORDER BY tw_date ASC
\end{lstlisting}

\begin{itemize}
\item
  (01/41) TW @kyodo\_official(共同通信公式) 日時: 2021-06-24
  15:55:59 +0900 URL:
  \url{https://twitter.com/kyodo\_official/status/1407955647216320512\textgreater} {}
  福島2人ひき逃げ殺人に死刑判決\textgreater{} \url{https://t.co/4yEFi8WCqU} 
\item
  (04/41) TW @sakamotomasayuk(坂本正幸) 日時: 2021-06-24 15:59:07
  +0900 URL:
  \url{https://twitter.com/sakamotomasayuk/status/1407956437301547009\textgreater} {}
  @1961kumachin 死刑判決です
\end{itemize}

 坂本正幸弁護士の中村元弥弁護士への返信ツイートですが、これで最初に死刑判決のニュースを知ったように思います。Twitterのトレンドでも短時間でしたが、「死刑判決」というトレンドが出ていたかもしれません。

\begin{itemize}
\tightlist
\item
  (10/41) RT @uwaaaa(サイ太)|uwaaaa(サイ太) 日時:2021-06-24
  17:02:39 +0900/2019-01-10 10:01:00 +0900 URL:
  \url{https://twitter.com/uwaaaa/status/1407972423765434372} 
  \url{https://twitter.com/uwaaaa/status/1083166888534798336\textgreater} {}
  死刑判決を下した裁判官の苦悩は語られることがあるけど,冤罪で有罪判決を下した裁判官や無茶な起訴をして無罪を食らった検察官の苦悩が語られることはないよね。どうやって精神保ってるんだろうね。
\end{itemize}

 見覚えのある刑裁サイ太のツイートです。死刑判決のニュースと直接関連はなさそうですが、触発されたという可能性はあるのかもしれません。

 死刑判決が福島地裁郡山支部となっていたことを思い出したのですが、支部でも死刑判決が出るのかと思いました。事件の見出しに福島県三春町となっていたことも思い出したのですが、福島県のどの辺りかわからずにいます。いわき市方面という気はしています。

\begin{itemize}
\tightlist
\item
  三春町 - Google マップ \url{https://t.co/I7t7N0ExRy} 
\end{itemize}

 郡山市と隣接していますが、いわき市に向かう国道49号線とは離れていました。ほとんど見覚えがないですが福島県田村郡とあります。昨夜の大きな木の写真がありますが、これはテレビで中継を見たことがある桜の木かもしれません。

 死刑判決が出る前にもネットでニュースをみていたのですが、刑務所を出所して2日後、刑務所に戻ることを目的に起こした殺人事件で、盗んだトラックでボランティアで道路清掃中だった男女をはねて殺害したという事件でした。

 個人的には弁護士の無関心ぶりも気になっていたのですが、2日後に深澤諭史弁護士の過去のツイートが出てきました。

 Googleマップで確認したところ猪苗代湖の湖畔が猪苗代町となっていて、郡山市との間は記憶以上に集落が見当たらないのですが、磐梯熱海という温泉地のようなところがあって、その磐梯熱海駅の住所が郡山市熱海町となっていました。

 熱海といえば静岡県熱海市が有名ですが、磐梯熱海のことはテレビでも見かけた覚えがなくすっかり忘れていました。国道4号線に出る5分ぐらい手前に、右方向にトルコ風呂の怪しげな看板がやたらと目立っていたことは記憶にあります。ソープランドの前だったと思います。

 磐梯熱海になるのだと思いますが、ソープランドの看板を見たという記憶はなく、夜の暗い時間に通ることが昭和63年頃にはなくなっていたのかもしれません。あるいは廃業していた可能性もあるのかもしれないですが、暗がりにとても目立つ看板でした。

 昭和63年頃によく行ったのが中越運送の郡山支店ですが、トルコ風呂の看板を夜に見た後、記憶にあるのは郡山市内ではなく、東北自動車道の本宮インターでした。

\begin{itemize}
\tightlist
\item
  RT fukazawas(深澤諭史)|fukazawas(深澤諭史) 日時:2021-06-25
  20:29/2019-11-23 11:28 URL:
  \url{https://twitter.com/fukazawas/status/1408386985836695553} 
  \url{https://twitter.com/fukazawas/status/1198065660171096064} 
  \textgreater{} 裁判員裁判三傑\\
  \textgreater{}
  ①障害者は受け入れ先がないので求刑より長く刑務所入れるのが健全な社会常識\\
  \textgreater{} ②無罪っぽいんで刑を軽くします。\\
  \textgreater{}
  ③否認する被告人に真実話さないと償えないと問う裁判所←NEW!!
\end{itemize}

 深澤諭史弁護士がよくやる過去の自分のツイートのリツイートですが、元のツイートの日付が2019年11月23日となっていました。また深澤諭史弁護士のタイムラインでは更新があったようです。司法書士会がなんとかというツイートがありました。

〉〉〉 kk\_hironoのリツイート 〉〉〉

\begin{itemize}
\tightlist
\item
  RT
  kk\_hirono(刑事告発・非常上告_金沢地方検察庁御中)|perplex9(ぱ~ぷ。'13)
  日時:2021-06-26 17:46/2021/06/26 15:13 URL:
  \url{https://twitter.com/kk\_hirono/status/1408708128602935297} 
  \url{https://twitter.com/perplex9/status/1408669711009652736} 
  \textgreater{} @fukazawas 当会の会長です( ・∀・)
\end{itemize}

〉〉〉 kk\_hironoのリツイート 〉〉〉

\begin{itemize}
\tightlist
\item
  RT
  kk\_hirono(刑事告発・非常上告_金沢地方検察庁御中)|perplex9(ぱ~ぷ。'13)
  日時:2021-06-26 17:46/2021/06/26 14:50 URL:
  \url{https://twitter.com/kk\_hirono/status/1408708258823434240} 
  \url{https://twitter.com/perplex9/status/1408663980839563266} 
  \textgreater{} それにしても深澤先生、twitter好きだなあw
\end{itemize}

 1つは深澤諭史弁護士への返信ツイートですが、どちらも深澤諭史弁護士がタイムラインでリツイートをしています。

\begin{itemize}
\tightlist
\item
  RT
  fukazawas(深澤諭史)|mitsuba\_group(【公式】司法書士法人みつ葉グループ)
  日時:2021-06-26 14:36/2021-06-15 17:32 URL:
  \url{https://twitter.com/fukazawas/status/1408660450456133639} 
  \url{https://twitter.com/mitsuba\_group/status/1404718379412332544} 
  \textgreater{} 【みつ葉グループ】の司法書士採用説明会開催中!\\
  \textgreater{}
  先輩司法書士を交えた質疑応答やオフィス内の見学など、実際の職場の雰囲気や業務内容をより詳しく知ることが出来る内容になっています!\\
  \textgreater{} まずはお気軽にご応募ください。
\end{itemize}

 さきほども深澤諭史弁護士のタイムラインで見かけていたようなツイートですが、深澤諭史弁護士のリツイートが14時36分となっています。関連付けて考えることはなかったのですが、この司法書士採用説明会の講師を深澤諭史弁護士がやったのかと思えてきました。

\begin{quote}
《引用の始まり》
\end{quote}

\begin{quote}
【公式】司法書士法人みつ葉グループ@mitsuba\_group司法書士法人みつ葉グループ、司法書士採用の公式アカウントです。説明会やインターン情報、事務所の雰囲気などをお届けします。※質問にはお返事できません。お電話かHPのお問い合わせフォームからお願い致します。東京都港区赤坂7-2-21 草月会館7階recruit.mitsubagroup.co.jp2021年1月からTwitterを利用しています24
フォロー中53 フォロワー
\end{quote}

\begin{quote}
《引用の終わり》
\end{quote}

\begin{itemize}
\tightlist
\item
  【公式】司法書士法人みつ葉グループさん (@mitsuba\_group) / Twitter
  \url{https://twitter.com/mitsuba\_group} 
\end{itemize}

 フォロワーの数が53と少ないですが、住所が東京都港区赤坂となっています。港区自体の地価が高いと聞いたことがありますが、日本の中心地ともいえそうです。

〉〉〉 kk\_hironoのリツイート 〉〉〉

\begin{itemize}
\tightlist
\item
  RT
  kk\_hirono(刑事告発・非常上告_金沢地方検察庁御中)|mitsuba\_group(【公式】司法書士法人みつ葉グループ)
  日時:2021-06-26 18:01/2021/01/13 10:21 URL:
  \url{https://twitter.com/kk\_hirono/status/1408711992311832576} 
  \url{https://twitter.com/mitsuba\_group/status/1349164733736374275} 
  \textgreater{}
  2020年は世界中が変化を求められた激動の一年となりました。
  私たちもコロナ禍を契機に、テレワークの導入、オンライン面談など業務改革に取り組みました。
  今年は、社内体制の整備、人材育成やサービス向上に投資することで、品質向上に努めていく所存です。
  \url{https://t.co/gT9OOUjU6A} 
\end{itemize}

 私のリツイートが最初のリツイートとなったようですが、写真に書道の紙に「至誠通天」とあるのが気になりました。「至誠一貫」というのは時代劇の裁判所(たぶん奉行所)の場面でよく見かけたという記憶があります。

 似たような意味の言葉があったとしばらく考えていたのですが、思い出した中でいくらか近いと思ったのが「天佑神助」になりますが、長く見かけることのない言葉で、余り使われることはなさそうです。

 新潟県のことで思い出していたのが弥彦神社ですが、近年テレビの旅番組で知った神社で、当初は新発田市の辺りを思い浮かべていました。そして1年ほど前になるのか、昭和史の重大ニュースに、初詣で大きな事故があったことを知ったのです。

\begin{quote}
《引用の始まり》
\end{quote}

\begin{quote}
1月1日午前0時過ぎ、新潟県弥彦村の弥彦神社で初もうで客が餅まきに殺到、玉垣が崩れて将棋倒しになり、124人が死亡するという惨事が起きた。珍しく雪のない元旦になり、例年を大きく上回る3万人が参拝に訪れた。この事件では、事故防止のための対策が不十分であったことが指摘され、警備に当たった警察や神社側の責任が厳しく追及された。
\end{quote}

\begin{quote}
《引用の終わり》
\end{quote}

\begin{itemize}
\tightlist
\item
  新潟 弥彦神社初詣で大惨事 \textbar{} NHK放送史(動画・記事)
  \url{https://www2.nhk.or.jp/archives/tv60bin/detail/index.cgi?das\_id=D0009030019\_00000} 
\end{itemize}

 Googleの検索結果には、「2021/01/01 ---
珍しく雪のない元旦になり、例年を大きく上回る3万人が参拝に訪れた。この事件では、事故防止のための対策が不十分であったことが指摘され、警備に当たった警察や神社側の
\ldots」とあります。

 最初に知ったのが、この同じNHK放送史というページだったと思うのですが、2021年1月1日というのは今年のことです。

\begin{lstlisting}
base ❯ twilog-serch  弥彦神社
\end{lstlisting}

\begin{itemize}
\tightlist
\item
  ./hirono\_hideki2021-06-26\_163018.csv:2020-12-17 17:06:09
  ``新潟 弥彦神社初詣で大惨事 \textbar{} NHK放送史(動画・記事)
  \url{https://www2.nhk.or.jp/archives/tv60bin/detail/index.cgi?das\_id=D0009030019\_00000} 
  1月1日午前0時過ぎ、新潟県弥彦村の弥彦神社で初もうで客が餅まきに殺到、玉垣が崩れて将棋倒しになり、124人が死亡するという惨事が起きた。''
  \url{https://twitter.com/hirono\_hideki/status/1339481997836574726} 
\item
  ./kk\_hirono2021-06-26\_180003.csv:2019-08-15 15:29:24
  ``早送りでみましたが、ヒルナンデスに「いしかわ動物園」は出ていませんでした。新潟のロケで、半年ほど前にテレビ番組で初めて知った弥彦神社も出ていましたが、その後はスタジオで蚊の対策のようなコーナー、そのあとは八景島シーパラダイスでした。''
  \url{https://twitter.com/kk\_hirono/status/1161887597561405440} 
\end{itemize}

 検索結果は2件だけでしたが、弥彦神社については他にも調べているはずです。場所がまず意外だったのですが、国道160号線と国道8号線の三条市の間にあったと思います。違いました、国道116号線と日本海との間でした。

 北陸自動車道の三条インターには、三条と一緒に「燕」とあったことをよく憶えているのですが、燕という地名は改めて燕市になるようです。市街地の中心部を国道116号線が通過しているようにも見えます。

 この国道116号線も金沢市場輸送で途中からよく使う道路となったのですが、日曜日の夕方か夜に出発する運行では仙台方面に向かうのに決まって、この国道116号線を使い、柏崎市から国道116号線に入っていました。福島方面の場合は先程の五泉市方面になります。

 柏崎市からそのままの国道8号線で長岡市、見附市、三条市という経路もあったのですが、かなり遠回りに感じる道で、五泉市から福島方面に向かうときは、途中、北陸自動車道に乗って三条インターで降りていたような気もします。

 柿崎インターから高速道路の使用が認められていたのかもしれません。昭和59年の鮮魚の新潟定期でも金沢からずっと国道8号線を走り、柿崎インターから認められていたのですが、次の卸先が長岡市で、その次が小千谷だったので、長岡から小千谷は下道でした。

 長岡市の市場の周辺というのもそうでしたが、広い土地に建物も少なく、夜は交通量も少なくて、街灯の明かりも少なかったという記憶です。新潟県内にはそういう場所が多かったのですが、土地がとても広く感じられました。

 新潟の鮮魚の定期便は、夜中の2時からの荷降ろしと決まっていたかもしれません。帰りは全線下道でしたが、長岡市の手前、三条市の辺りでも夜が明けて明るくなっていたような記憶があります。店というのはゲームセンターぐらいで柏崎市の国道沿いにもあったと思います。

 新潟定期からの帰りは、そのまま富山県内で大阪方面行きの荷物を積むことが多かったのですが、最初が新潟県糸魚川市の油揚げの工場となっていました。国道8号線沿い、富山に向かって右側で、大きな団地のような建物でした。

\begin{itemize}
\tightlist
\item
  〈〈〈 2021/06/26 18:48:16 Linux Emacs: 〈〈〈
\end{itemize}

\hypertarget{ux88c1ux5224ux54e1ux5236ux5ea6ux3092ux76aeux8089ux308bux6df1ux6fa4ux8aedux53f2ux5f01ux8b77ux58ebux81eaux8eabux306bux3088ux308bux904eux53bbux306eux30c4ux30a4ux30fcux30c8ux306eux30eaux30c4ux30a4ux30fcux30c8ux304bux3089ux518dux767aux898bux306bux81f3ux3063ux305fux65b0ux6f5fux5973ux5150ux6bbaux5bb3ux4e8bux4ef6ux306bux304aux3051ux308bux9ed9ux79d8ux6a29ux3068ux3046ux306eux5b57ux3068ux306eux30b3ux30e9ux30dcux306bux3088ux308bux30d2ux30e3ux30c3ux30cfux30fcux306eux30c4ux30a4ux30fcux30c83}{%
\paragraph{裁判員制度を皮肉る深澤諭史弁護士自身による過去のツイートのリツイートから再発見に至った、新潟女児殺害事件における黙秘権とうの字とのコラボによるヒャッハーのツイート(3)}\label{ux88c1ux5224ux54e1ux5236ux5ea6ux3092ux76aeux8089ux308bux6df1ux6fa4ux8aedux53f2ux5f01ux8b77ux58ebux81eaux8eabux306bux3088ux308bux904eux53bbux306eux30c4ux30a4ux30fcux30c8ux306eux30eaux30c4ux30a4ux30fcux30c8ux304bux3089ux518dux767aux898bux306bux81f3ux3063ux305fux65b0ux6f5fux5973ux5150ux6bbaux5bb3ux4e8bux4ef6ux306bux304aux3051ux308bux9ed9ux79d8ux6a29ux3068ux3046ux306eux5b57ux3068ux306eux30b3ux30e9ux30dcux306bux3088ux308bux30d2ux30e3ux30c3ux30cfux30fcux306eux30c4ux30a4ux30fcux30c83}}

\begin{itemize}
\tightlist
\item
  〉〉〉 Linux Emacs: 2021/06/27 10:46:09 〉〉〉
\end{itemize}

:CATEGORIES: @kanazawabengosi \#金沢弁護士会 @JFBAsns
日本弁護士連合会(日弁連) \#法務省 @MOJ\_HOUMU \#深澤諭史弁護士
\#うの字

 昨夜、大きな発見が2つありました。共通するのは出版物ですが、2つ目は裁判員制度とも関連していて、深澤諭史弁護士や向原栄大朗弁護士、いわぽんこと岩田圭只弁護士の考えに近いものを感じました。そのきっかけというのも岩田圭只弁護士のツイートです。

 1つ目は和歌山カレー事件の新情報です。話題になっていないのですが、構図が浮かび上がり見えてきたという段取りを感じました。昨日書き忘れていましたが、6月24日のテレビのNHKで、再審請求の取り下げのニュースはなかったです。早送りにしていたので見落としの可能性は完全に否定できません。

 出版に強いこだわりを持ち思惑を実社会の軌道に乗せたと感じるのも深澤諭史弁護士になりますが、司法書士会から講演の依頼が入るのもその実績かもしれません。深澤諭史弁護士は自身でもそのような宣伝と受け取れるツイートをしていました。

 なお、私が和歌山カレー事件について最も関心を寄せるのは一審での黙秘のことです。本人は刑務所に入った夫を少しでも早く社会に出し子供の面倒をみてもらうのが目的だった、と述べているようです。その本人が著書を出していたことも最近になって知りました。

 3日ほど前、和歌山カレー事件を長年取材してきた片岡建氏のnoteの記事を読んだところ、動画に否定的なコメントを連発する人のコメントが掲載されていて、一審で黙秘をしたので弁護士との関係がおかしくなった、というものがありました。弁護士の全体的反応としては黙秘を是としたことは和歌山県弁護士会の声明で見ています。

\begin{itemize}
\tightlist
\item
  ./hirono\_hideki2021-06-27\_110002.csv:2021-06-25 13:58:31
  ``和歌山カレー事件に並々ならぬ関心を持ち、冤罪説を懸命に否定する田中舞子さんへ|片岡健|note
  \url{https://t.co/wXdSxpQ5yg''} 
  \url{https://twitter.com/hirono\_hideki/status/1408288475330224128} 
\item
  ./hirono\_hideki2021-06-27\_110002.csv:2021-06-25 14:20:05
  ``「和歌山カレー事件の犯人は、林眞須美の長女だったのでは・・・」という憶測が流布している件について|片岡健|note
  \url{https://t.co/nH5gpricX5''} 
  \url{https://twitter.com/hirono\_hideki/status/1408293901954273288} 
\end{itemize}

 次に昨夜見た出版物ですが、初めてみる出版社だと感じました。万代に続く漢字がもう1文字あったと記憶しますが、個人的に万代で思い出すのは、新潟ブルースという曲の歌詞にも出てくる万代橋です。見かけることの多い橋の名前で、有名な橋だと思います。歴史もありそうです。

\begin{itemize}
\item
  \begin{enumerate}
  \def\labelenumi{(\arabic{enumi})}
  \setcounter{enumi}{1}
  \tightlist
  \item
    和歌山 事件 万代 - Twitter検索 / Twitter \url{https://t.co/fSVntQW91l} 
    「和歌山 事件 万代」の検索結果はありません
  \end{enumerate}
\end{itemize}

 「生田暉雄 出版」、「生田暉雄 本」とTwitter検索しても見つからずにいます。

〉〉〉 kk\_hironoのリツイート 〉〉〉

\begin{itemize}
\tightlist
\item
  RT
  kk\_hirono(刑事告発・非常上告_金沢地方検察庁御中)|hirono\_hideki(奉納\さらば弁護士鉄道・泥棒神社の物語)
  日時:2021-06-27 11:18/2021/06/27 01:07 URL:
  \url{https://twitter.com/kk\_hirono/status/1408972966440837120} 
  \url{https://twitter.com/hirono\_hideki/status/1408819247174668294} 
  \textgreater{} 2021年06月27日01時07分の実行記録:
  twitterAPI-search-lawList-mydql-add.rb ``再審申立書''
  ツイート数:3/2492 リツイート数:0/2492 トータル:6
  ``再審申立書''の該当: hirono\_hideki 3/0件 kk\_hirono 0/0件
  s\_hirono 0/0件
\end{itemize}

〉〉〉 kk\_hironoのリツイート 〉〉〉

\begin{itemize}
\tightlist
\item
  RT
  kk\_hirono(刑事告発・非常上告_金沢地方検察庁御中)|hirono\_hideki(奉納\さらば弁護士鉄道・泥棒神社の物語)
  日時:2021-06-27 11:18/2021/06/27 01:02 URL:
  \url{https://twitter.com/kk\_hirono/status/1408972980583952384} 
  \url{https://twitter.com/hirono\_hideki/status/1408817928380706817} 
  \textgreater{}
  和歌山カレー「再審申立書」\textasciitilde 冤罪の大カラクリを根底から暴露
  -- 万代宝書房 \url{https://t.co/dug9OY5dif} 
  本書は、申立書とほぼ同一内容のものです。
  本件の発生は、平成10年7月25日午後6時ごろ、夏祭りで出されたカレーを食べた67名が身体に異常を起こしたことに発する。
\end{itemize}

〉〉〉 kk\_hironoのリツイート 〉〉〉

\begin{itemize}
\tightlist
\item
  RT
  kk\_hirono(刑事告発・非常上告_金沢地方検察庁御中)|hirono\_hideki(奉納\さらば弁護士鉄道・泥棒神社の物語)
  日時:2021-06-27 11:18/2021/06/27 00:49 URL:
  \url{https://twitter.com/kk\_hirono/status/1408972998250352642} 
  \url{https://twitter.com/hirono\_hideki/status/1408814721935892483} 
  \textgreater{}
  新刊本! 〜和歌山カレー事件「再審申立書」冤罪の大カラクリを根底から暴露〜
  - YouTube \url{https://t.co/iop2rNLULU}  149 回視聴 2021/06/26
\end{itemize}

〉〉〉 kk\_hironoのリツイート 〉〉〉

\begin{itemize}
\tightlist
\item
  RT
  kk\_hirono(刑事告発・非常上告_金沢地方検察庁御中)|MichikoKameishi(弁護士
  亀石倫子) 日時:2021-06-27 11:19/2020/05/25 20:15 URL:
  \url{https://twitter.com/kk\_hirono/status/1408973238785380353} 
  \url{https://twitter.com/MichikoKameishi/status/1264877837032386561} 
  \textgreater{}
  大崎事件第4次再審請求クラウドファンディングが900万円を突破・・・!ネクストゴール1千万まであと少しです。本当にありがとうございます🙇🏻‍♀️
  近日、再審申立書全文を公開予定。大崎事件を徹底解説するオンライン配信イベントもやります☺️
  引き続きよろしくお願いいたします! \url{https://t.co/aEXOBjm2xS} 
  \url{https://t.co/ePVqIJlvSF} 
\end{itemize}

 タイムラインがすぐに終点になったので、数えながら遡ると21件のツイートでした。私のツイートがそのうち5件となっています。

 「和歌山カレー「再審申立書」\textasciitilde 冤罪の大カラクリを根底から暴露
--
万代宝書房」が本のタイトルのようです。私は過去に何度もやった再審請求で、「再審請求書」を標題にしていたと記憶があるのですが、申立というのは控訴、上告で経験していましたが、拘置所では刑務官にやってもらう手続きでした。

 なお、最近は滅多に見かけないように思いますが、刑事裁判の控訴審では「控訴趣意書」、上告審では「上告趣意書」という書面に理由を記述するのが決まりとなっていました。控訴趣意書では5部の提出が必要で、カーボン紙というのを使っていましたが、時間と手間のかかる作業でした。

 平成4年から6年当時の金沢刑務所の拘置所では、全罫紙の薄と厚の購入があって、枚数に違いがありましたが、薄だと一冊1,150円ぐらいしたような記憶です。薄すぎるぐらい薄いもので、ボールペンの筆圧での破れに気を使いました。この薄でないとカーボン紙で他に4部は作成が無理そうでした。

 はっきりしたことは思い出せないですが、その全罫紙の薄で1冊は100枚だったと思います。ただ、当時はB4用紙の袋とじとなっていたので1枚が表裏の2ページ分になります。この袋とじや割り印の面倒さというのも嫌というほど経験しました。

\begin{itemize}
\item
  \begin{enumerate}
  \def\labelenumi{(\arabic{enumi})}
  \setcounter{enumi}{3}
  \tightlist
  \item
    新刊本! 〜和歌山カレー事件「再審申立書」冤罪の大カラクリを根底から暴露〜
    - YouTube \url{https://t.co/epJNRuV5bh}  349 回視聴2021/06/26 ¥\n  ¥\n
    14 ¥\n  ¥\n 1 ¥\n  ¥\n 共有 ¥\n  ¥\n 保存 ¥\n 
  \end{enumerate}
\end{itemize}

 YouTubeの再生回数が349回に増えていました。よくみると昨日の日付の投稿なので、これから再生回数が増えるのかもしれません。

 そういえば、と思ったのですが、忘れていた本の値段を確認すると、1,430円となっていました。他の本と比較してずいぶん安い設定だと思ったのですが、ページ数はまだ確認していません。YouTubeの動画には、裁判所に提出したPDFファイルのフォントを小さくして文字数を増やしたとかありました。

\begin{itemize}
\tightlist
\item
  和歌山カレー「再審申立書」\textasciitilde 冤罪の大カラクリを根底から暴露
  -- 万代宝書房 \url{https://t.co/WdAREQtw6h}  ¥\n 単行本 : 230ページ
\end{itemize}

 230ページとありました。裁判所に提出したのがA4用紙約300枚という話であったように思います。

 時刻は11時47です。Amazonで注文を確定しました。1,430円という値段が大きいですが、こういう買い物は初めてになります。全体としてのまとめかたなど、告発状の作成にも参考になるところはありそうと考えました。

 著者の生田暉雄弁護士といえば、香川県ですが、お隣の徳島県に徳島ラジオ商殺し事件があるのを思い出しました。いくつか著書が出ていることをネットで知ったのですが、再審請求の請求人が死亡した後に出た再審無罪判決だったように思います。以前はネットではない書籍でよく見かけた著名な事件です。

\begin{quote}
《引用の始まり》
\end{quote}

\begin{quote}
徳島ラジオ商殺し事件(とくしまラジオしょうごろしじけん)とは、1953年に徳島県徳島市で発生した強盗殺人事件。犯人とされた冨士茂子に対し、刑の確定および死後に再審によって無罪が言い渡された冤罪事件である。日本弁護士連合会が支援していた。日本初の死後再審が行われ、死後に無罪判例によって名誉回復がなされた。
\end{quote}

\begin{quote}
《引用の終わり》
\end{quote}

\begin{itemize}
\tightlist
\item
  徳島ラジオ商殺し事件 - Wikipedia
  \url{https://ja.wikipedia.org/wiki/\%E5\%BE\%B3\%E5\%B3\%B6\%E3\%83\%A9\%E3\%82\%B8\%E3\%82\%AA\%E5\%95\%86\%E6\%AE\%BA\%E3\%81\%97\%E4\%BA\%8B\%E4\%BB\%B6} 
\end{itemize}

 内容が余り記憶に残らない事件なのですが、犯人とされた再審請求の請求人が女性であることを確認しました。名前だけではなく、ネットで女性らしい人が街頭で活動する写真を見たような記憶があります。

\begin{quote}
《引用の始まり》
\end{quote}

\begin{quote}
林 伸豪(はやし のぶひで、1941年 -
)は日本の弁護士。日本弁護士連合会副会長や四国弁護士会連合会理事長などを歴任した[1]。信州大学文理学部社会科学科中退(1966年)[1]。同年司法試験合格。司法修習20期(同期に明治大学総長納谷廣美や検事総長松尾邦弘)を経て1968年弁護士登録。徳島ラジオ商殺人事件やトンネルじん肺訴訟などを担当した[1]。九条の会徳島呼びかけ人[2]、日本国民救援会徳島県本部会長[3]、日本共産党徳島後援会会長[4]。徳島県徳島市出身。
\end{quote}

\begin{quote}
《引用の終わり》
\end{quote}

\begin{itemize}
\tightlist
\item
  林伸豪 - Wikipedia
  \url{https://ja.wikipedia.org/wiki/\%E6\%9E\%97\%E4\%BC\%B8\%E8\%B1\%AA} 
\end{itemize}

 徳島ラジオ商殺し事件のWikipediaは前にも一通り読んでいるはずですが、今回は初めて知るような弁護士の名前に気が付きました。林伸豪という名前の弁護士です。昭和16年生まれで没年はありません。この後取り上げる昨夜見た2件目の本の弁護士も昭和17年生まれとなっていたと思います。

 1つ深澤諭史弁護士のタイムラインで更新がありました。過去の自分のツイートを引用で紹介したものです。

\begin{itemize}
\item
  TW fukazawas(深澤諭史) 日時: 2021/06/27 11:24:47 URL:
  \url{https://twitter.com/fukazawas/status/1408974559374548993} 
  \textgreater{} (^ω^)昨日の研修で取り上げましたお。
  \url{https://t.co/hdoXjAc1iU} 
\item
  TW fukazawas(深澤諭史) 日時: 2020/11/27 08:42:17 URL:
  \url{https://twitter.com/fukazawas/status/1332107438816784384} 
  \textgreater{}
  弁護士がTwitterやるとき、慣れるまでは匿名で、その後、実名アカウントに切り替える、のはお勧めだったりする。\\
  \textgreater{}
  ただ、匿名でやっていても、うまくいけばいくほど実名と結びつきが出るので、匿名時代の暴言について、実名化後もその評判を引き継ぐリスクはある。これも著書で解説予定。
\end{itemize}

 匿名から実名に切り替えたTwitterの弁護士アカウントというのは思い当たるものがないのですが、ローカスという匿名からリンクのブログなどで積極的に実名を出すようになったのが三浦義隆弁護士という記憶があるぐらいです。身バレというのも「ぽぽひと」だった柴田収弁護士ぐらいです。

 ただ柴田収弁護士の場合、他の発言と関連付けて実名を示唆するツイートとはあったという話でした。実名から匿名に切り替えたのが小田原市の村松謙弁護士で、鍵を掛けて非公開設定にした後しばらくして、アカウントを削除していました。

 なお、柴田収弁護士の場合、ぽぽひとというアカウントは非公開設定にしたままで、別に実名のTwitterアカウントがありました。いつからTwitterを始めていたのか確認しないでいましたが、ぽぽひとというアカウントとの関連付けというのはなさそうに思います。

\begin{quote}
《引用の始まり》
\end{quote}

\begin{quote}
弁護士 柴田収@DV・モラハラ離婚案件がメイン@themis\_okayama岡山市の駅前通りにある弁護士法人岡山テミス法律事務所の代表弁護士です。
離婚問題、特にDV・モラルハラスメント問題を中心に取り扱っています。Web会議システムを利用した相談も受け付け中です。岡山県岡山市北区磨屋町1-6 岡山磨屋町ビル3階themis-okayama.jp2015年3月からTwitterを利用しています222
フォロー中843 フォロワー
\end{quote}

\begin{quote}
《引用の終わり》
\end{quote}

\begin{itemize}
\tightlist
\item
  弁護士 柴田収@DV・モラハラ離婚案件がメインさん (@themis\_okayama)
  / Twitter \url{https://twitter.com/themis\_okayama} 
\end{itemize}

 2015年3月からTwitterを利用しています、とありました。ぽぽひとというアカウントの方も確認をしておきますが、そんなに前からあるアカウントではなかったような気がします。

\begin{quote}
《引用の始まり》
\end{quote}

\begin{quote}
ぽぽひと@悪徳の栄え@popohito新62 地方在住
フォローリクエストは法曹、司法修習生、司法試験受験生、法律事務所事務職員、裁判所・検察庁職員のみ受け付けます。2015年5月からTwitterを利用しています2,043
フォロー中3,741 フォロワー
\end{quote}

\begin{quote}
《引用の終わり》
\end{quote}

\begin{itemize}
\tightlist
\item
  ぽぽひと@悪徳の栄えさん (@popohito) / Twitter
  \url{https://twitter.com/popohito} 
\end{itemize}

 非公開設定にするとTwitterの利用開始時期が表示されないのかと思ったのですが、目を凝らすと「2015年5月からTwitterを利用しています」とありました。

 時刻は14時43分です。時間のことを全く気に掛けずにいたので、15時近くになっているのを見て軽く驚きました。昼は糠漬けのイワシを焼いて食べました。昨日の夕方Aコープ能都店で買ってきたもので、今回も鳥取産となっていました。1年ほど前にも同じものを食べています。

 「こんかイワシ」だったと思いネットで調べたところ、石川県では「こんかイワシ」と呼ぶようです。ここ10年で3,4回ぐらいしか食べた記憶がないのですが、子供の頃にも食べることはありました。

 食べた後でネットで調べたのですが、軽く炙って食べるのが良く、生のままでも食べられるということでした。サバの糠漬けは2年ほど前にそのままスライスで食べられると知ったのですが、そのときもけっこう驚いていました。

 スーパーでは、地元産を含めサバの糠漬けの方が多く売っていると思いますが、焼いて食べた記憶はないかもしれません。米ぬかと塩だけで作るらしいとも知ったのですが、ずいぶんと歴史を感じる味わいの食材だと思いました。

 歴史を感じるといえば、個人的な経験も交えて新潟市なのですが、さきほどGoogleマップで調べたところ裁判所と検察庁の場所がかなり離れていることがわかりました。裁判所の周辺を見てもわからなかったので、別の検索で調べ直したのですが、その前は新潟県司法書士会のホームページを見ていました。

\begin{itemize}
\tightlist
\item
  〈〈〈 2021/06/27 14:56:56 Linux Emacs: 〈〈〈
\end{itemize}

\hypertarget{ux88c1ux5224ux54e1ux5236ux5ea6ux3092ux76aeux8089ux308bux6df1ux6fa4ux8aedux53f2ux5f01ux8b77ux58ebux81eaux8eabux306bux3088ux308bux904eux53bbux306eux30c4ux30a4ux30fcux30c8ux306eux30eaux30c4ux30a4ux30fcux30c8ux304bux3089ux518dux767aux898bux306bux81f3ux3063ux305fux65b0ux6f5fux5973ux5150ux6bbaux5bb3ux4e8bux4ef6ux306bux304aux3051ux308bux9ed9ux79d8ux6a29ux3068ux3046ux306eux5b57ux3068ux306eux30b3ux30e9ux30dcux306bux3088ux308bux30d2ux30e3ux30c3ux30cfux30fcux306eux30c4ux30a4ux30fcux30c84}{%
\paragraph{裁判員制度を皮肉る深澤諭史弁護士自身による過去のツイートのリツイートから再発見に至った、新潟女児殺害事件における黙秘権とうの字とのコラボによるヒャッハーのツイート(4)}\label{ux88c1ux5224ux54e1ux5236ux5ea6ux3092ux76aeux8089ux308bux6df1ux6fa4ux8aedux53f2ux5f01ux8b77ux58ebux81eaux8eabux306bux3088ux308bux904eux53bbux306eux30c4ux30a4ux30fcux30c8ux306eux30eaux30c4ux30a4ux30fcux30c8ux304bux3089ux518dux767aux898bux306bux81f3ux3063ux305fux65b0ux6f5fux5973ux5150ux6bbaux5bb3ux4e8bux4ef6ux306bux304aux3051ux308bux9ed9ux79d8ux6a29ux3068ux3046ux306eux5b57ux3068ux306eux30b3ux30e9ux30dcux306bux3088ux308bux30d2ux30e3ux30c3ux30cfux30fcux306eux30c4ux30a4ux30fcux30c84}}

\begin{itemize}
\tightlist
\item
  〉〉〉 Linux Emacs: 2021/06/27 14:58:30 〉〉〉
\end{itemize}

:CATEGORIES: @kanazawabengosi \#金沢弁護士会 @JFBAsns
日本弁護士連合会(日弁連) \#法務省 @MOJ\_HOUMU \#深澤諭史弁護士
\#うの字 \#黙秘権

 これまでにけっこうな回数、Googleマップで新潟市の地図をみているのですが、今回はこれまでと違ったような見え方がして、新潟地方裁判所の前にある白山神社が、信濃川の対岸となっていました。前回見たときも思ったのですが、その対岸というのは海と川に挟まれ出島のような地形となっています。

 新潟市の白山神社の歴史は古かったはずで、水害が発生すれば孤立するような場所に大きな神社があるのも不思議に思え、信濃川が大きな氾濫を起こすという大水害も発生はしていないのかと考えたりしました。ある弁護士のブログ記事がきっかけと記憶にあるのですが、最初に白山神社を知ったのです。

\begin{itemize}
\tightlist
\item
  2021年06月27日15時05分の登録:
  「白山神社.*新潟」を@hirono\_hideki @kk\_hirono @s\_hironoで検索 14件の該当 2021-06-27\_15:05の記録
  \url{https://kk2020-09.blogspot.com/2021/06/hironohidekikkhironoshirono142021-06.html} 
\item
  2021年06月27日15時05分の登録:
  「新潟.*白山神社」を@hirono\_hideki @kk\_hirono @s\_hironoで検索 28件の該当 2021-06-27\_15:05の記録
  \url{https://kk2020-09.blogspot.com/2021/06/hironohidekikkhironoshirono282021-06.html} 
\end{itemize}

2014-08-08 12:52:48 ``RT @BarlKarth:
白山神社に「新潟地裁が死刑を言い渡しませんように。」と記載した絵馬を奉納しようかな。''
\url{https://twitter.com/hirono\_hideki/status/497591230293745665} 

2020-01-15 14:24:17 ``-*-
樋詰哲朗弁護士(金沢弁護士会)のツイートから知ることになった、新潟市の白山神社、新潟総鎮守、Googleマップで確認した場所は新潟地裁の向かいだった''
\url{https://twitter.com/kk\_hirono/status/1217316553806401536} 

2020-01-15 17:21:52 ``94:2020-01-15\_17:20:55 *
樋詰哲朗弁護士(金沢弁護士会)のツイートから知ることになった、新潟市の白山神社、新潟総鎮守、Googleマップで確認した場所は新潟地裁の向かいだった
\url{https://hirono-hideki.hatenadiary.jp/entry/2020/01/15/172050''} 
\url{https://twitter.com/hirono\_hideki/status/1217361241426972673} 

2020-12-14 18:32:24
``新潟総鎮守 白山神社|初詣、七五三、縁結び、安産、厄除けなど
\url{http://www.niigatahakusanjinja.or.jp/} 
思い出したけど,「新潟総鎮守」となっていた。一の宮はたまに見かける,能登一の宮は気多大社とか。''
\url{https://twitter.com/hirono\_hideki/status/1338416541591687173} 

 高島章弁護士(新潟県弁護士会)と神社というのは、「金玉踊り」と「金玉神社」しか記憶になかったのですが、ネットで高島章弁護士(新潟県弁護士会)の名前を見かけなくなって久しいです。

 樋詰哲朗弁護士(金沢弁護士会)と新潟総鎮守白山神社との関係というのも記憶になかったのですが、記事のようなものを作成していたようです。意外な発見です。

\begin{quote}
《引用の始まり》
\end{quote}

\begin{quote}
 新潟については、これまで何度か取り上げてきた経緯がありますが、大きなポイントが、「金玉神社」や「金玉踊り」にこだだわった高島章弁護士(新潟県弁護士会)、新発田市の連続強姦事件(逃走事件も)、そして昨年の12月の初めに無期懲役の判決が出た新潟西区女児殺害事件となります。

 白山神社についても取り上げてきましたが、最初に気になったのが輪島市名舟町の神社で、白山神社と奥津比咩神社の2つが神社名と知ったことです。高台に社殿があり、子供の頃に神社を見た記憶ははっきりしないのですが、ゲゲゲの鬼太郎の牛鬼の場面とイメージが重なっていました。
\end{quote}

\begin{quote}
《引用の終わり》
\end{quote}

\begin{itemize}
\item
  *
  樋詰哲朗弁護士(金沢弁護士会)のツイートから知ることになった、新潟市の白山神社、新潟総鎮守、Googleマップで確認した場所は新潟地裁の向かいだった
  - 告発\金沢地方検察庁\最高検察庁\法務省\石川県警察御中2020
  \url{https://hirono-hideki.hatenadiary.jp/entry/2020/01/15/172050} 
\item
  2020年01月15日15:07記録\法務検察・石川県警察宛\写真資料:輪島市名舟町 奥津比咩神社 白山神社(2014年年08月23日〜2019年06月01日:47件
  \url{https://t.co/hNOXwXfW3P} 
\end{itemize}

〉〉〉 kk\_hironoのリツイート 〉〉〉

\begin{itemize}
\tightlist
\item
  RT
  kk\_hirono(刑事告発・非常上告_金沢地方検察庁御中)|hirono\_hideki(奉納\さらば弁護士鉄道・泥棒神社の物語)
  日時:2021-06-27 15:19/2019/01/26 03:08 URL:
  \url{https://twitter.com/kk\_hirono/status/1409033614818385920} 
  \url{https://twitter.com/hirono\_hideki/status/1088861237314609152} 
  \textgreater{}
  2012年に失踪、愛知県の山中で元漫画喫茶店員女性=当時(41)=の遺体が見つかった事件では、傷害致死容疑で逮捕後、黙秘を続けて不起訴となった元同店経営者夫婦を相手取り、遺族が損害賠償を求めて提訴。
  \url{https://t.co/im99Z69swd} 
\end{itemize}

〉〉〉 kk\_hironoのリツイート 〉〉〉

\begin{itemize}
\tightlist
\item
  RT
  kk\_hirono(刑事告発・非常上告_金沢地方検察庁御中)|hirono\_hideki(奉納\さらば弁護士鉄道・泥棒神社の物語)
  日時:2021-06-27 15:19/2019/01/26 03:04 URL:
  \url{https://twitter.com/kk\_hirono/status/1409033663992393733} 
  \url{https://twitter.com/hirono\_hideki/status/1088860195973824512} 
  \textgreater{}
  「娘の最後話して」黙秘権の壁、遺族苦悩 転落死の損賠訴訟(西日本新聞)
  - Yahoo!ニュース \url{https://t.co/im99Z69swd} 
\end{itemize}

〉〉〉 kk\_hironoのリツイート 〉〉〉

\begin{itemize}
\tightlist
\item
  RT
  kk\_hirono(刑事告発・非常上告_金沢地方検察庁御中)|hirono\_hideki(奉納\さらば弁護士鉄道・泥棒神社の物語)
  日時:2021-06-27 15:20/2019/01/26 13:43 URL:
  \url{https://twitter.com/kk\_hirono/status/1409033756317339651} 
  \url{https://twitter.com/hirono\_hideki/status/1089020982008852485} 
  \textgreater{} 越後一宮 彌彦神社 \url{https://t.co/ai7hF1bQYC} 
\end{itemize}

※ @kk\_hironoのアカウントがブロックされ,リツイートに失敗したツイート

\begin{itemize}
\tightlist
\item
  TW hizumelaw(弁護士樋詰哲朗) 日時:2020/01/10 12:31:02 URL:
  \url{https://twitter.com/hizumelaw/status/1215476112358166529} 
  \textgreater{}
  米山一史先生のはてなダイアリー、新人の頃熟読していたのだけれど、はてなブログに移行されておらず、最近読めなくなって悲しい。
\end{itemize}

\begin{quote}
《引用の始まり》
\end{quote}

\begin{quote}
しっかりブロックされ続けていることを確認した樋詰哲朗弁護士(金沢弁護士会)のTwitterアカウントになります。奉納\さらば弁護士鉄道・泥棒神社の物語(@hirono\_hideki)の方で先にブロックされたことを確認していました。

 比較的最近になって存在を知ったアカウントになるかと思います。当初は石川県への移住者と考えていたことなど、前にも取り上げていると思います。金沢弁護士会所属の弁護士では珍しく、ネットを使った積極的な弁護士業務をされているようです。

 小学校の低学年と思われる息子がいるようで、子煩悩ぶりを伝わるツイートも見かけてきましたが、Twitterの更新の頻度はかなり低めです。何日も更新のないことがざらにあり、たまに更新をみると珍しく感じています。

 その樋詰哲朗弁護士(金沢弁護士会)がツイートで名前を出していたのが「米山一史先生」で、名前の部分をGoogle検索したところ、新潟の弁護士で事務所名が「白山パーク法律事務所」だと知りました。以前に少し見かけていたような気もしたのですが今回は「白山パーク」のことが気になりました。
\end{quote}

\begin{quote}
《引用の終わり》
\end{quote}

\begin{itemize}
\tightlist
\item
  *
  樋詰哲朗弁護士(金沢弁護士会)のツイートから知ることになった、新潟市の白山神社、新潟総鎮守、Googleマップで確認した場所は新潟地裁の向かいだった
  - 告発\金沢地方検察庁\最高検察庁\法務省\石川県警察御中2020
  \url{https://hirono-hideki.hatenadiary.jp/entry/2020/01/15/172050} 
\end{itemize}

\begin{quote}
《引用の始まり》
\end{quote}

\begin{quote}
上記に引用をしましたが、「白山神社の御祭神、菊理媛大神(白山大神)は別名を、白山比咩(しらやまひめ)大神と言い、加賀の霊峰白山頂上に祀られている女神さまで、
この神様を勧請して新潟の地に祀ったものです。」とありました。これはすごく納得の説明でありがたく感じました。

 「菊理媛大神は願うことを正しくよりよい方法でお導きくださり、乱れた糸をくくり整えるように融和され仲を取り持ち和す縁結びの神様であります。」という糸に関連付けられたお話は初めてみたように思いますが、個人的には福井県勝山市の繊維工場と繋がりがあります。

 その福井県勝山市の繊維工場のことは、石川県警察と関係があって、私の人生の岐路になった可能性もあることです。ずいぶん前から予定していた記述の1つですが、この機会に早めにやっておきたいと思います。
\end{quote}

\begin{quote}
《引用の終わり》
\end{quote}

\begin{itemize}
\tightlist
\item
  *
  樋詰哲朗弁護士(金沢弁護士会)のツイートから知ることになった、新潟市の白山神社、新潟総鎮守、Googleマップで確認した場所は新潟地裁の向かいだった
  - 告発\金沢地方検察庁\最高検察庁\法務省\石川県警察御中2020
  \url{https://hirono-hideki.hatenadiary.jp/entry/2020/01/15/172050} 
\end{itemize}

 「その福井県勝山市の繊維工場のことは、石川県警察と関係があって、私の人生の岐路になった可能性もあることです。」とありますが、この続きもまだ記述はしていなかったように思います。その勝山市の繊維工場に行った仕事は、国勝運送の仕事でした。

 たぶん一日、仕事をしただけだったと思うのですが、翌日の仕事のため松任市内に車を停めて寝ようとしていたところ、パトカーの警察に整備不良車だと言われ、今度見つけたら切符を切ると言われたのでした。当時はシャコタンと呼ばれていましたがノーサスというサスペンションを取り払ったものでした。

 急に直すことなどできるはずもなく、仕事の場所になっていたのでまた警察に見つかる可能性が高く、それで国勝運送の仕事をやめることにしたのです。面接をしたのも本社があったのも金沢市内の神田付近だったと思いますが、翌日の仕事の場所は松任市内で、8号線バイパスから少し入った場所でした。

 当時、国勝運送は金沢市周辺で2番目に大きいと言われた地場の運送会社で1番と言われていたのが北都運輸でした。北都運輸の会社は松任市内でパトカーに警告書をもらった場所の近くだったと思います。金沢市場輸送の仕事でその本社のような会社に行くことがありました。

 あのまま国勝運送で仕事を続けていれば、人生が変わっていたかもしれないし、勝山市まで同行した社員は年配でしたが、とても親切に仕事を教えてくれたという印象がありました。北都運輸も同じでしたが、運転手は若者が多く、出入りも激しいとは聞いていました。

\begin{itemize}
\tightlist
\item
  米山一史 - 新潟県弁護士会 \url{https://t.co/k5aLnUcQzm} 
\end{itemize}

 新潟県弁護士会の建物の写真がありますが、あらためて昭和56年当時の金沢家庭裁判所の建物に酷似していると思いました。米山一史弁護士の名前もすっかり忘れ、見かけることもなかったと思いますが、@のリンクがツイートで無効となっており、アカウントは削除されているようです。

\begin{itemize}
\tightlist
\item
  弁護士紹介 \textbar{} 白山パーク法律事務所 \textbar{}
  新潟市中央区の弁護士事務所 \url{https://t.co/RUWUGYBFA3} 
\end{itemize}

 新しい情報はほとんど見かけなかったですが、この白山パーク法律事務所の地図を知ったことで、白山神社と新潟地方裁判所の場所を知ることになりました。新潟の弁護士については、そのずっと前から高島章弁護士(新潟県弁護士会)と、新発田市の連続事件のことで注目していました。

\begin{itemize}
\item
  子の安全どう守る 事件から3年 新潟で見守り隊増加:朝日新聞デジタル
  \url{https://t.co/x7h5EpeViZ} 
\item
  「今すぐ退官して二度と法曹界に戻るべきでない」という深澤諭史弁護士のツイートをリツイート:新潟女児殺害事件小林遥被告の死刑求刑、「憲法ガール」著者、大島義則弁護士の反応
  - 告発\金沢地方検察庁\最高検察庁\法務省\石川県警察御中
  \url{https://t.co/2lf4dC3wJl} 
\end{itemize}

 また忘れていた自分の記事がGoogleの検索結果に出てきました。「小林遼 弁護士」という検索です。

\begin{quote}
《引用の始まり》
\end{quote}

\begin{quote}
深澤諭史弁護士は憲法の尊重を裁判官に求め、「今すぐ退官して二度と法曹界に戻るべきでない。」と厳しく指弾をしているようです。法曹界に戻るべきでない、というのは退官後に弁護士登録をするなと言っているのでしょう。

 もともと自分と自分の同調者以外に厳しい態度を示すのが、モトケンこと矢部善朗弁護士(京都弁護士会)と深澤諭史弁護士の際立つ特徴で、深澤諭史弁護士の方が影響を受けているのかと考えたこともありました。

 これは安易な言葉、感情で評価すべきものではなく、深刻重大なものとして受け止めております。それを憲法の専門家で研究や執筆もされているらしい大島義則弁護士がリツイートをしたわけで、さらに「辞表を提出」「それでもダメ」というさらに厳しい評価のツイートを行っています。
\end{quote}

\begin{quote}
《引用の終わり》
\end{quote}

\begin{itemize}
\tightlist
\item
  「今すぐ退官して二度と法曹界に戻るべきでない」という深澤諭史弁護士のツイートをリツイート:新潟女児殺害事件小林遥被告の死刑求刑、「憲法ガール」著者、大島義則弁護士の反応
  - 告発\金沢地方検察庁\最高検察庁\法務省\石川県警察御中
  \url{https://hirono-hideki.hatenablog.com/entry/2019/11/25/131610} 
\end{itemize}

 どうも深澤諭史弁護士のツイートが記事に見当たらないのですが、その深澤諭史弁護士のツイートを「憲法ガール」著者、大島義則弁護士がリツイートしていたらしいことが確認できました。

\begin{itemize}
\tightlist
\item
  〈〈〈 2021/06/27 16:30:50 Linux Emacs: 〈〈〈
\end{itemize}

\hypertarget{ux88c1ux5224ux54e1ux5236ux5ea6ux3092ux76aeux8089ux308bux6df1ux6fa4ux8aedux53f2ux5f01ux8b77ux58ebux81eaux8eabux306bux3088ux308bux904eux53bbux306eux30c4ux30a4ux30fcux30c8ux306eux30eaux30c4ux30a4ux30fcux30c8ux304bux3089ux518dux767aux898bux306bux81f3ux3063ux305fux65b0ux6f5fux5973ux5150ux6bbaux5bb3ux4e8bux4ef6ux306bux304aux3051ux308bux9ed9ux79d8ux6a29ux3068ux3046ux306eux5b57ux3068ux306eux30b3ux30e9ux30dcux306bux3088ux308bux30d2ux30e3ux30c3ux30cfux30fcux306eux30c4ux30a4ux30fcux30c85}{%
\paragraph{裁判員制度を皮肉る深澤諭史弁護士自身による過去のツイートのリツイートから再発見に至った、新潟女児殺害事件における黙秘権とうの字とのコラボによるヒャッハーのツイート(5)}\label{ux88c1ux5224ux54e1ux5236ux5ea6ux3092ux76aeux8089ux308bux6df1ux6fa4ux8aedux53f2ux5f01ux8b77ux58ebux81eaux8eabux306bux3088ux308bux904eux53bbux306eux30c4ux30a4ux30fcux30c8ux306eux30eaux30c4ux30a4ux30fcux30c8ux304bux3089ux518dux767aux898bux306bux81f3ux3063ux305fux65b0ux6f5fux5973ux5150ux6bbaux5bb3ux4e8bux4ef6ux306bux304aux3051ux308bux9ed9ux79d8ux6a29ux3068ux3046ux306eux5b57ux3068ux306eux30b3ux30e9ux30dcux306bux3088ux308bux30d2ux30e3ux30c3ux30cfux30fcux306eux30c4ux30a4ux30fcux30c85}}

\begin{itemize}
\tightlist
\item
  〉〉〉 Linux Emacs: 2021/06/27 16:32:59 〉〉〉
\end{itemize}

:CATEGORIES: @kanazawabengosi \#金沢弁護士会 @JFBAsns
日本弁護士連合会(日弁連) \#法務省 @MOJ\_HOUMU \#深澤諭史弁護士
\#うの字 \#黙秘権

\begin{itemize}
\tightlist
\item
  1430:2021-06-26\_16:28:55 \#告発状 \#\#\#\#
  裁判員制度を皮肉る深澤諭史弁護士自身による過去のツイートのリツイートから再発見に至った、新潟女児殺害事件における黙秘権とうの字とのコラボによるヒャッハーのツイート(1)
  \url{https://hirono-hideki.hatenadiary.jp/entry/2021/06/26/162853} 
\item
  1431:2021-06-26\_18:49:41 \#告発状 \#\#\#\#
  裁判員制度を皮肉る深澤諭史弁護士自身による過去のツイートのリツイートから再発見に至った、新潟女児殺害事件における黙秘権とうの字とのコラボによるヒャッハーのツイート(2)
  \url{https://hirono-hideki.hatenadiary.jp/entry/2021/06/26/184938} 
\item
  1432:2021-06-27\_14:57:35 \#告発状 \#\#\#\#
  裁判員制度を皮肉る深澤諭史弁護士自身による過去のツイートのリツイートから再発見に至った、新潟女児殺害事件における黙秘権とうの字とのコラボによるヒャッハーのツイート(3)
  \url{https://hirono-hideki.hatenadiary.jp/entry/2021/06/27/145733} 
\item
  1433:2021-06-27\_16:31:12 \#告発状 \#\#\#\#
  裁判員制度を皮肉る深澤諭史弁護士自身による過去のツイートのリツイートから再発見に至った、新潟女児殺害事件における黙秘権とうの字とのコラボによるヒャッハーのツイート(4)
  \url{https://hirono-hideki.hatenadiary.jp/entry/2021/06/27/163110} 
\end{itemize}

\begin{quote}
《引用の始まり》
\end{quote}

\begin{quote}
 iphoneのアラームはよく使ってきたのですが、同じ時計のアプリにストップウォッチがあることも知りませんでした。11月22日のNHKのNEWS7で、新潟女児殺害事件死刑求刑のニュースは48秒間でした。弁護士の懲役10年以下という量刑意見もすっきり省かれていたようです。

 深澤諭史弁護士の前田恒彦氏のYAHOOニュースのコメント記事を引用したツイートは、早い段階で目にしていたもので、前日に死刑求刑の公判があったということも、この深澤諭史弁護士のツイートがきっかけで初めて知ったように思います。スクリーンショットの記録も残しているかと思います。

 上記で全文を引用した朝日新聞デジタルの記事を読んだだけでは見えてこない、公判の様子が他のネット上の記事にはあって、特に死刑を求刑された被告本人の言葉には、深く考えさせられるところがあって、死刑求刑の公判では前例をみたことのない内容であったと思います。
\end{quote}

\begin{quote}
《引用の終わり》
\end{quote}

\begin{itemize}
\tightlist
\item
  「今すぐ退官して二度と法曹界に戻るべきでない」という深澤諭史弁護士のツイートをリツイート:新潟女児殺害事件小林遥被告の死刑求刑、「憲法ガール」著者、大島義則弁護士の反応
  - 告発\金沢地方検察庁\最高検察庁\法務省\石川県警察御中
  \url{https://hirono-hideki.hatenablog.com/entry/2019/11/25/131610} 
\end{itemize}

\begin{lstlisting}
base ❯ d|grep 退官して
\end{lstlisting}

\begin{itemize}
\tightlist
\item
  2014年07月21日00時53分の登録:
  検事は、転勤して栄進、幹部は退官して、大手企業の顧問弁護士、監査役、公証¥\n人等で左団扇の生活。検察審査員はどこの誰かもわからないし、/落合洋司弁護¥\n士
  \url{http://hirono2014sk.blogspot.com/2014/07/blog-post\_8937.html} 
\item
  2019年01月09日17時28分の登録:
  \たかしRX @yst64\もと検事の落合洋司先生の小説『ニチョウ』買った。30ページほど読んだけど、主役の経歴が新任明けで徳島地検へ赴任してたり、10年で退官して弁護士
  \url{http://hirono2014sk.blogspot.com/2019/01/rxyst643010.html} 
\item
  2020年01月03日21時36分の登録:
  \豚野郎 @butayar0\司法試験に受かったのに検察事務官になった(?)主人公のドラマが話題ですが,学者から副検事になり検事になったあと退官して弁護士になって,国選の
  \url{http://hirono2014sk.blogspot.com/2020/01/butayar0.html} 
\end{itemize}

 深澤諭史弁護士のツイートを記録した痕跡がありませんでした。「今すぐ退官して二度と法曹界に戻るべきでない」という内容を含むツイートです。

\begin{itemize}
\item
  2021年06月27日16時43分の登録:
  REGEXP:''法曹界''/深澤諭史(@fukazawas)の検索(2013-01-27〜2020-09-04/2021年06月27日16時43分の記録134件)
  \url{https://kk2020-09.blogspot.com/2021/06/regexpfukazawas2013-01-272020-09.html} 
\item
  2021年06月27日16時43分の登録:
  REGEXP:''今すぐ退官して二度と法曹界に戻るべきでない。''/データベース登録済みツイートの検索:2019-11-25〜2021-06-27/2021年06月27日16時42分の記録:ユーザ・投稿:1/3件
  \url{https://kk2020-09.blogspot.com/2021/06/regexp2019-11-252021-06-2720210627164213.html} 
\item
  (012/134) TW fukazawas(深澤諭史) 日時:2015-05-21 22:07:00 +0900
  URL:
  \url{https://twitter.com/fukazawas/status/601373846101299200\textgreater} {}
  @sakamotomasayuk @Noooooooorth\textgreater{}
  有為の人物を失ったのは,法曹界の損失ですね。\textgreater{}
  弔辞は当職が読みます(・∀・)\textgreater{}
  いや,ほんと,笑いが,じゃなかった涙が止まりません(T∀T)
\end{itemize}

 見覚えのないツイートがまとめ記事に記録されていましたが、メンションは坂本正幸弁護士と北周士弁護士のようです。

\begin{itemize}
\item
  非常上告-最高検察庁御中\_ツイッター(@s\_hirono)/「法曹界」の検索結果 -
  Twilog \url{https://t.co/Ud57bYGesj} 
\item
  非常上告-最高検察庁御中\_ツイッター(@s\_hirono)/「退官」の検索結果 -
  Twilog \url{https://t.co/j2oWWwMGdD} 
\end{itemize}

 探しているのですが、「法曹界」と「退官」を同時に含む深澤諭史弁護士のツイートが見つからずにいます。

※ @kk\_hironoのアカウントがブロックされ,リツイートに失敗したツイート

\begin{itemize}
\tightlist
\item
  TW fukazawas(深澤諭史) 日時:2019/11/23 10:08:47 URL:
  \url{https://twitter.com/fukazawas/status/1198045696521199616} 
  \textgreater{}
  【前田恒彦さんのコメント】「あなたにできる最低限の償いは真実を述べることだと思いますが、真実を述べないこ\ldots{}
  ▼新潟女児殺害、被告に死刑求刑「まれにみる非道な犯行」
  \url{https://t.co/28p6lROCsV} 
  \textgreater{}
  これ本当にいったなら、今すぐ退官して二度と法曹界に戻るべきでない。\\
  \textgreater{} (^ω^#)
\end{itemize}

 見つかったのですが、d\textbar grep
`/深澤諭史(@fukazawas)/''2019年11月23日''' という検索から出てきた次のまとめ記事です。これは振り出しに戻るかっこうになります。

\begin{itemize}
\tightlist
\item
  2019年11月24日01時17分の登録:
  ツイートの記録資料:\法務検察・石川県警察宛\/深澤諭史(@fukazawas)/''2019年11月23日'':86件
  \url{http://hirono2014sk.blogspot.com/2019/11/fukazawas2019112386.html} 
\end{itemize}

 夜中に目が覚めて具合が悪かったときですが、深澤諭史弁護士のリツイートにあった日付で、上記の通り調べようとしたところ、2019とするつもりが、キーボードで0を打っても2が入力されたり、他にも違って数字となっていました。

 端末のアプリを変えてみたところ普通に入力が出来、そのあとおかしな現象を起こしていた端末で再度やってみると、今度は普通に入力ができました。

 まるで怪奇現象のようだったのですが、重苦しい心身の不調も相まって、これは深澤諭史弁護士を徹底的に記録せよというお告げではないかと思い、実行を始めたのです。

 深澤諭史弁護士のツイートの返信欄に次のツイートがありました。1つは昨日見かけたアカウントです。

〉〉〉 kk\_hironoのリツイート 〉〉〉

\begin{itemize}
\tightlist
\item
  RT
  kk\_hirono(刑事告発・非常上告_金沢地方検察庁御中)|perplex9(ぱ~ぷ。'13)
  日時:2021-06-27 17:10/2019/11/23 11:44 URL:
  \url{https://twitter.com/kk\_hirono/status/1409061581414010882} 
  \url{https://twitter.com/perplex9/status/1198069804206186496} 
  \textgreater{} @fukazawas
  \textgreater 裁判長は「全てをうそとは言わないが」と前置きした上で、「最低限の償いは真実を述べることだ。真実を話さないのなら、最低限の償いの気持ちもないと理解される」と語り掛けた。
  「信じてもらえると思うのか」新潟女児殺害 裁判長、異例の質問
  \url{https://t.co/YgiqLColwR} 
\end{itemize}

〉〉〉 kk\_hironoのリツイート 〉〉〉

\begin{itemize}
\tightlist
\item
  RT
  kk\_hirono(刑事告発・非常上告_金沢地方検察庁御中)|hkteti(kkthikt)
  日時:2021-06-27 17:11/2019/11/24 11:29 URL:
  \url{https://twitter.com/kk\_hirono/status/1409061713832398850} 
  \url{https://twitter.com/hkteti/status/1198428478808281088} 
  \textgreater{} @fukazawas @okumuraosaka
  一般人としての論評、意見ならアリなのかもしれませんが、公正中立であることが求められる裁判官としてはいけなかったんですね。
\end{itemize}

〉〉〉 kk\_hironoのリツイート 〉〉〉

\begin{itemize}
\tightlist
\item
  RT
  kk\_hirono(刑事告発・非常上告_金沢地方検察庁御中)|anjela81(あみお😷🧼🏠👈6ft👉💉💉)
  日時:2021-06-27 17:11/2019/11/23 10:26 URL:
  \url{https://twitter.com/kk\_hirono/status/1409061770052866048} 
  \url{https://twitter.com/anjela81/status/1198050159453708288} 
  \textgreater{} @fukazawas
  はじめまして。憲法38条の「何人も、自己に不利益な供述を強要されない」に抵触するから、という理解で合ってますでしょうか?
  もし良かったら教えていただけると幸いです。
\end{itemize}

〉〉〉 kk\_hironoのリツイート 〉〉〉

\begin{itemize}
\item
  RT
  kk\_hirono(刑事告発・非常上告_金沢地方検察庁御中)|tks13ten(高野和美)
  日時:2021-06-27 17:11/2019/11/24 13:06 URL:
  \url{https://twitter.com/kk\_hirono/status/1409061893189214212} 
  \url{https://twitter.com/tks13ten/status/1198452798024667136} 
  \textgreater{} @fukazawas @anjela81
  これはお白州でお奉行様が咎人に申しつけるのと同じ感覚ですね(・∀・)
\item
  〉〉〉 アカウント(@fukazawas)は,@kk\_hironoをブロックしています。リツイートできませんでした。
  〉〉〉 ¥\n ¥\n \url{https://t.co/kLUvEAUVP3} 
\item
  〉〉〉 アカウント(@fukazawas)は,@kk\_hironoをブロックしています。リツイートできませんでした。
  〉〉〉 ¥\n ¥\n \url{https://t.co/fhFhm6EyHk} 
\end{itemize}

※ @kk\_hironoのアカウントがブロックされ,リツイートに失敗したツイート

\begin{itemize}
\tightlist
\item
  TW fukazawas(深澤諭史) 日時:2019/11/23 10:29:51 URL:
  \url{https://twitter.com/fukazawas/status/1198050998108876801} 
  \textgreater{} @anjela81 そうです。\\
  \textgreater{}
  それと憲法31条の適正手続保障、37条1項の公平な裁判所で裁判を受ける権利の侵害にも当たると考えます。
\end{itemize}

※ @kk\_hironoのアカウントがブロックされ,リツイートに失敗したツイート

\begin{itemize}
\item
  TW fukazawas(深澤諭史) 日時:2019/11/24 09:20:34 URL:
  \url{https://twitter.com/fukazawas/status/1198395950542151680} 
  \textgreater{} @shinobuhome
  私はその可能性が高いと思います。もっとも,公正らしさを害しますし,その心証に重きを置くことも不適切です。\\
  \textgreater{}
  連日開廷の裁判員裁判で,評議の日程前の心証ですから,この時点で心証形成は難しいはずです。
\item
  〈〈〈 2021/06/27 17:13:47 Linux Emacs: 〈〈〈
\end{itemize}

\hypertarget{ux88c1ux5224ux54e1ux5236ux5ea6ux3092ux76aeux8089ux308bux6df1ux6fa4ux8aedux53f2ux5f01ux8b77ux58ebux81eaux8eabux306bux3088ux308bux904eux53bbux306eux30c4ux30a4ux30fcux30c8ux306eux30eaux30c4ux30a4ux30fcux30c8ux304bux3089ux518dux767aux898bux306bux81f3ux3063ux305fux65b0ux6f5fux5973ux5150ux6bbaux5bb3ux4e8bux4ef6ux306bux304aux3051ux308bux9ed9ux79d8ux6a29ux3068ux3046ux306eux5b57ux3068ux306eux30b3ux30e9ux30dcux306bux3088ux308bux30d2ux30e3ux30c3ux30cfux30fcux306eux30c4ux30a4ux30fcux30c86}{%
\paragraph{裁判員制度を皮肉る深澤諭史弁護士自身による過去のツイートのリツイートから再発見に至った、新潟女児殺害事件における黙秘権とうの字とのコラボによるヒャッハーのツイート(6)}\label{ux88c1ux5224ux54e1ux5236ux5ea6ux3092ux76aeux8089ux308bux6df1ux6fa4ux8aedux53f2ux5f01ux8b77ux58ebux81eaux8eabux306bux3088ux308bux904eux53bbux306eux30c4ux30a4ux30fcux30c8ux306eux30eaux30c4ux30a4ux30fcux30c8ux304bux3089ux518dux767aux898bux306bux81f3ux3063ux305fux65b0ux6f5fux5973ux5150ux6bbaux5bb3ux4e8bux4ef6ux306bux304aux3051ux308bux9ed9ux79d8ux6a29ux3068ux3046ux306eux5b57ux3068ux306eux30b3ux30e9ux30dcux306bux3088ux308bux30d2ux30e3ux30c3ux30cfux30fcux306eux30c4ux30a4ux30fcux30c86}}

\begin{itemize}
\tightlist
\item
  〉〉〉 Linux Emacs: 2021/06/27 17:15:25 〉〉〉
\end{itemize}

:CATEGORIES: @kanazawabengosi \#金沢弁護士会 @JFBAsns
日本弁護士連合会(日弁連) \#法務省 @MOJ\_HOUMU \#深澤諭史弁護士
\#うの字 \#黙秘権

\begin{itemize}
\item
  1434:2021-06-27\_17:14:10 \#告発状 \#\#\#\#
  裁判員制度を皮肉る深澤諭史弁護士自身による過去のツイートのリツイートから再発見に至った、新潟女児殺害事件における黙秘権とうの字とのコラボによるヒャッハーのツイート(5)
  \url{https://hirono-hideki.hatenadiary.jp/entry/2021/06/27/171407} 
\item
  TW fukazawas(深澤諭史) 日時: 2019/11/23 10:08:47 URL:
  \url{https://twitter.com/fukazawas/status/1198045696521199616} 
  \textgreater{}
  【前田恒彦さんのコメント】「あなたにできる最低限の償いは真実を述べることだと思いますが、真実を述べないこ\ldots{}
  ▼新潟女児殺害、被告に死刑求刑「まれにみる非道な犯行」
  \url{https://t.co/28p6lROCsV} 
  \textgreater{}
  これ本当にいったなら、今すぐ退官して二度と法曹界に戻るべきでない。\\
  \textgreater{} (^ω^#)
\end{itemize}

 次に上記の2019年11月23日10時08分47秒の深澤諭史弁護士のツイートの引用ツイートです。

〉〉〉 kk\_hironoのリツイート 〉〉〉

\begin{itemize}
\tightlist
\item
  RT
  kk\_hirono(刑事告発・非常上告_金沢地方検察庁御中)|ahoaho1818(みみずく)
  日時:2021-06-27 17:19/2019/11/24 12:16 URL:
  \url{https://twitter.com/kk\_hirono/status/1409063931197353986} 
  \url{https://twitter.com/ahoaho1818/status/1198440236876877824} 
  \textgreater{}
  J「あなたには、黙秘権が保証されています。言いたくないことは言わなくても構いません。また法廷内での発言はすべて証拠となりますので、注意してください」
  J「真実を述べないことは最低限の償いをするつもりがないというものだとみなされることは分かりますか?」
  ?????? \url{https://t.co/DGv9pMPp5U} 
\end{itemize}

〉〉〉 kk\_hironoのリツイート 〉〉〉

\begin{itemize}
\tightlist
\item
  RT
  kk\_hirono(刑事告発・非常上告_金沢地方検察庁御中)|std\_clam(あさりしょかん)
  日時:2021-06-27 17:20/2019/11/23 21:49 URL:
  \url{https://twitter.com/kk\_hirono/status/1409063966001696771} 
  \url{https://twitter.com/std\_clam/status/1198221970015277056} 
  \textgreater{} えぇ・・・・・・、これ本当?
  冤罪だったらどうするつもりなんだろ。 \url{https://t.co/9Fob6S1Gss} 
\end{itemize}

〉〉〉 kk\_hironoのリツイート 〉〉〉

\begin{itemize}
\tightlist
\item
  RT kk\_hirono(刑事告発・非常上告_金沢地方検察庁御中)|G6RKz2(Toast
  Rider@636) 日時:2021-06-27 17:20/2019/11/23 21:28 URL:
  \url{https://twitter.com/kk\_hirono/status/1409064020640878596} 
  \url{https://twitter.com/G6RKz2/status/1198216641579909121} 
  \textgreater{}
  被告人に同調するつもりはないが、これでは公正中立な裁判が担保されているとは言い難いな。
  \url{https://t.co/egkCQWtNjR} 
\end{itemize}

〉〉〉 kk\_hironoのリツイート 〉〉〉

\begin{itemize}
\tightlist
\item
  RT
  kk\_hirono(刑事告発・非常上告_金沢地方検察庁御中)|misagoya(みさご)
  日時:2021-06-27 17:20/2019/11/23 18:28 URL:
  \url{https://twitter.com/kk\_hirono/status/1409064074302746624} 
  \url{https://twitter.com/misagoya/status/1198171569488482304} 
  \textgreater{}
  裁判長が先入観をもってする裁判が公平な裁判と言えるのかななどと考えてしまいますね
  \url{https://t.co/Tjb97IZ4us} 
\end{itemize}

〉〉〉 kk\_hironoのリツイート 〉〉〉

\begin{itemize}
\tightlist
\item
  RT kk\_hirono(刑事告発・非常上告_金沢地方検察庁御中)|ja1hss(John
  Doe) 日時:2021-06-27 17:20/2019/11/23 17:41 URL:
  \url{https://twitter.com/kk\_hirono/status/1409064166837551104} 
  \url{https://twitter.com/ja1hss/status/1198159687939588096} 
  \textgreater{} @antadare\_mayune
  弁護士の深澤先生のコメント。他にも同様のコメントをされてる弁護士さん多い。
  私は直ちに法曹追放しろとまでは思わないけど、裁判官が諭す意味というか合理的理由ないと思います。
  \url{https://t.co/HZ3GwJnLp3} 
\end{itemize}

〉〉〉 kk\_hironoのリツイート 〉〉〉

\begin{itemize}
\tightlist
\item
  RT
  kk\_hirono(刑事告発・非常上告_金沢地方検察庁御中)|boosterbrain(宇桐
  慶英) 日時:2021-06-27 17:21/2019/11/23 17:05 URL:
  \url{https://twitter.com/kk\_hirono/status/1409064247414247429} 
  \url{https://twitter.com/boosterbrain/status/1198150563579383808} 
  \textgreater{}
  これ、昔司法試験予備校で勉強してたときに最も違和感のあった通説。曰く、黙秘権はあるが有罪が確定した場合、黙秘したことを反省がないとして量刑にネガティブに作用させることは認められる、と。
  \url{https://t.co/c9w1WCPvOD} 
\end{itemize}

〉〉〉 kk\_hironoのリツイート 〉〉〉

\begin{itemize}
\tightlist
\item
  RT
  kk\_hirono(刑事告発・非常上告_金沢地方検察庁御中)|tsuittarian(家に帰ったら手を洗い、うがいをしましょう@tsuittarian)
  日時:2021-06-27 17:21/2019/11/23 15:53 URL:
  \url{https://twitter.com/kk\_hirono/status/1409064296860975107} 
  \url{https://twitter.com/tsuittarian/status/1198132398921269248} 
  \textgreater{} 弁護士さんの頭の中はよく判らん \url{https://t.co/kjLofWbmSS} 
\end{itemize}

〉〉〉 kk\_hironoのリツイート 〉〉〉

\begin{itemize}
\tightlist
\item
  RT
  kk\_hirono(刑事告発・非常上告_金沢地方検察庁御中)|tsundereblog(ツンデレブログ 喧嘩腰じゃねーよ)
  日時:2021-06-27 17:21/2019/11/23 15:15 URL:
  \url{https://twitter.com/kk\_hirono/status/1409064345112248324} 
  \url{https://twitter.com/tsundereblog/status/1198122910696660992} 
  \textgreater{} とりあえず忌避かなあ \url{https://t.co/6HwxQQBZRO} 
\end{itemize}

〉〉〉 kk\_hironoのリツイート 〉〉〉

\begin{itemize}
\tightlist
\item
  RT
  kk\_hirono(刑事告発・非常上告_金沢地方検察庁御中)|hironobu\_fumi(園田広宣@たけのこ党・親きのこ派 隠れブルボン信者)
  日時:2021-06-27 17:21/2019/11/23 14:20 URL:
  \url{https://twitter.com/kk\_hirono/status/1409064415052193793} 
  \url{https://twitter.com/hironobu\_fumi/status/1198109067928948739} 
  \textgreater{} この発言が事実なら、かなり立場も危ういと思うが
  その心情に至るまでの経緯が分からないので、なんとも・・・。しかし裁判官がここまで感情を外に出して発言するのは珍しい・・・。
  \url{https://t.co/0ousMaJCWU} 
\end{itemize}

〉〉〉 kk\_hironoのリツイート 〉〉〉

\begin{itemize}
\tightlist
\item
  RT
  kk\_hirono(刑事告発・非常上告_金沢地方検察庁御中)|wakateben(若手弁)
  日時:2021-06-27 17:22/2019/11/23 12:38 URL:
  \url{https://twitter.com/kk\_hirono/status/1409064495251558404} 
  \url{https://twitter.com/wakateben/status/1198083258790989826} 
  \textgreater{}
  ある意味で、世間一般の素朴な感情に沿った対応といえそうですね。
  しかし、司法権を担う裁判官がこのような言動をするというのは、憲法や刑訴法のルールを守り適正にジャッジするという自身の職責を放棄したも同然では・・・
  \url{https://t.co/tlxeT6OfIf} 
\end{itemize}

〉〉〉 kk\_hironoのリツイート 〉〉〉

\begin{itemize}
\tightlist
\item
  RT
  kk\_hirono(刑事告発・非常上告_金沢地方検察庁御中)|gs25920855(GS)
  日時:2021-06-27 17:22/2019/11/23 11:54 URL:
  \url{https://twitter.com/kk\_hirono/status/1409064587815571457} 
  \url{https://twitter.com/gs25920855/status/1198072196368781312} 
  \textgreater{}
  裁判長は知っている方だけど、軽率にこのようなことを言う方ではないので、本当に言っていたとしたら、いろいろと事情があるのではないかと思う。裁判員の方からそのような意見が噴出したけれど、裁判員の方が言うと、黙秘権への配慮ができないと感じて、裁判長が言葉を選んで代わりに言ったとか。
  \url{https://t.co/RIy5A1VpNs} 
\end{itemize}

〉〉〉 kk\_hironoのリツイート 〉〉〉

\begin{itemize}
\tightlist
\item
  RT kk\_hirono(刑事告発・非常上告_金沢地方検察庁御中)|mofjd(モフ)
  日時:2021-06-27 17:22/2019/11/23 11:25 URL:
  \url{https://twitter.com/kk\_hirono/status/1409064637530705921} 
  \url{https://twitter.com/mofjd/status/1198064979703820288} 
  \textgreater{} この人は裁判員にもおかしな影響与えそう。
  \url{https://t.co/6unFxu2eqW} 
\end{itemize}

〉〉〉 kk\_hironoのリツイート 〉〉〉

\begin{itemize}
\tightlist
\item
  RT
  kk\_hirono(刑事告発・非常上告_金沢地方検察庁御中)|sugisima\_smile(弁護士 杉島健文)
  日時:2021-06-27 17:22/2019/11/23 10:42 URL:
  \url{https://twitter.com/kk\_hirono/status/1409064679029104642} 
  \url{https://twitter.com/sugisima\_smile/status/1198054225193750529} 
  \textgreater{} 黙秘権行使を邪魔する裁判官、、、
  \url{https://t.co/WnMCp0BG0u} 
\end{itemize}

〉〉〉 kk\_hironoのリツイート 〉〉〉

\begin{itemize}
\item
  RT
  kk\_hirono(刑事告発・非常上告_金沢地方検察庁御中)|komatta0215(【喪2】たまにくまみたい)
  日時:2021-06-27 17:23/2019/11/23 10:14 URL:
  \url{https://twitter.com/kk\_hirono/status/1409064732330369032} 
  \url{https://twitter.com/komatta0215/status/1198047069862187008} 
  \textgreater{} え、判決前にお前有罪って裁判長が言ったの!?
  \url{https://t.co/ErzC9i9H6Q} 
\item
  〉〉〉 アカウント(@O59K2dPQH59QEJx)は,@kk\_hironoをブロックしています。リツイートできませんでした。
  〉〉〉 ¥\n ¥\n \url{https://t.co/PdpmkjZ8QB} 
\end{itemize}

※ @kk\_hironoのアカウントがブロックされ,リツイートに失敗したツイート

\begin{itemize}
\tightlist
\item
  TW O59K2dPQH59QEJx(ピピピーッ) 日時:2019/11/23 10:13:58 URL:
  \url{https://twitter.com/O59K2dPQH59QEJx/status/1198047002082213888} 
  \textgreater{} ワイならその場で異議を出す。\\
  \textgreater{} 却下されマスコミなどで叩かれるだろうけどね。
  \url{https://t.co/89zLzX9PF5} 
\end{itemize}

 深澤諭史弁護士のツイートを引用し疑問を呈するツイートは、実質、「弁護士さんの頭の中はよく判らん」という内容のツイート1件ぐらいで、かなり積極的に深澤諭史弁護士のツイートを肯定的に評価し、同調すると思われるツイートが大勢を締めていました。

 この新潟市西区小2女児殺害事件については、最近になって知った情報もあるのですが、まだまだ知られていない情報もありそうです。弁護士の姿がテレビの法廷に出てこなかったのも奇怪に思える現象となっていました。顔を出せというわけではなく、主張とのバランスが著しく欠けていると思えたことです。

 検察の求刑が死刑で、一審の判決が無期懲役となっていましたが、電車に遺体を衝突させた事情は相当悪いとは思うものの、犯行の発覚を免れるための突発的な殺意で死亡させたもので、他の同種事件と比較しても有期懲役の余地があったように思われる刑事裁判でした。

 弁護士が被告人の立場を悪くさせたとも思えるのですが、世論の反発を招来した弁護士祭り、弁護士踊りの弁護士音頭とも思えるすこぶる面妖な現象でした。弁護士の宴、弁護士広場の儀式とも思えるもので、それに共通したものを(^ω^#)というマークで締めた深澤諭史弁護士のツイートに感じました。

 かなり前に始まると思いますが、「宴」という言葉を意識するようになったのも、深澤諭史弁護士のツイートがきっかけとなっていました。

\begin{itemize}
\item
  2021年06月27日17時39分の登録:
  REGEXP:''宴''/深澤諭史(@fukazawas)の検索(2012-11-16〜2019-10-22/2021年06月27日17時39分の記録23件)
  \url{https://kk2020-09.blogspot.com/2021/06/regexpfukazawas2012-11-162019-10.html} 
\item
  (14/23) TW fukazawas(深澤諭史) 日時:2017-09-12 12:44:00 +0900
  URL:
  \url{https://twitter.com/fukazawas/status/907449802702458880\textgreater} {}
  オープンだっ・・!\textgreater{} もうすぐ始まる・・・!\textgreater{}
  今年の受験生を招き入れての共演の宴・・!\textgreater{}
  合格の扉が開く・・・!\textgreater{}
  ククククク・・・・・。\textgreater{} \#司法黙示録
\item
  (17/23) TW fukazawas(深澤諭史) 日時:2018-07-17 17:53:00 +0900
  URL:
  \url{https://twitter.com/fukazawas/status/1019142955519508480\textgreater} {}
  うまー(・∀・)\textgreater{} 非弁ハンターの宴だお(^ω^)
  \url{https://t.co/8KPNVAJESm} 
\item
  (20/23) TW fukazawas(深澤諭史) 日時:2019-08-17 11:12:00 +0900
  URL:
  \url{https://twitter.com/fukazawas/status/1162547752204558336\textgreater} {}
  昔「お前ら最前線で餓死,俺は安全な司令部で芸者読んで宴会。お前ら餓死で俺芸者遊び」\textgreater{}
  今「お前ら成仏,俺モリハマ」 \#司法改革インパール
\item
  〈〈〈 2021/06/27 17:57:28 Linux Emacs: 〈〈〈
\end{itemize}

\hypertarget{ux88c1ux5224ux54e1ux5236ux5ea6ux3092ux76aeux8089ux308bux6df1ux6fa4ux8aedux53f2ux5f01ux8b77ux58ebux81eaux8eabux306bux3088ux308bux904eux53bbux306eux30c4ux30a4ux30fcux30c8ux306eux30eaux30c4ux30a4ux30fcux30c8ux304bux3089ux518dux767aux898bux306bux81f3ux3063ux305fux65b0ux6f5fux5973ux5150ux6bbaux5bb3ux4e8bux4ef6ux306bux304aux3051ux308bux9ed9ux79d8ux6a29ux3068ux3046ux306eux5b57ux3068ux306eux30b3ux30e9ux30dcux306bux3088ux308bux30d2ux30e3ux30c3ux30cfux30fcux306eux30c4ux30a4ux30fcux30c87}{%
\paragraph{裁判員制度を皮肉る深澤諭史弁護士自身による過去のツイートのリツイートから再発見に至った、新潟女児殺害事件における黙秘権とうの字とのコラボによるヒャッハーのツイート(7)}\label{ux88c1ux5224ux54e1ux5236ux5ea6ux3092ux76aeux8089ux308bux6df1ux6fa4ux8aedux53f2ux5f01ux8b77ux58ebux81eaux8eabux306bux3088ux308bux904eux53bbux306eux30c4ux30a4ux30fcux30c8ux306eux30eaux30c4ux30a4ux30fcux30c8ux304bux3089ux518dux767aux898bux306bux81f3ux3063ux305fux65b0ux6f5fux5973ux5150ux6bbaux5bb3ux4e8bux4ef6ux306bux304aux3051ux308bux9ed9ux79d8ux6a29ux3068ux3046ux306eux5b57ux3068ux306eux30b3ux30e9ux30dcux306bux3088ux308bux30d2ux30e3ux30c3ux30cfux30fcux306eux30c4ux30a4ux30fcux30c87}}

\begin{itemize}
\tightlist
\item
  〉〉〉 Linux Emacs: 2021/06/28 10:48:07 〉〉〉
\end{itemize}

:CATEGORIES: @kanazawabengosi \#金沢弁護士会 @JFBAsns
日本弁護士連合会(日弁連) \#法務省 @MOJ\_HOUMU \#深澤諭史弁護士
\#うの字 \#黙秘権

 昨夜は憶えているだけで3つ気になるあるいは考えさせられる内容の記事を読んでいました。もともと本件告発事件は、再審請求という流れを辿っており、再審請求については調べたり、読んだ記事あるいは情報の数も多く私の中で蓄積されています。裁判のやり直しといえば通常は再審請求になります。

 本稿のエントリーの作成を始めたタイミングで、次のまとめ記事の作成に取り掛かりました。Twitter検索から始めています。

\begin{itemize}
\tightlist
\item
  2021年06月28日11時13分の登録:
  REGEXP:''黙秘''/???(@un\_co\_the2nd)の検索(2016-03-11〜2021-06-13/2021年06月28日11時13分の記録45件)
  \url{https://kk2020-09.blogspot.com/2021/06/regexpuncothe2nd2016-03-112021-06.html} 
\end{itemize}

 のちほどご紹介したいと思いますが、予想以上の内容がありました。弁護士の生態について調査する資料です。

 次に奉納\さらば弁護士鉄道・泥棒神社の物語(@hirono\_hideki)のタイムラインから昨夜の流れをポイントとしておさえておきたいと思います。

〉〉〉 kk\_hironoのリツイート 〉〉〉

\begin{itemize}
\tightlist
\item
  RT
  kk\_hirono(刑事告発・非常上告_金沢地方検察庁御中)|hirono\_hideki(奉納\さらば弁護士鉄道・泥棒神社の物語)
  日時:2021-06-28 11:24/2021/06/28 06:47 URL:
  \url{https://twitter.com/kk\_hirono/status/1409336940059533312} 
  \url{https://twitter.com/hirono\_hideki/status/1409267127618473984} 
  \textgreater{} 事故被害者の賠償金約1300万円を着服か
  東京の弁護士を逮捕(TBS系(JNN)) - Yahoo!ニュース
  \url{https://t.co/6Wl2kmZbSN} 
  受け取る金額が少ないと感じた男性が保険会社に問い合わせ、犯行が発覚しました。永井容疑者は容疑を認め、「競馬につぎこんで借金がかさんだ」と供述しているということです
\end{itemize}

〉〉〉 kk\_hironoのリツイート 〉〉〉

\begin{itemize}
\tightlist
\item
  RT
  kk\_hirono(刑事告発・非常上告_金沢地方検察庁御中)|hirono\_hideki(奉納\さらば弁護士鉄道・泥棒神社の物語)
  日時:2021-06-28 11:24/2021/06/28 06:36 URL:
  \url{https://twitter.com/kk\_hirono/status/1409336996695212033} 
  \url{https://twitter.com/hirono\_hideki/status/1409264314452561923} 
  \textgreater{}
  冤罪は人道問題・・・宗教者も協力を 鴨志田祐美弁護士|文化時報社|note
  \url{https://t.co/pOc0U7BXBA} 
  鹿児島で弁護士登録し、今年4月に京都で登録替えした。早稲田リーガルコモンズ法律事務所京都オフィス所属。「落ち着いたら京都でもお寺巡りを始めたい」という。
\end{itemize}

〉〉〉 kk\_hironoのリツイート 〉〉〉

\begin{itemize}
\tightlist
\item
  RT
  kk\_hirono(刑事告発・非常上告_金沢地方検察庁御中)|hirono\_hideki(奉納\さらば弁護士鉄道・泥棒神社の物語)
  日時:2021-06-28 11:26/2021/06/27 22:44 URL:
  \url{https://twitter.com/kk\_hirono/status/1409337342146400256} 
  \url{https://twitter.com/hirono\_hideki/status/1409145489610612737} 
  \textgreater{} ▶
  ブロックされたツイート%thermalpaper00(感熱紙(疑似太陽炉))%2021/06/27
  21:09:39% \url{https://t.co/eUHjV5GTLS}  \textgreater{}
  自称リベラルな人たちは犯罪を擁護しているというか、身内の犯罪に「だけ」異常に甘いという印象だよなあ。程度の低い「造反有理」な。
\end{itemize}

〉〉〉 kk\_hironoのリツイート 〉〉〉

\begin{itemize}
\tightlist
\item
  RT
  kk\_hirono(刑事告発・非常上告_金沢地方検察庁御中)|hirono\_hideki(奉納\さらば弁護士鉄道・泥棒神社の物語)
  日時:2021-06-28 11:26/2021/06/27 22:38 URL:
  \url{https://twitter.com/kk\_hirono/status/1409337383447715840} 
  \url{https://twitter.com/hirono\_hideki/status/1409144201070080003} 
  \textgreater{}
  (5ページ目)「死を恐れるのは人間の本能です」10年前、立花隆が``最後のゼミ生''に伝えていたメッセージ
  \textbar{} 文春オンライン \url{https://t.co/tBqjb2nPfn} 
  こういうアドバイスなら、今の若い連中にしておけるかなと考えたのが、70歳になった今、この場を設けた理由です。
\end{itemize}

〉〉〉 kk\_hironoのリツイート 〉〉〉

\begin{itemize}
\tightlist
\item
  RT
  kk\_hirono(刑事告発・非常上告_金沢地方検察庁御中)|hirono\_hideki(奉納\さらば弁護士鉄道・泥棒神社の物語)
  日時:2021-06-28 11:27/2021/06/27 21:36 URL:
  \url{https://twitter.com/kk\_hirono/status/1409337522434445317} 
  \url{https://twitter.com/hirono\_hideki/status/1409128463433306116} 
  \textgreater{}
  「数年以内に君たちは人生最大の失敗をする」立花隆が``6時間の最終講義''で東大生に語っていたこと
  / Twitter \url{https://t.co/rHskpzbBWA} 
\end{itemize}

〉〉〉 kk\_hironoのリツイート 〉〉〉

\begin{itemize}
\tightlist
\item
  RT
  kk\_hirono(刑事告発・非常上告_金沢地方検察庁御中)|hirono\_hideki(奉納\さらば弁護士鉄道・泥棒神社の物語)
  日時:2021-06-28 11:27/2021/06/27 21:31 URL:
  \url{https://twitter.com/kk\_hirono/status/1409337556123017221} 
  \url{https://twitter.com/hirono\_hideki/status/1409127210603057155} 
  \textgreater{} 北海道新聞記者を釈放 旭川医科大への侵入容疑で逮捕 -
  弁護士落合洋司(東京弁護士会)の日々是好日 \url{https://t.co/9vcwm42LmO} 
\end{itemize}

〉〉〉 kk\_hironoのリツイート 〉〉〉

\begin{itemize}
\tightlist
\item
  RT
  kk\_hirono(刑事告発・非常上告_金沢地方検察庁御中)|chikumashobo(筑摩書房)
  日時:2021-06-28 11:27/2021/06/26 09:13 URL:
  \url{https://twitter.com/kk\_hirono/status/1409337646707474433} 
  \url{https://twitter.com/chikumashobo/status/1408579174495752196} 
  \textgreater{}
  「私も調べてゆくうちに、一三〇人の子供たちが一二八四年六月二六日にハーメルンの町で行方不明になった、ということが歴史的事実であることを発見し・・・」
  ~ 737年前の今日6/26に一体何が⁉︎ まるで推理小説!
  伝説化した未解決事件の謎を解く歴史学の名著。解説:石牟礼道子
  \url{https://t.co/I4gPdCuK8i} 
\end{itemize}

〉〉〉 kk\_hironoのリツイート 〉〉〉

\begin{itemize}
\tightlist
\item
  RT
  kk\_hirono(刑事告発・非常上告_金沢地方検察庁御中)|KBAsns(京都弁護士会 広報委員会)
  日時:2021-06-28 11:27/2021/06/26 14:03 URL:
  \url{https://twitter.com/kk\_hirono/status/1409337686062624780} 
  \url{https://twitter.com/KBAsns/status/1408652204014071812} 
  \textgreater{} 【当番弁護士制度】
  逮捕されている人に弁護士が無料で面会に駆けつける「当番弁護士制度」があります。
  お申込みは 075-212-0010(ふいにトーバン) までお電話下さい(24時間受付)
  \url{https://t.co/IqrGvAn51L}  \#京都弁護士会 \#刑事弁護 \#当番弁護
\end{itemize}

〉〉〉 kk\_hironoのリツイート 〉〉〉

\begin{itemize}
\tightlist
\item
  RT
  kk\_hirono(刑事告発・非常上告_金沢地方検察庁御中)|hirono\_hideki(奉納\さらば弁護士鉄道・泥棒神社の物語)
  日時:2021-06-28 11:28/2021/06/27 21:18 URL:
  \url{https://twitter.com/kk\_hirono/status/1409337765410480129} 
  \url{https://twitter.com/hirono\_hideki/status/1409123977096617990} 
  \textgreater{} 2021年06月27日21時18分の実行記録:
  twitterAPI-search-lawList-mydql-add.rb
  ``このトンネルでトラックにはねられました。'' ツイート数:3/2492
  リツイート数:2/2492 トータル:164
  ``このトンネルでトラックにはねられました。''の該当: hirono\_hideki
  0/1件 kk\_hirono 0/0件 s\_hirono 0/0件
\end{itemize}

〉〉〉 kk\_hironoのリツイート 〉〉〉

\begin{itemize}
\tightlist
\item
  RT
  kk\_hirono(刑事告発・非常上告_金沢地方検察庁御中)|hirono\_hideki(奉納\さらば弁護士鉄道・泥棒神社の物語)
  日時:2021-06-28 11:28/2021/06/27 21:15 URL:
  \url{https://twitter.com/kk\_hirono/status/1409337775791349763} 
  \url{https://twitter.com/hirono\_hideki/status/1409123306649710595} 
  \textgreater{} 2021年06月27日21時15分の実行記録:
  twitterAPI-search-lawList-mydql-add.rb ``トンネルではねられ死亡''
  ツイート数:2/2492 リツイート数:0/2492 トータル:111
  ``トンネルではねられ死亡''の該当: hirono\_hideki 2/0件 kk\_hirono
  0/0件 s\_hirono 0/0件
\end{itemize}

〉〉〉 kk\_hironoのリツイート 〉〉〉

\begin{itemize}
\tightlist
\item
  RT
  kk\_hirono(刑事告発・非常上告_金沢地方検察庁御中)|hirono\_hideki(奉納\さらば弁護士鉄道・泥棒神社の物語)
  日時:2021-06-28 11:28/2021/06/27 21:13 URL:
  \url{https://twitter.com/kk\_hirono/status/1409337802219671554} 
  \url{https://twitter.com/hirono\_hideki/status/1409122769271345160} 
  \textgreater{} 電動車いすの女性 トンネルではねられ死亡
  事故の背景には数センチの段差が|NHK事件記者取材note
  \url{https://t.co/7mMvfLp3RG}  2021年6月18日事故
\end{itemize}

〉〉〉 kk\_hironoのリツイート 〉〉〉

\begin{itemize}
\tightlist
\item
  RT
  kk\_hirono(刑事告発・非常上告_金沢地方検察庁御中)|hirono\_hideki(奉納\さらば弁護士鉄道・泥棒神社の物語)
  日時:2021-06-28 11:28/2021/06/27 21:03 URL:
  \url{https://twitter.com/kk\_hirono/status/1409337822373322754} 
  \url{https://twitter.com/hirono\_hideki/status/1409120311417528321} 
  \textgreater{} 電動車いすの女性 トンネルではねられ死亡
  事故の背景には数センチの段差が|NHK事件記者取材note
  \url{https://t.co/7mMvfLp3RG} 
  事故の背景にあったのは、わずか数センチの段差です。 (金沢放送局
  松葉翼)
\end{itemize}

〉〉〉 kk\_hironoのリツイート 〉〉〉

\begin{itemize}
\tightlist
\item
  RT
  kk\_hirono(刑事告発・非常上告_金沢地方検察庁御中)|nhk\_news(NHKニュース)
  日時:2021-06-28 11:28/2021/06/26 15:30 URL:
  \url{https://twitter.com/kk\_hirono/status/1409337848914800644} 
  \url{https://twitter.com/nhk\_news/status/1408673882710724608} 
  \textgreater{}
  40年苦楽を共にした妻はこのトンネルでトラックにはねられました。
  ``どうして歩道を通らなかったの''
  そう思ったあなたに聞いてほしい、夫の言葉があります。
  \url{https://t.co/59ebm6XErQ} 
\end{itemize}

〉〉〉 kk\_hironoのリツイート 〉〉〉

\begin{itemize}
\tightlist
\item
  RT
  kk\_hirono(刑事告発・非常上告_金沢地方検察庁御中)|amneris84(Shoko
  Egawa) 日時:2021-06-28 11:28/2021/06/27 07:09 URL:
  \url{https://twitter.com/kk\_hirono/status/1409337902597767170} 
  \url{https://twitter.com/amneris84/status/1408910297755095044} 
  \textgreater{}
  なるほど〜。 元々、菅氏は「勝負=政治的な賭け」を好む政治家で、そんな彼を「決断力がある」とほめそやす人がいる。今、彼は賭けに高揚し、異論は耳に入らない状態では、と →熱血!与良政談:「一か八か」は菅政治の本質=与良正男
  \textbar{} 毎日新聞 \url{https://t.co/Nh7SGRtszJ} 
\end{itemize}

〉〉〉 kk\_hironoのリツイート 〉〉〉

\begin{itemize}
\tightlist
\item
  RT
  kk\_hirono(刑事告発・非常上告_金沢地方検察庁御中)|amneris84(Shoko
  Egawa) 日時:2021-06-28 11:28/2021/06/27 09:49 URL:
  \url{https://twitter.com/kk\_hirono/status/1409337918636781574} 
  \url{https://twitter.com/amneris84/status/1408950567511097348} 
  \textgreater{} 大阪府知事、暴力には屈します、というメッセージ。
  \url{https://t.co/iRsdon3phP} 
\end{itemize}

〉〉〉 kk\_hironoのリツイート 〉〉〉

\begin{itemize}
\tightlist
\item
  RT
  kk\_hirono(刑事告発・非常上告_金沢地方検察庁御中)|1961kumachin(くまちん(弁護士中村元弥))
  日時:2021-06-28 11:28/2021/06/27 12:12 URL:
  \url{https://twitter.com/kk\_hirono/status/1409337978707578881} 
  \url{https://twitter.com/1961kumachin/status/1408986476365242373} 
  \textgreater{} 今日は27年前に裁判官官舎へのテロが行われた日なのだな
  松本サリン事件 \textbar{} NHK放送史(動画・記事)
  \url{https://t.co/GSKYgo9u64} 
\end{itemize}

〉〉〉 kk\_hironoのリツイート 〉〉〉

\begin{itemize}
\tightlist
\item
  RT
  kk\_hirono(刑事告発・非常上告_金沢地方検察庁御中)|mstk\_Horiguchi(ほりぐちです)
  日時:2021-06-28 11:29/2021/06/27 17:05 URL:
  \url{https://twitter.com/kk\_hirono/status/1409338067027038211} 
  \url{https://twitter.com/mstk\_Horiguchi/status/1409060410594975746} 
  \textgreater{}
  取材目的ならば施設側が「立ち入り禁止」と明示していても何ら問題が無いというのが報道側の理屈だそうなので東京五輪でバブル方式により取材制限をしていますが取材目的で記者の方々が立ち入る可能性が高いと言うことです
  選手間で感染拡大したら報道機関の所為だと予め確定しておきますね
  \url{https://t.co/6TwFFK5651} 
\end{itemize}

〉〉〉 kk\_hironoのリツイート 〉〉〉

\begin{itemize}
\tightlist
\item
  RT
  kk\_hirono(刑事告発・非常上告_金沢地方検察庁御中)|hirono\_hideki(奉納\さらば弁護士鉄道・泥棒神社の物語)
  日時:2021-06-28 11:29/2021/06/27 20:09 URL:
  \url{https://twitter.com/kk\_hirono/status/1409338121649487876} 
  \url{https://twitter.com/hirono\_hideki/status/1409106642398969860} 
  \textgreater{} 【石川】ブルーベリー壊滅 能登町特産 大打撃
  マイマイガ幼虫大発生:北陸中日新聞Web \url{https://t.co/0rzGUU7YAb} 
  近くの柳田植物公園の一ヘクタールの摘み取り園では、五月末から手作業で毛虫を取り除いていたところ、六日間で駆除幼虫は計百二十キロにも上った。
\end{itemize}

〉〉〉 kk\_hironoのリツイート 〉〉〉

\begin{itemize}
\tightlist
\item
  RT
  kk\_hirono(刑事告発・非常上告_金沢地方検察庁御中)|hirono\_hideki(奉納\さらば弁護士鉄道・泥棒神社の物語)
  日時:2021-06-28 11:29/2021/06/27 20:04 URL:
  \url{https://twitter.com/kk\_hirono/status/1409338224430915589} 
  \url{https://twitter.com/hirono\_hideki/status/1409105452403597313} 
  \textgreater{}
  業務上横領容疑で弁護士逮捕 依頼人の賠償金600万円着服―警視庁:時事ドットコム
  \url{https://t.co/RIV6LJFAaM} 
  東京弁護士会所属の弁護士永井博也容疑者(46)=渋谷区恵比寿西=を逮捕した。容疑を認め、「借金があり、返済するために競馬で増やそうと思った」と供述しているという。
\end{itemize}

〉〉〉 kk\_hironoのリツイート 〉〉〉

\begin{itemize}
\tightlist
\item
  RT
  kk\_hirono(刑事告発・非常上告_金沢地方検察庁御中)|hirono\_hideki(奉納\さらば弁護士鉄道・泥棒神社の物語)
  日時:2021-06-28 11:30/2021/06/27 19:52 URL:
  \url{https://twitter.com/kk\_hirono/status/1409338314864222209} 
  \url{https://twitter.com/hirono\_hideki/status/1409102310500802561} 
  \textgreater{}
  【フェイクポルノ】ひろゆき「名誉毀損になるんすか?」女性タレントの顔をAVに合成した動画を制作・公開した男2人逮捕・・・どう規制すれば?現行法では限界!?
  \url{https://t.co/deUXGvhxP1} 
  終わり近く、リツイートについても過去に刑事責任に問われたケースがあるとかいう深澤諭史弁護士
\end{itemize}

〉〉〉 kk\_hironoのリツイート 〉〉〉

\begin{itemize}
\tightlist
\item
  RT
  kk\_hirono(刑事告発・非常上告_金沢地方検察庁御中)|hirono\_hideki(奉納\さらば弁護士鉄道・泥棒神社の物語)
  日時:2021-06-28 11:30/2021/06/27 17:47 URL:
  \url{https://twitter.com/kk\_hirono/status/1409338411115159563} 
  \url{https://twitter.com/hirono\_hideki/status/1409070956513959944} 
  \textgreater{} -
  【画像】新潟に旅行したんだが最高だったから写真貼る:キニ速
  \url{https://t.co/EqodyLx2dt} 
\end{itemize}

〉〉〉 kk\_hironoのリツイート 〉〉〉

\begin{itemize}
\tightlist
\item
  RT
  kk\_hirono(刑事告発・非常上告_金沢地方検察庁御中)|hirono\_hideki(奉納\さらば弁護士鉄道・泥棒神社の物語)
  日時:2021-06-28 11:31/2021/06/27 14:34 URL:
  \url{https://twitter.com/kk\_hirono/status/1409338548147212288} 
  \url{https://twitter.com/hirono\_hideki/status/1409022256878002177} 
  \textgreater{} -
  新潟県司法書士会|身近な生活の法律問題は新潟県の司法書士へ
  \url{https://t.co/SoT7jbBejq} 
\end{itemize}

〉〉〉 kk\_hironoのリツイート 〉〉〉

\begin{itemize}
\tightlist
\item
  RT
  kk\_hirono(刑事告発・非常上告_金沢地方検察庁御中)|hirono\_hideki(奉納\さらば弁護士鉄道・泥棒神社の物語)
  日時:2021-06-28 11:31/2021/06/27 14:33 URL:
  \url{https://twitter.com/kk\_hirono/status/1409338556942671873} 
  \url{https://twitter.com/hirono\_hideki/status/1409021949552992256} 
  \textgreater{} - プロサム法務事務所 --
  プロサム法務事務所公式ホームページ \url{https://t.co/yK8oYaSQsl} 
\end{itemize}

〉〉〉 kk\_hironoのリツイート 〉〉〉

\begin{itemize}
\tightlist
\item
  RT
  kk\_hirono(刑事告発・非常上告_金沢地方検察庁御中)|hirono\_hideki(奉納\さらば弁護士鉄道・泥棒神社の物語)
  日時:2021-06-28 11:32/2021/06/27 13:21 URL:
  \url{https://twitter.com/kk\_hirono/status/1409338941690372102} 
  \url{https://twitter.com/hirono\_hideki/status/1409004052369723397} 
  \textgreater{} 2021-06-27\_13:19
  奉納\\#危険生物・弁護士脳汚染除去装置\\#金沢地方検察庁御中\_2020:
  \スラ弁(弁護士大西洋一) @o2441\1974年生まれの同じ年か・・。ダメ。ゼッタイ。依頼人の保険金600万円横領した疑い、弁護士を逮捕:朝日新聞デジタル
  \url{https://t.co/nUD4Of9uEe} 
\end{itemize}

〉〉〉 kk\_hironoのリツイート 〉〉〉

\begin{itemize}
\tightlist
\item
  RT
  kk\_hirono(刑事告発・非常上告_金沢地方検察庁御中)|hirono\_hideki(奉納\さらば弁護士鉄道・泥棒神社の物語)
  日時:2021-06-28 11:32/2021/06/27 13:21 URL:
  \url{https://twitter.com/kk\_hirono/status/1409338993292890113} 
  \url{https://twitter.com/hirono\_hideki/status/1409003920098136066} 
  \textgreater{} 2021-06-27\_12:47
  奉納\\#危険生物・弁護士脳汚染除去装置\\#金沢地方検察庁御中\_2020:
  REGEXP:''阿部泰隆''/データベース登録済みツイートの検索:2010-12-01〜2021-06-27/2021年06月27日12時46分の記録:ユーザ・投稿:26/48件
  \url{https://t.co/mQE4KYWmvN} 
\end{itemize}

 横領で弁護士逮捕のニュースですが、最初は大西洋一弁護士のツイートで知ったことを思い出しました。ニュース動画で被疑者の弁護士の映像がありましたが、薄ら笑いを浮かべているように見える表情で、新潟市西区小2女児殺害事件の被疑者の逮捕時の報道に似たものを感じました。

 ニュースには弁護士が競馬などで借金を作り返済目的で横領したとするものと、弁護士が謝金の返済目的で競馬に横領したお金を使ったとするものがありました。競馬で勝っていれば横領が発覚することもなく他の借金の返済に充当できたということなのかと考えました。

 弁護士の一発屋という性格を強く感じた事件でもあり、薄ら笑いは、最初さばさばとすっきりした表情にも見えていました。ふてぶてしさなのかわからないですが、顔を隠すこともなく堂々としていて、全面的に非を認めている様子ではありました。

 トンネル事故の記事は、京都弁護士会の秋重実弁護士のツイートで知りました。

 秋重実弁護士のツイートではなくリツイートでした。

\begin{itemize}
\item
  RT
  akishigemakoto(MakotoAkishige(civilista))|nhk\_news(NHKニュース)
  日時:2021-06-26 15:34/2021-06-26 15:30 URL:
  \url{https://twitter.com/akishigemakoto/status/1408675067626618885} 
  \url{https://twitter.com/nhk\_news/status/1408673882710724608} 
  \textgreater{}
  40年苦楽を共にした妻はこのトンネルでトラックにはねられました。\\
  \textgreater{}\\
  \textgreater{} ``どうして歩道を通らなかったの''\\
  \textgreater{}\\
  \textgreater{} そう思ったあなたに聞いてほしい、夫の言葉があります。
  \url{https://t.co/59ebm6XErQ} 
\item
  RT
  hirono\_hideki(奉納\さらば弁護士鉄道・泥棒神社の物語)|nhk\_news(NHKニュース)
  日時:2021-06-27 21:02/2021-06-26 15:30 URL:
  \url{https://twitter.com/hirono\_hideki/status/1409120010933473281} 
  \url{https://twitter.com/nhk\_news/status/1408673882710724608} 
\end{itemize}

 私のリツイートが昨夜の21時02分となっていましたが、元のNHKニュースのツイートは前日の26日15時30分で、秋重実弁護士のリツイートが4分後の34分となっています。最初にトンネル内の写真が目に入ったのですが、見たことのある風景だと思いました。

\begin{itemize}
\tightlist
\item
  電動車いすの女性 トンネルではねられ死亡
  事故の背景には数センチの段差が|NHK事件記者取材note
  \url{https://t.co/1alYrftnQo}  (金沢放送局 松葉翼)
\end{itemize}

 日付が違っているように思っていたのですが、やはり記事の日付は6月18日でした。「電動車いす」と「(金沢放送局
松葉翼)」ということで間違いないと確信しましたが、私の住む石川県鳳珠郡能登町宇出津で起きた死亡事故です。

 この「NHK事件記者取材note」というシリーズは、前に2,3記事を読んでいますが、最初が山梨県での死亡事故だったかもしれません。単独記者の密着取材という感じで斬新なものを感じていました。

 地元の事故なのでいくつか報道にはない話も聞いていますが、何よりこの電動車いすの死亡事故が気になったのは、前日に家の中で事故で亡くなった中国人女性の声を聞いていたことです。それも初めてのことで、最初は男性の声なのかと考えていたのですが、人に話すとその女性らしいとわかりました。

\begin{itemize}
\tightlist
\item
  奉納\さらば弁護士鉄道・泥棒神社の物語(@hirono\_hideki)/2020年06月01日
  - Twilog \url{https://t.co/F8lk7Bll6a} 
\end{itemize}

 記事に事故があったという昨年6月1日のTwilogです。ざっと見たところ、事故に関するツイートは見当たりませんでした。能登町の告知放送でした。宇出津第一隧道とかで事故があり、たぶん通行止めの知らせだったと思います。

 下岩屋の方のトンネルをイメージしていたのですが、第一、第二という区別は知りませんでした。珠洲市に向かう宇出津の町の出口のようなトンネルが下岩屋という町内のトンネル(厳密には川原町かも)で、その反対側で金沢方面に向かうトンネルが事故のあったトンネルでした。

〉〉〉 kk\_hironoのリツイート 〉〉〉

\begin{itemize}
\item
  RT
  kk\_hirono(刑事告発・非常上告_金沢地方検察庁御中)|hirono\_hideki(奉納\さらば弁護士鉄道・泥棒神社の物語)
  日時:2021-06-28 12:16/2020/06/01 21:19 URL:
  \url{https://twitter.com/kk\_hirono/status/1409349985791930377} 
  \url{https://twitter.com/hirono\_hideki/status/1267430691903385600} 
  \textgreater{} 2020年06月01日08時51分の登録:
  \モトケン @motoken\_tw\あなたとの議論のための議論は飽きたので、ミュートしますね。
  言うべきことは言ったので。 あとは、他の人にお任せします。
  \url{https://t.co/ZbhiW9KZ4p} 
\item
  2020年06月01日08時51分の登録:
  \モトケン @motoken\_tw\あなたとの議論のための議論は飽きたので、ミュートしますね。
  言うべきことは言ったので。 あとは、他の人にお任せします。
  \url{http://hirono2014sk.blogspot.com/2020/06/motokentw.html} 
\end{itemize}

 モトケンこと矢部善朗弁護士(京都弁護士会)らしい内容のツイートですが、このモトケンこと矢部善朗弁護士が牽引する弁護士鉄道の路線と、私の母親の人生というのは重要な関わりのあるもので、私と母親の関係の原点に近いのが、宇出津の上田町と辺田の浜の間にある事故のあったトンネルになります。

〉〉〉 kk\_hironoのリツイート 〉〉〉

\begin{itemize}
\item
  RT
  kk\_hirono(刑事告発・非常上告_金沢地方検察庁御中)|hirono\_hideki(奉納\さらば弁護士鉄道・泥棒神社の物語)
  日時:2021-06-28 12:34/2020/06/30 23:20 URL:
  \url{https://twitter.com/kk\_hirono/status/1409354440373460995} 
  \url{https://twitter.com/hirono\_hideki/status/1277970221978943495} 
  \textgreater{} 隧道データベース・詳細表示:宇出津第一 隧道(能都町藤波
  ) \url{https://t.co/9oWWWRNtaz}  トンネル名宇出津第一 個所名能都町藤波
  延長(m)138 車道幅員(m)5.5 限界高(m)4.5 竣工年度(和暦)S32
  竣工年度(西暦)1957
\item
  〈〈〈 2021/06/28 13:00:51 Linux Emacs: 〈〈〈
\end{itemize}

\hypertarget{ux88c1ux5224ux54e1ux5236ux5ea6ux3092ux76aeux8089ux308bux6df1ux6fa4ux8aedux53f2ux5f01ux8b77ux58ebux81eaux8eabux306bux3088ux308bux904eux53bbux306eux30c4ux30a4ux30fcux30c8ux306eux30eaux30c4ux30a4ux30fcux30c8ux304bux3089ux518dux767aux898bux306bux81f3ux3063ux305fux65b0ux6f5fux5973ux5150ux6bbaux5bb3ux4e8bux4ef6ux306bux304aux3051ux308bux9ed9ux79d8ux6a29ux3068ux3046ux306eux5b57ux3068ux306eux30b3ux30e9ux30dcux306bux3088ux308bux30d2ux30e3ux30c3ux30cfux30fcux306eux30c4ux30a4ux30fcux30c88}{%
\paragraph{裁判員制度を皮肉る深澤諭史弁護士自身による過去のツイートのリツイートから再発見に至った、新潟女児殺害事件における黙秘権とうの字とのコラボによるヒャッハーのツイート(8)}\label{ux88c1ux5224ux54e1ux5236ux5ea6ux3092ux76aeux8089ux308bux6df1ux6fa4ux8aedux53f2ux5f01ux8b77ux58ebux81eaux8eabux306bux3088ux308bux904eux53bbux306eux30c4ux30a4ux30fcux30c8ux306eux30eaux30c4ux30a4ux30fcux30c8ux304bux3089ux518dux767aux898bux306bux81f3ux3063ux305fux65b0ux6f5fux5973ux5150ux6bbaux5bb3ux4e8bux4ef6ux306bux304aux3051ux308bux9ed9ux79d8ux6a29ux3068ux3046ux306eux5b57ux3068ux306eux30b3ux30e9ux30dcux306bux3088ux308bux30d2ux30e3ux30c3ux30cfux30fcux306eux30c4ux30a4ux30fcux30c88}}

\begin{itemize}
\tightlist
\item
  〉〉〉 Linux Emacs: 2021/06/29 11:10:40 〉〉〉
\end{itemize}

:CATEGORIES: @kanazawabengosi \#金沢弁護士会 @JFBAsns
日本弁護士連合会(日弁連) \#法務省 @MOJ\_HOUMU \#深澤諭史弁護士
\#うの字 \#坂本正幸弁護士 \#黙秘権 \#裁判員制度

 「2021/06/28 13:00:51 Linux
Emacs:」という時刻でいったん締めていましたが、再開することはなくエントリーを切り替えシリーズ8となっています。テーマ性を重視しているので、もうしばらく続きます。

\begin{itemize}
\tightlist
\item
  1436:2021-06-29\_11:09:04 \#告発状 \#\#\#\#
  裁判員制度を皮肉る深澤諭史弁護士自身による過去のツイートのリツイートから再発見に至った、新潟女児殺害事件における黙秘権とうの字とのコラボによるヒャッハーのツイート(7)
  \url{https://hirono-hideki.hatenadiary.jp/entry/2021/06/29/110902} 
\end{itemize}

 Amazonからの通知で午前中に配達があると思っていたのですが、午後の遅い時間になりそうだと出かける支度にかかるタイミングで、前日に注文した本の配達がありました。

 パラパラと内容に目を通してすぐに出掛けました。能登町上町の母親のいる病院に靴を届けに行く用事で、その前に天坂のラーメン屋で味噌バターラーメンを食べました。ちょうど店に入ったタイミングで、テレビに神社の賽銭泥棒が捕まる防犯カメラの映像が流れていました。

 これまでの賽銭泥棒のイメージを根底から覆すもので、白昼に堂々と伸縮する棒を使って盗むという手口でした。これは初めてみるニュースで、まだ調べてもいないのでネットでも情報はみていません。

 次あたりが、同じく白昼、全裸でパトカーを壊した事件で、これはネットで少しニュースを見かけていました。番組はバイキングでしたが、また道路の映像が出てきて、今度は車が歩道を走行した、近くに小学生の列がいて危険だったという話で、警察に通報したで締めくくられていたように思います。

 映像はあまり良く見ていなかったのですが車で歩道を走行しただけでテレビのニュースになるのかと思っていました。家に戻ってからTwitterに「小学生の列」というトレンドがあって、同じ話題なのかと思ったところ、今度は2人の小学生が心肺停止、他にも重傷者というニュースでした。

\begin{itemize}
\item
  \begin{enumerate}
  \def\labelenumi{(\arabic{enumi})}
  \setcounter{enumi}{7}
  \tightlist
  \item
    埼玉・秩父市のさい銭泥棒、64歳男を逮捕 - YouTube
    \url{https://t.co/s9SafLZurn} 
    神社の境内に設置されたさい銭箱から現金2000円を盗んだ疑いがもたれてます。平容疑者は警察の取り調べに対し「(逮捕されたことや容疑に)納得いかない」と否認しています。
  \end{enumerate}
\item
  \begin{enumerate}
  \def\labelenumi{(\arabic{enumi})}
  \setcounter{enumi}{7}
  \tightlist
  \item
    千葉・八街 小学生5人死傷事故 トラック運転手の自宅など家宅捜索 -
    YouTube \url{https://t.co/045P52nEa5} 
    容疑者の呼気からは基準値以上のアルコールが検出されていて、警察は、過失運転致死傷の疑いで飲酒の状況など事故の経緯を調べる方針です。(29日09:15)
  \end{enumerate}
\item
  \begin{enumerate}
  \def\labelenumi{(\arabic{enumi})}
  \setcounter{enumi}{7}
  \tightlist
  \item
    全裸襲撃も関与か パトカー損壊事件・・・``酒''は否定(2021年6月27日)
    - YouTube \url{https://t.co/MQwJWcYmmU} 
    26日、茨城県で全裸の男が車に立ちふさがり、運転手に暴行を加えるなどしました。パトカーを壊したとして逮捕された男が、事件の直前に車を襲った疑いもあるとみて調べています。
  \end{enumerate}
\end{itemize}

〉〉〉 kk\_hironoのリツイート 〉〉〉

\begin{itemize}
\tightlist
\item
  RT
  kk\_hirono(刑事告発・非常上告_金沢地方検察庁御中)|morimitsu5130(みつき)
  日時:2021-06-29 11:50/2021/05/21 13:28 URL:
  \url{https://twitter.com/kk\_hirono/status/1409705867482603533} 
  \url{https://twitter.com/morimitsu5130/status/1395597326920556550} 
  \textgreater{}
  バイキングつけてたら、視聴者提供のドラレコの映像で歩道を車が走ってるって言うやつあったけどめちゃめちゃ地元でビビった
  \url{https://t.co/ITGMjfsmMV} 
\end{itemize}

 「バイキング 歩道」とTwitterで検索したのですが、それらしいツイートが1つだけあり、テレビの画面を撮影した画像があって、画面の右上に滋賀県大津市とありました。数年前の大津園児死傷事故の当時の映像なのかと思ったのですが、そうではなさそうでした。

 もう一度画像をみると、テレビの画面の左下に「2年前の事故現場のそばで・・・車が歩道を走る危険運転」とありました。昨日、天坂のラーメン店でテレビを見ていたときは、まったく気が付かずにいました。テレビの画面は13時28分となっています。店を出る時間に近かったように思います。

 昨日撮影した写真を確認すると、天坂の店の前で撮影した写真の時刻が13時30分となっていました。店を出た時間というのは全くの勘違いで、店に入った時間が13時30分頃でした。よくみると先程の大津市の映像もツイートの日付が5月21日となっていました。

 Amazonから本が届いた時間も12時50分頃と確認しました。宇出津新港のアルプで靴を買ってきたのも6月23日だったと写真の時刻ですぐにわかりました。夕方遅くに漆原に行ったのが24日、本の返却に行ったときコンセールのとでキリコ祭りの展示を見たのが25日でした。

 HDMIのケーブル接続でパソコンの画面をテレビに映し出すことができると知ったのもその時でした。漠然と考えたことはあったのですが、その日まで調べることがなく、その1時間ほど前に別のことがきっかけで少し調べていたのです。その間、玉串の新札を興能信用金庫で降ろしてから向かっていました。

 Amazonで購入した本ですが、Twitterで検索しても反応が信じがたいほど乏しく、神がかった現象のように思えていました。とても簡単に手元に来たのですが、自分だけが見えている現象なのかと多少不安をおぼえたほどです。キーワードは簡単で「再審申立書」です。

\begin{itemize}
\item
  2021年06月27日01時08分の登録:
  REGEXP:''再審申立書''/データベース登録済みツイートの検索:2020-05-25〜2021-06-27/2021年06月27日01時07分の記録:ユーザ・投稿:4/8件
  \url{https://kk2020-09.blogspot.com/2021/06/regexp2020-05-252021-06-2720210627010748.html} 
\item
  2021年06月27日23時43分の登録:
  REGEXP:''再審申立書''/データベース登録済みツイートの検索:2021-06-27〜2021-06-27/2021年06月27日23時43分の記録:ユーザ・投稿:2/13件
  \url{https://kk2020-09.blogspot.com/2021/06/regexp2021-06-272021-06.html} 
\item
  2021年06月28日19時23分の登録:
  REGEXP:''再審申立書''/データベース登録済みツイートの検索:2021-06-27〜2021-06-28/2021年06月28日19時22分の記録:ユーザ・投稿:2/14件
  \url{https://kk2020-09.blogspot.com/2021/06/regexp2021-06-272021-06\_28.html} 
\item
  (1/8) TW @MichikoKameishi(弁護士 亀石倫子) 日時: 2020-05-25
  20:15:00 +0900 URL:
  \url{https://twitter.com/MichikoKameishi/status/1264877837032386561\textgreater} {}
  大崎事件第4次再審請求クラウドファンディングが900万円を突破・・・!ネクストゴール1千万まであと少しです。本当にありがとうございます??‍♀️\textgreater{}
  近日、再審申立書全文を公開予定。大崎事件を徹底解説するオンライン配信イベントもやります・・・
  \url{https://t.co/BoXFtAx75J} 
\end{itemize}

 余り記憶になかった亀石倫子弁護士のツイートですが、大崎事件の再審申立書というのは聞いた覚えがなく、他のツイートの数も乏しいので話題にはなっていなかったようです。いちおう調べて確認をしておきます。

\begin{itemize}
\tightlist
\item
  大崎事件 再審申立書 - Google 検索 \url{https://t.co/rb1cx5O2Df} 
\end{itemize}

 検索結果の1ページ目はほとんどが「異議申立書」となっているのですが、第4次再審申立報告集会〜扉を開いて無罪を掴め!〜、というのが昨年3月30日の日付で見えます。

 亀石倫子弁護士の2020年5月25日のツイートでは、クラウドファンディングの募集も大詰めで、ネクストゴール1千万円まであと少し、として、画像には弁護士らの集合写真とともに「ご支援額9,000,000円突破!」とあります。

 弁護士のクラウドファンディングというのは、銀行の通帳の表紙だけの写真で、使いみちも明記したものを見かけたことがなく、かねて疑問に思っていました。多少話が違うと揉めたツイートぐらいで、大きな疑問の声、追求というのは見かけたことがなく、不思議な弁護士現象に思っていました。

\begin{lstlisting}
2021年06月29日12時58分の実行記録: twitterAPI-search-lawList-mydql-add.rb "再審申立書" ツイート数:19/2492 リツイート数:6/2492 トータル:115
"再審申立書"の該当: hirono_hideki 8/2件 kk_hirono 11/4件 s_hirono 0/0件
\end{lstlisting}

 あらためてTwitterAPIの検索を行いましたが、トータルで115という結果です。

 今思い出したので忘れないうちに記録しておきたいのですが、昨日は、うの字と坂本正幸弁護士の間でツイートのやりとりを確認しました。時間のことは夕方の遅い時間か、夜の早い時間だったように思うのですが、記録を残しているはずです。

\begin{itemize}
\tightlist
\item
  2021年06月28日10時26分の登録:
  \7286 @jmjhjmwtad\その手の人らは過激派になったものと認識すべきやな。過去の犯罪被害は自分の攻撃性を正当化せんよ
  \url{https://kk2020-09.blogspot.com/2021/06/7286jmjhjmwtad\_91.html} 
\item
  2021年06月28日11時13分の登録:
  REGEXP:''黙秘''/???(@un\_co\_the2nd)の検索(2016-03-11〜2021-06-13/2021年06月28日11時13分の記録45件)
  \url{https://kk2020-09.blogspot.com/2021/06/regexpuncothe2nd2016-03-112021-06.html} 
\item
  2021年06月28日18時43分の登録: \??? @un\_co\_the2nd\返信先:
  @sakamotomasayukさんうんこ製造パイプでいかがでしょう
  \url{https://kk2020-09.blogspot.com/2021/06/uncothe2nd-sakamotomasayuk.html} 
\item
  2021年06月28日18時43分の登録: \坂本正幸 @sakamotomasayuk\返信先:
  @un\_co\_the2ndさん糞袋と言いましてですね
  \url{https://kk2020-09.blogspot.com/2021/06/sakamotomasayuk-uncothe2nd.html} 
\item
  2021年06月28日18時47分の登録:
  \川口創 @kahajime\弁護士つく、ただそれだけで、そもそも判断基準が大きく変わるんです。
  \url{https://kk2020-09.blogspot.com/2021/06/kahajime.html} 
\end{itemize}

 時刻は13時09分です。視界に入った状態でスマホの画面が明るくなり、ポケモンの通知だったのですが、そのままポケモンを開くと、ずっとオフにしていた効果音や音楽がオンになっていました。最近はアップデートもないはずですが、これまでにないことです。

\begin{itemize}
\item
  TW un\_co\_the2nd(🔥💩🔥) 日時: 2021/06/28 16:36:56 URL:
  \url{https://twitter.com/un\_co\_the2nd/status/1409415504431702018} 
  \textgreater{} もともとヒトはキモい生き物だという自覚が足らんのだよ
\item
  TW sakamotomasayuk(坂本正幸) 日時: 2021/06/28 16:39:10 URL:
  \url{https://twitter.com/sakamotomasayuk/status/1409416064165769218} 
  \textgreater{} @un\_co\_the2nd 糞袋と言いましてですね
\end{itemize}

 「トイレット博士」という漫画のことを思い出すのですが、週刊少年ジャンプの連載で、まだ辺田の浜の家に住んでいる頃によく見た漫画として記憶にあります。週刊少年チャンピオンの「恐怖新聞」や「魔太郎がくる!!」も同じ頃のこととして記憶にあります。

\begin{itemize}
\tightlist
\item
  トイレット博士 - Wikipedia \url{https://t.co/bUlQKTL8lE} 
  『トイレット博士』(トイレットはかせ)は、とりいかずよしにより1970年から1977年まで『週刊少年ジャンプ』誌上に連載されたギャグ漫画作品。
\end{itemize}

 「静子(スナミの妻)、一城(長男)、双葉(長女)」とありました。別の漫画だったのかと少し不安になったのですが、双葉という女の子が生まれたあたりは、宇出津に引っ越したあと、話題になっていたと記憶にあります。漫画の内容が話題になることは余りなかったと思うので、よく憶えています。

 昭和45年から連載が始まっていたというのは驚きました。昭和52年までということに着目し、ちょうど同じ週刊少年ジャンプで「東大一直線」の連載が始まるのと同時期で、1年ほどかぶりがあったのかとも思えてきました。昨夜になるのかツイートの動画を少しみていた作者です。

〉〉〉 kk\_hironoのリツイート 〉〉〉

\begin{itemize}
\tightlist
\item
  RT
  kk\_hirono(刑事告発・非常上告_金沢地方検察庁御中)|kinjitou2010(🌷🌻金字塔
  ぽんぽこ商事CEO🌺🌻) 日時:2021-06-29 13:29/2021/06/28 19:17 URL:
  \url{https://twitter.com/kk\_hirono/status/1409730765206671364} 
  \url{https://twitter.com/kinjitou2010/status/1409455980572676097} 
  \textgreater{} 【小林よしのり】河野太郎大臣に告ぐ!!
  \url{https://t.co/tXeeXDZ38Y}  @YouTube \url{https://t.co/7EuyAQsJ9g} 
\end{itemize}

 小林よしのり氏は、「東大一直線」の舞台と同じ福岡市の出身となっていたように思うのですが、「わし」という言葉の使い方というか発音などが、昭和50年代の金沢市の若者とよく似ているようで、前から気になっていました。

 福岡へは長距離トラック運転手の仕事でよく行き、一緒に仕事をする運転手もいたのですが、聞いた憶えのない発音です。考えてみると他の地域でも「わし」というのは余り聞いた覚えがないのですが、法クラの弁護士が「ワイ」というのはよく見かけています。うの字もそうだったと思います。

 時刻は15時11分です。14時から歯医者の予約があったのですが、その暫く前に深澤諭史弁護士のタイムラインで、また大きく考えさせられるツイートの発見がありました。1つは奥村徹弁護士とのコラボ調のもので、もう一つはうの字のツイートの連続リツイートでした。

 今回は待合室に先に人がいたのですが、先に診察室に入り、治療が終わるのも早く、家に戻った時刻が14時15分でした。テレビをつけた時間かもしれません。チャンネルがNHKになっていて、長崎市が出ていました。新日本紀行などとあったので、古い番組の再放送かと思います。

 歯医者の待合室に入った頃はテレビでミヤネ屋の放送が始まっていて、千葉県八街市の小学生死傷事故でした。深澤諭史弁護士のタイムラインでみていたうの字のツイートというのも、この事故の報道を批判するものでした。

\begin{itemize}
\item
  RT fukazawas(深澤諭史)|GUv4i6(北白川) 日時:2021-06-29
  11:01/2021-06-29 09:30 URL:
  \url{https://twitter.com/fukazawas/status/1409693527550824453} 
  \url{https://twitter.com/GUv4i6/status/1409670566563303427} 
  \textgreater{} これは・・・w \url{https://t.co/72jdOErbBT} 
\item
  RT fukazawas(深澤諭史)|un\_co\_the2nd(🔥💩🔥) 日時:2021-06-29
  11:21/2021-06-29 09:49 URL:
  \url{https://twitter.com/fukazawas/status/1409698440687153154} 
  \url{https://twitter.com/un\_co\_the2nd/status/1409675352419430406} 
  \textgreater{}
  昨日の飲酒運転事故の直後にマスコミが運転手の両親宅に凸したらしい。そんな知能も知性もないことよくやるな。
\item
  RT fukazawas(深澤諭史)|un\_co\_the2nd(🔥💩🔥) 日時:2021-06-29
  11:21/2021-06-29 09:52 URL:
  \url{https://twitter.com/fukazawas/status/1409698469825048587} 
  \url{https://twitter.com/un\_co\_the2nd/status/1409676000355524611} 
  \textgreater{}
  老親や雇用主がカメラの前で首くくれば満足なのか?ほんとクソだな
\item
  RT fukazawas(深澤諭史)|un\_co\_the2nd(🔥💩🔥) 日時:2021-06-29
  11:21/2021-06-29 10:14 URL:
  \url{https://twitter.com/fukazawas/status/1409698474543648770} 
  \url{https://twitter.com/un\_co\_the2nd/status/1409681602347630600} 
  \textgreater{}
  記者が捕まったら全力無理筋擁護を見た直後なので、はーん社会の木鐸ねぇ?って鼻で笑うとこ
\item
  RT fukazawas(深澤諭史)|un\_co\_the2nd(🔥💩🔥) 日時:2021-06-29
  11:21/2021-06-29 10:50 URL:
  \url{https://twitter.com/fukazawas/status/1409698527987396612} 
  \url{https://twitter.com/un\_co\_the2nd/status/1409690824892317700} 
  \textgreater{}
  そんなもん見てる視聴者も、知性も理性もないんだよ(おこおこぷんぷん
  \url{https://t.co/Oouy7iTY3N} 
\end{itemize}

 深澤諭史弁護士のリツイートは連続して午前11時21分ですが、うの字のツイートは9時49分、9時52分、10時14分、10時50分と不思議にも思える時間の間隔があり、「そんなもん見てる視聴者も、知性も理性もないんだよ」が深澤諭史弁護士のタイムラインでは締めくくりのようになっています。

 そういえば、暫く前からうの字のプロフィールの名前が、3つの絵文字で、前後でウンコマークをはさむ炎となっています。絵文字のコピーをしようとしたのですが右クリックのメニューに画像をコピーとありました。その前にマウスポインターを近づけると、「炎」、「うんち」と出ていました。

 TwitterAPIで取得したテキストも絵文字は変な記号に置き換わっています。普通に文字入力をすればLinuxでも絵文字は打てそうです。Konsoleの端末では絵文字の入力が出来ましたが、Emacsでは変換候補の時点で文字化けになっていました。そのうち対応してくれるかもしれないです。

 歯医者に出かける前、遠くで雷がゴロゴロと鳴っていたのですが、家を出るとちょうど小雨が降り出していました。雷のあとは大雨になることがあったと思うのですが、不思議な天候で、うの字の「そんなもん見てる視聴者も、知性も理性もないんだよ(おこおこぷんぷん」と重なりを感じました。

 テレビを消す前、東京でコロナの感染が増えているという話がありましたが、深澤諭史弁護士やうの字のような弁護士が、神様を怒らせたり、天変地異を誘引しているのではと考えることがあります。あるいは逆に憑依してシグナルを送っているように思えることもあります。

 うの字と深澤諭史弁護士は考えがよく似ているのだとあらためて思ったのですが、深澤諭史弁護士がその本領を発揮したのも熊本地震の報道が契機となっていました。

\begin{lstlisting}
base ❯ d|grep 地震|grep 深澤
\end{lstlisting}

\begin{itemize}
\tightlist
\item
  2018年09月06日10時32分の登録:
  \深澤諭史 @fukazawas\ライオン脱走、井戸に毒・・・・・・。地震関連デマ。「業務妨害罪になりうる」と弁護士
  \url{http://hirono2014sk.blogspot.com/2018/09/fukazawas\_6.html} 
\item
  2019年06月19日22時25分の登録:
  \深澤諭史 @fukazawas\ほんこれ。¥\nというか、時の政権にそんな力があったら、もっと地震おこすタイミングが他に山ほどありますよね。¥\n(*・∀・)国会会期中とか予算
  \url{http://hirono2014sk.blogspot.com/2019/06/fukazawas\_98.html} 
\end{itemize}

 あるかと思ったエントリーは見つからなかったのですが、エントリーとして記録するにはこまめな見出しの設定が必要になります。テーマのまとまりを分散を加速させるという弊害の方が大きいように思えて、今は同じテーマでのシリーズ化を優先させています。

\begin{itemize}
\tightlist
\item
  ./kk\_hirono2021-06-29\_155007.csv:2016-05-24 10:09:40 ``****
  2016年4月14日夜、熊本地震が起こる前の、深澤諭史弁護士や法クラの「天の怒り?地の声?」と考えさせられるツイート''
  \url{https://twitter.com/kk\_hirono/status/734914225886224384} 
\end{itemize}

 記憶になかったですが、気になるものが見つかりました。

\begin{itemize}
\item
  刑事告発・非常上告_金沢地方検察庁御中(@kk\_hirono)/2016年05月24日 -
  Twilog \url{https://t.co/bP4EoQltIx} 
\item
  告発\金沢地方検察庁\最高検察庁\法務省\石川県警察御中
  \url{https://t.co/To1DUbofbN} 
\end{itemize}

 現在は基本的に更新していないブログですが、最後の投稿となっている記事に次のツイートがありました。「想像を絶する知的レベルの自治体もあるんだよ。」とありますが、私が生まれ住む、石川県鳳珠郡能登町のことです。

〉〉〉 kk\_hironoのリツイート 〉〉〉

\begin{itemize}
\tightlist
\item
  RT
  kk\_hirono(刑事告発・非常上告_金沢地方検察庁御中)|aphros67(小動物を愛するしんさん)
  日時:2021-06-29 16:01/2021/05/02 21:55 URL:
  \url{https://twitter.com/kk\_hirono/status/1409769045621051397} 
  \url{https://twitter.com/aphros67/status/1388839497203281920} 
  \textgreater{} 政府自民党、ふるさと創生事業で何も学んでいない(・ω・)
  想像を絶する知的レベルの自治体はあるんだよ。
  /コロナ対策の臨時交付金2500万円使い巨大イカのモニュメント設置 問われるお金の使い方【石川発】
  \textbar{} FNNプライムオンライン \url{https://t.co/PWzRWHRYIl} 
\end{itemize}

 おやっと思ったのですが、告発\市場急配センター殺人未遂事件\金沢地方検察庁・石川県警察御中(@kk\_hirono)でリツイートができました。ずっと前からずっとブロックされているという印象が強かったTwitterアカウントです。

〉〉〉 kk\_hironoのリツイート 〉〉〉

\begin{itemize}
\item
  RT
  kk\_hirono(刑事告発・非常上告_金沢地方検察庁御中)|aphros67(小動物を愛するしんさん)
  日時:2021-06-29 16:03/2021/06/09 12:58 URL:
  \url{https://twitter.com/kk\_hirono/status/1409769528037306369} 
  \url{https://twitter.com/aphros67/status/1402475092341235712} 
  \textgreater{} こっちにぶらさげておこう(・ω・)
  死刑制度違憲説について|小動物を愛するしんさん @aphros67 \#note
  \url{https://t.co/bVofrNNC5M} 
\item
  死刑制度違憲説について|小動物を愛するしんさん|note
  \url{https://t.co/LFCb8J3PyD} 
\end{itemize}

 Twitterのプロフィールにジャーナリストとありますが、死刑の研究をしているような情報がずっと前からありました。ただずいぶん古い内容のページがリンクにありました。このアカウントのnotoの記事をみるのは初めてになるかもしれません。先程最高裁の死刑判決のニュースをミヤネ屋でみたところです。

\begin{itemize}
\tightlist
\item
  〈〈〈 2021/06/29 16:10:25 Linux Emacs: 〈〈〈
\end{itemize}

\hypertarget{ux88c1ux5224ux54e1ux5236ux5ea6ux3092ux76aeux8089ux308bux6df1ux6fa4ux8aedux53f2ux5f01ux8b77ux58ebux81eaux8eabux306bux3088ux308bux904eux53bbux306eux30c4ux30a4ux30fcux30c8ux306eux30eaux30c4ux30a4ux30fcux30c8ux304bux3089ux518dux767aux898bux306bux81f3ux3063ux305fux65b0ux6f5fux5973ux5150ux6bbaux5bb3ux4e8bux4ef6ux306bux304aux3051ux308bux9ed9ux79d8ux6a29ux3068ux3046ux306eux5b57ux3068ux306eux30b3ux30e9ux30dcux306bux3088ux308bux30d2ux30e3ux30c3ux30cfux30fcux306eux30c4ux30a4ux30fcux30c89}{%
\paragraph{裁判員制度を皮肉る深澤諭史弁護士自身による過去のツイートのリツイートから再発見に至った、新潟女児殺害事件における黙秘権とうの字とのコラボによるヒャッハーのツイート(9)}\label{ux88c1ux5224ux54e1ux5236ux5ea6ux3092ux76aeux8089ux308bux6df1ux6fa4ux8aedux53f2ux5f01ux8b77ux58ebux81eaux8eabux306bux3088ux308bux904eux53bbux306eux30c4ux30a4ux30fcux30c8ux306eux30eaux30c4ux30a4ux30fcux30c8ux304bux3089ux518dux767aux898bux306bux81f3ux3063ux305fux65b0ux6f5fux5973ux5150ux6bbaux5bb3ux4e8bux4ef6ux306bux304aux3051ux308bux9ed9ux79d8ux6a29ux3068ux3046ux306eux5b57ux3068ux306eux30b3ux30e9ux30dcux306bux3088ux308bux30d2ux30e3ux30c3ux30cfux30fcux306eux30c4ux30a4ux30fcux30c89}}

\begin{itemize}
\tightlist
\item
  〉〉〉 Linux Atom: 2021-06-30 14:06 〉〉〉
\end{itemize}

:CATEGORIES: @kanazawabengosi \#金沢弁護士会 @JFBAsns
日本弁護士連合会(日弁連) \#法務省 @MOJ\_HOUMU \#深澤諭史弁護士
\#うの字

 「2021/06/29 16:10:25 Linux
Emacs:」からの中断となっていました。昨日も大きな発見があったのですが、その前の日の夕方から記録しておきたいことがあります。白山島に始まります。

\begin{itemize}
\item
  〈〈〈 Linux Atom: 2021-06-30 14:15 〈〈〈
\item
  〉〉〉 Linux Emacs: 2021/06/30 14:17:04 〉〉〉
\end{itemize}

 少しAtomエディタを使ったのですが、動作が重く感じました。日本語変換もワンテンポ遅れが出る感じです。VSCodeやAtomでは絵文字が表示されるのですが、Emacsではされないものが多く、朝はそれについて調べたりしていました。

 白山島から始めたいところですが、深澤諭史弁護士のタイムラインに気になるツイートを見かけていて、過去のツイートの連続したリツイートなのですが、いずれも引用されているのが過去に印象的だった深澤諭史弁護士のツイートです。

 なぜこのタイミングなのかというのが気になるのですが、今回は前に関連の有りそうなツイートが見当たりませんでした。子供の養育費と弁護士報酬に関する内容のツイートで、けっこう挑発的だとは以前から思っていました。

\begin{itemize}
\item
  TW fukazawas(深澤諭史) 日時: 2021-06-29 17:09 URL:
  \url{https://twitter.com/fukazawas/status/1409786036159320064} 
  \textgreater{}
  弁護士会の退会命令を受けても、再登録(再入会)、つまり弁護士資格そのものは失いません。\\
  \textgreater{}
  ですが、入会審査があるので、そこで認められないことがほとんどです。\\
  \textgreater{} 実務上、入会審査は結構厳しいです。
\item
  TW fukazawas(深澤諭史) 日時: 2021-06-29 17:18 URL:
  \url{https://twitter.com/fukazawas/status/1409788254027583496} 
  \textgreater{} @asty\_md
  はい。というか、裁判例で認められたケースもございます。\\
  \textgreater{} もっとも、基本的に、かなり難しいのではないかと。
\item
  RT fukazawas(深澤諭史)|k\_sawmen(泥濘大魔王サイケ)
  日時:2021-06-29 18:42/2019-07-22 14:58 URL:
  \url{https://twitter.com/fukazawas/status/1409809448990777344} 
  \url{https://twitter.com/k\_sawmen/status/1153182405018079232} 
  \textgreater{}
  他は措くとしても、「同性婚」も「夫婦別姓」も、「日常困っていることから距離のある」「政治的なお念仏」に聞こえるのか・・・・・・
  \url{https://t.co/rXfGErmoTp} 
\item
  RT fukazawas(深澤諭史)|kmrysyk(Yasuyuki
  KIMURA(弁護士ときどきランナー)) 日時:2021-06-29 18:43/2021-06-28
  22:42 URL: \url{https://twitter.com/fukazawas/status/1409809684077240325} 
  \url{https://twitter.com/kmrysyk/status/1409507573208735744} 
  \textgreater{}
  今現在、夫婦別姓を切望している当事者の訴えに「早くて10年はかかるでしょうね。」って平然と言い放てるのすごいですね。\\
  \textgreater{} さすが元最高裁判事様。 \url{https://t.co/lG6g1hFFDS} 
\item
  RT fukazawas(深澤諭史)|motaberarenaiyo(過食B) 日時:2021-06-29
  19:08/2021-06-29 19:02 URL:
  \url{https://twitter.com/fukazawas/status/1409816079216484356} 
  \url{https://twitter.com/motaberarenaiyo/status/1409814481622863873} 
  \textgreater{}
  漫画やアニメを攻撃する人と表現の不自由展を攻撃する人は割と同じ目つきをしている。
\item
  RT fukazawas(深澤諭史)|matimura(田丁木寸) 日時:2021-06-29
  22:31/2021-06-29 19:12 URL:
  \url{https://twitter.com/fukazawas/status/1409867116384198656} 
  \url{https://twitter.com/matimura/status/1409817141121937411} 
  \textgreater{}
  東京都がやるべきことは、酒の提供の禁止より先に世界的スポーツ大イベントの自粛じゃね―か?
\item
  RT fukazawas(深澤諭史)|katepanda2(弁護士
  太田啓子 「これからの男の子たちへ」(大月書店)) 日時:2021-06-29
  22:38/2021-06-29 17:08 URL:
  \url{https://twitter.com/fukazawas/status/1409868763386810368} 
  \url{https://twitter.com/katepanda2/status/1409785796610064384} 
  \textgreater{}
  川崎支部での壮絶なボスからイソへのパワハラ事案での重大な懲戒処分が今日出たようですね。\\
  \textgreater{}
  弁護士間のパワハラ、あると思います、、 このような窓口がもうけられているようですので、悩んでいる方がいたらぜひ。
  \url{https://t.co/cxP4WYjOlL} 
\item
  RT fukazawas(深澤諭史)|y00black(ユンダ) 日時:2021-06-29
  22:41/2021-06-29 11:56 URL:
  \url{https://twitter.com/fukazawas/status/1409869569385189389} 
  \url{https://twitter.com/y00black/status/1409707196930150403} 
  \textgreater{}
  夫婦別姓、婚姻届を出した際にランダムでどちらかの姓に統一される制度を導入すれば一気に気運が高まりそう
\item
  RT fukazawas(深澤諭史)|k\_sawmen(泥濘大魔王サイケ)
  日時:2021-06-29 22:55/2021-06-29 19:33 URL:
  \url{https://twitter.com/fukazawas/status/1409873142005350400} 
  \url{https://twitter.com/k\_sawmen/status/1409822422287536129} 
  \textgreater{} 2年ほど前のツイートに反応ありがとうございます。\\
  \textgreater{}
  「多くの人(要するにマジョリティ)にとって」という主語をつければそうなるでしょうね。その主語を当然のように前提にして、少数者の権利を「念仏」呼ばわりすることの残酷さを指摘してます。
  \url{https://t.co/GpXCDNNy1t} 
\item
  RT fukazawas(深澤諭史)|kobateck(小林哲朗 写真家)
  日時:2021-06-29 23:18/2021-06-29 02:27 URL:
  \url{https://twitter.com/fukazawas/status/1409878874415042565} 
  \url{https://twitter.com/kobateck/status/1409564169641205768} 
  \textgreater{} 15年くらい前まで大阪にあった軍艦アパート
  \url{https://t.co/4HzeQXgfn2} 
\item
  RT fukazawas(深澤諭史)|tokuda101(徳田しんのすけ🍉)
  日時:2021-06-29 23:18/2021-06-29 22:04 URL:
  \url{https://twitter.com/fukazawas/status/1409878888667299841} 
  \url{https://twitter.com/tokuda101/status/1409860332722348037} 
  \textgreater{}
  アイスコーヒー抽出に失敗(豆が少ないのにうっかりいつもの量で出しちゃった)
\item
  TW fukazawas(深澤諭史) 日時: 2021-06-29 23:18 URL:
  \url{https://twitter.com/fukazawas/status/1409879029675610116} 
  \textgreater{} >RT\\
  \textgreater{} コーヒー、ちょっと変わったモノが飲みたい(・∀・)
\item
  RT fukazawas(深澤諭史)|fukazawas(深澤諭史) 日時:2021-06-30
  10:48/2019-02-03 09:14 URL:
  \url{https://twitter.com/fukazawas/status/1410052655712325633} 
  \url{https://twitter.com/fukazawas/status/1091852421695889408} 
  \textgreater{}
  「弁護士が養育費から報酬をとるのはけしからん!」って思っている方々、それを100%阻止するすごいアイディアがありますので、お伝えいたします。\\
  \textgreater{} それは、「ちゃんと養育費を支払う」ことです。\\
  \textgreater{}
  こうしてしまえば、弁護士は手も足も出ません。お役御免です。\\
  \textgreater{} (・∀・)
\item
  RT fukazawas(深澤諭史)|civil\_law1(民法(親族相続)くん)
  日時:2021-06-30 10:48/2019-02-03 20:30 URL:
  \url{https://twitter.com/fukazawas/status/1410052666642767876} 
  \url{https://twitter.com/civil\_law1/status/1092022426265763840} 
  \textgreater{}
  リプライの中には、「本来払うべきお金を請求するなんて酷い!」というレベルのものがありますね。\\
  \textgreater{}
  他人の家庭のことに損をしながら介入してくれる人がいたとして、「他人の家庭事情に関して身銭を切らない弁護士はケチ!」と、異常な水準を置いてしまうことの是非ですよね、要は。
  \url{https://t.co/LrOeCgAhbI} 
\item
  RT fukazawas(深澤諭史)|himaben1st(暇弁@会務やめたい)
  日時:2021-06-30 10:49/2019-02-03 09:28 URL:
  \url{https://twitter.com/fukazawas/status/1410052757055102976} 
  \url{https://twitter.com/himaben1st/status/1091855802602639360} 
  \textgreater{} これに尽きるwwwwwww \url{https://t.co/EjbzhzR08r} 
\item
  RT fukazawas(深澤諭史)|rippy08(りっぴぃ) 日時:2021-06-30
  10:49/2020-07-27 11:12 URL:
  \url{https://twitter.com/fukazawas/status/1410052781168205826} 
  \url{https://twitter.com/rippy08/status/1287571459397869568} 
  \textgreater{}
  実際、義務者側からご相談受けたらこのように言うことはありますしね。。
  \url{https://t.co/o3SIzYHAT4} 
\item
  RT fukazawas(深澤諭史)|sonzaix(同志カルロ・ゼン@〆切が苦手)
  日時:2021-06-30 10:49/2019-11-23 20:45 URL:
  \url{https://twitter.com/fukazawas/status/1410052794204135427} 
  \url{https://twitter.com/sonzaix/status/1198205854991839233} 
  \textgreater{} 養育費の完璧計画\_(:3 」∠)\_\\
  \textgreater{}\\
  \textgreater{} ①まず国家から受け取る権利のある人の口座に振り込まれる\\
  \textgreater{}\\
  \textgreater{}
  ②国家は課税するのと同じノリで支払い義務のある人から徴収する\\
  \textgreater{}\\
  \textgreater{}
  ③養育費の未納というトラブルの被害者を、受取手から国家に移管する!\\
  \textgreater{}\\
  \textgreater{}
  ④行政は徴税には妙に熱心(個人的見解です)だ、税務署を信じよう。
  \url{https://t.co/vq1Cby5sXx} 
\item
  RT fukazawas(深澤諭史)|youarethehero(メタルヘッド背神的悪意者)
  日時:2021-06-30 10:49/2019-11-23 14:57 URL:
  \url{https://twitter.com/fukazawas/status/1410052803855142913} 
  \url{https://twitter.com/youarethehero/status/1198118400024662016} 
  \textgreater{} リプ欄 \url{https://t.co/1sVbBbEaRB} 
\item
  RT fukazawas(深澤諭史)|nodahayato(野田隼人 Atty. NODA Hayato
  J.D.) 日時:2021-06-30 10:49/2019-08-27 08:30 URL:
  \url{https://twitter.com/fukazawas/status/1410052837694791681} 
  \url{https://twitter.com/nodahayato/status/1166130894026887168} 
  \textgreater{}
  いきなり調停を起こされたぞ!という人がいるけど、「いや離婚した時点で払えよ」という感想。離婚調停に付随する分以外の例外はない。
  \url{https://t.co/PhhFtOaVkE} 
\item
  RT fukazawas(深澤諭史)|hosokattawa(細川啓%求職中断)
  日時:2021-06-30 10:49/2019-02-03 21:26 URL:
  \url{https://twitter.com/fukazawas/status/1410052891293876225} 
  \url{https://twitter.com/hosokattawa/status/1092036542174908417} 
  \textgreater{} こんな自明のツイートにすらクソリプがついている。
  \url{https://t.co/5Ws29nwQ2v} 
\item
  RT fukazawas(深澤諭史)|misoka09(長月みそか) 日時:2021-06-30
  10:59/2021-06-30 10:52 URL:
  \url{https://twitter.com/fukazawas/status/1410055398481022983} 
  \url{https://twitter.com/misoka09/status/1410053519063666691} 
  \textgreater{} 多くの作家さんの帰結に同意する。\\
  \textgreater{}
  やはり商業である以上、企画またはゴーサインを出した編集部に責任がある。\\
  \textgreater{}
  世間から叩かれても、作家の盾になって反論できるぐらいの責任が持てないのなら、企画を通すべきではないと思う。
\item
  RT fukazawas(深澤諭史)|noooooooorth(教皇ノースライム)
  日時:2021-06-30 12:01/2021-06-30 07:39 URL:
  \url{https://twitter.com/fukazawas/status/1410071042790330369} 
  \url{https://twitter.com/noooooooorth/status/1410004974499549190} 
  \textgreater{}
  「老婆心ながら」とわざわざ付け加えてくる忠告の役に立たなさ率は異常。
\item
  RT fukazawas(深澤諭史)|tamitin0914(たみちん) 日時:2021-06-30
  13:39/2021-06-29 14:21 URL:
  \url{https://twitter.com/fukazawas/status/1410095626415599622} 
  \url{https://twitter.com/tamitin0914/status/1409743863762231296} 
  \textgreater{} 小4長女のイラストを見てやって下さい
  \url{https://t.co/1HCS9WVeWy} 
\end{itemize}

 「RT fukazawas(深澤諭史)|fukazawas(深澤諭史) 日時:2021-06-30
10:48/2019-02-03
09:14」というリツイートから始まっていました。本日午前10時48分から始まっています。

\begin{itemize}
\tightlist
\item
  〈〈〈 2021/06/30 14:35:32 Linux Emacs: 〈〈〈
\end{itemize}

\hypertarget{ux88c1ux5224ux54e1ux5236ux5ea6ux3092ux76aeux8089ux308bux6df1ux6fa4ux8aedux53f2ux5f01ux8b77ux58ebux81eaux8eabux306bux3088ux308bux904eux53bbux306eux30c4ux30a4ux30fcux30c8ux306eux30eaux30c4ux30a4ux30fcux30c8ux304bux3089ux518dux767aux898bux306bux81f3ux3063ux305fux65b0ux6f5fux5973ux5150ux6bbaux5bb3ux4e8bux4ef6ux306bux304aux3051ux308bux9ed9ux79d8ux6a29ux3068ux3046ux306eux5b57ux3068ux306eux30b3ux30e9ux30dcux306bux3088ux308bux30d2ux30e3ux30c3ux30cfux30fcux306eux30c4ux30a4ux30fcux30c810}{%
\paragraph{裁判員制度を皮肉る深澤諭史弁護士自身による過去のツイートのリツイートから再発見に至った、新潟女児殺害事件における黙秘権とうの字とのコラボによるヒャッハーのツイート(10)}\label{ux88c1ux5224ux54e1ux5236ux5ea6ux3092ux76aeux8089ux308bux6df1ux6fa4ux8aedux53f2ux5f01ux8b77ux58ebux81eaux8eabux306bux3088ux308bux904eux53bbux306eux30c4ux30a4ux30fcux30c8ux306eux30eaux30c4ux30a4ux30fcux30c8ux304bux3089ux518dux767aux898bux306bux81f3ux3063ux305fux65b0ux6f5fux5973ux5150ux6bbaux5bb3ux4e8bux4ef6ux306bux304aux3051ux308bux9ed9ux79d8ux6a29ux3068ux3046ux306eux5b57ux3068ux306eux30b3ux30e9ux30dcux306bux3088ux308bux30d2ux30e3ux30c3ux30cfux30fcux306eux30c4ux30a4ux30fcux30c810}}

\begin{itemize}
\tightlist
\item
  〉〉〉 Linux Emacs: 2021/06/30 14:37:14 〉〉〉
\end{itemize}

:CATEGORIES: @kanazawabengosi \#金沢弁護士会 @JFBAsns
日本弁護士連合会(日弁連) \#法務省 @MOJ\_HOUMU \#深澤諭史弁護士
\#うの字 \#黙秘権 \#退会命令 \#懲戒処分 \#神奈川県弁護士会

 一昨日の夕方になると思います。新潟の白山神社のホームページで白山島というのを見たのですが、あとで調べてみると以前にもツイートに含めていました。そのときはそれほど強く印象には残らなかったようです。

 新潟市の白山神社ある一角で、川と海に挟まれた中洲のような地形になっているのですが、昔は白山島と呼ばれたということでした。

〉〉〉 kk\_hironoのリツイート 〉〉〉

\begin{itemize}
\tightlist
\item
  RT
  kk\_hirono(刑事告発・非常上告_金沢地方検察庁御中)|hirono\_hideki(奉納\さらば弁護士鉄道・泥棒神社の物語)
  日時:2021-06-30 14:43/2021/06/30 08:33 URL:
  \url{https://twitter.com/kk\_hirono/status/1410111841909436421} 
  \url{https://twitter.com/hirono\_hideki/status/1410018500727300096} 
  \textgreater{} 神奈川県弁護士会 古澤眞尋弁護士 退会命令の懲戒処分 -
  弁護士を考える 旧(雑記) \url{https://t.co/l0S5DccznI} 
\end{itemize}

〉〉〉 kk\_hironoのリツイート 〉〉〉

\begin{itemize}
\tightlist
\item
  RT
  kk\_hirono(刑事告発・非常上告_金沢地方検察庁御中)|hirono\_hideki(奉納\さらば弁護士鉄道・泥棒神社の物語)
  日時:2021-06-30 14:44/2021/06/30 08:29 URL:
  \url{https://twitter.com/kk\_hirono/status/1410111874524319746} 
  \url{https://twitter.com/hirono\_hideki/status/1410017736873238534} 
  \textgreater{}
  弁護士法人古澤総合法律事務所所属の古澤 眞尋元弁護士に依頼された方へ|神奈川県弁護士会
  \url{https://t.co/Bb0QWZ3Z5x} 
\end{itemize}

〉〉〉 kk\_hironoのリツイート 〉〉〉

\begin{itemize}
\tightlist
\item
  RT
  kk\_hirono(刑事告発・非常上告_金沢地方検察庁御中)|hirono\_hideki(奉納\さらば弁護士鉄道・泥棒神社の物語)
  日時:2021-06-30 14:44/2021/06/30 02:45 URL:
  \url{https://twitter.com/kk\_hirono/status/1410111981130899458} 
  \url{https://twitter.com/hirono\_hideki/status/1409931031726104578} 
  \textgreater{} 裁判の証拠に虚偽内容メール提出 弁護士に退会命令|NHK
  神奈川県のニュース \url{https://t.co/g4V52hKs7F} 
  弁護士会が確認したところ、司法修習生から送られた挨拶のメールを加工して古澤弁護士がねつ造した可能性があるということです。
\end{itemize}

〉〉〉 kk\_hironoのリツイート 〉〉〉

\begin{itemize}
\tightlist
\item
  RT
  kk\_hirono(刑事告発・非常上告_金沢地方検察庁御中)|hirono\_hideki(奉納\さらば弁護士鉄道・泥棒神社の物語)
  日時:2021-06-30 14:45/2021/06/29 20:46 URL:
  \url{https://twitter.com/kk\_hirono/status/1410112252204650498} 
  \url{https://twitter.com/hirono\_hideki/status/1409840809944379394} 
  \textgreater{}
  「地方移住」激増のウラで・・・田舎暮らしで「地獄を見た」人たちの恐怖体験(清泉
  亮) \textbar{} 現代ビジネス \textbar{} 講談社(4/4)
  \url{https://t.co/YdU52wiBsw} こっちの言葉では、よそ者を『きたりもん』と呼ぶらしいのですが、きたりもんはもう死ぬまできたりもんだって、やっぱり仲良くなった、ヨソから嫁いできた90
\end{itemize}

〉〉〉 kk\_hironoのリツイート 〉〉〉

\begin{itemize}
\tightlist
\item
  RT
  kk\_hirono(刑事告発・非常上告_金沢地方検察庁御中)|hirono\_hideki(奉納\さらば弁護士鉄道・泥棒神社の物語)
  日時:2021-06-30 14:45/2021/06/29 20:26 URL:
  \url{https://twitter.com/kk\_hirono/status/1410112349755772928} 
  \url{https://twitter.com/hirono\_hideki/status/1409835655018913797} 
  \textgreater{} 清泉亮 - Wikiwand \url{https://t.co/5jn2DT5ViF} 
  『吉原まんだら: 色街の女帝が駆け抜けた戦後』徳間書店、2015年
  『十字架を背負った尾根: 日航機墜落現場の知られざる四季』草思社、2015年
  『誰も教えてくれない田舎暮らしの教科書』東洋経済新報社、2018年
\end{itemize}

〉〉〉 kk\_hironoのリツイート 〉〉〉

\begin{itemize}
\tightlist
\item
  RT
  kk\_hirono(刑事告発・非常上告_金沢地方検察庁御中)|hirono\_hideki(奉納\さらば弁護士鉄道・泥棒神社の物語)
  日時:2021-06-30 14:46/2021/06/29 20:26 URL:
  \url{https://twitter.com/kk\_hirono/status/1410112364024766466} 
  \url{https://twitter.com/hirono\_hideki/status/1409835567374749709} 
  \textgreater{} 清泉亮 - Wikiwand \url{https://t.co/5jn2DT5ViF} 
  Twitterアカウントによると、2021年1月15日に長野県内の国道で乗っていた自動車が、逆走してきた軽トラックと正面衝突し、翌日死去した{[}3{]}。
\end{itemize}

〉〉〉 kk\_hironoのリツイート 〉〉〉

\begin{itemize}
\tightlist
\item
  RT
  kk\_hirono(刑事告発・非常上告_金沢地方検察庁御中)|hirono\_hideki(奉納\さらば弁護士鉄道・泥棒神社の物語)
  日時:2021-06-30 14:47/2021/06/29 19:53 URL:
  \url{https://twitter.com/kk\_hirono/status/1410112679771971587} 
  \url{https://twitter.com/hirono\_hideki/status/1409827327836516353} 
  \textgreater{} ▶
  ブロックされたツイート%akishigemakoto(MakotoAkishige(civilista))%2021/06/28
  21:18:01% \url{https://t.co/XiB5IlHWeA}  \textgreater{}
  発言内容を改変したうえ、誤った用語にすり替える報道機関
  \url{https://t.co/AwAN82GwhX} 
\end{itemize}

〉〉〉 kk\_hironoのリツイート 〉〉〉

\begin{itemize}
\tightlist
\item
  RT
  kk\_hirono(刑事告発・非常上告_金沢地方検察庁御中)|hirono\_hideki(奉納\さらば弁護士鉄道・泥棒神社の物語)
  日時:2021-06-30 14:47/2021/06/29 19:18 URL:
  \url{https://twitter.com/kk\_hirono/status/1410112776102547463} 
  \url{https://twitter.com/hirono\_hideki/status/1409818476957503491} 
  \textgreater{} 2021-06-29\_18:57
  奉納\\#危険生物・弁護士脳汚染除去装置\\#金沢地方検察庁御中\_2020:
  REGEXP:''退会.*命令''/データベース登録済みツイートの検索:2010-09-04〜2021-06-29/2021年06月29日18時55分の記録:ユーザ・投稿:71/138件
  \url{https://t.co/NVoVja6vM7} 
\end{itemize}

〉〉〉 kk\_hironoのリツイート 〉〉〉

\begin{itemize}
\tightlist
\item
  RT
  kk\_hirono(刑事告発・非常上告_金沢地方検察庁御中)|hirono\_hideki(奉納\さらば弁護士鉄道・泥棒神社の物語)
  日時:2021-06-30 14:48/2021/06/29 18:29 URL:
  \url{https://twitter.com/kk\_hirono/status/1410112906981744641} 
  \url{https://twitter.com/hirono\_hideki/status/1409806293976584195} 
  \textgreater{}
  「これで有罪になれば大変なことになる」孤立出産で死産した技能実習生の起訴に対して医師が示した危機感(望月優大)
  - 個人 - Yahoo!ニュース \url{https://t.co/a9mbY1l47l} 
  主任弁護人の石黒大貴弁護士が会見で言っていたように、この裁判は検察側と弁護側がその争点をめぐって真正面から争うという形に
\end{itemize}

〉〉〉 kk\_hironoのリツイート 〉〉〉

\begin{itemize}
\tightlist
\item
  RT
  kk\_hirono(刑事告発・非常上告_金沢地方検察庁御中)|hirono\_hideki(奉納\さらば弁護士鉄道・泥棒神社の物語)
  日時:2021-06-30 14:48/2021/06/29 16:19 URL:
  \url{https://twitter.com/kk\_hirono/status/1410113071805210627} 
  \url{https://twitter.com/hirono\_hideki/status/1409773420762386434} 
  \textgreater{} - 死刑制度違憲説について|小動物を愛するしんさん|note
  \url{https://t.co/qGjP3z0lWl} 
\end{itemize}

〉〉〉 kk\_hironoのリツイート 〉〉〉

\begin{itemize}
\tightlist
\item
  RT
  kk\_hirono(刑事告発・非常上告_金沢地方検察庁御中)|hirono\_hideki(奉納\さらば弁護士鉄道・泥棒神社の物語)
  日時:2021-06-30 14:49/2021/06/29 08:35 URL:
  \url{https://twitter.com/kk\_hirono/status/1410113252676182018} 
  \url{https://twitter.com/hirono\_hideki/status/1409656818989637633} 
  \textgreater{}
  岡口基一民事裁判傍聴記・2021年6月25日(被告・岡口基一、原告・岩瀬正史外)|相馬獄長|note
  \url{https://t.co/PkwyhfcxTx} 
\end{itemize}

〉〉〉 kk\_hironoのリツイート 〉〉〉

\begin{itemize}
\tightlist
\item
  RT
  kk\_hirono(刑事告発・非常上告_金沢地方検察庁御中)|hirono\_hideki(奉納\さらば弁護士鉄道・泥棒神社の物語)
  日時:2021-06-30 14:49/2021/06/29 08:02 URL:
  \url{https://twitter.com/kk\_hirono/status/1410113311287373824} 
  \url{https://twitter.com/hirono\_hideki/status/1409648525667749888} 
  \textgreater{}
  アンミカ、ブリーフ裁判官騒動に「人として恥ずかしい」「犯罪的なことをあおっている」とバッサリ
  \textbar{} ニコニコニュース \url{https://t.co/kqqA6uVSta}  2021/06/28 16:30
\end{itemize}

〉〉〉 kk\_hironoのリツイート 〉〉〉

\begin{itemize}
\tightlist
\item
  RT
  kk\_hirono(刑事告発・非常上告_金沢地方検察庁御中)|nemuri\_ayaaya(眠りのあや@無職非主婦アタオカ万歳)
  日時:2021-06-30 14:50/2021/06/28 14:23 URL:
  \url{https://twitter.com/kk\_hirono/status/1410113476689756161} 
  \url{https://twitter.com/nemuri\_ayaaya/status/1409381937345482760} 
  \textgreater{} バイキングでやってるね。
  岡口裁判官を提訴 投稿は「被害者の尊厳への配慮欠く」:朝日新聞デジタル
  \url{https://t.co/Ghl7rjviIG} 
\end{itemize}

〉〉〉 kk\_hironoのリツイート 〉〉〉

\begin{itemize}
\tightlist
\item
  RT
  kk\_hirono(刑事告発・非常上告_金沢地方検察庁御中)|hirono\_hideki(奉納\さらば弁護士鉄道・泥棒神社の物語)
  日時:2021-06-30 14:51/2021/06/28 23:07 URL:
  \url{https://twitter.com/kk\_hirono/status/1410113834992357376} 
  \url{https://twitter.com/hirono\_hideki/status/1409513871635718149} 
  \textgreater{} -
  袴田弁護団が期待感「再審開始は目前」 静岡で集会|あなたの静岡新聞
  \url{https://t.co/vSfgaLvOGl} 
\end{itemize}

〉〉〉 kk\_hironoのリツイート 〉〉〉

\begin{itemize}
\tightlist
\item
  RT
  kk\_hirono(刑事告発・非常上告_金沢地方検察庁御中)|adire\_kouhou(【公式】弁護士法人アディーレ法律事務所)
  日時:2021-06-30 14:52/2021/06/28 19:17 URL:
  \url{https://twitter.com/kk\_hirono/status/1410113958036467713} 
  \url{https://twitter.com/adire\_kouhou/status/1409455876218310656} 
  \textgreater{}
  6/16(水)5:50~8:00 読売テレビ【朝生ワイド す・またん!&ZIP!】から、``和歌山毒物カレー事件 再審請求受理''について長井健一弁護士が取材を受けました。
  \url{https://t.co/6BAV6WgBWX} 
\end{itemize}

〉〉〉 kk\_hironoのリツイート 〉〉〉

\begin{itemize}
\tightlist
\item
  RT
  kk\_hirono(刑事告発・非常上告_金沢地方検察庁御中)|wakayamacurry(和歌山カレー事件
  長男) 日時:2021-06-30 14:52/2021/06/28 20:27 URL:
  \url{https://twitter.com/kk\_hirono/status/1410114046276227075} 
  \url{https://twitter.com/wakayamacurry/status/1409473448267567107} 
  \textgreater{}
  和歌山カレー「再審申立書」\textasciitilde 冤罪の大カラクリを根底から暴露
  \url{https://t.co/gBfBgvzGgO}  \#Amazon
\end{itemize}

〉〉〉 kk\_hironoのリツイート 〉〉〉

\begin{itemize}
\tightlist
\item
  RT
  kk\_hirono(刑事告発・非常上告_金沢地方検察庁御中)|hirono\_hideki(奉納\さらば弁護士鉄道・泥棒神社の物語)
  日時:2021-06-30 14:52/2021/06/28 22:56 URL:
  \url{https://twitter.com/kk\_hirono/status/1410114115310292998} 
  \url{https://twitter.com/hirono\_hideki/status/1409510932636569603} 
  \textgreater{} 2021-06-28\_22:50
  奉納\\#危険生物・弁護士脳汚染除去装置\\#金沢地方検察庁御中\_2020:
  REGEXP:''キラ様''/データベース登録済みツイートの検索:2015-04-06〜2020-07-27/2021年06月28日22時48分の記録:ユーザ・投稿:45/120件
  \url{https://t.co/bWbxZWysDF} 
\end{itemize}

〉〉〉 kk\_hironoのリツイート 〉〉〉

\begin{itemize}
\tightlist
\item
  RT
  kk\_hirono(刑事告発・非常上告_金沢地方検察庁御中)|hirono\_hideki(奉納\さらば弁護士鉄道・泥棒神社の物語)
  日時:2021-06-30 14:53/2021/06/28 22:46 URL:
  \url{https://twitter.com/kk\_hirono/status/1410114180196093959} 
  \url{https://twitter.com/hirono\_hideki/status/1409508503480258563} 
  \textgreater{} 宇出津 崎山 火葬場 - Google 検索
  \url{https://t.co/TIurpQZiq1} 
\end{itemize}

〉〉〉 kk\_hironoのリツイート 〉〉〉

\begin{itemize}
\tightlist
\item
  RT
  kk\_hirono(刑事告発・非常上告_金沢地方検察庁御中)|1961kumachin(くまちん(弁護士中村元弥))
  日時:2021-06-30 14:53/2012/05/10 23:43 URL:
  \url{https://twitter.com/kk\_hirono/status/1410114245971169282} 
  \url{https://twitter.com/1961kumachin/status/200596852670865408} 
  \textgreater{} @piyorinchi FBの刑事訴訟愛好会に招待しておいた
\end{itemize}

〉〉〉 kk\_hironoのリツイート 〉〉〉

\begin{itemize}
\tightlist
\item
  RT
  kk\_hirono(刑事告発・非常上告_金沢地方検察庁御中)|BarlKarth(高島章)
  日時:2021-06-30 14:53/2018/08/31 01:11 URL:
  \url{https://twitter.com/kk\_hirono/status/1410114309779124231} 
  \url{https://twitter.com/BarlKarth/status/1035198320904261632} 
  \textgreater{} @nipox25
  私がフェイスブックで運営している「刑事訴訟愛好会」はもうちょっと入会条件が厳しいです。大学院博士後期過程在学中以上でないと入れません。崩壊大学院終了でも、司法試験合格が入会条件です。
\end{itemize}

〉〉〉 kk\_hironoのリツイート 〉〉〉

\begin{itemize}
\tightlist
\item
  RT
  kk\_hirono(刑事告発・非常上告_金沢地方検察庁御中)|hirono\_hideki(奉納\さらば弁護士鉄道・泥棒神社の物語)
  日時:2021-06-30 14:54/2021/06/28 21:39 URL:
  \url{https://twitter.com/kk\_hirono/status/1410114407552602118} 
  \url{https://twitter.com/hirono\_hideki/status/1409491689253707778} 
  \textgreater{} 刑事訴訟愛好会 \textbar{} Facebook
  \url{https://t.co/63C8VEgMW1}  メンバー · 818人
\end{itemize}

〉〉〉 kk\_hironoのリツイート 〉〉〉

\begin{itemize}
\tightlist
\item
  RT
  kk\_hirono(刑事告発・非常上告_金沢地方検察庁御中)|hirono\_hideki(奉納\さらば弁護士鉄道・泥棒神社の物語)
  日時:2021-06-30 14:54/2021/06/28 20:53 URL:
  \url{https://twitter.com/kk\_hirono/status/1410114467950567433} 
  \url{https://twitter.com/hirono\_hideki/status/1409480039486222339} 
  \textgreater{}
  『弁護士懲戒王座決定戦』(個人戦)何回まで懲戒処分を受けられるか!現役最多は9回(香川)8回(一弁)8回(新潟)3人が1差の中で競り合う!5月25日6回目処分5位に順位上げる。4位まで1差(更新)≪弁護士自治を考える会≫
  -- 弁護士自治を考える会 \url{https://t.co/PJTlo6RDj0} 
\end{itemize}

〉〉〉 kk\_hironoのリツイート 〉〉〉

\begin{itemize}
\tightlist
\item
  RT
  kk\_hirono(刑事告発・非常上告_金沢地方検察庁御中)|hirono\_hideki(奉納\さらば弁護士鉄道・泥棒神社の物語)
  日時:2021-06-30 14:54/2021/06/28 20:46 URL:
  \url{https://twitter.com/kk\_hirono/status/1410114512296968193} 
  \url{https://twitter.com/hirono\_hideki/status/1409478423211175937} 
  \textgreater{} 〒950-2051 新潟県新潟市西区寺尾朝日通1−13 - Google
  マップ \url{https://t.co/UDluyBo4on} 
\end{itemize}

〉〉〉 kk\_hironoのリツイート 〉〉〉

\begin{itemize}
\tightlist
\item
  RT
  kk\_hirono(刑事告発・非常上告_金沢地方検察庁御中)|hirono\_hideki(奉納\さらば弁護士鉄道・泥棒神社の物語)
  日時:2021-06-30 14:54/2021/06/28 20:41 URL:
  \url{https://twitter.com/kk\_hirono/status/1410114523642568706} 
  \url{https://twitter.com/hirono\_hideki/status/1409477062268698626} 
  \textgreater{}
  高島章弁護士(新潟)業務停止6月 費用を返金せず。6回目の処分「毎日地方有料版」
  -- 弁護士自治を考える会 \url{https://t.co/zoqmYW0FE4}  投稿日 : 2021年2月20日
  \textbar{} カテゴリー : 弁護士に関する記事
\end{itemize}

〉〉〉 kk\_hironoのリツイート 〉〉〉

\begin{itemize}
\tightlist
\item
  RT
  kk\_hirono(刑事告発・非常上告_金沢地方検察庁御中)|hirono\_hideki(奉納\さらば弁護士鉄道・泥棒神社の物語)
  日時:2021-06-30 14:54/2021/06/28 20:41 URL:
  \url{https://twitter.com/kk\_hirono/status/1410114580127240192} 
  \url{https://twitter.com/hirono\_hideki/status/1409476932274561024} 
  \textgreater{}
  高島章弁護士(新潟)に六度目の懲戒処分は業務停止6月 カネがらみのトラブルが続きますので復活は困難ではないでしょうか?
  -- 鎌倉九郎 \url{https://t.co/RssxVFAxfe} 
  カネのために非弁屋に飼われる弁護士も減る事は間違いないのである。本気で日弁連・各単位会の偉いさん方は検討して欲しい。
\end{itemize}

〉〉〉 kk\_hironoのリツイート 〉〉〉

\begin{itemize}
\tightlist
\item
  RT
  kk\_hirono(刑事告発・非常上告_金沢地方検察庁御中)|hirono\_hideki(奉納\さらば弁護士鉄道・泥棒神社の物語)
  日時:2021-06-30 14:55/2021/06/28 20:32 URL:
  \url{https://twitter.com/kk\_hirono/status/1410114680094265349} 
  \url{https://twitter.com/hirono\_hideki/status/1409474807456956418} 
  \textgreater{} 新潟地方検察庁・新潟区検察庁・新津区検察庁 から
  新潟地方裁判所 - Google マップ \url{https://t.co/dxrEo8I014}  7 分(1.6 km)
  東中通り 経由 最速ルート
\end{itemize}

〉〉〉 kk\_hironoのリツイート 〉〉〉

\begin{itemize}
\tightlist
\item
  RT
  kk\_hirono(刑事告発・非常上告_金沢地方検察庁御中)|hirono\_hideki(奉納\さらば弁護士鉄道・泥棒神社の物語)
  日時:2021-06-30 14:55/2021/06/28 20:23 URL:
  \url{https://twitter.com/kk\_hirono/status/1410114728140021760} 
  \url{https://twitter.com/hirono\_hideki/status/1409472567933501445} 
  \textgreater{} 東北の江ノ島!?『白山島』をご紹介! \textbar{}
  海と日本PROJECT in 山形 \url{https://t.co/729ZP5ERGn} 
\end{itemize}

〉〉〉 kk\_hironoのリツイート 〉〉〉

\begin{itemize}
\tightlist
\item
  RT
  kk\_hirono(刑事告発・非常上告_金沢地方検察庁御中)|hirono\_hideki(奉納\さらば弁護士鉄道・泥棒神社の物語)
  日時:2021-06-30 14:55/2021/06/28 20:07 URL:
  \url{https://twitter.com/kk\_hirono/status/1410114857773395969} 
  \url{https://twitter.com/hirono\_hideki/status/1409468513085902860} 
  \textgreater{} 白山島 -- つるおか観光ナビ \url{https://t.co/XLfhT9rzCe} 
  由良海岸のシンボル「白山島」は、火山性噴火によってできたといわれ、高さ70m、周囲436mの小島。島を周遊する散策路も整備されている。
\end{itemize}

〉〉〉 kk\_hironoのリツイート 〉〉〉

\begin{itemize}
\tightlist
\item
  RT
  kk\_hirono(刑事告発・非常上告_金沢地方検察庁御中)|hirono\_hideki(奉納\さらば弁護士鉄道・泥棒神社の物語)
  日時:2021-06-30 14:56/2021/06/28 19:27 URL:
  \url{https://twitter.com/kk\_hirono/status/1410114974488305667} 
  \url{https://twitter.com/hirono\_hideki/status/1409458350807388160} 
  \textgreater{} -
  新潟総鎮守 白山神社|初詣、七五三、縁結び、安産、厄除けなど
  \url{https://t.co/6Oa1GcX7JK} 
\end{itemize}

〉〉〉 kk\_hironoのリツイート 〉〉〉

\begin{itemize}
\tightlist
\item
  RT
  kk\_hirono(刑事告発・非常上告_金沢地方検察庁御中)|hirono\_hideki(奉納\さらば弁護士鉄道・泥棒神社の物語)
  日時:2021-06-30 14:56/2021/06/28 19:11 URL:
  \url{https://twitter.com/kk\_hirono/status/1410115066515447808} 
  \url{https://twitter.com/hirono\_hideki/status/1409454515938480135} 
  \textgreater{} - 【石川】町、照明のLED化検討
  能登町トンネルで女性死亡:北陸中日新聞Web \url{https://t.co/AbkZLC0zNW} 
\end{itemize}

〉〉〉 kk\_hironoのリツイート 〉〉〉

\begin{itemize}
\tightlist
\item
  RT
  kk\_hirono(刑事告発・非常上告_金沢地方検察庁御中)|hirono\_hideki(奉納\さらば弁護士鉄道・泥棒神社の物語)
  日時:2021-06-30 14:56/2021/06/28 19:01 URL:
  \url{https://twitter.com/kk\_hirono/status/1410115090460790785} 
  \url{https://twitter.com/hirono\_hideki/status/1409451768971337731} 
  \textgreater{} - 悪女の法律―女をバカにするとどうなるか (ワニ文庫)
  \textbar{} 和久 峻三 \textbar 本 \textbar{} 通販 \textbar{} Amazon
  \url{https://t.co/DMzg7ZvMyD} 
\end{itemize}

〉〉〉 kk\_hironoのリツイート 〉〉〉

\begin{itemize}
\item
  RT
  kk\_hirono(刑事告発・非常上告_金沢地方検察庁御中)|hirono\_hideki(奉納\さらば弁護士鉄道・泥棒神社の物語)
  日時:2021-06-30 14:57/2021/06/28 16:35 URL:
  \url{https://twitter.com/kk\_hirono/status/1410115200326373376} 
  \url{https://twitter.com/hirono\_hideki/status/1409415137161740288} 
  \textgreater{}
  国民民主の衆院選候補、公認取り消し 万引き容疑で捜査:朝日新聞デジタル
  \url{https://t.co/qcpAQOOdCi} 
  折川氏は東京都出身。毎日新聞記者やフリージャーナリストとして活動した後、同党の候補者公募に応募。今月1日に島根2区の公認候補として発表されたばかりだった。
\item
  〈〈〈 2021/06/30 14:58:13 Linux Emacs: 〈〈〈
\end{itemize}

\hypertarget{ux88c1ux5224ux54e1ux5236ux5ea6ux3092ux76aeux8089ux308bux6df1ux6fa4ux8aedux53f2ux5f01ux8b77ux58ebux81eaux8eabux306bux3088ux308bux904eux53bbux306eux30c4ux30a4ux30fcux30c8ux306eux30eaux30c4ux30a4ux30fcux30c8ux304bux3089ux518dux767aux898bux306bux81f3ux3063ux305fux65b0ux6f5fux5973ux5150ux6bbaux5bb3ux4e8bux4ef6ux306bux304aux3051ux308bux9ed9ux79d8ux6a29ux3068ux3046ux306eux5b57ux3068ux306eux30b3ux30e9ux30dcux306bux3088ux308bux30d2ux30e3ux30c3ux30cfux30fcux306eux30c4ux30a4ux30fcux30c811}{%
\paragraph{裁判員制度を皮肉る深澤諭史弁護士自身による過去のツイートのリツイートから再発見に至った、新潟女児殺害事件における黙秘権とうの字とのコラボによるヒャッハーのツイート(11)}\label{ux88c1ux5224ux54e1ux5236ux5ea6ux3092ux76aeux8089ux308bux6df1ux6fa4ux8aedux53f2ux5f01ux8b77ux58ebux81eaux8eabux306bux3088ux308bux904eux53bbux306eux30c4ux30a4ux30fcux30c8ux306eux30eaux30c4ux30a4ux30fcux30c8ux304bux3089ux518dux767aux898bux306bux81f3ux3063ux305fux65b0ux6f5fux5973ux5150ux6bbaux5bb3ux4e8bux4ef6ux306bux304aux3051ux308bux9ed9ux79d8ux6a29ux3068ux3046ux306eux5b57ux3068ux306eux30b3ux30e9ux30dcux306bux3088ux308bux30d2ux30e3ux30c3ux30cfux30fcux306eux30c4ux30a4ux30fcux30c811}}

\begin{itemize}
\tightlist
\item
  〉〉〉 Linux Emacs: 2021/06/30 15:01:19 〉〉〉
\end{itemize}

:CATEGORIES: @kanazawabengosi \#金沢弁護士会 @JFBAsns
日本弁護士連合会(日弁連) \#法務省 @MOJ\_HOUMU \#深澤諭史弁護士
\#うの字 \#高島章弁護士(新潟県弁護士会) \#退会命令 \#神奈川県弁護士会

\begin{itemize}
\item
  1439:2021-06-30\_14:58:47 \#告発状 \#\#\#\#
  裁判員制度を皮肉る深澤諭史弁護士自身による過去のツイートのリツイートから再発見に至った、新潟女児殺害事件における黙秘権とうの字とのコラボによるヒャッハーのツイート(10)
  \url{https://hirono-hideki.hatenadiary.jp/entry/2021/06/30/145844} 
\item
  奉納\危険生物・弁護士脳汚染除去装置\金沢地方検察庁御中\_2020:
  REGEXP:''退会.*命令''/データベース登録済みツイートの検索:2010-09-04〜2021-06-29/2021年06月29日18時55分の記録:ユーザ・投稿:71/138件
  \url{https://t.co/OTfTwkCt4T} 
\item
  (004/138) TW @okumuraosaka(TORU OKUMURA) 日時: 2013-09-06
  21:08:00 +0900 URL:
  \url{https://twitter.com/okumuraosaka/status/375953768719736832\textgreater} {}
  富山県弁護士会は6日、犯罪被害者にしつこく示談を持ちかけたなどとして、高森浩弁護士(46)を4日付で、退会命令の懲戒処分
  富山の弁護士に退会命令 懲戒5回目、弁護士会 -
  47NEWS(よんななニュース) \url{http://t.co/Hjuvqps6ao} 
\end{itemize}

 2013年9月6日のツイートに、「富山県弁護士会は6日、犯罪被害者にしつこく示談を持ちかけたなどとして、高森浩弁護士(46)を4日付で、退会命令の懲戒処分」とありますが、その後、日弁連の審査で退会命令は取り消しとなっていました。

\begin{itemize}
\tightlist
\item
  高森 浩弁護士(富山)懲戒処分の変更の要旨 退会命令⇒業務停止2年 --
  弁護士自治を考える会 \url{https://t.co/KIaLDyL32I} 
\end{itemize}

 確認をしておきました。

\begin{itemize}
\tightlist
\item
  (066/138) TW @un\_co\_the2nd(糞) 日時: 2020-11-17 08:01:00 +0900
  URL:
  \url{https://twitter.com/un\_co\_the2nd/status/1328473184358649859\textgreater} {}
  一晩たって読んでも会立件での電通の顧問の退会命令だすくらいでなきゃバランスが悪いなーと思う。すでに少なくとも二人殺した会社だからな・・・\textgreater\textgreater{}
  「戒告軽すぎる」、AV出演助長の弁護士に業務停止処分 日弁連
  \url{https://t.co/6QpBIRVkBX}  @Sankei\_newsより
\end{itemize}

 うの字のツイートで「会立件」が出てきました。通常、弁護士の懲戒処分は外から申し立てを受けてやるそうですが、弁護士会が独自に調べて審査するのを会立件と呼ぶようです。栃木県弁護士会になるのか、女性を長時間、事務所にとどめ置いたことで、最初に見たと記憶にあります。

\begin{itemize}
\tightlist
\item
  国選弁護を担当した当時20歳代の女性に対し不適切な対応をしたとして、栃木県弁護士会が、宇都宮市内の男性弁護士を懲戒処分する方針を固めたことが26日、県弁護士会関係者らへの取材でわかった。 /のまとめ
  - Togetter \url{https://t.co/QzUgYlJDOF} 
\end{itemize}

\begin{quote}
《引用の始まり》
\end{quote}

\begin{quote}
女性は、「強引に体を触られた」などとして県警に被害届を出し、弁護士会にも懲戒請求。昨春に示談となったのを受けて立件は見送られた。懲戒請求も取り下げられたが、事態を重くみた同会が処分を検討していた。横田弁護士は女性に数百万円の示談金を支払ったという。弁護士会関係者によると、処分の検討過程では退会命令などの処分も検討されたが、女性が示談後、弁護士会の事情聞き取りに応じなかったことから詳しい事情がわからず、戒告にとどまった。あるベテラン弁護士は「一晩一緒にいただけで大問題。国選弁護したことにつけ込んだとしか考えられず、卑劣極まりない。戒告にとどまったことがおかしい」と話した。(2013年4月4日12時05分
読売新聞)
\end{quote}

\begin{quote}
《引用の終わり》
\end{quote}

\begin{itemize}
\tightlist
\item
  横田康行弁護士(栃木)懲戒処分の要旨 -- 弁護士自治を考える会
  \url{https://jlfmt.com/2013/07/16/29412/} 
\end{itemize}

 記憶とは少し違っているように感じたのですが、「「強引に体を触られた」などとして県警に被害届を出し、弁護士会にも懲戒請求。」という部分は忘れていました。「懲戒請求も取り下げられたが、事態を重くみた同会が処分を検討していた。」とあります。

 弁護士がツイートでこれを会立件としていたと記憶にあるのですが、最初は懲戒請求され、示談の成立で取り下げになっていたのを、栃木県弁護士会が検討し処分したということのようです。

\begin{itemize}
\item
  (086/138) TW
  @hirono\_hideki(奉納\さらば弁護士鉄道・泥棒神社の物語) 日時:
  2020-11-20 13:47:00 +0900 URL:
  \url{https://twitter.com/hirono\_hideki/status/1329647409287233542\textgreater} {}
  3000万円横領の弁護士
  山本賢一弁護士(34)長野県弁護士会退会命令 県弁護士会事件としての告発せず
  成年後見人の立場を悪用 - YouTube \url{https://t.co/kFH6dygNFR} 
\item
  (097/138) TW @masaki\_kito(紀藤正樹 MasakiKito) 日時: 2021-05-07
  07:05:51 +0900 URL:
  \url{https://twitter.com/masaki\_kito/status/1390427615517765635\textgreater} {}
  弁護士が同じことを行えば退会命令以上に値する行為ですから登録できるのはおかしいと思います\textgreater 江川紹子氏が証拠改ざん隠蔽事件有罪の元特捜部長の弁護士登録に「驚き!」(東スポWeb)
  \url{https://t.co/XmpT9t6F6T} 
\item
  (098/138) TW @mamenske(?まめ之助?) 日時: 2021-06-29 16:32:29
  +0900 URL:
  \url{https://twitter.com/mamenske/status/1409776770811785220\textgreater} {}
  偽造した証拠を提出して、退会命令・・・。\textgreater\textgreater{}
  違う理由かと思ったけど、ここまでいくものなのですか(・・?)
\end{itemize}

 ようやく出てきましたが、昨日29日の16時32分のツイートとなっています。

\begin{itemize}
\item
  (099/138) TW @sakamotomasayuk(坂本正幸) 日時: 2021-06-29
  16:44:06 +0900 URL:
  \url{https://twitter.com/sakamotomasayuk/status/1409779694044516360\textgreater} {}
  退会命令か \url{https://t.co/eBBc4oTGpo} 
\item
  (113/138) TW @ekinan\_lawyer(えきなんローヤー?) 日時: 2021-06-29
  17:02:49 +0900 URL:
  \url{https://twitter.com/ekinan\_lawyer/status/1409784406735155208\textgreater} {}
  しかし、その退会命令受けた弁護士、非常勤裁判官とかやってたみたいよ。\textgreater\textgreater{}
  やりたくてもやれない、選ばれし弁護士のみができる仕事よね。\textgreater\textgreater{}
  外面が良くて、人望はかなりあったんだろうな・・・
\item
  (116/138) TW @GUv4i6(北白川) 日時: 2021-06-29 17:09:08 +0900
  URL:
  \url{https://twitter.com/GUv4i6/status/1409785997110382593\textgreater} {}
  退会命令は強い。なんどもいうがお客さんのお金ぽっけないないマンとの平仄が気になる。
\end{itemize}

 「退会命令」をキーワードにしたツイートでは、なかったか、目立たなかったと思いますが、川崎の法律事務所でパワハラという問題は、結構前から法クラで話題となり、ツイートで見かけていました。

 今朝見た情報になるのか、司法修習生から送信されたメールの内容を偽造して、懲戒請求された元部下の弁護士の悪評をメールに書き込み証拠提出したという話ではなかったかと思います。

 メールの内容の偽造というのは余り聞いたことのない話ですが、いったん受信したメールを改ざんしても、メールサーバに送信メールが残っていれば明らかに偽造だと判明しそうですが、そこまで踏み込んだ内容の情報は見ていません。可能性がある、という弁護士会の判断のようでした。

\begin{quote}
《引用の始まり》
\end{quote}

\begin{quote}
神奈川県弁護士会は29日、自身のパワーハラスメント行為が指摘される訴訟で捏造した証拠を提出したとして、古沢真尋弁護士(55)を退会命令の懲戒処分にしたと発表した。古沢氏は捏造を否認しているという。
\end{quote}

\begin{quote}
《引用の終わり》
\end{quote}

\begin{itemize}
\tightlist
\item
  証拠捏造で弁護士退会命令 パワハラ訴訟で、神奈川(共同通信) -
  Yahoo!ニュース
  \url{https://news.yahoo.co.jp/articles/96943f1193e2bf31d37164d93c0d3e356c368c6c} 
\end{itemize}

\begin{quote}
《引用の始まり》
\end{quote}

\begin{quote}
このパワハラを巡る訴訟で2016年12月、男性弁護士の悪評を書いたメールをでっち上げ、証拠として提出するなどしたと弁護士会は指摘している。
\end{quote}

\begin{quote}
《引用の終わり》
\end{quote}

\begin{itemize}
\tightlist
\item
  証拠捏造で弁護士退会命令 パワハラ訴訟で、神奈川(共同通信) -
  Yahoo!ニュース
  \url{https://news.yahoo.co.jp/articles/96943f1193e2bf31d37164d93c0d3e356c368c6c} 
\end{itemize}

 実務家でもないのでメールの証拠提出というのは具体例を知りませんが、Twitterのツイートと同じようなもので、メールの場合はヘッダ情報が一緒に印刷されているとは考えられます。

 過去に偽メールを信じて不正の追求を行い、偽メールと発覚した後に自殺した国会議員がいたと記憶にあります。北九州とも記憶にあります。まだSNSが普及する前だったと思います。

\begin{itemize}
\tightlist
\item
  永田寿康元議員 自殺に追いやった元凶は?:|NetIB-NEWS|ネットアイビーニュース
  \url{https://t.co/X9yRRQzxK2} 
\end{itemize}

\begin{quote}
《引用の始まり》
\end{quote}

\begin{quote}
この裁判は2016年に始まり、翌17年2月に男性弁護士が古沢氏に対する懲戒請求をしていた。古沢氏は裁判で、男性弁護士の迷惑行為によって「多数の顧問先や顧客を失った」ことなどを示す証拠として、メールを印刷した書面を提出していた。

 裁判資料によると、事務所の就職説明会や食事会に参加した司法修習生からのお礼のメールで、書面には「(男性弁護士が)極めて最低の弁護士であるということを聞いておりました」「事務所に就職することは自分の将来に大きな影を落とすことになりかねないと考え、この度は応募を辞退させて頂く」などと記載されていた。

 これを受け、男性弁護士側が・・・
\end{quote}

\begin{quote}
《引用の終わり》
\end{quote}

\begin{itemize}
\tightlist
\item
  パワハラ訴訟でメールの文面偽造 弁護士会が退会処分:朝日新聞デジタル
  \url{https://www.asahi.com/articles/ASP6Y7W41P6YULOB01S.html} 
\end{itemize}

 しかし、メールの文面偽造が認められなければ、退会命令で弁護士の身分を失うこともなかったと思われますが、メール1つで運命が決まるというのもすごいことだと思いました。捏造を否認とあるので、日弁連に不服申立はするのでしょう。

 漫画のコマのような断片しかみていないので、不可解にも思えるのですが、メールの文面を偽造と認定した根拠が乏しいとも思えます。メールを送信した司法修習生が自分の書いた内容ではないと証言したのかもしれないですが、その辺りの経緯も明らかになっていません。

\begin{itemize}
\tightlist
\item
  『代表弁護士がパワハラ、元勤務弁護士へ200万の賠償命令』 -
  弁護士ドットコムタイムズ \url{https://t.co/R1FHLWBomo} 
  元勤務弁護士の代理人を務めた伊藤諭弁護士と髙木亮二弁護士は判決について、「報酬なしの契約内容がありながらも、(社員脱退後の)業務委託報酬が認められたのは評価できる」とした。
\end{itemize}

〉〉〉 kk\_hironoのリツイート 〉〉〉

\begin{itemize}
\item
  RT
  kk\_hirono(刑事告発・非常上告_金沢地方検察庁御中)|itosatoshi110(士業レスキュー【弁護士伊藤諭】)
  日時:2021-06-30 16:48/2021/06/18 21:43 URL:
  \url{https://twitter.com/kk\_hirono/status/1410143251219292161} 
  \url{https://twitter.com/itosatoshi110/status/1405868739212365824} 
  \textgreater{} これは壮絶。
  しかし、こういう事案、程度の差こそあれ暗数はかなりあるんでしょうね。
  私の経験でも逃れた人は口を揃えて「思い出したくもない」といって訴えることをしようとしません。
  \url{https://t.co/Pl6XSlpzi2} 
\item
  「これは現代の奴隷契約だ」と弁護士
  「パワハラで自殺」女性の遺族らが社長ら提訴 \textbar{} ハフポスト
  \url{https://t.co/r7Ep1VfyQi}  ¥\n 2018年10月18日 13時03分 JST \textbar{}
  更新 2018年10月19日 17時18分 JST
\end{itemize}

〉〉〉 kk\_hironoのリツイート 〉〉〉

\begin{itemize}
\tightlist
\item
  RT
  kk\_hirono(刑事告発・非常上告_金沢地方検察庁御中)|itosatoshi110(士業レスキュー【弁護士伊藤諭】)
  日時:2021-06-30 16:52/2021/06/30 01:03 URL:
  \url{https://twitter.com/kk\_hirono/status/1410144145197330432} 
  \url{https://twitter.com/itosatoshi110/status/1409905366834442242} 
  \textgreater{}
  今回は問題になっていませんが、代理人弁護士の立場で、自分の依頼者がもってきた証拠が偽造かも知れないと判断した際の行動も問われます。
  \url{https://t.co/8SGL0eP3ou} 
\end{itemize}

 どこかのタイムラインで見たツイートの内容と思ったのですが、本日6月30日未明の1時3分のツイートとなっています。

\begin{itemize}
\item
  伊藤 諭弁護士(弁護士法人ASK市役所通り法律事務所) - 神奈川県川崎市 -
  弁護士ドットコム \url{https://t.co/sB7aYtVQWi} 
\item
  弁護士のご紹介|弁護士法人ASK \url{https://t.co/jZ34Sio70V} 
\end{itemize}

 時刻は7月1日10時20分です。昨夜は17時過ぎに電話があって、18時頃に出掛けて珠洲市の山奥に蛍を見に行っていました。

 「2016年4月14日夜、熊本地震が起こる前の、深澤諭史弁護士や法クラの「天の怒り?地の声?」と考えさせられるツイート」というエントリーを前のはてなブログで探そうとしていたのですが、イカのモニュメントから遠回りをしていました。

 エントリーは見つからないのですが、告発\市場急配センター殺人未遂事件\金沢地方検察庁・石川県警察御中(@kk\_hirono)の2016年05月24日のツイートが、そのエントリーの内容に対応しているはずです。

\begin{itemize}
\item
  TW kk\_hirono(刑事告発・非常上告_金沢地方検察庁御中) 日時:
  2016/05/24 10:09:40 URL:
  \url{https://twitter.com/kk\_hirono/status/734914225886224384} 
  \textgreater{} ****
  2016年4月14日夜、熊本地震が起こる前の、深澤諭史弁護士や法クラの「天の怒り?地の声?」と考えさせられるツイート
\item
  TW kk\_hirono(刑事告発・非常上告_金沢地方検察庁御中) 日時:
  2016/05/24 10:16:31 URL:
  \url{https://twitter.com/kk\_hirono/status/734915950609190913} 
  \textgreater{}
  それは被災者の側に立った発言というスタイルですが、刑事事件にもそのまま当てはまるもので、それまでちょくちょく目にしてきたものの集大成のようになっていました。歴史的資料、歴史的記録としても意義深いものがあると思います。
\item
  TW kk\_hirono(刑事告発・非常上告_金沢地方検察庁御中) 日時:
  2016/05/24 10:20:02 URL:
  \url{https://twitter.com/kk\_hirono/status/734916836815339521} 
  \textgreater{}
  坂本正幸弁護士のツイートで見かけたのが「西郷隆盛の呪い」という話でした。法クラの間で話題になっているとのことで調べてみましたが、メーリングリストかFacebookの友達間のようなサークル的な閉じられた空間でのやり取りではないかと思え、片鱗にしか触れることはできませんでした。
\end{itemize}

 見出しの後、2行目2つ目のツイートになりますが、坂本正幸弁護士のツイートで「西郷隆盛の呪い」を見かけたとあります。これで坂本正幸弁護士の過去のツイートなど調べていたのですが、該当するようなツイートはありませんでした。

\begin{itemize}
\tightlist
\item
  ``呪い'' (from:sakamotomasayuk) - Twitter検索 / Twitter
  \url{https://twitter.com/search?q=\%22\%E5\%91\%AA\%E3\%81\%84\%22\%20} 
\end{itemize}

 ちょっと検索方法を変えてみました。

\begin{itemize}
\item
  (from:sakamotomasayuk) until:2016-05-24 since:2016-05-23 -
  Twitter検索 / Twitter
  \url{https://twitter.com/search?lang=ja\&q=(from\%3Asakamotomasayuk)\%20until\%3A2016-05-24\%20since\%3A2016-05-23\&src=typed\_query} 
\item
  TW sakamotomasayuk(坂本正幸) 日時: 2016/05/24 08:55:36 URL:
  \url{https://twitter.com/sakamotomasayuk/status/734895589129891840} 
  \textgreater{}
  たぶん警察なんかと同じ感覚で相談したらタダでやってくれると思っているのではと思う>RT
\item
  (from:sakamotomasayuk) until:2016-05-24 since:2016-05-10 -
  Twitter検索 / Twitter
  \url{https://twitter.com/search?q=(from\%3Asakamotomasayuk)\%20until\%3A2016-05-24\%20since\%3A2016-05-10\&src=typed\_query} 
\end{itemize}

 対象範囲を広げたのですが5月22日から24日のツイートまでしか取得できていません。

 「2016年4月14日夜、熊本地震が起こる前の、深澤諭史弁護士や法クラの」とあるので、そちらを調べた方が早いかもしれないです。

\begin{itemize}
\tightlist
\item
  〈〈〈 2021/07/01 10:40:10 Linux Emacs: 〈〈〈
\end{itemize}

\hypertarget{ux88c1ux5224ux54e1ux5236ux5ea6ux3092ux76aeux8089ux308bux6df1ux6fa4ux8aedux53f2ux5f01ux8b77ux58ebux81eaux8eabux306bux3088ux308bux904eux53bbux306eux30c4ux30a4ux30fcux30c8ux306eux30eaux30c4ux30a4ux30fcux30c8ux304bux3089ux518dux767aux898bux306bux81f3ux3063ux305fux65b0ux6f5fux5973ux5150ux6bbaux5bb3ux4e8bux4ef6ux306bux304aux3051ux308bux9ed9ux79d8ux6a29ux3068ux3046ux306eux5b57ux3068ux306eux30b3ux30e9ux30dcux306bux3088ux308bux30d2ux30e3ux30c3ux30cfux30fcux306eux30c4ux30a4ux30fcux30c812}{%
\paragraph{裁判員制度を皮肉る深澤諭史弁護士自身による過去のツイートのリツイートから再発見に至った、新潟女児殺害事件における黙秘権とうの字とのコラボによるヒャッハーのツイート(12)}\label{ux88c1ux5224ux54e1ux5236ux5ea6ux3092ux76aeux8089ux308bux6df1ux6fa4ux8aedux53f2ux5f01ux8b77ux58ebux81eaux8eabux306bux3088ux308bux904eux53bbux306eux30c4ux30a4ux30fcux30c8ux306eux30eaux30c4ux30a4ux30fcux30c8ux304bux3089ux518dux767aux898bux306bux81f3ux3063ux305fux65b0ux6f5fux5973ux5150ux6bbaux5bb3ux4e8bux4ef6ux306bux304aux3051ux308bux9ed9ux79d8ux6a29ux3068ux3046ux306eux5b57ux3068ux306eux30b3ux30e9ux30dcux306bux3088ux308bux30d2ux30e3ux30c3ux30cfux30fcux306eux30c4ux30a4ux30fcux30c812}}

\begin{itemize}
\tightlist
\item
  〉〉〉 Linux Emacs: 2021/07/01 10:42:18 〉〉〉
\end{itemize}

:CATEGORIES: @kanazawabengosi \#金沢弁護士会 @JFBAsns
日本弁護士連合会(日弁連) \#法務省 @MOJ\_HOUMU \#深澤諭史弁護士
\#うの字 \#坂本正幸弁護士

\begin{itemize}
\tightlist
\item
  1440:2021-07-01\_10:40:38 \#告発状 \#\#\#\#
  裁判員制度を皮肉る深澤諭史弁護士自身による過去のツイートのリツイートから再発見に至った、新潟女児殺害事件における黙秘権とうの字とのコラボによるヒャッハーのツイート(11)
  \url{https://hirono-hideki.hatenadiary.jp/entry/2021/07/01/104035} 
\end{itemize}

\begin{lstlisting}
base ❯ d|grep 2016年4月14日
\end{lstlisting}

\begin{itemize}
\item
  2019年06月19日01時26分の登録:
  %@fukazawas 深澤諭史%¥\n深澤諭史¥\n
  @fukazawas¥\n2016年4月14日¥\n¥\n交通事故審理、迅速化へモデルを作成・・・最高裁(読売新聞)
  - Ya
  \url{http://hirono2014sk.blogspot.com/2019/06/fukazawasnn-fukazawasn2016414nn-ya.html} 
\item
  TW fukazawas(深澤諭史) 日時: 2016/04/14 09:36:19 URL:
  \url{https://twitter.com/fukazawas/status/720410321001656320} 
  \textgreater{}
  交通事故審理、迅速化へモデルを作成・・・最高裁(読売新聞) -
  Yahoo!ニュース \url{https://t.co/7gWLmv0pPX}  \#Yahooニュース\\
  \textgreater{}\\
  \textgreater{}
  この記事じゃありませんが,以前,弁護士の金儲け?みたいなトーンの記事がありましたね(・∀・;)
\end{itemize}

 埋め込みツイートの表示が早いと思ったら単体のツイートの記事でした。

\begin{itemize}
\tightlist
\item
  奉納\さらば弁護士鉄道・泥棒神社の物語(@hirono\_hideki)/2016年04月14日
  - Twilog \url{https://t.co/QNIo8Dpl54} 
\end{itemize}

〉〉〉 kk\_hironoのリツイート 〉〉〉

\begin{itemize}
\tightlist
\item
  RT
  kk\_hirono(刑事告発・非常上告_金沢地方検察庁御中)|hirono\_hideki(奉納\さらば弁護士鉄道・泥棒神社の物語)
  日時:2021-07-01 10:48/2016/04/14 20:56 URL:
  \url{https://twitter.com/kk\_hirono/status/1410415024443625472} 
  \url{https://twitter.com/hirono\_hideki/status/720581370116853761} 
  \textgreater{}
  金沢地方検察庁、畝本毅検事正。11日付で就任という。NHKの石川県内ニュースが終わったところ。
\end{itemize}

〉〉〉 kk\_hironoのリツイート 〉〉〉

\begin{itemize}
\tightlist
\item
  RT
  kk\_hirono(刑事告発・非常上告_金沢地方検察庁御中)|hirono\_hideki(奉納\さらば弁護士鉄道・泥棒神社の物語)
  日時:2021-07-01 10:48/2016/04/14 21:33 URL:
  \url{https://twitter.com/kk\_hirono/status/1410415062213267457} 
  \url{https://twitter.com/hirono\_hideki/status/720590849831149568} 
  \textgreater{}
  緊急地震速報の後、1分後ぐらいに熊本地方で震度7という速報。NHK。
\end{itemize}

〉〉〉 kk\_hironoのリツイート 〉〉〉

\begin{itemize}
\tightlist
\item
  RT
  kk\_hirono(刑事告発・非常上告_金沢地方検察庁御中)|kk\_hirono(刑事告発・非常上告_金沢地方検察庁御中)
  日時:2021-07-01 10:50/2016/04/14 09:27 URL:
  \url{https://twitter.com/kk\_hirono/status/1410415555262091266} 
  \url{https://twitter.com/kk\_hirono/status/720408025492627458} 
  \textgreater{} ;; This buffer is for notes you don't want to save, and
  for Lisp evaluation.
\end{itemize}

〉〉〉 kk\_hironoのリツイート 〉〉〉

\begin{itemize}
\tightlist
\item
  RT
  kk\_hirono(刑事告発・非常上告_金沢地方検察庁御中)|hirono\_hideki(奉納\さらば弁護士鉄道・泥棒神社の物語)
  日時:2021-07-01 10:54/2016/04/15 23:19 URL:
  \url{https://twitter.com/kk\_hirono/status/1410416554148126720} 
  \url{https://twitter.com/hirono\_hideki/status/720979902070042629} 
  \textgreater{} モトケンさんはTwitterを使っています:
  ``私は、たいてい学生を殺してたな。主に最前列に座ってた学生w
  \url{https://t.co/JcRTrTwZkm''}  \url{https://t.co/0zcilU9ZS3} 
\end{itemize}

〉〉〉 kk\_hironoのリツイート 〉〉〉

\begin{itemize}
\tightlist
\item
  RT
  kk\_hirono(刑事告発・非常上告_金沢地方検察庁御中)|hirono\_hideki(奉納\さらば弁護士鉄道・泥棒神社の物語)
  日時:2021-07-01 10:54/2016/04/15 23:22 URL:
  \url{https://twitter.com/kk\_hirono/status/1410416585685143553} 
  \url{https://twitter.com/hirono\_hideki/status/720980618041950208} 
  \textgreater{} .@motoken\_tw
  法律家としての完全成仏の実刑刑務所ぐらしで、責任をとってもらいますよ。ブロックされて上で言いたい放題の、完全無視をされているモトケンこと矢部善朗弁護士(京都弁護士会)さん。貴方や家族の人生の完全破滅を警察、検察に求める手続きになることをご理解求めます。
\end{itemize}

〉〉〉 kk\_hironoのリツイート 〉〉〉

\begin{itemize}
\tightlist
\item
  RT
  kk\_hirono(刑事告発・非常上告_金沢地方検察庁御中)|hirono\_hideki(奉納\さらば弁護士鉄道・泥棒神社の物語)
  日時:2021-07-01 10:55/2016/04/15 23:36 URL:
  \url{https://twitter.com/kk\_hirono/status/1410416709228392449} 
  \url{https://twitter.com/hirono\_hideki/status/720984190246903809} 
  \textgreater{}
  能登町宇出津白山神社。曳山祭り人形見。「傾城 高尾太夫 身請け話」。2016年4月15日。
\end{itemize}

〉〉〉 kk\_hironoのリツイート 〉〉〉

\begin{itemize}
\item
  RT
  kk\_hirono(刑事告発・非常上告_金沢地方検察庁御中)|hirono\_hideki(奉納\さらば弁護士鉄道・泥棒神社の物語)
  日時:2021-07-01 10:55/2016/04/15 23:40 URL:
  \url{https://twitter.com/kk\_hirono/status/1410416799863107586} 
  \url{https://twitter.com/hirono\_hideki/status/720985149614075904} 
  \textgreater{} .@fukazawas
  貴方こと深澤諭史弁護士は、まず人間として社会汚染の大汚染だと思います。ブロックしているようですが、いいたいことがあれば是非。大々的に取り上げて差し上げます。
\item
  刑事告発・非常上告_金沢地方検察庁御中(@kk\_hirono)/2016年04月14日 -
  Twilog \url{https://t.co/RyqiBtF25C} 
\item
  非常上告-最高検察庁御中\_ツイッター(@s\_hirono)/2016年04月14日 -
  Twilog \url{https://t.co/wMV2eXh3dR}  ツイートが見つかりませんでした
\item
  奉納\さらば弁護士鉄道・泥棒神社の物語(@hirono\_hideki)/2016年04月15日
  - Twilog \url{https://t.co/QQqi6XXtG1} 
\end{itemize}

 熊本地震の発生が2016年4月14日らしいということはわかったのですが、当日に目立った弁護士のツイートを記録したものは見当たりませんでした。4月15日に宇出津の曳山祭で仙台高尾の人形見とありました。

 熊本地震は、最初の大地震の2,3日後にさらに大きな余震が発生して大きな被害を出したという他には見たことのないニュースの地震であったという記憶があります。

\begin{itemize}
\tightlist
\item
  熊本地震 (2016年) - Wikipedia \url{https://t.co/QcLjfi5gQj}  ¥\n
  気象庁震度階級では最も大きい震度7を観測する地震が4月14日夜(前記時刻)および4月16日未明に発生したほか、最大震度が6強の地震が2回、6弱の地震が3回発生している{[}6{]}。
\end{itemize}

 なぜ坂本正幸弁護士のツイートに西郷隆盛の呪いなのか理解できなかったのですが、これで合点がいきました。坂本正幸弁護士は2016年4月当時まだ鹿児島大学の教授か准教授だった可能性もあります。

\begin{itemize}
\tightlist
\item
  〈〈〈 2021/07/01 11:09:53 Linux Emacs: 〈〈〈
\end{itemize}

\hypertarget{ux88c1ux5224ux54e1ux5236ux5ea6ux3092ux76aeux8089ux308bux6df1ux6fa4ux8aedux53f2ux5f01ux8b77ux58ebux81eaux8eabux306bux3088ux308bux904eux53bbux306eux30c4ux30a4ux30fcux30c8ux306eux30eaux30c4ux30a4ux30fcux30c8ux304bux3089ux518dux767aux898bux306bux81f3ux3063ux305fux65b0ux6f5fux5973ux5150ux6bbaux5bb3ux4e8bux4ef6ux306bux304aux3051ux308bux9ed9ux79d8ux6a29ux3068ux3046ux306eux5b57ux3068ux306eux30b3ux30e9ux30dcux306bux3088ux308bux30d2ux30e3ux30c3ux30cfux30fcux306eux30c4ux30a4ux30fcux30c812-1}{%
\paragraph{裁判員制度を皮肉る深澤諭史弁護士自身による過去のツイートのリツイートから再発見に至った、新潟女児殺害事件における黙秘権とうの字とのコラボによるヒャッハーのツイート(12)}\label{ux88c1ux5224ux54e1ux5236ux5ea6ux3092ux76aeux8089ux308bux6df1ux6fa4ux8aedux53f2ux5f01ux8b77ux58ebux81eaux8eabux306bux3088ux308bux904eux53bbux306eux30c4ux30a4ux30fcux30c8ux306eux30eaux30c4ux30a4ux30fcux30c8ux304bux3089ux518dux767aux898bux306bux81f3ux3063ux305fux65b0ux6f5fux5973ux5150ux6bbaux5bb3ux4e8bux4ef6ux306bux304aux3051ux308bux9ed9ux79d8ux6a29ux3068ux3046ux306eux5b57ux3068ux306eux30b3ux30e9ux30dcux306bux3088ux308bux30d2ux30e3ux30c3ux30cfux30fcux306eux30c4ux30a4ux30fcux30c812-1}}

\begin{itemize}
\tightlist
\item
  〉〉〉 Linux Emacs: 2021/07/01 11:13:42 〉〉〉
\end{itemize}

:CATEGORIES: @kanazawabengosi \#金沢弁護士会 @JFBAsns
日本弁護士連合会(日弁連) \#法務省 @MOJ\_HOUMU \#深澤諭史弁護士
\#うの字 \#坂本正幸弁護士 \#モトケンこと矢部善朗弁護士(京都弁護士会)

\begin{itemize}
\tightlist
\item
  1441:2021-07-01\_11:10:44 \#告発状 \#\#\#\#
  裁判員制度を皮肉る深澤諭史弁護士自身による過去のツイートのリツイートから再発見に至った、新潟女児殺害事件における黙秘権とうの字とのコラボによるヒャッハーのツイート(12)
  \url{https://hirono-hideki.hatenadiary.jp/entry/2021/07/01/111042} 
\end{itemize}

 2016年4月という熊本地震と、曳山が横転し石川県内各地で被害を出した強風が同じ時期で余震の最中というのはまったく意外でした。強風で朝方は曇りがちでしたが、雨が降ることはなく外は割と明るかったと記憶にあります。

〉〉〉 kk\_hironoのリツイート 〉〉〉

\begin{itemize}
\tightlist
\item
  RT
  kk\_hirono(刑事告発・非常上告_金沢地方検察庁御中)|s\_hirono(非常上告-最高検察庁御中\_ツイッター)
  日時:2021-07-01 11:52/2021/07/01 11:44 URL:
  \url{https://twitter.com/kk\_hirono/status/1410431082458087424} 
  \url{https://twitter.com/s\_hirono/status/1410429070916030471} 
  \textgreater{} 2016-04-29\_20.39.48.jpg \url{https://t.co/PixLtpi3Wn} 
\end{itemize}

〉〉〉 kk\_hironoのリツイート 〉〉〉

\begin{itemize}
\tightlist
\item
  RT
  kk\_hirono(刑事告発・非常上告_金沢地方検察庁御中)|s\_hirono(非常上告-最高検察庁御中\_ツイッター)
  日時:2021-07-01 11:52/2021/07/01 11:44 URL:
  \url{https://twitter.com/kk\_hirono/status/1410431102565589002} 
  \url{https://twitter.com/s\_hirono/status/1410429057024466949} 
  \textgreater{} 2016-04-29\_09.34.21.jpg \url{https://t.co/Q3I4gq94mM} 
\end{itemize}

〉〉〉 kk\_hironoのリツイート 〉〉〉

\begin{itemize}
\tightlist
\item
  RT
  kk\_hirono(刑事告発・非常上告_金沢地方検察庁御中)|s\_hirono(非常上告-最高検察庁御中\_ツイッター)
  日時:2021-07-01 11:52/2021/07/01 11:44 URL:
  \url{https://twitter.com/kk\_hirono/status/1410431118164201475} 
  \url{https://twitter.com/s\_hirono/status/1410429042101129220} 
  \textgreater{} 2016-04-24\_18.44.15.jpg \url{https://t.co/caqnOh5CF4} 
\end{itemize}

〉〉〉 kk\_hironoのリツイート 〉〉〉

\begin{itemize}
\tightlist
\item
  RT
  kk\_hirono(刑事告発・非常上告_金沢地方検察庁御中)|s\_hirono(非常上告-最高検察庁御中\_ツイッター)
  日時:2021-07-01 11:52/2021/07/01 11:44 URL:
  \url{https://twitter.com/kk\_hirono/status/1410431135931240449} 
  \url{https://twitter.com/s\_hirono/status/1410429028121473025} 
  \textgreater{} 2016-04-21\_15.15.17.jpg \url{https://t.co/lTxQz1CF8P} 
\end{itemize}

〉〉〉 kk\_hironoのリツイート 〉〉〉

\begin{itemize}
\tightlist
\item
  RT
  kk\_hirono(刑事告発・非常上告_金沢地方検察庁御中)|s\_hirono(非常上告-最高検察庁御中\_ツイッター)
  日時:2021-07-01 11:52/2021/07/01 11:44 URL:
  \url{https://twitter.com/kk\_hirono/status/1410431149504008193} 
  \url{https://twitter.com/s\_hirono/status/1410429013818892291} 
  \textgreater{} 2016-04-21\_14.53.17.jpg \url{https://t.co/rubelikY2V} 
\end{itemize}

〉〉〉 kk\_hironoのリツイート 〉〉〉

\begin{itemize}
\tightlist
\item
  RT
  kk\_hirono(刑事告発・非常上告_金沢地方検察庁御中)|s\_hirono(非常上告-最高検察庁御中\_ツイッター)
  日時:2021-07-01 11:52/2021/07/01 11:44 URL:
  \url{https://twitter.com/kk\_hirono/status/1410431168323854337} 
  \url{https://twitter.com/s\_hirono/status/1410428999130517506} 
  \textgreater{} 2016-04-21\_14.32.04.jpg \url{https://t.co/tcl1VfLkR7} 
\end{itemize}

〉〉〉 kk\_hironoのリツイート 〉〉〉

\begin{itemize}
\tightlist
\item
  RT
  kk\_hirono(刑事告発・非常上告_金沢地方検察庁御中)|s\_hirono(非常上告-最高検察庁御中\_ツイッター)
  日時:2021-07-01 11:52/2021/07/01 11:44 URL:
  \url{https://twitter.com/kk\_hirono/status/1410431182051823624} 
  \url{https://twitter.com/s\_hirono/status/1410428984530141190} 
  \textgreater{} 2016-04-20\_13.44.05.jpg \url{https://t.co/Mj4w3iLkgQ} 
\end{itemize}

〉〉〉 kk\_hironoのリツイート 〉〉〉

\begin{itemize}
\tightlist
\item
  RT
  kk\_hirono(刑事告発・非常上告_金沢地方検察庁御中)|s\_hirono(非常上告-最高検察庁御中\_ツイッター)
  日時:2021-07-01 11:52/2021/07/01 11:44 URL:
  \url{https://twitter.com/kk\_hirono/status/1410431195637190669} 
  \url{https://twitter.com/s\_hirono/status/1410428970290405376} 
  \textgreater{} 2016-04-17\_16.43.20_能登宇出津曳山祭り.jpg
  \url{https://t.co/i3bh2106J5} 
\end{itemize}

〉〉〉 kk\_hironoのリツイート 〉〉〉

\begin{itemize}
\tightlist
\item
  RT
  kk\_hirono(刑事告発・非常上告_金沢地方検察庁御中)|s\_hirono(非常上告-最高検察庁御中\_ツイッター)
  日時:2021-07-01 11:52/2021/07/01 11:44 URL:
  \url{https://twitter.com/kk\_hirono/status/1410431210468171778} 
  \url{https://twitter.com/s\_hirono/status/1410428955581059072} 
  \textgreater{} 2016-04-17\_12.19.33-1_能登宇出津曳山祭り.jpg
  \url{https://t.co/q1dplyJJ5W} 
\end{itemize}

〉〉〉 kk\_hironoのリツイート 〉〉〉

\begin{itemize}
\tightlist
\item
  RT
  kk\_hirono(刑事告発・非常上告_金沢地方検察庁御中)|s\_hirono(非常上告-最高検察庁御中\_ツイッター)
  日時:2021-07-01 11:53/2021/07/01 11:43 URL:
  \url{https://twitter.com/kk\_hirono/status/1410431226201079808} 
  \url{https://twitter.com/s\_hirono/status/1410428940871561219} 
  \textgreater{} 2016-04-17\_12.07.20_能登宇出津曳山祭り.jpg
  \url{https://t.co/SZhla58aZx} 
\end{itemize}

〉〉〉 kk\_hironoのリツイート 〉〉〉

\begin{itemize}
\tightlist
\item
  RT
  kk\_hirono(刑事告発・非常上告_金沢地方検察庁御中)|s\_hirono(非常上告-最高検察庁御中\_ツイッター)
  日時:2021-07-01 11:53/2021/07/01 11:43 URL:
  \url{https://twitter.com/kk\_hirono/status/1410431238775590919} 
  \url{https://twitter.com/s\_hirono/status/1410428926132854784} 
  \textgreater{} 2016-04-17\_12.06.02_能登宇出津曳山祭り.jpg
  \url{https://t.co/ZXtMNkNFeh} 
\end{itemize}

〉〉〉 kk\_hironoのリツイート 〉〉〉

\begin{itemize}
\tightlist
\item
  RT
  kk\_hirono(刑事告発・非常上告_金沢地方検察庁御中)|s\_hirono(非常上告-最高検察庁御中\_ツイッター)
  日時:2021-07-01 11:53/2021/07/01 11:43 URL:
  \url{https://twitter.com/kk\_hirono/status/1410431252696506372} 
  \url{https://twitter.com/s\_hirono/status/1410428911054323714} 
  \textgreater{} 2016-04-17\_12.01.52_能登宇出津曳山祭り.jpg
  \url{https://t.co/7W18kPFZTZ} 
\end{itemize}

〉〉〉 kk\_hironoのリツイート 〉〉〉

\begin{itemize}
\tightlist
\item
  RT
  kk\_hirono(刑事告発・非常上告_金沢地方検察庁御中)|s\_hirono(非常上告-最高検察庁御中\_ツイッター)
  日時:2021-07-01 11:53/2021/07/01 11:43 URL:
  \url{https://twitter.com/kk\_hirono/status/1410431267108130820} 
  \url{https://twitter.com/s\_hirono/status/1410428896755929089} 
  \textgreater{} 2016-04-17\_12.00.59_能登宇出津曳山祭り.jpg
  \url{https://t.co/BJD1pC6zuz} 
\end{itemize}

〉〉〉 kk\_hironoのリツイート 〉〉〉

\begin{itemize}
\tightlist
\item
  RT
  kk\_hirono(刑事告発・非常上告_金沢地方検察庁御中)|s\_hirono(非常上告-最高検察庁御中\_ツイッター)
  日時:2021-07-01 11:53/2021/07/01 11:43 URL:
  \url{https://twitter.com/kk\_hirono/status/1410431281389707266} 
  \url{https://twitter.com/s\_hirono/status/1410428882126213120} 
  \textgreater{} 2016-04-17\_07.23.04_能登宇出津曳山祭り.jpg
  \url{https://t.co/sunAOmdmc8} 
\end{itemize}

〉〉〉 kk\_hironoのリツイート 〉〉〉

\begin{itemize}
\tightlist
\item
  RT
  kk\_hirono(刑事告発・非常上告_金沢地方検察庁御中)|s\_hirono(非常上告-最高検察庁御中\_ツイッター)
  日時:2021-07-01 11:53/2021/07/01 11:43 URL:
  \url{https://twitter.com/kk\_hirono/status/1410431298506616835} 
  \url{https://twitter.com/s\_hirono/status/1410428867542618116} 
  \textgreater{} 2016-04-17\_06.10.10_能登宇出津曳山祭り.jpg
  \url{https://t.co/EnX1k6ceQb} 
\end{itemize}

〉〉〉 kk\_hironoのリツイート 〉〉〉

\begin{itemize}
\tightlist
\item
  RT
  kk\_hirono(刑事告発・非常上告_金沢地方検察庁御中)|s\_hirono(非常上告-最高検察庁御中\_ツイッター)
  日時:2021-07-01 11:53/2021/07/01 11:43 URL:
  \url{https://twitter.com/kk\_hirono/status/1410431312075235329} 
  \url{https://twitter.com/s\_hirono/status/1410428853089030144} 
  \textgreater{} 2016-04-17\_04.48.29_能登宇出津曳山祭り.jpg
  \url{https://t.co/BudhJvV1UL} 
\end{itemize}

〉〉〉 kk\_hironoのリツイート 〉〉〉

\begin{itemize}
\tightlist
\item
  RT
  kk\_hirono(刑事告発・非常上告_金沢地方検察庁御中)|s\_hirono(非常上告-最高検察庁御中\_ツイッター)
  日時:2021-07-01 11:53/2021/07/01 11:43 URL:
  \url{https://twitter.com/kk\_hirono/status/1410431326256209920} 
  \url{https://twitter.com/s\_hirono/status/1410428839151300614} 
  \textgreater{} 2016-04-17\_01.57.27_能登宇出津曳山祭り.jpg
  \url{https://t.co/vfyVHPCpcW} 
\end{itemize}

〉〉〉 kk\_hironoのリツイート 〉〉〉

\begin{itemize}
\tightlist
\item
  RT
  kk\_hirono(刑事告発・非常上告_金沢地方検察庁御中)|s\_hirono(非常上告-最高検察庁御中\_ツイッター)
  日時:2021-07-01 11:53/2021/07/01 11:43 URL:
  \url{https://twitter.com/kk\_hirono/status/1410431344987885571} 
  \url{https://twitter.com/s\_hirono/status/1410428825112965122} 
  \textgreater{} 2016-04-16\_17.28.04_能登宇出津曳山祭り.jpg
  \url{https://t.co/aOGGCnCO7a} 
\end{itemize}

〉〉〉 kk\_hironoのリツイート 〉〉〉

\begin{itemize}
\tightlist
\item
  RT
  kk\_hirono(刑事告発・非常上告_金沢地方検察庁御中)|s\_hirono(非常上告-最高検察庁御中\_ツイッター)
  日時:2021-07-01 11:53/2021/07/01 11:43 URL:
  \url{https://twitter.com/kk\_hirono/status/1410431360917938179} 
  \url{https://twitter.com/s\_hirono/status/1410428810676166659} 
  \textgreater{} 2016-04-16\_13.48.35_能登宇出津曳山祭り.jpg
  \url{https://t.co/ytkOWGQzRf} 
\end{itemize}

〉〉〉 kk\_hironoのリツイート 〉〉〉

\begin{itemize}
\tightlist
\item
  RT
  kk\_hirono(刑事告発・非常上告_金沢地方検察庁御中)|s\_hirono(非常上告-最高検察庁御中\_ツイッター)
  日時:2021-07-01 11:53/2021/07/01 11:43 URL:
  \url{https://twitter.com/kk\_hirono/status/1410431378345271297} 
  \url{https://twitter.com/s\_hirono/status/1410428795870269440} 
  \textgreater{} 2016-04-16\_13.43.05_能登宇出津曳山祭り.jpg
  \url{https://t.co/997P0MEh2N} 
\end{itemize}

〉〉〉 kk\_hironoのリツイート 〉〉〉

\begin{itemize}
\tightlist
\item
  RT
  kk\_hirono(刑事告発・非常上告_金沢地方検察庁御中)|s\_hirono(非常上告-最高検察庁御中\_ツイッター)
  日時:2021-07-01 11:53/2021/07/01 11:43 URL:
  \url{https://twitter.com/kk\_hirono/status/1410431408170885124} 
  \url{https://twitter.com/s\_hirono/status/1410428781316087808} 
  \textgreater{}
  2016-04-15\_17.39.17_宇出津曳山祭 人形見 仙台高尾.jpg
  \url{https://t.co/KzVMx4vAKL} 
\end{itemize}

〉〉〉 kk\_hironoのリツイート 〉〉〉

\begin{itemize}
\tightlist
\item
  RT
  kk\_hirono(刑事告発・非常上告_金沢地方検察庁御中)|s\_hirono(非常上告-最高検察庁御中\_ツイッター)
  日時:2021-07-01 11:53/2021/07/01 11:43 URL:
  \url{https://twitter.com/kk\_hirono/status/1410431426130964483} 
  \url{https://twitter.com/s\_hirono/status/1410428766581452805} 
  \textgreater{} 2016-04-08\_17.04.58_辺田の浜・神目神社.jpg
  \url{https://t.co/zOCbt650tp} 
\end{itemize}

〉〉〉 kk\_hironoのリツイート 〉〉〉

\begin{itemize}
\tightlist
\item
  RT
  kk\_hirono(刑事告発・非常上告_金沢地方検察庁御中)|s\_hirono(非常上告-最高検察庁御中\_ツイッター)
  日時:2021-07-01 11:53/2021/07/01 11:43 URL:
  \url{https://twitter.com/kk\_hirono/status/1410431456782938114} 
  \url{https://twitter.com/s\_hirono/status/1410428750676697088} 
  \textgreater{} 2016-04-08\_17.03.40_辺田の浜・神目神社.jpg
  \url{https://t.co/RkW87kg4i6} 
\end{itemize}

〉〉〉 kk\_hironoのリツイート 〉〉〉

\begin{itemize}
\tightlist
\item
  RT
  kk\_hirono(刑事告発・非常上告_金沢地方検察庁御中)|s\_hirono(非常上告-最高検察庁御中\_ツイッター)
  日時:2021-07-01 11:54/2021/07/01 11:43 URL:
  \url{https://twitter.com/kk\_hirono/status/1410431478039674885} 
  \url{https://twitter.com/s\_hirono/status/1410428736269213699} 
  \textgreater{} 2016-04-08\_17.02.29_辺田の浜・神目神社.jpg
  \url{https://t.co/QEccDNmaDR} 
\end{itemize}

〉〉〉 kk\_hironoのリツイート 〉〉〉

\begin{itemize}
\tightlist
\item
  RT
  kk\_hirono(刑事告発・非常上告_金沢地方検察庁御中)|s\_hirono(非常上告-最高検察庁御中\_ツイッター)
  日時:2021-07-01 11:54/2021/07/01 11:43 URL:
  \url{https://twitter.com/kk\_hirono/status/1410431495127273473} 
  \url{https://twitter.com/s\_hirono/status/1410428721232678912} 
  \textgreater{} 2016-04-08\_16.22.53.jpg \url{https://t.co/F63fRLOMoA} 
\end{itemize}

〉〉〉 kk\_hironoのリツイート 〉〉〉

\begin{itemize}
\tightlist
\item
  RT
  kk\_hirono(刑事告発・非常上告_金沢地方検察庁御中)|s\_hirono(非常上告-最高検察庁御中\_ツイッター)
  日時:2021-07-01 11:54/2021/07/01 11:43 URL:
  \url{https://twitter.com/kk\_hirono/status/1410431509450747905} 
  \url{https://twitter.com/s\_hirono/status/1410428705332105218} 
  \textgreater{} 2016-04-08\_16.22.20.jpg \url{https://t.co/SxJLmQvvK4} 
\end{itemize}

〉〉〉 kk\_hironoのリツイート 〉〉〉

\begin{itemize}
\tightlist
\item
  RT
  kk\_hirono(刑事告発・非常上告_金沢地方検察庁御中)|s\_hirono(非常上告-最高検察庁御中\_ツイッター)
  日時:2021-07-01 11:54/2021/07/01 11:42 URL:
  \url{https://twitter.com/kk\_hirono/status/1410431523606593540} 
  \url{https://twitter.com/s\_hirono/status/1410428690199027712} 
  \textgreater{} 2016-04-08\_16.18.44.jpg \url{https://t.co/ajuJ17GZAC} 
\end{itemize}

〉〉〉 kk\_hironoのリツイート 〉〉〉

\begin{itemize}
\tightlist
\item
  RT
  kk\_hirono(刑事告発・非常上告_金沢地方検察庁御中)|s\_hirono(非常上告-最高検察庁御中\_ツイッター)
  日時:2021-07-01 11:54/2021/07/01 11:42 URL:
  \url{https://twitter.com/kk\_hirono/status/1410431537082888192} 
  \url{https://twitter.com/s\_hirono/status/1410428675451879428} 
  \textgreater{} 2016-04-08\_16.17.30.jpg \url{https://t.co/kmh5k3hC5p} 
\end{itemize}

〉〉〉 kk\_hironoのリツイート 〉〉〉

\begin{itemize}
\tightlist
\item
  RT
  kk\_hirono(刑事告発・非常上告_金沢地方検察庁御中)|s\_hirono(非常上告-最高検察庁御中\_ツイッター)
  日時:2021-07-01 11:54/2021/07/01 11:42 URL:
  \url{https://twitter.com/kk\_hirono/status/1410431549787410438} 
  \url{https://twitter.com/s\_hirono/status/1410428660683661318} 
  \textgreater{} 2016-04-08\_16.16.24.jpg \url{https://t.co/bdmQ7gWdMU} 
\end{itemize}

〉〉〉 kk\_hironoのリツイート 〉〉〉

\begin{itemize}
\tightlist
\item
  RT
  kk\_hirono(刑事告発・非常上告_金沢地方検察庁御中)|s\_hirono(非常上告-最高検察庁御中\_ツイッター)
  日時:2021-07-01 11:54/2021/07/01 11:42 URL:
  \url{https://twitter.com/kk\_hirono/status/1410431562244521985} 
  \url{https://twitter.com/s\_hirono/status/1410428646217555970} 
  \textgreater{} 2016-04-07\_18.25.28.jpg \url{https://t.co/nqyPIldSgN} 
\end{itemize}

〉〉〉 kk\_hironoのリツイート 〉〉〉

\begin{itemize}
\tightlist
\item
  RT
  kk\_hirono(刑事告発・非常上告_金沢地方検察庁御中)|s\_hirono(非常上告-最高検察庁御中\_ツイッター)
  日時:2021-07-01 11:54/2021/07/01 11:42 URL:
  \url{https://twitter.com/kk\_hirono/status/1410431575792119808} 
  \url{https://twitter.com/s\_hirono/status/1410428631751352320} 
  \textgreater{} 2016-04-07\_18.24.57.jpg \url{https://t.co/URFjdBVcAH} 
\end{itemize}

〉〉〉 kk\_hironoのリツイート 〉〉〉

\begin{itemize}
\tightlist
\item
  RT
  kk\_hirono(刑事告発・非常上告_金沢地方検察庁御中)|s\_hirono(非常上告-最高検察庁御中\_ツイッター)
  日時:2021-07-01 11:54/2021/07/01 11:42 URL:
  \url{https://twitter.com/kk\_hirono/status/1410431589104832521} 
  \url{https://twitter.com/s\_hirono/status/1410428617398517760} 
  \textgreater{} 2016-04-07\_18.20.45.jpg \url{https://t.co/oOk6LluJWu} 
\end{itemize}

〉〉〉 kk\_hironoのリツイート 〉〉〉

\begin{itemize}
\tightlist
\item
  RT
  kk\_hirono(刑事告発・非常上告_金沢地方検察庁御中)|s\_hirono(非常上告-最高検察庁御中\_ツイッター)
  日時:2021-07-01 11:54/2021/07/01 11:42 URL:
  \url{https://twitter.com/kk\_hirono/status/1410431604955127812} 
  \url{https://twitter.com/s\_hirono/status/1410428603062444033} 
  \textgreater{} 2016-04-07\_18.17.28.jpg \url{https://t.co/aZch0qL70B} 
\end{itemize}

〉〉〉 kk\_hironoのリツイート 〉〉〉

\begin{itemize}
\tightlist
\item
  RT
  kk\_hirono(刑事告発・非常上告_金沢地方検察庁御中)|s\_hirono(非常上告-最高検察庁御中\_ツイッター)
  日時:2021-07-01 11:54/2021/07/01 11:42 URL:
  \url{https://twitter.com/kk\_hirono/status/1410431623946899462} 
  \url{https://twitter.com/s\_hirono/status/1410428588378132481} 
  \textgreater{} 2016-04-07\_18.16.17.jpg \url{https://t.co/qKC3MPopZq} 
\end{itemize}

〉〉〉 kk\_hironoのリツイート 〉〉〉

\begin{itemize}
\item
  RT
  kk\_hirono(刑事告発・非常上告_金沢地方検察庁御中)|s\_hirono(非常上告-最高検察庁御中\_ツイッター)
  日時:2021-07-01 11:54/2021/07/01 11:42 URL:
  \url{https://twitter.com/kk\_hirono/status/1410431662790385666} 
  \url{https://twitter.com/s\_hirono/status/1410428574323015680} 
  \textgreater{}
  2015-10-07-170518\_モトケンさんはTwitterを使っています: ''現実的対応としては、最寄りの警察に相談に行って刑事さんとお友達になるという手もあります。 R.jpg
  \url{https://t.co/TJZ94jlSnO} 
\item
  〈〈〈 2021/07/01 11:55:27 Linux Emacs: 〈〈〈
\end{itemize}

\hypertarget{ux88c1ux5224ux54e1ux5236ux5ea6ux3092ux76aeux8089ux308bux6df1ux6fa4ux8aedux53f2ux5f01ux8b77ux58ebux81eaux8eabux306bux3088ux308bux904eux53bbux306eux30c4ux30a4ux30fcux30c8ux306eux30eaux30c4ux30a4ux30fcux30c8ux304bux3089ux518dux767aux898bux306bux81f3ux3063ux305fux65b0ux6f5fux5973ux5150ux6bbaux5bb3ux4e8bux4ef6ux306bux304aux3051ux308bux9ed9ux79d8ux6a29ux3068ux3046ux306eux5b57ux3068ux306eux30b3ux30e9ux30dcux306bux3088ux308bux30d2ux30e3ux30c3ux30cfux30fcux306eux30c4ux30a4ux30fcux30c814}{%
\paragraph{裁判員制度を皮肉る深澤諭史弁護士自身による過去のツイートのリツイートから再発見に至った、新潟女児殺害事件における黙秘権とうの字とのコラボによるヒャッハーのツイート(14)}\label{ux88c1ux5224ux54e1ux5236ux5ea6ux3092ux76aeux8089ux308bux6df1ux6fa4ux8aedux53f2ux5f01ux8b77ux58ebux81eaux8eabux306bux3088ux308bux904eux53bbux306eux30c4ux30a4ux30fcux30c8ux306eux30eaux30c4ux30a4ux30fcux30c8ux304bux3089ux518dux767aux898bux306bux81f3ux3063ux305fux65b0ux6f5fux5973ux5150ux6bbaux5bb3ux4e8bux4ef6ux306bux304aux3051ux308bux9ed9ux79d8ux6a29ux3068ux3046ux306eux5b57ux3068ux306eux30b3ux30e9ux30dcux306bux3088ux308bux30d2ux30e3ux30c3ux30cfux30fcux306eux30c4ux30a4ux30fcux30c814}}

\begin{itemize}
\tightlist
\item
  〉〉〉 Linux Emacs: 2021/07/01 12:08:08 〉〉〉
\end{itemize}

:CATEGORIES: @kanazawabengosi \#金沢弁護士会 @JFBAsns
日本弁護士連合会(日弁連) \#法務省 @MOJ\_HOUMU \#深澤諭史弁護士
\#うの字 \#坂本正幸弁護士 \#モトケンこと矢部善朗弁護士(京都弁護士会)

\begin{itemize}
\item
  1442:2021-07-01\_11:56:04 \#告発状 \#\#\#\#
  裁判員制度を皮肉る深澤諭史弁護士自身による過去のツイートのリツイートから再発見に至った、新潟女児殺害事件における黙秘権とうの字とのコラボによるヒャッハーのツイート(13)
  \url{https://hirono-hideki.hatenadiary.jp/entry/2021/07/01/115601} 
\item
  (from:fukazawas) until:2016-04-20 since:2016-04-14 - Twitter検索 /
  Twitter
  \url{https://twitter.com/search?lang=ja\&q=(from\%3Afukazawas)\%20untl\%3A2016-04-20\%20since\%3A2016-04-14\&src=typed\_query} 
\item
  TW fukazawas(深澤諭史) 日時: 2016/04/15 18:45:28 URL:
  \url{https://twitter.com/fukazawas/status/720910904443871232} 
  \textgreater{}
  (^ω^)地方はエルドラドだお!仕事一杯需要一杯,夢一杯!!\\
  \textgreater{}\\
  \textgreater{} *某地方にて\\
  \textgreater{}
  (;#・∀・)(イノシシを追いかけながら)まてーー!!今晩の当職の晩飯ーーー!!!\\
  \textgreater{}\\
  \textgreater{} \#司法改革コント
\item
  TW fukazawas(深澤諭史) 日時: 2016/04/14 09:37:11 URL:
  \url{https://twitter.com/fukazawas/status/720410535943020545} 
  \textgreater{}
  あれだけ司法改革で,弁護士に気軽に依頼できる,頼りがいがあるって,喧伝したのに,いざ小さい事件まで「法の光」が行き渡ると,非難されるとか,なにをいいたいのかよくわからなかったあのとき(・∀・;)
\item
  TW fukazawas(深澤諭史) 日時: 2016/04/17 12:02:28 URL:
  \url{https://twitter.com/fukazawas/status/721534264429268995} 
  \textgreater{}
  所得の移転先が移転元から、「淘汰されろ」だのなんだのとボロクソ言われた挙げ句、被害者面されるとか、今の法曹志願者・新人は、きっと前世でかなり悪いことをしたんだと思う。\\
  \textgreater{}
  これに堪え忍んで、ひたすら他人様のお役に立てば、前世の悪行をぬぐい去り、なんとか成仏できるというものであろう。
\item
  TW fukazawas(深澤諭史) 日時: 2016/04/19 13:16:32 URL:
  \url{https://twitter.com/fukazawas/status/722277678368169984} 
  \textgreater{} ◯報道の公正とは何か??\\
  \textgreater{}
  それは,視聴者提供の写真を利用しても代金を支払わないことであり,\\
  \textgreater{}
  それは,Twitterアカウントで視聴者をフォローしないことであり,\\
  \textgreater{}
  その上で,トップが,現職の首相と会食をして寿司友になることである。
\item
  TW fukazawas(深澤諭史) 日時: 2016/04/18 12:51:05 URL:
  \url{https://twitter.com/fukazawas/status/721908884911181824} 
  \textgreater{}
  問題は,ちゃんと受診してくれるかどうか,というところですね。\\
  \textgreater{}
  症状が悪化すれば「全ては捜査機関や裁判所の陰謀で,あいつらが邪魔しなければ,俺は被害者と一緒になれたんだ!」とか思い込んだりするわけで,そういう人は治療には応じない。ここが本人も離脱したがっている覚醒剤依存とは違う。
\item
  TW fukazawas(深澤諭史) 日時: 2016/04/19 10:00:15 URL:
  \url{https://twitter.com/fukazawas/status/722228280930078720} 
  \textgreater{} わたしも例のアカウントにブロックされていた。\\
  \textgreater{} 非弁業者以外にブロックされたことは,ほぼ無いのだが。\\
  \textgreater{} (・∀・;;)
\item
  TW fukazawas(深澤諭史) 日時: 2016/04/18 12:46:56 URL:
  \url{https://twitter.com/fukazawas/status/721907843410976768} 
  \textgreater{}
  ストーカー被害、``治療''で歯止め 福岡県警が新たな取り組み(西日本新聞)
  - Yahoo!ニュース \url{https://t.co/Mqeu0B0DI6}  \#Yahooニュース\\
  \textgreater{}\\
  \textgreater{}
  「相手の気持ちを無視してつきまとうストーカー行為の防止には、感情の制御方法を身に付ける認知行動療法が有効」
\item
  TW fukazawas(深澤諭史) 日時: 2016/04/18 12:48:44 URL:
  \url{https://twitter.com/fukazawas/status/721908294982307840} 
  \textgreater{}
  ストーカー規制法の行為一覧をみても明らかだけど,つきまとい,いやがらせ,情報収集してそれをアピール,そんなことをやっても,誰一人好意を抱いてくれない,支持を得られないわけで,それでもそれを続けるというのは病的という他ない。\\
  \textgreater{}
  いくら処罰しても逆恨みになれば悪化するし,治療こそ大事。
\end{itemize}

 直接、熊本地震に関連する深澤諭史弁護士のツイートは見当たりませんでしたが、リツイートなどはけっこうあったと記憶します。ストーカーと同時期というのは記憶になかったですが、たびたび、ストーカーのことは治療と絡めて取り沙汰していました。

 「>RT私も検察官にそんな感じのことを言われたことがあります。」という内容の深澤諭史弁護士のツイートもあるのですが、その辺りも含めてスクリーンショットの記録を調べてみます。

〉〉〉 kk\_hironoのリツイート 〉〉〉

\begin{itemize}
\tightlist
\item
  RT
  kk\_hirono(刑事告発・非常上告_金沢地方検察庁御中)|s\_hirono(非常上告-最高検察庁御中\_ツイッター)
  日時:2021-07-01 12:42/2016/04/15 23:23 URL:
  \url{https://twitter.com/kk\_hirono/status/1410443675432194048} 
  \url{https://twitter.com/s\_hirono/status/720980965196046336} 
  \textgreater{}
  2016-04-15-232347\_奉納・社会汚染:弁護士泥棒神社\廣野秀樹さんはTwitterを使っています: ''。@motoken\_tw 法律家としての完全成仏の実刑刑務所.jpg
  \url{https://t.co/JXUU0Umzhu} 
\end{itemize}

〉〉〉 kk\_hironoのリツイート 〉〉〉

\begin{itemize}
\tightlist
\item
  RT
  kk\_hirono(刑事告発・非常上告_金沢地方検察庁御中)|s\_hirono(非常上告-最高検察庁御中\_ツイッター)
  日時:2021-07-01 12:42/2016/04/15 23:23 URL:
  \url{https://twitter.com/kk\_hirono/status/1410443712446943233} 
  \url{https://twitter.com/s\_hirono/status/720980808052264960} 
  \textgreater{}
  2016-04-15-232309\_モトケンさんはTwitterを使っています: ''私は、たいてい学生を殺してたな。主に最前列に座ってた学生w https://t。co/GsU.jpg
  \url{https://t.co/OWKx7v3Wng} 
\end{itemize}

〉〉〉 kk\_hironoのリツイート 〉〉〉

\begin{itemize}
\tightlist
\item
  RT
  kk\_hirono(刑事告発・非常上告_金沢地方検察庁御中)|s\_hirono(非常上告-最高検察庁御中\_ツイッター)
  日時:2021-07-01 12:44/2016/04/15 14:01 URL:
  \url{https://twitter.com/kk\_hirono/status/1410444119604817923} 
  \url{https://twitter.com/s\_hirono/status/720839517914017792} 
  \textgreater{}
  2016-04-15-140143\_モトケンさんはTwitterを使っています: ''本人訴訟をしている素人さんは、自分がしていることの法律的意味を理解してないと思うな。''.jpg
  \url{https://t.co/bPHbzUCK39} 
\end{itemize}

〉〉〉 kk\_hironoのリツイート 〉〉〉

\begin{itemize}
\tightlist
\item
  RT
  kk\_hirono(刑事告発・非常上告_金沢地方検察庁御中)|s\_hirono(非常上告-最高検察庁御中\_ツイッター)
  日時:2021-07-01 12:44/2016/04/15 13:54 URL:
  \url{https://twitter.com/kk\_hirono/status/1410444279512588288} 
  \url{https://twitter.com/s\_hirono/status/720837590526472192} 
  \textgreater{}
  2016-04-15-135404\_深澤諭史さんはTwitterを使っています: ''https://t。co/817yWxFPDJ 市民の利益を真に考えれば、担い手である弁護士.jpg
  \url{https://t.co/UrWWNG0brg} 
\end{itemize}

〉〉〉 kk\_hironoのリツイート 〉〉〉

\begin{itemize}
\tightlist
\item
  RT
  kk\_hirono(刑事告発・非常上告_金沢地方検察庁御中)|s\_hirono(非常上告-最高検察庁御中\_ツイッター)
  日時:2021-07-01 12:45/2016/04/15 08:10 URL:
  \url{https://twitter.com/kk\_hirono/status/1410444435737890816} 
  \url{https://twitter.com/s\_hirono/status/720751127692648449} 
  \textgreater{}
  2016-04-15-081030\_深澤諭史さんはTwitterを使っています: ''本人訴訟の一番恐ろしいところは、主張立証が稚拙なことよりも、法知識が不足なところよりも、①正.jpg
  \url{https://t.co/iPNlX8UQVs} 
\end{itemize}

〉〉〉 kk\_hironoのリツイート 〉〉〉

\begin{itemize}
\tightlist
\item
  RT
  kk\_hirono(刑事告発・非常上告_金沢地方検察庁御中)|s\_hirono(非常上告-最高検察庁御中\_ツイッター)
  日時:2021-07-01 12:47/2016/04/18 15:00 URL:
  \url{https://twitter.com/kk\_hirono/status/1410444974001311745} 
  \url{https://twitter.com/s\_hirono/status/721941514654384128} 
  \textgreater{}
  2016-04-18-150039\_深澤諭史さんはTwitterを使っています: ''問題は,ちゃんと受診してくれるかどうか,というところですね。 症状が悪化すれば「全ては捜査機.jpg
  \url{https://t.co/d9Xp7gOXld} 
\end{itemize}

〉〉〉 kk\_hironoのリツイート 〉〉〉

\begin{itemize}
\tightlist
\item
  RT
  kk\_hirono(刑事告発・非常上告_金沢地方検察庁御中)|s\_hirono(非常上告-最高検察庁御中\_ツイッター)
  日時:2021-07-01 12:50/2016/04/21 12:21 URL:
  \url{https://twitter.com/kk\_hirono/status/1410445656288727042} 
  \url{https://twitter.com/s\_hirono/status/722988724158492672} 
  \textgreater{}
  2016-04-21-122154\_深澤諭史さんはTwitterを使っています: ''>RT 「たいしたことじゃないんですけれど」「事件ではないんですが」「証拠もない話なんです」.jpg
  \url{https://t.co/K09v9YUkJ6} 
\end{itemize}

〉〉〉 kk\_hironoのリツイート 〉〉〉

\begin{itemize}
\tightlist
\item
  RT
  kk\_hirono(刑事告発・非常上告_金沢地方検察庁御中)|s\_hirono(非常上告-最高検察庁御中\_ツイッター)
  日時:2021-07-01 12:51/2016/04/21 12:13 URL:
  \url{https://twitter.com/kk\_hirono/status/1410445882407866370} 
  \url{https://twitter.com/s\_hirono/status/722986658006589440} 
  \textgreater{}
  2016-04-21-121342\_弁護士篠田奈保子さんはTwitterを使っています: ''弁護士が金目的で離婚を煽ってるとか、洗脳してるとか、DVをねつ造してるとか、言われ慣.jpg
  \url{https://t.co/QdXkDAzjw5} 
\end{itemize}

〉〉〉 kk\_hironoのリツイート 〉〉〉

\begin{itemize}
\tightlist
\item
  RT
  kk\_hirono(刑事告発・非常上告_金沢地方検察庁御中)|s\_hirono(非常上告-最高検察庁御中\_ツイッター)
  日時:2021-07-01 12:51/2016/04/21 12:12 URL:
  \url{https://twitter.com/kk\_hirono/status/1410445908999696388} 
  \url{https://twitter.com/s\_hirono/status/722986393182461953} 
  \textgreater{}
  2016-04-21-121239\_深澤諭史さんはTwitterを使っています: ''https://t。co/NCLZeQp8fM あるあるある。 そういう思いこみをする人だか.jpg
  \url{https://t.co/RVQYeAtuQN} 
\end{itemize}

〉〉〉 kk\_hironoのリツイート 〉〉〉

\begin{itemize}
\tightlist
\item
  RT
  kk\_hirono(刑事告発・非常上告_金沢地方検察庁御中)|s\_hirono(非常上告-最高検察庁御中\_ツイッター)
  日時:2021-07-01 12:51/2016/04/21 12:05 URL:
  \url{https://twitter.com/kk\_hirono/status/1410445947574767625} 
  \url{https://twitter.com/s\_hirono/status/722984694690631680} 
  \textgreater{}
  2016-04-21-120554\_深澤諭史さんはTwitterを使っています: ''大災害や大事故が起きるたんびに横行する不謹慎狩り。 「ヒャッハー!不謹慎な奴を見つけたぞーー.jpg
  \url{https://t.co/uTJ7whe0fo} 
\end{itemize}

〉〉〉 kk\_hironoのリツイート 〉〉〉

\begin{itemize}
\tightlist
\item
  RT
  kk\_hirono(刑事告発・非常上告_金沢地方検察庁御中)|s\_hirono(非常上告-最高検察庁御中\_ツイッター)
  日時:2021-07-01 12:52/2016/04/22 19:54 URL:
  \url{https://twitter.com/kk\_hirono/status/1410446253335277569} 
  \url{https://twitter.com/s\_hirono/status/723465088670167042} 
  \textgreater{}
  2016-04-22-195448\_モトケンさんのツイート: ''日本に関する限り、半分以上マスコミの問題だと思う。多分100%近く。.jpg
  \url{https://t.co/hnGYe668KG} 
\end{itemize}

〉〉〉 kk\_hironoのリツイート 〉〉〉

\begin{itemize}
\item
  RT
  kk\_hirono(刑事告発・非常上告_金沢地方検察庁御中)|s\_hirono(非常上告-最高検察庁御中\_ツイッター)
  日時:2021-07-01 12:52/2016/04/22 19:19 URL:
  \url{https://twitter.com/kk\_hirono/status/1410446309962575874} 
  \url{https://twitter.com/s\_hirono/status/723456180274618373} 
  \textgreater{}
  2016-04-22-191924\_深澤諭史さんのツイート: ''大メディアの方々に,是非,お願いしたいのは,事件関係者も被害者も被災者も,みんなみんな「あなたと同じ人間」だと忘.jpg
  \url{https://t.co/eQLDuGKWxx} 
\item
  非常上告-最高検察庁御中\_ツイッター(@s\_hirono)/2016年04月15日 -
  Twilog \url{https://t.co/Yzb880f945} 
\end{itemize}

 なかなかまずまずの収穫があったと思いますが、モトケンこと矢部善朗弁護士(京都弁護士会)のマスコミ批判と1つおいて深澤諭史弁護士のマスコミに対する説諭のようなツイートがありました。そういえばと思い出すツイートの1つです。

 熊本地震のような大災害が起こってマスコミが何も報道をしないのと、深澤諭史弁護士が簡単にストーカーと決めつけて医者の治療に解決を求めるのと、似たところがありそうですが、弁護士を世界の中心に据えた宗教性を感じ、社会的に有害でとてつもなく危険なものを感じます。

 なにやら公正世界仮説というのがありましたが、これも何度か見ています。熊本地震に近い時期とは考えになかったのですが、その被害者AAさん理も確認しておきます。

\begin{itemize}
\tightlist
\item
  〈〈〈 2021/07/01 13:03:38 Linux Emacs: 〈〈〈
\end{itemize}

\hypertarget{ux88c1ux5224ux54e1ux5236ux5ea6ux3092ux76aeux8089ux308bux6df1ux6fa4ux8aedux53f2ux5f01ux8b77ux58ebux81eaux8eabux306bux3088ux308bux904eux53bbux306eux30c4ux30a4ux30fcux30c8ux306eux30eaux30c4ux30a4ux30fcux30c8ux304bux3089ux518dux767aux898bux306bux81f3ux3063ux305fux65b0ux6f5fux5973ux5150ux6bbaux5bb3ux4e8bux4ef6ux306bux304aux3051ux308bux9ed9ux79d8ux6a29ux3068ux3046ux306eux5b57ux3068ux306eux30b3ux30e9ux30dcux306bux3088ux308bux30d2ux30e3ux30c3ux30cfux30fcux306eux30c4ux30a4ux30fcux30c815}{%
\paragraph{裁判員制度を皮肉る深澤諭史弁護士自身による過去のツイートのリツイートから再発見に至った、新潟女児殺害事件における黙秘権とうの字とのコラボによるヒャッハーのツイート(15)}\label{ux88c1ux5224ux54e1ux5236ux5ea6ux3092ux76aeux8089ux308bux6df1ux6fa4ux8aedux53f2ux5f01ux8b77ux58ebux81eaux8eabux306bux3088ux308bux904eux53bbux306eux30c4ux30a4ux30fcux30c8ux306eux30eaux30c4ux30a4ux30fcux30c8ux304bux3089ux518dux767aux898bux306bux81f3ux3063ux305fux65b0ux6f5fux5973ux5150ux6bbaux5bb3ux4e8bux4ef6ux306bux304aux3051ux308bux9ed9ux79d8ux6a29ux3068ux3046ux306eux5b57ux3068ux306eux30b3ux30e9ux30dcux306bux3088ux308bux30d2ux30e3ux30c3ux30cfux30fcux306eux30c4ux30a4ux30fcux30c815}}

\begin{itemize}
\tightlist
\item
  〉〉〉 Linux Emacs: 2021/07/01 13:05:08 〉〉〉
\end{itemize}

:CATEGORIES: @kanazawabengosi \#金沢弁護士会 @JFBAsns
日本弁護士連合会(日弁連) \#法務省 @MOJ\_HOUMU \#深澤諭史弁護士
\#うの字 \#坂本正幸弁護士 \#モトケンこと矢部善朗弁護士(京都弁護士会)

\begin{itemize}
\item
  1443:2021-07-01\_13:04:06 \#告発状 \#\#\#\#
  裁判員制度を皮肉る深澤諭史弁護士自身による過去のツイートのリツイートから再発見に至った、新潟女児殺害事件における黙秘権とうの字とのコラボによるヒャッハーのツイート(14)
  \url{https://hirono-hideki.hatenadiary.jp/entry/2021/07/01/130404} 
\item
  2021年07月01日13時01分の登録:
  REGEXP:''公正世界仮説''/データベース登録済みツイートの検索:2015-11-08〜2021-07-01/2021年07月01日13時00分の記録:ユーザ・投稿:13/41件
  \url{https://kk2020-09.blogspot.com/2021/07/regexp2015-11-082021-07.html} 
\item
  2021年07月01日13時06分の登録:
  REGEXP:''公正世界仮説''/深澤諭史(@fukazawas)の検索(2016-04-13〜2021-02-12/2021年07月01日13時06分の記録22件)
  \url{https://kk2020-09.blogspot.com/2021/07/regexpfukazawas2016-04-132021-02.html} 
\item
  (01/22) TW fukazawas(深澤諭史) 日時:2016-04-13 10:33:00 +0900
  URL:
  \url{https://twitter.com/fukazawas/status/720198185067515904\textgreater} {}
  本人訴訟の一番恐ろしいところは、主張立証が稚拙なことよりも、法知識が不足なところよりも、①正しい自分が負けるわけがない②間違っている相手の言い分が通るわけがない③いわゆる公正世界仮説、この3つが地獄のハーモニーを奏でてしまうところですね。\textgreater{}
  (・∀・;)
\end{itemize}

 この「公正世界仮説」というキーワードは、過去のツイートを検索してデータベースに追加するという作業はしていなかったように思うのですが、最初に出てきたのが2016年4月13日で、これは熊本地震の前日になりそうです。驚きです。

\begin{itemize}
\tightlist
\item
  (02/22) TW fukazawas(深澤諭史) 日時:2019-03-02 18:31:00 +0900
  URL:
  \url{https://twitter.com/fukazawas/status/1101777060609966085\textgreater} {}
  弁護士として膨大な紛争案件、ギリギリの状況での人の気持ちに触れる機会が多いと、本当に日本人って、「公正世界仮説」の虜なんだなって思う。\textgreater{}
  (・∀・;)
\end{itemize}

 深澤諭史弁護士のツイートにはチョロい国民性や肉屋を支持するブタというのもありました。弁護士をやっていて体験してきた本音の部分が出ているとも思います。成功体験の積み重ねかと思います。

\begin{itemize}
\tightlist
\item
  (03/22) TW fukazawas(深澤諭史) 日時:2019-06-06 08:59:00 +0900
  URL:
  \url{https://twitter.com/fukazawas/status/1136422354920255488\textgreater} {}
  日本人の「公正世界仮説への強固な信仰心」と「手続的正義の軽視」という特性は、「支配される民衆」としては、最高に素晴らしい特質だと思う。\textgreater{}
  ここまで支配する側、統治する側にとって都合の良い特性はない。\textgreater{}
  (・∀・) \url{https://t.co/vzIxq9yh6H} 
\end{itemize}

 黙秘権の保障も深澤諭史弁護士のいう「手続的正義の軽視」に入るのでしょう。

\begin{itemize}
\tightlist
\item
  (06/22) TW fukazawas(深澤諭史) 日時:2019-06-15 15:26:00 +0900
  URL:
  \url{https://twitter.com/fukazawas/status/1139781081232596992\textgreater} {}
  これすっごい不思議だけれども、公正世界仮説に囚われているのが原因じゃないかと思っています。\textgreater{}
  (・∀・;) \url{https://t.co/dtKv2B4ri2} 
\end{itemize}

 次の町村泰貴教授のツイートを印象したものです。

〉〉〉 kk\_hironoのリツイート 〉〉〉

\begin{itemize}
\tightlist
\item
  RT
  kk\_hirono(刑事告発・非常上告_金沢地方検察庁御中)|matimura(田丁木寸)
  日時:2021-07-01 13:39/2019/06/15 08:56 URL:
  \url{https://twitter.com/kk\_hirono/status/1410457917560475651} 
  \url{https://twitter.com/matimura/status/1139683033835900928} 
  \textgreater{}
  日本人の多くは個人情報の保護にことの外うるさく、プライバシー、肖像権とか、ある意味で自分勝手に解釈して、街角で写真に写りこむだけでも見知らぬ人にクレーム入れるくらいなのに、公権力のこういうのには頓着しないし、企業の情報収集も気にならない見たい。
\end{itemize}

〉〉〉 kk\_hironoのリツイート 〉〉〉

\begin{itemize}
\item
  RT
  kk\_hirono(刑事告発・非常上告_金沢地方検察庁御中)|matimura(田丁木寸)
  日時:2021-07-01 13:39/2019/06/15 08:56 URL:
  \url{https://twitter.com/kk\_hirono/status/1410457954759766019} 
  \url{https://twitter.com/matimura/status/1139683032393117696} 
  \textgreater{} SNS監視「自由侵害」の声 米ビザ申請で情報提供義務:
  日本経済新聞 \url{https://t.co/5ZTKCZFQHK} 
\item
  (13/22) TW fukazawas(深澤諭史) 日時:2020-04-02 20:56:00 +0900
  URL:
  \url{https://twitter.com/fukazawas/status/1245681502026739712\textgreater} {}
  (;・∀・)「公正世界仮説」を壊されるのがたまらなく我慢できない人って、それなりにいます。\textgreater{}
  私たち法曹は、そうでないと骨身に染みてわかっていますが、日本は特にそう思えない人がいっぱいいらっしゃいますので。\textgreater{}
  「政府がとんでもなく馬鹿・・・ \url{https://t.co/z4y0Em9J34} 
\item
  (18/22) TW fukazawas(深澤諭史) 日時:2020-06-09 18:57:00 +0900
  URL:
  \url{https://twitter.com/fukazawas/status/1270293869163233280\textgreater} {}
  日本社会って,ものすごい公正世界仮説が根強いと思っている。\textgreater{}
  性犯罪被害者への風当たりとか,デモに対する嫌悪感とか。\textgreater{}
  国家政府にとっては,とても都合が良い国民性ではあるが。\textgreater{}
  (・∀・)(^ω^)
\item
  (21/22) TW fukazawas(深澤諭史) 日時:2020-09-26 19:19:00 +0900
  URL:
  \url{https://twitter.com/fukazawas/status/1309799862267912198\textgreater} {}
  まさに公正世界仮説(・∀・) \url{https://t.co/YqHHDTIt7f} 
\end{itemize}

 これも次のツイートを引用した深澤諭史弁護士のツイートです。

〉〉〉 kk\_hironoのリツイート 〉〉〉

\begin{itemize}
\item
  RT
  kk\_hirono(刑事告発・非常上告_金沢地方検察庁御中)|bgsh\_owl(🦉ふくろう弁)
  日時:2021-07-01 13:44/2020/09/26 17:59 URL:
  \url{https://twitter.com/kk\_hirono/status/1410459170143313926} 
  \url{https://twitter.com/bgsh\_owl/status/1309779721861824512} 
  \textgreater{}
  なんの根拠もなく裁判所や警察が正しいって「お上が正しい」「お上にたてつくな」ってメンタルよね。
\item
  〈〈〈 2021/07/01 13:44:42 Linux Emacs: 〈〈〈
\end{itemize}

\hypertarget{ux88c1ux5224ux54e1ux5236ux5ea6ux3092ux76aeux8089ux308bux6df1ux6fa4ux8aedux53f2ux5f01ux8b77ux58ebux81eaux8eabux306bux3088ux308bux904eux53bbux306eux30c4ux30a4ux30fcux30c8ux306eux30eaux30c4ux30a4ux30fcux30c8ux304bux3089ux518dux767aux898bux306bux81f3ux3063ux305fux65b0ux6f5fux5973ux5150ux6bbaux5bb3ux4e8bux4ef6ux306bux304aux3051ux308bux9ed9ux79d8ux6a29ux3068ux3046ux306eux5b57ux3068ux306eux30b3ux30e9ux30dcux306bux3088ux308bux30d2ux30e3ux30c3ux30cfux30fcux306eux30c4ux30a4ux30fcux30c816}{%
\paragraph{裁判員制度を皮肉る深澤諭史弁護士自身による過去のツイートのリツイートから再発見に至った、新潟女児殺害事件における黙秘権とうの字とのコラボによるヒャッハーのツイート(16)}\label{ux88c1ux5224ux54e1ux5236ux5ea6ux3092ux76aeux8089ux308bux6df1ux6fa4ux8aedux53f2ux5f01ux8b77ux58ebux81eaux8eabux306bux3088ux308bux904eux53bbux306eux30c4ux30a4ux30fcux30c8ux306eux30eaux30c4ux30a4ux30fcux30c8ux304bux3089ux518dux767aux898bux306bux81f3ux3063ux305fux65b0ux6f5fux5973ux5150ux6bbaux5bb3ux4e8bux4ef6ux306bux304aux3051ux308bux9ed9ux79d8ux6a29ux3068ux3046ux306eux5b57ux3068ux306eux30b3ux30e9ux30dcux306bux3088ux308bux30d2ux30e3ux30c3ux30cfux30fcux306eux30c4ux30a4ux30fcux30c816}}

\begin{itemize}
\tightlist
\item
  〉〉〉 Linux Emacs: 2021/07/02 16:18:58 〉〉〉
\end{itemize}

:CATEGORIES: @kanazawabengosi \#金沢弁護士会 @JFBAsns
日本弁護士連合会(日弁連) \#法務省 @MOJ\_HOUMU \#深澤諭史弁護士
\#小倉秀夫弁護士 \#うの字

 昨夜、小倉秀夫弁護士に求意見とするツイートを送信していたのですが、返信はなかったようです。その前の夜、私が気がついたのは昨日の7月1日未明ですが、小倉秀夫弁護士のTwitterアカウントが凍結され、朝になると解除されていました。

〉〉〉 kk\_hironoのリツイート 〉〉〉

\begin{itemize}
\tightlist
\item
  RT
  kk\_hirono(刑事告発・非常上告_金沢地方検察庁御中)|hirono\_hideki(奉納\さらば弁護士鉄道・泥棒神社の物語)
  日時:2021-07-02 16:24/2021/07/02 14:33 URL:
  \url{https://twitter.com/kk\_hirono/status/1410862023240413185} 
  \url{https://twitter.com/hirono\_hideki/status/1410834084058652678} 
  \textgreater{} 2021-07-02\_09:18
  奉納\\#危険生物・弁護士脳汚染除去装置\\#金沢地方検察庁御中\_2020:
  \泥濘大魔王サイケ @k\_sawmen\育児中の女性に水子地蔵画像送りつけ太郎先生、自分は荒らし行為などしてないかのような言い種で草
  \url{https://t.co/0QEM1DjMjb} 
\end{itemize}

〉〉〉 kk\_hironoのリツイート 〉〉〉

\begin{itemize}
\tightlist
\item
  RT
  kk\_hirono(刑事告発・非常上告_金沢地方検察庁御中)|hirono\_hideki(奉納\さらば弁護士鉄道・泥棒神社の物語)
  日時:2021-07-02 16:24/2021/07/02 14:33 URL:
  \url{https://twitter.com/kk\_hirono/status/1410862032052621313} 
  \url{https://twitter.com/hirono\_hideki/status/1410834057701646337} 
  \textgreater{} 2021-07-02\_09:18
  奉納\\#危険生物・弁護士脳汚染除去装置\\#金沢地方検察庁御中\_2020:
  \🔥💩🔥 @un\_co\_the2nd\水子地蔵画像送りつけたろという発想が出てくるのがヤベーよ・・・しかも子育て中の人にさ・・・
  \url{https://t.co/Zx73xx4BhV} 
\end{itemize}

〉〉〉 kk\_hironoのリツイート 〉〉〉

\begin{itemize}
\tightlist
\item
  RT
  kk\_hirono(刑事告発・非常上告_金沢地方検察庁御中)|hirono\_hideki(奉納\さらば弁護士鉄道・泥棒神社の物語)
  日時:2021-07-02 16:24/2021/07/02 14:33 URL:
  \url{https://twitter.com/kk\_hirono/status/1410862041212981251} 
  \url{https://twitter.com/hirono\_hideki/status/1410834031155970052} 
  \textgreater{} 2021-07-02\_08:56
  奉納\\#危険生物・弁護士脳汚染除去装置\\#金沢地方検察庁御中\_2020:
  Ex-REGEXP:''生田暉雄''/データベース登録済みツイートの検索:2016-02-12〜2021-06-16/2021年07月02日08時55分の記録:ユーザ・投稿:5/6件
  \url{https://t.co/jqVacbp912} 
\end{itemize}

〉〉〉 kk\_hironoのリツイート 〉〉〉

\begin{itemize}
\tightlist
\item
  RT
  kk\_hirono(刑事告発・非常上告_金沢地方検察庁御中)|hirono\_hideki(奉納\さらば弁護士鉄道・泥棒神社の物語)
  日時:2021-07-02 16:24/2021/07/02 14:33 URL:
  \url{https://twitter.com/kk\_hirono/status/1410862049689624584} 
  \url{https://twitter.com/hirono\_hideki/status/1410834004731854848} 
  \textgreater{} 2021-07-02\_08:49
  奉納\\#危険生物・弁護士脳汚染除去装置\\#金沢地方検察庁御中\_2020:
  Ex-REGEXP:''再審申立書''/データベース登録済みツイートの検索:2020-05-25〜2020-05-25/2021年07月02日08時49分の記録:ユーザ・投稿:2/2件
  \url{https://t.co/kBsdxSkyA5} 
\end{itemize}

〉〉〉 kk\_hironoのリツイート 〉〉〉

\begin{itemize}
\tightlist
\item
  RT
  kk\_hirono(刑事告発・非常上告_金沢地方検察庁御中)|hirono\_hideki(奉納\さらば弁護士鉄道・泥棒神社の物語)
  日時:2021-07-02 16:25/2021/07/02 14:33 URL:
  \url{https://twitter.com/kk\_hirono/status/1410862106753126402} 
  \url{https://twitter.com/hirono\_hideki/status/1410833978274107397} 
  \textgreater{} 2021-07-02\_08:43
  奉納\\#危険生物・弁護士脳汚染除去装置\\#金沢地方検察庁御中\_2020:
  REGEXP:''岡口''/データベース登録済みツイートの検索:2021-06-28〜2021-07-01/2021年07月02日08時42分の記録:ユーザ・投稿:13/42件
  \url{https://t.co/4IECPvVkht} 
\end{itemize}

〉〉〉 kk\_hironoのリツイート 〉〉〉

\begin{itemize}
\tightlist
\item
  RT
  kk\_hirono(刑事告発・非常上告_金沢地方検察庁御中)|hirono\_hideki(奉納\さらば弁護士鉄道・泥棒神社の物語)
  日時:2021-07-02 16:25/2021/07/02 09:27 URL:
  \url{https://twitter.com/kk\_hirono/status/1410862145839927297} 
  \url{https://twitter.com/hirono\_hideki/status/1410756866565558272} 
  \textgreater{} -
  『「リーガルハイ」弁護士が選ぶドラマ1位に!脚本家・古沢良太氏インタビュー<完全版>』
  - 弁護士ドットコムタイムズ \url{https://t.co/6iBNdjJgGE} 
\end{itemize}

〉〉〉 kk\_hironoのリツイート 〉〉〉

\begin{itemize}
\tightlist
\item
  RT
  kk\_hirono(刑事告発・非常上告_金沢地方検察庁御中)|hirono\_hideki(奉納\さらば弁護士鉄道・泥棒神社の物語)
  日時:2021-07-02 16:25/2021/07/02 07:56 URL:
  \url{https://twitter.com/kk\_hirono/status/1410862284583313413} 
  \url{https://twitter.com/hirono\_hideki/status/1410734149170712577} 
  \textgreater{}
  「死刑」に被告の動き止まる 若き裁判員の葛藤 妻子6人殺害事件(毎日新聞)
  - goo ニュース \url{https://t.co/jOmphYNgLc} 
  小沼典彦弁護士は、小松被告の訴訟能力を認めた判決について、「(医師の)診断で判断すべきだった」と批判。
\end{itemize}

〉〉〉 kk\_hironoのリツイート 〉〉〉

\begin{itemize}
\tightlist
\item
  RT
  kk\_hirono(刑事告発・非常上告_金沢地方検察庁御中)|chosakukenho(小倉秀夫)
  日時:2021-07-02 16:26/2021/07/02 01:25 URL:
  \url{https://twitter.com/kk\_hirono/status/1410862316577386499} 
  \url{https://twitter.com/chosakukenho/status/1410635585631313920} 
  \textgreater{}
  一方的に相手を糾弾できるポジションだと思ったら反撃されたってパターンぽいんですよねえ。
  \url{https://t.co/elUjwByZ3S} 
\end{itemize}

〉〉〉 kk\_hironoのリツイート 〉〉〉

\begin{itemize}
\tightlist
\item
  RT
  kk\_hirono(刑事告発・非常上告_金沢地方検察庁御中)|hirono\_hideki(奉納\さらば弁護士鉄道・泥棒神社の物語)
  日時:2021-07-02 16:26/2021/07/02 07:50 URL:
  \url{https://twitter.com/kk\_hirono/status/1410862326853496837} 
  \url{https://twitter.com/hirono\_hideki/status/1410732472514482176} 
  \textgreater{} @chosakukenho
  ここに名誉毀損の刑事告訴の理由があります。求意見です。→ -
  2020年12月08日09時38分の登録:
  REGEXP:''(hirono\_hideki|kk\_hirono|s\_hirono)''/小倉秀夫(@Hideo\_Ogura)の検索(2010-05-06〜2018-08-30/2020年12月08日09時38分の記録98件)
  \url{https://t.co/oP3z75UVpW} 
\end{itemize}

〉〉〉 kk\_hironoのリツイート 〉〉〉

\begin{itemize}
\tightlist
\item
  RT
  kk\_hirono(刑事告発・非常上告_金沢地方検察庁御中)|hitotokikiken(将鼓(ネトウヨアルバイトリーダー、元:ひととき融資は犯罪です
  bot)) 日時:2021-07-02 16:26/2021/07/01 16:53 URL:
  \url{https://twitter.com/kk\_hirono/status/1410862398773235717} 
  \url{https://twitter.com/hitotokikiken/status/1410506719210266627} 
  \textgreater{} まあしかし、小倉秀夫弁護士は昔こんなんやったからなあ。
  わい、斜めに見てる。 \url{https://t.co/5j63sAQd3K} 
\end{itemize}

〉〉〉 kk\_hironoのリツイート 〉〉〉

\begin{itemize}
\tightlist
\item
  RT
  kk\_hirono(刑事告発・非常上告_金沢地方検察庁御中)|chosakukenho(小倉秀夫)
  日時:2021-07-02 16:26/2021/07/01 08:48 URL:
  \url{https://twitter.com/kk\_hirono/status/1410862462455349249} 
  \url{https://twitter.com/chosakukenho/status/1410384807779979265} 
  \textgreater{}
  多くの弁護士は納得のいかない敗訴を繰り返しているので、「正義は勝つ」って単純に思ってなんていないと思うんですよ。
\end{itemize}

〉〉〉 kk\_hironoのリツイート 〉〉〉

\begin{itemize}
\tightlist
\item
  RT
  kk\_hirono(刑事告発・非常上告_金沢地方検察庁御中)|chosakukenho(小倉秀夫)
  日時:2021-07-02 16:26/2021/07/01 22:56 URL:
  \url{https://twitter.com/kk\_hirono/status/1410862478393712642} 
  \url{https://twitter.com/chosakukenho/status/1410598153347166213} 
  \textgreater{} この人だったのか。 \url{https://t.co/7yI5bbjKbZ} 
\end{itemize}

〉〉〉 kk\_hironoのリツイート 〉〉〉

\begin{itemize}
\tightlist
\item
  RT
  kk\_hirono(刑事告発・非常上告_金沢地方検察庁御中)|hirono\_hideki(奉納\さらば弁護士鉄道・泥棒神社の物語)
  日時:2021-07-02 16:26/2021/07/01 23:31 URL:
  \url{https://twitter.com/kk\_hirono/status/1410862491454738434} 
  \url{https://twitter.com/hirono\_hideki/status/1410606988891803650} 
  \textgreater{} @chosakukenho - 2021年06月17日11時07分の登録:
  2021年06月17日の記録:写真資料:2021-06-17\_名誉毀損の刑事告訴に関連した小倉秀夫弁護士(東京弁護士会)のツイートの記録
  \url{https://t.co/atNwjOStxG} 
\end{itemize}

〉〉〉 kk\_hironoのリツイート 〉〉〉

\begin{itemize}
\tightlist
\item
  RT
  kk\_hirono(刑事告発・非常上告_金沢地方検察庁御中)|chosakukenho(小倉秀夫)
  日時:2021-07-02 16:27/2021/07/01 23:09 URL:
  \url{https://twitter.com/kk\_hirono/status/1410862573730209793} 
  \url{https://twitter.com/chosakukenho/status/1410601493674172421} 
  \textgreater{} そんなことして何が嬉しいのですか。
  \url{https://t.co/7yI5bbjKbZ} 
\end{itemize}

〉〉〉 kk\_hironoのリツイート 〉〉〉

\begin{itemize}
\tightlist
\item
  RT
  kk\_hirono(刑事告発・非常上告_金沢地方検察庁御中)|hirono\_hideki(奉納\さらば弁護士鉄道・泥棒神社の物語)
  日時:2021-07-02 16:32/2021/07/01 19:26 URL:
  \url{https://twitter.com/kk\_hirono/status/1410863825159561219} 
  \url{https://twitter.com/hirono\_hideki/status/1410545331549343744} 
  \textgreater{} 2021-07-01\_18:32
  奉納\\#危険生物・弁護士脳汚染除去装置\\#金沢地方検察庁御中\_2020:
  \モトケン @motoken\_tw\おニューのほうですね。私と同じパターンかも。運営は文脈読まないからね。オグリンも読まない(と言うか読ませない)けどw
  \url{https://t.co/ZpgDkfer8y} 
\end{itemize}

〉〉〉 kk\_hironoのリツイート 〉〉〉

\begin{itemize}
\tightlist
\item
  RT
  kk\_hirono(刑事告発・非常上告_金沢地方検察庁御中)|chosakukenho(小倉秀夫)
  日時:2021-07-02 16:32/2021/06/30 11:36 URL:
  \url{https://twitter.com/kk\_hirono/status/1410863888778743808} 
  \url{https://twitter.com/chosakukenho/status/1410064623085056000} 
  \textgreater{}
  刑事告訴するという話なんですか?民事訴訟を提起するのではなく。
  \url{https://t.co/cJLIsdyRik} 
\end{itemize}

〉〉〉 kk\_hironoのリツイート 〉〉〉

\begin{itemize}
\tightlist
\item
  RT
  kk\_hirono(刑事告発・非常上告_金沢地方検察庁御中)|chosakukenho(小倉秀夫)
  日時:2021-07-02 16:32/2021/07/01 08:58 URL:
  \url{https://twitter.com/kk\_hirono/status/1410863952007860226} 
  \url{https://twitter.com/chosakukenho/status/1410387356180357122} 
  \textgreater{}
  侮辱罪で告訴だと、侮辱罪に該当する心証を検事さんに与えても、普通起訴しないので、つらい戦いですね。
  \url{https://t.co/Vrv9zudX1Q} 
\end{itemize}

〉〉〉 kk\_hironoのリツイート 〉〉〉

\begin{itemize}
\tightlist
\item
  RT
  kk\_hirono(刑事告発・非常上告_金沢地方検察庁御中)|chosakukenho(小倉秀夫)
  日時:2021-07-02 16:32/2021/07/01 17:15 URL:
  \url{https://twitter.com/kk\_hirono/status/1410864047369515010} 
  \url{https://twitter.com/chosakukenho/status/1410512305410830336} 
  \textgreater{} 何の通知も来ていないので何も分からないんですよ。
  \url{https://t.co/lNM1UlirMx} 
\end{itemize}

〉〉〉 kk\_hironoのリツイート 〉〉〉

\begin{itemize}
\tightlist
\item
  RT
  kk\_hirono(刑事告発・非常上告_金沢地方検察庁御中)|hirono\_hideki(奉納\さらば弁護士鉄道・泥棒神社の物語)
  日時:2021-07-02 16:33/2021/07/01 18:50 URL:
  \url{https://twitter.com/kk\_hirono/status/1410864100884713474} 
  \url{https://twitter.com/hirono\_hideki/status/1410536233017638914} 
  \textgreater{} 1位 あわや大事故 子供が歩く歩道を車が走行か \textbar{}
  バイキングMORE 2021/06/28(月)11:55のニュース \textbar{} TVでた蔵
  \url{https://t.co/kGU209xIdB}  2021年6月28日放送 13:37 - 13:38 フジテレビ
  バイキングMORE バイキング的BUZZニュースランキング
\end{itemize}

〉〉〉 kk\_hironoのリツイート 〉〉〉

\begin{itemize}
\tightlist
\item
  RT
  kk\_hirono(刑事告発・非常上告_金沢地方検察庁御中)|hirono\_hideki(奉納\さらば弁護士鉄道・泥棒神社の物語)
  日時:2021-07-02 16:33/2021/07/01 18:48 URL:
  \url{https://twitter.com/kk\_hirono/status/1410864138549559296} 
  \url{https://twitter.com/hirono\_hideki/status/1410535868071170048} 
  \textgreater{}
  <Buzzニュースランキング>1位・あわや大事故・子供が歩く歩道を車が走行か
  フジテレビ【バイキングMORE】|JCCテレビすべて
  \url{https://t.co/IjbiFhMp1b} 
  今月8日午後4時頃に熊本県菊陽町で、歩道から交差点に進入する青い車の様子を、近くを走行していた車のドライブレコーダーが捉えた。
\end{itemize}

〉〉〉 kk\_hironoのリツイート 〉〉〉

\begin{itemize}
\tightlist
\item
  RT
  kk\_hirono(刑事告発・非常上告_金沢地方検察庁御中)|hirono\_hideki(奉納\さらば弁護士鉄道・泥棒神社の物語)
  日時:2021-07-02 16:33/2021/07/01 18:21 URL:
  \url{https://twitter.com/kk\_hirono/status/1410864258443710470} 
  \url{https://twitter.com/hirono\_hideki/status/1410528906935816193} 
  \textgreater{}
  菅首相が献花 犠牲の男児2人悼む 八街児童5人死傷事故(千葉日報オンライン)
  - Yahoo!ニュース \url{https://t.co/fwH91bKDmT} 
\end{itemize}

〉〉〉 kk\_hironoのリツイート 〉〉〉

\begin{itemize}
\tightlist
\item
  RT
  kk\_hirono(刑事告発・非常上告_金沢地方検察庁御中)|hirono\_hideki(奉納\さらば弁護士鉄道・泥棒神社の物語)
  日時:2021-07-02 16:34/2021/07/01 17:57 URL:
  \url{https://twitter.com/kk\_hirono/status/1410864350181486592} 
  \url{https://twitter.com/hirono\_hideki/status/1410522918459559946} 
  \textgreater{}
  北海道新聞が速やかに果たすべき説明責任とは――「記者逮捕」を考える〈上〉
  - 高田昌幸|論座 - 朝日新聞社の言論サイト \url{https://t.co/PaKp5UogvR} 
  ・・・ログインして読む (残り:約390文字/本文:約5591文字)
\end{itemize}

 小倉秀夫弁護士の凍結と凍結の解除については、反応するツイートが多く、かなりの数奉納\さらば弁護士鉄道・泥棒神社の物語(@hirono\_hideki)のアカウントでリツイートをしたのですが、夜遅くには沈静化し、何事もなかったような状況に戻っていました。

\begin{itemize}
\item
  2021年07月02日16時36分の登録:
  @hirono\_hideki(奉納\さらば弁護士鉄道・泥棒神社の物語)のツイート ''.*'' 3238/3238:2021-06-12\_2005〜2021-07-02\_1609 2021年07月02日16時36分の記録
  \url{https://kk2020-09.blogspot.com/2021/07/hironohideki323832382021-06-1220052021.html} 
\item
  〈〈〈 2021/07/02 16:40:10 Linux Emacs: 〈〈〈
\end{itemize}

\hypertarget{ux88c1ux5224ux54e1ux5236ux5ea6ux3092ux76aeux8089ux308bux6df1ux6fa4ux8aedux53f2ux5f01ux8b77ux58ebux81eaux8eabux306bux3088ux308bux904eux53bbux306eux30c4ux30a4ux30fcux30c8ux306eux30eaux30c4ux30a4ux30fcux30c8ux304bux3089ux518dux767aux898bux306bux81f3ux3063ux305fux65b0ux6f5fux5973ux5150ux6bbaux5bb3ux4e8bux4ef6ux306bux304aux3051ux308bux9ed9ux79d8ux6a29ux3068ux3046ux306eux5b57ux3068ux306eux30b3ux30e9ux30dcux306bux3088ux308bux30d2ux30e3ux30c3ux30cfux30fcux306eux30c4ux30a4ux30fcux30c817}{%
\paragraph{裁判員制度を皮肉る深澤諭史弁護士自身による過去のツイートのリツイートから再発見に至った、新潟女児殺害事件における黙秘権とうの字とのコラボによるヒャッハーのツイート(17)}\label{ux88c1ux5224ux54e1ux5236ux5ea6ux3092ux76aeux8089ux308bux6df1ux6fa4ux8aedux53f2ux5f01ux8b77ux58ebux81eaux8eabux306bux3088ux308bux904eux53bbux306eux30c4ux30a4ux30fcux30c8ux306eux30eaux30c4ux30a4ux30fcux30c8ux304bux3089ux518dux767aux898bux306bux81f3ux3063ux305fux65b0ux6f5fux5973ux5150ux6bbaux5bb3ux4e8bux4ef6ux306bux304aux3051ux308bux9ed9ux79d8ux6a29ux3068ux3046ux306eux5b57ux3068ux306eux30b3ux30e9ux30dcux306bux3088ux308bux30d2ux30e3ux30c3ux30cfux30fcux306eux30c4ux30a4ux30fcux30c817}}

\begin{itemize}
\tightlist
\item
  〉〉〉 Linux Emacs: 2021/07/02 16:41:52 〉〉〉
\end{itemize}

:CATEGORIES: @kanazawabengosi \#金沢弁護士会 @JFBAsns
日本弁護士連合会(日弁連) \#法務省 @MOJ\_HOUMU \#深澤諭史弁護士
\#うの字 \#小倉秀夫弁護士 \#モトケンこと矢部善朗弁護士(京都弁護士会)
\#再審請求

 午前中、6月の主な写真を集めたまとめ記事を作成しました。後になってスクリプトを修正し、ページのアンカーのリンクをテキストとして表示させるコードを追加しました。ページタイトルやURLは、投稿時には不明なもので、アクセス時にjavascriptで取得し表示するようにしました。

 ざっと一月の経過がわかりやすいと感じました。まだ決算報告書の作成に取り掛かっていないのですが、一区切り入れて置きたいと思って作成したのが6月分の写真のまとめ記事になります。次に再開する準備も兼ねてやってみたのですが、記憶と記録の整理に役立つと感じました。

 以下にページ内リンクで写真を紹介しながら簡単なご説明を記しておきたいと思います。

\begin{itemize}
\tightlist
\item
  奉納\危険生物・弁護士脳汚染除去装置\金沢地方検察庁御中\_2020:
  2021年07月02日の記録:写真資料:2021-06 2021-06-01\_141353_.jpg
  \url{https://t.co/W5KGQQWoLs}  \url{https://t.co/MY94vM3Psk} 
\end{itemize}

 6月1日の撮影となっていますが、平成4年か平成5年の金沢市内の地図で、被告発人木梨松嗣弁護士の法律事務所を調べていました。福井刑務所で訴訟用として使っていた地図の1つです。地図は結構な数まとめ買いをしていました。郵送の差し入れと思います。

\begin{itemize}
\tightlist
\item
  奉納\危険生物・弁護士脳汚染除去装置\金沢地方検察庁御中\_2020:
  2021年07月02日の記録:写真資料:2021-06 2021-06-01\_174443_.jpg
  \url{https://kk2020-09.blogspot.com/2021/07/202107022021-06.html\#72\#5} 
\end{itemize}

 図書館にある金沢市内の住宅地図で、被告発人木梨松嗣弁護士の法律事務所を調べています。

\begin{itemize}
\tightlist
\item
  奉納\危険生物・弁護士脳汚染除去装置\金沢地方検察庁御中\_2020:
  2021年07月02日の記録:写真資料:2021-06 2021-06-02\_205705_.jpg
  \url{https://kk2020-09.blogspot.com/2021/07/202107022021-06.html\#72\#10} 
\end{itemize}

 Amazonプライムで視聴した映画「不敵に笑う男」の場面ですが、北陸刑務所とあるのは、実際に金沢市小立野にあった当時の金沢刑務所とネットで知り、調べ直したときの撮影です。昭和33年か34年の撮影という情報があったと思います。

 小立野についていろいろと調べていた時期ですが、金沢市の精神病院の沿革や歴史についても調べたことになります。けっこう時間を使いましたが、まだ十分に調べたとは言えず、被告発人木梨松嗣弁護士の精神鑑定請求という重要な事実関係に関わります。

\begin{itemize}
\tightlist
\item
  奉納\危険生物・弁護士脳汚染除去装置\金沢地方検察庁御中\_2020:
  2021年07月02日の記録:写真資料:2021-06 2021-06-05\_055933_.jpg
  \url{https://kk2020-09.blogspot.com/2021/07/202107022021-06.html\#72\#19} 
\end{itemize}

 スクリプト実行の端末表示で、絵文字を使うようになりました。

\begin{itemize}
\tightlist
\item
  奉納\危険生物・弁護士脳汚染除去装置\金沢地方検察庁御中\_2020:
  2021年07月02日の記録:写真資料:2021-06 2021-06-05\_121155_.jpg
  \url{https://kk2020-09.blogspot.com/2021/07/202107022021-06.html\#72\#21} 
\end{itemize}

 図書館でみた北陸中日新聞です。月に数回しかみない新聞で、ほとんどが図書館ですが、金沢刑務所の写真が出ていて所長も顔出しのようでした。数日後にネットで調べると情報が少しは見つかりましたが、ずっと乏しい情報になっていたように思います。

\begin{itemize}
\tightlist
\item
  奉納\危険生物・弁護士脳汚染除去装置\金沢地方検察庁御中\_2020:
  2021年07月02日の記録:写真資料:2021-06 2021-06-05\_121815_.jpg
  \url{https://kk2020-09.blogspot.com/2021/07/202107022021-06.html\#72\#22} 
\end{itemize}

 図書館の金沢市内の住宅地図で、被告発人若杉幸平弁護士の金沢市田井町の住所を調べたものですが、表記は「若杉」だけになっていることを確認しました。金沢弁護士会のホームページでも若杉法律事務所の住所となっていた番地です。

\begin{itemize}
\tightlist
\item
  奉納\危険生物・弁護士脳汚染除去装置\金沢地方検察庁御中\_2020:
  2021年07月02日の記録:写真資料:2021-06 2021-06-11\_083801_.jpg
  \url{https://kk2020-09.blogspot.com/2021/07/202107022021-06.html\#72\#34} 
\end{itemize}

 Googleマップとストリートビューで鳥取市周辺を調べたときのものです。ストリートビューに切り替えるといきなりトンネルの中が出てきたのですが、初めての経験でした。平成の初め頃とは道路状況がまるで変わっていて、国道9号線の市内の入口は痕跡も掴めませんでした。

 本書では記述をしていなかったと思いますが、昭和の終わりから平成の初め頃の鳥取市内、国道9号線沿いの入口のような辺りというのは、西部劇に出てくる砂漠の中の町という雰囲気で、銀河鉄道999に出てくる西部劇のような町とも雰囲気が似ていて、印象深いものがありました。

\begin{itemize}
\tightlist
\item
  奉納\危険生物・弁護士脳汚染除去装置\金沢地方検察庁御中\_2020:
  2021年07月02日の記録:写真資料:2021-06 2021-06-11\_114138_.jpg
  \url{https://kk2020-09.blogspot.com/2021/07/202107022021-06.html\#72\#37} 
\end{itemize}

 鳥取の事件のPDFファイルで、広島高裁松江支部の逆転無罪判決になります。鳥取の弁護士の3人で900万円という国選弁護費用がきっかけでしたが、その後に情報は見ていません。銀河鉄道999にでも出てくる幻のような話でしたが、かなり時間を掛けて調べ記録を残しているはずです。

\begin{itemize}
\tightlist
\item
  奉納\危険生物・弁護士脳汚染除去装置\金沢地方検察庁御中\_2020:
  2021年07月02日の記録:写真資料:2021-06 2021-06-11\_141110_.jpg
  \url{https://kk2020-09.blogspot.com/2021/07/202107022021-06.html\#72\#40} 
\end{itemize}

 久しぶりに起動したWindows10のログイン画面の撮影ですが、対応に時間をかけるトラブルに遭遇したような記憶があります。よく憶えていません。

\begin{itemize}
\tightlist
\item
  奉納\危険生物・弁護士脳汚染除去装置\金沢地方検察庁御中\_2020:
  2021年07月02日の記録:写真資料:2021-06 2021-06-11\_154011_.jpg
  \url{https://kk2020-09.blogspot.com/2021/07/202107022021-06.html\#72\#41} 
\end{itemize}

 これは図書館の新聞ではないように思えますが、富山の交番襲撃で遺族が富山県を提訴したという記事でした。あとでネットで調べましたが、大きなニュースにはなっておらず、弁護士の関心もすこぶる低かったように思います。

\begin{itemize}
\tightlist
\item
  奉納\危険生物・弁護士脳汚染除去装置\金沢地方検察庁御中\_2020:
  2021年07月02日の記録:写真資料:2021-06 2021-06-11\_155451_.jpg
  \url{https://kk2020-09.blogspot.com/2021/07/202107022021-06.html\#72\#43} 
\end{itemize}

 旧能登町役場近くの公衆トイレ内の写真ですが、平成21年の10月頃、ここでその後の人生を帰るような出来事がありました。遅いか早いかの状態だったとは思いますが、当時のことが思い出され懐かしさもあって写真を撮っておきました。生活保護生活のきっかけのようなことです。

\begin{itemize}
\tightlist
\item
  奉納\危険生物・弁護士脳汚染除去装置\金沢地方検察庁御中\_2020:
  2021年07月02日の記録:写真資料:2021-06 2021-06-11\_165834_.jpg
  \url{https://kk2020-09.blogspot.com/2021/07/202107022021-06.html\#72\#45} 
\end{itemize}

 図書館で見つけて借りてきたHTMLの書籍ですが、家では一度も開くことなく、返却日の6月25日に返しにいきました。珍しく返却日を初めからずっと憶えていました。

\begin{itemize}
\tightlist
\item
  奉納\危険生物・弁護士脳汚染除去装置\金沢地方検察庁御中\_2020:
  2021年07月02日の記録:写真資料:2021-06 2021-06-11\_181837_.jpg
  \url{https://kk2020-09.blogspot.com/2021/07/202107022021-06.html\#72\#46} 
\end{itemize}

 宇出津の観音寺橋の写真です。蛍を見に国重に向かうところですが、源平の南宮神社と国重の白山神社に立ち寄ることを予定し、早めの時間の出発となりました。

\begin{itemize}
\tightlist
\item
  奉納\危険生物・弁護士脳汚染除去装置\金沢地方検察庁御中\_2020:
  2021年07月02日の記録:写真資料:2021-06 2021-06-11\_182705_.jpg
  \url{https://kk2020-09.blogspot.com/2021/07/202107022021-06.html\#72\#47} 
\end{itemize}

 源平の南宮神社の鳥居前です。先に違った道に入って戻っていたのですが、ちょうどバイクを停めたタイミングで、能登町の告知放送が始まり、矢波の火事が鎮火したという放送でした。結局、火事の話は聞くことがなかったので、火災の規模や内容もわかっていません。

\begin{itemize}
\tightlist
\item
  奉納\危険生物・弁護士脳汚染除去装置\金沢地方検察庁御中\_2020:
  2021年07月02日の記録:写真資料:2021-06 2021-06-11\_183000_.jpg
  \url{https://kk2020-09.blogspot.com/2021/07/202107022021-06.html\#72\#48} 
\end{itemize}

 写真ではそれほど目立っていませんが、南宮神社の後ろにとても細くて高い木がありました。しばらく前に、Twitterで見ていた昭和30年代前半の宇出津の仙人町付近の写真で見た高い木とよく似ていました。後でネットで調べたところクロマツという情報がありました。

\begin{itemize}
\tightlist
\item
  奉納\危険生物・弁護士脳汚染除去装置\金沢地方検察庁御中\_2020:
  2021年07月02日の記録:写真資料:2021-06 2021-06-11\_184714_.jpg
  \url{https://kk2020-09.blogspot.com/2021/07/202107022021-06.html\#72\#51} 
\end{itemize}

 Googleマップで見ていた情報で当たりをつけて向かった国重の白山神社ですが、建物が今まで自分が見てきた寺社仏閣ではもっとも古びていて、海の底から引き上げられた蛸壷のように見えました。海に入っていた木造船の船体で付着した貝殻を取り除いた状態にも思えました。

\begin{itemize}
\tightlist
\item
  奉納\危険生物・弁護士脳汚染除去装置\金沢地方検察庁御中\_2020:
  2021年07月02日の記録:写真資料:2021-06 2021-06-11\_191641_.jpg
  \url{https://kk2020-09.blogspot.com/2021/07/202107022021-06.html\#72\#53} 
\end{itemize}

 暗くなるまで山口ダムで時間を潰したのですが、かなり長く感じる時間を過ごしていました。古びた建物が「銀の龍の背に乗って」が主題歌のドラマの診療所に似ているとも思えました。ドラマのタイトルが思い出せないですが、3年ほど前に再放送をテレビで見ています。

〉〉〉 kk\_hironoのリツイート 〉〉〉

\begin{itemize}
\item
  RT
  kk\_hirono(刑事告発・非常上告_金沢地方検察庁御中)|hirono\_hideki(奉納\さらば弁護士鉄道・泥棒神社の物語)
  日時:2021-07-02 18:03/2019/07/05 00:17 URL:
  \url{https://twitter.com/kk\_hirono/status/1410886900257329153} 
  \url{https://twitter.com/hirono\_hideki/status/1146800087164604417} 
  \textgreater{} 幕末~明治初期の吉原遊郭の風景 - YouTube
  \url{https://t.co/VZsbbKsOcN} 
\item
  奉納\さらば弁護士鉄道・泥棒神社の物語(@hirono\_hideki)/2019年07月05日
  - Twilog \url{https://t.co/iwO73RuAFc} 
\end{itemize}

 リンクのYouTube動画を作成しましたが、見た覚えがないような内容の吉原遊廓の風景でした。2019年7月5日00時17分のツイートとなっています。

〉〉〉 kk\_hironoのリツイート 〉〉〉

\begin{itemize}
\tightlist
\item
  RT
  kk\_hirono(刑事告発・非常上告_金沢地方検察庁御中)|hirono\_hideki(奉納\さらば弁護士鉄道・泥棒神社の物語)
  日時:2021-07-02 18:07/2019/07/05 10:56 URL:
  \url{https://twitter.com/kk\_hirono/status/1410887906168762370} 
  \url{https://twitter.com/hirono\_hideki/status/1146960961552519168} 
  \textgreater{} 刑事弁護オアシス \textbar{} 刑事弁護の情報が集まる
  刑事弁護情報サイト \url{https://t.co/Z9vrAeao7r} 
\end{itemize}

〉〉〉 kk\_hironoのリツイート 〉〉〉

\begin{itemize}
\tightlist
\item
  RT
  kk\_hirono(刑事告発・非常上告_金沢地方検察庁御中)|hirono\_hideki(奉納\さらば弁護士鉄道・泥棒神社の物語)
  日時:2021-07-02 18:07/2019/07/05 11:01 URL:
  \url{https://twitter.com/kk\_hirono/status/1410887959235076103} 
  \url{https://twitter.com/hirono\_hideki/status/1146962241196974080} 
  \textgreater{} 2019年7月 1日「大崎事件」最高裁決定に対する抗議声明 -
  えん罪救済センター \url{https://t.co/T7grluZLXj} 
\end{itemize}

〉〉〉 kk\_hironoのリツイート 〉〉〉

\begin{itemize}
\tightlist
\item
  RT
  kk\_hirono(刑事告発・非常上告_金沢地方検察庁御中)|ihokujp(i北陸|イベント・観光地情報)
  日時:2021-07-02 18:08/2019/07/01 17:09 URL:
  \url{https://twitter.com/kk\_hirono/status/1410888067217457154} 
  \url{https://twitter.com/ihokujp/status/1145605234980315136} 
  \textgreater{} あばれ祭 7月5日(金)~6日(土) 能登町宇出津地区
  宇出津地区のキリコ祭。以前の様子はこちらから\url{https://t.co/Uhld8mwTsT} 
  \url{https://t.co/YlHMEsq4rw} 
\end{itemize}

〉〉〉 kk\_hironoのリツイート 〉〉〉

\begin{itemize}
\tightlist
\item
  RT
  kk\_hirono(刑事告発・非常上告_金沢地方検察庁御中)|hirono\_hideki(奉納\さらば弁護士鉄道・泥棒神社の物語)
  日時:2021-07-02 18:09/2019/07/05 13:05 URL:
  \url{https://twitter.com/kk\_hirono/status/1410888315419598855} 
  \url{https://twitter.com/hirono\_hideki/status/1146993565077106688} 
  \textgreater{} 宇出津あばれ祭り 白山神社 神輿
  2019-07-05\_122353_.jpg 2019-07-05\_122511_.jpg
  2019-07-05\_122935_.jpg 2019-07-05\_122936_.jpg
  \url{https://t.co/KUhkJdrSmV} 
\end{itemize}

〉〉〉 kk\_hironoのリツイート 〉〉〉

\begin{itemize}
\tightlist
\item
  RT
  kk\_hirono(刑事告発・非常上告_金沢地方検察庁御中)|hirono\_hideki(奉納\さらば弁護士鉄道・泥棒神社の物語)
  日時:2021-07-02 18:10/2019/07/05 14:31 URL:
  \url{https://twitter.com/kk\_hirono/status/1410888667715887104} 
  \url{https://twitter.com/hirono\_hideki/status/1147015050445541376} 
  \textgreater{} 2019年07月05日14時24分の登録:
  \ぽぽひと@内閣調査室所属 @popohito\当会で民事裁判のIT化検討を扱う委員会が、パソコンをろくに使えない重鎮弁護士だらけという情報に接して驚愕している。
  \url{https://t.co/vIxcK6C6jS} 
\end{itemize}

〉〉〉 kk\_hironoのリツイート 〉〉〉

\begin{itemize}
\tightlist
\item
  RT
  kk\_hirono(刑事告発・非常上告_金沢地方検察庁御中)|hirono\_hideki(奉納\さらば弁護士鉄道・泥棒神社の物語)
  日時:2021-07-02 18:11/2019/07/05 15:47 URL:
  \url{https://twitter.com/kk\_hirono/status/1410888803099627531} 
  \url{https://twitter.com/hirono\_hideki/status/1147034333334556672} 
  \textgreater{} 2019年07月05日15時46分の登録:
  REGEXP:''刑事弁護オアシス''/データベース登録済みツイート:2019年07月05日15時45分の記録:ユーザ・投稿:24/48件
  \url{https://t.co/O1CvZGWl9S} 
\end{itemize}

〉〉〉 kk\_hironoのリツイート 〉〉〉

\begin{itemize}
\tightlist
\item
  RT
  kk\_hirono(刑事告発・非常上告_金沢地方検察庁御中)|keiben\_oasis(刑事弁護オアシス)
  日時:2021-07-02 18:11/2019/07/04 12:32 URL:
  \url{https://twitter.com/kk\_hirono/status/1410888839355269121} 
  \url{https://twitter.com/keiben\_oasis/status/1146622679786512384} 
  \textgreater{} 「和歌山カレー事件から21年
  7月20日、和歌山カレー事件再審請求弁護団報告集会」を公開いたしました。
  \#刑事弁護オアシス,\#刑事弁護 \url{https://t.co/N4HBrtFTSO} 
\end{itemize}

〉〉〉 kk\_hironoのリツイート 〉〉〉

\begin{itemize}
\tightlist
\item
  RT
  kk\_hirono(刑事告発・非常上告_金沢地方検察庁御中)|hirono\_hideki(奉納\さらば弁護士鉄道・泥棒神社の物語)
  日時:2021-07-02 18:11/2019/07/05 16:39 URL:
  \url{https://twitter.com/kk\_hirono/status/1410888936486957058} 
  \url{https://twitter.com/hirono\_hideki/status/1147047232383811584} 
  \textgreater{}
  ごご☆プレ・Dr.~コトー診療所 【病気を診るな、人を診ろ】\#4
  2019年7月5日(金) 15時53分~16時50分 の放送内容
  \url{https://t.co/XJ7MhaGjtN} 
\end{itemize}

〉〉〉 kk\_hironoのリツイート 〉〉〉

\begin{itemize}
\tightlist
\item
  RT
  kk\_hirono(刑事告発・非常上告_金沢地方検察庁御中)|hirono\_hideki(奉納\さらば弁護士鉄道・泥棒神社の物語)
  日時:2021-07-02 18:12/2019/07/05 19:02 URL:
  \url{https://twitter.com/kk\_hirono/status/1410889099632803840} 
  \url{https://twitter.com/hirono\_hideki/status/1147083377721262080} 
  \textgreater{}
  爆報!THE フライデー【萩原流行の事故死 裁判の結末を妻が初告白】
  2019年7月5日(金) 19時00分~20時00分 の放送内容
  \url{https://t.co/5PZnEgKfNv} 
\end{itemize}

 時刻は18時14分です。ふと気がついたのですが、今日は7月の第一金曜日で、本来なら宇出津のあばれ祭りの初日になります。

 新型コロナで昨年の2020年と2年続けての中止となった宇出津のあばれ祭りですが、最後の通常開催となったのが2019年でした。夕方になると石川県のローカル番組で、1つの放送局ぐらいは毎年あばれ祭りの中継をやっているので、テレビのチャンネルを変えて探しました。

 1つか2つチャンネルを変えたタイミングだったと思うのですが、そこで見た場面というのがDr.コトー診療所で、テレビでみたのは初めてだったと思います。

 そのあとに、テレビであばれ祭りの中継をみたのですが、その場所というのが音羽町で、宇出津の遊郭跡地になります。

 本来は宇出津の新村の数馬酒造の前が神輿と曳山が休憩をする場所となっているが、白山神社の演目が遊女の仙台高尾となっているので、今年は音羽町で休憩することになっていると聞いていたのが2016年の4月のことで、当日、強風でその曳山が転倒したのでその場で中止となっていたのです。

 2016年、2017年と2年間、祭礼委員をやり、2018年、2019年と音羽町の人が祭礼委員をやり、また2020年、2021年と小棚木で祭礼委員となっていたのですが、まともな実行となったのは2020年の秋季例祭ぐらいで、4月の曳山祭も7月のあばれ祭りも2年続けて中止となっています。

 祭りは中止となっていますが、

\begin{itemize}
\tightlist
\item
  〈〈〈 2021/07/02 18:58:46 Linux Emacs: 〈〈〈
\end{itemize}

\hypertarget{ux30e2ux30c8ux30b1ux30f3ux3053ux3068ux77e2ux90e8ux5584ux6717ux5f01ux8b77ux58ebux4eacux90fdux5f01ux8b77ux58ebux4f1a}{%
\subsubsection{モトケンこと矢部善朗弁護士(京都弁護士会)}\label{ux30e2ux30c8ux30b1ux30f3ux3053ux3068ux77e2ux90e8ux5584ux6717ux5f01ux8b77ux58ebux4eacux90fdux5f01ux8b77ux58ebux4f1a}}

\hypertarget{ux3053ux306eux4ebaux81eaux5206ux306eux5bb6ux306eux4e2dux3067ux72ecux308aux8a00ux3092ux3064ux3076ux3084ux3044ux3066ux3044ux308bux306eux3068ux5e83ux5834ux3067ux5927ux58f0ux3067ux81eaux5206ux306eux4e3bux5f35ux3092ux53ebux3093ux3067ux3044ux308bux3053ux3068ux306eux533aux5225ux304cux3064ux304bux306aux3044ux306eux306dux3068ux3044ux3046ux30c4ux30a4ux30fcux30c8}{%
\paragraph{「この人、自分の家の中で独り言をつぶやいているのと、広場で大声で自分の主張を叫んでいることの区別がつかないのね。」というツイート}\label{ux3053ux306eux4ebaux81eaux5206ux306eux5bb6ux306eux4e2dux3067ux72ecux308aux8a00ux3092ux3064ux3076ux3084ux3044ux3066ux3044ux308bux306eux3068ux5e83ux5834ux3067ux5927ux58f0ux3067ux81eaux5206ux306eux4e3bux5f35ux3092ux53ebux3093ux3067ux3044ux308bux3053ux3068ux306eux533aux5225ux304cux3064ux304bux306aux3044ux306eux306dux3068ux3044ux3046ux30c4ux30a4ux30fcux30c8}}

\begin{itemize}
\tightlist
\item
  〉〉〉 Linux Emacs: 2021/04/22 10:45:39 〉〉〉
\end{itemize}

:CATEGORIES: @kanazawabengosi \#金沢弁護士会 @JFBAsns
日本弁護士連合会(日弁連) \#法務省 @MOJ\_HOUMU
\#モトケンこと矢部善朗弁護士(京都弁護士会)

 引用されたツイートを遡ると,昨日辺りに見かけた「介入されたくなかったらブロックすればいんじゃね?」というモトケンこと矢部善朗弁護士(京都弁護士会)のツイートが会話の流れにありました。

〉〉〉 kk\_hironoのリツイート 〉〉〉

\begin{itemize}
\tightlist
\item
  RT
  kk\_hirono(刑事告発・非常上告_金沢地方検察庁御中)|amazakke(あ◯さん)
  日時:2021-04-22 10:50/2021/04/22 01:56 URL:
  \url{https://twitter.com/kk\_hirono/status/1385048405643128844} 
  \url{https://twitter.com/amazakke/status/1384913943643185155} 
  \textgreater{} @motoken\_tw
  論理で語るモトケンさんには、思いつき感情論で反論してくる人と相性が悪そうだ
\end{itemize}

〉〉〉 kk\_hironoのリツイート 〉〉〉

\begin{itemize}
\tightlist
\item
  RT
  kk\_hirono(刑事告発・非常上告_金沢地方検察庁御中)|uEhcJcGVcY92Z23(ちゃうんや 備蓄は適宜補充を。)
  日時:2021-04-22 10:50/2021/04/22 06:31 URL:
  \url{https://twitter.com/kk\_hirono/status/1385048439117910019} 
  \url{https://twitter.com/uEhcJcGVcY92Z23/status/1384983209239187457} 
  \textgreater{} @motoken\_tw
  締め出したいのなら、ブロックなり鍵かけるなりすれば良いのに。
\end{itemize}

〉〉〉 kk\_hironoのリツイート 〉〉〉

\begin{itemize}
\tightlist
\item
  RT
  kk\_hirono(刑事告発・非常上告_金沢地方検察庁御中)|nattosansai(伊久新之助@GooglePlusRefugee)
  日時:2021-04-22 10:51/2021/04/22 01:34 URL:
  \url{https://twitter.com/kk\_hirono/status/1385048572287021059} 
  \url{https://twitter.com/nattosansai/status/1384908331471544326} 
  \textgreater{} @motoken\_tw 文脈がわからんので判断はできないが、
  「リアルでそこが人んち」にいる人が「去れって言われて去らな」かったのか?
  それとも、別にリアルでもなく、人んちでもない場(例えば、ツイッター等)で、個人的に「去れ」って言われたのか?
\end{itemize}

〉〉〉 kk\_hironoのリツイート 〉〉〉

\begin{itemize}
\tightlist
\item
  RT
  kk\_hirono(刑事告発・非常上告_金沢地方検察庁御中)|underwriter00(アンダーライター)
  日時:2021-04-22 10:51/2021/04/22 05:21 URL:
  \url{https://twitter.com/kk\_hirono/status/1385048611734528002} 
  \url{https://twitter.com/underwriter00/status/1384965573487038464} 
  \textgreater{} @motoken\_tw
  何言ってるんだろ、この人。Twitterで表で話してる事の異論が来て不退去(笑)
  嫌なら鍵かけて話せば良いのに。 Twitterの仕様を知らないのかな。
  自分が世界中に発信しているとは思っていないのか
\end{itemize}

〉〉〉 kk\_hironoのリツイート 〉〉〉

\begin{itemize}
\tightlist
\item
  RT
  kk\_hirono(刑事告発・非常上告_金沢地方検察庁御中)|2goRPeSy5hsxd37(いまじん)
  日時:2021-04-22 10:51/2021/04/22 01:07 URL:
  \url{https://twitter.com/kk\_hirono/status/1385048691044614147} 
  \url{https://twitter.com/2goRPeSy5hsxd37/status/1384901627144986624} 
  \textgreater{} @motoken\_tw @FvTF2DxpChUP4pM @NwFle6q9vQTXb4q
  @aurXCanJyw3ePxS @den198804 @X2Can3joy @otsukaikumasan @yone717
  @kt\_lec\_serious @nana77rey1 @vB6T5dn8OdRiTmB @runrunpon
  @csyuvMhPcpyuLE1 @atr46056249 @REON60206927 @ibuki\_teika
  @SeitokaiYakuinF @edogawahanako55 @schmalkaizen @atr58748830
  @Nzdaisukiikoiko @dozeen3 @VNrespectFN @madaoorz2 @Ta\_Howait
  @Magmag40887776 @wworeww1 @moe303060 @TEAR\_ACCEL @hdr\_mck
  @Junito04899663 @kmm295 @nyanhechan @Sf8Mf3o2JFKv5uk @karusatonotora
  @E27FdOJwjnhdfpd @ha8ka0 @cp0GUGoG5QAaeRd @Ron82794189
  @Emmanuel\_Chanel
  去れって言われても去らないのは、リアルでそこが人んちなら不退去ですよwwwww
  議論の相手としても歓迎されてないのに、そこまでして我を通したいなら、泥棒と同じで自分と他人の境界線がおかしいよ?
  他にも話せる場所はいくらでもあるでしょうが。 何やってんの?
\end{itemize}

〉〉〉 kk\_hironoのリツイート 〉〉〉

\begin{itemize}
\tightlist
\item
  RT
  kk\_hirono(刑事告発・非常上告_金沢地方検察庁御中)|2goRPeSy5hsxd37(いまじん)
  日時:2021-04-22 10:52/2021/04/22 01:02 URL:
  \url{https://twitter.com/kk\_hirono/status/1385048733985820672} 
  \url{https://twitter.com/2goRPeSy5hsxd37/status/1384900448352616450} 
  \textgreater{} @motoken\_tw @FvTF2DxpChUP4pM @NwFle6q9vQTXb4q
  @aurXCanJyw3ePxS @den198804 @X2Can3joy @otsukaikumasan @yone717
  @kt\_lec\_serious @nana77rey1 @vB6T5dn8OdRiTmB @runrunpon
  @csyuvMhPcpyuLE1 @atr46056249 @REON60206927 @ibuki\_teika
  @SeitokaiYakuinF @edogawahanako55 @schmalkaizen @atr58748830
  @Nzdaisukiikoiko @dozeen3 @VNrespectFN @madaoorz2 @Ta\_Howait
  @Magmag40887776 @wworeww1 @moe303060 @TEAR\_ACCEL @hdr\_mck
  @Junito04899663 @kmm295 @nyanhechan @Sf8Mf3o2JFKv5uk @karusatonotora
  @E27FdOJwjnhdfpd @ha8ka0 @cp0GUGoG5QAaeRd @Ron82794189
  @Emmanuel\_Chanel あのね、元検事さん。
  Twitterだから仕様内であれば何をやってもいいと思ってるかもしれないけど、仲間内のツリーなんだよ。
  どういう扱いのツリーにするかはスレ主と我々が決めること。
  知らずに来るのは仕方ないとしても、事情がわかれば退くのがマナーでしょうが。
\end{itemize}

〉〉〉 kk\_hironoのリツイート 〉〉〉

\begin{itemize}
\tightlist
\item
  RT
  kk\_hirono(刑事告発・非常上告_金沢地方検察庁御中)|2goRPeSy5hsxd37(いまじん)
  日時:2021-04-22 10:52/2021/04/22 00:42 URL:
  \url{https://twitter.com/kk\_hirono/status/1385048779460542469} 
  \url{https://twitter.com/2goRPeSy5hsxd37/status/1384895431520444418} 
  \textgreater{} @motoken\_tw @FvTF2DxpChUP4pM @NwFle6q9vQTXb4q
  @aurXCanJyw3ePxS @den198804 @X2Can3joy @otsukaikumasan @yone717
  @kt\_lec\_serious @nana77rey1 @vB6T5dn8OdRiTmB @runrunpon
  @csyuvMhPcpyuLE1 @atr46056249 @REON60206927 @ibuki\_teika
  @SeitokaiYakuinF @edogawahanako55 @schmalkaizen @atr58748830
  @Nzdaisukiikoiko @dozeen3 @VNrespectFN @madaoorz2 @Ta\_Howait
  @Magmag40887776 @wworeww1 @moe303060 @TEAR\_ACCEL @hdr\_mck
  @Junito04899663 @kmm295 @nyanhechan @Sf8Mf3o2JFKv5uk @karusatonotora
  @E27FdOJwjnhdfpd @ha8ka0 @cp0GUGoG5QAaeRd @Ron82794189
  @Emmanuel\_Chanel
  相手が不快に感じるような発言をすることが当然の権利とは、さすが、「元」検事さんは言うことが違いますねw
  それとも、裁判所が判断しない限り、それをしてもよいとでも言うつもりですかねwwwww
\end{itemize}

〉〉〉 kk\_hironoのリツイート 〉〉〉

\begin{itemize}
\tightlist
\item
  RT
  kk\_hirono(刑事告発・非常上告_金沢地方検察庁御中)|2goRPeSy5hsxd37(いまじん)
  日時:2021-04-22 10:52/2021/04/22 00:38 URL:
  \url{https://twitter.com/kk\_hirono/status/1385048835286716419} 
  \url{https://twitter.com/2goRPeSy5hsxd37/status/1384894381400936449} 
  \textgreater{} @motoken\_tw @FvTF2DxpChUP4pM @NwFle6q9vQTXb4q
  @aurXCanJyw3ePxS @den198804 @X2Can3joy @otsukaikumasan @yone717
  @kt\_lec\_serious @nana77rey1 @vB6T5dn8OdRiTmB @runrunpon
  @csyuvMhPcpyuLE1 @atr46056249 @REON60206927 @ibuki\_teika
  @SeitokaiYakuinF @edogawahanako55 @schmalkaizen @atr58748830
  @Nzdaisukiikoiko @dozeen3 @VNrespectFN @madaoorz2 @Ta\_Howait
  @Magmag40887776 @wworeww1 @moe303060 @TEAR\_ACCEL @hdr\_mck
  @Junito04899663 @kmm295 @nyanhechan @Sf8Mf3o2JFKv5uk @karusatonotora
  @E27FdOJwjnhdfpd @ha8ka0 @cp0GUGoG5QAaeRd @Ron82794189
  @Emmanuel\_Chanel 当然の権利? ほーw
  相手がやめてくれと不作為の要求をしてもあなたはご自身の権利とやらを主張しますか?☺️
\end{itemize}

〉〉〉 kk\_hironoのリツイート 〉〉〉

\begin{itemize}
\tightlist
\item
  RT
  kk\_hirono(刑事告発・非常上告_金沢地方検察庁御中)|FvTF2DxpChUP4pM(m-k)
  日時:2021-04-22 10:52/2021/04/21 15:30 URL:
  \url{https://twitter.com/kk\_hirono/status/1385048899002372097} 
  \url{https://twitter.com/FvTF2DxpChUP4pM/status/1384756499612798977} 
  \textgreater{} @motoken\_tw @NwFle6q9vQTXb4q @aurXCanJyw3ePxS
  @den198804 @X2Can3joy @otsukaikumasan @yone717 @kt\_lec\_serious
  @nana77rey1 @vB6T5dn8OdRiTmB @runrunpon @csyuvMhPcpyuLE1 @atr46056249
  @REON60206927 @ibuki\_teika @SeitokaiYakuinF @edogawahanako55
  @schmalkaizen @atr58748830 @Nzdaisukiikoiko @dozeen3 @VNrespectFN
  @madaoorz2 @2goRPeSy5hsxd37 @Ta\_Howait @Magmag40887776 @wworeww1
  @moe303060 @TEAR\_ACCEL @hdr\_mck @Junito04899663 @kmm295 @nyanhechan
  @Sf8Mf3o2JFKv5uk @karusatonotora @E27FdOJwjnhdfpd @ha8ka0
  @cp0GUGoG5QAaeRd @Ron82794189 @Emmanuel\_Chanel
  おやめくださいと告知して、それを続けるというのは嫌がらせともとれる発言になりますが?
\end{itemize}

〉〉〉 kk\_hironoのリツイート 〉〉〉

\begin{itemize}
\tightlist
\item
  RT
  kk\_hirono(刑事告発・非常上告_金沢地方検察庁御中)|FvTF2DxpChUP4pM(m-k)
  日時:2021-04-22 10:53/2021/04/21 15:09 URL:
  \url{https://twitter.com/kk\_hirono/status/1385048964819394561} 
  \url{https://twitter.com/FvTF2DxpChUP4pM/status/1384751062716862464} 
  \textgreater{} @motoken\_tw @NwFle6q9vQTXb4q @aurXCanJyw3ePxS
  @den198804 @X2Can3joy @otsukaikumasan @yone717 @kt\_lec\_serious
  @nana77rey1 @vB6T5dn8OdRiTmB @runrunpon @csyuvMhPcpyuLE1 @atr46056249
  @REON60206927 @ibuki\_teika @SeitokaiYakuinF @edogawahanako55
  @schmalkaizen @atr58748830 @Nzdaisukiikoiko @dozeen3 @VNrespectFN
  @madaoorz2 @2goRPeSy5hsxd37 @Ta\_Howait @Magmag40887776 @wworeww1
  @moe303060 @TEAR\_ACCEL @hdr\_mck @Junito04899663 @kmm295 @nyanhechan
  @Sf8Mf3o2JFKv5uk @karusatonotora @E27FdOJwjnhdfpd @ha8ka0
  @cp0GUGoG5QAaeRd @Ron82794189 @Emmanuel\_Chanel
  じゃ、このツリーでツイートをおやめください。
  反対派を説得するんじゃないんですか?
  説明の放棄ととらえてよろしいでしょうか?
\end{itemize}

〉〉〉 kk\_hironoのリツイート 〉〉〉

\begin{itemize}
\tightlist
\item
  RT
  kk\_hirono(刑事告発・非常上告_金沢地方検察庁御中)|NwFle6q9vQTXb4q(Deep
  sea creatures) 日時:2021-04-22 10:53/2021/04/21 14:58 URL:
  \url{https://twitter.com/kk\_hirono/status/1385049046654488576} 
  \url{https://twitter.com/NwFle6q9vQTXb4q/status/1384748457718521857} 
  \textgreater{} @aurXCanJyw3ePxS @motoken\_tw @FvTF2DxpChUP4pM
  @den198804 @X2Can3joy @otsukaikumasan @yone717 @kt\_lec\_serious
  @nana77rey1 @vB6T5dn8OdRiTmB @runrunpon @csyuvMhPcpyuLE1 @atr46056249
  @REON60206927 @ibuki\_teika @SeitokaiYakuinF @edogawahanako55
  @schmalkaizen @atr58748830 @Nzdaisukiikoiko @dozeen3 @VNrespectFN
  @madaoorz2 @2goRPeSy5hsxd37 @Ta\_Howait @Magmag40887776 @wworeww1
  @moe303060 @TEAR\_ACCEL @hdr\_mck @Junito04899663 @kmm295 @nyanhechan
  @Sf8Mf3o2JFKv5uk @karusatonotora @E27FdOJwjnhdfpd @ha8ka0
  @cp0GUGoG5QAaeRd @Ron82794189 @Emmanuel\_Chanel
  事実婚と法律婚は法的保護が違いますが、モトケン先生は全て(相続含め)等しくすべきだとお考えですか?
  現在事実婚を選んでいる方は改姓だけでない理由のようですがその方達はこのまま?
  日本の婚姻受理はフランスのPACSよりも簡易ですけど、導入後もl婚姻受理条件も変えない方が良いと考えですか?
  \url{https://t.co/fH36ui7kqu}  \url{https://t.co/WLw9EGmni9} 
\end{itemize}

〉〉〉 kk\_hironoのリツイート 〉〉〉

\begin{itemize}
\tightlist
\item
  RT
  kk\_hirono(刑事告発・非常上告_金沢地方検察庁御中)|NwFle6q9vQTXb4q(Deep
  sea creatures) 日時:2021-04-22 10:53/2021/04/21 14:44 URL:
  \url{https://twitter.com/kk\_hirono/status/1385049106343632900} 
  \url{https://twitter.com/NwFle6q9vQTXb4q/status/1384744881780756481} 
  \textgreater{} @aurXCanJyw3ePxS @motoken\_tw @FvTF2DxpChUP4pM
  @den198804 @X2Can3joy @otsukaikumasan @yone717 @kt\_lec\_serious
  @nana77rey1 @vB6T5dn8OdRiTmB @runrunpon @csyuvMhPcpyuLE1 @atr46056249
  @REON60206927 @ibuki\_teika @SeitokaiYakuinF @edogawahanako55
  @schmalkaizen @atr58748830 @Nzdaisukiikoiko @dozeen3 @VNrespectFN
  @madaoorz2 @2goRPeSy5hsxd37 @Ta\_Howait @Magmag40887776 @wworeww1
  @moe303060 @TEAR\_ACCEL @hdr\_mck @Junito04899663 @kmm295 @nyanhechan
  @Sf8Mf3o2JFKv5uk @karusatonotora @E27FdOJwjnhdfpd @ha8ka0
  @cp0GUGoG5QAaeRd @Ron82794189 @Emmanuel\_Chanel
  モトケン先生、ご出張ですか?
  ひよこまるさんは先生のクライアントさんなんですか?
  また、話しを全て見ないで参加されてますの?
\end{itemize}

〉〉〉 kk\_hironoのリツイート 〉〉〉

\begin{itemize}
\tightlist
\item
  RT
  kk\_hirono(刑事告発・非常上告_金沢地方検察庁御中)|X2Can3joy(ひよこまる)
  日時:2021-04-22 10:53/2021/04/20 11:45 URL:
  \url{https://twitter.com/kk\_hirono/status/1385049162337591297} 
  \url{https://twitter.com/X2Can3joy/status/1384337378324606976} 
  \textgreater{} @otsukaikumasan @FvTF2DxpChUP4pM @den198804 @yone717
  @kt\_lec\_serious @nana77rey1 @vB6T5dn8OdRiTmB @runrunpon
  @csyuvMhPcpyuLE1 @atr46056249 @REON60206927 @aurXCanJyw3ePxS
  @ibuki\_teika @SeitokaiYakuinF @edogawahanako55 @schmalkaizen
  @atr58748830 @Nzdaisukiikoiko @dozeen3 @VNrespectFN @madaoorz2
  @2goRPeSy5hsxd37 @NwFle6q9vQTXb4q @Ta\_Howait @Magmag40887776
  @wworeww1 @moe303060 @TEAR\_ACCEL @hdr\_mck @Junito04899663 @kmm295
  @nyanhechan @Sf8Mf3o2JFKv5uk @karusatonotora @E27FdOJwjnhdfpd @ha8ka0
  @cp0GUGoG5QAaeRd @Ron82794189 @Emmanuel\_Chanel
  \textgreater 同姓が最もシンプル
  いいえ、実際の家族を証明する事が戸籍の機能であり、姓は情報項目の一つなので、同姓である必要はありません。
  私は、法律婚が優先されるべきとは思っておりません。同姓も別姓も等しく家族だと主張しただけです。
\end{itemize}

〉〉〉 kk\_hironoのリツイート 〉〉〉

\begin{itemize}
\tightlist
\item
  RT
  kk\_hirono(刑事告発・非常上告_金沢地方検察庁御中)|otsukaikumasan(おつかいくまさん)
  日時:2021-04-22 10:53/2021/04/20 10:04 URL:
  \url{https://twitter.com/kk\_hirono/status/1385049204746199041} 
  \url{https://twitter.com/otsukaikumasan/status/1384312030346170372} 
  \textgreater{} @X2Can3joy @FvTF2DxpChUP4pM @den198804 @yone717
  @kt\_lec\_serious @nana77rey1 @vB6T5dn8OdRiTmB @runrunpon
  @csyuvMhPcpyuLE1 @atr46056249 @REON60206927 @aurXCanJyw3ePxS
  @ibuki\_teika @SeitokaiYakuinF @edogawahanako55 @schmalkaizen
  @atr58748830 @Nzdaisukiikoiko @dozeen3 @VNrespectFN @madaoorz2
  @2goRPeSy5hsxd37 @NwFle6q9vQTXb4q @Ta\_Howait @Magmag40887776
  @wworeww1 @moe303060 @TEAR\_ACCEL @hdr\_mck @Junito04899663 @kmm295
  @nyanhechan @Sf8Mf3o2JFKv5uk @karusatonotora @E27FdOJwjnhdfpd @ha8ka0
  @cp0GUGoG5QAaeRd @Ron82794189 @Emmanuel\_Chanel
  住民基本台帳法に基づくのは「通称」だけです。身分関係の記録まで住民基本台帳で管理しろとは言っていません。
  通称については、戸籍に住民基本台帳法に情報を反映させるべきか?方法は?という問題はあるが、通称の活用で人格権の問題はほぼ充足されますね。
  \url{https://t.co/En2F1RpzcS} 
\end{itemize}

〉〉〉 kk\_hironoのリツイート 〉〉〉

\begin{itemize}
\tightlist
\item
  RT
  kk\_hirono(刑事告発・非常上告_金沢地方検察庁御中)|yone717(yone)
  日時:2021-04-22 10:54/2021/04/19 02:26 URL:
  \url{https://twitter.com/kk\_hirono/status/1385049456588980227} 
  \url{https://twitter.com/yone717/status/1383834434055462923} 
  \textgreater{} @kt\_lec\_serious @nana77rey1 @vB6T5dn8OdRiTmB
  @runrunpon @csyuvMhPcpyuLE1 @FvTF2DxpChUP4pM @X2Can3joy @atr46056249
  @REON60206927 @aurXCanJyw3ePxS @ibuki\_teika @SeitokaiYakuinF
  @edogawahanako55 @schmalkaizen @atr58748830 @den198804
  @Nzdaisukiikoiko @dozeen3 @VNrespectFN @madaoorz2 @2goRPeSy5hsxd37
  @NwFle6q9vQTXb4q @Ta\_Howait @otsukaikumasan @Magmag40887776 @wworeww1
  @moe303060 @TEAR\_ACCEL @hdr\_mck @Junito04899663 @kmm295 @nyanhechan
  @Sf8Mf3o2JFKv5uk @karusatonotora @E27FdOJwjnhdfpd @ha8ka0
  @cp0GUGoG5QAaeRd @Ron82794189 @Emmanuel\_Chanel
  話変わるけど、相手が名字を変えたくない場合に結婚を諦める人がこんなに居るのですね。
  結婚を諦めるという事は相手と同じく自分も名字を変えたくないという事。
  選択的夫婦別姓なら結婚出来るのにね。
  未婚率増加と少子化が問題視されているのに法改正反対だなんて、改善する気無いって事だよね。
  \url{https://t.co/8SaZD8diSf} 
\end{itemize}

〉〉〉 kk\_hironoのリツイート 〉〉〉

\begin{itemize}
\tightlist
\item
  RT
  kk\_hirono(刑事告発・非常上告_金沢地方検察庁御中)|kt\_lec\_serious(行太@人権やらCOVID-19やらのシリアス垢🎗🗣)
  日時:2021-04-22 10:55/2021/04/18 17:37 URL:
  \url{https://twitter.com/kk\_hirono/status/1385049595814707207} 
  \url{https://twitter.com/kt\_lec\_serious/status/1383701297988194307} 
  \textgreater{} @vB6T5dn8OdRiTmB @runrunpon @csyuvMhPcpyuLE1
  @FvTF2DxpChUP4pM @X2Can3joy @atr46056249 @REON60206927
  @aurXCanJyw3ePxS @ibuki\_teika @SeitokaiYakuinF @edogawahanako55
  @schmalkaizen @atr58748830 @den198804 @nana77rey1 @Nzdaisukiikoiko
  @dozeen3 @VNrespectFN @madaoorz2 @2goRPeSy5hsxd37 @NwFle6q9vQTXb4q
  @Ta\_Howait @otsukaikumasan @Magmag40887776 @wworeww1 @moe303060
  @TEAR\_ACCEL @hdr\_mck @Junito04899663 @kmm295 @nyanhechan
  @Sf8Mf3o2JFKv5uk @karusatonotora @E27FdOJwjnhdfpd @ha8ka0
  @cp0GUGoG5QAaeRd @Ron82794189 @Emmanuel\_Chanel
  別姓推進派の誰かさんも「どちらとも言えない」を含めたからですよ?
\end{itemize}

〉〉〉 kk\_hironoのリツイート 〉〉〉

\begin{itemize}
\tightlist
\item
  RT
  kk\_hirono(刑事告発・非常上告_金沢地方検察庁御中)|vB6T5dn8OdRiTmB(ツッキーだよ)
  日時:2021-04-22 10:55/2021/04/18 17:34 URL:
  \url{https://twitter.com/kk\_hirono/status/1385049653855412224} 
  \url{https://twitter.com/vB6T5dn8OdRiTmB/status/1383700374519894021} 
  \textgreater{} @kt\_lec\_serious @runrunpon @csyuvMhPcpyuLE1
  @FvTF2DxpChUP4pM @X2Can3joy @atr46056249 @REON60206927
  @aurXCanJyw3ePxS @ibuki\_teika @SeitokaiYakuinF @edogawahanako55
  @schmalkaizen @atr58748830 @den198804 @nana77rey1 @Nzdaisukiikoiko
  @dozeen3 @VNrespectFN @madaoorz2 @2goRPeSy5hsxd37 @NwFle6q9vQTXb4q
  @Ta\_Howait @otsukaikumasan @Magmag40887776 @wworeww1 @moe303060
  @TEAR\_ACCEL @hdr\_mck @Junito04899663 @kmm295 @nyanhechan
  @Sf8Mf3o2JFKv5uk @karusatonotora @E27FdOJwjnhdfpd @ha8ka0
  @cp0GUGoG5QAaeRd @Ron82794189 @Emmanuel\_Chanel
  なぜ、どちらとも言えないを含めるの?
\end{itemize}

〉〉〉 kk\_hironoのリツイート 〉〉〉

\begin{itemize}
\tightlist
\item
  RT
  kk\_hirono(刑事告発・非常上告_金沢地方検察庁御中)|csyuvMhPcpyuLE1(ウェプエス)
  日時:2021-04-22 10:56/2021/04/18 16:02 URL:
  \url{https://twitter.com/kk\_hirono/status/1385049747736514560} 
  \url{https://twitter.com/csyuvMhPcpyuLE1/status/1383677393445953542} 
  \textgreater{} @runrunpon @FvTF2DxpChUP4pM @kt\_lec\_serious
  @X2Can3joy @atr46056249 @REON60206927 @aurXCanJyw3ePxS @ibuki\_teika
  @SeitokaiYakuinF @edogawahanako55 @schmalkaizen @atr58748830
  @den198804 @nana77rey1 @Nzdaisukiikoiko @dozeen3 @VNrespectFN
  @madaoorz2 @vB6T5dn8OdRiTmB @2goRPeSy5hsxd37 @NwFle6q9vQTXb4q
  @Ta\_Howait @otsukaikumasan @Magmag40887776 @wworeww1 @moe303060
  @TEAR\_ACCEL @hdr\_mck @Junito04899663 @kmm295 @nyanhechan
  @Sf8Mf3o2JFKv5uk @karusatonotora @E27FdOJwjnhdfpd @ha8ka0
  @cp0GUGoG5QAaeRd @Ron82794189 @DJ\_aka\_F2G @Emmanuel\_Chanel
  私の感性なのか、選択的夫婦別姓と同性婚は全然別の問題としか思えないのですが
  同性婚は高額な性転換手術が必要ですが、選択的夫婦別姓は二人の話し合いだけで法律婚可能なはず
  選択的夫婦別姓を人権問題というのは、無理があると思います
\end{itemize}

〉〉〉 kk\_hironoのリツイート 〉〉〉

\begin{itemize}
\tightlist
\item
  RT
  kk\_hirono(刑事告発・非常上告_金沢地方検察庁御中)|FvTF2DxpChUP4pM(m-k)
  日時:2021-04-22 10:56/2021/04/18 09:14 URL:
  \url{https://twitter.com/kk\_hirono/status/1385049811171233793} 
  \url{https://twitter.com/FvTF2DxpChUP4pM/status/1383574643773112332} 
  \textgreater{} @runrunpon @kt\_lec\_serious @X2Can3joy @atr46056249
  @REON60206927 @aurXCanJyw3ePxS @ibuki\_teika @SeitokaiYakuinF
  @edogawahanako55 @schmalkaizen @atr58748830 @den198804 @nana77rey1
  @Nzdaisukiikoiko @dozeen3 @VNrespectFN @madaoorz2 @vB6T5dn8OdRiTmB
  @2goRPeSy5hsxd37 @NwFle6q9vQTXb4q @Ta\_Howait @otsukaikumasan
  @Magmag40887776 @wworeww1 @moe303060 @TEAR\_ACCEL @hdr\_mck
  @Junito04899663 @kmm295 @nyanhechan @Sf8Mf3o2JFKv5uk @karusatonotora
  @E27FdOJwjnhdfpd @ha8ka0 @cp0GUGoG5QAaeRd @Ron82794189 @DJ\_aka\_F2G
  @Emmanuel\_Chanel
  そう思うなら、人権侵害ではないし、裁判でもその判決が出てる
  婚姻しなきゃいいよ 事実婚でいいじゃん 権利もある程度認めてる
  \url{https://t.co/8do9J0bnk7} 
\end{itemize}

〉〉〉 kk\_hironoのリツイート 〉〉〉

\begin{itemize}
\tightlist
\item
  RT
  kk\_hirono(刑事告発・非常上告_金沢地方検察庁御中)|X2Can3joy(ひよこまる)
  日時:2021-04-22 10:56/2021/04/15 17:15 URL:
  \url{https://twitter.com/kk\_hirono/status/1385049893421547521} 
  \url{https://twitter.com/X2Can3joy/status/1382608405899087876} 
  \textgreater{} @REON60206927 @FvTF2DxpChUP4pM @kt\_lec\_serious
  @aurXCanJyw3ePxS @ibuki\_teika @SeitokaiYakuinF @edogawahanako55
  @schmalkaizen @atr58748830 @den198804 @nana77rey1 @Nzdaisukiikoiko
  @dozeen3 @VNrespectFN @madaoorz2 @vB6T5dn8OdRiTmB @2goRPeSy5hsxd37
  @NwFle6q9vQTXb4q @Ta\_Howait @BuenaVista000 @otsukaikumasan
  @Magmag40887776 @wworeww1 @moe303060 @TEAR\_ACCEL @hdr\_mck
  @Junito04899663 @kmm295 @nyanhechan @Sf8Mf3o2JFKv5uk @karusatonotora
  @E27FdOJwjnhdfpd @ha8ka0 @cp0GUGoG5QAaeRd @Ron82794189 @DJ\_aka\_F2G
  @Emmanuel\_Chanel 選択的夫婦別姓 自民党内の攻防の裏側 \textbar{}
  NHK政治マガジン \url{https://t.co/iM90APAi7p} 
  反対派は、旧姓使用を推してるのですよね?
  どちらが費用が掛からないか、合理的か、ニーズに合うかの視点も必要。
\end{itemize}

〉〉〉 kk\_hironoのリツイート 〉〉〉

\begin{itemize}
\item
  RT
  kk\_hirono(刑事告発・非常上告_金沢地方検察庁御中)|edogawahanako55(ポンコツ)
  日時:2021-04-22 10:57/2021/04/06 22:33 URL:
  \url{https://twitter.com/kk\_hirono/status/1385050161609547778} 
  \url{https://twitter.com/edogawahanako55/status/1379427149447274504} 
  \textgreater{} @2goRPeSy5hsxd37 @NwFle6q9vQTXb4q @Ta\_Howait
  @BuenaVista000 @ibuki\_teika @Nzdaisukiikoiko @nana77rey1
  @otsukaikumasan @Magmag40887776 @REON60206927 @wworeww1
  @vB6T5dn8OdRiTmB @moe303060 @TEAR\_ACCEL @den198804 @hdr\_mck
  @kt\_lec\_serious @FvTF2DxpChUP4pM @Junito04899663 @kmm295 @X2Can3joy
  @schmalkaizen @nyanhechan @VNrespectFN @Sf8Mf3o2JFKv5uk
  @karusatonotora @E27FdOJwjnhdfpd @ha8ka0 @cp0GUGoG5QAaeRd @Ron82794189
  @DJ\_aka\_F2G @Emmanuel\_Chanel @AmemaTV\_kaiino
  外国人との結婚は別姓が認められます
  例外的に、外国人との結婚として扱って別姓で結婚できるようにするのはありじゃないでしょうか
  便宜上、それぞれがどちらも外国人と結婚したものとして扱う事で、同じ処理方法で戸籍登録をするとか
\item
  〉〉〉 アカウント(@motoken\_tw)は,@kk\_hironoをブロックしています。リツイートできませんでした。
  〉〉〉 ¥\n ¥\n \url{https://t.co/HXg39ggz8D} 
\item
  〉〉〉 アカウント(@motoken\_tw)は,@kk\_hironoをブロックしています。リツイートできませんでした。
  〉〉〉 ¥\n ¥\n \url{https://t.co/zL80jyXzQy} 
\item
  〉〉〉 アカウント(@motoken\_tw)は,@kk\_hironoをブロックしています。リツイートできませんでした。
  〉〉〉 ¥\n ¥\n \url{https://t.co/xwsjCQgdna} 
\item
  〉〉〉 アカウント(@motoken\_tw)は,@kk\_hironoをブロックしています。リツイートできませんでした。
  〉〉〉 ¥\n ¥\n \url{https://t.co/yGYleaPtGV} 
\item
  〉〉〉 アカウント(@den198804)は,@kk\_hironoをブロックしています。リツイートできませんでした。
  〉〉〉 ¥\n ¥\n \url{https://t.co/v6dpRuaThL} 
\item
  〉〉〉 アカウント(@den198804)は,@kk\_hironoをブロックしています。リツイートできませんでした。
  〉〉〉 ¥\n ¥\n \url{https://t.co/ISBvV1hZKc} 
\item
  〉〉〉 アカウント(@nana77rey1)は,@kk\_hironoをブロックしています。リツイートできませんでした。
  〉〉〉 ¥\n ¥\n \url{https://t.co/X5M5tiB7Dm} 
\item
  〉〉〉 アカウント(@nana77rey1)は,@kk\_hironoをブロックしています。リツイートできませんでした。
  〉〉〉 ¥\n ¥\n \url{https://t.co/SfMn7i5zCT} 
\item
  〉〉〉 アカウント(@ibuki\_teika)は,@kk\_hironoをブロックしています。リツイートできませんでした。
  〉〉〉 ¥\n ¥\n \url{https://t.co/4bneLlXbfA} 
\end{itemize}

※ @kk\_hironoのアカウントがブロックされ,リツイートに失敗したツイート

\begin{itemize}
\tightlist
\item
  TW motoken\_tw(モトケン) 日時:2021/04/22 01:12:08 URL:
  \url{https://twitter.com/motoken\_tw/status/1384902783829151751} 
  \textgreater{}
  この人、自分の家の中で独り言をつぶやいているのと、広場で大声で自分の主張を叫んでいることの区別がつかないのね。
  \url{https://t.co/1T1ut0qEQT} 
\end{itemize}

※ @kk\_hironoのアカウントがブロックされ,リツイートに失敗したツイート

\begin{itemize}
\tightlist
\item
  TW motoken\_tw(モトケン) 日時:2021/04/21 16:07:55 URL:
  \url{https://twitter.com/motoken\_tw/status/1384765827543552001} 
  \textgreater{} @FvTF2DxpChUP4pM @NwFle6q9vQTXb4q @aurXCanJyw3ePxS
  @den198804 @X2Can3joy @otsukaikumasan @yone717 @kt\_lec\_serious
  @nana77rey1 @vB6T5dn8OdRiTmB @runrunpon @csyuvMhPcpyuLE1 @atr46056249
  @REON60206927 @ibuki\_teika @SeitokaiYakuinF @edogawahanako55
  @schmalkaizen @atr58748830 @Nzdaisukiikoiko @dozeen3 @VNrespectFN
  @madaoorz2 @2goRPeSy5hsxd37 @Ta\_Howait @Magmag40887776 @wworeww1
  @moe303060 @TEAR\_ACCEL @hdr\_mck @Junito04899663 @kmm295 @nyanhechan
  @Sf8Mf3o2JFKv5uk @karusatonotora @E27FdOJwjnhdfpd @ha8ka0
  @cp0GUGoG5QAaeRd @Ron82794189 @Emmanuel\_Chanel 鍵でもかけたら。\\
  \textgreater{} ブロックしてもミュートしてもいいし。\\
  \textgreater{}
  それをしないのなら、公開の場の議論として、あらゆるツイートに批判コメントを書き込むのは、ツイッターユーザーとしての当然の権利。\\
  \textgreater{} 内容的に嫌がらせだと言うのなら通報すればいいのでは?
\end{itemize}

※ @kk\_hironoのアカウントがブロックされ,リツイートに失敗したツイート

\begin{itemize}
\tightlist
\item
  TW motoken\_tw(モトケン) 日時:2021/04/21 15:21:32 URL:
  \url{https://twitter.com/motoken\_tw/status/1384754152119566344} 
  \textgreater{} @FvTF2DxpChUP4pM @NwFle6q9vQTXb4q @aurXCanJyw3ePxS
  @den198804 @X2Can3joy @otsukaikumasan @yone717 @kt\_lec\_serious
  @nana77rey1 @vB6T5dn8OdRiTmB @runrunpon @csyuvMhPcpyuLE1 @atr46056249
  @REON60206927 @ibuki\_teika @SeitokaiYakuinF @edogawahanako55
  @schmalkaizen @atr58748830 @Nzdaisukiikoiko @dozeen3 @VNrespectFN
  @madaoorz2 @2goRPeSy5hsxd37 @Ta\_Howait @Magmag40887776 @wworeww1
  @moe303060 @TEAR\_ACCEL @hdr\_mck @Junito04899663 @kmm295 @nyanhechan
  @Sf8Mf3o2JFKv5uk @karusatonotora @E27FdOJwjnhdfpd @ha8ka0
  @cp0GUGoG5QAaeRd @Ron82794189 @Emmanuel\_Chanel
  介入されたくなかったらブロックすればいんじゃね?\\
  \textgreater{} ミュートしてもいいし。
\end{itemize}

※ @kk\_hironoのアカウントがブロックされ,リツイートに失敗したツイート

\begin{itemize}
\tightlist
\item
  TW motoken\_tw(モトケン) 日時:2021/04/21 15:07:25 URL:
  \url{https://twitter.com/motoken\_tw/status/1384750600433201152} 
  \textgreater{} @NwFle6q9vQTXb4q @aurXCanJyw3ePxS @FvTF2DxpChUP4pM
  @den198804 @X2Can3joy @otsukaikumasan @yone717 @kt\_lec\_serious
  @nana77rey1 @vB6T5dn8OdRiTmB @runrunpon @csyuvMhPcpyuLE1 @atr46056249
  @REON60206927 @ibuki\_teika @SeitokaiYakuinF @edogawahanako55
  @schmalkaizen @atr58748830 @Nzdaisukiikoiko @dozeen3 @VNrespectFN
  @madaoorz2 @2goRPeSy5hsxd37 @Ta\_Howait @Magmag40887776 @wworeww1
  @moe303060 @TEAR\_ACCEL @hdr\_mck @Junito04899663 @kmm295 @nyanhechan
  @Sf8Mf3o2JFKv5uk @karusatonotora @E27FdOJwjnhdfpd @ha8ka0
  @cp0GUGoG5QAaeRd @Ron82794189 @Emmanuel\_Chanel
  あなたは私のクライアントではないから、私に答える義務も責任もないんだよね。\\
  \textgreater{}
  事実婚と法律婚を全く同じにしたら法律婚の意味がないでしょ。
\end{itemize}

※ @kk\_hironoのアカウントがブロックされ,リツイートに失敗したツイート

\begin{itemize}
\tightlist
\item
  TW den198804(ででんのでん) 日時:2021/04/20 06:25:01 URL:
  \url{https://twitter.com/den198804/status/1384256746244296714} 
  \textgreater{} @X2Can3joy @yone717 @kt\_lec\_serious @nana77rey1
  @vB6T5dn8OdRiTmB @runrunpon @csyuvMhPcpyuLE1 @FvTF2DxpChUP4pM
  @atr46056249 @REON60206927 @aurXCanJyw3ePxS @ibuki\_teika
  @SeitokaiYakuinF @edogawahanako55 @schmalkaizen @atr58748830
  @Nzdaisukiikoiko @dozeen3 @VNrespectFN @madaoorz2 @2goRPeSy5hsxd37
  @NwFle6q9vQTXb4q @Ta\_Howait @otsukaikumasan @Magmag40887776 @wworeww1
  @moe303060 @TEAR\_ACCEL @hdr\_mck @Junito04899663 @kmm295 @nyanhechan
  @Sf8Mf3o2JFKv5uk @karusatonotora @E27FdOJwjnhdfpd @ha8ka0
  @cp0GUGoG5QAaeRd @Ron82794189 @Emmanuel\_Chanel
  いいことを言い連ねているだけのあなたはリスク管理ゼロですけどね。\\
  \textgreater{}
  別に夫婦別姓をやりたいわけでもない私が積極的にリスク管理する必要もない。\\
  \textgreater{} 現時点で反対なのだから。\\
  \textgreater{}
  賛成に転じるには最低限、様々なところで出ている問題、懸念についてリスク管理して不安を払拭されてないと。
\end{itemize}

※ @kk\_hironoのアカウントがブロックされ,リツイートに失敗したツイート

\begin{itemize}
\tightlist
\item
  TW den198804(ででんのでん) 日時:2021/04/19 23:28:17 URL:
  \url{https://twitter.com/den198804/status/1384151871938449414} 
  \textgreater{} @yone717 @kt\_lec\_serious @nana77rey1 @vB6T5dn8OdRiTmB
  @runrunpon @csyuvMhPcpyuLE1 @FvTF2DxpChUP4pM @X2Can3joy @atr46056249
  @REON60206927 @aurXCanJyw3ePxS @ibuki\_teika @SeitokaiYakuinF
  @edogawahanako55 @schmalkaizen @atr58748830 @Nzdaisukiikoiko @dozeen3
  @VNrespectFN @madaoorz2 @2goRPeSy5hsxd37 @NwFle6q9vQTXb4q @Ta\_Howait
  @otsukaikumasan @Magmag40887776 @wworeww1 @moe303060 @TEAR\_ACCEL
  @hdr\_mck @Junito04899663 @kmm295 @nyanhechan @Sf8Mf3o2JFKv5uk
  @karusatonotora @E27FdOJwjnhdfpd @ha8ka0 @cp0GUGoG5QAaeRd @Ron82794189
  @Emmanuel\_Chanel いいんじゃない。\\
  \textgreater{} 特に反対はしないよ。\\
  \textgreater{}
  法的な結婚まで踏み切れないのだから中途半端だけど、待遇も中途半端だし、分相応だと思うしね。
\end{itemize}

※ @kk\_hironoのアカウントがブロックされ,リツイートに失敗したツイート

\begin{itemize}
\tightlist
\item
  TW nana77rey1(井田奈穂/Naho Ida/選択的夫婦別姓・全国陳情アクション)
  日時:2021/04/19 00:02:28 URL:
  \url{https://twitter.com/nana77rey1/status/1383798086686822408} 
  \textgreater{} @yone717 @kt\_lec\_serious @vB6T5dn8OdRiTmB @runrunpon
  @csyuvMhPcpyuLE1 @FvTF2DxpChUP4pM @X2Can3joy @atr46056249
  @REON60206927 @aurXCanJyw3ePxS @ibuki\_teika @SeitokaiYakuinF
  @edogawahanako55 @schmalkaizen @atr58748830 @den198804
  @Nzdaisukiikoiko @dozeen3 @VNrespectFN @madaoorz2 @2goRPeSy5hsxd37
  @NwFle6q9vQTXb4q @Ta\_Howait @otsukaikumasan @Magmag40887776 @wworeww1
  @moe303060 @TEAR\_ACCEL @hdr\_mck @Junito04899663 @kmm295 @nyanhechan
  @Sf8Mf3o2JFKv5uk @karusatonotora @E27FdOJwjnhdfpd @ha8ka0
  @cp0GUGoG5QAaeRd @Ron82794189 @Emmanuel\_Chanel
  お相手あっての結婚だからねえ。
\end{itemize}

※ @kk\_hironoのアカウントがブロックされ,リツイートに失敗したツイート

\begin{itemize}
\tightlist
\item
  TW nana77rey1(井田奈穂/Naho Ida/選択的夫婦別姓・全国陳情アクション)
  日時:2021/04/06 11:39:31 URL:
  \url{https://twitter.com/nana77rey1/status/1379262463682613249} 
  \textgreater{} @Ta\_Howait @otsukaikumasan @BuenaVista000
  @Nzdaisukiikoiko @Magmag40887776 @ibuki\_teika @REON60206927 @wworeww1
  @2goRPeSy5hsxd37 @vB6T5dn8OdRiTmB @moe303060 @TEAR\_ACCEL @den198804
  @hdr\_mck @kt\_lec\_serious @FvTF2DxpChUP4pM @Junito04899663 @kmm295
  @X2Can3joy @schmalkaizen @nyanhechan @VNrespectFN @Sf8Mf3o2JFKv5uk
  @karusatonotora @E27FdOJwjnhdfpd @ha8ka0 @cp0GUGoG5QAaeRd @Ron82794189
  @DJ\_aka\_F2G @Emmanuel\_Chanel @AmemaTV\_kaiino
  この点ではおつかいくまさん分かっててくださって良かったです。父母が生まれ持った氏名の方がより辿りやすい、はあると思いますが、別にそうでなくても辿りやすさは今と変わりません。\\
  \textgreater{} 記載項目なにも変わらないので。 \url{https://t.co/MPKce2Nv0p} 
\end{itemize}

※ @kk\_hironoのアカウントがブロックされ,リツイートに失敗したツイート

\begin{itemize}
\tightlist
\item
  TW ibuki\_teika(🇯🇵息吹定家「パパ保守侍」) 日時:2021/02/14 08:04:16
  URL: \url{https://twitter.com/ibuki\_teika/status/1360726514199760897} 
  \textgreater{} 【憲法改正するチャンスを私達国民に下さい🇯🇵】\\
  \textgreater{}\\
  \textgreater{} @BuenaVista000\\
  \textgreater{} @Sf8Mf3o2JFKv5uk\\
  \textgreater{} @2goRPeSy5hsxd37\\
  \textgreater{} @karusatonotora\\
  \textgreater{} @E27FdOJwjnhdfpd\\
  \textgreater{} @ha8ka0\\
  \textgreater{} @den198804\\
  \textgreater{} @cp0GUGoG5QAaeRd\\
  \textgreater{} @kt\_lec\_serious\\
  \textgreater{} @Ron82794189\\
  \textgreater{} @DJ\_aka\_F2G\\
  \textgreater{} @TEAR\_ACCEL\\
  \textgreater{} @Emmanuel\_Chanel\\
  \textgreater{} @AmemaTV\_kaiino\\
  \textgreater{}\\
  \textgreater{} \#憲法改正を望みます \url{https://t.co/WxG9GfAsMF} 
\end{itemize}

 3つほどモトケンこと矢部善朗弁護士(京都弁護士会)の他にブロックされているアカウントがありました。前にもメンションが入っていたものと思われますが,モトケンこと矢部善朗弁護士(京都弁護士会)との関係性が理由ということもあるのかもしれません。

 モトケンこと矢部善朗弁護士(京都弁護士会)のプロフィールにある元検事という肩書も影響がありそうに思えます。モトケンこと矢部善朗弁護士(京都弁護士会)の強引さも際立つところで,今後の参考のためにも取り上げておきました。

\begin{itemize}
\tightlist
\item
  〈〈〈 2021/04/22 11:09:21 Linux Emacs: 〈〈〈
\end{itemize}

\hypertarget{ux7d20ux4ebaux3084ux3069ux7d20ux4ebaux306bux5f37ux3044ux3053ux3060ux308fux308aux3092ux793aux3057ux3066ux304dux305fux30e2ux30c8ux30b1ux30f3ux3053ux3068ux77e2ux90e8ux5584ux6717ux5f01ux8b77ux58ebux4eacux90fdux5f01ux8b77ux58ebux4f1aux306eux30c4ux30a4ux30fcux30c8ux306eux8a18ux9332}{%
\paragraph{「素人」や「ど素人」に強いこだわりを示してきたモトケンこと矢部善朗弁護士(京都弁護士会)のツイートの記録}\label{ux7d20ux4ebaux3084ux3069ux7d20ux4ebaux306bux5f37ux3044ux3053ux3060ux308fux308aux3092ux793aux3057ux3066ux304dux305fux30e2ux30c8ux30b1ux30f3ux3053ux3068ux77e2ux90e8ux5584ux6717ux5f01ux8b77ux58ebux4eacux90fdux5f01ux8b77ux58ebux4f1aux306eux30c4ux30a4ux30fcux30c8ux306eux8a18ux9332}}

\begin{itemize}
\tightlist
\item
  〉〉〉 Linux Emacs: 2021/04/22 11:21:53 〉〉〉
\end{itemize}

:CATEGORIES: @kanazawabengosi \#金沢弁護士会 @JFBAsns
日本弁護士連合会(日弁連) \#法務省 @MOJ\_HOUMU
\#モトケンこと矢部善朗弁護士(京都弁護士会)

\begin{lstlisting}
py37_env ❯ d|grep @motoken_tw|grep 素人
\end{lstlisting}

\begin{itemize}
\tightlist
\item
  2017年10月02日18時08分の登録:
  REGEXP:''ど素人''/モトケンこと矢部善朗弁護士(京都弁護士会)(@motoken\_tw)のツイートの記録(2010-09-25〜2017-07-28/2017年10月02日18時07分・44件)
  \url{http://hirono2014sk.blogspot.com/2017/10/regexpmotokentw2010-09-252017-07.html} 
\item
  2017年10月02日18時10分の登録:
  %@motoken\_tw モトケン%ど素人が安田弁護士を批判するのはまあ仕方がないとして、マスコミはもうちっと刑事弁護を勉強したほうがいいんでないかな
  \url{http://hirono2014sk.blogspot.com/2017/10/motokentw\_12.html} 
\item
  2017年10月26日16時13分の登録:
  REGEXP:''素人''/モトケン(@motoken\_tw)の検索(2017-01-10〜2017-09-23/2017年10月26日16時13分の記録60件)
  \url{http://hirono2014sk.blogspot.com/2017/10/regexpmotokentw2017-01-102017-09.html} 
\item
  2017年11月22日21時58分の登録:
  \モトケン @motoken\_tw RT: @lawkus\どうしてツイッタラーは素人のくせにプロに独自見解を披露しようと思ってしまうのか
  \url{http://hirono2014sk.blogspot.com/2017/11/motokentwrtlawkus.html} 
\item
  2017年12月07日04時31分の登録:
  %@motoken\_tw モトケン%想定しているギャラリーは素人さんなのでご容赦を
  m(\_ \_)m \url{http://hirono2014sk.blogspot.com/2017/12/motokentw-m-m.html} 
\item
  2017年12月16日07時21分の登録:
  %@motoken\_tw モトケン%そのド素人が俳優でも中学の教師でも信頼感が増すというわけではないのだが。実名を尊重するのであれば、当然そのバックグラウンドも尊重すべきはずなんだけどな。
  \url{http://hirono2014sk.blogspot.com/2017/12/motokentw\_74.html} 
\item
  2017年12月30日13時07分の登録:
  \モトケン @motoken\_tw\ツイッターあるあるなんだけど、素人が玄人に意見を言って(そのこと自体は何の問題もない)、反論されたときに、ひょっとしたら自分が間違って
  \url{http://hirono2014sk.blogspot.com/2017/12/motokentw\_91.html} 
\item
  2017年12月30日13時08分の登録:
  \モトケン @motoken\_tw\ツイッターあるあるなんだけど、素人が玄人に意見を言って(そのこと自体は何の問題もない)、反論されたときに、ひょっとしたら自分が間違って
  \url{http://hirono2014sk.blogspot.com/2017/12/motokentw\_49.html} 
\item
  2017年12月30日23時33分の登録:
  \モトケン @motoken\_tw\法律の素人だからといって、他人が言ってもいないことを言ったと言っていいことにはならない。そういうデマは、言論界における冤罪と言ってもい
  \url{http://hirono2014sk.blogspot.com/2017/12/motokentw\_23.html} 
\item
  2018年01月01日21時50分の登録:
  \モトケン @motoken\_tw\ああいう文章テクニックはね、たちの悪い刑事が純朴な被疑者を騙して虚偽自白調書を作るときと同じなんだよ。悪徳弁護士が素人を騙して一方的に
  \url{http://hirono2014sk.blogspot.com/2018/01/motokentw.html} 
\item
  2018年01月04日20時15分の登録:
  \モトケン @motoken\_tw\素人さん相手に揚げ足取りっぽくて申し訳ないが、本件は司法の問題じゃないですからね。司法は逮捕令状を出してる。それを執行しなかったのは行
  \url{http://hirono2014sk.blogspot.com/2018/01/motokentw\_68.html} 
\item
  2018年01月12日09時31分の登録:
  \モトケン @motoken\_tw\あなたが何も知らない素人だと言ってるのではないですよ。私はフォロワーを意識してますのでね。ここは二人で議論しているのではなく公開の場な
  \url{http://hirono2014sk.blogspot.com/2018/01/motokentw\_0.html} 
\item
  2018年01月17日11時21分の登録:
  \モトケン @motoken\_tw\勘違いしてるかも知れないけど、君のツイートに注目してるのは、素人さんじゃなくて同業者さんだよ。
  \url{http://hirono2014sk.blogspot.com/2018/01/motokentw\_40.html} 
\item
  2018年01月23日10時37分の登録:
  \モトケン @motoken\_tw\刑事司法の仕組みや実情を何も知らない記者とかジャーナリストが妄想記事を書き、それを鵜呑みにする刑事司法ど素人の大学教授とかなんちゃらが
  \url{http://hirono2014sk.blogspot.com/2018/01/motokentw\_81.html} 
\item
  2018年02月14日10時37分の登録:
  \モトケン @motoken\_tw\素人がそれを言い出しても何の益もないし害ばかり、ということだと思います。
  \url{http://hirono2014sk.blogspot.com/2018/02/motokentw\_98.html} 
\item
  2018年03月07日14時04分の登録:
  \モトケン @motoken\_tw\法律のプロを相手にど素人が法律の解釈論の議論をするつもりですか?
  私はそんな暇はない。
  \url{http://hirono2014sk.blogspot.com/2018/03/motokentw\_63.html} 
\item
  2018年03月24日02時26分の登録:
  \モトケン @motoken\_tw\伝聞推定という言葉は知らないな。「素人のうろ覚えの知ったかぶりの知識」をつぶやいて恥ずかしくないかね。ないんだろうな。つぶやいてるんだ
  \url{http://hirono2014sk.blogspot.com/2018/03/motokentw\_72.html} 
\item
  2018年03月28日18時54分の登録:
  \モトケン @motoken\_tw\素人が専門職につっ掛かるのはどう言うべきかな?
  \url{http://hirono2014sk.blogspot.com/2018/03/motokentw\_49.html} 
\item
  2018年04月01日21時38分の登録:
  \モトケン @motoken\_tw\この自称「素人」さんは、正確な知識を持っているのだろうか?「素人」なのに?
  \url{http://hirono2014sk.blogspot.com/2018/04/motokentw\_56.html} 
\item
  2018年04月11日21時26分の登録:
  \モトケン @motoken\_tw\素人かと思ったらプロフに弁護士と書いてあったw
  以前はツイートを見れば弁護士かどうかかなりの確率で分かったんだけどな。
  \url{http://hirono2014sk.blogspot.com/2018/04/motokentww\_11.html} 
\item
  2018年04月13日05時46分の登録:
  REGEXP:''素人''/モトケン(@motoken\_tw)の検索(2010-04-06〜2018-04-11/2018年04月13日05時46分の記録375件)
  \url{http://hirono2014sk.blogspot.com/2018/04/regexpmotokentw2010-04-062018-04.html} 
\item
  2018年04月13日05時46分の登録:
  REGEXP:''素人さん''/モトケン(@motoken\_tw)の検索(2010-11-13〜2018-01-16/2018年04月13日05時46分の記録47件)
  \url{http://hirono2014sk.blogspot.com/2018/04/regexpmotokentw2010-11-132018-01\_13.html} 
\item
  2018年07月14日20時44分の登録:
  \モトケン @motoken\_tw\この人、自分が何の議論をしているのか分かってない。ほとんどの素人さんはそうだけど。
  \url{http://hirono2014sk.blogspot.com/2018/07/motokentw\_14.html} 
\item
  2018年07月25日09時01分の登録:
  \モトケン @motoken\_tw\ど素人がプロ相手に言いがかりつけてるっていう自覚ある?
  \url{http://hirono2014sk.blogspot.com/2018/07/motokentw\_25.html} 
\item
  2018年08月08日17時47分の登録:
  REGEXP:''素人''/モトケン(@motoken\_tw)の検索(2010-04-06〜2018-07-26/2018年08月08日17時47分の記録387件)
  \url{http://hirono2014sk.blogspot.com/2018/08/regexpmotokentw2010-04-062018-07.html} 
\item
  2018年09月27日19時49分の登録:
  \モトケン @motoken\_tw\自分は素人だということを忘れている人がいるようだ。そういう人がツイッターを使うとどうなるかということを考えたほうがいいな。
  \url{http://hirono2014sk.blogspot.com/2018/09/motokentw\_11.html} 
\item
  2018年10月05日22時41分の登録:
  \モトケン @motoken\_tw\社会学者さんたちは、バーチャルアイドルに対する性的搾取とか市民的公共性とか素人にはとてもわかりにくい言葉を使うので、その真意を推し量る
  \url{http://hirono2014sk.blogspot.com/2018/10/motokentw\_5.html} 
\item
  2018年11月03日00時04分の登録:
  \モトケン @motoken\_tw\素人がそんな理屈で動くと思ってるんですか?裁判を起こせば金になる、と知ったら裁判を起こしますよ。判決であろうが和解であろうが。
  \url{http://hirono2014sk.blogspot.com/2018/11/motokentw\_3.html} 
\item
  2018年11月26日14時13分の登録:
  \モトケン @motoken\_tw\民事も刑事も司法というのはとても技術的なものだ、ということを素人さんたちには理解してほしいな。
  \url{http://hirono2014sk.blogspot.com/2018/11/motokentw\_26.html} 
\item
  2018年12月14日18時30分の登録:
  \モトケン @motoken\_tw\丁寧に説明すればするほど、素人さんにはわかりにくくなるのが、過失犯や結果的加重犯。
  \url{http://hirono2014sk.blogspot.com/2018/12/motokentw\_14.html} 
\item
  2019年01月30日13時31分の登録:
  \モトケン @motoken\_tw\法律問題で、弁護士の見解よりど素人の見解を信じるメンタリティは不思議としか言いようがないのだが、そういう人がたくさんいることをツイッタ
  \url{http://hirono2014sk.blogspot.com/2019/01/motokentw\_30.html} 
\item
  2019年03月13日15時33分の登録:
  \モトケン @motoken\_tw\ど素人が何が問題なのか理解できずにプロに絡んでいる、という自覚はないの?
  \url{http://hirono2014sk.blogspot.com/2019/03/motokentw\_73.html} 
\item
  2019年03月22日20時03分の登録:
  \モトケン @motoken\_tw\¥\n法律素人の一般人が判決の原文を読んでどの程度理解できるかも疑問です。
  \url{http://hirono2014sk.blogspot.com/2019/03/motokentw\_74.html} 
\item
  2019年04月13日23時45分の登録:
  \モトケン @motoken\_tw\無罪判決批判をするど素人に対して、弁護士から「判決を読んだのか?」「判決を読んでから批判しろ!」という言葉が投げつけられているが、判決
  \url{http://hirono2014sk.blogspot.com/2019/04/motokentw\_19.html} 
\item
  2019年04月14日01時27分の登録:
  \モトケン @motoken\_tw\(承前)¥\nしかし、判決が公開されておらず読む手段もないど素人に対してそのように言うと、言われたど素人としては、「要するに、ど素人は無罪
  \url{http://hirono2014sk.blogspot.com/2019/04/motokentw\_33.html} 
\item
  2019年04月24日10時38分の登録:
  \モトケン @motoken\_tw\素人が無罪判決を批判したときに、法クラが判決を読まずに批判できない、と言うと、その法クラも判決を見ていない場合は、法クラ自身が判決は正
  \url{http://hirono2014sk.blogspot.com/2019/04/motokentw\_30.html} 
\item
  2019年05月06日15時01分の登録:
  \モトケン @motoken\_tw\ということを素人のみなさんは理解してほしい。¥\n一つのツイートの背景で検討している情報量が桁違いに違うんです。
  \url{http://hirono2014sk.blogspot.com/2019/05/motokentw\_82.html} 
\item
  2019年05月09日10時56分の登録:
  \モトケン @motoken\_tw\根本的な問題として、あなたは報道でしか情報を得ていないだろう。私もそうだ。ど素人と違って、プロはその程度の情報で確定的なことは言わんの
  \url{http://hirono2014sk.blogspot.com/2019/05/motokentw\_54.html} 
\item
  2019年05月12日12時51分の登録: \モトケン @motoken\_tw\返信先:
  @Hideo\_Oguraさん¥\n静岡地裁の無罪判決に対する素人さんの批判はまだ目にしてないので、私のツイートの対象外。
  \url{http://hirono2014sk.blogspot.com/2019/05/motokentw-hideoogura.html} 
\item
  2019年05月14日08時02分の登録: \モトケン @motoken\_tw\返信先:
  @imarockcaster42さん¥\n「ど素人」のほうがよかったですか?
  \url{http://hirono2014sk.blogspot.com/2019/05/motokentw-imarockcaster42.html} 
\item
  2019年05月20日18時59分の登録:
  \モトケン @motoken\_tw\素人さん向けに補足説明¥\n人の死亡の立証は、医師による医学的基準に基づく死亡診断書等で立証します。¥\n人の傷害の立証は、これも医師による診
  \url{http://hirono2014sk.blogspot.com/2019/05/motokentw\_261.html} 
\item
  2019年07月05日17時24分の登録:
  \モトケン @motoken\_tw\不同意性交罪関係の議論で感じるのは、実務家は常に立証というものを考えているので、その点で、素人の一般人はもちろん学者とも感覚が違うとい
  \url{http://hirono2014sk.blogspot.com/2019/07/motokentw\_16.html} 
\item
  2019年07月10日16時36分の登録:
  \モトケン @motoken\_tw\結果として朝日は誤報したことに変わりはない。情報戦の世界では騙される方がバカ。新聞社は素人投資家ではなくて情報処理のプロですよ。
  \url{http://hirono2014sk.blogspot.com/2019/07/motokentw\_39.html} 
\item
  2019年09月06日21時41分の登録:
  \モトケン @motoken\_tw\ツイッターで素人さんの質問に答えるのは、依頼者に説明するときの練習になるので弁護士として有益なんだけど、素人なのに「俺の方がわかってる
  \url{http://hirono2014sk.blogspot.com/2019/09/motokentw\_6.html} 
\item
  2019年09月06日21時49分の登録:
  \モトケン @motoken\_tw\法律問題に関するツイートに対してど素人さんから「貴方本当に弁護士なんですか?」というようなリプをもらうことが珍しくないツイッターw¥\nど
  \url{http://hirono2014sk.blogspot.com/2019/09/motokentww.html} 
\item
  2019年10月04日12時25分の登録:
  \モトケン @motoken\_tw\裁判を起こしたり起こされた素人さんの中には、自分が正しい、と思っている人が多い。実際、その人が正しい事件は多い。¥\nそういう人は、自分が
  \url{http://hirono2014sk.blogspot.com/2019/10/motokentw\_4.html} 
\item
  2019年10月04日12時27分の登録:
  \モトケン @motoken\_tw\多くの素人さんは、自分が言いたいことを言うのに急で、勝つために言う必要があることは何かを考えない。¥\n言い分が食い違った時にどうするのが
  \url{http://hirono2014sk.blogspot.com/2019/10/motokentw\_33.html} 
\item
  2019年10月04日12時28分の登録:
  \モトケン @motoken\_tw\こういう場合に弁護士を頼まないというのは、家を建てるときに、大工に金を払うのはもったいないと言って、素人が自分で家を建てて、雨漏りに悩
  \url{http://hirono2014sk.blogspot.com/2019/10/motokentw\_68.html} 
\item
  2019年10月31日19時44分の登録:
  \モトケン @motoken\_tw\「専門家が素人をタコ殴りにしてる。」¥\nそんなことを言ってる間抜けな人がいるんですか?¥\nそういう問題じゃないですよね。¥\n市政に影響力のあ
  \url{http://hirono2014sk.blogspot.com/2019/10/motokentw\_48.html} 
\item
  2019年11月13日10時26分の登録:
  \モトケン @motoken\_tw\同じような俺なんでも知ってるタイプのど素人が毎日のように現れるなw
  \url{http://hirono2014sk.blogspot.com/2019/11/motokentww.html} 
\item
  2019年12月19日21時59分の登録:
  REGEXP:''素人''/モトケン(@motoken\_tw)の検索(2010-04-06〜2019-12-09/2019年12月19日21時59分の記録476件)
  \url{http://hirono2014sk.blogspot.com/2019/12/regexpmotokentw2010-04-062019-12\_19.html} 
\item
  2020年01月05日16時52分の登録:
  REGEXP:''素人''/モトケン(@motoken\_tw)の検索(2010-04-06〜2020-01-01/2020年01月05日16時52分の記録482件)
  \url{http://hirono2014sk.blogspot.com/2020/01/regexpmotokentw2010-04-062020-01.html} 
\item
  2020年01月05日17時02分の登録:
  REGEXP:''素人''/モトケン(@motoken\_tw)の検索(2019-12-09〜2020-01-01/2020年01月05日17時02分の記録7件)
  \url{http://hirono2014sk.blogspot.com/2020/01/regexpmotokentw2019-12-092020-01.html} 
\item
  2020年01月05日18時17分の登録:
  REGEXP:''素人さん''/モトケン(@motoken\_tw)の検索(2019-12-09〜2019-12-31/2020年01月05日18時17分の記録2件)
  \url{http://hirono2014sk.blogspot.com/2020/01/regexpmotokentw2019-12-092019-12.html} 
\item
  2020年01月11日13時40分の登録:
  \モトケン @motoken\_tw\「ヤメ検がかつてのコネ使って不起訴『事実上の無罪』に誘導」¥\n素人さんが思うほどコネはものを言わない。偉いさんでも辞めた瞬間に過去の人。
  \url{http://hirono2014sk.blogspot.com/2020/01/motokentw\_22.html} 
\item
  2020年01月13日18時07分の登録:
  REGEXP:''素人さん''/モトケン(@motoken\_tw)の検索(2010-11-13〜2020-01-11/2020年01月13日18時07分の記録78件)
  \url{http://hirono2014sk.blogspot.com/2020/01/regexpmotokentw2010-11-132020-01.html} 
\item
  2020年05月12日14時14分の登録:
  \モトケン @motoken\_tw\元検事で現弁護士の検察庁法改正法案批判に対して、どうしても論破したいという素人さんがけっこういるんですね。
  \url{http://hirono2014sk.blogspot.com/2020/05/motokentw\_12.html} 
\item
  2020年05月17日12時12分の登録:
  \モトケン @motoken\_tw\検察庁と内閣の問題は選挙で解決すればいいと言う人が多いのだが、こういう素人さんたちが、内閣と検察庁の問題について、選挙で正しい判断がで
  \url{http://hirono2014sk.blogspot.com/2020/05/motokentw\_74.html} 
\item
  2020年08月13日10時40分の登録:
  REGEXP:''素人''/モトケン(@motoken\_tw)の検索(2010-04-06〜2020-08-13/2020年08月13日10時40分の記録507件)
  \url{http://hirono2014sk.blogspot.com/2020/08/regexpmotokentw2010-04-062020-08.html} 
\item
  2020年09月27日22時55分の登録:
  \モトケン @motoken\_tw\私のツイートに対して、犯罪論も規範論も行動心理学もかじったことも舐めたこともないど素人が、小児性犯罪を助長する、と批判(より正確にはは
  \url{http://kk2020-09.blogspot.com/2020/09/motokentw\_55.html} 
\item
  2020年10月09日21時39分の登録:
  \モトケン @motoken\_tw\専門家の説明を聞いた時の素人の対応は大きく分けて二つに別れる。 一つは、専門家の話を尊重して理解しようと努める人。 もう一つは、自分の
  \url{http://kk2020-09.blogspot.com/2020/10/motokentw\_29.html} 
\item
  2020年10月09日21時48分の登録:
  \モトケン @motoken\_tw\>どうも一般人を見下しているようだ。 素人は専門分野の知識や理解が乏しいから素人なんです。 何か法律的な問題が起こった時は自分で解決し
  \url{http://kk2020-09.blogspot.com/2020/10/motokentw\_77.html} 
\item
  2020年10月11日22時46分の登録:
  \モトケン @motoken\_tw\私のツイートを見て素人を「排除」すると読み取る人がいるんだな。 そういう人とはまともなコミュニケーションが取れそうもないな。
  \url{http://kk2020-09.blogspot.com/2020/10/motokentw\_11.html} 
\item
  2021年01月06日20時33分の登録:
  \モトケン @motoken\_tw\こういう問題における素人と専門家の違いは、制度の一部を変えた場合に、その影響が制度全体のどこにどのように及ぶかという点についての視野の
  \url{http://kk2020-09.blogspot.com/2021/01/motokentw\_91.html} 
\item
  2021年02月17日23時57分の登録:
  REGEXP:''素人''/モトケン(@motoken\_tw)の検索(2010-04-06〜2021-01-06/2021年02月17日23時57分の記録539件)
  \url{https://kk2020-09.blogspot.com/2021/02/regexpmotokentw2010-04-062021-01.html} 
\item
  2021年02月23日10時56分の登録:
  REGEXP:''素人''/モトケン(@motoken\_tw)の検索(2010-04-06〜2021-01-06/2021年02月23日10時56分の記録539件)
  \url{https://kk2020-09.blogspot.com/2021/02/regexpmotokentw2010-04-062021-01\_23.html} 
\item
  2021年03月21日00時15分の登録: \モトケン @motoken\_tw\返信先:
  @y6xMU5fk07Py0Xsさんいるでしょうね。素人に比べたら圧倒的に少ないと思いますけど。
  \url{https://kk2020-09.blogspot.com/2021/03/motokentw-y6xmu5fk07py0xs.html} 
\item
  2021年03月21日00時18分の登録: \モトケン @motoken\_tw\返信先:
  @PQDTGqUcrUgrm0yさん他の弁護士や裁判官。少なくとも法律の素人ではない。
  \url{https://kk2020-09.blogspot.com/2021/03/motokentw-pqdtgqucrugrm0y.html} 
\item
  2021年03月21日16時27分の登録:
  \モトケン @motoken\_tw\弁護士垢が自分の専門分野である法律問題について、素人垢と「議論」する意味はほとんどない。法的解決の基礎または前提となる事実関係について
  \url{https://kk2020-09.blogspot.com/2021/03/motokentw\_34.html} 
\item
  2021年03月22日16時52分の登録:
  REGEXP:''(素人|専門家)''/モトケン(@motoken\_tw)の検索(2010-03-15〜2021-03-22/2021年03月22日16時52分の記録856件)
  \url{https://kk2020-09.blogspot.com/2021/03/regexpmotokentw2010-03-152021-03.html} 
\item
  2021年03月24日01時38分の登録:
  \モトケン @motoken\_tw\素人同士が議論しても仕方がない問題ですが、素人に毛が生えた程度にはデータベースを知っているものとしては、大量の人的リソースを必要とする
  \url{https://kk2020-09.blogspot.com/2021/03/motokentw\_24.html} 
\item
  2021年03月30日13時36分の登録:
  \モトケン @motoken\_tw\この人、ど素人がプロと会話しているという意識がないのだろうか?
  \url{https://kk2020-09.blogspot.com/2021/03/motokentw\_8.html} 
\item
  2021年04月22日11時11分の登録:
  REGEXP:''素人''/モトケン(@motoken\_tw)の検索(2010-04-06〜2021-03-29/2021年04月22日11時11分の記録552件)
  \url{https://kk2020-09.blogspot.com/2021/04/regexpmotokentw2010-04-062021-03.html} 
\item
  2021年04月22日11時12分の登録:
  REGEXP:''ど素人''/モトケン(@motoken\_tw)の検索(2010-09-25〜2021-03-29/2021年04月22日11時12分の記録95件)
  \url{https://kk2020-09.blogspot.com/2021/04/regexpmotokentw2010-09-252021-03.html} 
\end{itemize}

 思ったよりだいぶん多く数がありました。「❯ d\textbar grep
@motoken\_tw\textbar grep 素人\textbar wc
-l」というコマンドで数えると74件です。

 ここ最近は見ていないのですが,「REGEXP:''ど素人''/モトケン(@motoken\_tw)の検索(2010-09-25〜2021-03-29/2021年04月22日11時12分の記録95件)」から確認してみます。

\begin{itemize}
\item
  (89/95) TW motoken\_tw(モトケン) 日時:2020-05-19 07:45:00 +0900
  URL:
  \url{https://twitter.com/motoken\_tw/status/1262514560705032192\textgreater} {}
  @ddslumber
  ど素人からさんざんクソリプを投げつけられたプロから言わせてもらうと、素人が勉強不足なのは当たり前なので、勉強不足であることは何の問題もないが、勉強不足であるということを自覚しないで(プロより自分のほうがわかっ・・・
  \url{https://t.co/i75AdrWOxW} 
\item
  (90/95) TW motoken\_tw(モトケン) 日時:2020-09-27 12:43:00 +0900
  URL:
  \url{https://twitter.com/motoken\_tw/status/1310062535291281409\textgreater} {}
  私のツイートに対して、犯罪論も規範論も行動心理学もかじったことも舐めたこともないど素人が、小児性犯罪を助長する、と批判(より正確にはは誤読)することは想定内だが、ツイッターの運営がそのレベルだとツイッターが議論のプラットフォームとして地位を主張することはおこがまし過ぎる。\textgreater{}
  (続く
\item
  (91/95) TW motoken\_tw(モトケン) 日時:2020-10-10 12:40:00 +0900
  URL:
  \url{https://twitter.com/motoken\_tw/status/1314772770203426816\textgreater} {}
  @saori3 思ってるんでしょう。\textgreater{}
  責任の有無程度を決めるのはマスコミでもど素人の感情的な人でもなく裁判所です。
\item
  (92/95) TW motoken\_tw(モトケン) 日時:2020-11-01 19:53:00 +0900
  URL:
  \url{https://twitter.com/motoken\_tw/status/1322854183943380992\textgreater} {}
  @nanchattei\_f @h\_jisaburou 法律ど素人の精神科医がLegal
  riskを語る(騙る)わけですね。\textgreater{}
  そこまで仰るのなら、実名でAED配備反対運動でもされたらいかがですか?
\item
  (93/95) TW motoken\_tw(モトケン) 日時:2021-03-20 12:13:24 +0900
  URL:
  \url{https://twitter.com/motoken\_tw/status/1373110398673584132\textgreater} {}
  @turbo0421 @abengers2 @code\_dia
  ど素人さんに対して\textgreater\textgreater{}
  私は、判決のこの部分(具体的に引用)を読んでこう思いました\textgreater\textgreater{}
  ということを素人に求めることが不当ですか。\textgreater\textgreater{}
  >高市案\textgreater\textgreater{}
  の通称姓についての連続ツイート\textgreater{} \url{https://t.co/TSyQVFuk4w} 
\item
  (94/95) TW motoken\_tw(モトケン) 日時:2021-03-21 14:48:52 +0900
  URL:
  \url{https://twitter.com/motoken\_tw/status/1373511909639557121\textgreater} {}
  つまり、専門家が自分の専門分野についてツイートするというのは、ニュースの解説をしたり、世間の人の誤解を正したり、または専門家同士の議論をしている場合が多いわけです。\textgreater{}
  そこにど素人さんが割って入って自己主張される勇気には敬服いたします。\textgreater{}
  専門家の意見の誤りは別の専門家が正します。
\item
  (95/95) TW motoken\_tw(モトケン) 日時:2021-03-29 18:30:44 +0900
  URL:
  \url{https://twitter.com/motoken\_tw/status/1376466846010859521\textgreater} {}
  あまり言いたくはないが(と言いつつしばしば言っていることなのだが)、この人、ど素人がプロと会話しているという意識がないのだろうか?
  \url{https://t.co/HeOjtYP1j4} 
\end{itemize}

 3月21日のツイートも3月29日のツイートも余り見覚えがないと思ったのですが,データベースの記録を確認したところ次のものが記録されていました。

\begin{itemize}
\item
  2021年03月30日13時36分の登録:
  \モトケン @motoken\_tw\この人、ど素人がプロと会話しているという意識がないのだろうか?
  \url{https://kk2020-09.blogspot.com/2021/03/motokentw\_8.html} 
\item
  2021年03月21日16時27分の登録:
  \モトケン @motoken\_tw\弁護士垢が自分の専門分野である法律問題について、素人垢と「議論」する意味はほとんどない。法的解決の基礎または前提となる事実関係について
  \url{https://kk2020-09.blogspot.com/2021/03/motokentw\_34.html} 
\item
  〈〈〈 2021/04/22 11:44:00 Linux Emacs: 〈〈〈
\end{itemize}

\hypertarget{ux88c1ux5224ux306bux5f37ux3044ux5f01ux8b77ux58ebux306fux8133ux307fux305dux306eux5165ux308cux66ffux3048ux3067ux66f8ux9762ux306eux7df4ux5ea6ux3092ux4e0aux3052ux308bux306eux304cux4e0aux624bux3044ux3068ux3044ux3046ux30e2ux30c8ux30b1ux30f3ux3053ux3068ux77e2ux90e8ux5584ux6717ux5f01ux8b77ux58ebux4eacux90fdux5f01ux8b77ux58ebux4f1aux306eux5f15ux7528ux30c4ux30a4ux30fcux30c8}{%
\paragraph{「裁判に強い弁護士は「脳みその入れ替え」で書面の練度を上げるのが上手い。」というモトケンこと矢部善朗弁護士(京都弁護士会)の引用ツイート}\label{ux88c1ux5224ux306bux5f37ux3044ux5f01ux8b77ux58ebux306fux8133ux307fux305dux306eux5165ux308cux66ffux3048ux3067ux66f8ux9762ux306eux7df4ux5ea6ux3092ux4e0aux3052ux308bux306eux304cux4e0aux624bux3044ux3068ux3044ux3046ux30e2ux30c8ux30b1ux30f3ux3053ux3068ux77e2ux90e8ux5584ux6717ux5f01ux8b77ux58ebux4eacux90fdux5f01ux8b77ux58ebux4f1aux306eux5f15ux7528ux30c4ux30a4ux30fcux30c8}}

\begin{itemize}
\tightlist
\item
  〉〉〉 Linux Emacs: 2021/04/26 05:53:23 〉〉〉
\end{itemize}

:CATEGORIES: @kanazawabengosi \#金沢弁護士会 @JFBAsns
日本弁護士連合会(日弁連) \#法務省 @MOJ\_HOUMU
\#モトケンこと矢部善朗弁護士(京都弁護士会)

※ @kk\_hironoのアカウントがブロックされ,リツイートに失敗したツイート

\begin{itemize}
\tightlist
\item
  TW motoken\_tw(モトケン) 日時:2021/04/25 18:48:43 URL:
  \url{https://twitter.com/motoken\_tw/status/1386255844904374272} 
  \textgreater{} これ、私が何度か言ってる「三手の読み」の話。\\
  \textgreater{} 一晩あけるかどうかは別にして。\\
  \textgreater{}
  「脳みその入れ替え」が早くできる人はあける必要はないし、人によっては三日あけたほうがいいかも。
  \url{https://t.co/AU9KnSBYnB} 
\end{itemize}

〉〉〉 kk\_hironoのリツイート 〉〉〉

\begin{itemize}
\tightlist
\item
  RT
  kk\_hirono(刑事告発・非常上告_金沢地方検察庁御中)|kosuke\_shina(品川皓亮|法律入門書の著者|弁護士のキャリア支援)
  日時:2021-04-26 05:54/2021/04/23 07:30 URL:
  \url{https://twitter.com/kk\_hirono/status/1386423474978054145} 
  \url{https://twitter.com/kosuke\_shina/status/1385360263784779776} 
  \textgreater{}
  裁判に強い弁護士は「脳みその入れ替え」で書面の練度を上げるのが上手い。
  ①全力でクライアントの主張を書面に落とす ↓
  ②脳みそを「相手方」に入れ替えて、①の書面を徹底的に批判する ↓
  ③脳みそを再びクライアントに戻して、②に耐えうる書面にする
  ※①と②の間で一晩空けるのがポイント。
\end{itemize}

〉〉〉 kk\_hironoのリツイート 〉〉〉

\begin{itemize}
\item
  RT
  kk\_hirono(刑事告発・非常上告_金沢地方検察庁御中)|s\_hirono(非常上告-最高検察庁御中\_ツイッター)
  日時:2021-04-26 05:55/2021/04/25 22:42 URL:
  \url{https://twitter.com/kk\_hirono/status/1386423706113642497} 
  \url{https://twitter.com/s\_hirono/status/1386314574819852289} 
  \textgreater{}
  2021-04-25-200944\_モトケン@motoken\_tw·1時間これ、私が何度か言ってる「三手の読み」の話。一晩あけるかどうかは別にして。「脳みその入れ替え」が早くで.jpg
  \url{https://t.co/VDjnOpLHH4} 
\item
  2021年04月26日05時57分の登録:
  REGEXP:''三手の読み''/モトケン(@motoken\_tw)の検索(2019-12-12〜2021-04-25/2021年04月26日05時57分の記録2件)
  \url{https://kk2020-09.blogspot.com/2021/04/regexpmotokentw2019-12-122021-04.html} 
\item
  2021年04月26日05時57分の登録:
  \モトケン @motoken\_tw\これ、私が何度か言ってる「三手の読み」の話。一晩あけるかどうかは別にして。「脳みその入れ替え」が早くできる人はあける必要はないし、人に
  \url{https://kk2020-09.blogspot.com/2021/04/motokentw\_26.html} 
\item
  2021年04月26日05時59分の登録:
  REGEXP:''三手''/モトケン(@motoken\_tw)の検索(2019-12-12〜2021-04-25/2021年04月26日05時59分の記録2件)
  \url{https://kk2020-09.blogspot.com/2021/04/regexpmotokentw2019-12-122021-04\_26.html} 
\item
  〈〈〈 2021/04/26 06:01:13 Linux Emacs: 〈〈〈
\end{itemize}

\hypertarget{ux30deux30b9ux30b3ux30dfux306eux8a18ux4e8bux306eux4fe1ux983cux6027ux304cux4f4eux4e0bux3057ux4fe1ux983cux56deux5fa9ux306aux3069ux5922ux306eux307eux305fux5922ux3068ux3044ux3046ux4e2dux65e5ux65b0ux805eux4ecaux4e95ux667aux6587ux8a18ux8005ux306bux5411ux3051ux305fux30e2ux30c8ux30b1ux30f3ux3053ux3068ux77e2ux90e8ux5584ux6717ux5f01ux8b77ux58ebux4eacux90fdux5f01ux8b77ux58ebux4f1aux306eux30c4ux30a4ux30fcux30c8}{%
\paragraph{マスコミの記事の信頼性が低下し信頼回復など夢のまた夢,という中日新聞今井智文記者に向けたモトケンこと矢部善朗弁護士(京都弁護士会)のツイート}\label{ux30deux30b9ux30b3ux30dfux306eux8a18ux4e8bux306eux4fe1ux983cux6027ux304cux4f4eux4e0bux3057ux4fe1ux983cux56deux5fa9ux306aux3069ux5922ux306eux307eux305fux5922ux3068ux3044ux3046ux4e2dux65e5ux65b0ux805eux4ecaux4e95ux667aux6587ux8a18ux8005ux306bux5411ux3051ux305fux30e2ux30c8ux30b1ux30f3ux3053ux3068ux77e2ux90e8ux5584ux6717ux5f01ux8b77ux58ebux4eacux90fdux5f01ux8b77ux58ebux4f1aux306eux30c4ux30a4ux30fcux30c8}}

\begin{itemize}
\tightlist
\item
  〉〉〉 Linux Emacs: 2021/05/17 07:20:28 〉〉〉
\end{itemize}

:CATEGORIES: @kanazawabengosi \#金沢弁護士会 @JFBAsns
\#日本弁護士連合会(日弁連) \#法務省 @MOJ\_HOUMU \#マスコミ
\#モトケンこと矢部善朗弁護士(京都弁護士会)

 6時半過ぎに目が覚めました。LEDの卓上スタンドの眩しい電気をつけたまま寝ていたのですが,熟睡し子供が沢山集まった八坂神社の拝殿の夢をみていました。データベースの記録をみると日付が変わる前には寝ていたようですが,久しぶりに長い時間,熟睡していたようです。

\begin{itemize}
\tightlist
\item
  2021年05月16日22時12分の登録:
  REGEXP:''能登町''/データベース登録済みツイートの検索:2021-04-30〜2021-05-15/2021年05月16日22時12分の記録:ユーザ・投稿:14/154件
  \url{https://kk2020-09.blogspot.com/2021/05/regexp2021-04-302021-05\_16.html} 
\item
  2021年05月16日23時29分の登録:
  \うの字を名乗る?物 @un\_co\_the2nd\刑弁畑の人らは日々警察とか検察とバトってるからなのか戦闘意欲のレベルが違うような。ま、非弁の害悪の度合いも刑事だと比喩で
  \url{https://kk2020-09.blogspot.com/2021/05/uncothe2nd\_16.html} 
\item
  2021年05月16日23時30分の登録:
  \ねこパ〜スタ @abcabcabc999666\非弁が刑事に口出しするのはやめといたほうがいい刑弁畑の人は戦闘民族みたいなのが多いから喧嘩になると大変や
  \url{https://kk2020-09.blogspot.com/2021/05/abcabcabc999666.html} 
\item
  2021年05月16日23時33分の登録:
  REGEXP:''刑弁畑''/データベース登録済みツイートの検索:2018-02-26〜2021-05-16/2021年05月16日23時32分の記録:ユーザ・投稿:14/18件
  \url{https://kk2020-09.blogspot.com/2021/05/regexp2018-02-262021-05.html} 
\item
  2021年05月16日23時36分の登録:
  REGEXP:''青木理''/データベース登録済みツイートの検索:2011-05-29〜2021-05-16/2021年05月16日23時34分の記録:ユーザ・投稿:38/108件
  \url{https://kk2020-09.blogspot.com/2021/05/regexp2011-05-292021-05.html} 
\item
  2021年05月17日06時46分の登録:
  2021-05-16の投稿一覧\検察・石川県警察宛記録資料\奉納\危険生物・弁護士脳汚染除去装置\金沢地方検察庁御中:40件
  \url{https://kk2020-09.blogspot.com/2021/05/2021-05-1640.html} 
\item
  2021年05月17日06時46分の登録:
  ツイートの記録資料:\法務検察・石川県警察宛\/小倉秀夫(@chosakukenho)/''2021年05月16日'':11件
  \url{https://kk2020-09.blogspot.com/2021/05/chosakukenho2021051611.html} 
\item
  2021年05月17日06時46分の登録:
  ツイートの記録資料:\法務検察・石川県警察宛\/モトケン(@motoken\_tw)/''2021年05月16日'':12件
  \url{https://kk2020-09.blogspot.com/2021/05/motokentw2021051612.html} 
\item
  2021年05月17日06時55分の登録:
  \小倉秀夫 @chosakukenho\まず、そう言う法律を制定した場合に憲法違反となるかどうかを検討するのが先ですね。
  \url{https://kk2020-09.blogspot.com/2021/05/chosakukenho\_21.html} 
\item
  2021年05月17日07時00分の登録:
  \嶋﨑量(弁護士) @shima\_chikara\子どもたちには、そんな当たり前の人権教育こそが必要で、政治権力の誤った「道徳」を教育現場に押しつけさせてはなりません。
  \url{https://kk2020-09.blogspot.com/2021/05/shimachikara\_17.html} 
\end{itemize}

 投稿一覧とモトケンこと矢部善朗弁護士(京都弁護士会),小倉秀夫弁護士のツイートの記録資料は毎日,日付が変わった頃に実行するday-ago-script.shという自作コマンドで作成しています。深澤諭史弁護士のTwitterは更新がないままなので記事が作成されません。

 そういえば寝る少し前にジャーナリストの青木理氏の名前を見かけていたのですが,ずいぶん久しぶりに見かけた名前で,すっかり忘れていたことに気が付きました。

 どうも昨夜辺りからGoogle
Chromeで私のまとめ記事の埋め込みツイートが表示されなくなっているのですが,「【宮台真司x青木理】字幕作成中:学力の低い首相が国を亡ぼす
\url{https://t.co/bvJsfcqYFU}  @YouTubeより」というツイートも見当たりました。

 青木理氏は週刊誌でも検察を痛烈に批判していた時期があり,PC遠隔操作事件のときも警察や検察を批判していたような記憶です。一方で,被疑者だった人物を,ネコ男などと妖怪のように週刊誌の記事にしていたのも印象的でした。茂平食堂で見た週刊誌だったと思います。

 告訴状の作成は次のエントリーまで進んでいます。

\begin{itemize}
\tightlist
\item
  1373:2021-05-16\_18:33:29 \#告発状 \#\#\#\#
  舞鶴事件の京田辺市の細川治弁護士,被害者安藤文さん家族と被告発人長谷川紘之弁護士の関係性
  \url{https://hirono-hideki.hatenadiary.jp/entry/2021/05/16/183326} 
\end{itemize}

 次のツイートは中断した時のものですが,次の見出しだけ作成しています。

\begin{itemize}
\tightlist
\item
  TW kk\_hirono(刑事告発・非常上告_金沢地方検察庁御中) 日時:
  2021-05-16 18:36 URL:
  \url{https://twitter.com/kk\_hirono/status/1393862919067029504} 
  \textgreater{} \#\#\#\#
  被害者安藤文さん家族と被告発人長谷川紘之弁護士の関係性,舞鶴事件で京田辺市の細川治弁護士から思うこと
\end{itemize}

 宇出津新港に買い物に出掛けるつもりだったのですが,外を見ると小雨が降っていて薄暗く,バイクの運転が危ないということもあるので,行くのをやめにして,冷蔵庫にあった大根で久しぶりの味噌汁を作りました。

 前の日の土曜日は,夕方,Aコープ能都店に買い物に行ったのですが,惣菜の弁当類や揚げ物のコーナーが全滅で,明日の日曜日が休みということに気がついたのですが,鮮魚コーナーに行くと,立派な赤カレイが半額で売っていました。消費期限が翌日なので半額ですが,休み前は多いことです。

〉〉〉 kk\_hironoのリツイート 〉〉〉

\begin{itemize}
\tightlist
\item
  RT
  kk\_hirono(刑事告発・非常上告_金沢地方検察庁御中)|s\_hirono(非常上告-最高検察庁御中\_ツイッター)
  日時:2021-05-17 08:09/2021/05/17 08:08 URL:
  \url{https://twitter.com/kk\_hirono/status/1394067395115851779} 
  \url{https://twitter.com/s\_hirono/status/1394067244011773954} 
  \textgreater{}
  2021-05-15\_211501_Aコープ能都店の赤かれいのお腹から出てきたイワシ.jpg
  \url{https://t.co/mDHQM1HvTF} 
\end{itemize}

〉〉〉 kk\_hironoのリツイート 〉〉〉

\begin{itemize}
\tightlist
\item
  RT
  kk\_hirono(刑事告発・非常上告_金沢地方検察庁御中)|s\_hirono(非常上告-最高検察庁御中\_ツイッター)
  日時:2021-05-17 08:09/2021/05/17 08:08 URL:
  \url{https://twitter.com/kk\_hirono/status/1394067406839074818} 
  \url{https://twitter.com/s\_hirono/status/1394067229621161984} 
  \textgreater{}
  2021-05-15\_205051_赤かれい 100g当り88円 488g 429円 半額.jpg
  \url{https://t.co/5QKhapJB3W} 
\end{itemize}

〉〉〉 kk\_hironoのリツイート 〉〉〉

\begin{itemize}
\tightlist
\item
  RT
  kk\_hirono(刑事告発・非常上告_金沢地方検察庁御中)|s\_hirono(非常上告-最高検察庁御中\_ツイッター)
  日時:2021-05-17 08:09/2021/05/17 08:08 URL:
  \url{https://twitter.com/kk\_hirono/status/1394067423041449984} 
  \url{https://twitter.com/s\_hirono/status/1394067215213678594} 
  \textgreater{} 2021-05-15\_205036_赤かれい.jpg
  \url{https://t.co/QGjlqHETgk} 
\end{itemize}

〉〉〉 kk\_hironoのリツイート 〉〉〉

\begin{itemize}
\tightlist
\item
  RT
  kk\_hirono(刑事告発・非常上告_金沢地方検察庁御中)|s\_hirono(非常上告-最高検察庁御中\_ツイッター)
  日時:2021-05-17 08:09/2021/05/17 08:08 URL:
  \url{https://twitter.com/kk\_hirono/status/1394067443643916288} 
  \url{https://twitter.com/s\_hirono/status/1394067200793747459} 
  \textgreater{} 2021-05-15\_190143_Aコープ能都店の前,まだ薄明るい.jpg
  \url{https://t.co/HHm2YV0MMs} 
\end{itemize}

 母親がたまにこの赤かれいの煮付けを作っていたのですが,玉ねぎと豆腐が入っていました。

 カレイというのは海の底,砂に潜って上を通る獲物を狙っているというイメージなのですが,お腹からけっこうな大きさのイワシが入ってきました。イワシは大きな群れの回遊魚で,海底の底付近をウロウロするイメージがなかったので,これは意外なことでした。

 どんたく宇出津店でh他の種類のカレイで安く大量に袋に入っているものもあるのですが,赤カレイ以外のカレイは自分で買って調理をしたことがありません。赤カレイは内臓が他の魚とは違っていて,消化が早いのかわからないですが,すぐに内蔵が腐って痛みそうな魚だと思いました。

\begin{itemize}
\tightlist
\item
  赤カレイとは?その生態や美味しく食べられる人気レシピ5選もご紹介!
  \textbar{} 暮らし〜の \url{https://t.co/MIzhsWLr35} 
\end{itemize}

 赤カレイに旬があると書いてありましたが,聞いたことはなく,滅多に食べる魚ではありませんでした。調理済みのものもいつものように売っていて,高い値段ではないのですが,半額以外の魚を買うことはあまりないという生活です。

 今井智文記者の名前を見たのも久しぶりのことでしたが,モトケンこと矢部善朗弁護士(京都弁護士会)のタイムラインで昨日のことだったと思います。刑事裁判の被告人をマスコミが被告と呼称することで,他の法クラにも批判を受け論戦のようになっていたことがありました。

 昨日もさほど気に留めないでいたのですが,今朝,モトケンこと矢部善朗弁護士(京都弁護士会)のタイムラインでみた次のツイートは,さすがにどうかと思いました。正確には昨日のツイートらしく16日午前10時58分となっています。

\begin{itemize}
\tightlist
\item
  TW motoken\_tw(モトケン) 日時: 2021/05/16 10:58:33 URL:
  \url{https://twitter.com/motoken\_tw/status/1393747668506071045} 
  \textgreater{}
  根本的には、マスコミの記事の信頼性が低下して、読者の中に第一次資料に当たりたいというニーズが増えてきているということだと思う。\\
  \textgreater{}
  そういうニーズに反論しているだけだと、信頼回復なんか夢のまた夢。
  \url{https://t.co/FcUOFe3oVl} 
\end{itemize}

 次の今井智文記者のツイートを引用していますが,その今井智文記者のツイートもモトケンこと矢部善朗弁護士(京都弁護士会)のツイートに対する返信となっています。

〉〉〉 kk\_hironoのリツイート 〉〉〉

\begin{itemize}
\tightlist
\item
  RT
  kk\_hirono(刑事告発・非常上告_金沢地方検察庁御中)|imaicn21(今井
  智文) 日時:2021-05-17 08:34/2021/05/16 10:45 URL:
  \url{https://twitter.com/kk\_hirono/status/1394073884937232384} 
  \url{https://twitter.com/imaicn21/status/1393744443300216836} 
  \textgreater{} @motoken\_tw
  「省略形ではなく正式名称を使わなければいけない」という理論が、多くの文字数制限の厳しい媒体で適用されないのは、矛盾だと思います。
\end{itemize}

 モトケンこと矢部善朗弁護士(京都弁護士会)は議論だと思っているのでしょうが,「マスコミの記事の信頼性が低下」というのは具体的にどんな記事あるいはニュースなのかという疑問です。このような相手を畳み掛ける批判はモトケンこと矢部善朗弁護士(京都弁護士会)のお家芸でもあります。

 サスペンスドラマでは京都が舞台になることが多く,事件の関係者に家元というのが出てくることもよくありました。伝統芸能のようなものを感じさせるモトケンこと矢部善朗弁護士(京都弁護士会)ですが,その批判の矛先は,検察,裁判官にも向けられてきました。

\begin{itemize}
\tightlist
\item
  TW motoken\_tw(モトケン) 日時: 2021/05/16 10:58:33 URL:
  \url{https://twitter.com/motoken\_tw/status/1393747668506071045} 
  \textgreater{}
  根本的には、マスコミの記事の信頼性が低下して、読者の中に第一次資料に当たりたいというニーズが増えてきているということだと思う。\\
  \textgreater{}
  そういうニーズに反論しているだけだと、信頼回復なんか夢のまた夢。
  \url{https://t.co/FcUOFe3oVl} 
\end{itemize}

 上記の件のモトケンこと矢部善朗弁護士(京都弁護士会)のツイートを再掲しましたが,読めば読むほど味わい深い内容となっています。老舗の味というのか漬物屋の伝統製法のようでもあります。 

 「信頼回復なんか夢のまた夢。」という短文の締めくくりの部分ですが,夢の旅というSFの世界を連想させるもので,現実のこの社会に,それほどマスコミの信頼を揺るがす報道があったのかと考える。個人的には森友学園などがその部類ですが,そういう評価は一般化されていないはずです。

 それほど見かける機会はないので知らない人が多いかもしれませんが,東京新聞は中日新聞の子会社らしく,安倍政権批判の斬込み隊というイメージが個人的にあります。しかし,そのような批判をモトケンこと矢部善朗弁護士(京都弁護士会)のツイートで見た覚えはありません。

 理由はわかりませんが,Twitterの通知におすすめツイートなどとして望月衣塑子記者のツイートが表示されることがあり,たまに読むことがあります。

 「読者の中に第一次資料に当たりたい」というのもモトケンこと矢部善朗弁護士(京都弁護士会)の主観のようですが,第一次資料として思い当たるのは,弁護士が公開する資料ぐらいのものです。個人の告発が資料として評価されたのは,弁護士が絡んで記者会見をした案件ぐらいです。

 具体的には赤木ファイルとして,数日前,ちらほらと見かけ,土曜日の夕方に銭湯を出るタイミングでも報道番組の次回予告にそのような話が聞こえました。早い段階で弁護士2人の活動や記者会見が多く,一人の弁護士の名前は松丸正弁護士ではなかったかと思います。

\begin{itemize}
\tightlist
\item
  2021年05月17日08時56分の登録:
  「松丸正」を@hirono\_hideki @kk\_hirono @s\_hironoで検索 34件の該当 2021-05-17\_08:56の記録
  \url{https://kk2020-09.blogspot.com/2021/05/hironohidekikkhironoshirono342021-05.html} 
\end{itemize}

2020-03-19 19:17:16
``「内閣吹っ飛ぶ」森友文書改ざんで職員 遺書は震える字:朝日新聞デジタル
\url{https://www.asahi.com/articles/ASN3L6HX4N3LPTIL01F.html} 
提訴後の会見で、人事院から開示されたほとんどが黒塗りにされた赤木俊夫さんに関する資料を見せる代理人の松丸正弁護士(左)と生越照幸弁護士=2020年3月18日午後、大阪市内、小川智撮影''
\url{https://twitter.com/hirono\_hideki/status/1240583107121573889} 

 個人的な印象というか感想ですが,松丸正弁護士がロケットを打ち上げ,衛星が軌道をまわり始めたという弁護士鉄道の千夜一夜の1つで星巡りのような物語です。

 twilog-serch-post
で望月衣塑子記者についてもまとめ記事の作成をしましたが,99件となっていました。最近はツイートをリンクで開いても軽く本文に目を通す程度だったので,リツイートの数など見ていないですが,以前はかなり多くの数のリツイートなど大きな反響があったようです。

\begin{itemize}
\tightlist
\item
  2021年05月17日09時02分の登録:
  「望月衣塑子」を@hirono\_hideki @kk\_hirono @s\_hironoで検索 99件の該当 2021-05-17\_09:02の記録
  \url{https://kk2020-09.blogspot.com/2021/05/hironohidekikkhironoshirono992021-05.html} 
\end{itemize}

\begin{quote}
《引用の始まり》
\end{quote}

\begin{quote}
望月衣塑子@ISOKO\_MOCHIZUKI新聞記者。千葉、埼玉など各県警、東京地検特捜部、社会部遊軍でもりかけ疑惑、セクハラ問題、武器輸出、軍学共同等を取材。著書に「武器輸出と日本企業」「新聞記者」「独裁者」「『安倍晋三』大研究」「新聞と権力の大問題」「同調圧力」ツィートはあくまでも個人の見解です2010年3月からTwitterを利用しています1,378
フォロー中23.5万 フォロワー
\end{quote}

\begin{quote}
《引用の終わり》
\end{quote}

\begin{itemize}
\tightlist
\item
  望月衣塑子さん (@ISOKO\_MOCHIZUKI) /
  Twitter \url{https://twitter.com/ISOKO\_MOCHIZUKIn} 
\end{itemize}

 プロフィールにずいぶん沢山の著書が紹介されていました。これまで著書を出しているというイメージは全くなかったのですが,よく見るとプロフィールに東京新聞はありません。赤木ファイルの問題も最近は名前を見かけないですが,NHKを退職したという男性記者の奮闘があったようです。

〉〉〉 kk\_hironoのリツイート 〉〉〉

\begin{itemize}
\tightlist
\item
  RT
  kk\_hirono(刑事告発・非常上告_金沢地方検察庁御中)|ISOKO\_MOCHIZUKI(望月衣塑子)
  日時:2021-05-17 09:12/2020/09/11 09:20 URL:
  \url{https://twitter.com/kk\_hirono/status/1394083326059708419} 
  \url{https://twitter.com/ISOKO\_MOCHIZUKI/status/1304213128347938816} 
  \textgreater{} 共著の新書2冊でました。 \#田原総一朗 さん「
  \#嫌われるジャーナリスト 」(SB新書) \#佐高信 さん「
  \#なぜ日本のジャーナリズムは崩壊したのか 」(講談社)
  政権のおごり、ゆがみを生んだ安倍・菅 \#官邸
  によるメディア支配は次の政権でも続くのか。安倍政権の検証は不可欠です。ぜひお読みください
  \url{https://t.co/QVQcVJ47QA} 
\end{itemize}

 ハッシュタグにもなっているようですが,「 \#嫌われるジャーナリスト
」(SB新書)というのは,面白そうなタイトルで,書店でふらっと買って帰る人も多そうなキャッチコピーだと思いました。\#田原総一朗 さんも一時期,検察批判をしていたように思います。

 同じ固定ツイートで紹介されているもう1冊の,「
\#なぜ日本のジャーナリズムは崩壊したのか
」(講談社)という本もインパクトが大きいですが,表紙に「権力が隠し,メディアが伝えない この国の「中枢」の真実!」とあります。

 わかりやすく扱いやすいワード,「赤木ファイル」でまとめ記事の作成を始めました。確か自殺した赤木さんは神戸市だったと思いますが,昨日,集中的に取り上げ記録にした舞鶴女子高校生殺害事件との比較においても参考になるところで,「弁護士」という要素が大きいといえます。

 2021年05月17日09時21分の実行記録:8500で処理を終了
twitterAPI-search-lawList-mydql-add.rb ``赤木ファイル''
ツイート数:6/2416 リツイート数:5/2416 トータル:8500 ¥\n
``赤木ファイル''の該当: hirono\_hideki 0/0件 kk\_hirono 2/0件
s\_hirono 0/0件

 松丸正弁護士と似た弁護士に川人博弁護士がいるのですが,過労自殺などで有名,第一人者やエキスパートのような弁護士になるかと思います。これまでにも指摘をしてきましたが,本件告発事件は,被害者安藤文さんを被害者とする殺人未遂事件で,労働災害という側面も大きなものになります。

 昨日,集中的に取り上げ記録にした舞鶴女子高校生殺害事件ですが,細川治弁護士のブログに,とても参考になることが書いてありました。マスコミと弁護士の関係性,決定権は裁判所にありますが,弁護士が与える社会への影響の大きさを垣間見る事例でした。

 本来,舞鶴女子高校生殺害事件のことは被告発人長谷川紘之弁護士と関連付ける方向で記述をしてきました。確定してしまった無罪判決は,この国の誰にもどうすることもできません。憲法で決まっているからですが,一事不再理や被告人の不利益となる再審請求が認められていないからです。

 まだ知ってから1年は経っていないと思いますが,台湾では被告人を有罪とする検察の再審請求が認められているそうです。同じ頃に,再審請求で台湾の再審の制度を高く評価し,紹介する弁護士か学者の発言も見かけていました。これもtwilog-serch-post
で作っておきます。

\begin{itemize}
\item
  2021年05月17日08時56分の登録:
  「松丸正」を@hirono\_hideki @kk\_hirono @s\_hironoで検索 34件の該当 2021-05-17\_08:56の記録
  \url{https://kk2020-09.blogspot.com/2021/05/hironohidekikkhironoshirono342021-05.html} 
\item
  2021年05月17日09時02分の登録:
  「望月衣塑子」を@hirono\_hideki @kk\_hirono @s\_hironoで検索 99件の該当 2021-05-17\_09:02の記録
  \url{https://kk2020-09.blogspot.com/2021/05/hironohidekikkhironoshirono992021-05.html} 
\item
  2021年05月17日09時25分の登録:
  「川人博」を@hirono\_hideki @kk\_hirono @s\_hironoで検索 101件の該当 2021-05-17\_09:25の記録
  \url{https://kk2020-09.blogspot.com/2021/05/hironohidekikkhironoshirono1012021-05.html} 
\item
  2021年05月17日09時25分の登録:
  REGEXP:''赤木ファイル''/データベース登録済みツイートの検索:2021-05-01〜2021-05-17/2021年05月17日09時23分の記録:ユーザ・投稿:34/61件
  \url{https://kk2020-09.blogspot.com/2021/05/regexp2021-05-012021-05\_17.html} 
\item
  2021年05月17日09時36分の登録:
  「台湾」を@hirono\_hideki @kk\_hirono @s\_hironoで検索 126件の該当 2021-05-17\_09:36の記録
  \url{https://kk2020-09.blogspot.com/2021/05/hironohidekikkhironoshirono1262021-05.html} 
\item
  2021年05月17日09時37分の登録:
  REGEXP:''台湾.*再審''/データベース登録済みツイートの検索:2019-05-23〜2021-05-17/2021年05月17日09時37分の記録:ユーザ・投稿:16/19件
  \url{https://kk2020-09.blogspot.com/2021/05/regexp2019-05-232021-05.html} 
\item
  (01/19) TW kidkaido(海渡雄一) 日時: 2019-05-23 19:19:00 +0900
  URL:
  \url{https://twitter.com/kidkaido/status/1131504993557630976\textgreater} {}
  台湾では、刑事事件について再審請求を求める弁護人は公判に提出されなかった証拠を含めて、すべての証拠の閲覧ができるそうです。日本では、裁判官が勧告しなければ、検察庁は何も開示しないのと大違い。\textgreater{}
  台湾は、知らない間に人権先進国になって・・・ \url{https://t.co/wiWnLRXXy1} 
\item
  (03/19) TW kamo629782(かもん弓(鴨志田 祐美)) 日時: 2020-01-12
  07:21:00 +0900 URL:
  \url{https://twitter.com/kamo629782/status/1216123060538920960\textgreater} {}
  私は昨年11月にかの地で江恵民検事総長と面談し、その直後に台湾は再審法改正を成し遂げました。\textgreater{}
  刑事司法改革の流れも、ますます確固たるものになっていくでしょう。\textgreater\textgreater{}
  私も、もっと台湾との絆を深めたい。 \url{https://t.co/JVG3QZ8KoJ} 
\end{itemize}

 リツイートされているツイートですが,岡口基一裁判官のFacebook投稿をひたすら転載しているというアカウントのツイートのようです。ツイートの記事の要約と写真があるのですが,その写真が大津地裁の大西直樹裁判長です。

\begin{itemize}
\tightlist
\item
  TW okaguchikii(岡ロ基ー) 日時: 2020/05/05 02:54:21 URL:
  \url{https://twitter.com/okaguchikii/status/1257367980490600455} 
  \textgreater{} どんどん、台湾や韓国に追い越されていくニッポン\\
  \textgreater{}\\
  \textgreater{}
  「日本には台湾や韓国のように、捜査機関を検証する公的な独立機関はない。日本の裁判所は、再審無罪となった事件の誤判を検証し、結果を公表したことは、過去に一度もない。」\\
  \textgreater{} \url{https://t.co/IoNK6LBLJN} 
\end{itemize}

 2,3日前になるのか久しぶりに岡口基一裁判官に関する情報をTwitterで検索したのですが,数は少なく関心自体が薄れているようでした。国がどこかのタイミングで,岡口基一裁判官の弾劾裁判の開始をニュースとしてぶつけてくる可能性というのは,ずっと前から想定しています。

 本来,岡口基一裁判官の弾劾裁判の開始か不開始を決める重要な判断が予定されていたそうですが,それが大崎事件の最高裁決定とほぼ同じ日で,直前に取り消しか延期となっていました。手続きの名前は忘れましたが,国会議員が集まって協議し決めるという話でした。

\begin{itemize}
\tightlist
\item
  日本は冤罪対策「後進国」、調査結果を明かさない捜査機関  冤罪生み出した「黒い正義」~湖東記念病院再審から考える|社会|地域のニュース|京都新聞
  \url{https://t.co/WXRGAYmQWd}  2020年4月20日 11:00
\end{itemize}

 リンクを開くまで気が付かなかったのですが,京都新聞の記事でした。おまけにバーナー広告のような画像で,「冤罪を生み出した「黒い正義」 〜湖東記念病院再審から考える〜」とあります。前に見ていれば記憶に残っていると思うのですが,全く心当たりがありません。

 モトケンこと矢部善朗弁護士(京都弁護士会)のマスコミ批判の延長でここまで記述を続けてきたのですが,この続きは別のかたちで取り上げていきたいと思います。

 締めくくりに,データベースに登録済みのモトケンこと矢部善朗弁護士(京都弁護士会)のツイートから「再審」を含むものをキーワードにまとめ記事を作成し,ご紹介をしておきたいと思います。

\begin{itemize}
\item
  2021年05月17日09時58分の登録:
  REGEXP:''再審''/モトケン(@motoken\_tw)の検索(2010-04-17〜2020-11-06/2021年05月17日09時58分の記録51件)
  \url{https://kk2020-09.blogspot.com/2021/05/regexpmotokentw2010-04-172020-11.html} 
\item
  (49/51) TW motoken\_tw(モトケン) 日時:2020-05-16 01:34:00 +0900
  URL:
  \url{https://twitter.com/motoken\_tw/status/1261334211815862272\textgreater} {}
  @Jichael\_MacSon
  裁判において、冤罪(誤判)発生は不可避。\textgreater{}
  だから再審規定があるんです。\textgreater{}
  再審規定によっても完全に防ぐことはできませんけどね。\textgreater\textgreater{}
  本来、検察には証拠を捏造してまで起訴する動機はないんです。検察は成果主・・・
  \url{https://t.co/AnBqaPcDvq} 
\item
  (50/51) TW motoken\_tw(モトケン) 日時:2020-05-17 00:02:00 +0900
  URL:
  \url{https://twitter.com/motoken\_tw/status/1261673289304309760\textgreater} {}
  @naitohayato334 裁判所に言ってください。\textgreater{}
  裁判所が無罪にすれば再審にはならないのだから。
\item
  (51/51) TW motoken\_tw(モトケン) 日時:2020-11-06 08:13:00 +0900
  URL:
  \url{https://twitter.com/motoken\_tw/status/1324489957113651200\textgreater} {}
  司法がつまり裁判官が、常に正しい判断をするという保証は全くない。\textgreater{}
  だから三審制が取られているし再審手続があるし、冤罪事件が何件も発覚している。\textgreater{}
  では、司法不信を公言する医療界は常に正しい判断をするのか?\textgreater{}
  医師は絶対誤診をしないのか?手術でミスをすることは絶対ないのか?
\end{itemize}

 たぶん,前回同じまとめ記事を作成してから追加がないのではと思うのですが,再審を含むモトケンこと矢部善朗弁護士(京都弁護士会)のツイートは,昨年2020年11月6日を最後に,出現がありません。

 昨日発見し修正した理由で,ツイートの全文が取得されていませんが,残りの部分は「検察は成果主義ではないから。しかし、警察と特捜は成果主義です。」と,モトケンこと矢部善朗弁護士(京都弁護士会)らしいことが書いてあります。

 とりあえず,警察や検察,ときに裁判所を批判し悪者にしておけば,客が寄ってきて金儲けが出来るという,弁護士の火事場泥棒的,盗人泥棒根性が,ご本尊のように浮かび妖しい光を放つ,象徴的なモトケンこと矢部善朗弁護士(京都弁護士会)のツイートの1つです。

 「湖東」でページ内検索をすると1件だけ該当のツイートがありました。文章の長いツイートで,最後にURLがありますが,kyoto-np.co.jpとあるのは,京都新聞のドメインかもしれません。

※ @kk\_hironoのアカウントがブロックされ,リツイートに失敗したツイート

\begin{itemize}
\item
  TW motoken\_tw(モトケン) 日時:2020/03/31 11:20:44 URL:
  \url{https://twitter.com/motoken\_tw/status/1244811838534979585} 
  \textgreater{}
  この事件は自白がなければ起訴できなかった事件。そういう事件であれば尚更に自白の任意性と信用性が吟味されなければならない。そして、この人に虚偽自白をさせた刑事は本当にクソ。詳細はググってください。>再審で無罪 殺人罪で服役した女性の名誉回復 湖東記念病院
  \url{https://t.co/qDN19kSkSV} 
\item
  お探しのページが見つかりません\textbar 京都新聞
  \url{https://t.co/OTWddA4AlW}  ¥\n お探しのページが見つかりませんでした ¥\n
  お探しのページは削除されたか、URLが変更されています。 ¥\n ¥\n
  トップページよりご確認ください。
\end{itemize}

 京都新聞のサイトで,この「お探しのページが見つかりませんでした」「お探しのページは削除されたか、URLが変更されています。」というメッセージを見たのは,初めてになるのではと思います。

 幸い,記事の見出しの一部がモトケンこと矢部善朗弁護士(京都弁護士会)のツイートにあるようなので,これで探し出せるものがあるかもしれません。なぜ京都新聞が記事を削除したのかも気になり,あるいはURLの変更で元の記事が見つかるかもしれません。

\begin{itemize}
\tightlist
\item
  再審で無罪 殺人罪で服役した女性の名誉回復 湖東記念病院 - Google 検索
  \url{https://t.co/9WJYyaNDIf} 
\end{itemize}

 「「人工呼吸器外し」再審で無罪 殺人罪で女性の名誉・・・」という見出しが上位に並んでいて,京都新聞のURLが見えるのですが,記事の見出しが一部変更され,別に投稿された記事かと思ったところ,これも記事が見つからないというメッセージが出ました。

 大崎事件の西日本新聞でも同じことがありましたが,何かの理由で記事を削除するのであれば,訂正の説明をするのが報道機関としての責務ではないのかという思いがあります。疑念や憶測が広がる可能性があり,それこそ信頼の低下になりそうです。

\begin{itemize}
\tightlist
\item
  【京都新聞】「人工呼吸器外し」再審で無罪 殺人罪で服役した女性の名誉回復 湖東記念病院
  \textbar{} ふらっと 人権情報ネットワーク \url{https://t.co/jXPfm2jymO} 
\end{itemize}

 内容は京都新聞の記事の引用だと思いますが,続きを読む↓京都新聞,という京都新聞の部分のリンクを開くと,先程から繰り返される記事が見つからないというメッセージがでました。京都新聞では「ページが」となっているようですが。

 「www.kyoto-np.co.jp/articles/-/203847」というhttpsのプロトコルを除いたURLに気がついたのですが,ハイフンがあるのが削除設定した記事の保管場所なのかもしれません。

 試しに「mkdir
-」とやると,そのままのディレクトリが作成できたのですが,「cd -
」とやっても「cd `-'」とやっても移動できません。「cd
-」とやってしまうと,前にいたディレクトリに移動することはわかりきっています。

 本来使えない半角記号でファイルを作成するとシングルクォートで囲まれます。スマホやデジカメの写真ファイルに多いのですが,半角スペースが含まれているとそうなります。不思議なことに-
はそのまま表示されています。

 試しに1つスマホで写真ファイルを作成したのですが,「`2021-05-17
10.42.42.jpg'」と端末のlsコマンドでは表示されています。

 もっともWebアプリでは,Webサーバーのディレクトリがファイルやデータの保管場所になっているとは限らず,データベースからデータの読み込みを伴うページ遷移にも使われるということは,実際にWebサーバーの開発をやっていて経験をしています。

 本来,Webページのドメインから下は,半角の/(スラッシュ)で区切られたディレクトリの階層になるのですが,半角のハイフンのみのディレクトリというのは,見たことも聞いたこともありませんでした。

 よく考えてみると半角のハイフン自体は特殊文字ではなく,エスケープの必要がなく,ファイル名やディレクトリ名に使えていたと思います。写真ファイルの年月日の区切りにも普通に使っていました。

\begin{itemize}
\tightlist
\item
  先頭にハイフンが付くファイルを削除できない - ITmedia エンタープライズ
  \url{https://t.co/LynYxPW8aV} 
  ファイル名の前に「--」を付加すれば,以降の指定はパラメータとして認識されなくなるのだ。
\end{itemize}

 面白い発見をしたと思ったのですが,「cd --
-」とやっても前のディレクトリに飛ばされてしまいました。

 「rm -rf --
-」とやるとディレクトリを削除することが出来ましたが,いくつか試したものの移動することは出来ませんでした。ファイルブラウザからだと普通に操作できると思います。やらなかったですが。

 よく見ると内容が表示されている,次の気にもURLに半角スラッシュで囲まれた半角ハイフンだけの記号がありました。

\begin{itemize}
\tightlist
\item
  日本は冤罪対策「後進国」、調査結果を明かさない捜査機関  冤罪生み出した「黒い正義」~湖東記念病院再審から考える|社会|地域のニュース|京都新聞
  \url{https://www.kyoto-np.co.jp/articles/-/210523} 
\end{itemize}

\begin{quote}
《引用の始まり》
\end{quote}

\begin{quote}
なぜ、裁判所は何度も誤判を積み重ねてしまったのか。再審判決で刑事司法の在り方を問うた大西裁判長も、裁判所自身の原因には言及しなかった。 冤罪を解き明かすことは、刑事司法に多くの教訓を生み出すはずだ。だが、再審無罪判決が確定した刑事事件のうち、捜査機関が自ら検証結果を公表したのは、足利事件や氷見事件などわずかにすぎない。裁判所が検証結果を公表した例は、いまだない。
\end{quote}

\begin{quote}
《引用の終わり》
\end{quote}

\begin{itemize}
\tightlist
\item
  日本は冤罪対策「後進国」、調査結果を明かさない捜査機関  冤罪生み出した「黒い正義」~湖東記念病院再審から考える|社会|地域のニュース|京都新聞(2/4) \url{https://www.kyoto-np.co.jp/articles/-/210523?page=2n} 
\end{itemize}

 上記に引用しましたが,「再審無罪判決が確定した刑事事件のうち、捜査機関が自ら検証結果を公表したのは、足利事件や氷見事件などわずかにすぎない。裁判所が検証結果を公表した例は、いまだない。」とあります。

 氷見事件は,お隣の県で,さらに近年知ったことですが同じ能登半島,長距離トラックの仕事でも何度か鮮魚を魚市場に積みに行ったことがあり,国道160号線は,平成3年の秋から,集中して,七尾市から関東方面に向かうのに通行するようになった国道です。

 そういうこともあり身近に感じる冤罪事件でした。柳原さんという名前だったと思いますが,強姦という性犯罪の罪で服役していたというのも私と同じ福井刑務所でした。そういうことば注目していた事件ですが,捜査機関が自ら検証結果を公表というのは記憶にないというか印象にありません。

\begin{quote}
《引用の始まり》
\end{quote}

\begin{quote}
■検察幹部が死刑囚の再審請求、無罪判決得る

 2016年3月、現職の検察幹部が、10年前に確定した拳銃殺人事件死刑囚の無罪を訴え、裁判所に再審請求した。警察が暴力で自白させ、検察も否認の姿勢に対し、罵倒していた。裁判所は約1年半後、再審で無罪を言い渡した。この検察幹部は18年、検察トップの検事総長に就任し、冤罪(えんざい)について考える民間団体の集会で「冤罪は裁判所、弁護士とも協力して取り組まないといけない」と力強く主張した。
\end{quote}

\begin{quote}
《引用の終わり》
\end{quote}

\begin{itemize}
\tightlist
\item
  日本は冤罪対策「後進国」、調査結果を明かさない捜査機関  冤罪生み出した「黒い正義」~湖東記念病院再審から考える|社会|地域のニュース|京都新聞(3/4) \url{https://www.kyoto-np.co.jp/articles/-/210523?page=3n} 
\end{itemize}

 3ページ目になりますが,「2016年3月、現職の検察幹部が、10年前に確定した拳銃殺人事件死刑囚の無罪を訴え、裁判所に再審請求した。」とありますが,全く記憶にない話です。唐突なので自分の頭がどうかしているのか,改ざんされたページを読んでいるのかと不安になってきました。

〉〉〉 kk\_hironoのリツイート 〉〉〉

\begin{itemize}
\tightlist
\item
  RT
  kk\_hirono(刑事告発・非常上告_金沢地方検察庁御中)|s\_hirono(非常上告-最高検察庁御中\_ツイッター)
  日時:2021-05-17 11:19/2021/05/17 11:17 URL:
  \url{https://twitter.com/kk\_hirono/status/1394115435541368840} 
  \url{https://twitter.com/s\_hirono/status/1394114838754848769} 
  \textgreater{}
  2021-05-17-110641\_日本は冤罪対策「後進国」、調査結果を明かさない捜査機関  冤罪生み出した「黒い正義」~湖東記念病院再審から考える|社会|地域のニュース|京都.jpg
  \url{https://t.co/sGLPSqMKWU} 
\end{itemize}

〉〉〉 kk\_hironoのリツイート 〉〉〉

\begin{itemize}
\tightlist
\item
  RT
  kk\_hirono(刑事告発・非常上告_金沢地方検察庁御中)|s\_hirono(非常上告-最高検察庁御中\_ツイッター)
  日時:2021-05-17 11:20/2021/05/17 11:17 URL:
  \url{https://twitter.com/kk\_hirono/status/1394115452956078081} 
  \url{https://twitter.com/s\_hirono/status/1394114911479894026} 
  \textgreater{}
  2021-05-17-111721\_2016年3月、現職の検察幹部が、10年前に確定した拳銃殺人事件死刑囚の無罪を訴え、裁判所に再審請求した。警察が暴力で自白させ、検察も否認の.jpg
  \url{https://t.co/USnLXmzVwP} 
\end{itemize}

 一通り記事を読み終えてから調べてみたいと思いますが,大きなニュースにならなかったのが不思議で,たまたま私が見逃していただけであれば,それもすごい確立をくぐり抜けた偶然になると思います。

 思わずため息が出て拍子抜けをしたのですが,「もちろん、日本の話ではない。しかし、かつて日本の統治下にあり、ほぼ同じ刑事訴訟法を持っていた台湾で実際にあったことだ。」と,ここで再び台湾が出てきました。

\begin{quote}
《引用の始まり》
\end{quote}

\begin{quote}
日本には台湾や韓国のように、捜査機関を検証する公的な独立機関はない。冤罪が確定しても、捜査機関が自ら検証報告を公開するのはごくわずかだ。裁判所に至っては、再審無罪となった事件の誤判を検証し、結果を公表したことは、過去に一度もない。

 ただ、イノセンス運動は、日本でも一歩を踏み出している。京都を拠点に「えん罪救済センター」が2016年に発足。受刑者らの依頼を受け、法学者や弁護士らが無償で冤罪の解明を進めている。支援は審査で決めるが、申し込みは現在、約330件に上る。

 副代表の笹倉香奈甲南大教授(刑事訴訟法)は「冤罪の救済、検証の面で、日本は世界的に見て遅れている。まずは個別の事件で雪冤(せつえん)を進めて共通の原因を見つけ出し、刑事司法の改革につなげたい」と力を込める。

       ◆ 「事件性すら証明されていない」。再審で無罪判決となった湖東病院の患者死亡。ないはずの「殺人事件」を生み出したのは、虚偽の自白を誘導し、強引な有罪立証を進めた警察と検察だった。16年にわたり無実の罪を着せた「正義」とは何か、問う。
\end{quote}

\begin{quote}
《引用の終わり》
\end{quote}

\begin{itemize}
\tightlist
\item
  日本は冤罪対策「後進国」、調査結果を明かさない捜査機関  冤罪生み出した「黒い正義」~湖東記念病院再審から考える|社会|地域のニュース|京都新聞(4/4) \url{https://www.kyoto-np.co.jp/articles/-/210523?page=4n} 
\end{itemize}

 最後に長めの引用をしましたが,ちょうどツイートできる文字数だったので,そのまま再捜査要請書_警察庁・石川県警察御中(@kk\_hirono)でツイートをしました。私は常にツイートできる文字数を意識して改行を入れるのですが,新聞社の記事ではとてもめずらしく感じました。

 台湾と韓国の再審の事例紹介があわせて出てきたことで,以前に読んでいる記事ではないかと思えてきたのですが,3ページ目の最初の当たりまでは全く違った印象で読み進めていました。ただ,バーナーの「冤罪を生み出した「黒い正義」」というのは,記憶にも印象にも残っていません。

\begin{itemize}
\tightlist
\item
  海上自衛隊員をわいせつ疑いで現行犯逮捕 喫煙所で下半身露出|社会|地域のニュース|京都新聞
  \url{https://t.co/Dr5P9jAynp} 
  舞鶴署によると、喫煙所にいた65歳女性が通報した。容疑者は、容疑を認めているという。
\end{itemize}

 京都新聞のサイトのホームで見つけた記事です。舞鶴ということで気になったのですが,容疑の内容云々というより,実名が出ていないことに着目しました。一昔前ならば,当たり前のように実名が出ていたように思うのですが,弁護士の活動の功績になるのか,よくみるようになった傾向です。

 記事は本日の10時46分で,昨夜午後8時半頃の通報,駆けつけた警察官による現行犯逮捕のようですが,弁護士が動いて実名報道を阻止させたという可能性のあるのかと,銀河鉄道999の場面を思い出すように想像しました。京都弁護士会ならではという地域の特殊性も感じます。

 京都弁護士会では,モトケンこと矢部善朗弁護士(京都弁護士会)なども完全放置なのでしょう。ひところはモトケンこと矢部善朗弁護士(京都弁護士会)を京都弁護士会に懲戒請求するという話もツイートで見かけることがありましたが,何事もなく済んだのかと考えています。

\begin{itemize}
\tightlist
\item
  困ったときは、弁護士に相談を!あなたのまちの京都弁護士会【相続/交通事故/借金/離婚/雇用関係】
  <特集PR>|PR|京都新聞 \url{https://t.co/cNd4VLtjkl} 
\end{itemize}

 京都弁護士会のホームページへのリンクかと思ったのですが,同じ京都新聞の記事で「<特集PR>」とあります。

 京都新聞のサイトというかホームページは,ページをマウスでスクロールするタイミングで,最上部にメニューが表示されたり,されなかったりするのですが,そのメニューに検索のボタンが見えたので,「遠山大輔」でやってみました。結果は2件です。

\begin{itemize}
\item
  検索結果|京都新聞 \url{https://t.co/nYusaNvnqb} 
\item
  弁護と被害者 遠山大輔 現代のことば|社会|地域のニュース|京都新聞
  \url{https://t.co/MqNx5GrXHF}  京都新聞IDへの会員登録・ログイン ¥\n
  続きを読むには会員登録やプランの利用申し込みが必要です。
\item
  「『死刑』という響き」遠山大輔
  現代のことば|社会|地域のニュース|京都新聞 \url{https://t.co/SfIxAuFMYO} 
  ¥\n 京都新聞IDへの会員登録・ログイン ¥\n
  続きを読むには会員登録やプランの利用申し込みが必要です。
\end{itemize}

 なぜ,京都新聞で「遠山大輔」という検索をしたかと言うと,犯罪史においても空前絶後の大事件となったはずの京アニ放火殺人事件で,2人の国選弁護人の一人となったのが遠山大輔弁護士だからです。しばらくして情報は一切見かけなくなり,辞任の可能性を含め調べてみました。

 京都新聞というのはその名にある通り,京都市あるいは京都府を代表する地元紙ということになると普通に思いますが,だんだんと京都弁護士会や京都の弁護士と,思想信条が一体化した一蓮托生ではないのかと思えてきました。批判的,チェック機能の要素がまるで見られず,垂れ流しのようです。

 あらためて「そういうニーズに反論しているだけだと、信頼回復なんか夢のまた夢。」というモトケンこと矢部善朗弁護士(京都弁護士会)のツイートですが,同じツイートの「読者の中に第一次資料に当たりたいというニーズ」を指しているのでしょう。

 ツイートのやりとりを遡っても「第一次資料」の意味が全く見えてこないのですが,省略形と正式名称の話で,どこから飛び出したのか不思議でなりません。埋め込みツイートの数の問題もあるので,もう一つエントリーを増やすかたちで対応します。

\begin{itemize}
\tightlist
\item
  〈〈〈 2021/05/17 12:15:54 Linux Emacs: 〈〈〈
\end{itemize}

\hypertarget{ux4e2dux65e5ux65b0ux805eux8a18ux8005ux4ecaux4e95ux667aux6587ux6c0fimaicn21ux3068ux306eux30c4ux30a4ux30fcux30c8ux306eux3084ux308aux3068ux308aux3067ux30e2ux30c8ux30b1ux30f3ux3053ux3068ux77e2ux90e8ux5584ux6717ux5f01ux8b77ux58ebux4eacux90fdux5f01ux8b77ux58ebux4f1aux306eux7b2cux4e00ux6b21ux8cc7ux6599ux306fux4f55ux3092ux610fux5473ux3059ux308bux306eux304bux3068ux3044ux3046ux8b0eux306bux8febux308b}{%
\paragraph{中日新聞記者,今井智文氏@imaicn21とのツイートのやりとりで,モトケンこと矢部善朗弁護士(京都弁護士会)の第一次資料は何を意味するのかという謎に迫る}\label{ux4e2dux65e5ux65b0ux805eux8a18ux8005ux4ecaux4e95ux667aux6587ux6c0fimaicn21ux3068ux306eux30c4ux30a4ux30fcux30c8ux306eux3084ux308aux3068ux308aux3067ux30e2ux30c8ux30b1ux30f3ux3053ux3068ux77e2ux90e8ux5584ux6717ux5f01ux8b77ux58ebux4eacux90fdux5f01ux8b77ux58ebux4f1aux306eux7b2cux4e00ux6b21ux8cc7ux6599ux306fux4f55ux3092ux610fux5473ux3059ux308bux306eux304bux3068ux3044ux3046ux8b0eux306bux8febux308b}}

\begin{itemize}
\tightlist
\item
  〉〉〉 Linux Emacs: 2021/05/17 12:19:01 〉〉〉
\end{itemize}

:CATEGORIES: @kanazawabengosi \#金沢弁護士会 @JFBAsns
日本弁護士連合会(日弁連) \#法務省 @MOJ\_HOUMU
\#モトケンこと矢部善朗弁護士(京都弁護士会) \#マスコミ

\begin{itemize}
\tightlist
\item
  1374:2021-05-17\_12:16:43 \#告発状 \#\#\#\#
  マスコミの記事の信頼性が低下し信頼回復など夢のまた夢,という中日新聞今井智文記者に向けたモトケンこと矢部善朗弁護士(京都弁護士会)のツイート
  \url{https://hirono-hideki.hatenadiary.jp/entry/2021/05/17/121639} 
\end{itemize}

 上記のエントリーの続きになります。モトケンこと矢部善朗弁護士(京都弁護士会)のツイートをブラウザで開いたままにしとどまっていると,とめどもなく「読者の中に第一次資料に当たりたいというニーズ」が何を指し示しているのかという謎があふれだしてきました。

 関連したツイートをまとめてリツイートして行きたいと思います。モトケンこと矢部善朗弁護士(京都弁護士会)のツイートは100%ブロックされていて,未来永劫に解除されることはないと判断しています。

〉〉〉 kk\_hironoのリツイート 〉〉〉

\begin{itemize}
\tightlist
\item
  RT
  kk\_hirono(刑事告発・非常上告_金沢地方検察庁御中)|imaicn21(今井
  智文) 日時:2021-05-17 12:24/2021/05/16 10:09 URL:
  \url{https://twitter.com/kk\_hirono/status/1394131740684259334} 
  \url{https://twitter.com/imaicn21/status/1393735217177788419} 
  \textgreater{}
  電光掲示板ニュースを提供しているのも新聞社です。「正式名称を書け」論は、メディア企業の出すニュースの、多くの媒体(例えば電光掲示板、ラジオ、テレビ、メール速報)などで実現不可能だし、「同じ話題のニュースなのに、電光掲示板では不要でウェブだと正式名称が必要」だと矛盾も抱えていると。
  \url{https://t.co/tOGSXcOrdu} 
\end{itemize}

〉〉〉 kk\_hironoのリツイート 〉〉〉

\begin{itemize}
\tightlist
\item
  RT
  kk\_hirono(刑事告発・非常上告_金沢地方検察庁御中)|yotajirosan(B54なぶさん)
  日時:2021-05-17 12:24/2021/05/15 22:06 URL:
  \url{https://twitter.com/kk\_hirono/status/1394131795679973378} 
  \url{https://twitter.com/yotajirosan/status/1393553406015336448} 
  \textgreater{}
  巨人と阪神の正式名称とレベルが全然違う話ではないか。そして新聞は速報性においては他のメディアと勝負にならず、正確性等で勝負しなければならないのに電光掲示板ニュースと比較してどうするのか。ただでさえ広告が減ってるんだから正式名称ぐらい書くスペースはあるはず。
  \url{https://t.co/T67AtbBlLq} 
\end{itemize}

〉〉〉 kk\_hironoのリツイート 〉〉〉

\begin{itemize}
\tightlist
\item
  RT
  kk\_hirono(刑事告発・非常上告_金沢地方検察庁御中)|imaicn21(今井
  智文) 日時:2021-05-17 12:25/2021/05/15 07:52 URL:
  \url{https://twitter.com/kk\_hirono/status/1394131951628423170} 
  \url{https://twitter.com/imaicn21/status/1393338495393943553} 
  \textgreater{}
  ・渋谷駅前の大型ビジョンに流れるのもニュース。それに正式名称とリンクって可能なの?
  ・「阪神2-1巨人 阪神4連勝」のニュースにリンクってあるの?
  ・「阪神」と「巨人」の正式名称ってほんとに必要?
  などをすっ飛ばして、「とにかく正式名称とリンクを貼るのが当然」という話がされている
\end{itemize}

〉〉〉 kk\_hironoのリツイート 〉〉〉

\begin{itemize}
\tightlist
\item
  RT
  kk\_hirono(刑事告発・非常上告_金沢地方検察庁御中)|imaicn21(今井
  智文) 日時:2021-05-17 12:25/2021/05/15 07:52 URL:
  \url{https://twitter.com/kk\_hirono/status/1394132005890138112} 
  \url{https://twitter.com/imaicn21/status/1393338494072758278} 
  \textgreater{}
  正式名称を表示してリンクを貼るのが当然という議論の多くが
  ・字数に制限がないウェブ媒体 ・正式名称やリンクが存在するニュース
  ・正式名称やリンクの必要性が高いニュース
  と、条件を限定しています。その条件を満たすのはニュース500本に1本ぐらいしかないように思われ、全然一般論ではないんです
  \url{https://t.co/AsLoA56mpD} 
\end{itemize}

〉〉〉 kk\_hironoのリツイート 〉〉〉

\begin{itemize}
\tightlist
\item
  RT
  kk\_hirono(刑事告発・非常上告_金沢地方検察庁御中)|itotakeru(伊藤たける|憲法マニアの弁護士@とやま)
  日時:2021-05-17 12:26/2021/05/14 23:23 URL:
  \url{https://twitter.com/kk\_hirono/status/1394132172202680322} 
  \url{https://twitter.com/itotakeru/status/1393210424942755845} 
  \textgreater{}
  正式名称を「表示しない」、あるいはリンクを「貼らない」理由は、どこにあるのでしょうか。
  記事の末尾に記載すれば、わかりやすさの観点と両立できると思うのですが。
  \url{https://t.co/1ItvNYpEIL} 
\end{itemize}

〉〉〉 kk\_hironoのリツイート 〉〉〉

\begin{itemize}
\tightlist
\item
  RT
  kk\_hirono(刑事告発・非常上告_金沢地方検察庁御中)|imaicn21(今井
  智文) 日時:2021-05-17 12:26/2021/05/14 21:42 URL:
  \url{https://twitter.com/kk\_hirono/status/1394132275965554688} 
  \url{https://twitter.com/imaicn21/status/1393184985788555265} 
  \textgreater{}
  正確に理解を深めるという記事での目標と、「正式名称とリンクを貼る」という理解促進の外部委託が、必ずしもリンクしないように感じます。
  \url{https://t.co/GQ3mTn796p} 
\end{itemize}

〉〉〉 kk\_hironoのリツイート 〉〉〉

\begin{itemize}
\tightlist
\item
  RT
  kk\_hirono(刑事告発・非常上告_金沢地方検察庁御中)|itotakeru(伊藤たける|憲法マニアの弁護士@とやま)
  日時:2021-05-17 12:27/2021/05/14 21:05 URL:
  \url{https://twitter.com/kk\_hirono/status/1394132367963430916} 
  \url{https://twitter.com/itotakeru/status/1393175601285009411} 
  \textgreater{}
  「正確に理解を深める」ならば、なおさら正式名称と、その報告書の議事録へのリンクを貼るべきではないでしょうか。
  \url{https://t.co/knI8ZFsQyQ} 
\end{itemize}

〉〉〉 kk\_hironoのリツイート 〉〉〉

\begin{itemize}
\tightlist
\item
  RT
  kk\_hirono(刑事告発・非常上告_金沢地方検察庁御中)|imaicn21(今井
  智文) 日時:2021-05-17 12:27/2021/05/13 07:56 URL:
  \url{https://twitter.com/kk\_hirono/status/1394132441812508674} 
  \url{https://twitter.com/imaicn21/status/1392614664106778626} 
  \textgreater{} 1つの記事だけで
  ・ニュースについて正確に理解を深めること(大多数の読者に必要なこと)
  と ・正しく情報源にたどり着けること(一部の読者に必要なこと)
  を同時に行おうというのが、無理な話です。メディアは読者からの質問に応じる窓口を設けており、そこで対応します。
  \url{https://t.co/IsKmapYyuk} 
\end{itemize}

〉〉〉 kk\_hironoのリツイート 〉〉〉

\begin{itemize}
\tightlist
\item
  RT
  kk\_hirono(刑事告発・非常上告_金沢地方検察庁御中)|gk1024(Kamei,
  Gentaro) 日時:2021-05-17 12:28/2021/05/13 07:31 URL:
  \url{https://twitter.com/kk\_hirono/status/1394132605889441792} 
  \url{https://twitter.com/gk1024/status/1392608478997323776} 
  \textgreater{}
  国民が正しく情報源にたどり着けることが「特殊な用途」であり、小中学生やおじいさんおばあさんはより詳しい情報に辿り着ける必要がないというなら、恐るべき勘違いだと思います。
  \url{https://t.co/bYtBK4dtOs} 
\end{itemize}

〉〉〉 kk\_hironoのリツイート 〉〉〉

\begin{itemize}
\tightlist
\item
  RT
  kk\_hirono(刑事告発・非常上告_金沢地方検察庁御中)|imaicn21(今井
  智文) 日時:2021-05-17 12:29/2021/05/12 22:34 URL:
  \url{https://twitter.com/kk\_hirono/status/1394132829882126341} 
  \url{https://twitter.com/imaicn21/status/1392473199611289601} 
  \textgreater{}
  正式名称を必要とするごくわずかな読者より、まずは分かりやすい説明を必要とする小中学生からおじいさんおばあさんまで幅広い読者のために書かれるのが新聞記事です。特殊な用途には向いていませんのでご容赦ください。
  \url{https://t.co/qcIiDWtpNf} 
\end{itemize}

〉〉〉 kk\_hironoのリツイート 〉〉〉

\begin{itemize}
\item
  RT
  kk\_hirono(刑事告発・非常上告_金沢地方検察庁御中)|gk1024(Kamei,
  Gentaro) 日時:2021-05-17 12:29/2021/05/12 20:21 URL:
  \url{https://twitter.com/kk\_hirono/status/1394132952557101059} 
  \url{https://twitter.com/gk1024/status/1392439665139929097} 
  \textgreater{}
  こういう記事は「有識者会議」と書くのではなく、正式名称を書いて頂きたい(定期)。厚労省の有識者会議は複数あるので、これでは当該会議を特定したことにならず、調べ物に不便なのですよ。
  歯科医による接種認める 新型コロナワクチン―厚労省有識者会議:時事ドットコム
  \url{https://t.co/t8ttHXQR74} 
\item
  〉〉〉 アカウント(@motoken\_tw)は,@kk\_hironoをブロックしています。リツイートできませんでした。
  〉〉〉 ¥\n ¥\n \url{https://t.co/7eOgk9Ge4J} 
\item
  〉〉〉 アカウント(@motoken\_tw)は,@kk\_hironoをブロックしています。リツイートできませんでした。
  〉〉〉 ¥\n ¥\n \url{https://t.co/DcDi1ZcLiH} 
\end{itemize}

※ @kk\_hironoのアカウントがブロックされ,リツイートに失敗したツイート

\begin{itemize}
\tightlist
\item
  TW motoken\_tw(モトケン) 日時:2021/05/16 10:58:33 URL:
  \url{https://twitter.com/motoken\_tw/status/1393747668506071045} 
  \textgreater{}
  根本的には、マスコミの記事の信頼性が低下して、読者の中に第一次資料に当たりたいというニーズが増えてきているということだと思う。\\
  \textgreater{}
  そういうニーズに反論しているだけだと、信頼回復なんか夢のまた夢。
  \url{https://t.co/FcUOFe3oVl} 
\end{itemize}

※ @kk\_hironoのアカウントがブロックされ,リツイートに失敗したツイート

\begin{itemize}
\tightlist
\item
  TW motoken\_tw(モトケン) 日時:2021/05/16 10:20:32 URL:
  \url{https://twitter.com/motoken\_tw/status/1393738099839029252} 
  \textgreater{} @imaicn21
  文字数制限の厳しい媒体と緩い媒体で表現を変える(厳しい媒体では省略形を使う)のが「矛盾」ですか?
\end{itemize}

 予想より多い紆余曲折を経て,最後にたどり着いたのが次の記事になります。中日新聞ではなくJIJI.COMとあるので,正式名称を忘れましたが,時事通信になるのか,そういうメディアではないかと思われます。

\begin{quote}
《引用の始まり》
\end{quote}

\begin{quote}
 新型コロナウイルスのワクチン接種について、厚生労働省の有識者会議は23日、集団接種に携わる医師らが不足する場合、歯科医師がワクチンを打つことを認める方針を承認した。
\end{quote}

\begin{quote}
《引用の終わり》
\end{quote}

\begin{itemize}
\tightlist
\item
  歯科医による接種認める 新型コロナワクチン―厚労省有識者会議:時事ドットコム
  \url{https://www.jiji.com/jc/article?k=2021042300861\&g=soc} 
\end{itemize}

 厚生労働省の有識者会議というのが情報不足として批判され,その実態というのが,モトケンこと矢部善朗弁護士(京都弁護士会)の指し示した第一次資料になるのかと考えました。これは確かに,厚労省有識者会議の中の部会の特定が,一般には難しそうに思えます。

 コロナ禍の問題は素人考えでも多岐にわたり,厚労省有識者会議というのも複数ありそうです。ここで思い出したのが,検察のありかた,なんとか会議です。郷原信郎弁護士とジャーナリストの江川紹子氏がメンバーになっていました。あれはわかりやすい名称でした。

\begin{itemize}
\tightlist
\item
  法務省:検察の在り方検討会議 \url{https://t.co/xK5Zv7g9u2}  ¥\n
  検察の再生に向けて【概要版】{[}PDF:156KB{]} ¥\n
  検察の再生に向けて{[}PDF:573KB{]}
\end{itemize}

 今まで考えたこともなかった資料がPDFファイルとして公開されているようです。これも面白いタイミングで,迷路のような紆余曲折を経た発見となりました。なにかの導きかもしれないので,これはおろそかに出来ないと思います。弁護士鉄道からの脱出に向けて。検察の在り方検討会議です。

 検察の再生に向けて【概要版】{[}PDF:156KB{]},は5ページで,ちょっと安心したのですが,検察の再生に向けて{[}PDF:573KB{]}は不意をつかれ現実に引き戻されたように42ページとなっていました。

 気がつくまで時間がかかったのですが,2つの異なる文書と思っていたのが,同じ「検察の再生に向けて」と題する文書で,概要版と全文に分かれているようです。概要版というのもおそらく初めて目にした言葉に思えます。要約版ならあったように思います。

 全文と概要,あるいは要約という取り扱いの違いは,私が長く頭を悩ませてきた告発状作成での問題で,大きな時間のロスにもなっていました。そういう意味でもこの2つのPDFファイルは参考になりそうです。

 時刻は14時19分です。輪島から生活保護の担当者の訪問があり,銀行に通帳の記帳に行ったり,書面に記入したりで時間がかかっていました。5分程前に帰られたところです。丁度,通帳が繰越になるというタイミングでもあり,余計に時間がかかりました。

 時刻は5月18日9時8分です。輪島からの生活保護の担当者の訪問があった後,1つツイートしてそのまま中断になっていました。昨日は夕方か夜の早めの時間だったと思うのですが,再審について驚くような発見がありました。

 モトケンこと矢部善朗弁護士(京都弁護士会)と今井智文氏のツイートについては一通りのことを書いたと思います。今朝になって他にモトケンこと矢部善朗弁護士(京都弁護士会)のことで取り上げたいことが出てきたのですが,参考資料ではなく,石川県警察の問題として扱います。

\begin{itemize}
\tightlist
\item
  〈〈〈 2021/05/18 09:13:54 Linux Emacs: 〈〈〈
\end{itemize}

\hypertarget{ux5e02ux5dddux5bdbux5f01ux8b77ux58ebux7b2cux4e8cux6771ux4eacux5f01ux8b77ux58ebux4f1aux5143ux691cux4e8b}{%
\subsubsection{市川寛弁護士(第二東京弁護士会・元検事)}\label{ux5e02ux5dddux5bdbux5f01ux8b77ux58ebux7b2cux4e8cux6771ux4eacux5f01ux8b77ux58ebux4f1aux5143ux691cux4e8b}}

\hypertarget{ux670827ux65e5ux306bux59cbux307eux308bux518dux5be9ux6cd5ux6539ux6b63ux3092ux3081ux3056ux3059ux5e02ux6c11ux306eux4f1aux306eux7d50ux62102ux5468ux5e74ux3092ux8a18ux5ff5ux304bux3089ux548cux6b4cux5c71ux306eux4fddux967aux91d1ux76eeux7684ux6bbaux5bb3ux4e8bux4ef6ux95a2ux9023ux306eux5e02ux5dddux5bdbux5f01ux8b77ux58ebux30c4ux30a4ux30fcux30c8ux306eux8a18ux93321}{%
\paragraph{4月27日に始まる「「再審法改正をめざす市民の会」の結成2周年を記念」から和歌山の保険金目的殺害事件関連の市川寛弁護士ツイートの記録(1)}\label{ux670827ux65e5ux306bux59cbux307eux308bux518dux5be9ux6cd5ux6539ux6b63ux3092ux3081ux3056ux3059ux5e02ux6c11ux306eux4f1aux306eux7d50ux62102ux5468ux5e74ux3092ux8a18ux5ff5ux304bux3089ux548cux6b4cux5c71ux306eux4fddux967aux91d1ux76eeux7684ux6bbaux5bb3ux4e8bux4ef6ux95a2ux9023ux306eux5e02ux5dddux5bdbux5f01ux8b77ux58ebux30c4ux30a4ux30fcux30c8ux306eux8a18ux93321}}

\begin{itemize}
\tightlist
\item
  〉〉〉 Linux Emacs: 2021/05/09 12:28:33 〉〉〉
\end{itemize}

:CATEGORIES: @kanazawabengosi \#金沢弁護士会 @JFBAsns
日本弁護士連合会(日弁連) \#法務省 @MOJ\_HOUMU \#市川寛弁護士
\#再審請求 \#冤罪 \#マスコミ

 ブログで埋め込みツイートの表示が多くなるので,予めエントリーを3つにわけました。(1)から(3)になります。再審請求の最新の情報から始まり,保険金殺人疑惑で警察やマスコミを批判する市川寛弁護士の意見は,私の告発・非常上告事件でもとても参考になりそうな資料です。

〉〉〉 kk\_hironoのリツイート 〉〉〉

\begin{itemize}
\tightlist
\item
  RT kk\_hirono(刑事告発・非常上告_金沢地方検察庁御中)|tongu(惇兄)
  日時:2021-05-09 12:33/2021/04/27 14:34 URL:
  \url{https://twitter.com/kk\_hirono/status/1391234778372866048} 
  \url{https://twitter.com/tongu/status/1386916603271413764} 
  \textgreater{}
  ナリ検よかった。いやあ検察庁は怖いところやで。霞ヶ関は多分あの何倍も怖いんやろうけど。
\end{itemize}

〉〉〉 kk\_hironoのリツイート 〉〉〉

\begin{itemize}
\item
  RT
  kk\_hirono(刑事告発・非常上告_金沢地方検察庁御中)|oilysalt(ぽんてむ君@Baron
  von Drafting) 日時:2021-05-09 12:33/2021/04/27 19:08 URL:
  \url{https://twitter.com/kk\_hirono/status/1391234950075097088} 
  \url{https://twitter.com/oilysalt/status/1386985617267904513} 
  \textgreater{}
  今年のGWも感染症対策で外出しないでステイホームになるということで本を仕入れた。
  前から気になっていた『ナリ検』(日本評論社)を購入、あと検察が舞台の小説ということで刑事手続きの副読本として『基本刑事訴訟法Ⅰ手続き理解編』(日本評論社)も買った。
  \url{https://t.co/izgPgwdhxK} 
\item
  〉〉〉 アカウント(@imarockcaster42)は,@kk\_hironoをブロックしています。リツイートできませんでした。
  〉〉〉 ¥\n ¥\n \url{https://t.co/N0EbTx04z3} 
\item
  〉〉〉 アカウント(@imarockcaster42)は,@kk\_hironoをブロックしています。リツイートできませんでした。
  〉〉〉 ¥\n ¥\n \url{https://t.co/o21hxDAfu5} 
\item
  〉〉〉 アカウント(@imarockcaster42)は,@kk\_hironoをブロックしています。リツイートできませんでした。
  〉〉〉 ¥\n ¥\n \url{https://t.co/4cVvIdV0oe} 
\item
  〉〉〉 アカウント(@imarockcaster42)は,@kk\_hironoをブロックしています。リツイートできませんでした。
  〉〉〉 ¥\n ¥\n \url{https://t.co/GSltGkoj34} 
\item
  〉〉〉 アカウント(@imarockcaster42)は,@kk\_hironoをブロックしています。リツイートできませんでした。
  〉〉〉 ¥\n ¥\n \url{https://t.co/WcQlYP4MZJ} 
\item
  〉〉〉 アカウント(@imarockcaster42)は,@kk\_hironoをブロックしています。リツイートできませんでした。
  〉〉〉 ¥\n ¥\n \url{https://t.co/Pib5kCxnUf} 
\item
  〉〉〉 アカウント(@imarockcaster42)は,@kk\_hironoをブロックしています。リツイートできませんでした。
  〉〉〉 ¥\n ¥\n \url{https://t.co/XqTrYcP2og} 
\item
  〉〉〉 アカウント(@imarockcaster42)は,@kk\_hironoをブロックしています。リツイートできませんでした。
  〉〉〉 ¥\n ¥\n \url{https://t.co/pDEa6B94mc} 
\item
  〉〉〉 アカウント(@imarockcaster42)は,@kk\_hironoをブロックしています。リツイートできませんでした。
  〉〉〉 ¥\n ¥\n \url{https://t.co/v0cSvs9f6l} 
\item
  〉〉〉 アカウント(@imarockcaster42)は,@kk\_hironoをブロックしています。リツイートできませんでした。
  〉〉〉 ¥\n ¥\n \url{https://t.co/LlOJtaWjVV} 
\item
  〉〉〉 アカウント(@imarockcaster42)は,@kk\_hironoをブロックしています。リツイートできませんでした。
  〉〉〉 ¥\n ¥\n \url{https://t.co/Q55P6i9Uwb} 
\item
  〉〉〉 アカウント(@imarockcaster42)は,@kk\_hironoをブロックしています。リツイートできませんでした。
  〉〉〉 ¥\n ¥\n \url{https://t.co/KiRfPU06Vp} 
\item
  〉〉〉 アカウント(@imarockcaster42)は,@kk\_hironoをブロックしています。リツイートできませんでした。
  〉〉〉 ¥\n ¥\n \url{https://t.co/eNhjstxJVh} 
\item
  〉〉〉 アカウント(@imarockcaster42)は,@kk\_hironoをブロックしています。リツイートできませんでした。
  〉〉〉 ¥\n ¥\n \url{https://t.co/ZRQcUPw7Bv} 
\item
  〉〉〉 アカウント(@imarockcaster42)は,@kk\_hironoをブロックしています。リツイートできませんでした。
  〉〉〉 ¥\n ¥\n \url{https://t.co/CW1CsS1KRq} 
\item
  〉〉〉 アカウント(@imarockcaster42)は,@kk\_hironoをブロックしています。リツイートできませんでした。
  〉〉〉 ¥\n ¥\n \url{https://t.co/zll4jBHO2a} 
\item
  〉〉〉 アカウント(@imarockcaster42)は,@kk\_hironoをブロックしています。リツイートできませんでした。
  〉〉〉 ¥\n ¥\n \url{https://t.co/6ykNUroVYz} 
\end{itemize}

※ @kk\_hironoのアカウントがブロックされ,リツイートに失敗したツイート

\begin{itemize}
\tightlist
\item
  TW imarockcaster42(弁護士 市川 寛) 日時:2021/04/27 12:14:52 URL:
  \url{https://twitter.com/imarockcaster42/status/1386881507067760642} 
  \textgreater{}
  5月20日午後2時から、私も運営委員を務めている「再審法改正をめざす市民の会」の結成2周年を記念しての集会(ライブ配信形式)が催されます。冤罪被害者による鼎談や、私も登壇するシンポジウムが行われます。よろしければご覧ください
  \url{https://t.co/vaF1BXh5wI} 
\end{itemize}

※ @kk\_hironoのアカウントがブロックされ,リツイートに失敗したツイート

\begin{itemize}
\tightlist
\item
  TW imarockcaster42(弁護士 市川 寛) 日時:2021/04/28 12:56:48 URL:
  \url{https://twitter.com/imarockcaster42/status/1387254444358651905} 
  \textgreater{}
  その結果、証拠構造に問題のある事件であることがわかり、ダメ出ししたことが複数回あります。このため警察からは嫌われましたが、そんなことよりヤバい事件の身柄を取って起訴してしまうことの方がずっと怖いので、自分の対処は正しかったと思っています
\end{itemize}

※ @kk\_hironoのアカウントがブロックされ,リツイートに失敗したツイート

\begin{itemize}
\tightlist
\item
  TW imarockcaster42(弁護士 市川 寛) 日時:2021/04/28 12:54:49 URL:
  \url{https://twitter.com/imarockcaster42/status/1387253948411650051} 
  \textgreater{}
  私もP時代にこういう事前相談を受けたことがありますが、なにしろゴーサイン=起訴なので神経を遣います。警察は説明用のペーパーくらいしか持参しないので、私は「これまでとれているKSを全部見せてほしい」と求めたり、場合によってはVら参考人を調べさせてもらいました
\end{itemize}

※ @kk\_hironoのアカウントがブロックされ,リツイートに失敗したツイート

\begin{itemize}
\tightlist
\item
  TW imarockcaster42(弁護士 市川 寛) 日時:2021/04/28 12:30:51 URL:
  \url{https://twitter.com/imarockcaster42/status/1387247915140149257} 
  \textgreater{}
  あの手の事件は逮捕前に警察が検察に相談し、了承を得ています。この「了承」は「起訴する」というものです。つまり、あの事件はよほどの誤算がない限り、起訴が決まっています。もちろん、起訴されたとしても有罪つまり犯人と断定することはできません。警察と検察の勝手な取り決めにすぎませんから
\end{itemize}

※ @kk\_hironoのアカウントがブロックされ,リツイートに失敗したツイート

\begin{itemize}
\tightlist
\item
  TW imarockcaster42(弁護士 市川 寛) 日時:2021/04/28 14:35:10 URL:
  \url{https://twitter.com/imarockcaster42/status/1387279199237382147} 
  \textgreater{}
  逮捕状が出たり勾留されるだけの証拠があっても、無実だった人がたくさんいます。全く証拠(らしきもの)がなければ捜査の対象にすらならないのですから、当然です。まして、無実でも起訴された人も少なくありません。大事件でこそ、捜査機関のやることを批判的に眺めることが大切だと思います
\end{itemize}

※ @kk\_hironoのアカウントがブロックされ,リツイートに失敗したツイート

\begin{itemize}
\tightlist
\item
  TW imarockcaster42(弁護士 市川 寛) 日時:2021/04/28 16:57:13 URL:
  \url{https://twitter.com/imarockcaster42/status/1387314946745147395} 
  \textgreater{}
  殺人という重罪で起訴されれば逃亡のおそれは誰にもあるでしょうが、仮にそれが勾留を必要とする理由なら、在宅求令状または逮捕中求令状で起訴してもいいはず。起訴前勾留が欲しいのは、身柄拘束下での被疑者取調べをしたいからですよね
\end{itemize}

※ @kk\_hironoのアカウントがブロックされ,リツイートに失敗したツイート

\begin{itemize}
\tightlist
\item
  TW imarockcaster42(弁護士 市川 寛) 日時:2021/04/29 11:10:16 URL:
  \url{https://twitter.com/imarockcaster42/status/1387590022119841796} 
  \textgreater{}
  大事件のときこそ、過去の同種冤罪の反省を踏まえた報道が求められるはずですが、マスコミはただただ警察発表と「被疑者はこんな(妙な)人でした~」報道しかしませんね。で、この事件が無罪になると捜査批判でしょ?今の記事に「まだ分からない」の一文がなぜ入れられないのでしょうか
\end{itemize}

※ @kk\_hironoのアカウントがブロックされ,リツイートに失敗したツイート

\begin{itemize}
\tightlist
\item
  TW imarockcaster42(弁護士 市川 寛) 日時:2021/04/29 11:15:08 URL:
  \url{https://twitter.com/imarockcaster42/status/1387591246529449986} 
  \textgreater{}
  見出しの「一致か」もミソ。「なんとかスポーツ」紙の手法ですね。「か」としておけばウソにはならない。NHKがこんな保険をかけながら犯罪報道に邁進していいんですかね
\end{itemize}

※ @kk\_hironoのアカウントがブロックされ,リツイートに失敗したツイート

\begin{itemize}
\tightlist
\item
  TW imarockcaster42(弁護士 市川 寛) 日時:2021/04/29 11:13:02 URL:
  \url{https://twitter.com/imarockcaster42/status/1387590720106553346} 
  \textgreater{} ドン・ファン事件 元妻と覚醒剤密売人
  スマホ位置情報一致か \url{https://t.co/pIaw5rLZ0M} 
  容疑者は事件前、SNSを通じて覚醒剤の密売人と知り合い、連絡を取っていたとみられる←「みられる」がミソ。警察の主観的評価です。法廷で弁護人に崩される可能性が十分に残っている証拠と読むべきでしょう
\end{itemize}

※ @kk\_hironoのアカウントがブロックされ,リツイートに失敗したツイート

\begin{itemize}
\tightlist
\item
  TW imarockcaster42(弁護士 市川 寛) 日時:2021/04/29 11:32:04 URL:
  \url{https://twitter.com/imarockcaster42/status/1387595509678804995} 
  \textgreater{}
  動機の解明などが焦点になる。←はい、これも犯罪報道のワンパターン・フレーズです
\end{itemize}

※ @kk\_hironoのアカウントがブロックされ,リツイートに失敗したツイート

\begin{itemize}
\tightlist
\item
  TW imarockcaster42(弁護士 市川 寛) 日時:2021/04/29 11:32:04 URL:
  \url{https://twitter.com/imarockcaster42/status/1387595508575707138} 
  \textgreater{}
  直接的な証拠が乏しい中、警察や検察は状況証拠の積み重ねによる難しい立証を強いられる可能性がある。←そこで、自白という「直接的な証拠」が欲しいんですよ。毎日新聞はこの事件で警察・検察を応援するスタンスなのですね
\end{itemize}

※ @kk\_hironoのアカウントがブロックされ,リツイートに失敗したツイート

\begin{itemize}
\tightlist
\item
  TW imarockcaster42(弁護士 市川 寛) 日時:2021/04/29 11:32:03 URL:
  \url{https://twitter.com/imarockcaster42/status/1387595507019571203} 
  \textgreater{}
  紀州ドン・ファン「非常に難しい事件」捜査員ら状況証拠積み重ね
  \textbar \url{https://t.co/0vVUXPCcPT} 
  「公判に耐えうるレベルで立証するには非常に難しい事件だった」←まるで有罪判決後の表現ですね。この記事を読んだ人が、被疑者が犯人だと決めつけないはずがありません。過去の冤罪の反省はどこに?
\end{itemize}

※ @kk\_hironoのアカウントがブロックされ,リツイートに失敗したツイート

\begin{itemize}
\tightlist
\item
  TW imarockcaster42(弁護士 市川 寛) 日時:2021/04/29 11:34:07 URL:
  \url{https://twitter.com/imarockcaster42/status/1387596025926348806} 
  \textgreater{}
  百歩譲って、今目の前にある事件が「固い」つまり限りなく有罪に見えるとしても、その視点をそのまま報じることにより、そうではない事件つまり無罪の可能性が十分にある報道の際、果たして読者が区別して読めるのか、ひいては「報道=有罪」という決めつけにならないか、そんな用心が欲しいんですよね
\end{itemize}

※ @kk\_hironoのアカウントがブロックされ,リツイートに失敗したツイート

\begin{itemize}
\tightlist
\item
  TW imarockcaster42(弁護士 市川 寛) 日時:2021/04/29 11:40:39 URL:
  \url{https://twitter.com/imarockcaster42/status/1387597670848471043} 
  \textgreater{}
  マスコミも、ただただ警察発表を垂れ流すのではなく、独自に裏付けを行って「固い」と言える事実だけ報じていると聞いたことがありますが、守秘義務のある弁護人は、むやみに事実を口外できないことから起きる不公平をどう考えているんですかね
\end{itemize}

※ @kk\_hironoのアカウントがブロックされ,リツイートに失敗したツイート

\begin{itemize}
\tightlist
\item
  TW imarockcaster42(弁護士 市川 寛) 日時:2021/04/29 11:40:39 URL:
  \url{https://twitter.com/imarockcaster42/status/1387597669753778176} 
  \textgreater{}
  刑事手続には、検察が有罪方向の証拠しか出さず、無罪方向の証拠を弁護人に見せないという大問題がありますが、マスコミも全く同じ姿勢ですね。警察が発表する、有罪方向の事実ばかり報じて、無罪方向の事実は報じない。そして市民は「そんなものか」と思い込む。これでは冤罪が減るはずがありません。
\end{itemize}

※ @kk\_hironoのアカウントがブロックされ,リツイートに失敗したツイート

\begin{itemize}
\tightlist
\item
  TW imarockcaster42(弁護士 市川 寛) 日時:2021/04/29 12:45:22 URL:
  \url{https://twitter.com/imarockcaster42/status/1387613955154059266} 
  \textgreater{} @hiromomosetsu
  外国の捜査機関も積極的に発表していますし、知る権利に資する効果もありますから、一律に禁じるのは難しいでしょうね。発表した証拠や事実をその後の公判で出さなかったり、弁護人に弾劾されて立証に失敗した場合は公訴棄却するとかの制裁はどうでしょうか。無責任な垂れ流しがなくなる気がしますが・・・
\end{itemize}

※ @kk\_hironoのアカウントがブロックされ,リツイートに失敗したツイート

\begin{itemize}
\item
  TW imarockcaster42(弁護士 市川 寛) 日時:2021/04/29 12:54:17 URL:
  \url{https://twitter.com/imarockcaster42/status/1387616198582407170} 
  \textgreater{}
  例の事件は殺人「など」で逮捕とありますし、覚醒剤密売人を絡めての捜査にしているのは、それによって接見禁止がつけやすいことも計算したんでしょうね。20日間勝負になるからには、被疑者にはかけられるだけのプレッシャーをかけたい。警察と検察のそれなりの準備が見て取れますね
\item
  〈〈〈 2021/05/09 12:41:31 Linux Emacs: 〈〈〈
\end{itemize}

\hypertarget{ux670827ux65e5ux306bux59cbux307eux308bux518dux5be9ux6cd5ux6539ux6b63ux3092ux3081ux3056ux3059ux5e02ux6c11ux306eux4f1aux306eux7d50ux62102ux5468ux5e74ux3092ux8a18ux5ff5ux304bux3089ux548cux6b4cux5c71ux306eux4fddux967aux91d1ux76eeux7684ux6bbaux5bb3ux4e8bux4ef6ux95a2ux9023ux306eux5e02ux5dddux5bdbux5f01ux8b77ux58ebux30c4ux30a4ux30fcux30c8ux306eux8a18ux93322}{%
\paragraph{4月27日に始まる「「再審法改正をめざす市民の会」の結成2周年を記念」から和歌山の保険金目的殺害事件関連の市川寛弁護士ツイートの記録(2)}\label{ux670827ux65e5ux306bux59cbux307eux308bux518dux5be9ux6cd5ux6539ux6b63ux3092ux3081ux3056ux3059ux5e02ux6c11ux306eux4f1aux306eux7d50ux62102ux5468ux5e74ux3092ux8a18ux5ff5ux304bux3089ux548cux6b4cux5c71ux306eux4fddux967aux91d1ux76eeux7684ux6bbaux5bb3ux4e8bux4ef6ux95a2ux9023ux306eux5e02ux5dddux5bdbux5f01ux8b77ux58ebux30c4ux30a4ux30fcux30c8ux306eux8a18ux93322}}

\begin{itemize}
\tightlist
\item
  〉〉〉 Linux Emacs: 2021/05/09 12:42:39 〉〉〉
\end{itemize}

:CATEGORIES: @kanazawabengosi \#金沢弁護士会 @JFBAsns
日本弁護士連合会(日弁連) \#法務省 @MOJ\_HOUMU \#市川寛弁護士
\#再審請求 \#冤罪 \#マスコミ

〉〉〉 kk\_hironoのリツイート 〉〉〉

\begin{itemize}
\tightlist
\item
  RT
  kk\_hirono(刑事告発・非常上告_金沢地方検察庁御中)|yomu\_kokkai(平河エリ
  Eri Hirakawa \textbar{} 読む国会) 日時:2021-05-09 12:45/2021/04/30
  09:11 URL: \url{https://twitter.com/kk\_hirono/status/1391237752683864068} 
  \url{https://twitter.com/yomu\_kokkai/status/1387922462885617665} 
  \textgreater{} @imarockcaster42
  なるほど・・・・・・。言葉の定義は難しいですね。ありがとうございます。勉強になりました。
\end{itemize}

〉〉〉 kk\_hironoのリツイート 〉〉〉

\begin{itemize}
\item
  RT
  kk\_hirono(刑事告発・非常上告_金沢地方検察庁御中)|<LaTeXのエラーで削除>
  日時:2021-05-09 12:47/2021/05/01 20:19 URL:
  \url{https://twitter.com/kk\_hirono/status/1391238337759899648} 
  <LaTeXのエラーで削除>\textgreater{}
  ナリ検、まだ途中だけど、め〜〜〜〜〜〜ちゃくちゃ面白いな〜〜〜〜〜〜!!!!!
\item
  〉〉〉 アカウント(@imarockcaster42)は,@kk\_hironoをブロックしています。リツイートできませんでした。
  〉〉〉 ¥\n ¥\n \url{https://t.co/wQIl0K7jc8} 
\item
  〉〉〉 アカウント(@luckymangan)は,@kk\_hironoをブロックしています。リツイートできませんでした。
  〉〉〉 ¥\n ¥\n \url{https://t.co/SuEX9vtZUv} 
\item
  〉〉〉 アカウント(@luckymangan)は,@kk\_hironoをブロックしています。リツイートできませんでした。
  〉〉〉 ¥\n ¥\n \url{https://t.co/s5OHjbTrCF} 
\item
  〉〉〉 アカウント(@imarockcaster42)は,@kk\_hironoをブロックしています。リツイートできませんでした。
  〉〉〉 ¥\n ¥\n \url{https://t.co/DeXkGIF4BN} 
\item
  〉〉〉 アカウント(@imarockcaster42)は,@kk\_hironoをブロックしています。リツイートできませんでした。
  〉〉〉 ¥\n ¥\n \url{https://t.co/oJhqMSfY7y} 
\item
  〉〉〉 アカウント(@imarockcaster42)は,@kk\_hironoをブロックしています。リツイートできませんでした。
  〉〉〉 ¥\n ¥\n \url{https://t.co/NcoqSxtER7} 
\item
  〉〉〉 アカウント(@imarockcaster42)は,@kk\_hironoをブロックしています。リツイートできませんでした。
  〉〉〉 ¥\n ¥\n \url{https://t.co/DLAIKXNxyN} 
\item
  〉〉〉 アカウント(@imarockcaster42)は,@kk\_hironoをブロックしています。リツイートできませんでした。
  〉〉〉 ¥\n ¥\n \url{https://t.co/oqhem3V5d4} 
\item
  〉〉〉 アカウント(@imarockcaster42)は,@kk\_hironoをブロックしています。リツイートできませんでした。
  〉〉〉 ¥\n ¥\n \url{https://t.co/ldHKPmTO5R} 
\item
  〉〉〉 アカウント(@imarockcaster42)は,@kk\_hironoをブロックしています。リツイートできませんでした。
  〉〉〉 ¥\n ¥\n \url{https://t.co/GKxYWKtRA4} 
\item
  〉〉〉 アカウント(@imarockcaster42)は,@kk\_hironoをブロックしています。リツイートできませんでした。
  〉〉〉 ¥\n ¥\n \url{https://t.co/sufiiQR4Jv} 
\item
  〉〉〉 アカウント(@imarockcaster42)は,@kk\_hironoをブロックしています。リツイートできませんでした。
  〉〉〉 ¥\n ¥\n \url{https://t.co/xpU2XOws9t} 
\item
  〉〉〉 アカウント(@imarockcaster42)は,@kk\_hironoをブロックしています。リツイートできませんでした。
  〉〉〉 ¥\n ¥\n \url{https://t.co/SfwLHvC1Gw} 
\item
  〉〉〉 アカウント(@imarockcaster42)は,@kk\_hironoをブロックしています。リツイートできませんでした。
  〉〉〉 ¥\n ¥\n \url{https://t.co/Yjyl1DnX51} 
\item
  〉〉〉 アカウント(@imarockcaster42)は,@kk\_hironoをブロックしています。リツイートできませんでした。
  〉〉〉 ¥\n ¥\n \url{https://t.co/YNuXMrojA9} 
\item
  〉〉〉 アカウント(@imarockcaster42)は,@kk\_hironoをブロックしています。リツイートできませんでした。
  〉〉〉 ¥\n ¥\n \url{https://t.co/qJ4WIiPQhq} 
\item
  〉〉〉 アカウント(@norinori1968)は,@kk\_hironoをブロックしています。リツイートできませんでした。
  〉〉〉 ¥\n ¥\n \url{https://t.co/vTq4U1F2Mh} 
\item
  〉〉〉 アカウント(@imarockcaster42)は,@kk\_hironoをブロックしています。リツイートできませんでした。
  〉〉〉 ¥\n ¥\n \url{https://t.co/jinKeFX24J} 
\item
  〉〉〉 アカウント(@imarockcaster42)は,@kk\_hironoをブロックしています。リツイートできませんでした。
  〉〉〉 ¥\n ¥\n \url{https://t.co/eUET997xNY} 
\item
  〉〉〉 アカウント(@imarockcaster42)は,@kk\_hironoをブロックしています。リツイートできませんでした。
  〉〉〉 ¥\n ¥\n \url{https://t.co/MT1oRX4qAY} 
\item
  〉〉〉 アカウント(@imarockcaster42)は,@kk\_hironoをブロックしています。リツイートできませんでした。
  〉〉〉 ¥\n ¥\n \url{https://t.co/VDcaeVxVQm} 
\item
  〉〉〉 アカウント(@imarockcaster42)は,@kk\_hironoをブロックしています。リツイートできませんでした。
  〉〉〉 ¥\n ¥\n \url{https://t.co/vS5yX6ih1K} 
\item
  〉〉〉 アカウント(@imarockcaster42)は,@kk\_hironoをブロックしています。リツイートできませんでした。
  〉〉〉 ¥\n ¥\n \url{https://t.co/ZWf6M154Iu} 
\item
  〉〉〉 アカウント(@imarockcaster42)は,@kk\_hironoをブロックしています。リツイートできませんでした。
  〉〉〉 ¥\n ¥\n \url{https://t.co/tKdL95SsaN} 
\end{itemize}

※ @kk\_hironoのアカウントがブロックされ,リツイートに失敗したツイート

\begin{itemize}
\tightlist
\item
  TW imarockcaster42(弁護士 市川 寛) 日時:2021/04/29 13:10:55 URL:
  \url{https://twitter.com/imarockcaster42/status/1387620386628534276} 
  \textgreater{} @luckymangan
  大いにやりそうですよね。元法務大臣の買収事件で、カネをもらった側を軒並み不起訴にしたやり口といい、最近は対向犯の片方を野放しにする手法が頻発している気がします
\end{itemize}

※ @kk\_hironoのアカウントがブロックされ,リツイートに失敗したツイート

\begin{itemize}
\tightlist
\item
  TW luckymangan(リーチ一発ツモ裏1) 日時:2021/04/29 13:06:51 URL:
  \url{https://twitter.com/luckymangan/status/1387619361880346625} 
  \textgreater{} @imarockcaster42
  ご教示ありがとうございます。仮に別件でやられている場合、その段階次第でそれも交渉材料になる余地はありそうと邪推してしまいます・・・
\end{itemize}

※ @kk\_hironoのアカウントがブロックされ,リツイートに失敗したツイート

\begin{itemize}
\tightlist
\item
  TW luckymangan(リーチ一発ツモ裏1) 日時:2021/04/29 12:57:32 URL:
  \url{https://twitter.com/luckymangan/status/1387617019713900545} 
  \textgreater{}
  覚醒剤密売人の方は敢えて逮捕せず、「逮捕を交渉材料に」(捜査機関に有利な)調書を巻いた上、(逮捕してない密売人に対する)罪証隠滅の余地があるとして、接見禁止や勾留を正当化する要素ともして、一石二鳥なのね・・・
  \url{https://t.co/8u85QQzVfL} 
\end{itemize}

※ @kk\_hironoのアカウントがブロックされ,リツイートに失敗したツイート

\begin{itemize}
\tightlist
\item
  TW imarockcaster42(弁護士 市川 寛) 日時:2021/04/29 13:16:50 URL:
  \url{https://twitter.com/imarockcaster42/status/1387621874369110016} 
  \textgreater{}
  仮に自白しても、それが身柄拘束後のものであるというだけで、信用性を割り引いて評価すべきなんですよね。報道によれば任意での調べでは否認したそうですが、なぜ捕まると自白するのかを虚心坦懐に見つめるべきだと思います。得てして「逮捕で懲らしめられて参ったからだろう」とみなしがちですからね
\end{itemize}

※ @kk\_hironoのアカウントがブロックされ,リツイートに失敗したツイート

\begin{itemize}
\tightlist
\item
  TW imarockcaster42(弁護士 市川 寛) 日時:2021/04/29 14:14:08 URL:
  \url{https://twitter.com/imarockcaster42/status/1387636293945892864} 
  \textgreater{}
  もっとも、法は「絶対に有罪と確信できるだけの証拠がないうちに起訴してはならない」とは定めていないので、起訴後に客観証拠を集める補充捜査を続けてもいいとは思うんですけどね。それでも、例えば恐喝未遂のVPSがないままに起訴された事件を受け取ると、公判検事は青ざめます
\end{itemize}

※ @kk\_hironoのアカウントがブロックされ,リツイートに失敗したツイート

\begin{itemize}
\tightlist
\item
  TW imarockcaster42(弁護士 市川 寛) 日時:2021/04/29 14:10:33 URL:
  \url{https://twitter.com/imarockcaster42/status/1387635391981711365} 
  \textgreater{}
  PとBの両方を経験した立場から言うと、「20日間という時間は、被疑者を取り調べられる時間としては長過ぎる。が、(逮捕前の内偵捜査のない場合)客観証拠を集められる時間としては短過ぎる」という感覚です。後者のうらみから、つい前者に傾いて自白をとりに行きがちなのではないかと思います
\end{itemize}

※ @kk\_hironoのアカウントがブロックされ,リツイートに失敗したツイート

\begin{itemize}
\tightlist
\item
  TW imarockcaster42(弁護士 市川 寛) 日時:2021/04/29 20:27:20 URL:
  \url{https://twitter.com/imarockcaster42/status/1387730216122912771} 
  \textgreater{} \#Nowplaying
  ジョー山中「人間の証明のテーマ」。歌もメロディもいいのですが、まるで韻を踏んでいない英詩はどうにかできなかったのかなと。角川映画の主題歌にはいい曲が多いですね
  \url{https://t.co/twfT26Sauy} 
\end{itemize}

※ @kk\_hironoのアカウントがブロックされ,リツイートに失敗したツイート

\begin{itemize}
\tightlist
\item
  TW imarockcaster42(弁護士 市川 寛) 日時:2021/04/29 20:37:22 URL:
  \url{https://twitter.com/imarockcaster42/status/1387732740955512833} 
  \textgreater{}
  せめて「被疑者が真犯人であってほしい。が、万が一にもそうでなければ、冤罪を生む上に真犯人を取り逃がし、地域住民を再び不安に陥れる結果になる。捜査の行方を見守りたい」くらいは書いてほしいですよね
\end{itemize}

※ @kk\_hironoのアカウントがブロックされ,リツイートに失敗したツイート

\begin{itemize}
\tightlist
\item
  TW imarockcaster42(弁護士 市川 寛) 日時:2021/04/29 20:35:38 URL:
  \url{https://twitter.com/imarockcaster42/status/1387732303288291328} 
  \textgreater{}
  第一、なんで、どこもかしこもこんな「警察よくやったぁ!」報道しかできないんですかね。マスコミは、逮捕時には警察にご祝儀を配ると約束しているんですかね
\end{itemize}

※ @kk\_hironoのアカウントがブロックされ,リツイートに失敗したツイート

\begin{itemize}
\tightlist
\item
  TW imarockcaster42(弁護士 市川 寛) 日時:2021/04/29 20:35:38 URL:
  \url{https://twitter.com/imarockcaster42/status/1387732301748989953} 
  \textgreater{}
  「紀州のドン・ファン」元妻が遺産相続に執着 和歌山カレー事件と同じ消去法で立件〈dot.〉\url{https://t.co/8Fe3rDPMOb} 
  和歌山県警では過去にそっくりの事件があった。(略)和歌山カレー事件だ。←これ、冤罪ですよ
\end{itemize}

※ @kk\_hironoのアカウントがブロックされ,リツイートに失敗したツイート

\begin{itemize}
\tightlist
\item
  TW imarockcaster42(弁護士 市川 寛) 日時:2021/04/29 20:41:50 URL:
  \url{https://twitter.com/imarockcaster42/status/1387733864450183173} 
  \textgreater{}
  「マスコミは犯人視報道を慎んで欲しい」と訴えると、ひょっとしたら記者あたりから「それは分かっているが、渦中にあるときは諸般の事情でそうもいかないんだよ」みたいな声が漏れるかも。が、無罪判決後の捜査批判記事を読んだ警察・検察関係者も、全く同じ愚痴をこぼすでしょうね
\end{itemize}

※ @kk\_hironoのアカウントがブロックされ,リツイートに失敗したツイート

\begin{itemize}
\tightlist
\item
  TW imarockcaster42(弁護士 市川 寛) 日時:2021/04/29 21:14:23 URL:
  \url{https://twitter.com/imarockcaster42/status/1387742055640690697} 
  \textgreater{}
  捜査はともすれば「この被疑者は絶対に犯人」という思い込みで一方通行になりがちですが、そんな危ない暴走に「ちょっと待て」と再考を促すのが報道機関の使命ではないのでしょうか。警察と同じ方向に歩むだけなら、報道機関ではなく広報機関になってしまうと思います
\end{itemize}

※ @kk\_hironoのアカウントがブロックされ,リツイートに失敗したツイート

\begin{itemize}
\tightlist
\item
  TW imarockcaster42(弁護士 市川 寛) 日時:2021/04/30 08:57:14 URL:
  \url{https://twitter.com/imarockcaster42/status/1387918932053430275} 
  \textgreater{}
  「合理的な疑い」があるのに有罪にしていれば冤罪ですよね。また、「真相がわからない」から冤罪と言うのに抵抗があるというのは、「それなら真犯人を連れてこい」とばかりに被告人側に立証責任を転換するような考え方だと思います。有罪にできない事件を有罪にしたら、それはすなわち冤罪でしょう
  \url{https://t.co/RAvuTD9ENu} 
\end{itemize}

※ @kk\_hironoのアカウントがブロックされ,リツイートに失敗したツイート

\begin{itemize}
\tightlist
\item
  TW imarockcaster42(弁護士 市川 寛) 日時:2021/04/30 09:12:43 URL:
  \url{https://twitter.com/imarockcaster42/status/1387922831221035010} 
  \textgreater{} @yomu\_kokkai \url{https://t.co/RFD52DlxsL} 
  ウィキにはこんな記載がありました
\end{itemize}

※ @kk\_hironoのアカウントがブロックされ,リツイートに失敗したツイート

\begin{itemize}
\tightlist
\item
  TW imarockcaster42(弁護士 市川 寛) 日時:2021/04/30 08:58:44 URL:
  \url{https://twitter.com/imarockcaster42/status/1387919311776456705} 
  \textgreater{} \url{https://t.co/RLAb09Wzt9} 
  この本を読むだけでも、冤罪だと思えるのではないでしょうか
\end{itemize}

※ @kk\_hironoのアカウントがブロックされ,リツイートに失敗したツイート

\begin{itemize}
\tightlist
\item
  TW imarockcaster42(弁護士 市川 寛) 日時:2021/04/30 21:27:37 URL:
  \url{https://twitter.com/imarockcaster42/status/1388107771007037443} 
  \textgreater{}
  「紀州のドンなんとか」事件の取材に駆り出されてGWが潰れかけているマスコミ関係者の皆様、そんな取材は大して意味がないので、ゆっくり休暇をとってください
\end{itemize}

※ @kk\_hironoのアカウントがブロックされ,リツイートに失敗したツイート

\begin{itemize}
\tightlist
\item
  TW norinori1968(国立研究開発個人原田知世研究機構) 日時:2021/05/01
  10:07:41 URL:
  \url{https://twitter.com/norinori1968/status/1388299051645370368} 
  \textgreater{}
  「冤罪」という言葉の意味を、あるべき内容にくみかえて伝えていくことが重要だと改めて思います。\\
  \textgreater{}\\
  \textgreater{}
  「有罪と証明できなければ無罪」なのであって、「無罪と証明できなければ有罪」なのではなく、従って、「真相」なるものが刑事手続の中でわからなくても、冤罪は冤罪。
  \url{https://t.co/x3Gms0M8Ns} 
\end{itemize}

※ @kk\_hironoのアカウントがブロックされ,リツイートに失敗したツイート

\begin{itemize}
\tightlist
\item
  TW imarockcaster42(弁護士 市川 寛) 日時:2021/05/01 14:33:47 URL:
  \url{https://twitter.com/imarockcaster42/status/1388366016292425730} 
  \textgreater{}
  付言すると、検察は無罪に傾く証拠は弁護人に見せず、裁判所にも出さないので、法廷に出される証拠は捜査で集められた証拠の一部です。この「証拠隠し」によってたくさんの冤罪が生まれてきました。目下捜査中の事件も、冤罪にならない保障など全くありません
\end{itemize}

※ @kk\_hironoのアカウントがブロックされ,リツイートに失敗したツイート

\begin{itemize}
\tightlist
\item
  TW imarockcaster42(弁護士 市川 寛) 日時:2021/05/01 14:29:56 URL:
  \url{https://twitter.com/imarockcaster42/status/1388365047957581825} 
  \textgreater{}
  さらに、起訴されたからと言って、その事件の証拠が足りているとも限りません。あくまで検察がそう判断しただけで、法廷で弁護人と裁判所の批判的検討に耐えられるだけの証拠があるかは分からないからです。特に弁護人のチェックのない証拠で「証明十分」と即断することは危険です
\end{itemize}

※ @kk\_hironoのアカウントがブロックされ,リツイートに失敗したツイート

\begin{itemize}
\tightlist
\item
  TW imarockcaster42(弁護士 市川 寛) 日時:2021/05/01 14:26:41 URL:
  \url{https://twitter.com/imarockcaster42/status/1388364229686661121} 
  \textgreater{}
  捜査中の事件に関する報道に接して「この被疑者は犯人に間違いない」と思うのは自由ですが、もしそうなら検察がとっくに起訴しています。起訴しないのは、その時点ではまだ証拠が十分ではないと検察が考えているからです
\end{itemize}

※ @kk\_hironoのアカウントがブロックされ,リツイートに失敗したツイート

\begin{itemize}
\tightlist
\item
  TW imarockcaster42(弁護士 市川 寛) 日時:2021/05/01 14:39:59 URL:
  \url{https://twitter.com/imarockcaster42/status/1388367576309243909} 
  \textgreater{} \url{https://t.co/8Fe3rDPMOb} 
  覚せい剤殺人のトリックの全貌は明らかになるのか。←こんなことを初めから書いているマスコミが、弁護人と同じレベルで事実や証拠をチェックしているとは思えません
\end{itemize}

※ @kk\_hironoのアカウントがブロックされ,リツイートに失敗したツイート

\begin{itemize}
\tightlist
\item
  TW imarockcaster42(弁護士 市川 寛) 日時:2021/05/01 14:36:57 URL:
  \url{https://twitter.com/imarockcaster42/status/1388366812778172416} 
  \textgreater{}
  報道される事実や証拠は、専らマスコミがそうだと判断して流しているだけで、弁護人のチェックを経ていません。マスコミもチェックしているのでしょうが、犯罪報道を眺める限り、警察と歩調を合わせるマスコミが多いので、チェックは甘いと思います。つまり報道される事実をたやすく信じるのは危険です
\end{itemize}

※ @kk\_hironoのアカウントがブロックされ,リツイートに失敗したツイート

\begin{itemize}
\tightlist
\item
  TW imarockcaster42(弁護士 市川 寛) 日時:2021/05/01 14:44:08 URL:
  \url{https://twitter.com/imarockcaster42/status/1388368620045012993} 
  \textgreater{}
  マスコミからは「弁護人が情報を流さないからしょうがない」との反論があるでしょうが、弁護人に頼らずに公平化を図れる方策があります。それは全ての記事に「有罪判決があるまでは分からない」という一文を添えることです
\end{itemize}

※ @kk\_hironoのアカウントがブロックされ,リツイートに失敗したツイート

\begin{itemize}
\item
  TW imarockcaster42(弁護士 市川 寛) 日時:2021/05/01 14:44:08 URL:
  \url{https://twitter.com/imarockcaster42/status/1388368618941947905} 
  \textgreater{}
  マスコミも公平な犯罪報道を願っているとは思います。が、被疑者の言い分を聞いているのは弁護人だけです(マスコミが接見すれば別ですが)。弁護人には守秘義務があるので、警察のように情報を垂れ流すことができません。マスコミはそれを知りつつ、警察発表に立脚しての犯罪報道を続けるのでしょうか
\item
  〈〈〈 2021/05/09 13:39:23 Linux Emacs: 〈〈〈
\end{itemize}

\hypertarget{ux670827ux65e5ux306bux59cbux307eux308bux518dux5be9ux6cd5ux6539ux6b63ux3092ux3081ux3056ux3059ux5e02ux6c11ux306eux4f1aux306eux7d50ux62102ux5468ux5e74ux3092ux8a18ux5ff5ux304bux3089ux548cux6b4cux5c71ux306eux4fddux967aux91d1ux76eeux7684ux6bbaux5bb3ux4e8bux4ef6ux95a2ux9023ux306eux5e02ux5dddux5bdbux5f01ux8b77ux58ebux30c4ux30a4ux30fcux30c8ux306eux8a18ux93323}{%
\paragraph{4月27日に始まる「「再審法改正をめざす市民の会」の結成2周年を記念」から和歌山の保険金目的殺害事件関連の市川寛弁護士ツイートの記録(3)}\label{ux670827ux65e5ux306bux59cbux307eux308bux518dux5be9ux6cd5ux6539ux6b63ux3092ux3081ux3056ux3059ux5e02ux6c11ux306eux4f1aux306eux7d50ux62102ux5468ux5e74ux3092ux8a18ux5ff5ux304bux3089ux548cux6b4cux5c71ux306eux4fddux967aux91d1ux76eeux7684ux6bbaux5bb3ux4e8bux4ef6ux95a2ux9023ux306eux5e02ux5dddux5bdbux5f01ux8b77ux58ebux30c4ux30a4ux30fcux30c8ux306eux8a18ux93323}}

\begin{itemize}
\tightlist
\item
  〉〉〉 Linux Emacs: 2021/05/09 13:40:38 〉〉〉
\end{itemize}

:CATEGORIES: @kanazawabengosi \#金沢弁護士会 @JFBAsns
日本弁護士連合会(日弁連) \#法務省 @MOJ\_HOUMU \#市川寛弁護士
\#再審請求 \#冤罪 \#マスコミ

〉〉〉 kk\_hironoのリツイート 〉〉〉

\begin{itemize}
\tightlist
\item
  RT
  kk\_hirono(刑事告発・非常上告_金沢地方検察庁御中)|SaKka\_romana(SaKka
  romana) 日時:2021-05-09 13:42/2021/05/07 07:42 URL:
  \url{https://twitter.com/kk\_hirono/status/1391252225939283969} 
  \url{https://twitter.com/SaKka\_romana/status/1390436848644739073} 
  \textgreater{} 『ナリ検 ある次席検事の挑戦』\#読了
  検察官から弁護士になるのを「ヤメ検」と言いますが、その逆の弁護士から検察官になるのを評して「ナリ検」としたのが本作です😄
  ある冤罪事件を舞台にした小説で、検察官・弁護士の立場から一つの事件を考察していて、とても面白いストーリーと結末でした✨
\end{itemize}

〉〉〉 kk\_hironoのリツイート 〉〉〉

\begin{itemize}
\item
  RT
  kk\_hirono(刑事告発・非常上告_金沢地方検察庁御中)|todateyoshiyuki(弁護士戸舘圭之オフィシャル/とってぃ/袴田事件弁護団)
  日時:2021-05-09 13:43/2021/05/08 14:19 URL:
  \url{https://twitter.com/kk\_hirono/status/1391252483691859969} 
  \url{https://twitter.com/todateyoshiyuki/status/1390899169892593667} 
  \textgreater{}
  こと刑事に関していえば、将来、検察官になることがない限り、弁護士が論文をあれこれ書いても特に不都合はなさそうですね。立場の互換性がある分野だと困るということですよね。
\item
  〉〉〉 アカウント(@imarockcaster42)は,@kk\_hironoをブロックしています。リツイートできませんでした。
  〉〉〉 ¥\n ¥\n \url{https://t.co/0myMNGtzTz} 
\item
  〉〉〉 アカウント(@imarockcaster42)は,@kk\_hironoをブロックしています。リツイートできませんでした。
  〉〉〉 ¥\n ¥\n \url{https://t.co/1nc0FjAUYH} 
\item
  〉〉〉 アカウント(@imarockcaster42)は,@kk\_hironoをブロックしています。リツイートできませんでした。
  〉〉〉 ¥\n ¥\n \url{https://t.co/CqrisHwJf2} 
\end{itemize}

※ @kk\_hironoのアカウントがブロックされ,リツイートに失敗したツイート

\begin{itemize}
\tightlist
\item
  TW imarockcaster42(弁護士 市川 寛) 日時:2021/05/03 12:07:28 URL:
  \url{https://twitter.com/imarockcaster42/status/1389053968362344448} 
  \textgreater{} 桜井昌司さんが語るー冤罪・布川事件ー
  \url{https://t.co/GuJls71U5a}  @YouTubeより
  私も運営委員を務める「再審法改正をめざす市民の会」の活動内容も分かる映像です。ぜひご覧ください
\end{itemize}

※ @kk\_hironoのアカウントがブロックされ,リツイートに失敗したツイート

\begin{itemize}
\tightlist
\item
  TW imarockcaster42(弁護士 市川 寛) 日時:2021/05/06 10:17:12 URL:
  \url{https://twitter.com/imarockcaster42/status/1390113383018467332} 
  \textgreater{} 2週間後になります。よろしくお願いします
  \url{https://t.co/A28TslDawX} 
\end{itemize}

※ @kk\_hironoのアカウントがブロックされ,リツイートに失敗したツイート

\begin{itemize}
\item
  TW imarockcaster42(弁護士 市川 寛) 日時:2021/05/08 15:33:57 URL:
  \url{https://twitter.com/imarockcaster42/status/1390917871673376771} 
  \textgreater{} @todateyoshiyuki
  しばらくの間でも、検事になってみてください(爆
\item
  〈〈〈 2021/05/09 13:45:26 Linux Emacs: 〈〈〈
\end{itemize}

\hypertarget{ux5f01ux8b77ux4ebaux306bux306fux5b88ux79d8ux7fa9ux52d9ux304cux3042ux308bux306eux3067ux8b66ux5bdfux306eux3088ux3046ux306bux60c5ux5831ux3092ux5782ux308cux6d41ux3059ux3053ux3068ux304cux3067ux304dux307eux305bux3093ux30deux30b9ux30b3ux30dfux306fux305dux308cux3092ux77e5ux308aux3064ux3064ux8b66ux5bdfux767aux8868ux306bux7acbux811aux3057ux3066ux306eux72afux7f6aux5831ux9053ux3068ux3044ux3046ux5e02ux5dddux5bdbux5f01ux8b77ux58ebux306eux30c4ux30a4ux30fcux30c8}{%
\paragraph{「弁護人には守秘義務があるので、警察のように情報を垂れ流すことができません。マスコミはそれを知りつつ、警察発表に立脚しての犯罪報道」という市川寛弁護士のツイート}\label{ux5f01ux8b77ux4ebaux306bux306fux5b88ux79d8ux7fa9ux52d9ux304cux3042ux308bux306eux3067ux8b66ux5bdfux306eux3088ux3046ux306bux60c5ux5831ux3092ux5782ux308cux6d41ux3059ux3053ux3068ux304cux3067ux304dux307eux305bux3093ux30deux30b9ux30b3ux30dfux306fux305dux308cux3092ux77e5ux308aux3064ux3064ux8b66ux5bdfux767aux8868ux306bux7acbux811aux3057ux3066ux306eux72afux7f6aux5831ux9053ux3068ux3044ux3046ux5e02ux5dddux5bdbux5f01ux8b77ux58ebux306eux30c4ux30a4ux30fcux30c8}}

\begin{itemize}
\tightlist
\item
  〉〉〉 Linux Emacs: 2021/05/09 14:07:12 〉〉〉
\end{itemize}

:CATEGORIES: @kanazawabengosi \#金沢弁護士会 @JFBAsns
日本弁護士連合会(日弁連) \#法務省 @MOJ\_HOUMU \#市川寛弁護士
\#マスコミ \#守秘義務

\begin{quote}
《引用の始まり》
\end{quote}

\begin{quote}
1965(昭和40)年、神奈川県生れ。中央大学卒。1990(平成2)年に司法試験に合格し、1993年検事任官。2000年から佐賀地検に三席検事として勤務、佐賀市農協背任事件の主任検事を務める。同事件の被疑者に不当な取調べを行ったことについて法廷で証言し、大きく報道される。その後、被告人は無罪となった。2005年に辞職し、2007年弁護士登録。2014年からアパリ法律事務所に所属。
\end{quote}

\begin{quote}
《引用の終わり》
\end{quote}

\begin{itemize}
\tightlist
\item
  市川寛 \textbar{} 著者プロフィール \textbar{}
  新潮社 \url{https://www.shinchosha.co.jp/writer/5052/n} 
\end{itemize}

 上記は先に引用していた市川寛弁護士のプロフィールになりますが,新潮社だと気が付きました。ドメインもローマ字でそのようになっていますが,今初めて気がついたドメインになります。

 市川寛弁護士のツイートは,次の3つのエントリーにまとめ予め記録をしておきました。ツイートの削除が著しく多いのも市川寛弁護士ならではの特徴です。

\begin{itemize}
\tightlist
\item
  1351:2021-05-09\_12:42:00 \#告発状 \#\#\#\#
  4月27日に始まる「「再審法改正をめざす市民の会」の結成2周年を記念」から和歌山の保険金目的殺害事件関連の市川寛弁護士ツイートの記録(1)
  \url{https://hirono-hideki.hatenadiary.jp/entry/2021/05/09/124157} 
\item
  1352:2021-05-09\_13:39:49 \#告発状 \#\#\#\#
  4月27日に始まる「「再審法改正をめざす市民の会」の結成2周年を記念」から和歌山の保険金目的殺害事件関連の市川寛弁護士ツイートの記録(2)
  \url{https://hirono-hideki.hatenadiary.jp/entry/2021/05/09/133946} 
\item
  1353:2021-05-09\_13:45:54 \#告発状 \#\#\#\#
  4月27日に始まる「「再審法改正をめざす市民の会」の結成2周年を記念」から和歌山の保険金目的殺害事件関連の市川寛弁護士ツイートの記録(3)
  \url{https://hirono-hideki.hatenadiary.jp/entry/2021/05/09/134552} 
\end{itemize}

 やっておこうと思いながらツイートの数を記載していなかったのですが,60ぐらいはあったような気がします。すべてのツイートには目を通しきれておらず,少なめに見て9割ぐらいと思いますが,1つだけでもとても意味深い資料性の高いツイートがいくつかありました。

 取り上げ方というか見出しの付け方にも頭を悩ませたのですが,最新のツイートから遡って,最初に強くインパクトを感じた標題のツイートに決定しました。次のツイートになります。

\begin{itemize}
\tightlist
\item
  TW imarockcaster42(弁護士 市川 寛) 日時: 2021/05/01 14:44:08 URL:
  \url{https://twitter.com/imarockcaster42/status/1388368618941947905} 
  \textgreater{}
  マスコミも公平な犯罪報道を願っているとは思います。が、被疑者の言い分を聞いているのは弁護人だけです(マスコミが接見すれば別ですが)。弁護人には守秘義務があるので、警察のように情報を垂れ流すことができません。マスコミはそれを知りつつ、警察発表に立脚しての犯罪報道を続けるのでしょうか
\end{itemize}

 他にも見出し付きで記録しておきたいツイートがいくつもあるのですが,そう時間をかけ告発状の紙面を割くわけにもいきません。再審請求については,掲載によるご紹介程度にとどめておきたいと思います。

 とてもわかりやすい市川寛弁護士の理屈ですが,被疑者本人の話というのは弁護士しか聞いておらず,弁護士は守秘義務があるので話せない,警察は情報を都合よく垂れ流し,マスコミが見境なしに飛びつき無批判に拡散し,被疑者を貶め冤罪の危険性を高めている,といったところでしょう。

 なかなかここまではっきり断言する弁護士も見当たらなかったので,従来の認識を整理するかたちで再認識できました。弁護士が真剣に労を厭わずに取り組む刑事事件という前提もつきそうですが,弁護士に放置された冤罪事件というのもかなりありそうです。

 割と最近,再審で無罪判決となった滋賀県の湖東病院事件も,両親が滋賀県内中の弁護士を訪ね廻り,井戸謙一弁護士に出会ったことで,刑事裁判官の経験はあるものの民事専門だったという同弁護士が,京都の刑事弁護を専門とする弁護士の協力を得て,再審請求につなげたという話です。

 冤罪事件で弁護士の責任というのはほとんど問題になりません。問題にされたことはあるようですが,富山県の弁護士会の会長による歴史的発言がありました。数年前からネットで情報を探すのも難しそうでしたが,私の記録したものがいくつか残されていると思います。

 そもそも富山弁護士会なのか富山県弁護士会なのか思い出せないのですが,このように県のつく弁護士会とつかない弁護士会が混在していて分かりづらくなっています。滋賀県の場合は滋賀県弁護士会であったように思います。

\begin{lstlisting}
py37_env ❯ d|grep 富山県弁護士会
\end{lstlisting}

\begin{itemize}
\tightlist
\item
  2019年02月20日22時19分の登録:
  \たけるbot @itotakeru\富山県弁護士会では、単位会と執行部一任の票が賛成になりました¥\n¥\n当日100人の出席が必要です。¥\n出席可能で代理人になれる弁護士を募
  \url{http://hirono2014sk.blogspot.com/2019/02/botitotakeru-100.html} 
\item
  2020年06月09日21時08分の登録:
  \えきなんローヤー? @ekinan\_lawyer\富山県警察による違法捜査に抗議し再発防止を求める会長声明|決議文・意見書・会長声明|富山県弁護士会
  \url{http://hirono2014sk.blogspot.com/2020/06/ekinanlawyer.html} 
\item
  2020年06月09日21時40分の登録:
  REGEXP:''富山県弁護士会''/データベース登録済みツイート:2020年06月09日21時38分の記録:ユーザ・投稿:48/60件
  \url{http://hirono2014sk.blogspot.com/2020/06/regexp2020060921384860.html} 
\end{itemize}

 Bloggerでは記録がなかったようですが,富山弁護士会では該当がなかったので,富山県弁護士会が正しいようです。これでtwilog-serch-post
のまとめを作成します。

\begin{itemize}
\tightlist
\item
  2021年05月09日14時41分の登録:
  「富山県弁護士会」を@hirono\_hideki @kk\_hirono @s\_hironoで検索 48件の該当 2021-05-09\_14:41の記録
  \url{https://kk2020-09.blogspot.com/2021/05/hironohidekikkhironoshirono482021-05.html} 
\item
  2021年05月09日14時42分の登録:
  「滋賀県弁護士会」を@hirono\_hideki @kk\_hirono @s\_hironoで検索 2件の該当 2021-05-09\_14:42の記録
  \url{https://kk2020-09.blogspot.com/2021/05/hironohidekikkhironoshirono22021-05.html} 
\item
  2021年05月09日14時43分の登録:
  「滋賀弁護士会」を@hirono\_hideki @kk\_hirono @s\_hironoで検索 74件の該当 2021-05-09\_14:43の記録
  \url{https://kk2020-09.blogspot.com/2021/05/hironohidekikkhironoshirono742021-05.html} 
\end{itemize}

 滋賀県弁護士会だと思っていたのですが,該当が2件と少なすぎるので,ネットで調べたところ滋賀弁護士会が正しいようでした。たしか,よくみる野田隼人弁護士のTwitterのプロフィールでも弁護士会の名前を度々見かけていたはずですが,滋賀弁護士会の方に違和感をおぼえます。

\begin{quote}
《引用の始まり》
\end{quote}

\begin{quote}
野田隼人 Atty. NODA
Hayato@nodahayato弁護士12年目。専門はフィリピン法務(投資・進出・現地許認可・現地労使関係その他企業法務,結婚離婚・相続,その他)と国内不動産と刑事法。ITで弁護士の敷居を下げる実験中。Pythonista。同志社高→上智→東北ロー→京大博士後期。Beretta
A400 Xplor Unico。銃猟。滋賀弁護士会。兵貴勝不貴久。Makati
Citynodalaw.jp2008年6月からTwitterを利用しています5,635 フォロー中1.4万
フォロワー
\end{quote}

\begin{quote}
《引用の終わり》
\end{quote}

\begin{itemize}
\tightlist
\item
  野田隼人 Atty. NODA Hayato(@nodahayato)さんの返信があるツイート /
  Twitter \url{https://twitter.com/nodahayato/with\_replies} 
\end{itemize}

 確認したところ野田隼人弁護士のTwitterのプロフィールにも滋賀弁護士会となっていました。富山県弁護士会の氷見強姦冤罪事件と,滋賀弁護士会の湖東病院事件は,かなり類似点の多い冤罪事件だと私は分析しています。

2010-05-08 23:38:27 ``富山えん罪事件についての報告書 by 富山県弁護士会
\url{http://goo.gl/rGyl} 
当時の国選弁護人の責任には言及せず、記者会見を開いた今村元・会長は「冤罪を生んだ責任の99・9%は捜査機関にあり、弁護士の処分や注意は行わない」と述べた。(引用)''
\url{https://twitter.com/hirono\_hideki/status/13611376145} 

2010-05-08 23:42:22 ``富山えん罪事件についての報告書 by 富山県弁護士会
\url{http://goo.gl/rGyl} 
これを読んで、ずいぶんと被疑者、国民をナメ腐った弁護士だと憤りを感じました。また、総体的な刑事弁護力、弁護士自身の地位の低下をも招いたのではないかと、個人的に考えております。''
\url{https://twitter.com/hirono\_hideki/status/13611576453} 

2013-09-06 23:58:01
``富山県弁護士会は6日、犯罪被害者にしつこく示談を持ちかけたなどとして、高森浩弁護士(46)を4日付で、退会命令の懲戒処分にしたと発表した。
\url{http://ow.ly/oD6Jh''} 
\url{https://twitter.com/hirono\_hideki/status/375996315144323072} 

2013-09-08 23:10:50
``富山県弁護士会は6日、犯罪被害者にしつこく示談を持ちかけたなどとして、高森浩弁護士(46)を4日付で、退会命令の懲戒処分にしたと発表した。''``富山の弁護士、5回目の懲戒で退会命令
- MSN産経ニュース''" -
\url{http://sankei.jp.msn.com/affairs/news/130906/crm13090621170009-n1.htm}  "
\url{https://twitter.com/hirono\_hideki/status/376709217068912641} 

\begin{itemize}
\item
  今村 元|弁護士を探す|富山県弁護士会 \url{https://t.co/lnyHA7L3qk}  ¥\n
  平成18年4月から平成19年3月まで富山県弁護士会副会長 ¥\n
  平成19年4月から平成20年3月まで富山県弁護士会会長
\item
  今村記念法律事務所 \url{https://t.co/9EozHKSWl9}  ¥\n
  ご相談の問題点・要点を掴み取り(理解力)、解決策を考案し(発想力)、さまざまな見地から実行に移す(瞬発力)の三要素が、今村記念法律事務所に所属する弁護士の基本姿勢です。
\end{itemize}

 ホームページに法律事務所の住所が見当たらないのですが,範囲選択が出来ない色付きの見出しに「理解力,発想力,瞬発力で,真実を追い求める」とあります。

\begin{itemize}
\tightlist
\item
  事務所概要 \textbar{} 今村記念法律事務所 \url{https://t.co/xEouPgL0lZ}  ¥\n
  所在地 ¥\n
  〒101-0051 東京都千代田区神田神保町2-8-3 専修大学8号館1階
\end{itemize}

 変わった名称の弁護士事務所だと思ったのですが,住所が東京都千代田区神田神保町となっていました。銅像になるような創設者が今村という名前なのかと想像しますが,「今村記念法律事務所」ということで,歴史的博物館にも近いものを感じました。

\begin{itemize}
\tightlist
\item
  今村記念法律事務所の由来 \textbar{} 今村記念法律事務所
  \url{https://t.co/vFjOORgTsN} 
  事務所の名に冠してある「今村」とは、事務所のある専修大学でかつて総長を務めた今村力三郎氏からいただいています。
\end{itemize}

 「足尾銅山鉱毒事件」,「幸徳秋水らの大逆事件」に関わっているとあります。半年ほど前になるのか足尾銅山鉱毒事件の義民のような人物のお話を読んだのですが,余り記憶にないものの弁護士のことは話になかったような気がします。

\begin{itemize}
\tightlist
\item
  2021年05月09日15時07分の登録:
  「足尾銅山」を@hirono\_hideki @kk\_hirono @s\_hironoで検索 10件の該当 2021-05-09\_15:06の記録
  \url{https://kk2020-09.blogspot.com/2021/05/hironohidekikkhironoshirono102021-05.html} 
\end{itemize}

2020-01-10 00:42:24
``大貫大八弁護士(故人)と大貫正一弁護士の父娘弁護士は、それだけ義侠心が厚い人物であったのかもしれません。栃木県には足尾銅山の公害問題も歴史にあったはずですが、これを確かに知ったのは山本太郎という政治家が天皇陛下に書簡を手渡そうとした問題でした。''
\url{https://twitter.com/kk\_hirono/status/1215297780349984768} 

 父娘弁護士というのは記憶になかったのですが,弁護士の名前で親子の間違いだとわかりました。見覚えのない名前で大貫といえば,大貫憲介弁護士のことが頭に浮かびました。親子弁護士と栃木県で記憶にあるのは,ジャガイモの報酬で尊属殺違憲判決に尽力した弁護士のことかと思います。

\begin{itemize}
\tightlist
\item
  「父殺しの女性」を救った日本初の法令違憲判決:日経ビジネス電子版
  \url{https://t.co/0hSWiwKnTG} 
  事件から数日後、宇都宮市内で事務所を構える大貫大八弁護士(故人)のところにひとりの依頼人が現れた。殺されたXさんの別居中の妻Yさんだった。依頼内容は娘のAの弁護だった。
\end{itemize}

 団地の写真が印象的に記憶にある記事でした。昭和の終わりから平成の初めでも弁護士というのは,紹介者なしの一見さんはお断りという話が普通になっていました。にわかに信じがたい話ですが,同じ栃木県の足利事件で佐藤博史弁護士が関わったきっかけというのもどうかという気がしていました。

 富山,滋賀に続き栃木が出てきましたが,ここも弁護士鉄道の歴史として,いわくつきの刑事事件や刑事裁判の多いところです。

2020-01-10 00:44:01 ``»
65回 足尾銅山鉱毒事件 ~田中正造と山本太郎の違い~ \textbar{}
日本人のための日本近現代史
\url{https://ameblo.jp/kitatyu79/entry-12108089836.html''} 
\url{https://twitter.com/kk\_hirono/status/1215298186643832833} 

 昨年の2020年1月10日に読んでいた記事のようです。記事を見つけたきっかけは憶えていないですが,かなり長文の詳しい資料を読み込んだような憶えがあります。

\begin{itemize}
\tightlist
\item
  今村元 弁護士 氷見 - Google 検索 \url{https://t.co/9G54KIqvWU} 
  最も的確な検索結果を表示するために、上の 17
  件と似たページは除外されています。 ¥\n
  検索結果をすべて表示するには、ここから再検索してください。
\end{itemize}

 「今村元 弁護士 氷見」というGoogle検索の2ページ目に,この先に結果はないという表示が出てきましたが,やはり富山県弁護士会の会長だった今村元弁護士と氷見強姦冤罪事件を結びつける情報はありませんでした。業者に頼んで削除したのかと前から疑問に思っています。

 今村元弁護士の名前に見覚えはあるのですが,今になって初めて弁護士を辞めた元弁護士に読めるということに気が付き,思い出したのが,その点を指摘されていた佐野元弁護士です。ここ2,3年はすっかり名前を忘れていたようですが,一頃は話題になった弁護士です。

 twilog-serch-post
を佐田元弁護士で実行すると結果が1件しかなく,間違いに気がついたのですが,佐田元弁護士でした。思ったほど数はなかったですが,42件の該当となっています。

\begin{itemize}
\tightlist
\item
  2021年05月09日15時31分の登録:
  「佐田元弁護士」を@hirono\_hideki @kk\_hirono @s\_hironoで検索 42件の該当 2021-05-09\_15:31の記録
  \url{https://kk2020-09.blogspot.com/2021/05/hironohidekikkhironoshirono422021-05.html} 
\end{itemize}

2013-06-17 00:40:48
``【裁判記録は誰のものか】「これは国民の知る権利の問題です」(江川 紹子)
- 個人 - Yahoo!ニュース \url{http://ow.ly/m4Wnx} 
佐田元真己弁護士というのは佐田元という姓らしく佐田元弁護士という表記が多くて、弁護士を辞めた人かと思った。''
\url{https://twitter.com/hirono\_hideki/status/346291274317443072} 

2014-01-08 09:13:31
``代理人の一人で、刑事弁護のプロとして名高い後藤貞人弁護士は、次のように熱く佐田元弁護士を弁護した。
\url{http://ow.ly/smHPm''} 
\url{https://twitter.com/hirono\_hideki/status/420709816924127232} 

 刑事事件になってはいなかったと思いますが,検察が弁護士会に懲戒請求したという珍しいケースでした。

2017-01-27 22:21:02
``2017年01月27日04時17分_誤った佐野元弁護士の検索から佐田元弁護士の横領1年間の業務停止命令懲戒処分の情報を発見
\url{https://www.youtube.com/watch?v=UeQnnchi0jE''} 
\url{https://twitter.com/s\_hirono/status/824970471447801856} 

 佐田元弁護士の横領と懲戒処分のことも思い出していたのですが,検察の懲戒請求が2013年で,横領が2017年というのは間が長く意外でした。2017年1月27日は私が情報を見つけた時点になるのかもしれません。

\begin{quote}
《引用の始まり》
\end{quote}

\begin{quote}
【無罪判決の獲得経験あり】【谷町四丁目駅】【初回相談無料】【無罪判決の獲得経験あり】刑事事件・交通事故・不動産/マンショントラブル・法人倒産お任せください。困っている方の助けになれるよう尽力します。どんな弁護士ですか?◆志・理念━━━━━━━━━━━「困っている人を助けたい。」少しでも多くの方の法的トラブルの解決をお手伝いしたいと思っております。

特に刑事事件に力を入れております。「無実なのに勾留・逮捕されてしまった」「犯罪を犯して逮捕されてしまった」等、刑事事件でお困りの場合はご相談ください。

傷害致死事件の裁判員裁判での無罪判決の獲得経験もあります。

◆経歴・所属━━━━━━━━━━━1960年5月18日生まれ 大阪府大阪市出身1979年3月 大阪府立桜塚高等学校 卒業1985年3月 私立関西大学法学部法律学科 卒業(学術研究会法律相談所、体育会相撲部に所属)1989年11月 司法試験合格1992年4月 弁護士登録1999年4月 さだもと法律事務所 設立2009年4月 弁護士法人さだもと法律事務所設立
\end{quote}

\begin{quote}
《引用の終わり》
\end{quote}

\begin{itemize}
\tightlist
\item
  佐田元 眞己弁護士 さだもと法律事務所 \textbar{}
  ココナラ法律相談 \url{https://legal.coconala.com/lawyers/1771n} 
\end{itemize}

 「まさみ」というふり仮名があったので,同性の別人という可能性も頭をよぎったのですが,プロフィールに「傷害致死事件の裁判員裁判での無罪判決の獲得経験もあります。」と宣伝しているので間違いはなさそうです。

 小さい顔写真がありますが,話題になっていた当時に見かけていた顔写真とはずいぶん違った印象で,優しくものわかりのようさそうな裁判官に見えます。経歴から思い切った実行力はありそうです。復活する強靭さもありそうです。

\begin{quote}
《引用の始まり》
\end{quote}

\begin{quote}
弁護士会によりますと、佐田弁護士は男性の依頼者から預かった約105万円をおととし3月ころから約20回にわたり着服。さらにこの穴埋めをするため、自らが成年後見人として管理している口座から38回にわたり現金を引き出すなどして会わせて約570万円を横領したということです。去年1月自らが裁判所に申告をして横領が発覚し弁済したということです。佐田元弁護士は4年前、取り調べの録音映像をNHKに提供し、目的外使用にあたると指摘される問題を起こしていました。
\end{quote}

\begin{quote}
《引用の終わり》
\end{quote}

\begin{itemize}
\tightlist
\item
  佐田元眞已弁護士(大阪)懲戒処分の要旨 --
  弁護士自治を考える会 \url{https://jlfmt.com/2017/05/30/31314/n} 
\end{itemize}

 「去年1月自らが裁判所に申告をして横領が発覚し弁済したということです。」というのは初めて知ったように思いますが,弁護士の横領事件で初めて見るケースでもあります。刑事事件にはなっていないようですが,発覚を察知したのか改心なのか,詳細はわからないようです。

\begin{itemize}
\tightlist
\item
  佐田元 眞己弁護士 さだもと法律事務所 \textbar{} ココナラ法律相談
  \url{https://legal.coconala.com/lawyers/1771} 
\end{itemize}

 数日前にもココナラ法律相談というのを見かけていたのですが,情報がないとか変わった表示になっていました。

\begin{itemize}
\tightlist
\item
  2021年05月09日15時55分の登録:
  「ココナラ法律相談」を@hirono\_hideki @kk\_hirono @s\_hironoで検索 24件の該当 2021-05-09\_15:55の記録
  \url{https://kk2020-09.blogspot.com/2021/05/hironohidekikkhironoshirono242021-05.html} 
\end{itemize}

2020-12-15 20:19:24 ``新潟県の放火に強い弁護士 \textbar{}
ココナラ法律相談
\url{https://legal.coconala.com/search/result?prefecture\_id=15\&sub\_category\_id=215''} 
\url{https://twitter.com/hirono\_hideki/status/1338805856016551936} 

 ジョークなのかと思ったのですが,リンクが開かれました。

2021-04-10 21:28:05 ``武田 祐介弁護士 武田総合法律事務所 \textbar{}
ココナラ法律相談 \url{https://t.co/MYcC0wAB46} 
私は弁護士の仕事を「過去の事実を探究する職業」だと捉えています。¥\n依頼者の話を伺い、客観的な証拠の積み重ねを行い、できる限り納得される結果へと導いていく作業をするからです。''
\url{https://twitter.com/hirono\_hideki/status/1380860135002673154} 

2021-05-07 03:52:19 ``ご指定のページが見つかりません \textbar{}
ココナラ法律相談 \url{https://t.co/Y2taC49wmK''} 
\url{https://twitter.com/hirono\_hideki/status/1390378911876349953} 

 4月10日には表示されたページを見ていたようですが記憶にありません。次に調べたのが5月7日でした。一昨日になります。

\begin{itemize}
\tightlist
\item
  性的暴行容疑で弁護士を再逮捕|NHK 千葉県のニュース
  \url{https://t.co/jAmx4ziyy6}  ¥\n 05月06日 14時45分 ¥\n ¥\n
  先月、女性を自宅に連れ込んで、性的暴行をしようとしたとして起訴された36歳の弁護士が、別の女性に対する性的暴行の疑いで再逮捕されました。
\end{itemize}

 更新された最新ニュースというのは見当たりませんでしたが,「警察は武田弁護士が容疑を認めているかどうか明らかにしていません。」とNHKのニュースが締めくくられています。ここ数年,3,4年ぐらいでしょうか。よく見かけるようになったパターンです。

 あくまで乏しい断片的な情報をつなぎ合わせた想像の1つですが,コロナ禍で逼迫した飲食店従業員の女性が,お金がありそうな弁護士をターゲットに誘いをかけ,捨て身で挑発をしたという可能性も物語として成立するのかもしれません。

 弁護士が示談に応じていれば不起訴になっていた可能性もあり,弁護士にもそれだけ言い分があったのかもしれないですが,何も伝わっていないようです。当番弁護士というのもあるはずですが,弁護士のツイートで余り話題にもされていませんでした。

 報道には軽傷ともあったと思うのですが,なかには怪我の程度が全治2週間というものがあり,これはかなり大きな怪我になると思います。酔っていたという話も見かけないのですが,警察の情報漏らしをリークとして咎める弁護士業界には皮肉に映る事件でもあります。

\begin{quote}
《引用の始まり》
\end{quote}

\begin{quote}
起訴状などによると、武田被告は今月8日、飲食店従業員の女性(24)を千葉市中央区の自宅に連れ込み、顔を殴ってわいせつな行為をしようとし、女性に全治2週間のけがを負わせたとしている。
\end{quote}

\begin{quote}
《引用の終わり》
\end{quote}

\begin{itemize}
\tightlist
\item
  女性に乱暴容疑で弁護士を起訴 千葉地検 -
  産経ニュース \url{https://www.sankei.com/affairs/news/210429/afr2104290016-n1.htmln} 
\end{itemize}

 どうも4月8日の事件が全治2週間のけがだったようです。「千葉地検は28日、武田容疑者を強制性交等致傷の罪で起訴した。」とあるのですが,「顔を殴ってわいせつな行為をしようとし、」とあるので,こちらも未遂の可能性があるのかもしれません。どちらかが未遂となっていました。

\begin{itemize}
\tightlist
\item
  武田祐介弁護士(千葉)を傷害容疑で逮捕 強制性交等致傷容疑で送検されたとのことなので、弁護士としてあるまじき行為を取ったことは間違いないはずです。
  -- 鎌倉九郎 \url{https://t.co/KLSj7t7pUo} 
\end{itemize}

 上記のページには,11日付の毎日新聞として「顔を複数回殴るなどして性的暴行を加えようとしたとしている。逃げ出した女性の110番通報で駆け付けた同署員が自宅近くにいた武田容疑者を傷害容疑で緊急逮捕した。」とあります。少なくともこちらは未遂のようです。

 傷害というのは事実として間違いなさそうですが,「性的暴行を加えようとした」という主観面が,強制性交等致傷という重い罪の容疑になっているようです。無期懲役もある大罪なのかもしれません。前の強姦罪でも3年以上の有期懲役などとなっていたはずです。

 今は違っているのかもしれないですが,強姦罪は親告罪,強姦致傷罪は非親告罪という大きな違いもありました。

 この千葉の弁護士の強制性交等致傷事件については,この事件で思い出した我孫子警察署の警察官の事件と一緒に取り上げておきたいと考えていました。弁護士が警察に掛けた過大な負担や,水面下での攻防のようなものを想像していました。

\hypertarget{section-11}{%
\paragraph{}\label{section-11}}

\hypertarget{ux5cb8ux672cux5b66ux5f01ux8b77ux58ebux7b2cux4e00ux6771ux4eacux5f01ux8b77ux58ebux4f1aux6240ux5c5e}{%
\subsubsection{岸本学弁護士(第一東京弁護士会所属)}\label{ux5cb8ux672cux5b66ux5f01ux8b77ux58ebux7b2cux4e00ux6771ux4eacux5f01ux8b77ux58ebux4f1aux6240ux5c5e}}

\hypertarget{ux75f4ux6f22ux88abux7591ux8005ux306eux793aux8ac7ux91d130ux4e07ux5186ux3092ux63d0ux793aux3057ux6700ux7d42ux7684ux306bux306f9ux5272ux4ee5ux4e0aux306eux30b1ux30fcux30b9ux3067210ux500dux306eux793aux8ac7ux91d1ux984dux3067ux6c7aux7740ux3059ux308bux3068ux3044ux3046ux4fe1ux3058ux304cux305fux3044ux5cb8ux672cux5b66ux5f01ux8b77ux58ebux306eux30c4ux30a4ux30fcux30c8ux3068ux4ed6ux306eux5f01ux8b77ux58ebux306eux53cdux5fdc1}{%
\paragraph{痴漢被疑者の示談金,30万円を提示し,最終的には9割以上のケースで2\textasciitilde10倍の示談金額で決着する,という信じがたい岸本学弁護士のツイートと,他の弁護士の反応(1)}\label{ux75f4ux6f22ux88abux7591ux8005ux306eux793aux8ac7ux91d130ux4e07ux5186ux3092ux63d0ux793aux3057ux6700ux7d42ux7684ux306bux306f9ux5272ux4ee5ux4e0aux306eux30b1ux30fcux30b9ux3067210ux500dux306eux793aux8ac7ux91d1ux984dux3067ux6c7aux7740ux3059ux308bux3068ux3044ux3046ux4fe1ux3058ux304cux305fux3044ux5cb8ux672cux5b66ux5f01ux8b77ux58ebux306eux30c4ux30a4ux30fcux30c8ux3068ux4ed6ux306eux5f01ux8b77ux58ebux306eux53cdux5fdc1}}

\begin{itemize}
\tightlist
\item
  〉〉〉 Linux Emacs: 2021/05/26 10:33:08 〉〉〉
\end{itemize}

:CATEGORIES: @kanazawabengosi \#金沢弁護士会 @JFBAsns
\#日本弁護士連合会(日弁連) \#法務省 @MOJ\_HOUMU \#岸本学弁護士
\#小倉秀夫弁護士

 最初,小倉秀夫弁護士のタイムラインで岸本学弁護士のツイートを引用した小倉秀夫弁護士の弁護士のツイートがきっかけだったのですが,いろいろと遡っていくと,最終的に数日前に見かけていた,見出しに取り込んだ内容の岸本学弁護士のツイートに行き着きました。

 うまく行くかわからないですが,連続テレビ小説あまちゃん,で印象的だった逆回転時計のような感じで岸本学弁護士の件のツイートをその反応を交えながらリツイートしていきたいと思います。なかなか機序は保てない予想です。

〉〉〉 kk\_hironoのリツイート 〉〉〉

\begin{itemize}
\tightlist
\item
  RT
  kk\_hirono(刑事告発・非常上告_金沢地方検察庁御中)|9jtCdbGf3lih8Fe(弁護士
  岸本 学) 日時:2021-05-26 10:37/2021/05/21 13:32 URL:
  \url{https://twitter.com/kk\_hirono/status/1397366368199004160} 
  \url{https://twitter.com/9jtCdbGf3lih8Fe/status/1395598298300354560} 
  \textgreater{} 痴漢被疑者の弁護士
  「示談金額として、30万円を提示します。被害の程度は重くなく、慰謝料請求の裁判例からいっても、刑事処罰時の罰金額からいっても、この金額が妥当と考えます」
  わい 「あ、その金額だとお断りします」 以上。 だいたいこんな感じ。
\end{itemize}

〉〉〉 kk\_hironoのリツイート 〉〉〉

\begin{itemize}
\tightlist
\item
  RT
  kk\_hirono(刑事告発・非常上告_金沢地方検察庁御中)|9jtCdbGf3lih8Fe(弁護士
  岸本 学) 日時:2021-05-26 10:38/2021/05/21 15:48 URL:
  \url{https://twitter.com/kk\_hirono/status/1397366379930480643} 
  \url{https://twitter.com/9jtCdbGf3lih8Fe/status/1395632513859559426} 
  \textgreater{} 最終的には、9割以上のケースで
  2\textasciitilde10倍の示談金額で決着する。
\end{itemize}

〉〉〉 kk\_hironoのリツイート 〉〉〉

\begin{itemize}
\tightlist
\item
  RT
  kk\_hirono(刑事告発・非常上告_金沢地方検察庁御中)|bengoshimentaru(家系弁護士)
  日時:2021-05-26 10:39/2021/05/21 23:27 URL:
  \url{https://twitter.com/kk\_hirono/status/1397366649112600580} 
  \url{https://twitter.com/bengoshimentaru/status/1395748062790705156} 
  \textgreater{} 悪質な強姦が300万は安いと思うけど
  ちょっと触れた迷惑防止条例違反は10万円でええんやないかと
  \url{https://t.co/STsF9DIlyo} 
\end{itemize}

〉〉〉 kk\_hironoのリツイート 〉〉〉

\begin{itemize}
\tightlist
\item
  RT
  kk\_hirono(刑事告発・非常上告_金沢地方検察庁御中)|SUGIYAMA766(弁護士杉山程彦5東慶應三丁目)
  日時:2021-05-26 10:39/2021/05/22 09:11 URL:
  \url{https://twitter.com/kk\_hirono/status/1397366679793901569} 
  \url{https://twitter.com/SUGIYAMA766/status/1395895135003549696} 
  \textgreater{} @bengoshimentaru
  岸本弁護士のように言えば世間受けはするだろうけど、慰謝料インフレ化させるのもよくないと思う。
  私なら、被害者代理人に岸本弁護士がついたら、交渉経過を検察に上申します。
  これでは交渉になりません。
\end{itemize}

〉〉〉 kk\_hironoのリツイート 〉〉〉

\begin{itemize}
\tightlist
\item
  RT
  kk\_hirono(刑事告発・非常上告_金沢地方検察庁御中)|SUGIYAMA766(弁護士杉山程彦5東慶應三丁目)
  日時:2021-05-26 10:39/2021/05/22 09:35 URL:
  \url{https://twitter.com/kk\_hirono/status/1397366792545210370} 
  \url{https://twitter.com/SUGIYAMA766/status/1395901059139211267} 
  \textgreater{} @bengoshimentaru
  私なら20万提示して、Maxでも30万だなあ。
  それ以上なら、被疑者にも相場を青天井にするのはよくないことだと説得する。
  ヘタレな被疑者で示談のためならどれだけでもカネ払うというなら仕方がないけど。
\end{itemize}

〉〉〉 kk\_hironoのリツイート 〉〉〉

\begin{itemize}
\tightlist
\item
  RT
  kk\_hirono(刑事告発・非常上告_金沢地方検察庁御中)|relaxpop(おかまつさん)
  日時:2021-05-26 10:40/2021/05/22 18:08 URL:
  \url{https://twitter.com/kk\_hirono/status/1397366976817729539} 
  \url{https://twitter.com/relaxpop/status/1396030160420233217} 
  \textgreater{}
  素朴な疑問として、示談しなかった結果、刑罰を受けず、かつ、被害者が示談金も受け取れない場合は、どのようにお考えになってらっしゃるのでしょうか?
  \url{https://t.co/AAMM7R4wUj} 
\end{itemize}

〉〉〉 kk\_hironoのリツイート 〉〉〉

\begin{itemize}
\tightlist
\item
  RT
  kk\_hirono(刑事告発・非常上告_金沢地方検察庁御中)|9jtCdbGf3lih8Fe(弁護士
  岸本 学) 日時:2021-05-26 10:40/2021/05/22 18:14 URL:
  \url{https://twitter.com/kk\_hirono/status/1397366988607942658} 
  \url{https://twitter.com/9jtCdbGf3lih8Fe/status/1396031646650933254} 
  \textgreater{} @relaxpop ここで全部説明する積りは無いのですが、当然、
  そのリスクを織り込んで、見極める方策を取りながら、示談交渉をするのです。
  失礼ながら、とても初歩的なご質問と思いますが。
\end{itemize}

〉〉〉 kk\_hironoのリツイート 〉〉〉

\begin{itemize}
\tightlist
\item
  RT
  kk\_hirono(刑事告発・非常上告_金沢地方検察庁御中)|9jtCdbGf3lih8Fe(弁護士
  岸本 学) 日時:2021-05-26 10:40/2021/05/22 17:56 URL:
  \url{https://twitter.com/kk\_hirono/status/1397367014771990528} 
  \url{https://twitter.com/9jtCdbGf3lih8Fe/status/1396027259585134594} 
  \textgreater{} @relaxpop え?
  もちろん、払わないのもご自由だと思いますけど。
  被疑者は示談せずに刑罰を受ける選択も当然あると思います。
\end{itemize}

〉〉〉 kk\_hironoのリツイート 〉〉〉

\begin{itemize}
\tightlist
\item
  RT
  kk\_hirono(刑事告発・非常上告_金沢地方検察庁御中)|relaxpop(おかまつさん)
  日時:2021-05-26 10:40/2021/05/22 16:26 URL:
  \url{https://twitter.com/kk\_hirono/status/1397367035252797445} 
  \url{https://twitter.com/relaxpop/status/1396004550646013952} 
  \textgreater{}
  被害実態が重いのであれば、民事裁判で主張・立証するしかないですよね。被疑者段階の示談交渉にでは、期間の制約がありますし、被害者側の立証にも限界があるでしょう。それなりの金額を支払うのに、本人が言っているからだけでは、払えない気がします。
  \url{https://t.co/nAmBonNnpu} 
\end{itemize}

〉〉〉 kk\_hironoのリツイート 〉〉〉

\begin{itemize}
\tightlist
\item
  RT
  kk\_hirono(刑事告発・非常上告_金沢地方検察庁御中)|9jtCdbGf3lih8Fe(弁護士
  岸本 学) 日時:2021-05-26 10:40/2021/05/22 01:48 URL:
  \url{https://twitter.com/kk\_hirono/status/1397367060435378176} 
  \url{https://twitter.com/9jtCdbGf3lih8Fe/status/1395783578420604930} 
  \textgreater{}
  被害が「その場でちょっと触っただけ」に留まると、思っているとすれば、そういう意見にもなるでしょうね。
  被害実態が見えていないのではないでしょうか。
\end{itemize}

〉〉〉 kk\_hironoのリツイート 〉〉〉

\begin{itemize}
\tightlist
\item
  RT
  kk\_hirono(刑事告発・非常上告_金沢地方検察庁御中)|9jtCdbGf3lih8Fe(弁護士
  岸本 学) 日時:2021-05-26 10:40/2021/05/21 23:37 URL:
  \url{https://twitter.com/kk\_hirono/status/1397367098284773377} 
  \url{https://twitter.com/9jtCdbGf3lih8Fe/status/1395750608137691138} 
  \textgreater{} 「10万円でええんやないかと」 とのことです。
  これが日本の弁護士のご意見。 \url{https://t.co/7DbY5kA1GI} 
\end{itemize}

〉〉〉 kk\_hironoのリツイート 〉〉〉

\begin{itemize}
\tightlist
\item
  RT
  kk\_hirono(刑事告発・非常上告_金沢地方検察庁御中)|relaxpop(おかまつさん)
  日時:2021-05-26 10:41/2021/05/22 18:16 URL:
  \url{https://twitter.com/kk\_hirono/status/1397367274609074178} 
  \url{https://twitter.com/relaxpop/status/1396032171786182660} 
  \textgreater{}
  そうだとすると、先生が想定されているような金額は、到底無理なような気がいたします。私の経験不足だけかもしれませんが。
  \url{https://t.co/JkeRRcnX7F} 
\end{itemize}

〉〉〉 kk\_hironoのリツイート 〉〉〉

\begin{itemize}
\tightlist
\item
  RT
  kk\_hirono(刑事告発・非常上告_金沢地方検察庁御中)|9jtCdbGf3lih8Fe(弁護士
  岸本 学) 日時:2021-05-26 10:41/2021/05/22 18:26 URL:
  \url{https://twitter.com/kk\_hirono/status/1397367283819810816} 
  \url{https://twitter.com/9jtCdbGf3lih8Fe/status/1396034717598707716} 
  \textgreater{} @relaxpop
  先生のご理解を得ようとは思いませんし、ここで全部説明する気はさらさらありませんが、
  私は空想や理論を話しているのではなく、実務経験をお話ししてます。
  先生も、痴漢の被害者側代理人で示談交渉をされれば、すぐに分かると思いますよ。
\end{itemize}

〉〉〉 kk\_hironoのリツイート 〉〉〉

\begin{itemize}
\tightlist
\item
  RT
  kk\_hirono(刑事告発・非常上告_金沢地方検察庁御中)|relaxpop(おかまつさん)
  日時:2021-05-26 10:41/2021/05/22 18:31 URL:
  \url{https://twitter.com/kk\_hirono/status/1397367309287624705} 
  \url{https://twitter.com/relaxpop/status/1396036032680763400} 
  \textgreater{} @9jtCdbGf3lih8Fe
  先生のようなお考えをお持ちの被害者代理人とは、これまで、交渉したことがなく、単純な疑問で質問をさせていただきました。失礼な質問で、ご気分を害されたならお詫びいたします。
\end{itemize}

〉〉〉 kk\_hironoのリツイート 〉〉〉

\begin{itemize}
\tightlist
\item
  RT
  kk\_hirono(刑事告発・非常上告_金沢地方検察庁御中)|9jtCdbGf3lih8Fe(弁護士
  岸本 学) 日時:2021-05-26 10:41/2021/05/22 18:41 URL:
  \url{https://twitter.com/kk\_hirono/status/1397367334621224960} 
  \url{https://twitter.com/9jtCdbGf3lih8Fe/status/1396038479042080769} 
  \textgreater{} @relaxpop すぐわかると思いますよ。
  被害者側が「300万円でなければ示談しない。お嫌なら刑罰を受けてください」と
  言って来たときに、被疑者本人がどう考えるか、代理人としてどう動くか、ということです。
  示談を断念することも選択肢でしょうね。
  こちらはむやみに大きい金額をいっているわけではない。
\end{itemize}

〉〉〉 kk\_hironoのリツイート 〉〉〉

\begin{itemize}
\tightlist
\item
  RT
  kk\_hirono(刑事告発・非常上告_金沢地方検察庁御中)|relaxpop(おかまつさん)
  日時:2021-05-26 10:42/2021/05/22 18:46 URL:
  \url{https://twitter.com/kk\_hirono/status/1397367467765235712} 
  \url{https://twitter.com/relaxpop/status/1396039877037215746} 
  \textgreater{}
  被疑者弁護人として、裁判例と乖離した示談は受けないとおもいます。示談経緯報告書、資料、供託で終わりが多いのではないでしょうか。示談金額などについては、被害者代理人、刑事弁護人の間で勉強会などができるといいですね。
  \url{https://t.co/Rb95DMBZ0x} 
\end{itemize}

〉〉〉 kk\_hironoのリツイート 〉〉〉

\begin{itemize}
\item
  RT
  kk\_hirono(刑事告発・非常上告_金沢地方検察庁御中)|9jtCdbGf3lih8Fe(弁護士
  岸本 学) 日時:2021-05-26 10:42/2021/05/22 18:56 URL:
  \url{https://twitter.com/kk\_hirono/status/1397367484445970434} 
  \url{https://twitter.com/9jtCdbGf3lih8Fe/status/1396042280717352961} 
  \textgreater{} @relaxpop
  被疑者弁護人が受け入れるかどうかではなく、被疑者本人の意向が全てと思いますが。
  刑罰を受けるのも、サイフが痛むのも被疑者ですから。
  すぐわかると思いますが、条例違反で自白していれば、検察庁は示談成立しないと、被疑者が何をやっても略式で落とします。
\item
  〈〈〈 2021/05/26 10:43:16 Linux Emacs: 〈〈〈
\end{itemize}

\hypertarget{ux75f4ux6f22ux88abux7591ux8005ux306eux793aux8ac7ux91d130ux4e07ux5186ux3092ux63d0ux793aux3057ux6700ux7d42ux7684ux306bux306f9ux5272ux4ee5ux4e0aux306eux30b1ux30fcux30b9ux3067210ux500dux306eux793aux8ac7ux91d1ux984dux3067ux6c7aux7740ux3059ux308bux3068ux3044ux3046ux4fe1ux3058ux304cux305fux3044ux5cb8ux672cux5b66ux5f01ux8b77ux58ebux306eux30c4ux30a4ux30fcux30c8ux3068ux4ed6ux306eux5f01ux8b77ux58ebux306eux53cdux5fdc2}{%
\paragraph{痴漢被疑者の示談金,30万円を提示し,最終的には9割以上のケースで2\textasciitilde10倍の示談金額で決着する,という信じがたい岸本学弁護士のツイートと,他の弁護士の反応(2)}\label{ux75f4ux6f22ux88abux7591ux8005ux306eux793aux8ac7ux91d130ux4e07ux5186ux3092ux63d0ux793aux3057ux6700ux7d42ux7684ux306bux306f9ux5272ux4ee5ux4e0aux306eux30b1ux30fcux30b9ux3067210ux500dux306eux793aux8ac7ux91d1ux984dux3067ux6c7aux7740ux3059ux308bux3068ux3044ux3046ux4fe1ux3058ux304cux305fux3044ux5cb8ux672cux5b66ux5f01ux8b77ux58ebux306eux30c4ux30a4ux30fcux30c8ux3068ux4ed6ux306eux5f01ux8b77ux58ebux306eux53cdux5fdc2}}

\begin{itemize}
\tightlist
\item
  〉〉〉 Linux Emacs: 2021/05/26 10:45:05 〉〉〉
\end{itemize}

:CATEGORIES: @kanazawabengosi \#金沢弁護士会 @JFBAsns
\#日本弁護士連合会(日弁連) \#法務省 @MOJ\_HOUMU \#岸本学弁護士
\#示談金 \#性犯罪 \#検察官

〉〉〉 kk\_hironoのリツイート 〉〉〉

\begin{itemize}
\tightlist
\item
  RT
  kk\_hirono(刑事告発・非常上告_金沢地方検察庁御中)|jmjhjmwtad(7286)
  日時:2021-05-26 10:46/2021/05/22 18:59 URL:
  \url{https://twitter.com/kk\_hirono/status/1397368398909100034} 
  \url{https://twitter.com/jmjhjmwtad/status/1396043013751775232} 
  \textgreater{}
  実際、私が、数年前に刑事弁護やった痴漢(迷惑防止条例違反)のケースで、示談不成立でも不起訴になったケースがあるんよな。
  何をやっても略式で落とす? \url{https://t.co/QlR4WCkt5N} 
\end{itemize}

〉〉〉 kk\_hironoのリツイート 〉〉〉

\begin{itemize}
\tightlist
\item
  RT
  kk\_hirono(刑事告発・非常上告_金沢地方検察庁御中)|9jtCdbGf3lih8Fe(弁護士
  岸本 学) 日時:2021-05-26 10:46/2021/05/22 19:00 URL:
  \url{https://twitter.com/kk\_hirono/status/1397368432438374402} 
  \url{https://twitter.com/9jtCdbGf3lih8Fe/status/1396043372796649477} 
  \textgreater{} @jmjhjmwtad じゃ、不起訴を期待なさったらよいのでは?
\end{itemize}

〉〉〉 kk\_hironoのリツイート 〉〉〉

\begin{itemize}
\tightlist
\item
  RT
  kk\_hirono(刑事告発・非常上告_金沢地方検察庁御中)|jmjhjmwtad(7286)
  日時:2021-05-26 10:46/2021/05/22 19:01 URL:
  \url{https://twitter.com/kk\_hirono/status/1397368441749708802} 
  \url{https://twitter.com/jmjhjmwtad/status/1396043648307924996} 
  \textgreater{} @9jtCdbGf3lih8Fe
  何を根拠に何をしても略式で落とすと言っておられるんですかね?
\end{itemize}

〉〉〉 kk\_hironoのリツイート 〉〉〉

\begin{itemize}
\tightlist
\item
  RT
  kk\_hirono(刑事告発・非常上告_金沢地方検察庁御中)|9jtCdbGf3lih8Fe(弁護士
  岸本 学) 日時:2021-05-26 10:46/2021/05/22 19:08 URL:
  \url{https://twitter.com/kk\_hirono/status/1397368459730718722} 
  \url{https://twitter.com/9jtCdbGf3lih8Fe/status/1396045416949497861} 
  \textgreater{} @jmjhjmwtad
  加害者の弁護人に対してと、被害者の代理人に対してとではね、
  検事が行うコミュニケーションの内容が違うのですよ。
\end{itemize}

〉〉〉 kk\_hironoのリツイート 〉〉〉

\begin{itemize}
\tightlist
\item
  RT
  kk\_hirono(刑事告発・非常上告_金沢地方検察庁御中)|jmjhjmwtad(7286)
  日時:2021-05-26 10:46/2021/05/22 19:11 URL:
  \url{https://twitter.com/kk\_hirono/status/1397368474268168196} 
  \url{https://twitter.com/jmjhjmwtad/status/1396046090600816647} 
  \textgreater{} @9jtCdbGf3lih8Fe 具体的にどう違うんですか?
\end{itemize}

〉〉〉 kk\_hironoのリツイート 〉〉〉

\begin{itemize}
\tightlist
\item
  RT
  kk\_hirono(刑事告発・非常上告_金沢地方検察庁御中)|9jtCdbGf3lih8Fe(弁護士
  岸本 学) 日時:2021-05-26 10:46/2021/05/22 19:13 URL:
  \url{https://twitter.com/kk\_hirono/status/1397368490856620034} 
  \url{https://twitter.com/9jtCdbGf3lih8Fe/status/1396046525298483203} 
  \textgreater{} @jmjhjmwtad どうでしょうね。
  ご自身で確かめたらいかがですか?
\end{itemize}

〉〉〉 kk\_hironoのリツイート 〉〉〉

\begin{itemize}
\tightlist
\item
  RT
  kk\_hirono(刑事告発・非常上告_金沢地方検察庁御中)|jmjhjmwtad(7286)
  日時:2021-05-26 10:46/2021/05/22 19:17 URL:
  \url{https://twitter.com/kk\_hirono/status/1397368525648392193} 
  \url{https://twitter.com/jmjhjmwtad/status/1396047587245969411} 
  \textgreater{}
  私は「痴漢事件において示談不成立の場合検察官は被疑者が何をやっても略式で落とす」という貴職の見解の根拠を質問させて頂きました。
  当然「検察内部の運用を検事から聞いて知った」など、相応の根拠があって、ツイッターという公開の場で発言されていると思います。
  根拠を教えて頂けますか? \url{https://t.co/C9L6ZA5EZK} 
\end{itemize}

〉〉〉 kk\_hironoのリツイート 〉〉〉

\begin{itemize}
\tightlist
\item
  RT
  kk\_hirono(刑事告発・非常上告_金沢地方検察庁御中)|9jtCdbGf3lih8Fe(弁護士
  岸本 学) 日時:2021-05-26 10:46/2021/05/22 19:19 URL:
  \url{https://twitter.com/kk\_hirono/status/1397368604895551490} 
  \url{https://twitter.com/9jtCdbGf3lih8Fe/status/1396048044295086081} 
  \textgreater{} @jmjhjmwtad
  諸事情により、今までお答えした以上のお答えはお断りします。
  別に回答の義務は負っていません。
\end{itemize}

〉〉〉 kk\_hironoのリツイート 〉〉〉

\begin{itemize}
\tightlist
\item
  RT
  kk\_hirono(刑事告発・非常上告_金沢地方検察庁御中)|relaxpop(おかまつさん)
  日時:2021-05-26 10:49/2021/05/22 21:32 URL:
  \url{https://twitter.com/kk\_hirono/status/1397369331864932357} 
  \url{https://twitter.com/relaxpop/status/1396081502429929473} 
  \textgreater{}
  そうなんだ。被害者側、被疑者弁護人、両方やるけど、検察官のコミュニケーションが違うと感じたことは、ないけど、それは、私が鈍感だからだろうか・・・
  \url{https://t.co/DYvbSwsOcO} 
\end{itemize}

〉〉〉 kk\_hironoのリツイート 〉〉〉

\begin{itemize}
\tightlist
\item
  RT
  kk\_hirono(刑事告発・非常上告_金沢地方検察庁御中)|NwFle6q9vQTXb4q(Deep
  sea creatures) 日時:2021-05-26 10:51/2021/05/23 02:10 URL:
  \url{https://twitter.com/kk\_hirono/status/1397369678645796868} 
  \url{https://twitter.com/NwFle6q9vQTXb4q/status/1396151516663189510} 
  \textgreater{} #人権派の闇
  たくさんの弁護士さんが相場は数十万円と言っている中で、起訴されることを突きつけて相場を遥かに超える300万円を要求するんですね。
  \url{https://t.co/tSlJC6BL6x} 
\end{itemize}

〉〉〉 kk\_hironoのリツイート 〉〉〉

\begin{itemize}
\tightlist
\item
  RT
  kk\_hirono(刑事告発・非常上告_金沢地方検察庁御中)|SUGIYAMA766(弁護士杉山程彦5東慶應三丁目)
  日時:2021-05-26 10:51/2021/05/23 02:06 URL:
  \url{https://twitter.com/kk\_hirono/status/1397369739056271362} 
  \url{https://twitter.com/SUGIYAMA766/status/1396150381185363973} 
  \textgreater{} 絶対に許せないから刑事処罰してほしい。
  刑事事件が終わるまでは、示談交渉は一切しない。 天晴れと思う。
  300万円で示談したい。 不道徳だ。やめろ。 \url{https://t.co/2F98MKu2aJ} 
\end{itemize}

〉〉〉 kk\_hironoのリツイート 〉〉〉

\begin{itemize}
\tightlist
\item
  RT
  kk\_hirono(刑事告発・非常上告_金沢地方検察庁御中)|9jtCdbGf3lih8Fe(弁護士
  岸本 学) 日時:2021-05-26 10:54/2021/05/23 23:10 URL:
  \url{https://twitter.com/kk\_hirono/status/1397370622888484865} 
  \url{https://twitter.com/9jtCdbGf3lih8Fe/status/1396468519575822347} 
  \textgreater{} ツイッタランドでは、
  「弁護士アカウント」は、まず疑いの目で見られている、
  と言うのは事実のようです。
\end{itemize}

〉〉〉 kk\_hironoのリツイート 〉〉〉

\begin{itemize}
\tightlist
\item
  RT
  kk\_hirono(刑事告発・非常上告_金沢地方検察庁御中)|9jtCdbGf3lih8Fe(弁護士
  岸本 学) 日時:2021-05-26 10:54/2021/05/23 12:27 URL:
  \url{https://twitter.com/kk\_hirono/status/1397370633088995331} 
  \url{https://twitter.com/9jtCdbGf3lih8Fe/status/1396306729860669443} 
  \textgreater{}
  私の場合、痴漢・盗撮の被害者から、示談交渉を年間40\textsubscript{50件受任し、示談不成立はそのうち1}2件なのですが、
  その実情についてツイートすると、
  「想像」でそれを否定してくる弁護士アカが沸いてくるのが片腹痛い。
\end{itemize}

〉〉〉 kk\_hironoのリツイート 〉〉〉

\begin{itemize}
\tightlist
\item
  RT
  kk\_hirono(刑事告発・非常上告_金沢地方検察庁御中)|9jtCdbGf3lih8Fe(弁護士
  岸本 学) 日時:2021-05-26 10:55/2021/05/23 02:29 URL:
  \url{https://twitter.com/kk\_hirono/status/1397370654643474432} 
  \url{https://twitter.com/9jtCdbGf3lih8Fe/status/1396156206557720576} 
  \textgreater{} 条例違反の痴漢で、
  被害者か「300万円」を請求する場合、それは「金銭への欲求」が強いからではなく、
  「被害感情」が強いためです。
  そのことは、経験上、検事など関係者にそう説明すれば理解してもらえます。
\end{itemize}

〉〉〉 kk\_hironoのリツイート 〉〉〉

\begin{itemize}
\tightlist
\item
  RT
  kk\_hirono(刑事告発・非常上告_金沢地方検察庁御中)|9jtCdbGf3lih8Fe(弁護士
  岸本 学) 日時:2021-05-26 10:55/2021/05/23 02:42 URL:
  \url{https://twitter.com/kk\_hirono/status/1397370665053790208} 
  \url{https://twitter.com/9jtCdbGf3lih8Fe/status/1396159428689227779} 
  \textgreater{} そして被害者は、本
  当に「300万円」貰おうとほ思っておらず、
  被疑者の刑事処罰が基本方針である場合が多いです。
\end{itemize}

〉〉〉 kk\_hironoのリツイート 〉〉〉

\begin{itemize}
\tightlist
\item
  RT
  kk\_hirono(刑事告発・非常上告_金沢地方検察庁御中)|9jtCdbGf3lih8Fe(弁護士
  岸本 学) 日時:2021-05-26 10:55/2021/05/23 07:13 URL:
  \url{https://twitter.com/kk\_hirono/status/1397370682690838529} 
  \url{https://twitter.com/9jtCdbGf3lih8Fe/status/1396227803708678145} 
  \textgreater{} つまり、条例違反の痴漢で「300万円」請求する被害者は、
  「それくらい貰わなければ、犯人を許すことができない」 と思っています。
  全然普通だと思うのですけど。
  むしろ金銭賠償を重視する人は、もっとリスクの少ない額になります。
\end{itemize}

〉〉〉 kk\_hironoのリツイート 〉〉〉

\begin{itemize}
\tightlist
\item
  RT
  kk\_hirono(刑事告発・非常上告_金沢地方検察庁御中)|9jtCdbGf3lih8Fe(弁護士
  岸本 学) 日時:2021-05-26 10:55/2021/05/23 02:54 URL:
  \url{https://twitter.com/kk\_hirono/status/1397370702433456128} 
  \url{https://twitter.com/9jtCdbGf3lih8Fe/status/1396162560022388736} 
  \textgreater{} 痴漢の示談交渉で、被害者の「行動原理」は、
  被害「感情」であって、 被害「論理」ではないことを、
  全ての関係者が明記すべきです。
\end{itemize}

〉〉〉 kk\_hironoのリツイート 〉〉〉

\begin{itemize}
\tightlist
\item
  RT
  kk\_hirono(刑事告発・非常上告_金沢地方検察庁御中)|9jtCdbGf3lih8Fe(弁護士
  岸本 学) 日時:2021-05-26 10:55/2021/05/23 01:48 URL:
  \url{https://twitter.com/kk\_hirono/status/1397370721844617216} 
  \url{https://twitter.com/9jtCdbGf3lih8Fe/status/1396146009063837698} 
  \textgreater{} 痴漢事件の示談交渉に登場するプレーヤーは、
  ①被害者(代理人含む) ②被疑者 ③被疑者代理人 ④検事 です。
  ②被疑者と③被疑者代理人は、別々のプレーヤーと、私は見ています。
  私は、各プレーヤーの意向・方針・環境と、またその変化に、細心の注意を払います。
\end{itemize}

〉〉〉 kk\_hironoのリツイート 〉〉〉

\begin{itemize}
\tightlist
\item
  RT
  kk\_hirono(刑事告発・非常上告_金沢地方検察庁御中)|9jtCdbGf3lih8Fe(弁護士
  岸本 学) 日時:2021-05-26 10:55/2021/05/23 01:53 URL:
  \url{https://twitter.com/kk\_hirono/status/1397370740538634240} 
  \url{https://twitter.com/9jtCdbGf3lih8Fe/status/1396147244047364096} 
  \textgreater{} 私の場合、 痴漢の示談交渉で、
  被疑者の代理人が「私選」であるか、
  被疑者勾留中などで「国選弁護人」であるかによって、
  対応方針に影響が出ます。
\end{itemize}

〉〉〉 kk\_hironoのリツイート 〉〉〉

\begin{itemize}
\tightlist
\item
  RT
  kk\_hirono(刑事告発・非常上告_金沢地方検察庁御中)|9jtCdbGf3lih8Fe(弁護士
  岸本 学) 日時:2021-05-26 10:55/2021/05/24 17:22 URL:
  \url{https://twitter.com/kk\_hirono/status/1397370875834368001} 
  \url{https://twitter.com/9jtCdbGf3lih8Fe/status/1396743461353652225} 
  \textgreater{} 「110番」通報の練習ができる番号 て、あると良いのにな。
\end{itemize}

〉〉〉 kk\_hironoのリツイート 〉〉〉

\begin{itemize}
\tightlist
\item
  RT
  kk\_hirono(刑事告発・非常上告_金沢地方検察庁御中)|9jtCdbGf3lih8Fe(弁護士
  岸本 学) 日時:2021-05-26 10:55/2021/05/24 20:01 URL:
  \url{https://twitter.com/kk\_hirono/status/1397370899985100802} 
  \url{https://twitter.com/9jtCdbGf3lih8Fe/status/1396783492415909893} 
  \textgreater{}
  「匿名」かつ「集団」で、特定のターゲットにイジメを行うような連中。
  え? はい、弁護士ですよ、奴らは。
\end{itemize}

〉〉〉 kk\_hironoのリツイート 〉〉〉

\begin{itemize}
\tightlist
\item
  RT
  kk\_hirono(刑事告発・非常上告_金沢地方検察庁御中)|TomCat38386600(TomCat🇯🇵)
  日時:2021-05-26 10:56/2021/05/23 15:52 URL:
  \url{https://twitter.com/kk\_hirono/status/1397371058848628736} 
  \url{https://twitter.com/TomCat38386600/status/1396358303702159362} 
  \textgreater{} 弁護士どもはボロ儲けやなぁ。
  そりゃあ、「証拠不要で女の自己申告だけで痴漢被害を認めろ!」
  と発狂するアホ弁護士がウジヤウジヤ居るのも理解できるわ。
  \url{https://t.co/UOGS7vV6YZ} 
\end{itemize}

〉〉〉 kk\_hironoのリツイート 〉〉〉

\begin{itemize}
\tightlist
\item
  RT
  kk\_hirono(刑事告発・非常上告_金沢地方検察庁御中)|SUGIYAMA766(弁護士杉山程彦5東慶應三丁目)
  日時:2021-05-26 10:56/2021/05/23 14:19 URL:
  \url{https://twitter.com/kk\_hirono/status/1397371126074908673} 
  \url{https://twitter.com/SUGIYAMA766/status/1396335033485447173} 
  \textgreater{} これ事実ならボロい商売だなあ。
  刑事事件被害者側で尊敬する仕事は、警察がまともに取り扱わない事件を証拠そろえて告訴して、警察検察が動くように説得する仕事だ。
  すでに被疑者が逮捕されている事件なんておいしいところのつまみ食いだは。
  \url{https://t.co/1O8VlkHrpK} 
\end{itemize}

〉〉〉 kk\_hironoのリツイート 〉〉〉

\begin{itemize}
\item
  RT
  kk\_hirono(刑事告発・非常上告_金沢地方検察庁御中)|bengoshimentaru(家系弁護士)
  日時:2021-05-26 10:57/2021/05/23 12:30 URL:
  \url{https://twitter.com/kk\_hirono/status/1397371168974204936} 
  \url{https://twitter.com/bengoshimentaru/status/1396307446092550145} 
  \textgreater{} 被害者側って、報酬体系どんな感じですか?
  \url{https://t.co/je6ppTrQj3} 
\item
  〈〈〈 2021/05/26 10:57:27 Linux Emacs: 〈〈〈
\end{itemize}

\hypertarget{ux5cb8ux672cux5148ux751fux306fux4ed6ux306eux5f01ux8b77ux58ebux306bux5ac9ux59acux3055ux308cux3066ux3044ux308bux304cux6545ux306bux653bux6483ux3055ux308cux3066ux3044ux308bux3068ux3044ux3046ux81eaux5df1ux8a8dux8b58ux306aux3093ux3060ux308dux3046ux304bux3068ux3044ux3046ux5c0fux5009ux79c0ux592bux5f01ux8b77ux58ebux306eux30c4ux30a4ux30fcux30c8ux304bux3089ux59cbux307eux3063ux305fux9a5aux304dux306eux5cb8ux672cux5b66ux5f01ux8b77ux58ebux306eux30c4ux30a4ux30fcux30c8ux306eux8a18ux9332}{%
\paragraph{「岸本先生は、他の弁護士に嫉妬されているが故に攻撃されているという自己認識なんだろうか。」という小倉秀夫弁護士のツイートから始まった驚きの岸本学弁護士のツイートの記録}\label{ux5cb8ux672cux5148ux751fux306fux4ed6ux306eux5f01ux8b77ux58ebux306bux5ac9ux59acux3055ux308cux3066ux3044ux308bux304cux6545ux306bux653bux6483ux3055ux308cux3066ux3044ux308bux3068ux3044ux3046ux81eaux5df1ux8a8dux8b58ux306aux3093ux3060ux308dux3046ux304bux3068ux3044ux3046ux5c0fux5009ux79c0ux592bux5f01ux8b77ux58ebux306eux30c4ux30a4ux30fcux30c8ux304bux3089ux59cbux307eux3063ux305fux9a5aux304dux306eux5cb8ux672cux5b66ux5f01ux8b77ux58ebux306eux30c4ux30a4ux30fcux30c8ux306eux8a18ux9332}}

\begin{itemize}
\tightlist
\item
  〉〉〉 Linux Emacs: 2021/05/26 11:00:49 〉〉〉
\end{itemize}

:CATEGORIES: @kanazawabengosi \#金沢弁護士会 @JFBAsns
\#日本弁護士連合会(日弁連) \#法務省 @MOJ\_HOUMU \#岸本学弁護士
\#小倉秀夫弁護士

〉〉〉 kk\_hironoのリツイート 〉〉〉

\begin{itemize}
\tightlist
\item
  RT
  kk\_hirono(刑事告発・非常上告_金沢地方検察庁御中)|chosakukenho(小倉秀夫)
  日時:2021-05-26 11:01/2021/05/26 01:32 URL:
  \url{https://twitter.com/kk\_hirono/status/1397372289281249287} 
  \url{https://twitter.com/chosakukenho/status/1397229214328360972} 
  \textgreater{}
  岸本先生は、他の弁護士に嫉妬されているが故に攻撃されているという自己認識なんだろうか。
  \url{https://t.co/NGbACcQ8gv} 
\end{itemize}

〉〉〉 kk\_hironoのリツイート 〉〉〉

\begin{itemize}
\tightlist
\item
  RT
  kk\_hirono(刑事告発・非常上告_金沢地方検察庁御中)|9jtCdbGf3lih8Fe(弁護士
  岸本 学) 日時:2021-05-26 11:01/2021/05/26 01:05 URL:
  \url{https://twitter.com/kk\_hirono/status/1397372335561207809} 
  \url{https://twitter.com/9jtCdbGf3lih8Fe/status/1397222300135612417} 
  \textgreater{} 弁護士の「嫉妬」ほど、 始末におえないものはない。
\end{itemize}

 上記のツイートがきっかけで岸本学弁護士のタイムラインを開いたのですが,驚きの内容のツイートがたくさんならんでいて,実名の弁護士のツイートということもありますから,これは参考にしなければいけない,石川県警察,金沢地方検察庁にも大いに参考にしていただく記録にしようと決意しました。

 まず,1つのスレッドを遡るとやがて別のスレッドに分かれたのですが,最終的に行き着いたのは同じ岸本学弁護士のツイートで,これが反応を呼び起こした発端なのだと判断しました。さきほどの次の2つのエントリーに一連の流れをやや不規則ながら網羅的にまとめました。

\begin{itemize}
\tightlist
\item
  1384:2021-05-24\_20:52:14 \#告発状 \#\#\#\#
  検察をこれでもかと激烈に批判,罵倒する「再審法改正をめざす市民の会」の元検事でもある市川寛弁護士と,金沢刑務所の官本で読んだ「能登怪奇譚」
  \url{https://hirono-hideki.hatenadiary.jp/entry/2021/05/24/205208} 
\item
  1385:2021-05-26\_00:00:38 \#告発状 \#\#\#\#
  福井刑務所で平成6年11月11日受付となっていた副本と赤い判子のある甲号証(二)と(四)の書類,被告発人長谷川紘之弁護士からの原告側提出書証,ノートとの整合性
  \url{https://hirono-hideki.hatenadiary.jp/entry/2021/05/26/000035} 
\item
  1386:2021-05-26\_02:07:05 \#告発状 \#\#\#\#
  2021年5月26日未明,大きな発見となった石川県羽咋市の平鍛造株式会社と被告発人木梨松嗣弁護士との接点,評議員に長原悟弁護士,理事に米田弘幸弁護士という関係性
  \url{https://hirono-hideki.hatenadiary.jp/entry/2021/05/26/020703} 
\item
  1387:2021-05-26\_10:43:53 \#告発状 \#\#\#\#
  痴漢被疑者の示談金,30万円を提示し,最終的には9割以上のケースで2\textasciitilde10倍の示談金額で決着する,という信じがたい岸本学弁護士のツイートと,他の弁護士の反応(1)
  \url{https://hirono-hideki.hatenadiary.jp/entry/2021/05/26/104350} 
\item
  1388:2021-05-26\_10:57:58 \#告発状 \#\#\#\#
  痴漢被疑者の示談金,30万円を提示し,最終的には9割以上のケースで2\textasciitilde10倍の示談金額で決着する,という信じがたい岸本学弁護士のツイートと,他の弁護士の反応(2)
  \url{https://hirono-hideki.hatenadiary.jp/entry/2021/05/26/105755} 
\end{itemize}

 全体的なエントリーは上記の流れで,被告発人木梨松嗣弁護士について本格的に取り上げていたのですが,急遽,必要性を感じて岸本学弁護士の一連のツイートと他の弁護士アカウントの反応を記録しておきべきと判断しました。これが弁護士なのかと改めて目から鱗が落ちた思いです。

 再び,岸本学弁護士のタイムラインにあるツイートをリツイートで記録し,ご紹介しておきたいと思います。

〉〉〉 kk\_hironoのリツイート 〉〉〉

\begin{itemize}
\tightlist
\item
  RT
  kk\_hirono(刑事告発・非常上告_金沢地方検察庁御中)|9jtCdbGf3lih8Fe(弁護士
  岸本 学) 日時:2021-05-26 11:12/2021/05/06 20:22 URL:
  \url{https://twitter.com/kk\_hirono/status/1397375136446119938} 
  \url{https://twitter.com/9jtCdbGf3lih8Fe/status/1390265803614785539} 
  \textgreater{} 当職のインタビューを記事にして頂きました。
  声を封じる被害者たたき 「あなたが悪い」恐れる被害者:朝日新聞デジタル
  \url{https://t.co/5dINjwAks1} 
\end{itemize}

〉〉〉 kk\_hironoのリツイート 〉〉〉

\begin{itemize}
\tightlist
\item
  RT
  kk\_hirono(刑事告発・非常上告_金沢地方検察庁御中)|9jtCdbGf3lih8Fe(弁護士
  岸本 学) 日時:2021-05-26 11:12/2021/05/26 01:46 URL:
  \url{https://twitter.com/kk\_hirono/status/1397375169555931137} 
  \url{https://twitter.com/9jtCdbGf3lih8Fe/status/1397232694384373761} 
  \textgreater{}
  「君は本当に素晴らしい。他の女よりも遥かに優れた奴隷だ」
  と言われると、嬉しくなって「もっと良い奴隷になろう」と思ってしまうような人は、確かにいる。
  と言うか、少なくない。
  優れていようがいまいが、奴隷は奴隷だ。誉めてないから。
\end{itemize}

〉〉〉 kk\_hironoのリツイート 〉〉〉

\begin{itemize}
\tightlist
\item
  RT
  kk\_hirono(刑事告発・非常上告_金沢地方検察庁御中)|9jtCdbGf3lih8Fe(弁護士
  岸本 学) 日時:2021-05-26 11:13/2021/05/26 01:31 URL:
  \url{https://twitter.com/kk\_hirono/status/1397375193153085442} 
  \url{https://twitter.com/9jtCdbGf3lih8Fe/status/1397228902582521858} 
  \textgreater{} 日本人は、依然少なくない数が
  「女性たちは奴隷であること」を 当然のこととして受け入れているぞ。
  かなりやばいことだと思うぞ。
\end{itemize}

〉〉〉 kk\_hironoのリツイート 〉〉〉

\begin{itemize}
\tightlist
\item
  RT
  kk\_hirono(刑事告発・非常上告_金沢地方検察庁御中)|9jtCdbGf3lih8Fe(弁護士
  岸本 学) 日時:2021-05-26 11:13/2021/05/26 01:11 URL:
  \url{https://twitter.com/kk\_hirono/status/1397375218834755585} 
  \url{https://twitter.com/9jtCdbGf3lih8Fe/status/1397223761611563018} 
  \textgreater{}
  「私たち『奴隷』が『ご主人様』に文句を言うのは間違っているのよ。そんなことをしたら、もっと酷い目に遭うのよ」
  声を挙げる性犯罪被害者の女性を「叩く」女性は、これに似ている。
\end{itemize}

〉〉〉 kk\_hironoのリツイート 〉〉〉

\begin{itemize}
\tightlist
\item
  RT
  kk\_hirono(刑事告発・非常上告_金沢地方検察庁御中)|9jtCdbGf3lih8Fe(弁護士
  岸本 学) 日時:2021-05-26 11:13/2021/05/26 01:26 URL:
  \url{https://twitter.com/kk\_hirono/status/1397375233753948162} 
  \url{https://twitter.com/9jtCdbGf3lih8Fe/status/1397227564679237637} 
  \textgreater{}
  「日本の女性は、せっかく『従順な奴隷だ』と誉められているのに、レイプされて文句を言う女がいると台無しだわ」
  と、言っている。
\end{itemize}

〉〉〉 kk\_hironoのリツイート 〉〉〉

\begin{itemize}
\tightlist
\item
  RT
  kk\_hirono(刑事告発・非常上告_金沢地方検察庁御中)|9jtCdbGf3lih8Fe(弁護士
  岸本 学) 日時:2021-05-26 11:13/2021/05/26 01:08 URL:
  \url{https://twitter.com/kk\_hirono/status/1397375253634904066} 
  \url{https://twitter.com/9jtCdbGf3lih8Fe/status/1397223057996013576} 
  \textgreater{} 相手を下に見ているから、
  相手の意思を無視して、性暴力を振るう。
  下に見ているから、性被害に遭った者を笑い、「お前に抗議の資格はない」と攻撃する。
\end{itemize}

〉〉〉 kk\_hironoのリツイート 〉〉〉

\begin{itemize}
\tightlist
\item
  RT
  kk\_hirono(刑事告発・非常上告_金沢地方検察庁御中)|9jtCdbGf3lih8Fe(弁護士
  岸本 学) 日時:2021-05-26 11:13/2021/05/25 10:45 URL:
  \url{https://twitter.com/kk\_hirono/status/1397375296559411202} 
  \url{https://twitter.com/9jtCdbGf3lih8Fe/status/1397005930210889733} 
  \textgreater{}
  痴漢や盗撮、ストーカー含め性犯罪の被害をうけるリスクは男性よりも女性の方が格段に大きい。
  これは、進学や就労においても、無視できない大きさの「ハンデ」を、女性につけていると思う。
\end{itemize}

〉〉〉 kk\_hironoのリツイート 〉〉〉

\begin{itemize}
\tightlist
\item
  RT
  kk\_hirono(刑事告発・非常上告_金沢地方検察庁御中)|9jtCdbGf3lih8Fe(弁護士
  岸本 学) 日時:2021-05-26 11:13/2021/05/25 10:38 URL:
  \url{https://twitter.com/kk\_hirono/status/1397375321721085957} 
  \url{https://twitter.com/9jtCdbGf3lih8Fe/status/1397004164232740868} 
  \textgreater{}
  「10代女性宅の浴室を盗撮しようとカメラを設置し、つきまといなどのストーカー行為を繰り返した」(記事から引用)
  被害者は恐怖の日々だと思います。犯人逮捕でも、安心できるかわからない。
  加害者は59歳の中学教諭。 女性はとんでもない危険にさらされている。
  \url{https://t.co/RzUEvq767l} 
\end{itemize}

〉〉〉 kk\_hironoのリツイート 〉〉〉

\begin{itemize}
\tightlist
\item
  RT
  kk\_hirono(刑事告発・非常上告_金沢地方検察庁御中)|9jtCdbGf3lih8Fe(弁護士
  岸本 学) 日時:2021-05-26 11:13/2021/05/24 20:54 URL:
  \url{https://twitter.com/kk\_hirono/status/1397375345225916416} 
  \url{https://twitter.com/9jtCdbGf3lih8Fe/status/1396796659288272899} 
  \textgreater{} 弁護士の少なからぬ数が、
  「誤判でも無罪判決は目出度い」と 思っているふしを感じる。
\end{itemize}

〉〉〉 kk\_hironoのリツイート 〉〉〉

\begin{itemize}
\tightlist
\item
  RT
  kk\_hirono(刑事告発・非常上告_金沢地方検察庁御中)|H\_Tomoko\_\_(Tomoko)
  日時:2021-05-26 11:14/2021/05/23 01:22 URL:
  \url{https://twitter.com/kk\_hirono/status/1397375504613666816} 
  \url{https://twitter.com/H\_Tomoko\_\_/status/1396139528692264961} 
  \textgreater{}
  条例違反の痴漢事件で被害者が示談金として300万円を請求していると聞いたら・・・そして被疑者が否認に転じたら・・・私が検事だったら起訴を躊躇ってしまう。。被疑者が否認したら略式にはできないですしね。(警察官が複数現認なら公判請求できるかもですが)
  \url{https://t.co/xGRrqxHJNS} 
\end{itemize}

〉〉〉 kk\_hironoのリツイート 〉〉〉

\begin{itemize}
\tightlist
\item
  RT
  kk\_hirono(刑事告発・非常上告_金沢地方検察庁御中)|H\_Tomoko\_\_(Tomoko)
  日時:2021-05-26 11:16/2021/05/23 03:13 URL:
  \url{https://twitter.com/kk\_hirono/status/1397376096476094468} 
  \url{https://twitter.com/H\_Tomoko\_\_/status/1396167394264064004} 
  \textgreater{} @9jtCdbGf3lih8Fe
  「嫌疑不十分で不起訴になります(なりました)」と伝えるでしょうね。
\end{itemize}

〉〉〉 kk\_hironoのリツイート 〉〉〉

\begin{itemize}
\tightlist
\item
  RT
  kk\_hirono(刑事告発・非常上告_金沢地方検察庁御中)|9jtCdbGf3lih8Fe(弁護士
  岸本 学) 日時:2021-05-26 11:16/2021/05/23 06:45 URL:
  \url{https://twitter.com/kk\_hirono/status/1397376104789250055} 
  \url{https://twitter.com/9jtCdbGf3lih8Fe/status/1396220766463533057} 
  \textgreater{} @H\_Tomoko\_\_
  私は「微妙に」違う見方をしています。そして実際、検事の対応はそれから外れていません。
\end{itemize}

〉〉〉 kk\_hironoのリツイート 〉〉〉

\begin{itemize}
\tightlist
\item
  RT
  kk\_hirono(刑事告発・非常上告_金沢地方検察庁御中)|9jtCdbGf3lih8Fe(弁護士
  岸本 学) 日時:2021-05-26 11:16/2021/05/23 02:24 URL:
  \url{https://twitter.com/kk\_hirono/status/1397376140717613056} 
  \url{https://twitter.com/9jtCdbGf3lih8Fe/status/1396155028524605442} 
  \textgreater{} @HrBuy2 ネットイナゴとは良い表現ですね。
  数が多いと踏み潰すのも疲れる。 良い殺虫剤は無いものか。
\end{itemize}

〉〉〉 kk\_hironoのリツイート 〉〉〉

\begin{itemize}
\tightlist
\item
  RT
  kk\_hirono(刑事告発・非常上告_金沢地方検察庁御中)|kazu6445(カズタカ)
  日時:2021-05-26 11:17/2021/05/22 23:14 URL:
  \url{https://twitter.com/kk\_hirono/status/1397376336545472514} 
  \url{https://twitter.com/kazu6445/status/1396107147436388354} 
  \textgreater{} 不起訴になった例を知ってますが
  常に略式になるって何を根拠に仰ってるんですかね?
  \url{https://t.co/RQlKnPSQmw} 
\end{itemize}

〉〉〉 kk\_hironoのリツイート 〉〉〉

\begin{itemize}
\tightlist
\item
  RT
  kk\_hirono(刑事告発・非常上告_金沢地方検察庁御中)|9jtCdbGf3lih8Fe(弁護士
  岸本 学) 日時:2021-05-26 11:17/2021/05/22 23:57 URL:
  \url{https://twitter.com/kk\_hirono/status/1397376370934648834} 
  \url{https://twitter.com/9jtCdbGf3lih8Fe/status/1396118123296612354} 
  \textgreater{} @kazu6445
  貴職の目には「常に」という文字が見えるらしいが、
  書面読む時、不便で無いですか?
\end{itemize}

〉〉〉 kk\_hironoのリツイート 〉〉〉

\begin{itemize}
\tightlist
\item
  RT
  kk\_hirono(刑事告発・非常上告_金沢地方検察庁御中)|9jtCdbGf3lih8Fe(弁護士
  岸本 学) 日時:2021-05-26 11:18/2021/05/22 21:46 URL:
  \url{https://twitter.com/kk\_hirono/status/1397376480405954565} 
  \url{https://twitter.com/9jtCdbGf3lih8Fe/status/1396084959614357506} 
  \textgreater{} @relaxpop コミュニケーションの「内容」ですね。
  鈍感とかそういう問題ではないと思います。
  検事が被疑者弁護人に伝える内容と、 検事が被害者代理人に伝える内容が、
  全く同じであるはずはないと思いますが。
\end{itemize}

〉〉〉 kk\_hironoのリツイート 〉〉〉

\begin{itemize}
\tightlist
\item
  RT
  kk\_hirono(刑事告発・非常上告_金沢地方検察庁御中)|9jtCdbGf3lih8Fe(弁護士
  岸本 学) 日時:2021-05-26 11:18/2021/05/22 21:47 URL:
  \url{https://twitter.com/kk\_hirono/status/1397376503688564738} 
  \url{https://twitter.com/9jtCdbGf3lih8Fe/status/1396085305636126721} 
  \textgreater{} @relaxpop
  もちろん、ケースによっては差が出ないことも有るとはおもいますが。
\end{itemize}

〉〉〉 kk\_hironoのリツイート 〉〉〉

\begin{itemize}
\tightlist
\item
  RT
  kk\_hirono(刑事告発・非常上告_金沢地方検察庁御中)|9jtCdbGf3lih8Fe(弁護士
  岸本 学) 日時:2021-05-26 11:18/2021/05/22 20:14 URL:
  \url{https://twitter.com/kk\_hirono/status/1397376677802508289} 
  \url{https://twitter.com/9jtCdbGf3lih8Fe/status/1396061842124939264} 
  \textgreater{} あとね、
  痴漢の示談交渉の主体は「代理人弁護士」ではないよ。
  「代理人弁護士」は自分の状況分析は伝えるけれども、自分の考えを本人に押し付けてはいけない。
  特に、被害者が望む示談金額を、自分の考えで押さえつけてはいけない。
\end{itemize}

〉〉〉 kk\_hironoのリツイート 〉〉〉

\begin{itemize}
\tightlist
\item
  RT
  kk\_hirono(刑事告発・非常上告_金沢地方検察庁御中)|9jtCdbGf3lih8Fe(弁護士
  岸本 学) 日時:2021-05-26 11:19/2021/05/22 20:14 URL:
  \url{https://twitter.com/kk\_hirono/status/1397376702624329731} 
  \url{https://twitter.com/9jtCdbGf3lih8Fe/status/1396061843718692871} 
  \textgreater{} それと「代理人弁護士」は、それが本人の意向であれば、
  交渉に「負ける」ことをためらってはいけない。
\end{itemize}

〉〉〉 kk\_hironoのリツイート 〉〉〉

\begin{itemize}
\tightlist
\item
  RT
  kk\_hirono(刑事告発・非常上告_金沢地方検察庁御中)|9jtCdbGf3lih8Fe(弁護士
  岸本 学) 日時:2021-05-26 11:19/2021/05/22 20:16 URL:
  \url{https://twitter.com/kk\_hirono/status/1397376714833989634} 
  \url{https://twitter.com/9jtCdbGf3lih8Fe/status/1396062529399398405} 
  \textgreater{} 「変わり身」早いよ、 私は。
\end{itemize}

〉〉〉 kk\_hironoのリツイート 〉〉〉

\begin{itemize}
\tightlist
\item
  RT
  kk\_hirono(刑事告発・非常上告_金沢地方検察庁御中)|hiyokomamemame4(ヒヨコマメマメ)
  日時:2021-05-26 11:19/2021/05/22 20:14 URL:
  \url{https://twitter.com/kk\_hirono/status/1397376790025293826} 
  \url{https://twitter.com/hiyokomamemame4/status/1396061810617315330} 
  \textgreater{}
  そもそも性犯罪・性被害についての裁判所の慰謝料基準が低廉にすぎることが問題の根底にある気がする
  被害者からすると「そんなに低いんですか・・・?」となり、弁護人からすると「そうは言っても裁判所ではこんなもんですから〜なら訴訟します?」となり、起訴前ボーナスステージで妥結するしかないのだ
  \url{https://t.co/BUwNerNgvt} 
\end{itemize}

〉〉〉 kk\_hironoのリツイート 〉〉〉

\begin{itemize}
\tightlist
\item
  RT
  kk\_hirono(刑事告発・非常上告_金沢地方検察庁御中)|9jtCdbGf3lih8Fe(弁護士
  岸本 学) 日時:2021-05-26 11:19/2021/05/22 20:16 URL:
  \url{https://twitter.com/kk\_hirono/status/1397376910800216064} 
  \url{https://twitter.com/9jtCdbGf3lih8Fe/status/1396062326810308609} 
  \textgreater{} @hiyokomamemame4 そのとおりと思います。
  ましてや、赤の他人の痴漢犯人に対して強制執行はほぼ不可能です。
  痴漢被害者にとって、民事裁判は無いものと考える必要がある、
\end{itemize}

〉〉〉 kk\_hironoのリツイート 〉〉〉

\begin{itemize}
\item
  RT
  kk\_hirono(刑事告発・非常上告_金沢地方検察庁御中)|9jtCdbGf3lih8Fe(弁護士
  岸本 学) 日時:2021-05-26 11:20/2021/05/22 19:17 URL:
  \url{https://twitter.com/kk\_hirono/status/1397376999539167235} 
  \url{https://twitter.com/9jtCdbGf3lih8Fe/status/1396047568665145346} 
  \textgreater{} 当たり前ですが、
  痴漢の被害者は、示談交渉で、自分の希望に反した金額・条件に応じる法的義務はありません。
  それは被疑者も同様。 折り合いがつけば、示談が成立します。
  なので、誰かが「妥当な金額・条件」を決めたとしても意味がありません。
\item
  〈〈〈 2021/05/26 11:21:57 Linux Emacs: 〈〈〈
\end{itemize}

\hypertarget{ux793eux4f1aux554fux984cux4e8bux4ef6ux5831ux9053ux306bux5bfeux3059ux308bux5f01ux8b77ux58ebux306eux53cdux5fdc}{%
\subsection{社会問題,事件報道に対する弁護士の反応}\label{ux793eux4f1aux554fux984cux4e8bux4ef6ux5831ux9053ux306bux5bfeux3059ux308bux5f01ux8b77ux58ebux306eux53cdux5fdc}}

\hypertarget{ux65edux5ddd14ux6b73ux5c11ux5973ux30a4ux30b8ux30e1ux51cdux6b7bux4e8bux4ef6}{%
\subsubsection{旭川14歳少女イジメ凍死事件}\label{ux65edux5ddd14ux6b73ux5c11ux5973ux30a4ux30b8ux30e1ux51cdux6b7bux4e8bux4ef6}}

\hypertarget{ux65edux5ddd14ux6b73ux5c11ux5973ux30a4ux30b8ux30e1ux51cdux6b7bux4e8bux4ef6ux306bux5bfeux3059ux308bux5f01ux8b77ux58ebux30b8ux30e3ux30fcux30caux30eaux30b9ux30c8ux6cd5ux5b66ux8005ux306eux30c4ux30a4ux30fcux30c8ux306eux8a18ux9332}{%
\paragraph{「旭川14歳少女イジメ凍死事件」に対する弁護士,ジャーナリスト,法学者のツイートの記録}\label{ux65edux5ddd14ux6b73ux5c11ux5973ux30a4ux30b8ux30e1ux51cdux6b7bux4e8bux4ef6ux306bux5bfeux3059ux308bux5f01ux8b77ux58ebux30b8ux30e3ux30fcux30caux30eaux30b9ux30c8ux6cd5ux5b66ux8005ux306eux30c4ux30a4ux30fcux30c8ux306eux8a18ux9332}}

\begin{itemize}
\tightlist
\item
  〉〉〉 Linux Emacs: 2021/04/20 19:56:50 〉〉〉
\end{itemize}

\begin{quote}
《引用の始まり》
\end{quote}

\begin{quote}
アカウント名 ツイート数 リツイート数NEWS JAPAN(NEWS\_JAPAN\_S) 63
5文春オンライン(bunshun\_online) 43 0eiichi(eiichi24196) 0
1渡邉葉(YoWatShiinaEsq) 0 1本 秀紀(MotoBuch1228) 1
0奉納\さらば弁護士鉄道・泥棒神社の物語(hirono\_hideki) 8
3シン・へぼろーやー?残高22円?(i9wpevKQdGktbmY) 0
2うぱ弁(j9ginnga2lets) 0 1うそつきべ んごし。(LiarLawyer800) 0 1NEWS
JAPAN R(news\_type\_c) 3 0
\end{quote}

\begin{quote}
《引用の終わり》
\end{quote}

\begin{itemize}
\item
  奉納\危険生物・弁護士脳汚染除去装置\金沢地方検察庁御中\_2020:
  REGEXP:''旭川.*(女子|少女)''/データベース登録済みツイートの検索:2021-04-16〜2021-04-20/2021年04月20日19時45分の記録:ユーザ・投稿:10/132件
  \url{https://kk2020-09.blogspot.com/2021/04/regexp2021-04-162021-04\_20.html} 
\item
  〈〈〈 2021/04/20 19:58:25 Linux Emacs: 〈〈〈
\end{itemize}

\hypertarget{ux65edux5ddd14ux6b73ux5c11ux5973ux30a4ux30b8ux30e1ux51cdux6b7bux4e8bux4ef6ux306bux5411ux3051ux305fux3068ux601dux308fux308cux308bux6df1ux6fa4ux8aedux53f2ux5f01ux8b77ux58ebux306eux30c4ux30a4ux30fcux30c8ux30eaux30c4ux30a4ux30fcux30c8ux4e8bux4ef6ux306eux7279ux5b9aux306f4ux670820ux65e5ux6642ux70b9ux3067ux78baux8a8dux3055ux308cux306aux3044}{%
\paragraph{「旭川14歳少女イジメ凍死事件」に向けたと思われる深澤諭史弁護士のツイート・リツイート,事件の特定は4月20日時点で確認されない}\label{ux65edux5ddd14ux6b73ux5c11ux5973ux30a4ux30b8ux30e1ux51cdux6b7bux4e8bux4ef6ux306bux5411ux3051ux305fux3068ux601dux308fux308cux308bux6df1ux6fa4ux8aedux53f2ux5f01ux8b77ux58ebux306eux30c4ux30a4ux30fcux30c8ux30eaux30c4ux30a4ux30fcux30c8ux4e8bux4ef6ux306eux7279ux5b9aux306f4ux670820ux65e5ux6642ux70b9ux3067ux78baux8a8dux3055ux308cux306aux3044}}

\begin{itemize}
\tightlist
\item
  〉〉〉 Linux Emacs: 2021/04/20 20:01:24 〉〉〉
\end{itemize}

:CATEGORIES: @kanazawabengosi \#金沢弁護士会 @JFBAsns
日本弁護士連合会(日弁連) \#法務省 @MOJ\_HOUMU \#深澤諭史弁護士
\#いじめ \#自殺 \#学校

\begin{itemize}
\item
  奉納\危険生物・弁護士脳汚染除去装置\金沢地方検察庁御中\_2020:
  @fukazawas(深澤諭史)のツイート ''.*'' 3218/3218:2021-02-07\_1004〜2021-04-20\_1954 2021年04月20日20時03分の記録
  \url{https://t.co/uiucRE7d5i} 
\item
  \begin{itemize}
  \tightlist
  \item
    (5/3218)
    @fukazawas(深澤諭史)のツイート ''.*'' 3218/3218:2021-02-07\_1004〜2021-04-20\_1954
    2021年04月20日20時03分の記録 ¥\n RT
    fukazawas(深澤諭史)|63s244(TM) 日時:2021-04-20
    12:53/2021-04-20 10:28 URL: \url{https://t.co/OgfNbK0bsQ} 
    \url{https://t.co/soke8iPfZp}  \textgreater{}
    × 弁護士が入るとややこしくなる ¥\n \textgreater{} ¥\n
    \textgreater{}
    ○ 弁護士が入るとこちらに都合よく言いくるめられないから困る
  \end{itemize}
\item
  (15/3218)
  @fukazawas(深澤諭史)のツイート ''.*'' 3218/3218:2021-02-07\_1004〜2021-04-20\_1954
  2021年04月20日20時03分の記録\\
  RT fukazawas(深澤諭史)|O59K2dPQH59QEJx(ピピピーッ)
  日時:2021-04-20 08:41/2021-04-19 19:02 URL:
  \url{https://twitter.com/fukazawas/status/1384291075892781062} 
  \url{https://twitter.com/O59K2dPQH59QEJx/status/1384085051097325577} 
\item
  (16/3218)
  @fukazawas(深澤諭史)のツイート ''.*'' 3218/3218:2021-02-07\_1004〜2021-04-20\_1954
  2021年04月20日20時03分の記録\\
  RT fukazawas(深澤諭史)|uwaaaa(サイ太) 日時:2021-04-20
  08:25/2021-04-19 14:31 URL:
  \url{https://twitter.com/fukazawas/status/1384287155782971393} 
  \url{https://twitter.com/uwaaaa/status/1384016694281007110} 
  \textgreater{}
  他人を加害した人物を全くの第三者が正義棒でぶったたいてるとき,脳のどの部分が活性化するんですかね
\item
  (31/3218)
  @fukazawas(深澤諭史)のツイート ''.*'' 3218/3218:2021-02-07\_1004〜2021-04-20\_1954
  2021年04月20日20時03分の記録\\
  RT fukazawas(深澤諭史)|nohohon\_6098(nohohon6098)
  日時:2021-04-19 11:19/2021-04-19 09:28 URL:
  \url{https://twitter.com/fukazawas/status/1383968501623267328} 
  \url{https://twitter.com/nohohon\_6098/status/1383940467759276036} 
  \textgreater{}
  戦前の少年少女の犯罪史とか見ると、今と変わらない、今より酷いいじめや凶悪な犯罪が数多出てくる。\\
  \textgreater{}
  人権教育が行き届くようになった今のほうが明らかに減っていて、多いように見えるのはメディアの報じ方の問題で、問題が起きると集中的に報じるので多いと錯覚するけど、実は減ってるんですけどね。
  \url{https://t.co/VvAxyjC90j} 
\item
  (34/3218)
  @fukazawas(深澤諭史)のツイート ''.*'' 3218/3218:2021-02-07\_1004〜2021-04-20\_1954
  2021年04月20日20時03分の記録\\
  RT fukazawas(深澤諭史)|SotaKimura(木村草太) 日時:2021-04-18
  20:37/2021-04-18 19:24 URL:
  \url{https://twitter.com/fukazawas/status/1383746549671370769} 
  \url{https://twitter.com/SotaKimura/status/1383728248949940231} 
  \textgreater{}
  権力勾配を原因とするハラスメント事案を見ていると、「今までうまくいっていたのに、弁護士が余計なことをするからおかしくなった」みたいな主張に出くわすことがある。\\
  \textgreater{}
  我慢を強いられてきた周囲の人が、弁護士の援助で、ようやく適切な主張ができるようになってきただけだと思う。
\end{itemize}

 最後の「日時:2021-04-18 20:37/2021-04-18
19:24」は,旭川14歳少女イジメ凍死事件と無関係なのかもしれません。Twitterのt練度で見かけたのが最初だったのかはっきり思い出せないので,この辺りも確認をしておきたいところです。

\begin{itemize}
\tightlist
\item
  2021年04月20日20時22分の登録:
  「旭川」を@hirono\_hideki @kk\_hirono @s\_hironoで検索 253件の該当 2021-04-20\_20:22の記録
  \url{https://kk2020-09.blogspot.com/2021/04/hironohidekikkhironoshirono2532021-04.html} 
\end{itemize}

2021-01-15 17:08:27 ``\textgreater{}
旭川弁護士会館にて、全国冤罪事件弁護団連絡協議会第24回交流会「舞鶴事件にみる事実認定のあり方」をテレビ会議で視聴中。遠山大輔弁護士の報告。防犯カメラ画像分析についての千原・奈良先端科学技術大学院大学名誉教授報告等''
\url{https://twitter.com/kk\_hirono/status/1349991824530653185} 

2021-03-22 20:04:16 ``RT @1961kumachin: こ、これは、、、、
HTB北海道ニュース\textbar 旭川のホテルや商業施設で不審火 放火の可能性も
\url{https://www.htb.co.jp/news/archives\_10851.html}  \#北海道
\#HTB北海道ニュース''
\url{https://twitter.com/kk\_hirono/status/1373953672070135810} 

\begin{itemize}
\item
  2021-04-19 04:47:33 ``- 旭川14歳女子凍死
  教頭が「加害生徒にも未来がある」と母親に告げる? - ライブドアニュース
  \url{https://t.co/F7IfnafGCq''}  \url{https://t.co/T7S4HzNPN3} 
\item
  2021-04-19 05:03:06 ``旭川14歳女子凍死
  教頭が「加害生徒にも未来がある」と母親に告げる? - ライブドアニュース
  \url{https://t.co/F7IfnafGCq}  ¥\n¥\n2021年4月18日
  16時0分¥\n¥\n文春オンライン'' \url{https://t.co/3tavgV1MND} 
\item
  2021-04-19 05:05:28
  ``「娘の遺体は凍っていた」14歳少女がマイナス17℃の旭川で凍死 背景に上級生の凄惨イジメ《母親が涙の告白》
  \textbar{} 文春オンライン \url{https://t.co/QOzjW8erhZ''} 
  \url{https://t.co/IufE0ctSEh} 
\item
  2021-04-19 05:35:28
  ``廣瀬爽彩(さあや)いじめデマ続出「中尾まほ顔画像特定は嘘」北海道旭川事件加害者
  \textbar{} こねこのニュース調べ \url{https://t.co/rPl6IhX3tH''} 
  \url{https://t.co/xaiAIMlohd} 
\item
  2021-04-19 05:37:28
  ``廣瀬爽彩(さあや)中学いじめ無視「女子生徒の妄想」旭川北星校長が対応と謝罪拒否
  \textbar{} こねこのニュース調べ \url{https://t.co/F39w8yYM2H''} 
  \url{https://t.co/BnJvfgLUmx} 
\end{itemize}

2021-04-19 05:46:04
``「死ぬから画像を消してください」旭川14歳女子死亡``ウッペツ川飛び込み''イジメ事件の全貌《警察が出動》(文春オンライン)
- Yahoo!ニュース \url{https://t.co/SwM8CAnJmJ} 
この``飛び込み事件''は、地元の情報誌「メディアあさひかわ」(2019年10月号)が報じている。"
\url{https://twitter.com/hirono\_hideki/status/1383884557502488577} 

 ツイートの記録の旭川の検索結果が253件ということで意外に少なかったのですが,これまでも注目をしてきたのは旭川弁護士会や,旭川弁護士会に所属する2人の弁護士です。一人は実名のTwitterアカウントで,近年,旭川弁護士会の会長もされていました。2人ともブロックはされてないはずです。

〉〉〉 kk\_hironoのリツイート 〉〉〉

\begin{itemize}
\tightlist
\item
  RT
  kk\_hirono(刑事告発・非常上告_金沢地方検察庁御中)|1961kumachin(くまちん(弁護士中村元弥))
  日時:2021-04-20 20:36/2021/04/20 17:52 URL:
  \url{https://twitter.com/kk\_hirono/status/1384471035316117509} 
  \url{https://twitter.com/1961kumachin/status/1384429819232878592} 
  \textgreater{}
  日弁連刑事弁護センター無事終了(ZOOM参加)。今日も、性犯罪に関する刑事法検討会の議論状況や刑事司法IT化の議論状況その他、大変勉強になったわ。
\end{itemize}

〉〉〉 kk\_hironoのリツイート 〉〉〉

\begin{itemize}
\tightlist
\item
  RT
  kk\_hirono(刑事告発・非常上告_金沢地方検察庁御中)|1961kumachin(くまちん(弁護士中村元弥))
  日時:2021-04-20 20:36/2021/04/20 17:49 URL:
  \url{https://twitter.com/kk\_hirono/status/1384471104538877958} 
  \url{https://twitter.com/1961kumachin/status/1384428926060007427} 
  \textgreater{}
  今日まさにそういう状況があったけど、検察事務官としては取調べを終えて下におろしたのだから警察にもう戻っているだろうと思っていて、実際には警察の車は他の被疑者を待つためにまだ検察庁にいるという状況なのだろうなと
  \url{https://t.co/rT9LEdbxWk} 
\end{itemize}

〉〉〉 kk\_hironoのリツイート 〉〉〉

\begin{itemize}
\tightlist
\item
  RT
  kk\_hirono(刑事告発・非常上告_金沢地方検察庁御中)|1961kumachin(くまちん(弁護士中村元弥))
  日時:2021-04-20 20:37/2021/04/20 15:02 URL:
  \url{https://twitter.com/kk\_hirono/status/1384471194842198021} 
  \url{https://twitter.com/1961kumachin/status/1384387087055523840} 
  \textgreater{}
  はあちゅうさんについてTwitterでコメントした記憶はないが、念のためついログを検索するかな
\end{itemize}

〉〉〉 kk\_hironoのリツイート 〉〉〉

\begin{itemize}
\tightlist
\item
  RT
  kk\_hirono(刑事告発・非常上告_金沢地方検察庁御中)|nabeteru1Q78(渡辺輝人)
  日時:2021-04-20 20:37/2021/04/20 14:01 URL:
  \url{https://twitter.com/kk\_hirono/status/1384471303030132738} 
  \url{https://twitter.com/nabeteru1Q78/status/1384371699156545538} 
  \textgreater{} 再審請求大崎事件は新たなステージに移っている。 /
  1件のコメント \url{https://t.co/5VmhfxpKDF} 
  ``大崎事件、隣人供述の矛盾を動画に 弁護団「事故死」立証に挑む|【西日本新聞ニュース】''
  \url{https://t.co/Z2PrPd3Onz} 
\end{itemize}

〉〉〉 kk\_hironoのリツイート 〉〉〉

\begin{itemize}
\tightlist
\item
  RT
  kk\_hirono(刑事告発・非常上告_金沢地方検察庁御中)|ichikeinokarasu(イチケイのカラス【公式】)
  日時:2021-04-20 20:38/2021/04/19 15:45 URL:
  \url{https://twitter.com/kk\_hirono/status/1384471433770790912} 
  \url{https://twitter.com/ichikeinokarasu/status/1384035447362310146} 
  \textgreater{} ㊗️『\#イチケイのカラス 』1〜4巻
  またまた重版が決まりました!🥳🥳🥳
  5月下旬より全国の書店さんに並ぶそうです‼️ 下のコマは原作版
  坂間の第1話のシーンです📚ドラマにも出てきましたね〜‼️
  \url{https://t.co/i6PMieOnHs}  \url{https://t.co/LmrcvSgbUj} 
\end{itemize}

〉〉〉 kk\_hironoのリツイート 〉〉〉

\begin{itemize}
\tightlist
\item
  RT
  kk\_hirono(刑事告発・非常上告_金沢地方検察庁御中)|matimura(田丁木寸)
  日時:2021-04-20 20:38/2021/04/19 13:39 URL:
  \url{https://twitter.com/kk\_hirono/status/1384471502414770180} 
  \url{https://twitter.com/matimura/status/1384003704488005632} 
  \textgreater{}
  ゼミで学生がみんなに紹介していた本『ケーキを切れない非行少年たち』\url{https://t.co/OmNATEbz2o} 
  なかなかいい紹介で、少年法改正の背後にある刑罰至上主義に警鐘を鳴らす内容となっていた。
\end{itemize}

〉〉〉 kk\_hironoのリツイート 〉〉〉

\begin{itemize}
\tightlist
\item
  RT
  kk\_hirono(刑事告発・非常上告_金沢地方検察庁御中)|yinoue1975(井上雄樹)
  日時:2021-04-20 20:38/2021/04/19 12:19 URL:
  \url{https://twitter.com/kk\_hirono/status/1384471539542724609} 
  \url{https://twitter.com/yinoue1975/status/1383983672508878855} 
  \textgreater{}
  生保受給中の法テラス利用者から償還金天引きされたとの連絡受けたので、地方事務所にどういうことか尋ねたら免除申請書が期限までに提出されなかったからとの説明。いつからこんな鬼畜な運用になったのだ?
\end{itemize}

〉〉〉 kk\_hironoのリツイート 〉〉〉

\begin{itemize}
\tightlist
\item
  RT
  kk\_hirono(刑事告発・非常上告_金沢地方検察庁御中)|1961kumachin(くまちん(弁護士中村元弥))
  日時:2021-04-20 20:38/2021/04/19 08:58 URL:
  \url{https://twitter.com/kk\_hirono/status/1384471625689567238} 
  \url{https://twitter.com/1961kumachin/status/1383932887288803333} 
  \textgreater{} @to\_pamyu @hirohika777
  法廷で被告人と初めて会ったのかよ、笑
\end{itemize}

〉〉〉 kk\_hironoのリツイート 〉〉〉

\begin{itemize}
\tightlist
\item
  RT
  kk\_hirono(刑事告発・非常上告_金沢地方検察庁御中)|1961kumachin(くまちん(弁護士中村元弥))
  日時:2021-04-20 20:39/2021/04/18 22:57 URL:
  \url{https://twitter.com/kk\_hirono/status/1384471677103329288} 
  \url{https://twitter.com/1961kumachin/status/1383781649683402761} 
  \textgreater{}
  月曜日は弁護士任官、火曜日は刑事弁護センターと2日連続ZOOMで日弁連の会議。どちらも私が仕切らないといけない議題があるので、本当は日弁連に行きたいのだが
\end{itemize}

 上記9件のリツイートは,中村元弥弁護士のタイムラインのツイートになります。もう一人で元旭川弁護士会会長の井上雄樹弁護士のツイートもありました。次にそちらも見てみますが,中村元弥弁護士のタイムラインに旭川14歳少女イジメ凍死事件関連のツイートは確認されませんでした。

〉〉〉 kk\_hironoのリツイート 〉〉〉

\begin{itemize}
\tightlist
\item
  RT
  kk\_hirono(刑事告発・非常上告_金沢地方検察庁御中)|yinoue1975(井上雄樹)
  日時:2021-04-20 20:41/2021/04/20 18:08 URL:
  \url{https://twitter.com/kk\_hirono/status/1384472342068301832} 
  \url{https://twitter.com/yinoue1975/status/1384433682350411778} 
  \textgreater{}
  今はググっても出てこないけど、全裸弁護士ってのがあったよなあ
  \url{https://t.co/dskwSuN3CX} 
\end{itemize}

〉〉〉 kk\_hironoのリツイート 〉〉〉

\begin{itemize}
\tightlist
\item
  RT
  kk\_hirono(刑事告発・非常上告_金沢地方検察庁御中)|hdm1987(細々経営の若手弁護士)
  日時:2021-04-20 20:44/2021/04/20 10:39 URL:
  \url{https://twitter.com/kk\_hirono/status/1384472946727555084} 
  \url{https://twitter.com/hdm1987/status/1384320831174496258} 
  \textgreater{}
  もうこうなったら、めちゃくちゃ破天荒で型破りな裁判官・検察官・弁護士のドラマを作成したい。
  裁判官「審理してきたけど、分かんねーな!もうこうなったら、殴り合いで決めましょうよ!」
  検察官「そうですね!久しぶりにやりますか!」
  弁護人「ククク・・・、チカラを開放スル時がキタカ・・・」
\end{itemize}

〉〉〉 kk\_hironoのリツイート 〉〉〉

\begin{itemize}
\tightlist
\item
  RT
  kk\_hirono(刑事告発・非常上告_金沢地方検察庁御中)|battamonblack02(KBブラック02)
  日時:2021-04-20 20:44/2021/04/19 12:36 URL:
  \url{https://twitter.com/kk\_hirono/status/1384473082123878405} 
  \url{https://twitter.com/battamonblack02/status/1383987818662305796} 
  \textgreater{}
  生活保護の証明書を出していたら引き落としはされないはずでは・・・
  \url{https://t.co/0O7D8oClLD} 
\end{itemize}

〉〉〉 kk\_hironoのリツイート 〉〉〉

\begin{itemize}
\tightlist
\item
  RT
  kk\_hirono(刑事告発・非常上告_金沢地方検察庁御中)|yinoue1975(井上雄樹)
  日時:2021-04-20 20:44/2021/04/19 12:39 URL:
  \url{https://twitter.com/kk\_hirono/status/1384473118287167493} 
  \url{https://twitter.com/yinoue1975/status/1383988639743369220} 
  \textgreater{} @battamonblack02
  援助申し込み時に資料出しているので生保受給中であることはわかっているはずなんです。以前は改めて免除申請などする必要なかったと思うのですが、年々お役所仕事になってきてますね。
\end{itemize}

〉〉〉 kk\_hironoのリツイート 〉〉〉

\begin{itemize}
\tightlist
\item
  RT
  kk\_hirono(刑事告発・非常上告_金沢地方検察庁御中)|idleness\_venomy(venomy)
  日時:2021-04-20 20:45/2021/04/17 14:34 URL:
  \url{https://twitter.com/kk\_hirono/status/1384473209462951942} 
  \url{https://twitter.com/idleness\_venomy/status/1383292752780161027} 
  \textgreater{}
  「日本ほど犯罪者に優しい国はありません」っていう弁護士さんがいて、さすがにヤメ検以外でそんなこと言う弁護士初めてみたと思ったら、普通にヤメ検だった話。
\end{itemize}

〉〉〉 kk\_hironoのリツイート 〉〉〉

\begin{itemize}
\tightlist
\item
  RT
  kk\_hirono(刑事告発・非常上告_金沢地方検察庁御中)|noooooooorth(教皇ノースライム)
  日時:2021-04-20 20:45/2021/04/14 22:53 URL:
  \url{https://twitter.com/kk\_hirono/status/1384473323644493833} 
  \url{https://twitter.com/noooooooorth/status/1382331056020283397} 
  \textgreater{}
  完全にサンデル教授と同じことを言うている。どエンド君の方が数年早い。
  \url{https://t.co/Chnucjgbqh} 
\end{itemize}

〉〉〉 kk\_hironoのリツイート 〉〉〉

\begin{itemize}
\tightlist
\item
  RT
  kk\_hirono(刑事告発・非常上告_金沢地方検察庁御中)|yinoue1975(井上雄樹)
  日時:2021-04-20 20:45/2021/04/15 13:34 URL:
  \url{https://twitter.com/kk\_hirono/status/1384473336541978626} 
  \url{https://twitter.com/yinoue1975/status/1382552890087141377} 
  \textgreater{}
  共同親権制になったからといって、連れ去りや面交の問題がすぐに解決するわけではないというのはその通りだと思うが、何かを期待して共同親権を支持する人の気持ちを理解することも大事だと思う。
\end{itemize}

 上記7件が,井上雄樹弁護士のタイムラインで見かけたツイートになります。本人のツイートの他,リツイートと引用されたツイートを含みます。やはり旭川14歳少女イジメ凍死事件に関連したツイートは全く確認されませんでした。

\begin{itemize}
\tightlist
\item
  2020年09月13日07時44分の登録:
  \井上雄樹 @yinoue1975\「どのような取り調べが行われているのか」ってドヤってるけど、可視化しろ、で終わりだし
  \url{http://hirono2014sk.blogspot.com/2020/09/yinoue1975.html} 
\item
  2020年09月27日11時35分の登録:
  \井上雄樹 @yinoue1975\主たる監護者かつDV被害者という認定がされれば、ほぼ親権者に指定されると思うのだけど、他にどんな事情があったのか気になる。
  \url{http://kk2020-09.blogspot.com/2020/09/yinoue1975dv.html} 
\item
  2020年10月04日19時31分の登録:
  \井上雄樹 @yinoue1975\事務官出身、異色の新検事正 盛岡地検・岡田氏「若手の目標に」:中日新聞Web
  \url{http://kk2020-09.blogspot.com/2020/10/yinoue1975web.html} 
\item
  2020年10月05日06時45分の登録:
  #井上雄樹 @yinoue1975#のツイート/2020-07-02\_0812〜2020-09-29\_0911/法務検察・石川県警察宛参考資料/記録作成措置実行日時:2020年10月05日06時45分
  \url{http://kk2020-09.blogspot.com/2020/10/yinoue19752020-07-0208122020-09.html} 
\item
  2020年11月02日02時17分の登録:
  \井上雄樹 @yinoue1975\自分も10年前に同じようなツイートしてたよ。今はなんとも思わないよ。
  \url{http://kk2020-09.blogspot.com/2020/11/yinoue1975.html} 
\item
  2020年11月25日17時17分の登録: %@Handalaw 半田
  望%日弁連接見委員会でお世話になっている、東京の井上侑先生が執筆された「被疑者弁護マニュアル」を入手。¥\n流し読みですが、否認事件の対応から勾
  \url{http://kk2020-09.blogspot.com/2020/11/handalaw-n.html} 
\item
  2020年11月29日06時31分の登録:
  @yinoue1975(井上雄樹)のツイート ''.*'' 3148/3148:2018-08-23\_1115〜2020-11-28\_1234 2020年11月29日06時30分の記録
  \url{http://kk2020-09.blogspot.com/2020/11/yinoue1975314831482018-08-2311152020-11.html} 
\item
  2021年01月09日20時57分の登録:
  \井上雄樹 @yinoue1975\修習生の給費制やら予備試験やらに反対するところなんか既得権益側の態度そのものなんよな。
  \url{http://kk2020-09.blogspot.com/2021/01/yinoue1975.html} 
\item
  2021年01月24日16時38分の登録:
  \井上雄樹 @yinoue1975\弁護士が勝ち目があることにしかツイートしなくなったらTwitterじゃない
  \url{http://kk2020-09.blogspot.com/2021/01/yinoue1975twitter.html} 
\item
  2021年01月24日21時18分の登録: \Shoko
  Egawa @amneris84\オウム井上に関するご著書を拝読した時から、氏の「客観的な視点」には大きな疑問を抱いてきました。その謎が晴れたのが今回。客観
  \url{http://kk2020-09.blogspot.com/2021/01/shoko-egawaamneris84\_4.html} 
\item
  2021年02月24日09時31分の登録:
  \郷原信郎【長いものには巻かれない・権力と戦う弁護士】 @nobuogohara\東北新社特別調査委員会の構成委員長:井上真一郎(弁護士)委 員:伊藤 良平(当社取締役副
  \url{https://kk2020-09.blogspot.com/2021/02/nobuogohara\_24.html} 
\item
  2021年03月13日07時07分の登録:
  \井上雄樹 @yinoue1975\法曹人口増加と法テラスの拡大は新興系事務所にもとってもウェルカムな状況なのが、重鎮系の先生には面白くないでしょうけどね。引用ツイート
  \url{https://kk2020-09.blogspot.com/2021/03/yinoue1975.html} 
\item
  2021年03月13日07時17分の登録:
  @yinoue1975(井上雄樹)のツイート ''.*'' 3128/3128:2018-09-20\_1430〜2021-03-12\_1703 2021年03月13日07時16分の記録
  \url{https://kk2020-09.blogspot.com/2021/03/yinoue1975312831282018-09-2014302021-03.html} 
\item
  2021年03月19日21時32分の登録:
  \井上雄樹 @yinoue1975\経験的には、警察官から出された飲み物に混入されたなどという弁解は裁判所からは一顧だにされないことが普通で、検察官も特に弁解を潰す立証な
  \url{https://kk2020-09.blogspot.com/2021/03/yinoue1975\_19.html} 
\item
  2021年04月19日06時23分の登録:
  \井上雄樹 @yinoue1975\回収を求めるなどと言われるとかえって買えるうちに買おうと思うのが人情だろう。文春の思う壺ではないか。
  \url{https://kk2020-09.blogspot.com/2021/04/yinoue1975.html} 
\end{itemize}

 上記は,「d\textbar grep
井上」の検索結果で,目に着いた2020年09月13日07時44分の登録,以降の記録になります。

\begin{itemize}
\tightlist
\item
  〈〈〈 2021/04/20 20:51:03 Linux Emacs: 〈〈〈
\end{itemize}

\hypertarget{ux65edux5ddd14ux6b73ux5c11ux5973ux30a4ux30b8ux30e1ux51cdux6b7bux4e8bux4ef6ux306bux5411ux3051ux305fux3068ux601dux308fux308cux308bux5211ux88c1ux30b5ux30a4ux592auwaaaaux306eux30c4ux30a4ux30fcux30c8ux30eaux30c4ux30a4ux30fcux30c8ux306eux8a18ux9332}{%
\paragraph{「旭川14歳少女イジメ凍死事件」に向けたと思われる刑裁サイ太@uwaaaaのツイート・リツイートの記録}\label{ux65edux5ddd14ux6b73ux5c11ux5973ux30a4ux30b8ux30e1ux51cdux6b7bux4e8bux4ef6ux306bux5411ux3051ux305fux3068ux601dux308fux308cux308bux5211ux88c1ux30b5ux30a4ux592auwaaaaux306eux30c4ux30a4ux30fcux30c8ux30eaux30c4ux30a4ux30fcux30c8ux306eux8a18ux9332}}

\begin{itemize}
\tightlist
\item
  〉〉〉 Linux Emacs: 2021/04/20 20:53:07 〉〉〉
\end{itemize}

:CATEGORIES: @kanazawabengosi \#金沢弁護士会 @JFBAsns
日本弁護士連合会(日弁連) \#法務省 @MOJ\_HOUMU \#刑裁サイ太 \#自殺
\#学校 \#いじめ

\begin{itemize}
\tightlist
\item
  2021年04月20日20時54分の登録:
  @uwaaaa(サイ太)のツイート ''.*'' 3233/3233:2020-09-17\_2031〜2021-04-20\_1838 2021年04月20日20時54分の記録
  \url{https://kk2020-09.blogspot.com/2021/04/uwaaaa323332332020-09-1720312021-04.html} 
\end{itemize}

TW uwaaaa(サイ太) 日時: 2020-09-18 12:07 URL:
\url{https://twitter.com/uwaaaa/status/1306791986406592512\textgreater} {}
自分の担当している事件に関する論文を書いて引用するのって禁じ手なんじゃなかったでしたっけ? 相手方や裁判所を批判しなければOK?I

RT uwaaaa(サイ太)|atsumilaw(弁護士 渥美 陽子) 日時:2020-09-20
21:43/2020-09-20 21:32 URL:
\url{https://twitter.com/uwaaaa/status/1307661723273322498} 
\url{https://twitter.com/atsumilaw/status/1307658856093245446\textgreater} {}
明日20時からのTV東京「0.1%の奇跡!
逆転無罪ミステリー」に出演しています。番組内で取り上げられる袴田事件は、現在進行形の冤罪事件です。再審の扉を開くため今も闘っている袴田さんと弁護団にエールを送る意味でも、ぜひ多くの方にご覧いただきたいです。
¥\n \#袴田事件に無罪を \url{https://t.co/7hUIXl99AW} 

\begin{itemize}
\tightlist
\item
  RT uwaaaa(サイ太)|uwaaaa(サイ太) 日時:2020-09-23
  20:18/2020-07-03 10:46 URL: \url{https://t.co/ig0sIgbFcR} 
  \url{https://t.co/mJxwZX5dy6}  ¥\n \textgreater{}
  実名報道強化とネットリンチの抑止とを同時にやろうとしてるの,頭がおかしいんじゃないの?
\end{itemize}

RT uwaaaa(サイ太)|uwaaaa(サイ太) 日時:2020-09-23 20:19/2018-09-20
12:42 URL: \url{https://twitter.com/uwaaaa/status/1308727632368787456} 
\url{https://twitter.com/uwaaaa/status/1042619981735710720\textgreater} {}
事実上,実名報道は刑罰として機能しているのに,実名報道するか否かは報道機関のブラックボックスに委ねられていて,かつ簡易な不服申立手段がない現状が問題だよ

RT uwaaaa(サイ太)|uwaaaa(サイ太) 日時:2020-09-23 20:20/2019-08-22
11:45 URL: \url{https://twitter.com/uwaaaa/status/1308728008384024576} 
\url{https://twitter.com/uwaaaa/status/1164368072204963841\textgreater} {}
「なにが真実性の担保じゃ!!」 ¥\n
マスコミの実名報道に疑問を呈するため,真実の○ンポを晒した全裸中年男性。後日,実名報道された。

RT uwaaaa(サイ太)|uwaaaa(サイ太) 日時:2020-09-23 20:21/2019-04-16
19:23 URL: \url{https://twitter.com/uwaaaa/status/1308728206157963268} 
\url{https://twitter.com/uwaaaa/status/1118097542124724224\textgreater} {}
実名を「公表」される被疑者・被告人には事前に弁明の機会が与えられないばかりか,無罪になった瞬間に匿名報道になり,名誉回復の機会すら与えられない。

 「@uwaaaa(サイ太)のツイート ''.*'' 3233/3233:2020-09-17\_2031〜2021-04-20\_1838」という記録のはずが,2020-09-23
20:21のリツイートに突入していることに気が付きました。見たはずのツイートが幻のように見当たりませんでした。

\begin{itemize}
\item
  2021年04月01日06時29分の登録:
  \サイ太 @uwaaaa\戒告で済むのか,という印象。表現の自由ではあるけど,事件の相手方を呼び捨てにしたり誹謗中傷するのは行き過ぎていたよ
  \url{https://kk2020-09.blogspot.com/2021/04/uwaaaa.html} 
\item
  2021年04月01日22時13分の登録:
  \サイ太 @uwaaaa\これ、「全国の実名Twitter弁護士が大量に訴訟告知の怪」の方がそれっぽくてよかったなと投稿した直後から後悔してました
  \url{https://kk2020-09.blogspot.com/2021/04/uwaaaatwitter.html} 
\item
  2021年04月02日16時59分の登録:
  \サイ太 @uwaaaa\この種の話,「では問題です」みたいな形で聞くことによってノイズにしかならない深読み勢が発生してしまうメタ的な事情を考慮しないのかなっていつも思う
  \url{https://kk2020-09.blogspot.com/2021/04/uwaaaa\_2.html} 
\item
  2021年04月04日15時00分の登録:
  サイ太(@uwaaaa)/「国選」の検索結果 - Twilog:2021年04月04日15時00分 811件
  \url{https://kk2020-09.blogspot.com/2021/04/uwaaaa-twilog202104041500811.html} 
\item
  2021年04月04日16時18分の登録:
  REGEXP:''非常上告''/サイ太(@uwaaaa)の検索(2011-07-06〜2021-03-05/2021年04月04日16時18分の記録7件)
  \url{https://kk2020-09.blogspot.com/2021/04/regexpuwaaaa2011-07-062021-03.html} 
\item
  2021年04月04日16時18分の登録:
  REGEXP:''国選''/サイ太(@uwaaaa)の検索(2010-03-01〜2021-04-02/2021年04月04日16時17分の記録779件)
  \url{https://kk2020-09.blogspot.com/2021/04/regexpuwaaaa2010-03-012021-04.html} 
\item
  2021年04月04日16時41分の登録:
  REGEXP:''餌やり''/サイ太(@uwaaaa)の検索(2014-10-06〜2021-04-02/2021年04月04日16時41分の記録25件)
  \url{https://kk2020-09.blogspot.com/2021/04/regexpuwaaaa2014-10-062021-04.html} 
\item
  2021年04月04日16時42分の登録:
  REGEXP:''ペット''/サイ太(@uwaaaa)の検索(2014-10-06〜2021-04-02/2021年04月04日16時41分の記録55件)
  \url{https://kk2020-09.blogspot.com/2021/04/regexpuwaaaa2014-10-062021-04\_4.html} 
\item
  2021年04月05日21時38分の登録:
  \サイ太 @uwaaaa\お願い事を引き受けてるのに,お願い事してる側から舐めた対応されるのはなんなの?
  \url{https://kk2020-09.blogspot.com/2021/04/uwaaaa\_5.html} 
\item
  2021年04月05日21時40分の登録:
  \サイ太 @uwaaaa\読んでいて,教師と保護者の関係も全く同じ構造な気がしました
  \url{https://kk2020-09.blogspot.com/2021/04/uwaaaa\_23.html} 
\item
  2021年04月07日12時26分の登録:
  \サイ太 @uwaaaa\津谷先生の事件と同じ年に横浜弁護士会(当時)の若手弁護士も離婚事件の当事者に刺殺されています
  \url{https://kk2020-09.blogspot.com/2021/04/uwaaaa\_7.html} 
\item
  2021年04月07日14時06分の登録:
  \サイ太 @uwaaaa\利益相反で回避しまくってたら地方じゃやっていけないよ
  \url{https://kk2020-09.blogspot.com/2021/04/uwaaaa\_90.html} 
\item
  2021年04月07日14時08分の登録:
  \サイ太 @uwaaaa\弁護士殺害で無期懲役判決、横浜地裁: 日本経済新聞
  \url{https://kk2020-09.blogspot.com/2021/04/uwaaaa\_68.html} 
\item
  2021年04月11日00時55分の登録:
  \サイ太 @uwaaaa\川崎殺傷「1人で死ねば」の声 事件や自殺誘うと懸念も
  {[}川崎の19人殺傷事件{]}:朝日新聞デジタル
  \url{https://kk2020-09.blogspot.com/2021/04/uwaaaa\_11.html} 
\item
  2021年04月11日00時57分の登録:
  \サイ太 @uwaaaa\元棋士の話と小室さんの話で,当事者一方からの主張でしかない話を受け入れる態度に違いがある人が大量発生していそうな気がする
  \url{https://kk2020-09.blogspot.com/2021/04/uwaaaa\_75.html} 
\item
  2021年04月11日00時58分の登録:
  \サイ太 @uwaaaa\業務妨害案件に発展しそうで心配です
  \url{https://kk2020-09.blogspot.com/2021/04/uwaaaa\_93.html} 
\item
  2021年04月12日23時20分の登録:
  \サイ太 @uwaaaa\深澤先生は侮辱罪が厳罰化されれば確実に仕事が増える側の弁護士で,そういう弁護士が反対しているということは本質情報なわけです。
  \url{https://kk2020-09.blogspot.com/2021/04/uwaaaa\_12.html} 
\item
  2021年04月19日14時46分の登録:
  \サイ太 @uwaaaa\日々生起する事件にそんなに感情移入してたら生きるの辛くありませんか?
  \url{https://kk2020-09.blogspot.com/2021/04/uwaaaa\_19.html} 
\item
  2021年04月19日14時50分の登録:
  \サイ太 @uwaaaa\他人を加害した人物を全くの第三者が正義棒でぶったたいてるとき,脳のどの部分が活性化するんですかね
  \url{https://kk2020-09.blogspot.com/2021/04/uwaaaa\_26.html} 
\item
  2021年04月19日14時52分の登録:
  \サイ太 @uwaaaa\あんなド直球の非弁太郎からサイ太ブログを懲戒するとか言われてたと考えたら涙が出てきちゃった
  \url{https://kk2020-09.blogspot.com/2021/04/uwaaaa\_77.html} 
\item
  2021年04月19日14時55分の登録:
  \サイ太 @uwaaaa\本当に国は住民訴訟がないからやりたい放題だよなあ
  \url{https://kk2020-09.blogspot.com/2021/04/uwaaaa\_27.html} 
\item
  2021年04月19日22時11分の登録:
  \サイ太 @uwaaaa\月9「イチケイのカラス」に疑問の声 裁判官は「捜査」しない(デイリースポーツ)
  - Yahoo!ニュース
  \url{https://kk2020-09.blogspot.com/2021/04/uwaaaa-yahoo.html} 
\item
  2021年04月20日18時02分の登録:
  \サイ太 @uwaaaa\これで訴えられたらガンプラを経費で買えるのでトータルでプラス
  \url{https://kk2020-09.blogspot.com/2021/04/uwaaaa\_20.html} 
\item
  2021年04月20日20時54分の登録:
  @uwaaaa(サイ太)のツイート ''.*'' 3233/3233:2020-09-17\_2031〜2021-04-20\_1838 2021年04月20日20時54分の記録
  \url{https://kk2020-09.blogspot.com/2021/04/uwaaaa323332332020-09-1720312021-04.html} 
\item
  TW uwaaaa(サイ太) 日時: 2021/04/19 14:31:08 URL:
  \url{https://twitter.com/uwaaaa/status/1384016694281007110} 
  \textgreater{}
  他人を加害した人物を全くの第三者が正義棒でぶったたいてるとき,脳のどの部分が活性化するんですかね
\item
  (1/100) TW uwaaaa(サイ太) 日時: 2021-04-19 14:39 URL:
  \url{https://twitter.com/uwaaaa/status/1384018863252144134\textgreater} {}
  感情移入してなくても日々大変だし生きづらさを感じるのに,異常に感情移入してる人は大丈夫なのか本気で不安になる
\end{itemize}

※ @kk\_hironoのアカウントがブロックされ,リツイートに失敗したツイート

\begin{itemize}
\tightlist
\item
  TW uwaaaa(サイ太) 日時:2021/04/19 14:39:45 URL:
  \url{https://twitter.com/uwaaaa/status/1384018863252144134} 
  \textgreater{}
  感情移入してなくても日々大変だし生きづらさを感じるのに,異常に感情移入してる人は大丈夫なのか本気で不安になる
\end{itemize}

 ツイート自体は削除されていないようです。

\begin{itemize}
\item
  (2/100) TW uwaaaa(サイ太) 日時: 2021-04-19 14:39 URL:
  \url{https://twitter.com/uwaaaa/status/1384018694670405634\textgreater} {}
  日々生起する事件にそんなに感情移入してたら生きるの辛くありませんか?
\item
  (3/100) TW uwaaaa(サイ太) 日時: 2021-04-19 14:31 URL:
  \url{https://twitter.com/uwaaaa/status/1384016694281007110\textgreater} {}
  他人を加害した人物を全くの第三者が正義棒でぶったたいてるとき,脳のどの部分が活性化するんですかね
\item
  (4/100) TW uwaaaa(サイ太) 日時: 2021-04-19 13:22 URL:
  \url{https://twitter.com/uwaaaa/status/1383999498192506882\textgreater} {}
  何かあったときに責任を取って貰える人のことしか信用しちゃダメですよね。
\item
  (13/100) RT uwaaaa(サイ太)|kosaka\_daimaou(古坂大魔王)
  日時:2021-04-19 10:55/2021-04-19 09:49 URL:
  \url{https://twitter.com/uwaaaa/status/1383962470834851848} 
  \url{https://twitter.com/kosaka\_daimaou/status/1383945786879209475\textgreater} {}
  はらわた煮え繰り返る、悲惨な事件のニュースを読んで即感情的に反応したくなる。マジで責任者や加害者に。俺もかなり我慢してる。でも、我々は少ない情報しか持ってなく、前後関係が定かではない。そんな時に第二第三の被害者を作らないように心掛ける。難しいけど。部外者には司法を見守るしかない。
\item
  (15/100) TW uwaaaa(サイ太) 日時: 2021-04-18 01:02 URL:
  \url{https://twitter.com/uwaaaa/status/1383450774156374026\textgreater} {}
  @yattoraren 警察はおこがましいので,自敬団と名乗ってください
\item
  (17/100) TW uwaaaa(サイ太) 日時: 2021-04-18 00:27 URL:
  \url{https://twitter.com/uwaaaa/status/1383442031649820681\textgreater} {}
  本当に国は住民訴訟がないからやりたい放題だよなあ
\item
  奉納\危険生物・弁護士脳汚染除去装置\金沢地方検察庁御中\_2020:
  \サイ太 @uwaaaa\他人を加害した人物を全くの第三者が正義棒でぶったたいてるとき,脳のどの部分が活性化するんですかね
  \url{https://kk2020-09.blogspot.com/2021/04/uwaaaa\_26.html} 
\end{itemize}

 上記のページを「旭川」でページ内検索してもサイドバーの履歴以外に該当はありませんでした。

 次にスクリーンショットの記録です。

\begin{itemize}
\item
  2021-04-01 22:31:26
  ``2021-04-01-221336\_サイ太@uwaaaa·1時間これ、「全国の実名Twitter弁護士が大量に訴訟告知の''怪''」の方がそれっぽくてよかったなと投稿した直後から後.jpg
  \url{http://pic.twitter.com/Md4cN1t50A}  "
  \url{https://twitter.com/s\_hirono/status/1377614583872024580} 
\item
  2021-04-01 22:31:43
  ``2021-04-01-221448\_サイ太@uwaaaa他称・ビジネス法務系スター弁護士が,日常業務の話から法律ニュース・司法制度の問題点,法曹養成・司法試験関係,おもしろ裁判.jpg
  \url{http://pic.twitter.com/DiMHvetU98''} 
  \url{https://twitter.com/s\_hirono/status/1377614657238659073} 
\item
  2021-04-02 17:00:55
  ``2021-04-02-165951\_サイ太@uwaaaa·5時間この種の話,「では問題です」みたいな形で聞くことによってノイズにしかならない深読み勢が発生してしまうメタ的な事情.jpg
  \url{http://pic.twitter.com/Os5bMmk6yl''} 
  \url{https://twitter.com/s\_hirono/status/1377893796357107713} 
\item
  2021-04-02 17:01:13
  ``2021-04-02-170017\_サイ太@uwaaaa·6時間中心人物の精神がようやく安定しただけではないかと思いました.jpg
  \url{http://pic.twitter.com/KcRDIrcYvy''} 
  \url{https://twitter.com/s\_hirono/status/1377893869648371719} 
\item
  2021-04-02 17:11:35
  ``2021-04-02-170102\_サイ太@uwaaaa·1時間こういう人たちの動物愛護精神で,国選のペット餌やりにも報酬を付けて欲しい.jpg
  \url{http://pic.twitter.com/Rbkto12U5p''} 
  \url{https://twitter.com/s\_hirono/status/1377896479658889219} 
\item
  2021-04-04 14:54:15
  ``2021-04-04-145337\_サイ太(@uwaaaa) - Twilog - Google Chrome.jpg
  \url{http://pic.twitter.com/YNPj2f3FpA''} 
  \url{https://twitter.com/s\_hirono/status/1378586696086773765} 
\item
  2021-04-04 14:54:33
  ``2021-04-04-145353\_サイ太(@uwaaaa) - Twilog - Google Chrome.jpg
  \url{http://pic.twitter.com/z7jBindC3E''} 
  \url{https://twitter.com/s\_hirono/status/1378586768522375168} 
\item
  2021-04-04 21:05:51
  ``2021-04-04-160344\_サイ太@uwaaaa非常上告が役に立った事例.jpg
  \url{http://pic.twitter.com/ME7he9DmeS''} 
  \url{https://twitter.com/s\_hirono/status/1378680210464251911} 
\item
  2021-04-04 21:06:08
  ``2021-04-04-160840\_サイ太@uwaaaa非常上告が役に立った事例posted at 11:45:05.jpg
  \url{http://pic.twitter.com/5Dt9nJhl5n''} 
  \url{https://twitter.com/s\_hirono/status/1378680283059290115} 
\item
  2021-04-05 21:41:42
  ``2021-04-05-213846\_深澤諭史さんがリツイートサイ太@uwaaaa·5時間お願い事を引き受けてるのに,お願い事してる側から舐めた対応されるのはなんなの?.jpg
  \url{http://pic.twitter.com/4LhTP5gV2w''} 
  \url{https://twitter.com/s\_hirono/status/1379051620747472899} 
\item
  2021-04-05 21:41:59
  ``2021-04-05-213958\_サイ太@uwaaaa·5時間弁護士会という団体のことなんですけども.jpg
  \url{http://pic.twitter.com/H40curUJw6''} 
  \url{https://twitter.com/s\_hirono/status/1379051693363503108} 
\item
  2021-04-05 21:42:17
  ``2021-04-05-214057\_サイ太@uwaaaa読んでいて,教師と保護者の関係も全く同じ構造な気がしました.jpg
  \url{http://pic.twitter.com/zSClRfDh0q''} 
  \url{https://twitter.com/s\_hirono/status/1379051766042353664} 
\item
  2021-04-07 12:27:07
  ``2021-04-07-122621\_深澤諭史さんがリツイートサイ太@uwaaaa·15時間すっかり忘れ去られているけど,津谷先生の事件と同じ年に横浜弁護士会(当時)の若手弁護.jpg
  \url{https://t.co/Mqk2YBIgf7''} 
  \url{https://twitter.com/s\_hirono/status/1379636831122124804} 
\item
  2021-04-07 22:13:48
  ``2021-04-07-221226\_サイ太@uwaaaa日弁連に対する悪意を全面に押し出すとこういう紹介の仕方になるんですが,実際は,200以上ある資格関連の法律で一斉に,成年.jpg
  \url{https://t.co/57v0R0LJ0q''} 
  \url{https://twitter.com/s\_hirono/status/1379784473726701572} 
\item
  2021-04-07 22:14:05
  ``2021-04-07-221313\_サイ太@uwaaaa成年被後見人になっても弁護士業務をすることが可能になる制度が検討されてるの? しかも日弁連は阻止するのは無理そうだからと.jpg
  \url{https://t.co/zluTvlrGFV''} 
  \url{https://twitter.com/s\_hirono/status/1379784546284019712} 
\item
  2021-04-12 23:21:45
  ``2021-04-12-232058\_深澤諭史さんがリツイートサイ太@uwaaaa·3時間深澤先生は侮辱罪が厳罰化されれば確実に仕事が増える側の弁護士で,そういう弁護士が反対し.jpg
  \url{https://t.co/1yaeWLk1pz''} 
  \url{https://twitter.com/s\_hirono/status/1381613513890664457} 
\item
  2021-04-19 14:55:43
  ``2021-04-19-144612\_サイ太@uwaaaa·6分日々生起する事件にそんなに感情移入してたら生きるの辛くありませんか?.jpg
  \url{https://t.co/rrZRGxK5qR''} 
  \url{https://twitter.com/s\_hirono/status/1384022882364268547} 
\item
  2021-04-19 14:56:01
  ``2021-04-19-145045\_サイ太@uwaaaa·19分他人を加害した人物を全くの第三者が正義棒でぶったたいてるとき,脳のどの部分が活性化するんですかね.jpg
  \url{https://t.co/1kiLVsxuaN''} 
  \url{https://twitter.com/s\_hirono/status/1384022955307372546} 
\item
  2021-04-19 14:56:18
  ``2021-04-19-145230\_サイ太@uwaaaa·3時間あんなド直球の非弁太郎からサイ太ブログを懲戒するとか言われてたと考えたら涙が出てきちゃった.jpg
  \url{https://t.co/LeZhP1CZ7R''} 
  \url{https://twitter.com/s\_hirono/status/1384023027721990144} 
\item
  2021-04-19 14:56:53
  ``2021-04-19-145516\_サイ太@uwaaaa·4月18日本当に国は住民訴訟がないからやりたい放題だよなあ.jpg
  \url{https://t.co/c6utqc8wP0''} 
  \url{https://twitter.com/s\_hirono/status/1384023173272801286} 
\item
  〈〈〈 2021/04/20 21:27:13 Linux Emacs: 〈〈〈
\end{itemize}

\hypertarget{ux65edux5ddd14ux6b73ux5c11ux5973ux30a4ux30b8ux30e1ux51cdux6b7bux4e8bux4ef6ux306bux5411ux3051ux305fux3068ux601dux308fux308cux308bux3046ux306eux5b57un_co_the2ndux306eux30c4ux30a4ux30fcux30c8ux30eaux30c4ux30a4ux30fcux30c8ux306eux8a18ux9332}{%
\paragraph{旭川14歳少女イジメ凍死事件」に向けたと思われるうの字@un\_co\_the2ndのツイート・リツイートの記録}\label{ux65edux5ddd14ux6b73ux5c11ux5973ux30a4ux30b8ux30e1ux51cdux6b7bux4e8bux4ef6ux306bux5411ux3051ux305fux3068ux601dux308fux308cux308bux3046ux306eux5b57un_co_the2ndux306eux30c4ux30a4ux30fcux30c8ux30eaux30c4ux30a4ux30fcux30c8ux306eux8a18ux9332}}

\begin{itemize}
\tightlist
\item
  〉〉〉 Linux Emacs: 2021/04/20 21:50:08 〉〉〉
\end{itemize}

:CATEGORIES: @kanazawabengosi \#金沢弁護士会 @JFBAsns
日本弁護士連合会(日弁連) \#法務省 @MOJ\_HOUMU \#うの字 \#自殺 \#学校

\begin{itemize}
\tightlist
\item
  2021年04月01日18時40分の登録:
  \うの字 @un\_co\_the2nd\弁護人ぼうチン要旨プッチンプリンは礼拝所に該当しない。プッチンプリンはプッチンしなくても美味しく、美味しくいただく行為は不敬な行為た
  \url{https://kk2020-09.blogspot.com/2021/04/uncothe2nd.html} 
\item
  2021年04月02日13時05分の登録:
  \うの字 @un\_co\_the2nd\自閉症と方言についての研究書に書かれている例題の意味が分からないのは自閉症だからなのでしょうか?→問題について賛否様々な声
  - To \url{https://kk2020-09.blogspot.com/2021/04/uncothe2nd-to.html} 
\item
  2021年04月05日04時30分の登録:
  \うの字 @un\_co\_the2nd\心が弱ってる人に対してやってるんだからめちゃくちゃ悪質だよな。癌が治るっつって祈祷して壺売るのと大差ないで
  \url{https://kk2020-09.blogspot.com/2021/04/uncothe2nd\_5.html} 
\item
  2021年04月05日04時35分の登録:
  \うの字 @un\_co\_the2nd\子供は死んでも守らなければならないという言葉に喝采を送るけど養育費は払わない、まさかそんな人いないよね?
  \url{https://kk2020-09.blogspot.com/2021/04/uncothe2nd\_49.html} 
\item
  2021年04月05日04時37分の登録:
  \うの字 @un\_co\_the2nd\アディーレの人が「債務整理を処理するリソースが十分ある」って言ってても、弁護士が依頼者から送付された資料を全く見ず、「担当弁護士など
  \url{https://kk2020-09.blogspot.com/2021/04/uncothe2nd\_74.html} 
\item
  2021年04月05日05時05分の登録:
  \うの字 @un\_co\_the2nd\新しい相手みつけたら前婚の子のことなんかいないことにする(もう会わない、遺産を1円も渡したくないとか言い出す)奴を散々見てるからな、
  \url{https://kk2020-09.blogspot.com/2021/04/uncothe2nd1.html} 
\item
  2021年04月05日05時08分の登録:
  \うの字 @un\_co\_the2nd\\#デポン先生今ひとときの海鮮丼
  \url{https://kk2020-09.blogspot.com/2021/04/uncothe2nd\_60.html} 
\item
  2021年04月08日11時45分の登録:
  \うの字 @un\_co\_the2nd\モラハラ主張されるのもむべなるかな(何か見た)
  \url{https://kk2020-09.blogspot.com/2021/04/uncothe2nd\_8.html} 
\item
  2021年04月08日21時48分の登録:
  \うの字 @un\_co\_the2nd\「死んでやる!」って言われた場合、プライベートなら「おう死ね!さっさと死ねや!」ってお返しするけど、仕事で言われたら「それは脅しです
  \url{https://kk2020-09.blogspot.com/2021/04/uncothe2nd\_58.html} 
\item
  2021年04月08日21時49分の登録:
  \うの字 @un\_co\_the2nd\フフッてなった昔ながらの梅干しに負ける守護霊
  \url{https://kk2020-09.blogspot.com/2021/04/uncothe2nd\_94.html} 
\item
  2021年04月08日21時49分の登録: \らめーん @shouwarame\返信先:
  @un\_co\_the2ndさん「いつのご予定ですか?」って、聞いちゃったことがあります・・・。
  \url{https://kk2020-09.blogspot.com/2021/04/shouwarame-uncothe2nd.html} 
\item
  2021年04月08日21時50分の登録:
  \ぽぽひと@ダイエッター @popohito\返信先:
  @un\_co\_the2ndさん野田先生だったかが以前「それはそちらがお決めになることですので」と回答するというツイ
  \url{https://kk2020-09.blogspot.com/2021/04/popohito-uncothe2nd.html} 
\item
  2021年04月09日12時13分の登録:
  \うの字 @un\_co\_the2nd\登記を司法書士に頼んだら何十万も取られたというその明細を見て、登録免許税が何十万だったという定番の話。
  \url{https://kk2020-09.blogspot.com/2021/04/uncothe2nd\_9.html} 
\item
  2021年04月17日14時28分の登録: \坂本正幸 @sakamotomasayuk\返信先:
  @un\_co\_the2ndさん近くに住む
  \url{https://kk2020-09.blogspot.com/2021/04/sakamotomasayuk-uncothe2nd.html} 
\item
  2021年04月17日14時29分の登録:
  \うの字を名乗る?物 @un\_co\_the2nd\そういうとこやぞ・・・
  \url{https://kk2020-09.blogspot.com/2021/04/uncothe2nd\_17.html} 
\item
  2021年04月17日15時13分の登録:
  \うの字を名乗る?物 @un\_co\_the2nd\がやられたらよろしくです(今から言っとく)
  \url{https://kk2020-09.blogspot.com/2021/04/uncothe2nd\_35.html} 
\item
  2021年04月17日15時15分の登録:
  \うの字を名乗る?物 @un\_co\_the2nd\返信先:
  @popohitoさんいくさ・・・します?
  \url{https://kk2020-09.blogspot.com/2021/04/uncothe2nd-popohito.html} 
\item
  2021年04月20日02時06分の登録:
  \うの字を名乗る?物 @un\_co\_the2nd\弁護士同席の拒否=公の秩序を持ち込ませてなるものか→ウチらのノリですよねこれ
  \url{https://kk2020-09.blogspot.com/2021/04/uncothe2nd\_20.html} 
\item
  2021年04月20日09時22分の登録:
  \うの字を名乗る?物 @un\_co\_the2nd\自分の思い通りの条件じゃないと嫌だで子供との面会蹴る人って子供に会いたいんじゃなくて他人を自分の意思に従わせたいんでしょ
  \url{https://kk2020-09.blogspot.com/2021/04/uncothe2nd\_81.html} 
\item
  2021年04月20日21時46分の登録:
  \うの字を名乗る?物 @un\_co\_the2nd\自分の思い通りの条件じゃないと嫌だで子供との面会蹴る人って子供に会いたいんじゃなくて他人を自分の意思に従わせたいんでしょ
  \url{https://kk2020-09.blogspot.com/2021/04/uncothe2nd\_68.html} 
\item
  2021年04月20日21時47分の登録:
  \うの字を名乗る?物 @un\_co\_the2nd\校長までが、ウチらのノリ\textgreater\textgreater\textgreater\textgreater\textgreater\textgreater 公の秩序を主張しておる(「いじめの構造」内藤朝雄)
  \url{https://kk2020-09.blogspot.com/2021/04/uncothe2nd\_43.html} 
\item
  2021年04月20日21時48分の登録:
  \うの字を名乗る?物 @un\_co\_the2nd\弁護士同席の拒否=公の秩序を持ち込ませてなるものか→ウチらのノリですよねこれ
  \url{https://kk2020-09.blogspot.com/2021/04/uncothe2nd\_2.html} 
\end{itemize}

〉〉〉 kk\_hironoのリツイート 〉〉〉

\begin{itemize}
\item
  RT
  kk\_hirono(刑事告発・非常上告_金沢地方検察庁御中)|SotaKimura(木村草太)
  日時:2021-04-20 21:53/2021/04/20 09:22 URL:
  \url{https://twitter.com/kk\_hirono/status/1384490509599580161} 
  \url{https://twitter.com/SotaKimura/status/1384301373185871872} 
  \textgreater{}
  事実関係をろくに確認しない人に限って、「対話が大事」とか気軽に言うよね。
\item
  奉納\危険生物・弁護士脳汚染除去装置\金沢地方検察庁御中\_2020:
  \うの字を名乗る💩物 @un\_co\_the2nd\弁護士同席の拒否=公の秩序を持ち込ませてなるものか→ウチらのノリですよねこれ
  \url{https://t.co/RnSqwclQd2}  ¥\n 2021-04-19 12:07から100件:最新2021-04-20
  18:06という範囲(1日5時間59分)の取得
\item
  (3/100) RT
  un\_co\_the2nd(うの字を名乗る💩物)|Ichiro\_leadoff(山本一郎(Ichiro
  Yamamoto)) 日時:2021-04-20 17:09/2021-04-20 15:46 URL:
  \url{https://twitter.com/un\_co\_the2nd/status/1384418941477396481} 
  \url{https://twitter.com/Ichiro\_leadoff/status/1384398086252814341\textgreater} {}
  ワイが絶対に先に訴えられたい!\textgreater\textgreater{}
  これは譲れない! \url{https://t.co/Rhyq6LPnfn} 
\item
  (4/100) TW un\_co\_the2nd(うの字を名乗る💩物) 日時: 2021-04-20
  17:06 URL:
  \url{https://twitter.com/un\_co\_the2nd/status/1384418240168816641\textgreater} {}
  表示すると中身が見られる模様。保全出来る方はよろしく
  \url{https://t.co/76AGHj7UJG} 
\end{itemize}

〉〉〉 kk\_hironoのリツイート 〉〉〉

\begin{itemize}
\tightlist
\item
  RT
  kk\_hirono(刑事告発・非常上告_金沢地方検察庁御中)|s\_hirono(非常上告-最高検察庁御中\_ツイッター)
  日時:2021-04-20 22:02/2021/04/20 22:02 URL:
  \url{https://twitter.com/kk\_hirono/status/1384492710829039617} 
  \url{https://twitter.com/s\_hirono/status/1384492642889650182} 
  \textgreater{}
  2021-04-20-220016\_悲しい投稿を保存する弁護士集団@log4sadness【Twitter内の悪口対策】タレント・有名人等からの依頼に基づき、Twitter内の.jpg
  \url{https://t.co/T8kkCOrUgN} 
\end{itemize}

〉〉〉 kk\_hironoのリツイート 〉〉〉

\begin{itemize}
\item
  RT
  kk\_hirono(刑事告発・非常上告_金沢地方検察庁御中)|s\_hirono(非常上告-最高検察庁御中\_ツイッター)
  日時:2021-04-20 22:02/2021/04/20 22:00 URL:
  \url{https://twitter.com/kk\_hirono/status/1384492732844888073} 
  \url{https://twitter.com/s\_hirono/status/1384492089170223111} 
  \textgreater{}
  2021-04-20-215938\_悲しい投稿を保存する弁護士集団@log4sadness注意: このアカウントは一時的に制限されていますこのアカウントは不審な行為が確認されて.jpg
  \url{https://t.co/L14YhYdwo5} 
\item
  (6/100) RT un\_co\_the2nd(うの字を名乗る💩物)|lawkus(ystk)
  日時:2021-04-20 17:02/2021-04-20 16:41 URL:
  \url{https://twitter.com/un\_co\_the2nd/status/1384417153303019522} 
  \url{https://twitter.com/lawkus/status/1384411777123848194\textgreater} {}
  【お願い】【拡散希望】「誹謗中傷ログを保存する弁護士集団」→「悲しい投稿を保存する弁護士集団」の、できれば全てのツイートの魚拓持ってる人います?持ってたら是非ご提供いただきたいのですが。保存・警告された対象ツイートも合わせてご提供いただけると有り難いです。
\item
  (14/100) RT un\_co\_the2nd(うの字を名乗る💩物)|uwaaaa(サイ太)
  日時:2021-04-20 15:03/2021-04-20 15:01 URL:
  \url{https://twitter.com/un\_co\_the2nd/status/1384387158673297416} 
  \url{https://twitter.com/uwaaaa/status/1384386650361384961\textgreater} {}
  これはただの予言なのでセーフ \url{https://t.co/s9Q36iJOQr} 
\item
  (34/100) RT
  un\_co\_the2nd(うの字を名乗る💩物)|hdm1987(細々経営の若手弁護士)
  日時:2021-04-20 12:44/2021-04-20 10:45 URL:
  \url{https://twitter.com/un\_co\_the2nd/status/1384352203658326020} 
  \url{https://twitter.com/hdm1987/status/1384322259259584513\textgreater} {}
  弁護人「検察官よ、ニンゲンとは脆イモノダナ。これで被告人はムザイだ」\textgreater{}
  裁判官「無罪?まだ私との闘いが残っていますよ?まずは30\%からです」(ゴゴゴゴゴゴゴゴゴゴ)\textgreater{}
  書記官「傍聴人は逃げて下さい!闘気に巻き込まれ・・・
  \url{https://t.co/KHk2nMVeLL} 
\item
  (59/100) RT
  un\_co\_the2nd(うの字を名乗る💩物)|chemicalgroom(ゴルーグ28号)
  日時:2021-04-20 09:19/2021-04-20 08:54 URL:
  \url{https://twitter.com/un\_co\_the2nd/status/1384300554394824710} 
  \url{https://twitter.com/chemicalgroom/status/1384294245528334338\textgreater} {}
  記録読んでいて、証拠が統合されて簡素化されている結果、大人キッザニアみたいなことになっているんですね。今の裁判員裁判。
  \url{https://t.co/kQCe5CsIWC} 
\item
  (96/100) TW un\_co\_the2nd(うの字を名乗る💩物) 日時: 2021-04-19
  13:42 URL:
  \url{https://twitter.com/un\_co\_the2nd/status/1384004353434951681\textgreater} {}
  弁護士同席の拒否=公の秩序を持ち込ませてなるものか→ウチらのノリ\textgreater\textgreater{}
  ですよねこれ
\item
  (97/100) TW un\_co\_the2nd(うの字を名乗る💩物) 日時: 2021-04-19
  13:40 URL:
  \url{https://twitter.com/un\_co\_the2nd/status/1384003962974527491\textgreater} {}
  校長までが、ウチらのノリ\textgreater\textgreater\textgreater\textgreater\textgreater\textgreater 公の秩序を主張しておる(「いじめの構造」内藤朝雄)
  \url{https://t.co/5YwjfRWMtf} 
\item
  〈〈〈 2021/04/20 22:16:33 Linux Emacs: 〈〈〈
\end{itemize}

\hypertarget{ux65edux5ddd14ux6b73ux5c11ux5973ux30a4ux30b8ux30e1ux51cdux6b7bux4e8bux4ef6ux306bux5411ux3051ux305fux3068ux601dux308fux308cux308bux6751ux677eux8b19ux5f01ux8b77ux58ebux5c0fux7530ux539fux5e02ux795eux5948ux5dddux770cux5f01ux8b77ux58ebux4f1akm0bakeux306eux30c4ux30a4ux30fcux30c8ux30eaux30c4ux30a4ux30fcux30c8ux306eux8a18ux9332}{%
\paragraph{旭川14歳少女イジメ凍死事件」に向けたと思われる村松謙弁護士(小田原市・神奈川県弁護士会)@km0bakeのツイート・リツイートの記録}\label{ux65edux5ddd14ux6b73ux5c11ux5973ux30a4ux30b8ux30e1ux51cdux6b7bux4e8bux4ef6ux306bux5411ux3051ux305fux3068ux601dux308fux308cux308bux6751ux677eux8b19ux5f01ux8b77ux58ebux5c0fux7530ux539fux5e02ux795eux5948ux5dddux770cux5f01ux8b77ux58ebux4f1akm0bakeux306eux30c4ux30a4ux30fcux30c8ux30eaux30c4ux30a4ux30fcux30c8ux306eux8a18ux9332}}

\begin{itemize}
\tightlist
\item
  〉〉〉 Linux Emacs: 2021/04/20 22:33:21 〉〉〉
\end{itemize}

:CATEGORIES: @kanazawabengosi \#金沢弁護士会 @JFBAsns
日本弁護士連合会(日弁連) \#法務省 @MOJ\_HOUMU \#村松謙弁護士
\#神奈川県弁護士会 \#自殺 \#学校

※ @kk\_hironoのアカウントがブロックされ,リツイートに失敗したツイート

\begin{itemize}
\tightlist
\item
  TW km0bake(👁️つまらむ👁️) 日時:2021/04/19 12:48:02 URL:
  \url{https://twitter.com/km0bake/status/1383990747918069762} 
  \textgreater{} 校長がこれでは。。。 \url{https://t.co/sxsieGkCSp} 
\end{itemize}

〉〉〉 kk\_hironoのリツイート 〉〉〉

\begin{itemize}
\item
  RT
  kk\_hirono(刑事告発・非常上告_金沢地方検察庁御中)|idleness\_venomy(venomy)
  日時:2021-04-20 22:34/2021/04/19 12:42 URL:
  \url{https://twitter.com/kk\_hirono/status/1384500821765722113} 
  \url{https://twitter.com/idleness\_venomy/status/1383989438900932616} 
  \textgreater{}
  これをインタビューでベラベラしゃべる感覚が理解できない。
  \url{https://t.co/KAFzpjHqYL} 
\item
  「イジメはなかった。彼女の中には以前から死にたいって気持ちがあったんだと思います」旭川14歳女子凍死 中学校長を直撃
  \textbar{} 文春オンライン
  \url{https://bunshun.jp/articles/amp/44869?\_\_twitter\_impression=true} 
\item
  TW s\_hirono(非常上告-最高検察庁御中\_ツイッター) 日時: 2021-04-20
  22:36 URL:
  \url{https://twitter.com/s\_hirono/status/1384501317507244033\textgreater} {}
  2021-04-20-222951\_venomy@idleness\_venomyこれをインタビューでベラベラしゃべる感覚が理解できない。.jpg
  \url{https://t.co/O6taMpQAcq} 
\item
  TW s\_hirono(非常上告-最高検察庁御中\_ツイッター) 日時: 2021-04-20
  22:36 URL:
  \url{https://twitter.com/s\_hirono/status/1384501244924882953\textgreater} {}
  2021-04-20-222938\_つまらむ@km0bake·4月19日校長がこれでは。。。.jpg
  \url{https://t.co/p0i7j2ttq0} 
\item
  TW s\_hirono(非常上告-最高検察庁御中\_ツイッター) 日時: 2021-04-20
  22:36 URL:
  \url{https://twitter.com/s\_hirono/status/1384501172132671496\textgreater} {}
  2021-04-20-222723\_つまらむ@km0bake·4月19日をを、保護者のせいじゃない!というご意見で燃えとるな。。。いや、保護者や地域が先生の仕事を増やしている.jpg
  \url{https://t.co/4EJ3Pn8keZ} 
\item
  TW s\_hirono(非常上告-最高検察庁御中\_ツイッター) 日時: 2021-04-20
  22:35 URL:
  \url{https://twitter.com/s\_hirono/status/1384501098476544001\textgreater} {}
  2021-04-20-222617\_つまらむさんがリツイート嶋﨑量(弁護士)@shima\_chikara·4月18日#教師のバトン「教員が職場の労働条件に声をあげるのは、教.jpg
  \url{https://t.co/vSdYfHdZyv} 
\item
  TW s\_hirono(非常上告-最高検察庁御中\_ツイッター) 日時: 2021-04-20
  22:25 URL:
  \url{https://twitter.com/s\_hirono/status/1384498339291418627\textgreater} {}
  2021-04-20-222433\_つまらむ@km0bakeひええ。こわ。。。>「しつけとして」''保護命令''に反し元妻の家訪問・・・3人の子ども平手打ち 44歳男逮捕.jpg
  \url{https://t.co/UPNzObgUfl} 
\item
  TW s\_hirono(非常上告-最高検察庁御中\_ツイッター) 日時: 2021-04-20
  22:24 URL:
  \url{https://twitter.com/s\_hirono/status/1384498266591490052\textgreater} {}
  2021-04-20-222339\_弁護士 鬼澤秀昌@教育・NPO@onichang·4時間返信先: @km0bakeさんありがとうございます!!きっと弁護士の先生にも教師の先.jpg
  \url{https://t.co/w3Cx8y272X} 
\item
  2021年03月02日18時20分の登録:
  \?️つまらむ?️ @km0bake\「熊本地震の避難所で当時小学6年の女児に「わいせつな動画を見せた」として逮捕され、熊本家裁で刑事裁判の無罪に当たる不処分の決定を受け
  \url{https://kk2020-09.blogspot.com/2021/03/km0bake6.html} 
\item
  2021年03月03日15時16分の登録:
  \?️つまらむ?️ @km0bake\SNSやインターネット上における誹謗中傷等への対応について
  \textbar{} ニュース \textbar{} 宝塚歌劇公式ホームページ
  \url{https://kk2020-09.blogspot.com/2021/03/km0bakesns.html} 
\item
  2021年03月03日21時58分の登録:
  \?️つまらむ?️ @km0bake\黒島結菜きれいだよなぁ。。。目があると綺麗なものには目が行ってしまうよな。>ヒロイン黒島結菜「ちむどんどん」22年前期朝ドラ
  \url{https://kk2020-09.blogspot.com/2021/03/km0bake22.html} 
\item
  2021年03月07日19時34分の登録:
  \?️つまらむ?️ @km0bake\小田原なんか、つねに裁判官女性の方が多いんだけど。。。
  \url{https://kk2020-09.blogspot.com/2021/03/km0bake.html} 
\item
  2021年03月07日19時35分の登録:
  \?️つまらむ?️ @km0bake\100分で名著、寺田寅彦。どうしても「帝都物語」を思い出すわー。
  \url{https://kk2020-09.blogspot.com/2021/03/km0bake100.html} 
\item
  2021年03月07日19時36分の登録:
  \りっぴぃ @rippy08\この記事について、つまらむ先生 @km0bake
  のご意見聞きたいなと思っていたので、思い切ってリプ飛ばしてみます。私としては、共感するとこ
  \url{https://kk2020-09.blogspot.com/2021/03/rippy08-km0bake.html} 
\item
  2021年03月07日19時46分の登録:
  \?️つまらむ?️ @km0bake\マチ弁やっていると、社会の中でも大変困難なことを見聞きした気になるのですが、まだまだ、いろいろなことがあるんですね。。。。
  \url{https://kk2020-09.blogspot.com/2021/03/km0bake\_7.html} 
\item
  2021年03月07日19時49分の登録:
  \?️つまらむ?️ @km0bake\公式の垢がつけるハッシュタグかよ。。。
  \url{https://kk2020-09.blogspot.com/2021/03/km0bake\_3.html} 
\item
  2021年03月12日03時57分の登録:
  \?️つまらむ?️ @km0bake\これ、単位会の法教育委員会と完全にかぶってるんですけど。。。
  \url{https://kk2020-09.blogspot.com/2021/03/km0bake\_25.html} 
\item
  2021年03月12日03時57分の登録:
  \?️つまらむ?️ @km0bake\そういえば、コングレス、法教育関係のサイドイベントやるの今日かしら??
  \url{https://kk2020-09.blogspot.com/2021/03/km0bake\_12.html} 
\item
  2021年03月12日03時58分の登録:
  \?️つまらむ?️ @km0bake\「お父さんは人殺し・・・」東電社員の家族が受けた「壮絶すぎるいじめの実態」
  @gendai\_biz \url{https://gendai.ism} 
  \url{https://kk2020-09.blogspot.com/2021/03/km0bake-gendaibiz-httpsgendaiism.html} 
\item
  2021年03月12日03時59分の登録:
  \?️つまらむ?️ @km0bake\2万もいいねがついているのだから、分かってくれる人も多くなってるのか。Twitterわーるど にそういう人が多いのか。
  \url{https://kk2020-09.blogspot.com/2021/03/km0bake2twitter.html} 
\item
  2021年03月12日04時00分の登録:
  \?️つまらむ?️ @km0bake\モトケン先生が、選択的夫婦別姓反対派の人と百人組み手しているんだけど、なかなか興味深いね。意見の違う人たちの意見を真摯に聞くというの
  \url{https://kk2020-09.blogspot.com/2021/03/km0bake\_63.html} 
\item
  2021年03月15日18時39分の登録:
  \?️つまらむ?️ @km0bake\「共感力」?が異常に発達?しちゃっているのはたしかにそうだよね。。。そこらの絵にも共感して、震えたり吐き気がしたり涙が出たり。
  \url{https://kk2020-09.blogspot.com/2021/03/km0bake\_15.html} 
\item
  2021年03月15日22時01分の登録: \モトケン @motoken\_tw\返信先:
  @km0bakeさん表現の自由問題では、表現者側が最低限の自重をしないと権力に規制の口実を与えるという話をするのですが、校則
  \url{https://kk2020-09.blogspot.com/2021/03/motokentw-km0bake.html} 
\item
  2021年03月15日22時03分の登録:
  \?️つまらむ?️ @km0bake\10%でも不快だと思ったら引っ込めろ、って相当すごいな。コロナの影響で、すこし心身の調子を崩されてるのではないか。。。心配。
  \url{https://kk2020-09.blogspot.com/2021/03/km0bake10.html} 
\item
  2021年03月15日22時05分の登録:
  \?️つまらむ?️ @km0bake\疲れちゃったので、月刊懲戒でもよみますか。。。
  \url{https://kk2020-09.blogspot.com/2021/03/km0bake\_41.html} 
\item
  2021年03月15日22時07分の登録:
  \?️つまらむ?️ @km0bake\再犯防止や法教育など14件の報告書を採択 京都コングレスが閉会
  \textbar{} 京都新聞
  \url{https://kk2020-09.blogspot.com/2021/03/km0bake14.html} 
\item
  2021年03月15日22時13分の登録:
  \?️つまらむ?️ @km0bake\このやじ、大変けしからんな。。。あと、共産党えらい。>違法捜査被害に県議が「やじ」 熊本県議会、控訴を可決(西日本新聞)
  \url{https://kk2020-09.blogspot.com/2021/03/km0bake\_2.html} 
\item
  2021年03月16日22時18分の登録:
  \?️つまらむ?️ @km0bake\これはいい話。.@TokioHelvetica
  さんの「Twitterでめちゃくちゃ美人の女性ライダーのアカウント「正体はおっさんな
  \url{https://kk2020-09.blogspot.com/2021/03/km0baketokiohelvetica-twitter.html} 
\item
  2021年03月16日22時19分の登録:
  \?️つまらむ?️ @km0bake\Twitterだけでも、ナベテル先生もやったことがあるとツイートしているし、結構あるもんなんですねぇ。
  \url{https://kk2020-09.blogspot.com/2021/03/km0baketwitter.html} 
\item
  2021年03月16日22時20分の登録:
  \?️つまらむ?️ @km0bake\一緒に旅館・ホテルに泊まった。って証拠があっても、友達と旅行に行ったで済みそうだもんなぁ。
  \url{https://kk2020-09.blogspot.com/2021/03/km0bake\_16.html} 
\item
  2021年03月16日22時21分の登録:
  \?️つまらむ?️ @km0bake\「ウマ娘」の非実在型炎上をデータから見る(鳥海不二夫)
  - Y!ニュース \url{https://kk2020-09.blogspot.com/2021/03/km0bake-y.html} 
\item
  2021年03月17日22時17分の登録:
  \?️つまらむ?️ @km0bake\をー!NHK、綱森先生だー!
  \url{https://kk2020-09.blogspot.com/2021/03/km0bakenhk.html} 
\item
  2021年03月19日15時27分の登録:
  \?️つまらむ?️ @km0bake\ひょえ~。この記事では、なんだか全然わからないけど、すごいな。3月だからなのか。。。>警察官が覚醒剤混入疑い、被告に無罪
  \textbar{} 202 \url{https://kk2020-09.blogspot.com/2021/03/km0bake-202.html} 
\item
  2021年03月19日15時27分の登録:
  \?️つまらむ?️ @km0bake\神奈川県弁護士会、次期副会長2人女性。会長副会長6人の内2人だから、まずまずか。これまで、執行部に女性2人ってあったかな?と思ったら
  \url{https://kk2020-09.blogspot.com/2021/03/km0bake2.html} 
\item
  2021年03月23日21時39分の登録:
  \?️つまらむ?️ @km0bake\法教育でお世話になってる塩川泰子先生、ご活躍だなぁ。RT
  @wwd\_jp 弁護士・学者・元裁判官によるファッションロー専門チームが誕
  \url{https://kk2020-09.blogspot.com/2021/03/km0bakert-wwdjp.html} 
\item
  2021年03月23日21時41分の登録:
  \?️つまらむ?️ @km0bake\娘へのわいせつ「身に覚えなかった」 無罪判決の外国人男性、日本語の調書分からず署名させられ(京都新聞)
  \url{https://kk2020-09.blogspot.com/2021/03/km0bake\_23.html} 
\item
  2021年03月23日21時56分の登録:
  \?️つまらむ?️ @km0bake\三月、2件、性犯罪無罪のニュース見かけたな。とうなるんやろか。。。
  \url{https://kk2020-09.blogspot.com/2021/03/km0bake\_67.html} 
\item
  2021年03月24日18時31分の登録:
  \?️つまらむ?️ @km0bake\すごい、謝罪も景気よく燃えてるな。。。
  \url{https://kk2020-09.blogspot.com/2021/03/km0bake\_24.html} 
\item
  2021年03月26日20時45分の登録:
  \?️つまらむ?️ @km0bake\高山佳奈子先生には、表現の自由関係のあちこちの裁判でお世話になりまくっているし、表現の自由界隈は、応援していきたいところ。
  \url{https://kk2020-09.blogspot.com/2021/03/km0bake\_26.html} 
\item
  2021年03月26日20時49分の登録:
  \?️つまらむ?️ @km0bake\ナベテル先生、超がんばってほしい!!
  \url{https://kk2020-09.blogspot.com/2021/03/km0bake\_30.html} 
\item
  2021年03月29日22時16分の登録:
  \?️つまらむ?️ @km0bake\法教育授業でいつか使えないかと思って、昔話法廷シリーズは全部録画してある。監修されてる今井先生の学会発表も拝見してるのだが、まだやっ
  \url{https://kk2020-09.blogspot.com/2021/03/km0bake\_29.html} 
\item
  2021年03月29日22時18分の登録:
  \?️つまらむ?️ @km0bake\強盗殺人、認め。死刑を選択するか?なかなかおもしろかった。>検察官に天海祐希、弁護人に佐藤浩市、桃太郎に仲野太賀! 昔話法廷
  ~「桃 \url{https://kk2020-09.blogspot.com/2021/03/km0bake\_73.html} 
\item
  2021年03月31日19時30分の登録:
  \?️つまらむ?️ @km0bake\アレ、戒告なんだ。。
  \url{https://kk2020-09.blogspot.com/2021/03/km0bake\_31.html} 
\item
  2021年04月05日02時18分の登録:
  \?️つまらむ?️ @km0bake\仙台の神坪せんせ登場\#法と教育学会春季研究集会
  \url{https://kk2020-09.blogspot.com/2021/04/km0bake.html} 
\item
  2021年04月08日22時00分の登録:
  \?️つまらむ?️ @km0bake\「元裁判官の松原里美弁護士(67)も「やった行為の悪さに応じて処遇を決めるのは成人の刑事司法の基本原則。それを少年法の中に持ってきて
  \url{https://kk2020-09.blogspot.com/2021/04/km0bake67.html} 
\item
  2021年04月08日22時01分の登録:
  \?️つまらむ?️ @km0bake\松原さん退官されたんだ。小田原におられたことあるけど、こういう熱い人だとは、全然知らなかった。
  \url{https://kk2020-09.blogspot.com/2021/04/km0bake\_8.html} 
\item
  2021年04月08日22時08分の登録:
  \?️つまらむ?️ @km0bake\Twitterワールド、車椅子のはパフォーマンスで悪だ、我こそ正義の味方!叩き潰してくれる!ってひととアレに少しでも疑いを持つ奴は差
  \url{https://kk2020-09.blogspot.com/2021/04/km0baketwitter.html} 
\item
  2021年04月09日17時19分の登録: \?️つまらむ?️ @km0bake\返信先:
  @motoken\_twさんう~ん。この問題、障害者の人たちにもうちょっと寄り添っていけませんかね?すぐに「差別主義者だ!」
  \url{https://kk2020-09.blogspot.com/2021/04/km0bake-motokentw.html} 
\item
  2021年04月11日00時53分の登録:
  \?️Οἰδίπους?️ @km0bake\特定のtweetにかみついてくる人は、僕みたいな一般人相手でも時々いる。粘着する人はそんなもんなんやろね。
  \url{https://kk2020-09.blogspot.com/2021/04/km0baketweet.html} 
\item
  2021年04月13日20時23分の登録:
  \?️つまらむ?️ @km0bake\「司法に救われた」性暴力被害の女子高校生 裁判に参加、尊厳取り戻す(岐阜新聞Web)\#Yahooニュース
  \url{https://kk2020-09.blogspot.com/2021/04/km0bakeyahoo.html} 
\item
  2021年04月13日20時28分の登録:
  \?️つまらむ?️ @km0bake\侮辱罪ー厳罰化の前にやるべきこと
  \url{https://minatokokusai.jp/blog/13532/}  \#\#minatokoku
  \url{https://kk2020-09.blogspot.com/2021/04/km0bake-httpsminatokokusaijpblog13532.html} 
\item
  2021年04月13日20時40分の登録:
  \?️つまらむ?️ @km0bake\厳罰化は、とりあえずなんでも反対。>「ネットの中傷問題に詳しい深沢諭史弁護士は「権力者が悪用し、自身に批判的な言論を告訴する可能性が
  \url{https://kk2020-09.blogspot.com/2021/04/km0bake\_13.html} 
\item
  2021年04月13日21時07分の登録:
  \?️つまらむ?️ @km0bake\もちろん、京大教授(憲法)という権威に従え!ってわけじゃなくて、そうとういいこと仰っていると思うんだけどなぁ。
  \url{https://kk2020-09.blogspot.com/2021/04/km0bake\_4.html} 
\item
  2021年04月13日21時08分の登録:
  \?️つまらむ?️ @km0bake\こういう感じで理解した大人ばかりなら、Twitterランドもかなり平和になるだろう。引用ツイート
  \url{https://kk2020-09.blogspot.com/2021/04/km0baketwitter\_13.html} 
\item
  2021年04月16日11時54分の登録: \?️つまらむ?️ @km0bake\春だね。。。
  \url{https://kk2020-09.blogspot.com/2021/04/km0bake\_16.html} 
\item
  2021年04月19日06時17分の登録:
  \?️つまらむ?️ @km0bake\弁護士登録しなかったのかとおもったら登録してるのね。
  \url{https://kk2020-09.blogspot.com/2021/04/km0bake\_19.html} 
\item
  2021年04月19日15時46分の登録:
  \?️つまらむ?️ @km0bake\もう出てるのか。読まなきゃ。>性犯罪に関する刑事法検討会 第15回会議(令和3年4月12日)「性犯罪に関する刑事法検討会」
  取りまと \url{https://kk2020-09.blogspot.com/2021/04/km0bake\_65.html} 
\item
  2021年04月19日15時48分の登録:
  \?️つまらむ?️ @km0bake\「公共の品位を侵害したこと」なんて構成要件許されるのか。。。ツイッタラーなんか根こそぎ駆除できそうな。。。>RT
  \url{https://kk2020-09.blogspot.com/2021/04/km0bake\_89.html} 
\item
  2021年04月19日19時07分の登録:
  \?️つまらむ?️ @km0bake\をを、保護者のせいじゃない!というご意見で燃えとるな。。。いや、保護者や地域が先生の仕事を増やしているのは確実にあるよ。。。
  \url{https://kk2020-09.blogspot.com/2021/04/km0bake\_55.html} 
\item
  2021年04月20日22時20分の登録:
  \?️つまらむ?️ @km0bake\たかまつなな氏は、完全に界隈からロックオンされてんのな。
  \url{https://kk2020-09.blogspot.com/2021/04/km0bake\_20.html} 
\item
  2021年04月20日22時21分の登録:
  \?️つまらむ?️ @km0bake\このツイをたたく人たちよ。。。そら業務減らんわ。。。
  \url{https://kk2020-09.blogspot.com/2021/04/km0bake\_40.html} 
\item
  2021年04月20日22時23分の登録: \弁護士
  鬼澤秀昌@教育・NPO @onichang\返信先:
  @km0bakeさんありがとうございます!!きっと弁護士の先生にも教師の先生にも役立てていただけるのではな
  \url{https://kk2020-09.blogspot.com/2021/04/npoonichang-km0bake.html} 
\item
  2021年04月20日22時24分の登録:
  \?️つまらむ?️ @km0bake\ひええ。こわ。。。>「しつけとして」保護命令に反し元妻の家訪問・・・3人の子ども平手打ち
  44歳男逮捕 \url{https://kk2020-09.blogspot.com/2021/04/km0bake3-44.html} 
\item
  2021年04月20日22時25分の登録:
  \?️つまらむ?️ @km0bake\同情すべき窃盗犯というと、最近は、万引き依存みたいなやつですかね。女性の摂食障害と関連があるみたい。
  \url{https://kk2020-09.blogspot.com/2021/04/km0bake\_9.html} 
\item
  2021年04月20日22時26分の登録:
  \?️つまらむ?️ @km0bake\読んだ。すごい力作>中学校校則の見直しを求める意見書
  福岡県弁護士会 \url{https://kk2020-09.blogspot.com/2021/04/km0bake\_78.html} 
\item
  2021年04月20日22時27分の登録:
  \?️つまらむ?️ @km0bake\をを、保護者のせいじゃない!というご意見で燃えとるな。。。いや、保護者や地域が先生の仕事を増やしているのは確実にあるよ。。。
  \url{https://kk2020-09.blogspot.com/2021/04/km0bake\_2.html} 
\item
  2021年04月20日22時28分の登録:
  \?️つまらむ?️ @km0bake\社会を防衛するために、萌え絵を撲滅したい人たちと似ていると思う。
  \url{https://kk2020-09.blogspot.com/2021/04/km0bake\_11.html} 
\item
  2021年04月20日22時29分の登録:
  \?️つまらむ?️ @km0bake\校長がこれでは。。。
  \url{https://kk2020-09.blogspot.com/2021/04/km0bake\_79.html} 
\item
  〈〈〈 2021/04/20 22:45:50 Linux Emacs: 〈〈〈
\end{itemize}

\hypertarget{ux5927ux6d25ux5712ux5150ux6b7bux50b7ux4e8bux6545}{%
\subsubsection{大津園児死傷事故}\label{ux5927ux6d25ux5712ux5150ux6b7bux50b7ux4e8bux6545}}

\hypertarget{ux5e745ux67087ux65e5ux88abux5bb3ux8005ux5f01ux8b77ux56e3ux3068ux907aux65cfux304cux76f4ux9032ux8ecaux3092ux904bux8ee2ux3057ux3066ux3044ux305f64ux6b73ux5973ux6027ux3092ux4e0dux8d77ux8a34ux3068ux3057ux305fux5927ux6d25ux5730ux691cux306eux51e6ux5206ux3092ux4e0dux670dux3068ux3057ux3066ux5927ux6d25ux691cux5bdfux5be9ux67fbux4f1aux306bux5be9ux67fbux3068ux3044ux3046ux30cbux30e5ux30fcux30b9}{%
\paragraph{2021年5月7日,被害者弁護団と遺族が「直進車を運転していた64歳女性を不起訴とした大津地検の処分を不服として、大津検察審査会に審査」というニュース}\label{ux5e745ux67087ux65e5ux88abux5bb3ux8005ux5f01ux8b77ux56e3ux3068ux907aux65cfux304cux76f4ux9032ux8ecaux3092ux904bux8ee2ux3057ux3066ux3044ux305f64ux6b73ux5973ux6027ux3092ux4e0dux8d77ux8a34ux3068ux3057ux305fux5927ux6d25ux5730ux691cux306eux51e6ux5206ux3092ux4e0dux670dux3068ux3057ux3066ux5927ux6d25ux691cux5bdfux5be9ux67fbux4f1aux306bux5be9ux67fbux3068ux3044ux3046ux30cbux30e5ux30fcux30b9}}

\begin{itemize}
\tightlist
\item
  〉〉〉 Linux Emacs: 2021/05/08 10:22:48 〉〉〉
\end{itemize}

:CATEGORIES: @kanazawabengosi \#金沢弁護士会 @JFBAsns
日本弁護士連合会(日弁連) \#法務省 @MOJ\_HOUMU \#大津園児死亡事故

\begin{itemize}
\tightlist
\item
  2021年05月08日09時53分の登録:
  REGEXP:''大津.*事故''/データベース登録済みツイートの検索:2021-05-07〜2021-05-08/2021年05月08日09時52分の記録:ユーザ・投稿:14/72件
  \url{https://kk2020-09.blogspot.com/2021/05/regexp2021-05-072021-05.html} 
\end{itemize}

\begin{quote}
《引用の始まり》
\end{quote}

\begin{quote}
アカウント名 ツイート数 リツイート数毎日新聞(mainichi) 2 0NEWS
JAPAN(NEWS\_JAPAN\_S) 14 0朝日新聞(asahi shimbun)(asahi) 2
0共同通信公式(kyodo\_official) 3 047NEWS(47news) 2
0共同通信ニュース NEWSmart(kyodo\_newsmart) 1
0中川登志男(ToshioNakagawa) 1 0共同通信社大阪社会部(kyodonewsosaka)
2 0奉納\さらば弁護士鉄道・泥棒神社の物語(hirono\_hideki) 13
15刑事告発・非常上告_金沢地方検察庁御中(kk\_hirono) 12
0テレ朝news(tv\_asahi\_news) 1 0マニアの受難@日本酒沼タヒタヒ(mt1q7q) 1
0時事ドットコム(時事通信ニュース)(jijicom) 2 0小川慎一/Shinichi
Ogawa 原発、再生エネ、気候変動(ogawashinichi) 1 0
\end{quote}

\begin{quote}
《引用の終わり》
\end{quote}

\begin{itemize}
\tightlist
\item
  奉納\危険生物・弁護士脳汚染除去装置\金沢地方検察庁御中\_2020:
  REGEXP:''大津.*事故''/データベース登録済みツイートの検索:2021-05-07〜2021-05-08/2021年05月08日09時52分の記録:ユーザ・投稿:14/72件 \url{https://kk2020-09.blogspot.com/2021/05/regexp2021-05-072021-05.htmln} 
\end{itemize}

 報道機関以外のアカウントは私のものを除外し,3件でしょうか。見慣れないアカウントが2つあって,1つはプロフィールの名前が見たことがないぐらい長いものです。小川慎一という名前には見覚えがあるのですが,弁護士ではないと思います。

\begin{itemize}
\item
  (08/72) TW ToshioNakagawa(中川登志男) 日時: 2021-05-07 11:57:28
  +0900 URL:
  \url{https://twitter.com/ToshioNakagawa/status/1390501004345364480\textgreater} {}
  大津園児事故、直進車不起訴に不服申し立て 遺族ら「注意義務怠った」(京都新聞)\textgreater{}
  \#Yahooニュース\textgreater{} \url{https://t.co/atykN96rxT} 
\item
  (62/72) TW mt1q7q(マニアの受難@日本酒沼タヒタヒ) 日時: 2021-05-07
  20:25:36 +0900 URL:
  \url{https://twitter.com/mt1q7q/status/1390628882718416897\textgreater} {}
  いやこれは気持ちはわからんでもないが不起訴やむなしやろ。\textgreater\textgreater{}
  大津園児事故、直進車不起訴に不服申し立て 遺族ら「注意義務怠った」(京都新聞)\textgreater{}
  \#Yahooニュース\textgreater{} \url{https://t.co/n5BUCr3Ih5} 
\end{itemize}

 比較的見覚えのあるアカウントですが,アイコンの方が変わったようにも思えます。タヒタヒというのは死ね死ねの意味らしく,大阪の弁護士がそれで懲戒処分を受けたことにより,法クラで話題となり,大阪弁護士会への抗議の意味合いでタヒタヒを使っているようです。

〉〉〉 kk\_hironoのリツイート 〉〉〉

\begin{itemize}
\tightlist
\item
  RT
  kk\_hirono(刑事告発・非常上告_金沢地方検察庁御中)|mt1q7q(マニアの受難@日本酒沼タヒタヒ)
  日時:2021-05-08 10:36/2021/05/08 06:29 URL:
  \url{https://twitter.com/kk\_hirono/status/1390842895091961858} 
  \url{https://twitter.com/mt1q7q/status/1390780783753764864} 
  \textgreater{} あたたかく見守ってほしい、だと?何を言っているんだ。
  国民の生命身体財産を犠牲にしてまで自分達が優遇されることを容認するなら、もはやアスリート全員も同罪だ。
  鬼畜の所業であり、同情の余地はない。 \url{https://t.co/tDhHRPTDKZ} 
\end{itemize}

〉〉〉 kk\_hironoのリツイート 〉〉〉

\begin{itemize}
\tightlist
\item
  RT
  kk\_hirono(刑事告発・非常上告_金沢地方検察庁御中)|horitsusodan(ふくろん)
  日時:2021-05-08 10:37/2021/05/07 12:12 URL:
  \url{https://twitter.com/kk\_hirono/status/1390843149648502786} 
  \url{https://twitter.com/horitsusodan/status/1390504723359272963} 
  \textgreater{} み な す \url{https://t.co/ElqvqZNmZu} 
\end{itemize}

〉〉〉 kk\_hironoのリツイート 〉〉〉

\begin{itemize}
\tightlist
\item
  RT
  kk\_hirono(刑事告発・非常上告_金沢地方検察庁御中)|mt1q7q(マニアの受難@日本酒沼タヒタヒ)
  日時:2021-05-08 10:38/2021/05/06 23:29 URL:
  \url{https://twitter.com/kk\_hirono/status/1390843424748695552} 
  \url{https://twitter.com/mt1q7q/status/1390312752036470791} 
  \textgreater{}
  司法浪人していたころ,何もかもうまくいかなくて首を吊る三歩手前まで行った。文化芸術だけが僕を裏切らなかった。間違いなく今も自分を含め誰かの救いであり続けている。
  \#文化芸術は生きるために必要だ
\end{itemize}

〉〉〉 kk\_hironoのリツイート 〉〉〉

\begin{itemize}
\tightlist
\item
  RT
  kk\_hirono(刑事告発・非常上告_金沢地方検察庁御中)|mt1q7q(マニアの受難@日本酒沼タヒタヒ)
  日時:2021-05-08 10:38/2021/05/06 23:23 URL:
  \url{https://twitter.com/kk\_hirono/status/1390843462522527748} 
  \url{https://twitter.com/mt1q7q/status/1390311380213923843} 
  \textgreater{}
  再逮捕の段階だから何とも言えないけど,すべて事実だとしたら,もう病気なんだろうな・・・
  女性を自宅に連れ込み性的暴行容疑 千葉市の弁護士を再逮捕:イザ!
  \url{https://t.co/QeMtioUdOR}  @iza\_editより
\end{itemize}

 アカウントのタイムラインから4つリツイートをしました。最後の「女性を自宅に連れ込み性的暴行容疑 千葉市の弁護士を再逮捕:イザ!」は見覚えのあるツイートです。まとめ記事も作成しましたが,余り多くない弁護士らの反応の1つでした。

 次に3つ目のアカウントのツイートです。

\begin{itemize}
\tightlist
\item
  (72/72) TW ogawashinichi(小川慎一/Shinichi
  Ogawa 原発、再生エネ、気候変動) 日時: 2021-05-08 08:13:29 +0900
  URL:
  \url{https://twitter.com/ogawashinichi/status/1390807026083581954\textgreater} {}
  青信号で法定速度以下で直進していた車の運転手が、刑事責任を問われていいとはなかなか思えない。\textgreater\textgreater{}
  大津園児事故で検審に申し立て 16人死傷、直進車不起訴は不服:東京新聞
  TOKYO Web \url{https://t.co/iZqKJXYcfT} 
\end{itemize}

\begin{quote}
《引用の始まり》
\end{quote}

\begin{quote}
[name]ユーザ名称:小川慎一/Shinichi Ogawa 原発、再生エネ、気候変動

[screen\_name]ユーザ名:ogawashinichi

位置情報:

ユーザ説明:東京新聞(中日新聞)社会部原発取材班(@kochigen2017)キャップ。気候変動問題をテーマにした「地球異変」も担当。1975年生まれ。https://t.co/O8jNaGhAEY
staff writer of The Tokyo Simbun, Japanese newspaper

ユーザのフォロワー数:5179

ユーザのフォロー数:1362

ユーザがTwitterに登録した日時:2009-07-18 04:24:29 UTC

ユーザの投稿ツイート数:79528
\end{quote}

\begin{quote}
《引用の終わり》
\end{quote}

\begin{itemize}
\tightlist
\item
  奉納\危険生物・弁護士脳汚染除去装置\金沢地方検察庁御中\_2020:
  REGEXP:''大津.*事故''/データベース登録済みツイートの検索:2021-05-07〜2021-05-08/2021年05月08日09時52分の記録:ユーザ・投稿:14/72件 \url{https://kk2020-09.blogspot.com/2021/05/regexp2021-05-072021-05.html\#ogawashinichin} 
\end{itemize}

 プロフィールを上記に引用しましたが,やはり報道関係者でした。ジャーナリストかと思っていたのははずれでしたが,「東京新聞(中日新聞)社会部原発取材班(@kochigen2017)キャップ。」というのは新たな発見です。

 こちらもタイムラインから参考になりそうなツイートをいくつかリツイートをしておきたいと思います。

〉〉〉 kk\_hironoのリツイート 〉〉〉

\begin{itemize}
\tightlist
\item
  RT
  kk\_hirono(刑事告発・非常上告_金沢地方検察庁御中)|tokyonewsroom(東京新聞編集局)
  日時:2021-05-08 10:48/2021/05/08 07:10 URL:
  \url{https://twitter.com/kk\_hirono/status/1390846008804798464} 
  \url{https://twitter.com/tokyonewsroom/status/1390791069252476928} 
  \textgreater{}
  復興五輪「架空だった」・・・罪悪感抱く宮本亞門さん、IOCや政府を「利己的」と批判 
  大会経費は膨れ上がり、福島第一原発事故の後処理も進まない、全て誘致のための架空のものだった。悲惨な現実を見て「何ということに加担してしまったんだ」と罪悪感にさいなまれました
  \url{https://t.co/ihOLrvPdpF} 
\end{itemize}

〉〉〉 kk\_hironoのリツイート 〉〉〉

\begin{itemize}
\tightlist
\item
  RT
  kk\_hirono(刑事告発・非常上告_金沢地方検察庁御中)|ogawashinichi(小川慎一/Shinichi
  Ogawa 原発、再生エネ、気候変動) 日時:2021-05-08 10:48/2021/05/08
  07:55 URL: \url{https://twitter.com/kk\_hirono/status/1390846086395297794} 
  \url{https://twitter.com/ogawashinichi/status/1390802447187148800} 
  \textgreater{}
  「福島第一原発事故の後処理も進まない」とあるけど、政府と東電のロードマップが現実離れしているとはいえ、収束作業は確実に進んではいます。
  復興五輪「架空だった」・・・罪悪感抱く宮本亞門さん、IOCや政府を「利己的」と批判 インタビュー詳報:東京新聞
  TOKYO Web \url{https://t.co/CeH5ubRlsW} 
\end{itemize}

〉〉〉 kk\_hironoのリツイート 〉〉〉

\begin{itemize}
\tightlist
\item
  RT
  kk\_hirono(刑事告発・非常上告_金沢地方検察庁御中)|tokyonewsroom(東京新聞編集局)
  日時:2021-05-08 10:49/2021/05/07 22:49 URL:
  \url{https://twitter.com/kk\_hirono/status/1390846227261001729} 
  \url{https://twitter.com/tokyonewsroom/status/1390665197598101510} 
  \textgreater{} 競泳池江選手に五輪の辞退求める声「とても苦しい」
  東京新聞 TOKYO Web \url{https://t.co/IKJgkhX2CX} 
\end{itemize}

〉〉〉 kk\_hironoのリツイート 〉〉〉

\begin{itemize}
\tightlist
\item
  RT
  kk\_hirono(刑事告発・非常上告_金沢地方検察庁御中)|ogawashinichi(小川慎一/Shinichi
  Ogawa 原発、再生エネ、気候変動) 日時:2021-05-08 10:50/2021/05/07
  23:22 URL: \url{https://twitter.com/kk\_hirono/status/1390846413102194689} 
  \url{https://twitter.com/ogawashinichi/status/1390673489917476871} 
  \textgreater{}
  あまり賛成できない。専門的な教育受けて、資格を持った人が働けるようにした方がいいのでは。たぶん、そういう資格作ってるはず。
  【独自】「心のサポーター」100万人養成・・・うつ病など不調に悩む人、地域で支援(読売新聞オンライン)
  \#Yahooニュース \url{https://t.co/pTnKypKlMx} 
\end{itemize}

〉〉〉 kk\_hironoのリツイート 〉〉〉

\begin{itemize}
\tightlist
\item
  RT
  kk\_hirono(刑事告発・非常上告_金沢地方検察庁御中)|cnn\_co\_jp(cnn\_co\_jp)
  日時:2021-05-08 10:50/2021/05/07 18:32 URL:
  \url{https://twitter.com/kk\_hirono/status/1390846643621097472} 
  \url{https://twitter.com/cnn\_co\_jp/status/1390600341717192705} 
  \textgreater{} 米陸軍、女性兵士にポニーテールを許可 すべての制服で
  \url{https://t.co/R0UZ0bTsdk} 
\end{itemize}

〉〉〉 kk\_hironoのリツイート 〉〉〉

\begin{itemize}
\tightlist
\item
  RT
  kk\_hirono(刑事告発・非常上告_金沢地方検察庁御中)|ogawashinichi(小川慎一/Shinichi
  Ogawa 原発、再生エネ、気候変動) 日時:2021-05-08 10:52/2021/05/07
  10:24 URL: \url{https://twitter.com/kk\_hirono/status/1390846995237986304} 
  \url{https://twitter.com/ogawashinichi/status/1390477637995302912} 
  \textgreater{}
  新聞記者になる前から、社説は毎日載せる必要があるのか疑問に思っている。しかも2本もいるのか。
\end{itemize}

 最初のアカウントもプロフィールを調べました。どうも弁護士ではなさそうですが,「認定NPO法人理事など。著書(共著)『学校選択と教育バウチャー』『教育格差』『地方選挙と政治』など。」という社会活動ことで,リストに入れたのだと思います。大学関係者と勘違いしていたかも。

\begin{quote}
《引用の始まり》
\end{quote}

\begin{quote}
中川登志男@ToshioNakagawa1974年生まれ。神奈川県立茅ヶ崎西浜高→専修大(法)→東京学芸大院(修士課程、教育制度)→専修大院(修士・博士課程、選挙制度)。地方議員の際に議会制度も多少勉強。現在は認定NPO法人理事など。著書(共著)『学校選択と教育バウチャー』『教育格差』『地方選挙と政治』など。神奈川県(横浜市生まれ、湘南地域在住)sites.google.com/site/gendainor・・・2013年1月からTwitterを利用しています658
フォロー中872 フォロワー
\end{quote}

\begin{quote}
《引用の終わり》
\end{quote}

\begin{itemize}
\item
  \begin{enumerate}
  \def\labelenumi{(\arabic{enumi})}
  \setcounter{enumi}{13}
  \tightlist
  \item
    中川登志男さん (@ToshioNakagawa) /
    Twitter \url{https://twitter.com/ToshioNakagawan} 
  \end{enumerate}
\end{itemize}

 結局,弁護士の可能性のあるアカウントは匿名の1件だけでした。大津園児死亡事故の話題は触れたくないという傾向がうかがえます。

\hypertarget{ux6771ux4eacux30aaux30eaux30f3ux30d4ux30c3ux30afux95a2ux9023}{%
\subsubsection{東京オリンピック関連}\label{ux6771ux4eacux30aaux30eaux30f3ux30d4ux30c3ux30afux95a2ux9023}}

\hypertarget{ux5f01ux8b77ux58ebux306eux793eux4f1aux7684ux4fe1ux7528ux306bux3064ux3044ux3066ux5927ux3044ux306bux8003ux3048ux3055ux305bux3089ux308cux308bux5143ux65e5ux672cux5f01ux8b77ux58ebux4f1aux4f1aux9577ux5b87ux90fdux5baeux5065ux5150ux5f01ux8b77ux58ebux306eux6771ux4eacux30aaux30eaux30f3ux30d4ux30c3ux30afux4e2dux6b62ux306eux7f72ux540dux6d3bux52d5}{%
\paragraph{弁護士の社会的信用について大いに考えさせられる,元日本弁護士会会長,宇都宮健児弁護士の東京オリンピック中止の署名活動}\label{ux5f01ux8b77ux58ebux306eux793eux4f1aux7684ux4fe1ux7528ux306bux3064ux3044ux3066ux5927ux3044ux306bux8003ux3048ux3055ux305bux3089ux308cux308bux5143ux65e5ux672cux5f01ux8b77ux58ebux4f1aux4f1aux9577ux5b87ux90fdux5baeux5065ux5150ux5f01ux8b77ux58ebux306eux6771ux4eacux30aaux30eaux30f3ux30d4ux30c3ux30afux4e2dux6b62ux306eux7f72ux540dux6d3bux52d5}}

\begin{itemize}
\tightlist
\item
  〉〉〉 Linux Emacs: 2021/05/08 11:22:43 〉〉〉
\end{itemize}

:CATEGORIES: @kanazawabengosi \#金沢弁護士会 @JFBAsns
日本弁護士連合会(日弁連) \#法務省 @MOJ\_HOUMU \#宇都宮健児弁護士

\begin{quote}
《引用の始まり》
\end{quote}

\begin{quote}
2021年05月08日11時21分の実行記録APIのリミットに達するので8500で処理と中断しました。twitterAPI-search-lawList-mydql-add.rb
``宇都宮健児''ツイート数:12/2409 リツイート数:11/2409
トータル:8500hirono\_hideki 3/2件kk\_hirono 0/0件s\_hirono 0/0件
\end{quote}

\begin{quote}
《引用の終わり》
\end{quote}

\begin{itemize}
\tightlist
\item
  2021年05月08日11時24分の登録:
  REGEXP:''宇都宮健児''/データベース登録済みツイートの検索:2021-05-07〜2021-05-08/2021年05月08日11時23分の記録:ユーザ・投稿:15/23件
  \url{https://kk2020-09.blogspot.com/2021/05/regexp2021-05-072021-05\_8.html} 
\end{itemize}

 とりあえず3日間の期間指定でまとめ記事を作成しましたが,期間の定めのない「宇都宮健児」をキーワードにするまとめ記事も現在作成中です。Twitterでトレンドを見かけたように思いますが,昨日のことであったように思います。

\begin{itemize}
\tightlist
\item
  (01/23) RT
  ekinan\_lawyer(えきなんローヤー?)|akahataseiji(赤旗政治記者)
  日時:2021-05-07 08:44:52 +0900/2021-05-07 07:08:00 +0900 URL:
  \url{https://twitter.com/ekinan\_lawyer/status/1390452536314793985} 
  \url{https://twitter.com/akahataseiji/status/1390428182105321473\textgreater} {}
  東京五輪反対署名15万筆超に 5日正午からオンラインで開始(スポニチ)\textgreater{}
  《元日弁連会長の宇都宮健児氏が「東京五輪の開催中止を求める」とするオンライン署名を5日正午から始め、7日午前0時時点で15万筆を超えた》\textgreater{}
  \url{https://t.co/ukdBAqb0ao} 
\end{itemize}

 3日の期間指定でしたが,昨日5月7日の午前8時44分のえきなんローヤーのリツイートで午前7時8分のツイートから始まっていました。報道機関のアカウントのニュースを紹介するツイートかと思ったのですが,よくみると「赤旗政治記者」というアカウントのツイートでした。

\begin{itemize}
\item
  (04/23) RT tokyo\_shimbun(東京新聞(TOKYO
  Web))|tokyonewsroom(東京新聞編集局) 日時:2021-05-07 11:22:38
  +0900/2021-05-07 06:56:00 +0900 URL:
  \url{https://twitter.com/tokyo\_shimbun/status/1390492237675393030} 
  \url{https://twitter.com/tokyonewsroom/status/1390425292674506752\textgreater} {}
  元日弁連会長の宇都宮健児氏が、「東京五輪の開催中止を求める」とするオンライン署名を5日正午から始め、7日午前0時時点で15万筆を超えた。\textgreater{}
  #東京五輪 #新型コロナ \textgreater{} \url{https://t.co/qJka6yYX2O} 
\item
  (05/23) TW HuffPostJapan(ハフポスト日本版 /
  会話を生み出す国際メディア) 日時: 2021-05-07 13:06:22 +0900 URL:
  \url{https://twitter.com/HuffPostJapan/status/1390518343929630728\textgreater} {}
  弁護士で元日弁連会長の宇都宮健児さんが、東京五輪の開催中止を求める署名活動を開始。\textgreater\textgreater{}
  2日間ですでに18万人以上の署名が集まっています。\textgreater{}
  \url{https://t.co/JRlWzNvwYw} 
\item
  (08/23) TW ichifuna\_law(弁護士
  髙橋裕樹(アトム市川船橋法律事務所代表)) 日時: 2021-05-07 19:20:03
  +0900 URL:
  \url{https://twitter.com/ichifuna\_law/status/1390612386424393729\textgreater} {}
  「日本側から中止を言っても違約金は発生しない」\textgreater{}
  法的な根拠・・・\textgreater\textgreater{}
  ``五輪中止署名''発起人・宇都宮健児氏が独白「五輪にあてる財源を生活困窮者に」(東スポWeb)
  \url{https://t.co/rd9MluCnm0} 
\end{itemize}

 上記の髙橋裕樹弁護士のツイートにある「五輪中止署名''発起人・宇都宮健児氏が独白「五輪にあてる財源を生活困窮者に」(東スポWeb)」は初めてみるものですが,宇都宮健児弁護士らしさを感じました。弱者の味方のようですが,もともと疑問があります。

\begin{itemize}
\item
  (15/23) TW sanngatuusagino(弁護士 上瀧浩子) 日時: 2021-05-08
  01:50:53 +0900 URL:
  \url{https://twitter.com/sanngatuusagino/status/1390710742010056708\textgreater} {}
  「日本に余裕ない」都知事選に3度出馬の宇都宮健児氏が五輪中止を求める署名活動スタート(東スポWeb)\textgreater{}
  \#Yahooニュース\textgreater{} \url{https://t.co/vLGxqM7IOK} 
\item
  宇都宮健児氏が激白「バッハ会長は決断を」 東京五輪中止まで署名続ける決意示す〈dot.〉(AERA
  dot.) - Yahoo!ニュース \url{https://t.co/oa54rWjYYF}  ¥\n 5/8(土) 8:00配信
  ¥\n
  人々の命と暮らしを守るために東京五輪の開催中止を求めます――こう呼びかけたオンライン上の署名活動が、開始わずか3日で24万筆を超
\end{itemize}

 上記の本日8日午前8時00分の記事が,現時点での最新情報かと思われますが,24万筆とのことです。これを多いとみるか意外に少ないとみるかは,人それぞれとは思います。いわゆるインフルエンサーになるのかと宇都宮健児弁護士に対する認識を改めました。

 東京都知事選などにも出馬していた宇都宮健児弁護士ですが,事実上の泡沫候補で,絶対に支持したくないという人も一定数いそうで,それが東京オリンピックの中止に逆効果に働き,他に中止を願う人の迷惑となる可能性にも思いを巡らせるところです。

 これは弁護士の社会的信頼,信用にも直結するところかと思います。宇都宮健児弁護士は元日本弁護士会会長としても知られていますが,弁護士の過払い金バブルの象徴的存在でもあるからです。

 過払い金問題は複数の訴訟があり,全てが宇都宮健児弁護士の功績というわけでもなさそうですが,一般に象徴的存在となっているのは間違いないと思います。また,本人は弱者救済でやったのかもしれないですが,その後は弁護士業界の過払い金バブルとして相当の悪印象を与えていると思います。

 ふと思ったのですが,ニュース記事は宇都宮健児氏で,弁護士とはなっておらず,弁護士を廃業しているのかと考えたのですが,記事の本文には元日弁連会長という肩書もつけられていました。廃業というより登録抹消が多いかもしれません。

 小倉秀夫弁護士も以前,宇都宮健児弁護士の功績を高く評価し,それを疑問を呈する批判者に対する詭弁の材料として利用し,批判者を愚弄しているとも感じました。

\begin{itemize}
\tightlist
\item
  2021年05月08日11時27分の登録:
  REGEXP:''宇都宮健児''/データベース登録済みツイート:2021年05月08日11時25分の記録:ユーザ・投稿:34/68件
  \url{https://kk2020-09.blogspot.com/2021/05/regexp2021050811253468.html} 
\end{itemize}

 ちょっと失敗したことに気がついたのですが,コマンドに第2引数を指定していませんでした。いつもは期間指定なしの場合,1000-01-01/2021-05-07
16:46"などと指定しています。次の記事のタイトルとの違いが出るのですが,修正していません。

\begin{itemize}
\tightlist
\item
  2021年05月08日11時24分の登録:
  REGEXP:''宇都宮健児''/データベース登録済みツイートの検索:2021-05-07〜2021-05-08/2021年05月08日11時23分の記録:ユーザ・投稿:15/23件
  \url{https://kk2020-09.blogspot.com/2021/05/regexp2021-05-072021-05\_8.html} 
\end{itemize}

 「2021-05-07〜2021-05-08」という部分の有無です。最初と最後に該当したツイートの日付になり,ひと目で範囲がわかります。

 期間指定なしでも68件と意外に少なかったのですが,私がデータベースに登録したツイートなので,実際のツイートの数が十分に反映されているものではありません。

\begin{itemize}
\item
  (01/68) RT
  tetsurokokubo(小久保 哲郎)|harada\_akira(原田あきら(日本共産党都議会議員/杉並))
  日時:2012-11-15 09:02:00 +0900/2012-11-15 02:53:00 +0900 URL:
  \url{https://twitter.com/tetsurokokubo/status/268866446086508544} 
  \url{https://twitter.com/harada\_akira/status/268773628290674688\textgreater} {}
  拡散拡散!宇都宮健児弁護士が出演したNHK「プロフェッショナル」がすごい・・・宇都宮さんの強さとともに弱さ優しさがにじみ出る番組構成。絶対観るべし!!!【\url{http://t.co/RbvNTop2} 】
\item
  (15/68) TW fukazawas(深澤諭史) 日時: 2016-10-11 15:14:00 +0900
  URL:
  \url{https://twitter.com/fukazawas/status/785725143012118528\textgreater} {}
  要領の悪い弁護士で、事務所を2回クビに 宇都宮健児 (1/2)
  〈週刊朝日〉|\url{https://t.co/u8kqE939ld\textgreater} {}
  「多重債務事件が激減して‥家賃が安いところに移りました‥これは誇るべきことなんです。私たちの運動が成功したということですから。」
\end{itemize}

 見覚えのない深澤諭史弁護士のツイートから意外な発見がありました。リンクの記事ですが,過払い金バブルに浮かれた弁護士に対する宇都宮健児弁護士の苦々しい気持や皮肉のようなことが書いてありました。

 記事の内容と見出しに疑問を感じることは余りなかったのですが,今回は何が主題なのかと疑問に思いました。いくつか記事を引用しておきたいと思います。

\begin{quote}
《引用の始まり》
\end{quote}

\begin{quote}
林:弁護士さんでもすごくお金を儲けてる人っているじゃないですか。紀尾井町にビルを持ってたりして。宇都宮さんはそういう方向にはいかなかったんですね。

宇都宮:私は要領の悪い弁護士だったので、事務所を2回クビになっているんです。司法試験には早く受かったんですが、営業活動が苦手でクライアントが増えない。普通の弁護士さんは、既存の弁護士事務所に勤めながら同窓会やライオンズクラブ、ロータリークラブに通って名刺を配ったり、中小企業のおやじさんたちと一緒にゴルフに行ったりして人脈をつくっていくわけですね。事務所に勤めて給料をもらっている弁護士を業界では「イソ弁」と言うんですが・・・・・・。
\end{quote}

\begin{quote}
《引用の終わり》
\end{quote}

\begin{itemize}
\tightlist
\item
  要領の悪い弁護士で、事務所を2回クビに 宇都宮健児 (1/2)
  〈週刊朝日〉|AERA dot.
  (アエラドット) \url{https://dot.asahi.com/wa/2016100700098.html?page=1n} 
\end{itemize}

\begin{quote}
《引用の始まり》
\end{quote}

\begin{quote}
林:最近、「払いすぎていませんか?」とかいう広告をたくさん見かけますけど、ああいう弁護士事務所ってどうなんですか。

宇都宮:私は問題のある事務所だと思っています。昔は非常に困難な仕事だったので、誰もやりたがらなかったんですね。それでも判例を獲得したり法改正があった結果、過払い金の返還請求ができるようになった。私はずっとそういう運動をやってきましたが、彼らはそれにまったく参加していない人たちなんです。正直苦々しく思っています。
\end{quote}

\begin{quote}
《引用の終わり》
\end{quote}

\begin{itemize}
\tightlist
\item
  要領の悪い弁護士で、事務所を2回クビに 宇都宮健児 (1/2)
  〈週刊朝日〉|AERA dot.
  (アエラドット) \url{https://dot.asahi.com/wa/2016100700098.html?page=1n} 
\end{itemize}

\begin{quote}
《引用の始まり》
\end{quote}

\begin{quote}
林:宇都宮さんが一生懸命切り開いた耕地にスッと入ってきて、お金だけ取っていくわけですね。当時はコワい人とも対決したり?

宇都宮:かなり乱暴な取り立てがありましたね。私はケンカに強いほうじゃありませんが、私が逃げたら依頼者が被害にあってしまう。自分より弱い立場の人を守るためには、頑張らなきゃいけない。そういう思いは私に限らず、多くの人が持ってると思いますけどね。
\end{quote}

\begin{quote}
《引用の終わり》
\end{quote}

\begin{itemize}
\tightlist
\item
  要領の悪い弁護士で、事務所を2回クビに 宇都宮健児 (2/2)
  〈週刊朝日〉|AERA dot.
  (アエラドット) \url{https://dot.asahi.com/wa/2016100700098.html?page=2n} 
\end{itemize}

\begin{quote}
《引用の始まり》
\end{quote}

\begin{quote}
林:それなのになぜ、景気よくCMを流している事務所があるんですか。

宇都宮:一時は300万人ぐらいいた多重債務者が12万、13万人になったと言われていますが、それをCMでかき集めているんでしょう。

※週刊朝日 2016年10月14日号より抜粋
\end{quote}

\begin{quote}
《引用の終わり》
\end{quote}

\begin{itemize}
\tightlist
\item
  要領の悪い弁護士で、事務所を2回クビに 宇都宮健児 (2/2)
  〈週刊朝日〉|AERA dot.
  (アエラドット) \url{https://dot.asahi.com/wa/2016100700098.html?page=2n} 
\end{itemize}

 個人的に宇都宮健児弁護士に対するイメージがいくらか良い方向に変わったのですが,それが誰がなっても救いようのない弁護士業界の闇,ブラックホールのようなものをより強く感じさせる方向にも変わりました。汚物は消毒などとツイートしてきた弁護士の消毒,除去が必要なのかもしれません。

\begin{itemize}
\tightlist
\item
  〈〈〈 2021/05/08 12:29:12 Linux Emacs: 〈〈〈
\end{itemize}

\hypertarget{ux3053ux3046ux3044ux3046ux30aaux30eaux30f3ux30d4ux30c3ux30afux3068ux3044ux3046ux3082ux306eux3092ux7dbaux9e97ux4e8bux3084ux507dux5584ux3067ux5857ux308aux56faux3081ux3066ux3084ux308bux610fux5473ux3084ux4fa1ux5024ux304cux3042ux308bux306eux304bux518dux8003ux3055ux308cux308bux3079ux304dux6642ux671fux3060ux308dux3046ux3068ux3044ux3046ux843dux5408ux6d0bux53f8ux5f01ux8b77ux58ebux6771ux4eacux5f01ux8b77ux58ebux4f1aux306eux30c4ux30a4ux30fcux30c8}{%
\paragraph{「こういうオリンピックというものを、綺麗事や偽善で塗り固めてやる意味や価値があるのか、再考されるべき時期だろう。」という落合洋司弁護士(東京弁護士会)のツイート}\label{ux3053ux3046ux3044ux3046ux30aaux30eaux30f3ux30d4ux30c3ux30afux3068ux3044ux3046ux3082ux306eux3092ux7dbaux9e97ux4e8bux3084ux507dux5584ux3067ux5857ux308aux56faux3081ux3066ux3084ux308bux610fux5473ux3084ux4fa1ux5024ux304cux3042ux308bux306eux304bux518dux8003ux3055ux308cux308bux3079ux304dux6642ux671fux3060ux308dux3046ux3068ux3044ux3046ux843dux5408ux6d0bux53f8ux5f01ux8b77ux58ebux6771ux4eacux5f01ux8b77ux58ebux4f1aux306eux30c4ux30a4ux30fcux30c8}}

\begin{itemize}
\tightlist
\item
  〉〉〉 Linux Emacs: 2021/05/08 12:35:58 〉〉〉
\end{itemize}

:CATEGORIES: @kanazawabengosi \#金沢弁護士会 @JFBAsns
日本弁護士連合会(日弁連) \#法務省 @MOJ\_HOUMU
\#落合洋司弁護士(東京弁護士会)

※ @kk\_hironoのアカウントがブロックされ,リツイートに失敗したツイート

\begin{itemize}
\tightlist
\item
  TW
  yjochi(弁護士落合洋司🌸感染拡大を招く東京(頭狂)オリンピック中止!🌸)
  日時:2021/05/06 14:02:41 URL:
  \url{https://twitter.com/yjochi/status/1390170129279819780} 
  \textgreater{}
  こういうオリンピックというものを、綺麗事や偽善で塗り固めてやる意味や価値があるのか、再考されるべき時期だろう。
  \url{https://t.co/4RCHmiqvBk} 
\end{itemize}

〉〉〉 kk\_hironoのリツイート 〉〉〉

\begin{itemize}
\item
  RT
  kk\_hirono(刑事告発・非常上告_金沢地方検察庁御中)|HYamaguchi(Hiroshi
  Yamaguchi) 日時:2021-05-08 12:38/2021/05/06 10:47 URL:
  \url{https://twitter.com/kk\_hirono/status/1390873639285714947} 
  \url{https://twitter.com/HYamaguchi/status/1390120976801701888} 
  \textgreater{}
  IOCバッハ会長を、開催を強要し「収益のほとんどを自分たちのものにし、費用は全て開催国に押し付け」る「ぼったくり男爵」、と。→「地方行脚で食料を食い尽くす王族」に例え、米有力紙、日本に五輪中止促す IOC批判「開催国を食い物」
  \textbar{} 2021/5/6 - 共同通信 \url{https://t.co/oQ0ptrfdvg} 
\item
  米有力紙、日本に五輪中止促す IOC批判「開催国を食い物」 | 共同通信
  \url{https://this.kiji.is/762850499815833600} 
\end{itemize}

 Twitterのトレンドになっていた元記事のようですが,5月6日12時43分の投稿で同日の14時3分にアップデートとなっています。記事を開いて全部読む気持はなかったのですが,少しスクロールするとずいぶん短い記事だとわかり,目を通しました。

 落合洋司弁護士(東京弁護士会)の従来の主張,価値観を裏付けるような内容の記事となっていますが,落合洋司弁護士(東京弁護士会)が影響を受けてきた情報源の1つのようにも思えてきました。

 落合洋司弁護士(東京弁護士会)のオリンピックに対する見方,価値観は,新型コロナウィルス問題が起こる前から目を奪われるような特異なものがあり,いくつか記録したものがあったと思います。

\begin{lstlisting}
py37_env ❯ d|grep @yjochi |grep オリンピック
\end{lstlisting}

\begin{itemize}
\item
  2017年12月07日06時38分の登録: \落合洋司 Yoji
  Ochiai @yjochi\この弁解はすごい。弁解オリンピックに出られる。→わいせつ行為疑いの町長「女性記者を助けようと」:朝日新聞デジタル
  \url{http://hirono2014sk.blogspot.com/2017/12/yoji-ochiaiyjochi\_18.html} 
\item
  2018年02月07日06時03分の登録: \落合洋司 Yoji
  Ochiai @yjochi\日本中が、天皇退位、新元号、さらにオリンピックで大騒ぎしている間に、眞子さま、婚約者に諦めさせ、ひっそりと婚約解消とい
  \url{http://hirono2014sk.blogspot.com/2018/02/yoji-ochiaiyjochi\_58.html} 
\item
  2018年02月13日08時31分の登録: REGEXP:''オリンピック''/落合洋司 Yoji
  Ochiai(@yjochi)の検索(2009-08-08〜2018-02-11/2018年02月13日08時31分の記録112件)
  \url{http://hirono2014sk.blogspot.com/2018/02/regexp-yoji-ochiaiyjochi2009-08-082018.html} 
\item
  2019年01月13日16時14分の登録: \落合洋司?? 「ニチョウ
  東京地検特捜部特別分室」発売中! @yjochi\東京オリンピックの直前に公判開始決定出て、インターポールが国際指名手配、とか。ちなみにイン
  \url{http://hirono2014sk.blogspot.com/2019/01/yjochi\_13.html} 
\item
  2019年10月19日10時32分の登録: \?弁護士 落合洋司 YOJI
  OCHIAI @yjochi\酷い暑さの夏の東京で、ゼニカネのために、必死に飛んだり跳ねたり走ったりするオリンピックをやること自体が間違い
  \url{http://hirono2014sk.blogspot.com/2019/10/yoji-ochiaiyjochi\_19.html} 
\item
  2019年12月16日21時59分の登録: \?弁護士
  落合洋司 権力は腐敗する @yjochi\NHKは、2020年東京オリンピックを目指す、ドロドロした、金と利権まみれの、ハゲタカみたいなドラマにしたほうが視聴
  \url{http://hirono2014sk.blogspot.com/2019/12/yjochinhk2020.html} 
\item
  2020年01月31日03時55分の登録: \?弁護士 落合洋司 ときは今
  あめが下しる
  五月哉 @yjochi\やれば、史上最悪のパンデミック五輪だろうな。¥\n→東京オリンピック中止か、新型肺炎対策でIOCとWHO
  \url{http://hirono2014sk.blogspot.com/2020/01/yjochi-iocwho.html} 
\item
  2020年02月07日22時36分の登録: \?弁護士 落合洋司 ときは今
  あめが下しる
  五月哉 @yjochi\東京オリンピック中止も、徐々に視野に入ってくるだろう。現状で推移し、あるいは拡大するなら、とてもじゃ
  \url{http://hirono2014sk.blogspot.com/2020/02/yjochi\_0.html} 
\item
  2020年03月19日11時39分の登録:
  \弁護士落合洋司・高輪ゲートウェイ駅徒歩5分 @yjochi\金儲けのための世界大運動会だから、そうなるわね。¥\n→東京オリンピックのチケット、規約上は「払い戻し不可」。新
  \url{http://hirono2014sk.blogspot.com/2020/03/5yjochi\_7.html} 
\item
  2020年07月23日21時13分の登録:
  \弁護士落合洋司?高輪ゲートウェイ駅徒歩5分?泉岳寺駅徒歩1分 @yjochi\``ワクチン''オリンピックまで世界に供給は不透明
  新型コロナ \textbar{} NHKニュース ←早く中止
  \url{http://hirono2014sk.blogspot.com/2020/07/51yjochi-nhk.html} 
\item
  2021年01月02日19時56分の登録:
  \弁護士落合洋司?高輪ゲートウェイ駅徒歩5分?泉岳寺駅徒歩1分 @yjochi\オリンピックは、コロナが収束した後にやりたければいくらでもできるけど、死んだ人は元に戻らな
  \url{http://kk2020-09.blogspot.com/2021/01/51yjochi.html} 
\item
  2021年01月02日19時56分の登録:
  \弁護士落合洋司?高輪ゲートウェイ駅徒歩5分?泉岳寺駅徒歩1分 @yjochi\死屍累々の東京でオリンピックやる意味ないでしょ。
  \url{http://kk2020-09.blogspot.com/2021/01/51yjochi\_2.html} 
\item
  2021年01月19日22時02分の登録:
  \弁護士落合洋司?高輪ゲートウェイ駅徒歩5分?泉岳寺駅徒歩1分 @yjochi\コロナの薬だけでなく馬鹿につける薬も要るな。→菅首相「東京オリンピックはコロナに打ち勝った
  \url{http://kk2020-09.blogspot.com/2021/01/51yjochi\_48.html} 
\item
  2021年03月06日01時21分の登録:
  \弁護士落合洋司?高輪ゲートウェイ駅徒歩5分?泉岳寺駅徒歩1分 @yjochi\早く東京オリンピック中止に。引用ツイート
  \url{https://kk2020-09.blogspot.com/2021/03/51yjochi\_22.html} 
\item
  2021年04月16日01時19分の登録:
  \弁護士落合洋司?高輪ゲートウェイ駅徒歩5分?泉岳寺駅徒歩1分 @yjochi\やるべきことはいろいろあるが、まずはオリンピックを中止しようよ。頭狂オリンピックになるわね
  \url{https://kk2020-09.blogspot.com/2021/04/51yjochi\_12.html} 
\item
  2021年04月29日08時51分の登録:
  \弁護士落合洋司?高輪ゲートウェイ駅徒歩5分?泉岳寺駅徒歩1分 @yjochi\狂気の大感染頭狂オリンピックはじまりはじまりー。
  \url{https://kk2020-09.blogspot.com/2021/04/51yjochi\_29.html} 
\item
  2021年04月30日07時17分の登録:
  \弁護士落合洋司?感染拡大を招く東京(頭狂)オリンピック中止!? @yjochi\こいつ、感染大爆発になっても責任取れないだろう。粘り強さでなんとかなるんなら、現状はない
  \url{https://kk2020-09.blogspot.com/2021/04/yjochi.html} 
\item
  2021年05月03日20時40分の登録:
  \弁護士落合洋司?感染拡大を招く東京(頭狂)オリンピック中止!? @yjochi\他人はタダでこき使う奴隷ビジネスモデル。こんなものに協力するのは志願して奴隷になるのと同
  \url{https://kk2020-09.blogspot.com/2021/05/yjochi.html} 
\item
  2021年05月05日00時04分の登録:
  \弁護士落合洋司?感染拡大を招く東京(頭狂)オリンピック中止!? @yjochi\断じて行えば鬼神も之を避く(笑)。→「オールジャパンで五輪開催できる」安倍氏の``根性論''
  \url{https://kk2020-09.blogspot.com/2021/05/yjochi\_5.html} 
\item
  2021年05月06日18時32分の登録:
  \弁護士落合洋司?感染拡大を招く東京(頭狂)オリンピック中止!? @yjochi\こういうオリンピックというものを、綺麗事や偽善で塗り固めてやる意味や価値があるのか、再考
  \url{https://kk2020-09.blogspot.com/2021/05/yjochi\_6.html} 
\item
  2021年05月07日19時38分の登録:
  \弁護士落合洋司?感染拡大を招く東京(頭狂)オリンピック中止!? @yjochi\遂に逃亡かボッタクリ男爵。→17日予定のIOCバッハ会長来日は見送りへ 橋本聖子会長「非
  \url{https://kk2020-09.blogspot.com/2021/05/yjochi\_7.html} 
\item
  2021年05月07日19時39分の登録:
  \弁護士落合洋司?感染拡大を招く東京(頭狂)オリンピック中止!? @yjochi\こういうオリンピックというものを、綺麗事や偽善で塗り固めてやる意味や価値があるのか、再考
  \url{https://kk2020-09.blogspot.com/2021/05/yjochi\_90.html} 
\item
  〈〈〈 2021/05/08 12:48:47 Linux Emacs: 〈〈〈
\end{itemize}

\hypertarget{section-12}{%
\paragraph{}\label{section-12}}

\hypertarget{ux5343ux8449ux306eux5f01ux8b77ux58ebux306bux3088ux308b2ux3064ux306eux5f37ux5236ux6027ux4ea4ux7b49ux81f4ux50b7ux4e8bux4ef6}{%
\subsubsection{千葉の弁護士による2つの強制性交等致傷事件}\label{ux5343ux8449ux306eux5f01ux8b77ux58ebux306bux3088ux308b2ux3064ux306eux5f37ux5236ux6027ux4ea4ux7b49ux81f4ux50b7ux4e8bux4ef6}}

\hypertarget{ux4e0dux53efux89e3ux306aux7d4cux904eux3092ux8fbfux3063ux3066ux3044ux305fux5343ux8449ux306eux5f01ux8b77ux58ebux306bux3088ux308b2ux3064ux306eux5f37ux5236ux6027ux4ea4ux7b49ux81f4ux50b7ux4e8bux4ef6}{%
\paragraph{不可解な経過を辿っていた千葉の弁護士による2つの強制性交等致傷事件}\label{ux4e0dux53efux89e3ux306aux7d4cux904eux3092ux8fbfux3063ux3066ux3044ux305fux5343ux8449ux306eux5f01ux8b77ux58ebux306bux3088ux308b2ux3064ux306eux5f37ux5236ux6027ux4ea4ux7b49ux81f4ux50b7ux4e8bux4ef6}}

\begin{itemize}
\tightlist
\item
  〉〉〉 Linux Emacs: 2021/05/09 17:22:48 〉〉〉
\end{itemize}

:CATEGORIES: @kanazawabengosi \#金沢弁護士会 @JFBAsns
日本弁護士連合会(日弁連) \#法務省 @MOJ\_HOUMU

\begin{quote}
《引用の始まり》
\end{quote}

\begin{quote}
再逮捕容疑は3月7日午後8時ごろ~午後9時20分ごろ、飲食店従業員の女性(22)を自宅に連れ込み、顔面を殴るなどして性的暴行を加え、軽傷を負わせたとしている。

 署によると、この女性から事件2日後に被害届が出され、武田容疑者の犯行とみて捜査していたところ、別の飲食店従業員の女性(24)をわいせつ目的で4月8日に自宅に連れ込み、けがを負わせるなどした最初の事件が発生し、同容疑者を逮捕したという。
\end{quote}

\begin{quote}
《引用の終わり》
\end{quote}

\begin{itemize}
\tightlist
\item
  女性を自宅に連れ込み性的暴行容疑 千葉市の弁護士を再逮捕 -
  産経ニュース \url{https://www.sankei.com/affairs/news/210506/afr2105060010-n1.htmln} 
\end{itemize}

 4月6日の再逮捕容疑が,「3月7日午後8時ごろ~午後9時20分ごろ」とあり,「自宅に連れ込み、顔面を殴るなどして性的暴行を加え、軽傷を負わせたとしている。」とあるので,こちらは強制性交等の既遂となるのかもしれません。

 「署によると、この女性から事件2日後に被害届が出され、武田容疑者の犯行とみて捜査していたところ、別の飲食店従業員の女性(24)をわいせつ目的で4月8日に自宅に連れ込み、けがを負わせるなどした最初の事件が発生し、同容疑者を逮捕したという。」というのは内偵捜査のようです。

 張り込みをしていたのかどうか不明ですが,他にも複数の女性の被害があるらしく警察がマークをしていた可能性というのはあるのかもしれません。同じ店の従業員という可能性もありそうです。

 どちらも自宅に連れ込み,とあるので,どちらの女性も自宅に入るのは同意をしていたのでしょう。親しい間柄だったのか不明ですが,弁護士が性的暴行の手段として顔面を殴打したというのも不可解な話です。理性の問題というより,どれほどの重罪になるのか法律は知っていたのでしょう。

 強制性交等致傷と似た類型に,強盗致死傷罪がありますが,被害者が死んでいないのに強盗致死罪の成立を認めた判例もあったように思います。死刑判決が逆転の無罪となった山中事件ですが,共犯者とされた被害者に対する強盗致死罪でした。

 いずれも結果的加重犯という法律上の論点だったと思いますが,被害者が死んでいないのに強盗致死罪の成立を認めたのは,殺人未遂罪との違いでも気になっていました。また,余り問題にされていないようなのも,弁護士業界の対応として疑問に思っていました。

\begin{quote}
《引用の始まり》
\end{quote}

\begin{quote}
1990年7月27日に名古屋高裁(山本卓裁判長)は殺人事件について仁保事件以来18年ぶりとなるとなる死刑判決差し戻しでの戦後6件目の死刑求刑事案の無罪判決を言い渡し、確定した。差し戻し審では岡山大学の教授・石津日出雄の鑑定でAの供述していた被害者の血痕反応がなく、マット下のビニールカバー一か所から検出された血液も被害者の血液型と異なることが判明していた。

なお、男性Bの男性Aに対する殺人未遂事件については、1審では口封じのための殺人未遂を認定した。しかし、その後の審理で強盗致死未遂事件とされ、強盗致死未遂事件として有罪が確定して懲役8年が宣告されたが、未決勾留日数が計上され、既に服役済みとして放免された。
\end{quote}

\begin{quote}
《引用の終わり》
\end{quote}

\begin{itemize}
\tightlist
\item
  山中事件 -
  Wikipedia \url{https://ja.wikipedia.org/wiki/\%E5\%B1\%B1\%E4\%B8\%AD\%E4\%BA\%8B\%E4\%BB\%B6n} 
\end{itemize}

 確認のため調べてみると,「強盗致死未遂事件」となっていました。「蒔絵職人・霜上則男の冤罪―山中温泉殺人事件」では,未遂とはなっていなかったような気もするのですが,被害者が死んではいないので,致死の結果は発生しておらず,普通に考えて未遂事件だとは思います。

 単に「強盗致死未遂」という言葉に違和感を感じていただけなのかもしれません。強盗と殺人未遂の併合罪であれば,すっきり納得できるのですが,殺人罪とは異なり殺害の故意が必要とされていないのに関わらず,致死の結果が発生していないのに強盗致死未遂というのはやはり変に思えます。

 当時は殺人罪の下限が懲役3年で,強盗致死は死刑まはた無期となっていたように思います。有期懲役の選択がなかったように思いますし,これは現在も同じかと思われます。確認したところ同じ刑法第240条の条文内で,致傷と致死を場合分けしていました。

\begin{quote}
《引用の始まり》
\end{quote}

\begin{quote}
強盗致死傷罪(ごうとうちししょうざい)は刑法第240条で定められた罪。

刑法第240条において「強盗が、人を負傷させたときは無期または六年以上の懲役に処し、死亡させたときは死刑または無期懲役に処する。」と規定されている。

刑法第236条の強盗罪の加重類型である。未遂も処罰される(刑法第243条)。
\end{quote}

\begin{quote}
《引用の終わり》
\end{quote}

\begin{itemize}
\tightlist
\item
  強盗致死傷罪 -
  Wikipedia \url{https://ja.wikipedia.org/wiki/\%E5\%BC\%B7\%E7\%9B\%97\%E8\%87\%B4\%E6\%AD\%BB\%E5\%82\%B7\%E7\%BD\%AAn} 
\end{itemize}

 滅多に法律の条文は見なくなっているのですが,「六年以上の懲役」というのは初めて見たように思います。たぶんこれは,強制性交等致傷でも同じ量刑になるものと思われます。

 強盗罪は刑法236条となっていましたが,5年以上の有期懲役とありました。確認は出来ませんが平成5年当時からこの量刑は変わっていないように思います。強姦罪は当時,2年以上の有期懲役となっていました。

 たぶん忘れていたのだと思いますが,強制性交等罪も5年以上の有期懲役となっていました。平成10年代の中頃の改正で3年以上の有期懲役に引きあげられていたと思いますが,今では強盗罪と同じ量刑になっていたようです。改めて厳罰化が進んでいたのだと再認識しました。

\begin{quote}
《引用の始まり》
\end{quote}

\begin{quote}
結果的加重犯[編集]強制性交等罪、準強制性交等罪若しくは監護者性交等罪又はこれらの未遂罪を犯し、それによって被害者を死亡・負傷させた場合は、強制性交等致死傷罪(刑法181条2項)が成立し、無期又は6年[9]以上の懲役に処せられる。姦淫【性交等】に着手しその途中で死傷させれば、姦淫【性交等】は未遂でも、強姦致傷罪【強制性交等致死傷罪】が既遂で成立する[10]。
\end{quote}

\begin{quote}
《引用の終わり》
\end{quote}

\begin{itemize}
\tightlist
\item
  強制性交等罪 -
  Wikipedia \url{https://ja.wikipedia.org/wiki/\%E5\%BC\%B7\%E5\%88\%B6\%E6\%80\%A7\%E4\%BA\%A4\%E7\%AD\%89\%E7\%BD\%AA\#\%E5\%BC\%B7\%E5\%88\%B6\%E6\%80\%A7\%E4\%BA\%A4\%E7\%AD\%89\%E8\%87\%B4\%E6\%AD\%BB\%E5\%82\%B7\%E7\%BD\%AAn} 
\end{itemize}

 強制性交等罪は,致傷と致死の区別がなく強制性交等致死傷罪(刑法181条2項)として,無期または6年以上の懲役とあります。強盗致死とは違い死刑がないようです。致傷で刑の下限が1年の違いしかないのかと思っていたのですが,致傷でも無期懲役が成立するようです。強盗も同じでした。

 千葉の弁護士の事件は,報道されている情報も乏しく正確性もどうなのかと思うのですが,最初の3月7日の犯行が強制性交等致傷の既遂だった可能性があり,4月8日に別の女性に同じ強制性交等致傷の未遂事件を起こしたとあります。どちらも自宅に招き入れているので顔見知りなのでしょう。

 強制性交等致傷で思い出すのは俳優の高畑裕太氏の事件ですが,テレビでも連日の大きな報道となっていました。多額の示談金という報道も出ていましたが,示談も成立し不起訴処分となっていました。

 強制わいせつ罪にも致死傷罪があったように思いますが,性的暴行を目的に有形力の行使を行えば,とんでもない重罪になることを法律の専門家である弁護士が知らなかったとは考えにくく,性的意図がなかったので殴ったつもりだったのかもしれません。

 別の見方をすれば,そこまで弁護士の社会的信用は地に堕ちているのかということです。弁護士という属性で警察に睨まれ,あるいは見せしめのさらし者にされたという可能性が,あるいはあるのかもしれません。弁護士にさんざんと苦汁をなめされた警察官というのもいるのかもしれません。

 桶川ストーカー殺人事件などは,弁護士の関与がはっきりしないものの,世論を巻き起こして警察の責任が問われ,警察官に2人の自殺者を出し,そのうち1人は,警察幹部の自宅に放火未遂をしたという話で,服役もしていたように思います。

 千葉には,刑事弁護に精通した2人の弁護士がいました。一人はしばらくツイートを見かけていませんが,いずれも実名のTwitterアカウントがありました。一人はつい最近もツイートを見かけていると思います。

\begin{itemize}
\tightlist
\item
  〈〈〈 2021/05/09 18:33:36 Linux Emacs: 〈〈〈
\end{itemize}

\hypertarget{section-13}{%
\paragraph{}\label{section-13}}

\hypertarget{ux65b0ux578bux30b3ux30edux30caux30a6ux30a3ux30ebux30b9ux95a2ux9023}{%
\subsubsection{新型コロナウィルス関連}\label{ux65b0ux578bux30b3ux30edux30caux30a6ux30a3ux30ebux30b9ux95a2ux9023}}

\hypertarget{ux5831ux9053ux30b8ux30e3ux30fcux30caux30eaux30b9ux30c8}{%
\subsection{報道・ジャーナリスト}\label{ux5831ux9053ux30b8ux30e3ux30fcux30caux30eaux30b9ux30c8}}

\hypertarget{ux30b8ux30e3ux30fcux30caux30eaux30b9ux30c8ux306eux6c5fux5dddux7d39ux5b50ux6c0f}{%
\subsubsection{ジャーナリストの江川紹子氏}\label{ux30b8ux30e3ux30fcux30caux30eaux30b9ux30c8ux306eux6c5fux5dddux7d39ux5b50ux6c0f}}

\hypertarget{ux30b8ux30e3ux30fcux30caux30eaux30b9ux30c8ux306eux6c5fux5dddux7d39ux5b50ux6c0fux306eux30eaux30c4ux30a4ux30fcux30c8ux3067ux898bux3064ux3051ux305fux4e8bux4ef6ux304bux308921ux5e74ux3068ux3044ux3046ux6876ux5dddux30b9ux30c8ux30fcux30abux30fcux6bbaux4ebaux4e8bux4ef6ux306bux95a2ux3059ux308bux6e05ux6c34ux6f54ux6c0fux306eux30c4ux30a4ux30fcux30c8}{%
\paragraph{ジャーナリストの江川紹子氏のリツイートで見つけた,事件から21年という桶川ストーカー殺人事件に関する清水潔氏のツイート}\label{ux30b8ux30e3ux30fcux30caux30eaux30b9ux30c8ux306eux6c5fux5dddux7d39ux5b50ux6c0fux306eux30eaux30c4ux30a4ux30fcux30c8ux3067ux898bux3064ux3051ux305fux4e8bux4ef6ux304bux308921ux5e74ux3068ux3044ux3046ux6876ux5dddux30b9ux30c8ux30fcux30abux30fcux6bbaux4ebaux4e8bux4ef6ux306bux95a2ux3059ux308bux6e05ux6c34ux6f54ux6c0fux306eux30c4ux30a4ux30fcux30c8}}

\begin{itemize}
\tightlist
\item
  〉〉〉 Linux Emacs: 2021/05/11 10:55:49 〉〉〉
\end{itemize}

:CATEGORIES: @kanazawabengosi \#金沢弁護士会 @JFBAsns
日本弁護士連合会(日弁連) \#法務省 @MOJ\_HOUMU
\#ジャーナリストの江川紹子氏

\begin{itemize}
\tightlist
\item
  1359:2021-05-11\_10:55:02 \#告発状 \#\#\#\#
  「私たちはベテランの男性弁護士(あるいは裁判官)たちにお世話になると同時に性的に搾取される」という佐藤倫子弁護士のツイートと,三宅俊一郎裁判長(4)
  \url{https://hirono-hideki.hatenadiary.jp/entry/2021/05/11/105459} 
\end{itemize}

 上記エントリーで,埋め込みツイートの数からエントリーの切り替えが必要と考えていた折り,ジャーナリストの江川紹子氏のツイートのまとめから意外な発見があり,独立した項目として記録しておく必要があると判断しました。

 まず,アウトラインのレベル2として「報道・ジャーナリスト」を新設,レベル3にジャーナリストの江川紹子氏としました。

 奉納\さらば弁護士鉄道・泥棒神社の物語(@hirono\_hideki)のアカウントではブロックされている清水潔氏のTwitterアカウントですが,今のところ再捜査要請書_警察庁・石川県警察御中(@kk\_hirono)ではブロックされていないようなので,一連のツイートをリツイートします。

〉〉〉 kk\_hironoのリツイート 〉〉〉

\begin{itemize}
\tightlist
\item
  RT
  kk\_hirono(刑事告発・非常上告_金沢地方検察庁御中)|NOSUKE0607(清水
  潔) 日時:2021-05-11 11:00/2020/10/26 12:17 URL:
  \url{https://twitter.com/kk\_hirono/status/1391936277537120259} 
  \url{https://twitter.com/NOSUKE0607/status/1320565268251197440} 
  \textgreater{}
  事件から21年。被害者の名誉が少しでも回復するように祈ってツイートさせて頂きました。正義の実現に関わる警察官、検察官、裁判官の方にもぜひ知って頂きたいのです。
  「助けてください」と警察に救いを求めたにもかかわらず、21歳でその命を奪われた女性がいました。どうか忘れないでください。
\end{itemize}

〉〉〉 kk\_hironoのリツイート 〉〉〉

\begin{itemize}
\tightlist
\item
  RT
  kk\_hirono(刑事告発・非常上告_金沢地方検察庁御中)|NOSUKE0607(清水
  潔) 日時:2021-05-11 11:00/2020/10/26 12:11 URL:
  \url{https://twitter.com/kk\_hirono/status/1391936300878467072} 
  \url{https://twitter.com/NOSUKE0607/status/1320563752421318657} 
  \textgreater{}
  未だに女子大生の誤った風評は消えていません。それは警察のリーク情報と裏付けのない報道によってです。被害者がどんな人で、桶川事件の裏側にどんな現実があったのかは拙著に記しました。図書館でも、古本でも結構ですからぜひお読み頂きたく思います。まだ若かった自分が魂を込めて綴りました。
\end{itemize}

〉〉〉 kk\_hironoのリツイート 〉〉〉

\begin{itemize}
\tightlist
\item
  RT
  kk\_hirono(刑事告発・非常上告_金沢地方検察庁御中)|NOSUKE0607(清水
  潔) 日時:2021-05-11 11:01/2020/10/26 11:58 URL:
  \url{https://twitter.com/kk\_hirono/status/1391936418650333184} 
  \url{https://twitter.com/NOSUKE0607/status/1320560346571051009} 
  \textgreater{}
  桶川事件で捜査を怠り、殺人事件を防げなかった埼玉県警は、記者クラブを相手にこんな会見を開いて被害者像をミスリードしました。殺人事件発生直後にヘラヘラ笑う幹部と記者の迎合には今更ながら驚くしかありません。そして記者クラブの問題は今もほぼ変わっていないのです。
  \url{https://t.co/TwMAUPvFww} 
\end{itemize}

〉〉〉 kk\_hironoのリツイート 〉〉〉

\begin{itemize}
\tightlist
\item
  RT
  kk\_hirono(刑事告発・非常上告_金沢地方検察庁御中)|NOSUKE0607(清水
  潔) 日時:2021-05-11 11:01/2020/10/26 11:55 URL:
  \url{https://twitter.com/kk\_hirono/status/1391936474556166144} 
  \url{https://twitter.com/NOSUKE0607/status/1320559687826247681} 
  \textgreater{}
  桶川ストーカー殺人事件で女子大生の猪野詩織さんが亡くなってから今日で21年が経ちました。埼玉県警がストーカーに対して出された告訴状を改竄して放置。殺人に至る流れを止める事ができなかった残念な事件です。御命日の今日、改めて被害者のご冥福をお祈りします。
\end{itemize}

 いずれも昨年2020年10月26日のツイートとなっています。4件をリツイートしました。たぶん初めて見るツイートだと思うのですが,この内容だと見ていれば記録に追加していると思います。清水潔氏のツイートの記録は今のところ数は多くないと思うので,あるのか確認しておきます。

\begin{lstlisting}
py37_env ❯ d|grep @NOSUKE0607|grep 桶川
\end{lstlisting}

\begin{itemize}
\tightlist
\item
  2020年07月14日22時20分の登録: \清水
  潔 @NOSUKE0607\「『focus』」「えっ!『focus』って大人の写真ゴシップ記事ばかり書いて」
  桶川ストーカー事件の被害者遺族に聞く
  \url{http://hirono2014sk.blogspot.com/2020/07/nosuke0607focusfocus.html} 
\end{itemize}

 思いの外,数があったのですが,桶川で絞り込むと,これも意外なことに1件だけでした。

 その2020年10月26日の清水潔氏のツイートに桶川ストーカー殺人事件で被害者の女子大生が亡くなってから21年とあるのですが,この事件は平成11年の安藤健次郎さんに対する傷害事件で私が拘置所にいる時,発生した事件であったと思います。この時期はあとで調べて知りました。

\begin{itemize}
\tightlist
\item
  桶川ストーカー殺人事件 - Wikipedia \url{https://t.co/L1eq4fTE5V} 
  1999年(平成11年)10月26日に埼玉県桶川市のJR東日本高崎線桶川駅前で殺害された事件。警察捜査上の名称はJR桶川駅西口女子大生路上殺人事件{[}1{]}。桶川事件とも呼ばれる{[}2{]}。本件の発生が契機となり、2000年に「ストーカー規制法」が制定
\end{itemize}

 事件当時の被害者の年齢が見当たらないのですが,20歳か21歳だったように思います。

\begin{itemize}
\tightlist
\item
  「娘は3度殺された」 教訓を忘れるな―遺族の訴え 桶川ストーカー殺人20年(上)
  | 47NEWS \url{https://t.co/xQMtvAGN84} 
  大学生の猪野詩織さん=当時(21)=がストーカー被害の末に殺害された「桶川ストーカー殺人事件」から今年で20年。ストーカーという言葉が広まり、法整備のきっかけにも
\end{itemize}

 やはり21歳だったと確認しましたが,平成4年の傷害・準強姦被告事件の被害者安藤文さんの年齢も同じ21歳でした。平成11年の安藤健次郎さんの傷害事件が8月7日でしたが,それから21年経っているとは,ジャーナリストの江川紹子氏のリツイートがなければ気が付かなかったと思います。

\begin{quote}
《引用の始まり》
\end{quote}

\begin{quote}
7件見つかりました。タイトル著者名出版者出版年ISBN所蔵館遺言 :
桶川ストーカー殺人事件の深層清水潔著新潮社2000/1041044050196虚誕 :
警察につくられた桶川ストーカー殺人事件鳥越俊太郎,小林ゆうこ著岩波書店2002/1140002252277鉄路の果てに清水/潔‖著[シミズ,キヨシ]マガジンハウス2020.5978-4-8387-3097-12犯罪被害者の声が聞こえますか東/大作‖著[ヒガシ,ダイサク]講談社2006.44-06-212591-94裁判所の正体
: 法服を着た役人たち瀬木比呂志, 清水潔著東京 :
新潮社2017.541044050351「南京事件」を調査せよ : mission
70th清水潔著東京 : 文藝春秋2016.841639051461騙されてたまるか :
調査報道の裏側清水潔著東京 : 新潮社2015.741061062562
\end{quote}

\begin{quote}
《引用の終わり》
\end{quote}

\begin{itemize}
\tightlist
\item
  石川県内図書館横断検索 \url{https://www.library.pref.ishikawa.lg.jp/wo/cross/?q=\%E6\%A1\%B6\%E5\%B7\%9D\%E3\%82\%B9\%E3\%83\%88\%E3\%83\%BC\%E3\%82\%AB\%E3\%83\%BC\%E6\%AE\%BA\%E4\%BA\%BA\%E4\%BA\%8B\%E4\%BB\%B6n} 
\end{itemize}

 どうも著者のプロフィールから検索に該当する本も出てくるようですが,図書館の蔵書を調べることは考えてもいませんでした。これは読んでおくべき本なのかと思います。何かと被害者安藤文さんを被害者とする告発事件と共通点もあるからです。

 清水潔氏は第三者のジャーナリストという立場で,事件を調べ警察の怠慢,不正をも追求したということはよく知られているかと思います。調査報道かあるいは捜査報道という言葉もあったと思います。1つのメソッドあるいは方法論のようですが,その後に続くものが見当たらないのも気になっています。

 この前,図書館に行った時,正確には最後に図書館に行った前日になると思いますが,入り口に自由に持っていってくださいという本が置いてあり,2つの本をもらってきました。

 結構新しく傷や汚れもなさそうな本ですが,「二〇一〇年度版 全国のあいつぐ差別事件 部落開放・人権政策確立要求中央実行委員会編・発行」という本です。もう一冊が「お達者くらべ」で,社団法人 石川県看護協会とあります。

 二日目はいくらか閉館に近い時間で,正木ひろし弁護士の本を探したのですが見つからず,時間を掛けてあちこち見て回ったのに簡単に見つけられそうな本ではなさそうでした。

 被害者安藤文さんの告発事件のことは,桶川ストーカー殺人事件との関連性を交えて,レベル3の三宅俊一郎裁判長の項目で取り上げておきたいと思います。佐藤倫子弁護士の件を片付けてから,そちらに戻ります。

\begin{itemize}
\tightlist
\item
  〈〈〈 2021/05/11 11:40:51 Linux Emacs: 〈〈〈
\end{itemize}

\end{document}
