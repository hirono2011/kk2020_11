% Options for packages loaded elsewhere
\PassOptionsToPackage{unicode}{hyperref}
\PassOptionsToPackage{hyphens}{url}
%
\documentclass[
]{ltjarticle}
\usepackage{lmodern}
\usepackage{amssymb,amsmath}
\usepackage{ifxetex,ifluatex}
\ifnum 0\ifxetex 1\fi\ifluatex 1\fi=0 % if pdftex
  \usepackage[T1]{fontenc}
  \usepackage[utf8]{inputenc}
  \usepackage{textcomp} % provide euro and other symbols
\else % if luatex or xetex
  \usepackage{unicode-math}
  \defaultfontfeatures{Scale=MatchLowercase}
  \defaultfontfeatures[\rmfamily]{Ligatures=TeX,Scale=1}
\fi
% Use upquote if available, for straight quotes in verbatim environments
\IfFileExists{upquote.sty}{\usepackage{upquote}}{}
\IfFileExists{microtype.sty}{% use microtype if available
  \usepackage[]{microtype}
  \UseMicrotypeSet[protrusion]{basicmath} % disable protrusion for tt fonts
}{}
\makeatletter
\@ifundefined{KOMAClassName}{% if non-KOMA class
  \IfFileExists{parskip.sty}{%
    \usepackage{parskip}
  }{% else
    \setlength{\parindent}{0pt}
    \setlength{\parskip}{6pt plus 2pt minus 1pt}}
}{% if KOMA class
  \KOMAoptions{parskip=half}}
\makeatother
\usepackage{listings}
\lstset{
language=sh,
basicstyle=\ttfamily\scriptsize,
commentstyle=\textit,
classoffset=1,
keywordstyle=\bfseries,
frame=tRBl,
framesep=5pt,
showstringspaces=false,
%numbers=left,
%stepnumber=1,
%numberstyle=\tiny,
tabsize=2,
breaklines = true,
}


\usepackage{xcolor}
\IfFileExists{xurl.sty}{\usepackage{xurl}}{} % add URL line breaks if available
\IfFileExists{bookmark.sty}{\usepackage{bookmark}}{\usepackage{hyperref}}
\hypersetup{
  pdftitle={告発状},
  hidelinks,
  pdfcreator={LaTeX via pandoc}}
\urlstyle{same} % disable monospaced font for URLs
\usepackage[left=3cm,right=2cm,top=3.5cm,bottom=2.7cm]{geometry}
\setlength{\emergencystretch}{3em} % prevent overfull lines
\providecommand{\tightlist}{%
  \setlength{\itemsep}{0pt}\setlength{\parskip}{0pt}}
\setcounter{secnumdepth}{5}
\ifluatex
  \usepackage{selnolig}  % disable illegal ligatures
\fi

\title{告発状}
\author{}
\date{}

\begin{document}
\vspace{25mm}
\fontsize{16pt}{0cm}\selectfont
\begin{center}
告発状(作成中)
\par\end{center}
\vspace{10mm}
\fontsize{11pt}{11pt}\selectfont
\vspace{3mm}
\begin{flushright}
令和3年2月25日
\par\end{flushright}
〒920-0912金沢市大手町6番15号 金沢地方検察庁御中

\fontsize{10pt}{10pt}\selectfont
\vspace{3mm}
\hspace{90mm}被告発人 金沢弁護士会所属 木梨松嗣弁護士

\hspace{90mm}被告発人 金沢弁護士会所属 岡田進弁護士

\hspace{90mm}被告発人 金沢弁護士会所属 長谷川紘之弁護士 

\hspace{90mm}被告発人 金沢弁護士会所属 若杉幸平弁護士

\hspace{90mm}被告発人 元名古屋高裁金沢支部裁判長 小島裕史 

\hspace{90mm}被告発人 元金沢地方裁判所裁判官 古川龍一

\hspace{90mm}被告発人 松平日出男

\hspace{90mm}被告発人 池田宏美 

\hspace{90mm}被告発人 梅野博之

\hspace{90mm}被告発人 安田繁克 

\hspace{90mm}被告発人 安田敏 

\hspace{90mm}被告発人 東渡好信 

\hspace{90mm}被告発人 多田敏明 

\hspace{90mm}被告発人 浜口卓也 

\hspace{90mm}被告発人 大網健二

\hspace{90mm}

\fontsize{11pt}{11pt}\selectfont
告発人\\
\\
〒927-0431 石川県鳳珠郡能登町宇出津山分10-3 \hspace{60mm}廣野秀樹

\vspace{16mm}
\begin{center}
記
\par\end{center}
\vspace{16mm}

\section{告発の趣旨}
 被告発人らの所為は,市場急配センター(所在地: 〒920-0025 石川県金沢市駅西本町5丁目10−20)における殺人未遂(刑法第203条,第199条)の共謀共同正犯(刑法第60条)として法的評価すべきもの,また,弁護士,裁判官らの立場と職権で隠ぺいした幇助犯であると思料するので,犯情甚だ悪質につき,無期懲役刑として処罰することを求め,ここに告発に及びます。
\clearpage


\end{document}
